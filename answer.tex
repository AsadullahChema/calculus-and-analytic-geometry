 {\urduTechTermsfont {حصہ}} 11.5\hskip 1em\relax {\urduTechTermsfont {صفحہ}} 1398
\begin {description}\setlength {\parskip }{0pt} \setlength {\itemsep }{0pt plus 1pt}
\item [
1)
]
$x=3+t,\, y=-4+t,\,z=-1+t$
\item [
3)
]
$x=-2+5t,\,y=5t,\,z=3-5t$
\item [
5)
]
$x=0,\,y=2t,\,z=t$
\item [
7)
]
$x=1,\,y=1,\,z=1+t$
\item [
9)
]
$x=t,\,y=-7+2t,\,z=2t$
\item [
11)
]
$x=t,\,y=0,\,z=0$
\item [
13)
]
$x=t,\,y=t,\,z=3/2t,\,0\le t\le 1$
 \begin {center} \begin {tikzpicture}[font=\small ,x={(-0.5cm,-0.5cm)},y={(1cm,0)},z={(0,1cm)}] \draw [-latex](0,0,0)--(1.5,0,0)node[left]{$x$}; \draw [-latex](0,0,0)--(0,1.5,0)node[right]{$y$}; \draw [-latex](0,0,0)--(0,0,1.5)node[left]{$z$}; \draw [->-=0.5](0,0,0)node[circ]{}node[left]{$(0,0,0)$}--(1,1,1.5)coordinate(kA)node[circ]{}node[right]{$(1,1,\tfrac {3}{2})$}; \draw [dashed](kA)--(1,1,0)--(1,0,0); \draw [dashed](1,1,0)--(0,1,0); \end {tikzpicture} \end {center} 
\item [
15)
]
$x=1,\,y=1+t,\,z=0,\,-1\le t\le 0$
 \begin {center} \begin {tikzpicture}[font=\small ,x={(-0.5cm,-0.5cm)},y={(1cm,0)},z={(0,1cm)}] \draw [-latex](0,0,0)--(1.5,0,0)node[left]{$x$}; \draw [-latex](0,0,0)--(0,1.5,0)node[right]{$y$}; \draw [-latex](0,0,0)--(0,0,0.75)node[left]{$z$}; \draw [->-=0.5](1,0,0)node[circ]{}node[below,xshift={2.5ex}]{$(1,0,0)$}--(1,1,0)node[circ]{}node[right]{$(1,1,0)$}; \draw [dashed](1,1,0)--(0,1,0); \end {tikzpicture} \end {center} 
\item [
17)
]
$x=0,\,y=1-2t,\,z=1,\,0\le t\le 1$
 \begin {center} \begin {tikzpicture}[font=\small ,x={(-0.5cm,-0.5cm)},y={(1cm,0)},z={(0,1cm)}] \draw [-latex](0,0,0)--(0.5,0,0)node[left]{$x$}; \draw [-latex](0,-1.5,0)--(0,1.5,0)node[right]{$y$}; \draw [-latex](0,0,0)--(0,0,1.5)node[left]{$z$}; \draw [->-=0.25](0,1,1)node[circ]{}node[above]{$(0,1,1)$}--(0,-1,1)node[circ]{}node[above]{$(0,-1,1)$}; \end {tikzpicture} \end {center} 
\item [
19)
]
$x=2-2t,\,y=2t,\,z=2-2t,\,0\le t\le 1$
 \begin {center} \begin {tikzpicture}[font=\small ,x={(-0.5cm,-0.5cm)},y={(1cm,0)},z={(0,1cm)}] \draw [-latex](0,0,0)--(1.5,0,0)node[left]{$x$}; \draw [-latex](0,0,0)--(0,1.5,0)node[right]{$y$}; \draw [-latex](0,0,0)--(0,0,1.5)node[left]{$z$}; \draw [->-=0.5](1,0,1)node[circ]{}node[left]{$(2,0,2)$}--(0,1,0)node[circ]{}node[below]{$(0,2,0)$}; \draw [dashed](1,0,0)--(1,0,1)--(0,0,1); \end {tikzpicture} \end {center} 
\item [
21)
]
$3x-2y-z=-3$
\item [
23)
]
$7x-5y-4z=6$
\item [
25)
]
$x+3y+4z=34$
\item [
27)
]
$(1,2,3),\,\, -20x+12y+z=7$
\item [
29)
]
$y+z=3$
\item [
31)
]
$x-y+z=0$
\item [
33)
]
$2\sqrt {30}$
\item [
35)
]
$0$
\item [
37)
]
$\tfrac {9\sqrt {42}}{7}$
\item [
39)
]
$3$
\item [
41)
]
$19/5$
\item [
43)
]
$5/3$
\item [
45)
]
$9/\sqrt {41}$
\item [
47)
]
$\pi /4$
\item [
49)
]
$1.76$\, ریڈیئن
\item [
51)
]
$0.82$\,ریڈیئن
\item [
53)
]
$(3/2,-3/2,1/2)$
\item [
55)
]
$(1,1,0)$
\item [
57)
]
$x=1-t,\,y=1+t,\,z=-1$
\item [
59)
]
$x=4,\,y=3+6t,\,z=1+3t$
\item [
61)
]
 \عددی {L_1} اور \عددی {L_2} متقاطع ہیں؛ \عددی {L_2} اور \عددی {L_3} متوازی ہیں؛ \عددی {L_1} اور \عددی {L_3} غیر ہمسطحی ہیں۔ 
\item [
63)
]
$x=2+2t,\,y=-4-t,\,z=7+3t;\,x=-2-t,\, y=-2+t/2,\, z=1-3/2t$
\item [
65)
]
$(0,-\tfrac {1}{2},-\tfrac {3}{2}),\, (-1,0,-3),\,(1,-1,0)$
\item [
69)
]
 بہت سارے مختلف جوابات ممکن ہیں۔ ان میں سے ایک جواب ہے: $x+y=3,\,2y+z=7$ 
\item [
71)
]
 ماسوائے ان سطحوں کے جو مبدا سے گزرتے ہوں یا جو محددی محور کے متوازی ہوں تمام سطحوں کو \عددی {x/a+y/b+z/c=1} سے ظاہر کیا جا سکتا ہے۔ 
\end {description}
 {\urduTechTermsfont {حصہ}} 11.6\hskip 1em\relax {\urduTechTermsfont {صفحہ}} 1414
\begin {description}\setlength {\parskip }{0pt} \setlength {\itemsep }{0pt plus 1pt}
\item [
1)
]
 شکل \حوالہ {شکل_سوال_سمتیہ_سطح_اور_مساوات_ملائیں_الف}
\item [
3)
]
 شکل \حوالہ {شکل_سوال_سمتیہ_سطح_اور_مساوات_ملائیں_پ}
\item [
5)
]
 شکل \حوالہ {شکل_سوال_سمتیہ_سطح_اور_مساوات_ملائیں_ٹ}
\item [
7)
]
 شکل \حوالہ {شکل_سوال_سمتیہ_سطح_اور_مساوات_ملائیں_ج}
\item [
9)
]
 شکل \حوالہ {شکل_سوال_سمتیہ_سطح_اور_مساوات_ملائیں_ح}
\item [
11)
]
 شکل \حوالہ {شکل_سوال_سمتیہ_سطح_اور_مساوات_ملائیں_د}
\item [
13)
]
$x^2+y^2=4$
 \begin {center} \begin {tikzpicture}[font=\small ,declare function={fx(\r ,\t )=\r *cos(\t );fy(\r ,\t )=\r *sin(\t );}] \begin {axis}[view/h=135,axis equal,axis lines=center,xtick={\empty },ytick={\empty },ztick={\empty },xlabel={$x$},ylabel={$y$},zlabel={$z$},xlabel style={anchor=east},ylabel style={anchor=west},zlabel style={anchor=east}, axis x line=none,axis y line=none,axis z line=none] \addplot 3[domain y=0:360] ({fx(2,y)},{fy(2,y)},{3}); \addplot 3[domain y=0:360] ({fx(2,y)},{fy(2,y)},{0}); \addplot 3[domain y=0:360] ({fx(2,y)},{fy(2,y)},{-3}); \addplot 3[] coordinates {({fx(2,-30)},{fy(2,-30)},{-3})({fx(2,-30)},{fy(2,-30)},{3})}; \addplot 3[] coordinates {({fx(2,150)},{fy(2,150)},{-3})({fx(2,150)},{fy(2,150)},{3})}; \addplot 3[-latex] coordinates {(0,0,0)(6,0,0)}node[below]{$x$}; \addplot 3[-latex] coordinates {(0,0,0)(0,5,0)}node[below]{$y$}; \addplot 3[-latex] coordinates {(0,0,0)(0,0,6)}node[left]{$z$}; \end {axis} \end {tikzpicture} \end {center} 
\item [
15)
]
$z=y^2-1$
 \begin {center} \begin {tikzpicture}[font=\small ,declare function={fz(\y )=(\y )^2-1;}] \begin {axis}[view/h=135,axis equal,axis lines=center,xtick={\empty },ytick={\empty },ztick={\empty },xlabel={$x$},ylabel={$y$},zlabel={$z$},xlabel style={anchor=east},ylabel style={anchor=west},zlabel style={anchor=east}, axis x line=none,axis y line=none,axis z line=none] \addplot 3[samples y=0,domain=-1:1]({0},{x},{fz(x)}); \addplot 3[samples y=0,domain=-1:1]({-1},{x},{fz(x)}); \addplot 3[samples y=0,domain=-1:1]({1},{x},{fz(x)}); \addplot 3[] coordinates {({-1},{-1},{fz(-1)})({1},{-1},{fz(-1)})}; \addplot 3[] coordinates {({-1},{1},{fz(1)})({1},{1},{fz(1)})}; \addplot 3[-latex] coordinates {(0,0,0)(2,0,0)}node[below]{$x$}; \addplot 3[-latex] coordinates {(0,0,0)(0,2,0)}node[below]{$y$}; \addplot 3[-latex] coordinates {(0,0,0)(0,0,1.5)}node[left]{$z$}; \end {axis} \end {tikzpicture} \end {center} 
\item [
17)
]
 $x^2+4z^2=16$ \begin {center} \begin {tikzpicture}[font=\small ,declare function={fx(\r ,\t )=4*\r *cos(\t );fz(\r ,\t )=2*\r *sin(\t );}] \pgfmathsetmacro {\ta }{0} \pgfmathsetmacro {\tb }{90} \pgfmathsetmacro {\tc }{180} \pgfmathsetmacro {\td }{270} \pgfmathsetmacro {\ky }{12} \begin {axis}[view/h=135,axis equal,axis lines=center,xtick={\empty },ytick={\empty },ztick={\empty },xlabel={$x$},ylabel={$y$},zlabel={$z$},xlabel style={anchor=east},ylabel style={anchor=west},zlabel style={anchor=east}, axis x line=none,axis y line=none,axis z line=none] \addplot 3[domain y=0:360] ({fx(2,y)},-\ky ,{fz(2,y)}); \addplot 3[domain y=0:360] ({fx(2,y)},0,{fz(2,y)}); \addplot 3[domain y=0:360] ({fx(2,y)},\ky ,{fz(2,y)}); \addplot 3[] coordinates {({fx(2,\ta )},-\ky ,{fz(2,\ta )})({fx(2,\ta )},\ky ,{fz(2,\ta )})}; \addplot 3[] coordinates {({fx(2,\tb )},-\ky ,{fz(2,\tb )})({fx(2,\tb )},\ky ,{fz(2,\tb )})}; \addplot 3[] coordinates {({fx(2,\tc )},-\ky ,{fz(2,\tc )})({fx(2,\tc )},\ky ,{fz(2,\tc )})}; \addplot 3[] coordinates {({fx(2,\td )},-\ky ,{fz(2,\td )})({fx(2,\td )},\ky ,{fz(2,\td )})}; \addplot 3[-latex] coordinates {(0,0,0)(2.5,0,0)}node[below]{$x$}; \addplot 3[-latex] coordinates {(0,0,0)(0,\ky +0.5,0)}node[below]{$y$}; \addplot 3[-latex] coordinates {(0,0,0)(0,0,2.5)}node[left]{$z$}; \end {axis} \end {tikzpicture} \end {center} 
\item [
19)
]
 $z^2-y^2=1$ \begin {center} \begin {tikzpicture}[font=\small ,declare function={fz(\y )=sqrt((\y )^2+1);}] \pgfmathsetmacro {\kx }{2} \pgfmathsetmacro {\ky }{1} \begin {axis}[view/h=135,axis equal,axis lines=center,xtick={\empty },ytick={\empty },ztick={\empty },xlabel={$x$},ylabel={$y$},zlabel={$z$},xlabel style={anchor=east},ylabel style={anchor=west},zlabel style={anchor=east}, axis x line=none,axis y line=none,axis z line=none] \addplot 3[samples y=0,domain=-\ky :\ky ] (-\kx ,x,{fz(x)}); \addplot 3[samples y=0,domain=-\ky :\ky ] (\kx ,x,{fz(x)}); \addplot 3[] coordinates{(-\kx ,-\ky ,{fz(-\ky )}) (\kx ,-\ky ,{fz(-\ky )})}; \addplot 3[] coordinates{(-\kx ,\ky ,{fz(-\ky )}) (\kx ,\ky ,{fz(-\ky )})}; \addplot 3[samples y=0,domain=-\ky :\ky ] (-\kx ,x,{-fz(x)}); \addplot 3[samples y=0,domain=-\ky :\ky ] (\kx ,x,{-fz(x)}); \addplot 3[] coordinates{(-\kx ,-\ky ,{-fz(-\ky )}) (\kx ,-\ky ,{-fz(-\ky )})}; \addplot 3[] coordinates{(-\kx ,\ky ,{-fz(-\ky )}) (\kx ,\ky ,{-fz(-\ky )})}; \addplot 3[-latex] coordinates {(0,0,0)(2.5,0,0)}node[below]{$x$}; \addplot 3[-latex] coordinates {(0,0,0)(0,\ky +0.5,0)}node[below]{$y$}; \addplot 3[-latex] coordinates {(0,0,0)(0,0,2.5)}node[left]{$z$}; \end {axis} \end {tikzpicture} \end {center} 
\item [
21)
]
 $9x^2+y^2+z^2=9$ \begin {center} \begin {tikzpicture}[font=\small ,declare function={f(\y )=sqrt(9-(\y )^2);g(\x )=sqrt(9-9*(\x )^2);h(\x )=3*sqrt(1-(\x )^2);}] \begin {axis}[view/h=135,axis equal,axis lines=center,xtick={\empty },ytick={\empty },ztick={\empty },xlabel={$x$},ylabel={$y$},zlabel={$z$},xlabel style={anchor=east},ylabel style={anchor=west},zlabel style={anchor=east}, axis x line=none,axis y line=none,axis z line=none] \addplot 3[samples y=0,domain=-3:3] (0,x,{f(x)}); \addplot 3[samples y=0,domain=-3:3] (0,x,{-f(x)}); \addplot 3[samples y=0,domain=-1:1] (x,0,{g(x)}); \addplot 3[samples y=0,domain=-1:1] (x,0,{-g(x)}); \addplot 3[samples y=0,domain=-1:1] (x,{h(x)},0); \addplot 3[samples y=0,domain=-1:1] (x,{-h(x)},0); \addplot 3[-latex] coordinates {(0,0,0)(2.5,0,0)}node[below]{$x$}; \addplot 3[-latex] coordinates {(0,0,0)(0,2+0.5,0)}node[below]{$y$}; \addplot 3[-latex] coordinates {(0,0,0)(0,0,2.5)}node[left]{$z$}; \end {axis} \end {tikzpicture} \end {center} 
\item [
23)
]
 $4x^2+9y^2+4z^2=36$ \begin {center} \begin {tikzpicture}[font=\small ,declare function={f(\y )=1/2*sqrt(36-9*(\y )^2);g(\x )=sqrt(9-(\x )^2);h(\x )=1/3*sqrt(36-4*(\x )^2);}] \begin {axis}[view/h=135,axis equal,axis lines=center,xtick={\empty },ytick={\empty },ztick={\empty },xlabel={$x$},ylabel={$y$},zlabel={$z$},xlabel style={anchor=east},ylabel style={anchor=west},zlabel style={anchor=east}, axis x line=none,axis y line=none,axis z line=none] \addplot 3[samples y=0,domain=-2:2,smooth] (0,x,{f(x)}); \addplot 3[samples y=0,domain=-2:2,smooth] (0,x,{-f(x)}); \addplot 3[samples y=0,domain=-3:3,smooth] (x,0,{g(x)}); \addplot 3[samples y=0,domain=-3:3,smooth] (x,0,{-g(x)}); \addplot 3[samples y=0,domain=-3:3,smooth] (x,{h(x)},0); \addplot 3[samples y=0,domain=-3:3,smooth] (x,{-h(x)},0); \addplot 3[-latex] coordinates {(0,0,0)(4,0,0)}node[below]{$x$}; \addplot 3[-latex] coordinates {(0,0,0)(0,5,0)}node[below]{$y$}; \addplot 3[-latex] coordinates {(0,0,0)(0,0,3.5)}node[left]{$z$}; \end {axis} \end {tikzpicture} \end {center} 
\item [
25)
]
 $z=x^2+4y^2$ \begin {center} \begin {tikzpicture}[font=\small ,declare function={fx(\r ,\t )=2*\r *cos(\t );fy(\r ,\t )=\r *sin(\t );hz(\x ,\y )=(\x )^2+4*(\y )^2;}] \pgfmathsetmacro {\ra }{1} \pgfmathsetmacro {\rb }{4} \begin {axis}[view/h=135,axis equal,axis lines=center,xtick={\empty },ytick={\empty },ztick={\empty },xlabel={$x$},ylabel={$y$},zlabel={$z$},xlabel style={anchor=east},ylabel style={anchor=west},zlabel style={anchor=east}, axis x line=none,axis y line=none,axis z line=none] \addplot 3[domain=-2:2,domain y=0:360,smooth,variable=\r ,variable y=\t ] ({fx(\ra ,t)},{fy(\ra ,t)},4); \addplot 3[domain=-2:2,domain y=-1:1,smooth] (0,{y},{hz(0,y)}); \addplot 3[samples y=0,domain=-2:2,domain=-2:2,smooth] ({x},0,{hz(x,0)}); \addplot 3[-latex] coordinates {(0,0,0)(1,0,0)}node[below]{$x$}; \addplot 3[-latex] coordinates {(0,0,0)(0,1,0)}node[below]{$y$}; \addplot 3[-latex] coordinates {(0,0,0)(0,0,5.5)}node[left]{$z$}; \end {axis} \end {tikzpicture} \end {center} 
\item [
27)
]
 $z=8-x^2-y^2$ \begin {center} \begin {tikzpicture}[font=\small ,declare function={fx(\r ,\t )=sqrt(8)*\r *cos(\t );fy(\r ,\t )=sqrt(8)*\r *sin(\t );hz(\x ,\y )=8-(\x )^2-(\y )^2;}] \pgfmathsetmacro {\ra }{2*sqrt(2)} \begin {axis}[view/h=135,axis equal,axis lines=center,xtick={\empty },ytick={\empty },ztick={\empty },xlabel={$x$},ylabel={$y$},zlabel={$z$},xlabel style={anchor=east},ylabel style={anchor=west},zlabel style={anchor=east}, axis x line=none,axis y line=none,axis z line=none] \addplot 3[domain=-\ra :\ra ,domain y=0:360,smooth,variable=\r ,variable y=\t ] ({fx(1,t)},{fy(1,t)},0); \addplot 3[domain=-\ra :\ra ,domain y=-\ra :\ra ,smooth] (0,{y},{hz(0,y)}); \addplot 3[samples y=0,domain=-\ra :\ra ,domain=-\ra :\ra ,smooth] ({x},0,{hz(x,0)}); \addplot 3[-latex] coordinates {(0,0,0)(4,0,0)}node[below]{$x$}; \addplot 3[-latex] coordinates {(0,0,0)(0,4,0)}node[below]{$y$}; \addplot 3[-latex] coordinates {(0,0,0)(0,0,9)}node[left]{$z$}; \end {axis} \end {tikzpicture} \end {center} 
\item [
29)
]
 $x=4-4y^2-z^2$ \begin {center} \begin {tikzpicture}[font=\small ,declare function={fy(\r ,\t )=\r *cos(\t );fz(\r ,\t )=2*\r *sin(\t );hx(\y ,\z )=4-4*(\y )^2-(\z )^2;}] \pgfmathsetmacro {\ra }{2} \begin {axis}[view/h=135,axis equal,axis lines=center,xtick={\empty },ytick={\empty },ztick={\empty },xlabel={$x$},ylabel={$y$},zlabel={$z$},xlabel style={anchor=east},ylabel style={anchor=west},zlabel style={anchor=east}, axis x line=none,axis y line=none,axis z line=none] \addplot 3[domain=-\ra :\ra ,domain y=0:360,smooth,variable=\r ,variable y=\t ] (0,{fy(1,t)},{fz(1,t)}); \addplot 3[domain=-2:2,domain y=-2:2,smooth] ({hx(0,y)},0,y); \addplot 3[samples y=0,domain=-1:1,domain=-1:1,smooth] ({hx(x,0)},x,0); \addplot 3[-latex] coordinates {(0,0,0)(5,0,0)}node[below]{$x$}; \addplot 3[-latex] coordinates {(0,0,0)(0,1.5,0)}node[below]{$y$}; \addplot 3[-latex] coordinates {(0,0,0)(0,0,2.5)}node[left]{$z$}; \end {axis} \end {tikzpicture} \end {center} 
\item [
31)
]
 $x^2+y^2=z^2$ \begin {center} \begin {tikzpicture}[font=\small ,declare function={fx(\r ,\t )=\r *cos(\t );fy(\r ,\t )=\r *sin(\t );}] \pgfmathsetmacro {\ra }{2} \begin {axis}[view/h=135,axis equal,axis lines=center,xtick={\empty },ytick={\empty },ztick={\empty },xlabel={$x$},ylabel={$y$},zlabel={$z$},xlabel style={anchor=east},ylabel style={anchor=west},zlabel style={anchor=east}, axis x line=none,axis y line=none,axis z line=none] \addplot 3[domain=-\ra :\ra ,domain y=0:360,smooth,variable=\r ,variable y=\t ] ({fx(1,t)},{fy(1,t)},1); \addplot 3[domain=-\ra :\ra ,domain y=0:360,smooth,variable=\r ,variable y=\t ] ({fx(1,t)},{fy(1,t)},-1); \addplot 3[domain=-1:1,domain y=-1:1,smooth] (0,y,y); \addplot 3[domain=-1:1,domain y=-1:1,smooth] (0,y,-y); \addplot 3[samples y=0,domain=-1:1,domain y=-1:1,smooth] (x,0,x); \addplot 3[samples y=0,domain=-1:1,domain y=-1:1,smooth] (x,0,-x); \addplot 3[-latex] coordinates {(0,0,0)(2,0,0)}node[below]{$x$}; \addplot 3[-latex] coordinates {(0,0,0)(0,2,0)}node[below]{$y$}; \addplot 3[-latex] coordinates {(0,0,0)(0,0,2)}node[left]{$z$}; \end {axis} \end {tikzpicture} \end {center} 
\item [
33)
]
 $4x^2+9z^2=9y^2$ \begin {center} \begin {tikzpicture}[font=\small ,declare function={fx(\r ,\t )=3/2*\r *cos(\t );fz(\r ,\t )=\r *sin(\t );}] \pgfmathsetmacro {\ra }{2} \begin {axis}[view/h=135,axis equal,axis lines=center,xtick={\empty },ytick={\empty },ztick={\empty },xlabel={$x$},ylabel={$y$},zlabel={$z$},xlabel style={anchor=east},ylabel style={anchor=west},zlabel style={anchor=east}, axis x line=none,axis y line=none,axis z line=none] \addplot 3[domain=-\ra :\ra ,domain y=0:360,smooth,variable=\r ,variable y=\t ] ({fx(1,t)},1,{fz(1,t)}); \addplot 3[domain=-\ra :\ra ,domain y=0:360,smooth,variable=\r ,variable y=\t ] ({fx(1,t)},-1,{fz(1,t)}); \addplot 3[domain=-1:1,domain y=-1:1,smooth] (0,y,y); \addplot 3[domain=-1:1,domain y=-1:1,smooth] (0,y,-y); \addplot 3[-latex] coordinates {(0,0,0)(2,0,0)}node[below]{$x$}; \addplot 3[-latex] coordinates {(0,0,0)(0,2,0)}node[below]{$y$}; \addplot 3[-latex] coordinates {(0,0,0)(0,0,2)}node[left]{$z$}; \end {axis} \end {tikzpicture} \end {center} 
\item [
35)
]
 $x^2+y^2-z^2=1$ \begin {center} \begin {tikzpicture}[font=\small ,declare function={fy(\x )=sqrt(2-(\x )^2);hz(\x ,\y )=sqrt((\x )^2+(\y )^2-1);}] \pgfmathsetmacro {\ra }{sqrt(2)} \begin {axis}[view/h=135,axis equal,axis lines=center,xtick={\empty },ytick={\empty },ztick={\empty },xlabel={$x$},ylabel={$y$},zlabel={$z$},xlabel style={anchor=east},ylabel style={anchor=west},zlabel style={anchor=east}, axis x line=none,axis y line=none,axis z line=none] \addplot 3[samples y=0,domain=-sqrt(2):sqrt(2)] (x,{fy(x)},1); \addplot 3[samples y=0,domain=-sqrt(2):sqrt(2)] (x,{-fy(x)},1); \addplot 3[samples y=0,domain=-sqrt(2):sqrt(2)] (x,{fy(x)},-1); \addplot 3[samples y=0,domain=-sqrt(2):sqrt(2)] (x,{-fy(x)},-1); \addplot 3[domain=1:\ra ,domain y=1:\ra ,smooth] (0,y,{hz(0,y)}); \addplot 3[domain=1:\ra ,domain y=1:\ra ,smooth] (0,y,{-hz(0,y)}); \addplot 3[domain=1:\ra ,domain y=1:\ra ,smooth] (0,-y,{hz(0,y)}); \addplot 3[domain=1:\ra ,domain y=1:\ra ,smooth] (0,-y,{-hz(0,y)}); \addplot 3[samples y=0,domain=1:\ra ,domain y=1:\ra ,smooth] (x,0,{hz(x,0)}); \addplot 3[samples y=0,domain=1:\ra ,domain y=1:\ra ,smooth] (x,0,{-hz(x,0)}); \addplot 3[samples y=0,domain=1:\ra ,domain y=1:\ra ,smooth] (-x,0,{hz(x,0)}); \addplot 3[samples y=0,domain=1:\ra ,domain y=1:\ra ,smooth] (-x,0,{-hz(x,0)}); \addplot 3[-latex] coordinates {(0,0,0)(2.5,0,0)}node[below]{$x$}; \addplot 3[-latex] coordinates {(0,0,0)(0,2.5,0)}node[below]{$y$}; \addplot 3[-latex] coordinates {(0,0,0)(0,0,2.5)}node[left]{$z$}; \end {axis} \end {tikzpicture} \end {center} 
\item [
37)
]
 $\tfrac {y^2}{4}+\tfrac {z^2}{9}-\tfrac {x^2}{4}=1$ \begin {center} \begin {tikzpicture}[font=\small ,declare function={fz(\z )=3*sqrt(2-1/4*(\z )^2);hz(\x ,\y )=3/2*sqrt(4+(\x )^2-(\y )^2);}] \pgfmathsetmacro {\ra }{sqrt(2)} \begin {axis}[view/h=135,axis equal,axis lines=center,xtick={\empty },ytick={\empty },ztick={\empty },xlabel={$x$},ylabel={$y$},zlabel={$z$},xlabel style={anchor=east},ylabel style={anchor=west},zlabel style={anchor=east}, axis x line=none,axis y line=none,axis z line=none] \addplot 3[samples y=0,domain=-sqrt(8):sqrt(8)] (2,x,{fz(x)}); \addplot 3[samples y=0,domain=-sqrt(8):sqrt(8)] (2,x,{-fz(x)}); \addplot 3[samples y=0,domain=-sqrt(8):sqrt(8)] (-2,x,{fz(x)}); \addplot 3[samples y=0,domain=-sqrt(8):sqrt(8)] (-2,x,{-fz(x)}); \addplot 3[samples y=0,domain=-2:2,domain y=-2:2]({x},{0},{hz(x,0)}); \addplot 3[samples y=0,domain=-2:2,domain y=-2:2]({x},{0},{-hz(x,0)}); \addplot 3[domain=-2:2,domain y=-2:2]({0},{y},{hz(0,y)}); \addplot 3[domain=-2:2,domain y=-2:2]({0},{y},{-hz(0,y)}); \addplot 3[-latex] coordinates {(0,0,0)(5.5,0,0)}node[below]{$x$}; \addplot 3[-latex] coordinates {(0,0,0)(0,6,0)}node[below]{$y$}; \addplot 3[-latex] coordinates {(0,0,0)(0,0,6)}node[left]{$z$}; \end {axis} \end {tikzpicture} \end {center} 
\item [
39)
]
 $z^2-x^2-y^2=1$ \begin {center} \begin {tikzpicture}[font=\small ,declare function={fz(\x ,\y )=sqrt(1+(\x )^2+(\y )^2);hy(\x )=sqrt(8-(\x )^2);}] \pgfmathsetmacro {\ra }{sqrt(2)} \begin {axis}[view/h=135,axis equal,axis lines=center,xtick={\empty },ytick={\empty },ztick={\empty },xlabel={$x$},ylabel={$y$},zlabel={$z$},xlabel style={anchor=east},ylabel style={anchor=west},zlabel style={anchor=east}, axis x line=none,axis y line=none,axis z line=none] \addplot 3[samples y=0,domain=-sqrt(8):sqrt(8)] (x,{hy(x)},3); \addplot 3[samples y=0,domain=-sqrt(8):sqrt(8)] (x,{-hy(x)},3); \addplot 3[samples y=0,domain=-sqrt(8):sqrt(8)] (x,{hy(x)},-3); \addplot 3[samples y=0,domain=-sqrt(8):sqrt(8)] (x,{-hy(x)},-3); \addplot 3[samples y=0,domain=-sqrt(8):sqrt(8)] (x,0,{fz(x,0)}); \addplot 3[samples y=0,domain=-sqrt(8):sqrt(8)] (x,0,{-fz(x,0)}); \addplot 3[domain=-sqrt(8):sqrt(8)] (0,y,{fz(0,y)}); \addplot 3[domain=-sqrt(8):sqrt(8)] (0,y,{-fz(0,y)}); \addplot 3[-latex] coordinates {(0,0,0)(5.5,0,0)}node[below]{$x$}; \addplot 3[-latex] coordinates {(0,0,0)(0,6,0)}node[below]{$y$}; \addplot 3[-latex] coordinates {(0,0,0)(0,0,6)}node[left]{$z$}; \end {axis} \end {tikzpicture} \end {center} 
\item [
41)
]
 $x^2-y^2-\tfrac {z^2}{4}=1$ \begin {center} \begin {tikzpicture}[font=\small ,declare function={fx(\y ,\z )=sqrt(1+(\y )^2+1/4*(\z )^2);hz(\y )=2*sqrt(8-(\y )^2);}] \pgfmathsetmacro {\ra }{sqrt(8)} \begin {axis}[view/h=135,axis equal,axis lines=center,xtick={\empty },ytick={\empty },ztick={\empty },xlabel={$x$},ylabel={$y$},zlabel={$z$},xlabel style={anchor=east},ylabel style={anchor=west},zlabel style={anchor=east}, axis x line=none,axis y line=none,axis z line=none] \addplot 3[samples y=0,domain=-sqrt(8):sqrt(8)] (3,x,{hz(x)}); \addplot 3[samples y=0,domain=-sqrt(8):sqrt(8)] (3,x,{-hz(x)}); \addplot 3[samples y=0,domain=-sqrt(8):sqrt(8)] (-3,x,{hz(x)}); \addplot 3[samples y=0,domain=-sqrt(8):sqrt(8)] (-3,x,{-hz(x)}); \addplot 3[samples y=0,domain=-\ra :\ra ]({fx(x,0)},{x},{0}); \addplot 3[samples y=0,domain=-\ra :\ra ]({-fx(x,0)},{x},{0}); \addplot 3[domain y=-6:6]({fx(0,y)},{0},{y}); \addplot 3[domain y=-6:6]({-fx(0,y)},{0},{y}); \addplot 3[-latex] coordinates {(0,0,0)(5.5,0,0)}node[below]{$x$}; \addplot 3[-latex] coordinates {(0,0,0)(0,6,0)}node[below]{$y$}; \addplot 3[-latex] coordinates {(0,0,0)(0,0,6)}node[left]{$z$}; \end {axis} \end {tikzpicture} \end {center} 
\item [
43)
]
 $y^2-x^2=z$ \begin {center} \begin {tikzpicture}[font=\small ,declare function={fz(\x ,\y )=(\y )^2-(\x )^2;}] \pgfmathsetmacro {\ra }{1} \pgfmathsetmacro {\rb }{1} \begin {axis}[view/h=120,axis equal,axis lines=center,xtick={\empty },ytick={\empty },ztick={\empty },xlabel={$x$},ylabel={$y$},zlabel={$z$},xlabel style={anchor=east},ylabel style={anchor=west},zlabel style={anchor=east}, axis x line=none,axis y line=none,axis z line=none] \addplot 3[samples y=0,domain=-\ra :\ra ] (x,0,{fz(x,0)}); \addplot 3[samples y=0,domain=-\rb :\rb ] (x,1,{fz(x,1)}); \addplot 3[samples y=0,domain=-\rb :\rb ] (x,-1,{fz(x,-1)}); \addplot 3[domain y=-1:1] (0,y,{fz(0,y)}); \addplot 3[domain y=-1:1] (1,y,{fz(1,y)}); \addplot 3[domain y=-1:1] (-1,y,{fz(-1,y)}); \addplot 3[-latex] coordinates {(0,0,0)(2.5,0,0)}node[below]{$x$}; \addplot 3[-latex] coordinates {(0,0,0)(0,2,0)}node[below]{$y$}; \addplot 3[-latex] coordinates {(0,0,0)(0,0,1.5)}node[left]{$z$}; \end {axis} \end {tikzpicture} \end {center} 
\item [
45)
]
 $x^2+y^2+z^2=4$ \begin {center} \begin {tikzpicture}[font=\small ,declare function={fz(\x ,\y )=sqrt(4-(\x )^2-(\y )^2);fy(\x )=sqrt(4-(\x )^2);}] \pgfmathsetmacro {\ra }{2} \begin {axis}[view/h=120,axis equal,axis lines=center,xtick={\empty },ytick={\empty },ztick={\empty },xlabel={$x$},ylabel={$y$},zlabel={$z$},xlabel style={anchor=east},ylabel style={anchor=west},zlabel style={anchor=east}, axis x line=none,axis y line=none,axis z line=none] \addplot 3[samples y=0,domain=-\ra :\ra ] (x,0,{fz(x,0)}); \addplot 3[samples y=0,domain=-\ra :\ra ] (x,0,{-fz(x,0)}); \addplot 3[domain y=-\ra :\ra ] (0,y,{fz(0,y)}); \addplot 3[domain y=-\ra :\ra ] (0,y,{-fz(0,y)}); \addplot 3[samples y=0,domain=-\ra :\ra ] (x,{fy(x)},0); \addplot 3[samples y=0,domain=-\ra :\ra ] (x,{-fy(x)},0); \addplot 3[-latex] coordinates {(0,0,0)(3,0,0)}node[below]{$x$}; \addplot 3[-latex] coordinates {(0,0,0)(0,2.5,0)}node[below]{$y$}; \addplot 3[-latex] coordinates {(0,0,0)(0,0,2.5)}node[left]{$z$}; \end {axis} \end {tikzpicture} \end {center} 
\item [
47)
]
 $z=1+y^2-x^2$ \begin {center} \begin {tikzpicture}[font=\small ,declare function={fz(\x ,\y )=1+(\y )^2-(\x )^2;}] \pgfmathsetmacro {\ra }{1} \begin {axis}[view/h=120,axis equal,axis lines=center,xtick={\empty },ytick={\empty },ztick={\empty },xlabel={$x$},ylabel={$y$},zlabel={$z$},xlabel style={anchor=east},ylabel style={anchor=west},zlabel style={anchor=east}, axis x line=none,axis y line=none,axis z line=none] \addplot 3[samples y=0,domain=-\ra :\ra ] (x,1,{fz(x,1)}); \addplot 3[samples y=0,domain=-\ra :\ra ] (x,-1,{fz(x,-1)}); \addplot 3[domain y=-\ra :\ra ] (0,y,{fz(0,y)}); \addplot 3[domain y=-\ra :\ra ] (1,y,{fz(1,y)}); \addplot 3[domain y=-\ra :\ra ] (-1,y,{fz(-1,y)}); \addplot 3[-latex] coordinates {(0,0,0)(\ra +0.5,0,0)}node[below]{$x$}; \addplot 3[-latex] coordinates {(0,0,0)(0,\ra ,0)}node[below]{$y$}; \addplot 3[-latex] coordinates {(0,0,0)(0,0,1.5)}node[left]{$z$}; \end {axis} \end {tikzpicture} \end {center} 
\item [
49)
]
 $y=-x^2-z^2$ \begin {center} \begin {tikzpicture}[font=\small ,declare function={fy(\x ,\z )=-(\x )^2-(\z )^2;fz(\x )=sqrt(1-(\x )^2);}] \pgfmathsetmacro {\ra }{1} \begin {axis}[view/h=120,axis equal,axis lines=center,xtick={\empty },ytick={\empty },ztick={\empty },xlabel={$x$},ylabel={$y$},zlabel={$z$},xlabel style={anchor=east},ylabel style={anchor=west},zlabel style={anchor=east}, axis x line=none,axis y line=none,axis z line=none] \addplot 3[samples y=0,domain=-\ra :\ra ] (x,{fy(x,0)},0); \addplot 3[domain y=-\ra :\ra ] (0,{fy(0,y)},y); \addplot 3[samples y=0,domain=-\ra :\ra ] (x,-1,{fz(x)}); \addplot 3[samples y=0,domain=-\ra :\ra ] (x,-1,{-fz(x)}); \addplot 3[-latex] coordinates {(0,0,0)(\ra ,0,0)}node[below]{$x$}; \addplot 3[-latex] coordinates {(0,0,0)(0,0.5,0)}node[below]{$y$}; \addplot 3[-latex] coordinates {(0,0,0)(0,0,\ra )}node[left]{$z$}; \end {axis} \end {tikzpicture} \end {center} 
\item [
51)
]
 $16x^2+4y^2=1$ \begin {center} \begin {tikzpicture}[font=\small ,declare function={fy(\x )=1/2*sqrt(1-16*(\x )^2);}] \pgfmathsetmacro {\ra }{1/4} \pgfmathsetmacro {\rb }{1/2} \begin {axis}[view/h=120,axis equal,axis lines=center,xtick={\empty },ytick={\empty },ztick={\empty },xlabel={$x$},ylabel={$y$},zlabel={$z$},xlabel style={anchor=east},ylabel style={anchor=west},zlabel style={anchor=east}, axis x line=none,axis y line=none,axis z line=none] \addplot 3[samples y=0,domain=-\ra :\ra ] (x,{fy(x)},1); \addplot 3[samples y=0,domain=-\ra :\ra ] (x,{-fy(x)},1); \addplot 3[samples y=0,domain=-\ra :\ra ] (x,{fy(x)},0); \addplot 3[samples y=0,domain=-\ra :\ra ] (x,{-fy(x)},0); \addplot 3[samples y=0,domain=-\ra :\ra ] (x,{fy(x)},-1); \addplot 3[samples y=0,domain=-\ra :\ra ] (x,{-fy(x)},-1); \addplot 3[]coordinates {(\ra ,0,-1)(\ra ,0,1)}; \addplot 3[]coordinates {(-\ra ,0,-1)(-\ra ,0,1)}; \addplot 3[]coordinates {(0,\rb ,-1)(0,\rb ,1)}; \addplot 3[]coordinates {(0,-\rb ,-1)(0,-\rb ,1)}; \addplot 3[-latex] coordinates {(0,0,0)(1.5,0,0)}node[below]{$x$}; \addplot 3[-latex] coordinates {(0,0,0)(0,1,0)}node[below]{$y$}; \addplot 3[-latex] coordinates {(0,0,0)(0,0,1.5)}node[left]{$z$}; \end {axis} \end {tikzpicture} \end {center} 
\item [
53)
]
 $x^2+y^2-z^2=4$ \begin {center} \begin {tikzpicture}[font=\small ,declare function={fz(\x ,\y )=sqrt((\x )^2+(\y )^2-4);fy(\x )=sqrt(9-\x ^2);}] \pgfmathsetmacro {\ra }{2} \pgfmathsetmacro {\rb }{3} \pgfmathsetmacro {\rc }{sqrt(5)} \begin {axis}[view/h=120,axis equal,axis lines=center,xtick={\empty },ytick={\empty },ztick={\empty },xlabel={$x$},ylabel={$y$},zlabel={$z$},xlabel style={anchor=east},ylabel style={anchor=west},zlabel style={anchor=east}, axis x line=none,axis y line=none,axis z line=none] \addplot 3[samples y=0,domain=\ra :\rb ] (x,0,{fz(x,0)}); \addplot 3[samples y=0,domain=\ra :\rb ] (x,0,{-fz(x,0)}); \addplot 3[samples y=0,domain=\ra :\rb ] (-x,0,{fz(x,0)}); \addplot 3[samples y=0,domain=\ra :\rb ] (-x,0,{-fz(x,0)}); \addplot 3[domain y=\ra :\rb ] (0,y,{fz(0,y)}); \addplot 3[domain y=\ra :\rb ] (0,y,{-fz(0,y)}); \addplot 3[domain y=\ra :\rb ] (0,-y,{fz(0,y)}); \addplot 3[domain y=\ra :\rb ] (0,-y,{-fz(0,y)}); \addplot 3[samples y=0,domain=-3:3](x,{fy(x)},\rc ); \addplot 3[samples y=0,domain=-3:3](x,{-fy(x)},\rc ); \addplot 3[samples y=0,domain=-3:3](x,{fy(x)},-\rc ); \addplot 3[samples y=0,domain=-3:3](x,{-fy(x)},-\rc ); \addplot 3[-latex] coordinates {(0,0,0)(2.5,0,0)}node[below]{$x$}; \addplot 3[-latex] coordinates {(0,0,0)(0,2.5,0)}node[below]{$y$}; \addplot 3[-latex] coordinates {(0,0,0)(0,0,2.5)}node[left]{$z$}; \end {axis} \end {tikzpicture} \end {center} 
\item [
55)
]
 $x^2+z^2=y$ \begin {center} \begin {tikzpicture}[font=\small ,declare function={fy(\x ,\z )=\x ^2+\z ^2;fz(\x )=sqrt(1-\x ^2);}] \pgfmathsetmacro {\ra }{1} \begin {axis}[view/h=120,axis equal,axis lines=center,xtick={\empty },ytick={\empty },ztick={\empty },xlabel={$x$},ylabel={$y$},zlabel={$z$},xlabel style={anchor=east},ylabel style={anchor=west},zlabel style={anchor=east}, axis x line=none,axis y line=none,axis z line=none] \addplot 3[samples y=0,domain=-\ra :\ra ] (x,{fy(x,0)},0); \addplot 3[domain y=-\ra :\ra ] (0,{fy(0,y)},y); \addplot 3[samples y=0,domain=-\ra :\ra ](x,1,{fz(x)}); \addplot 3[samples y=0,domain=-\ra :\ra ](x,1,{-fz(x)}); \addplot 3[-latex] coordinates {(0,0,0)(1,0,0)}node[below]{$x$}; \addplot 3[-latex] coordinates {(0,0,0)(0,2,0)}node[below]{$y$}; \addplot 3[-latex] coordinates {(0,0,0)(0,0,0.5)}node[left]{$z$}; \end {axis} \end {tikzpicture} \end {center} 
\item [
57)
]
 $x^2+z^2=1$ \begin {center} \begin {tikzpicture}[font=\small ,declare function={fz(\x )=sqrt(1-\x ^2);}] \pgfmathsetmacro {\ra }{1} \begin {axis}[view/h=120,axis equal,axis lines=center,xtick={\empty },ytick={\empty },ztick={\empty },xlabel={$x$},ylabel={$y$},zlabel={$z$},xlabel style={anchor=east},ylabel style={anchor=west},zlabel style={anchor=east}, axis x line=none,axis y line=none,axis z line=none] \addplot 3[samples y=0,domain=-\ra :\ra ] (x,1,{fz(x)}); \addplot 3[samples y=0,domain=-\ra :\ra ] (x,1,{-fz(x)}); \addplot 3[samples y=0,domain=-\ra :\ra ] (x,-1,{fz(x)}); \addplot 3[samples y=0,domain=-\ra :\ra ] (x,-1,{-fz(x)}); \addplot 3[]coordinates {(-1,-1,0)(-1,1,0)}; \addplot 3[]coordinates {(1,-1,0)(1,1,0)}; \addplot 3[]coordinates {(0,-1,1)(0,1,1)}; \addplot 3[]coordinates {(0,-1,-1)(0,1,-1)}; \addplot 3[-latex] coordinates {(0,0,0)(2,0,0)}node[below]{$x$}; \addplot 3[-latex] coordinates {(0,0,0)(0,2,0)}node[below]{$y$}; \addplot 3[-latex] coordinates {(0,0,0)(0,0,1.25)}node[left]{$z$}; \end {axis} \end {tikzpicture} \end {center} 
\item [
59)
]
 $16y^2+9z^2=4x^2$ \begin {center} \begin {tikzpicture}[font=\small ,declare function={fx(\y ,\z )=1/2*sqrt(16*\y ^2+9*\z ^2);fz(\y )=1/3*sqrt(9-16*\y ^2);}] \pgfmathsetmacro {\ra }{1} \pgfmathsetmacro {\rb }{3/4} \begin {axis}[view/h=120,axis equal,axis lines=center,xtick={\empty },ytick={\empty },ztick={\empty },xlabel={$x$},ylabel={$y$},zlabel={$z$},xlabel style={anchor=east},ylabel style={anchor=west},zlabel style={anchor=east}, axis x line=none,axis y line=none,axis z line=none] \addplot 3[domain y=-\ra :\ra ] ({fx(0,y)},0,y); \addplot 3[domain y=-\ra :\ra ] ({-fx(0,y)},0,y); \addplot 3[samples y=0,domain=-\rb :\rb ] ({fx(x,0)},x,0); \addplot 3[samples y=0,domain=-\rb :\rb ] ({-fx(x,0)},x,0); \addplot 3[samples y=0,domain=-\rb :\rb ] (1.5,x,{fz(x)}); \addplot 3[samples y=0,domain=-\rb :\rb ] (1.5,x,{-fz(x)}); \addplot 3[samples y=0,domain=-\rb :\rb ] (-1.5,x,{fz(x)}); \addplot 3[samples y=0,domain=-\rb :\rb ] (-1.5,x,{-fz(x)}); \addplot 3[-latex] coordinates {(0,0,0)(2,0,0)}node[below]{$x$}; \addplot 3[-latex] coordinates {(0,0,0)(0,2,0)}node[below]{$y$}; \addplot 3[-latex] coordinates {(0,0,0)(0,0,1.25)}node[left]{$z$}; \end {axis} \end {tikzpicture} \end {center} 
\item [
61)
]
 $9x^2+4y^2+z^2=36$ \begin {center} \begin {tikzpicture}[font=\small ,declare function={fz(\x ,\y )=sqrt(36-9*(\x )^2-4*(\y )^2);fy(\x )=1/2*sqrt(36-9*\x ^2);}] \pgfmathsetmacro {\ra }{3} \pgfmathsetmacro {\rb }{2} \begin {axis}[view/h=120,axis equal,axis lines=center,xtick={\empty },ytick={\empty },ztick={\empty },xlabel={$x$},ylabel={$y$},zlabel={$z$},xlabel style={anchor=east},ylabel style={anchor=west},zlabel style={anchor=east}, axis x line=none,axis y line=none,axis z line=none] \addplot 3[domain y=-\ra :\ra ] (0,y,{fz(0,y)}); \addplot 3[domain y=-\ra :\ra ] (0,y,{-fz(0,y)}); \addplot 3[samples y=0,domain=-\rb :\rb ] (x,0,{fz(x,0)}); \addplot 3[samples y=0,domain=-\rb :\rb ] (x,0,{-fz(x,0)}); \addplot 3[samples y=0,domain=-\rb :\rb ](x,{fy(x)},0); \addplot 3[samples y=0,domain=-\rb :\rb ](x,{-fy(x)},0); \addplot 3[-latex] coordinates {(0,0,0)(6,0,0)}node[below]{$x$}; \addplot 3[-latex] coordinates {(0,0,0)(0,5,0)}node[below]{$y$}; \addplot 3[-latex] coordinates {(0,0,0)(0,0,8)}node[left]{$z$}; \end {axis} \end {tikzpicture} \end {center} 
\item [
63)
]
 $x^2+y^2-16z^2=16$ \begin {center} \pgfmathsetmacro {\ra }{4} \pgfmathsetmacro {\rb }{8} \pgfmathsetmacro {\rc }{64} \pgfmathsetmacro {\rd }{sqrt(3)} \begin {tikzpicture}[font=\small ,declare function={fz(\x ,\y )=1/4*sqrt(\x ^2+\y ^2-16);fy(\x )=sqrt(\rc -\x ^2);}] \begin {axis}[view/h=120,axis equal,axis lines=center,xtick={\empty },ytick={\empty },ztick={\empty },xlabel={$x$},ylabel={$y$},zlabel={$z$},xlabel style={anchor=east},ylabel style={anchor=west},zlabel style={anchor=east}, axis x line=none,axis y line=none,axis z line=none] \addplot 3[domain y=\ra :\rb ] (0,y,{fz(0,y)}); \addplot 3[domain y=\ra :\rb ] (0,y,{-fz(0,y)}); \addplot 3[domain y=\ra :\rb ] (0,-y,{fz(0,y)}); \addplot 3[domain y=\ra :\rb ] (0,-y,{-fz(0,y)}); \addplot 3[samples y=0,domain=\ra :\rb ] (x,0,{fz(x,0)}); \addplot 3[samples y=0,domain=\ra :\rb ] (x,0,{-fz(x,0)}); \addplot 3[samples y=0,domain=\ra :\rb ] (-x,0,{fz(x,0)}); \addplot 3[samples y=0,domain=\ra :\rb ] (-x,0,{-fz(x,0)}); \addplot 3[samples y=0,domain=-\rb :\rb ](x,{fy(x)},\rd ); \addplot 3[samples y=0,domain=-\rb :\rb ](x,{-fy(x)},\rd ); \addplot 3[samples y=0,domain=-\rb :\rb ](x,{fy(x)},-\rd ); \addplot 3[samples y=0,domain=-\rb :\rb ](x,{-fy(x)},-\rd ); \addplot 3[-latex] coordinates {(0,0,0)(\rb +4,0,0)}node[below]{$x$}; \addplot 3[-latex] coordinates {(0,0,0)(0,\rb +2,0)}node[below]{$y$}; \addplot 3[-latex] coordinates {(0,0,0)(0,0,8)}node[left]{$z$}; \end {axis} \end {tikzpicture} \end {center} 
\item [
65)
]
 $z=-x^2-y^2$ \begin {center} \pgfmathsetmacro {\ra }{1} \begin {tikzpicture}[font=\small ,declare function={fz(\x ,\y )=-\x ^2-\y ^2;fy(\x )=sqrt(1-\x ^2);}] \begin {axis}[view/h=135,axis equal,axis lines=center,xtick={\empty },ytick={\empty },ztick={\empty },xlabel={$x$},ylabel={$y$},zlabel={$z$},xlabel style={anchor=east},ylabel style={anchor=west},zlabel style={anchor=east}, axis x line=none,axis y line=none,axis z line=none] \addplot 3[domain y=-\ra :\ra ] (0,y,{fz(0,y)}); \addplot 3[samples y=0,domain=-\ra :\ra ] (x,0,{fz(x,0)}); \addplot 3[samples y=0,domain=-\ra :\ra ](x,{fy(x)},-1); \addplot 3[samples y=0,domain=-\ra :\ra ](x,{-fy(x)},-1); \addplot 3[-latex] coordinates {(0,0,0)(1.5,0,0)}node[below]{$x$}; \addplot 3[-latex] coordinates {(0,0,0)(0,1.5,0)}node[below]{$y$}; \addplot 3[-latex] coordinates {(0,0,0)(0,0,0.5)}node[left]{$z$}; \end {axis} \end {tikzpicture} \end {center} 
\item [
67)
]
 $x^2-4y^2=1$ \begin {center} \pgfmathsetmacro {\ra }{1} \pgfmathsetmacro {\rb }{1} \begin {tikzpicture}[font=\small ,declare function={fx(\y )=sqrt(1+4*\y ^2);}] \begin {axis}[view/h=135,axis equal,axis lines=center,xtick={\empty },ytick={\empty },ztick={\empty },xlabel={$x$},ylabel={$y$},zlabel={$z$},xlabel style={anchor=east},ylabel style={anchor=west},zlabel style={anchor=east}, axis x line=none,axis y line=none,axis z line=none] \addplot 3[domain y=-\ra :\ra ] ({fx(y)},y,\rb ); \addplot 3[domain y=-\ra :\ra ] ({-fx(y)},y,\rb ); \addplot 3[domain y=-\ra :\ra ] ({fx(y)},y,-\rb ); \addplot 3[domain y=-\ra :\ra ] ({-fx(y)},y,-\rb ); \addplot 3[]coordinates {({fx(-\ra )},{-\ra },{-\rb }) ({fx(-\ra )},{-\ra },{\rb })}; \addplot 3[]coordinates {({fx(\ra )},{\ra },{-\rb }) ({fx(\ra )},{\ra },{\rb })}; \addplot 3[]coordinates {({-fx(-\ra )},{-\ra },{-\rb }) ({-fx(-\ra )},{-\ra },{\rb })}; \addplot 3[]coordinates {({-fx(\ra )},{\ra },{-\rb }) ({-fx(\ra )},{\ra },{\rb })}; \addplot 3[]coordinates {({fx(0)},{0},{-\rb }) ({fx(0)},{0},{\rb })}; \addplot 3[]coordinates {({-fx(0)},{0},{-\rb }) ({-fx(0)},{0},{\rb })}; \addplot 3[-latex] coordinates {(0,0,0)(4,0,0)}node[below]{$x$}; \addplot 3[-latex] coordinates {(0,0,0)(0,3,0)}node[below]{$y$}; \addplot 3[-latex] coordinates {(0,0,0)(0,0,2)}node[left]{$z$}; \end {axis} \end {tikzpicture} \end {center} 
\item [
69)
]
 $4y^2+z^2-4x^2=4$ \begin {center} \begin {tikzpicture}[font=\small ,declare function={fx(\y ,\z )=1/2*sqrt(4*(\y )^2+(\z )^2-4);fz(\x ,\y )=2*sqrt(1+(\x )^2-(\y )^2);}] \pgfmathsetmacro {\ra }{3} \pgfmathsetmacro {\rb }{3/2-0.001} \pgfmathsetmacro {\rc }{sqrt(5)/2} \begin {axis}[view/h=135,axis equal,axis lines=center,xtick={\empty },ytick={\empty },ztick={\empty },xlabel={$x$},ylabel={$y$},zlabel={$z$},xlabel style={anchor=east},ylabel style={anchor=west},zlabel style={anchor=east}, axis x line=none,axis y line=none,axis z line=none] \addplot 3[domain y=2:\ra ] ({fx(0,y)},0,y); \addplot 3[domain y=2:\ra ] ({-fx(0,y)},0,y); \addplot 3[domain y=2:\ra ] ({fx(0,y)},0,-y); \addplot 3[domain y=2:\ra ] ({-fx(0,y)},0,-y); \addplot 3[samples y=0,domain=1:\rb ] ({fx(x,0)},x,0); \addplot 3[samples y=0,domain=1:\rb ] ({-fx(x,0)},x,0); \addplot 3[samples y=0,domain=1:\rb ] ({fx(x,0)},-x,0); \addplot 3[samples y=0,domain=1:\rb ] ({-fx(x,0)},-x,0); \addplot 3[domain y=-\rb :\rb ]({\rc },{y},{fz(\rc ,y)}); \addplot 3[domain y=-\rb :\rb ]({\rc },{y},{-fz(\rc ,y)}); \addplot 3[domain y=-\rb :\rb ]({-\rc },{y},{fz(-\rc ,y)}); \addplot 3[domain y=-\rb :\rb ]({-\rc },{y},{-fz(-\rc ,y)}); \addplot 3[-latex] coordinates {(0,0,0)(2.5,0,0)}node[below]{$x$}; \addplot 3[-latex] coordinates {(0,0,0)(0,4.5,0)}node[below]{$y$}; \addplot 3[-latex] coordinates {(0,0,0)(0,0,4.5)}node[left]{$z$}; \end {axis} \end {tikzpicture} \end {center} 
\item [
71)
]
 $x^2+y^2=z$ \begin {center} \begin {tikzpicture}[font=\small ,declare function={fz(\x ,\y )=\x ^2+\y ^2;fy(\x )=sqrt(1-\x ^2);}] \pgfmathsetmacro {\ra }{1} \begin {axis}[view/h=135,axis equal,axis lines=center,xtick={\empty },ytick={\empty },ztick={\empty },xlabel={$x$},ylabel={$y$},zlabel={$z$},xlabel style={anchor=east},ylabel style={anchor=west},zlabel style={anchor=east}, axis x line=none,axis y line=none,axis z line=none] \addplot 3[samples y=0,domain=-\ra :\ra ] (x,0,{fz(x,0)}); \addplot 3[domain y=-\ra :\ra ] (0,y,{fz(0,y)}); \addplot 3[samples y=0,domain=-\ra :\ra ](x,{fy(x)},1); \addplot 3[samples y=0,domain=-\ra :\ra ](x,{-fy(x)},1); \addplot 3[-latex] coordinates {(0,0,0)(0.5,0,0)}node[below]{$x$}; \addplot 3[-latex] coordinates {(0,0,0)(0,0.5,0)}node[below]{$y$}; \addplot 3[-latex] coordinates {(0,0,0)(0,0,1.75)}node[left]{$z$}; \end {axis} \end {tikzpicture} \end {center} 
\item [
73)
]
 $yz=1$ \begin {center} \begin {tikzpicture}[font=\small ,declare function={fz(\y )=1/\y ;}] \pgfmathsetmacro {\ra }{0.3} \pgfmathsetmacro {\rb }{1/\ra } \pgfmathsetmacro {\rc }{1} \begin {axis}[view/h=135,axis equal,axis lines=center,xtick={\empty },ytick={\empty },ztick={\empty },xlabel={$x$},ylabel={$y$},zlabel={$z$},xlabel style={anchor=east},ylabel style={anchor=west},zlabel style={anchor=east}, axis x line=none,axis y line=none,axis z line=none] \addplot 3[domain y=\ra :\rb ] (\rc ,\y ,{fz(y)}); \addplot 3[domain y=\ra :\rb ] (-\rc ,\y ,{fz(y)}); \addplot 3[]coordinates {(\rc ,\ra ,{fz(\ra )})(-\rc ,\ra ,{fz(\ra )})}; \addplot 3[]coordinates {(\rc ,\rb ,{fz(\rb )})(-\rc ,\rb ,{fz(\rb )})}; \addplot 3[domain y=-\ra :-\rb ] (\rc ,\y ,{fz(y)}); \addplot 3[domain y=-\ra :-\rb ] (-\rc ,\y ,{fz(y)}); \addplot 3[]coordinates {(\rc ,-\ra ,{fz(-\ra )})(-\rc ,-\ra ,{-fz(\ra )})}; \addplot 3[]coordinates {(\rc ,-\rb ,{fz(-\rb )})(-\rc ,-\rb ,{-fz(\rb )})}; \addplot 3[-latex] coordinates {(0,0,0)(\rb +0.5,0,0)}node[below]{$x$}; \addplot 3[-latex] coordinates {(0,0,0)(0,\rb +0.5,0)}node[below]{$y$}; \addplot 3[-latex] coordinates {(0,0,0)(0,0,\rb +0.5)}node[left]{$z$}; \end {axis} \end {tikzpicture} \end {center} 
\item [
75)
]
 $9x^2+16y^2=4z^2$ \begin {center} \begin {tikzpicture}[font=\small ,declare function={fz(\x ,\y )=1/2*sqrt(9*\x ^2+16*\y ^2);fy(\x )=1/4*sqrt(16-9*\x ^2);}] \pgfmathsetmacro {\ra }{1} \pgfmathsetmacro {\rb }{4/3} \begin {axis}[view/h=135,axis equal,axis lines=center,xtick={\empty },ytick={\empty },ztick={\empty },xlabel={$x$},ylabel={$y$},zlabel={$z$},xlabel style={anchor=east},ylabel style={anchor=west},zlabel style={anchor=east}, axis x line=none,axis y line=none,axis z line=none] \addplot 3[domain y=-\ra :\ra ] (0,\y ,{fz(0,y)}); \addplot 3[domain y=-\ra :\ra ] (0,\y ,{-fz(0,y)}); \addplot 3[samples y=0,domain=-\rb :\rb ] (\x ,0,{fz(x,0)}); \addplot 3[samples y=0,domain=-\rb :\rb ] (\x ,0,{-fz(x,0)}); \addplot 3[samples y=0,domain=-\rb :\rb ] (\x ,{fy(x)},2); \addplot 3[samples y=0,domain=-\rb :\rb ] (\x ,{-fy(x)},2); \addplot 3[samples y=0,domain=-\rb :\rb ] (\x ,{fy(x)},-2); \addplot 3[samples y=0,domain=-\rb :\rb ] (\x ,{-fy(x)},-2); \addplot 3[-latex] coordinates {(0,0,0)(\ra +0.5,0,0)}node[below]{$x$}; \addplot 3[-latex] coordinates {(0,0,0)(0,\ra +0.5,0)}node[right]{$y$}; \addplot 3[-latex] coordinates {(0,0,0)(0,0,\ra +2.5)}node[left]{$z$}; \end {axis} \end {tikzpicture} \end {center} 
\item [
77)
]
 (ا) \عددی {\tfrac {2\pi (9-c^2)}{9}}، (ب) \عددی {8\pi }، (ج) \عددی {\tfrac {4\pi a b c}{3}} 
\item [
81)
]
 راس \عددی {(0,y_1,cy_1^2/b^22)}،\\ ماسکہ \عددی {(0,y_1,cy_1^2/b^2-a^2/(4c))} 
\end {description}
 {\urduTechTermsfont {حصہ}} 11.7\hskip 1em\relax {\urduTechTermsfont {صفحہ}} 1430
\begin {description}\setlength {\parskip }{0pt} \setlength {\itemsep }{0pt plus 1pt}
\item [
1)
]
 نلکی \عددی {(0,0,0)}، کروی \عددی {(0,0,0)} 
\item [
3)
]
 نلکی \عددی {(1,\pi /2,0)}، کروی \عددی {(1,\pi /2,\pi /2)} 
\item [
5)
]
 کارتیسی \عددی {(1,0,0)}، کروی \عددی {(1,\pi /2,0)} 
\item [
7)
]
 کارتیسی \عددی {(0,1,1)}، کروی \عددی {(\sqrt {2},\pi /4,\pi /2)} 
\item [
9)
]
 کارتیسی \عددی {(0,-2\sqrt {2},0)}،نلکی \عددی {(2\sqrt {2},3\pi /2,0)} 
\item [
11)
]
 \عددی {x^2+y^2=0}، \عددی {\phi =0} یا \عددی {\phi =\pi } یعنی محور \عددی {z} 
\item [
13)
]
 \عددی {z=0}، \عددی {\theta =\pi /2}، مستوی \عددی {xy} 
\item [
15)
]
 \عددی {z=\rho }، \عددی {0\le \rho \le 1}؛ \عددی {\theta =\pi /4}، \عددی {0\le r\le \sqrt {2}}؛ ایک محدود ترخیم 
\item [
17)
]
 \عددی {x=0}، \عددی {\phi =\pi /2}، مستوی \عددی {yz} 
\item [
19)
]
 \عددی {\rho ^2+z^2=4}، \عددی {r=2}؛ رداس \عددی {2} کا کرہ جس کا مرکز مبدا پر ہے۔ 
\item [
21)
]
 \عددی {x^2+y^2+(z-5/2)^2=25/4}، \عددی {\rho ^2+z^2=5z}، رداس \عددی {5/2} کا کرہ جس کا مرکز \عددی {(0,0,5/2)} (کارتیسی) ہے۔ 
\item [
23)
]
 \عددی {y=1}، \عددی {r\sin \theta \sin \phi =1}، مستوی \عددی {y=1} 
\item [
25)
]
 \عددی {z=\sqrt {2}}، سطح \عددی {z=\sqrt {2}} 
\item [
27)
]
 \عددی {\rho ^2+z^2=2z}، \عددی {z\le 1}؛ \عددی {r=2\cos \theta }، \عددی {\pi /4\le \theta \le \pi /2}؛ نچلا نصف کرہ جس کا رداس \عددی {1} اور مرکز \عددی {(0,0,1)} (کارتیسی) ہے۔ 
\item [
29)
]
 \عددی {x^2+y^2+z^2=9}، \عددی {-3/2\le z\le 3/2}؛ \عددی {\rho ^2+z^2=9}، \عددی {-3/2\le z\le 3/2}، رداس \عددی {3} کے کرہ کا وہ حصہ جو سطح \عددی {z=-3/2} اور سطح \عددی {z=3/2} کے بیچ ہے۔ کرہ کا مرکز مبدا پر ہے۔ 
\item [
31)
]
 \عددی {z=4-4(x^2+y^2)}، \عددی {0\le z\le 4}؛ \عددی {r\cos \theta =4-4 r^2\sin ^2\theta }، \عددی {0\le \theta \le \pi /2}، قطع مکافی سطح \عددی {z=4-4(x^2+y^2)} کا بالائی حصہ جس کو مستوی \عددی {xy} کاٹتا ہے۔ 
\item [
33)
]
 \عددی {z=-\sqrt {x^2+y^2}}، \عددی {-1\le z\le 0}؛ \عددی {z=-\rho }، \عددی {0\le \rho \le 1}، ترخیم جس کا راس مبدا پر ہے، اس کا قاعدہ ، مستوی \عددی {z=-1} میں دائرہ \عددی {x^2+y^2=1} ہے، 
\item [
35)
]
 \عددی {z+x^2-y^2=0} یا \عددی {z=y^2-x^2}، \عددی {\cos \theta +r\sin ^2\theta \cos 2\phi =0}، قطع زائد قطع مکافی سطح 
\item [
37)
]
 \عددی {(2,3,1)} 
\item [
39)
]
 مستوی \عددی {\rho \phi } میں دائرہ \عددی {\rho =-2\sin \phi } کا پیدا کردہ ، محور \عددی {z} کا متوازی قائمہ دائری بیلن ۔ \begin {center} \begin {tikzpicture}[declare function={fr(\t )=-2*sin(\t );}] \pgfmathsetmacro {\ta }{20} \pgfmathsetmacro {\tb }{290} \begin {axis}[axis equal,axis lines=center,view/h=120, axis x line=none, axis y line=none, axis z line=none] \addplot 3[smooth,data cs=polar,samples y=0,domain=0:360](x,{fr(x)},1); \addplot 3[smooth,data cs=polar,samples y=0,domain=0:360](x,{fr(x)},0); \addplot 3[smooth,data cs=polar,samples y=0,domain=0:360](x,{fr(x)},-1); \addplot 3[data cs=polar]coordinates{(\ta ,{fr(\ta )},-1)(\ta ,{fr(\ta )},1)}; \addplot 3[data cs=polar]coordinates{(\tb ,{fr(\tb )},-1)(\tb ,{fr(\tb )},1)}; \addplot 3[-latex]coordinates {(0,0,0)(3,0,0)}node[left]{$x$}; \addplot 3[-latex]coordinates {(0,0,0)(0,1,0)}node[right]{$y$}; \addplot 3[-latex]coordinates {(0,0,0)(0,0,2)}node[right]{$z$}; \end {axis} \end {tikzpicture} \end {center} 
\item [
41)
]
 محور \عددی {z} کے متوازی لکیروں کی پیدا کردہ نلکی جس کو مستوی \عددی {\rho \phi } میں قلب نما \عددی {\rho =1-\cos \phi } پیدا کرتا ہے۔ \begin {center} \begin {tikzpicture}[declare function={fr(\t )=1-cos(\t );}] \pgfmathsetmacro {\ta }{135} \pgfmathsetmacro {\tb }{250} \begin {axis}[axis equal,axis lines=center,view/h=110,axis x line=none, axis y line=none, axis z line=none] \addplot 3[smooth,data cs=polar,samples y=0,domain=0:360](x,{fr(x)},1); \addplot 3[smooth,data cs=polar,samples y=0,domain=0:360](x,{fr(x)},0); \addplot 3[smooth,data cs=polar,samples y=0,domain=0:360](x,{fr(x)},-1); \addplot 3[data cs=polar]coordinates{(\ta ,{fr(\ta )},-1)(\ta ,{fr(\ta )},1)}; \addplot 3[data cs=polar]coordinates{(\tb ,{fr(\tb )},-1)(\tb ,{fr(\tb )},1)}; \addplot 3[-latex]coordinates {(0,0,0)(2.5,0,0)}node[left]{$x$}; \addplot 3[-latex]coordinates {(0,0,0)(0,2,0)}node[right]{$y$}; \addplot 3[-latex]coordinates {(0,0,0)(0,0,2.5)}node[left]{$z$}; \end {axis} \end {tikzpicture} \end {center} 
\item [
43)
]
 سطح طواف قلب نما جو محور \عددی {y} کے لحاض سے تشاکلی ہے۔ مبدا پر کنگرہ نیچے رخ ہے۔ \begin {center} \begin {tikzpicture}[declare function={fx(\t ,\p )=(1-cos(\t ))*sin(\t )*cos(\p );fy(\t ,\p )=(1-cos(\t ))*sin(\t )*sin(\p );fz(\t ,\p )=(1-cos(\t ))*cos(\t );}] \pgfmathsetmacro {\pa }{45} \pgfmathsetmacro {\pb }{90} \pgfmathsetmacro {\pc }{135} \pgfmathsetmacro {\ta }{85} \begin {axis}[axis equal,axis lines=center,view/h=110,axis x line=none, axis y line=none, axis z line=none] \addplot 3[smooth,domain y=0:360]({fx(\ta ,y)},{fy(\ta ,y)},{fz(\ta ,y)}); \addplot 3[smooth,domain=0:180,samples y=0]({fx(x,\pa )},{fy(x,\pa )},{fz(x,\pa )}); \addplot 3[smooth,domain=0:180,samples y=0]({fx(x,\pb )},{fy(x,\pb )},{fz(x,\pb )}); \addplot 3[smooth,domain=0:180,samples y=0]({fx(x,\pc )},{fy(x,\pc )},{fz(x,\pc )}); \addplot 3[smooth,domain=0:180,samples y=0]({fx(x,-\pa )},{fy(x,-\pa )},{fz(x,-\pa )}); \addplot 3[smooth,domain=0:180,samples y=0]({fx(x,-\pb )},{fy(x,-\pb )},{fz(x,-\pb )}); \addplot 3[smooth,domain=0:180,samples y=0]({fx(x,-\pc )},{fy(x,-\pc )},{fz(x,-\pc )}); \addplot 3[-latex]coordinates {(0,0,0)(3.5,0,0)}node[left]{$x$}; \addplot 3[-latex]coordinates {(0,0,0)(0,1.5,0)}node[right]{$y$}; \addplot 3[-latex]coordinates {(0,0,0)(0,0,1)}node[left]{$z$}; \end {axis} \end {tikzpicture} \end {center} 
\item [
45)
]
 (ب) \عددی {\theta =\pi /2} 
\item [
49)
]
 سطح کی مساوات \عددی {\rho =f(z)} ہمیں بتاتی ہے کہ نقطہ \عددی {(\rho ,\phi ,z)=(f(z),\phi ,z)} تمام \عددی {\phi } کے لئے سطح پر واقع ہو گا۔ بالخصوص جس بھی \عددی {(f(z),\phi ,z)} اس سطح پر پایا جاتا ہو اس وقت \عددی {(f(z),\phi +\pi ,z) } بھی اس سطح پر پایا جائے گا لہٰذا محور \عددی {z} کے لحاض سے یہ سطح تشاکلی ہے۔ 
\end {description}
 {\urduTechTermsfont {حصہ}} 12.1\hskip 1em\relax {\urduTechTermsfont {صفحہ}} 1447
\begin {description}\setlength {\parskip }{0pt} \setlength {\itemsep }{0pt plus 1pt}
\item [
1)
]
 $y=x^2-2x,\, \kvec {v}=\ai +2\aj ,\, \kvec {a}=2\aj $ 
\item [
3)
]
 $y=\tfrac {2}{9}x^2,\,\kvec {v}=3\ai +4\aj ,\,\kvec {a}=3\ai +8\aj $ 
\item [
5)
]
 $t=\tfrac {\pi }{4}:\,\kvec {v}=\tfrac {\sqrt {2}}{2}\ai -\tfrac {\sqrt {2}}{2}\aj ,\,\kvec {a}=-\tfrac {\sqrt {2}}{2}\ai -\tfrac {\sqrt {2}}{2}\aj ;\, t=\tfrac {\pi }{2}:\,\kvec {v}=-\aj ,\,\kvec {a}=-\ai $ \begin {center} \begin {tikzpicture}[font=\small ] \pgfmathsetmacro {\r }{1.25} \draw [-latex](-1.25*\r ,0)--(1.75*\r ,0)node[right]{$x$}; \draw [-latex](0,-1.25*\r )--(0,1.5*\r )node[right]{$y$}; \draw (0,\r )node[left,yshift=1ex]{$1$}; \draw (0,0)node[below left]{$O$} circle (\r ); \draw [thick,latex-](0,0)--++(45:\r )node[pos=0.6,shift={(135:0.2)},yshift=1ex]{$\kvec {a}(\tfrac {\pi }{4})$}; \draw [thick,-latex](45:\r )-++(-45:\r )node[pos=0.5,above right]{$\kvec {v}(\tfrac {\pi }{4})$}; \draw [thick,-latex](\r ,0)--++(0,-\r )node[pos=0.5,right]{$\kvec {v}(\tfrac {\pi }{2})$}; \draw [thick,-latex](\r ,0)--(0,0)node[pos=0.5,below]{$\kvec {a}(\tfrac {\pi }{1})$}; \end {tikzpicture} \end {center} 
\item [
7)
]
 $t=\pi :\, \kvec {v}=2\ai ,\,\kvec {a}=-\aj ; \, t=\tfrac {3\pi }{2}:\,\kvec {v}=\ai -\aj ,\, \kvec {a}=-\ai $ \begin {center} \begin {tikzpicture}[scale=0.75,declare function={fx(\x )=2*pi/360*\x -sin(\x );fy(\x )=1-cos(\x );}] \draw [-latex](0,0)--(7,0)node[right]{$x$}; \draw [-latex](0,0)--(0,2.25)node[right]{$y$}; \draw (0.1,1)--++(-0.2,0)node[left]{$1$}; \draw (0.1,2)--++(-0.2,0)node[left]{$2$}; \draw (pi,0.1)--++(0,-0.2)node[below]{$\pi $}; \draw (2*pi,0.1)--++(0,-0.2)node[below]{$2\pi $}; \draw [domain=0:360]plot ({fx(\x )},{fy(\x )}); \draw [-latex]({fx(180)},{fy(180)})node[circ]{}node[pin={[pin edge=-]135:{$t=\pi $}}]{}--++(2,0)node[pos=0.5,above]{$\kvec {v}(\pi )$}; \draw [-latex]({fx(180)},{fy(180)})--++(0,-1)node[pos=0.5,left]{$\kvec {a}(\pi )$}; \draw [-latex]({fx(270)},{fy(270)})node[circ]{}node[pin={[pin edge=-]45:{$t=\tfrac {3\pi }{2}$}}]{}--++(1,-1)node[pos=0.5,above right]{$\kvec {v}(\tfrac {3\pi }{2})$}; \draw [-latex]({fx(270)},{fy(270)})--++(-1,0)node[pos=0.5,below]{$\kvec {a}(\tfrac {3\pi }{2})$}; \end {tikzpicture} \end {center} 
\item [
9)
]
 $\kvec {v}=\ai +2t\aj +2\ak ;\, \kvec {a}=2\aj ;\, \text {رفتار} 3; \text {رخ} \tfrac {1}{3}\ai +\tfrac {2}{3}\aj +\tfrac {2}{3}\ak ;\kvec {v(1)}=3(\tfrac {1}{3}\ai +\tfrac {2}{3}\aj +\tfrac {2}{3}\ak )$ 
\item [
11)
]
 $\kvec {v}=(-2\sin t)\ai +(3\cos t)\aj +4\ak ;\, \kvec {a}=(-2\cos t)\ai -(3\sin t)\aj ;\text {رفتار}2\sqrt {5};$\\ $ \text {رخ}\tfrac {-1}{\sqrt {5}}\ai +\tfrac {2}{\sqrt {5}}\ak ;\kvec {v}(\pi /2)=2\sqrt {5}[-\tfrac {1}{\sqrt {5}}\ai +\tfrac {2}{\sqrt {5}}\ak ]$ 
\item [
13)
]
 $\kvec {v}=(\tfrac {2}{t+1})\ai +2t\aj +t\ak ;\kvec {a}=\tfrac {-2}{(t+1)^2}\ai +2\aj +\ak ; \text {رفتار} \sqrt {6};\text {رخ} \tfrac {1}{\sqrt {6}}\ai +\tfrac {2}{\sqrt {6}}\aj +\tfrac {1}{\sqrt {6}}\ak ;\, \kvec {v}(1)=\sqrt {6}(\tfrac {1}{\sqrt {6}}\ai +\tfrac {2}{\sqrt {6}}\aj +\tfrac {1}{\sqrt {6}}\ak )$ 
\item [
15)
]
 $\tfrac {\pi }{2}$ 
\item [
17)
]
 $\tfrac {\pi }{2}$ 
\item [
19)
]
 $t=0,\, \pi ,\, 2\pi $ 
\item [
21)
]
 $\tfrac {1}{4}\ai +7\aj +\tfrac {3}{2}\ak $ 
\item [
23)
]
 $\tfrac {\pi +2\sqrt {2}}{2}\aj +2\ak $ 
\item [
25)
]
 $(\ln 4)\ai +(\ln 4)\aj +(\ln 2)\ak $ 
\item [
27)
]
 $\kvec {r}(t)=(\tfrac {-t^2}{2}+1)\ai +(\tfrac {-t^2}{2}+2)\aj +(\tfrac {-t^2}{2}+3)\ak $ 
\item [
29)
]
 $\kvec {r}(t)=((t+1)^{3/2}-1)\ai +(-e^{-t}+1)\aj +(\ln (t+1)+1)\ak $ 
\item [
31)
]
 $\kvec {r}(t)=8t\ai +8t\aj +(-16t^2+100)\ak $ 
\item [
33)
]
 $x=t,\, y=-1,\, z=1+t$ 
\item [
35)
]
 $x=at,\, y=a,\, z=2\pi b+bt$ 
\item [
37)
]
 ا) (1) مستقل رفتار \عددی {1}؛ (2) جی ہاں (3) گھڑی کے مخالف رخ (4) جی ہاں\\ ب) (1) مستقل رفتار \عددی {2} (2) جی ہاں (3) گھڑی کے مخالف رخ (4) جی ہاں\\ ج) (1) مستقل رفتار \عددی {1} (2) جی ہاں (3) گھڑی کے مخالف رخ (4) یہ \عددی {(1,0)} کی بجائے \عددی {(0,-1)} سے ابتدا کرتا ہے\\ د) (1) مستقل رفتار \عددی {1} (2) جی ہاں (3) گھڑی کے رخ (4) جی ہاں\\ ہ) (1) متغیر رفتار (2) نہیں (3) گھڑی کے مخالف رخ (4) جی ہاں 
\item [
39)
]
 $\kvec {r}(t)=(\tfrac {3}{2}t^2+\tfrac {6}{\sqrt {11}}t+1)\ai -(\tfrac {1}{2}t^2+\tfrac {2}{\sqrt {11}}t-2)\aj +(\tfrac {1}{2}t^2+\tfrac {2}{\sqrt {11}}t+3)\ak =(\tfrac {1}{2}t^2+\tfrac {2t}{\sqrt {11}})(3\ai -\aj +\ak )+(\ai +2\aj +3\ak )$ 
\item [
41)
]
 $\kvec {v}(t)=2\sqrt {5}\ai +\sqrt {5}\aj $ 
\item [
43)
]
 زیادہ سے زیادہ \عددی {\abs {\kvec {v}}=3}، کم سے کم \عددی {\abs {\kvec {v}}=2}، زیادہ سے زیادہ \عددی {\abs {\kvec {a}}=3}، کم سے کم \عددی {\abs {\kvec {a}}=2} 
\end {description}
 {\urduTechTermsfont {حصہ}} 12.2\hskip 1em\relax {\urduTechTermsfont {صفحہ}} 1464
\begin {description}\setlength {\parskip }{0pt} \setlength {\itemsep }{0pt plus 1pt}
\item [
1)
]
 $\SI {50}{\second }$ 
\item [
3)
]
 (ا) \عددی {\SI {72.2}{\second }}، \عددی {\SI {25510}{\meter }}؛ (ب) \عددی {\SI {4020}{\meter }}؛ (ج) \عددی {\SI {6378}{\meter }} 
\item [
5)
]
 $t\approx \SI {2.1257}{\second },\quad x\approx \SI {20.14}{\meter }$ 
\item [
7)
]
 $v_0=\SI {9.9}{\meter \per \second },\,\alpha =18.4^{\circ },\alpha =71.6^{\circ }$ 
\item [
9)
]
 $\SI {174}{\kilo \meter \per \hour }$ 
\item [
11)
]
 گیند درخت کے پتوں کو چھوتا ہوا اسے پار کر پائے گا۔ 
\item [
13)
]
 \عددی {\SI {24.87}{\meter \per \second }} 
\item [
17)
]
 \عددی {\SI {141}{\percent }} 
\item [
21)
]
 \عددی {1.789} سیکنڈ، \عددی {\SI {19.92}{\meter }} 
\item [
25)
]
 $\kvec {v}(t)=-gt\ak +\kvec {v}_0,\, \kvec {r}(t)=-\tfrac {1}{2}gt^2\ak +\kvec {v}_0t$ 
\end {description}
 {\urduTechTermsfont {حصہ}} 12.3\hskip 1em\relax {\urduTechTermsfont {صفحہ}} 1473
\begin {description}\setlength {\parskip }{0pt} \setlength {\itemsep }{0pt plus 1pt}
\item [
1)
]
 $\kvec {T}=(-\tfrac {2}{3}\sin t)\ai +(\tfrac {2}{3}\cos t)\aj +\tfrac {\sqrt {5}}{3}\ak ,\quad 3\pi $ 
\item [
3)
]
 $\kvec {T}=\tfrac {1}{\sqrt {1+t}}\ai +\tfrac {\sqrt {t}}{\sqrt {1+t}}\ak ,\quad \tfrac {52}{3}$ 
\item [
5)
]
 $\kvec {T}=-\cos t\aj +\sin t\ak ,\quad \tfrac {3}{2}$ 
\item [
7)
]
 $\kvec {T}=(\tfrac {\cos t-t\sin t}{t+1})\ai +(\tfrac {\sin t+t\cos t}{t+1})\aj +(\tfrac {\sqrt {2}t^{1/2}}{t+1})\ak ,\quad \tfrac {\pi ^2}{2}+\pi $ 
\item [
9)
]
 $(0,5,24\pi )$ 
\item [
11)
]
 $s(t)=5t,\quad L=\tfrac {5\pi }{2}$ 
\item [
13)
]
 $s(t)=\sqrt {3}e^t-\sqrt {3},\quad L=\tfrac {3\sqrt {3}}{4}$ 
\item [
15)
]
 $\sqrt {2}+\ln (1+\sqrt {2})$ 
\item [
17)
]
 ا) نلکی \عددی {x^2+y^2=1} اور مستوی \عددی {x+z=1}\\ د) \عددی {L=\int _0^{2\pi }\sqrt {1+\sin ^2t}\dif t}\\ ہ) \عددی {L\approx 7.64} 
\end {description}
 {\urduTechTermsfont {حصہ}} 12.4\hskip 1em\relax {\urduTechTermsfont {صفحہ}} 1489
\begin {description}\setlength {\parskip }{0pt} \setlength {\itemsep }{0pt plus 1pt}
\item [
1)
]
 $\kvec {T}=(\cos t)\ai -(\sin t)\aj ,\, \kvec {N}=(-\sin t)\ai -(\cos t)\aj ,\,\kappa =\cos t$ 
\item [
3)
]
 $\kvec {T}=\tfrac {1}{\sqrt {1+t^2}}\ai -\tfrac {t}{\sqrt {1+t^2}}\aj ,\,\kvec {N}=\tfrac {-t}{\sqrt {1+t^2}}\ai -\tfrac {1}{\sqrt {1+t^2}}\aj ,\,\kappa =\tfrac {1}{2(\sqrt {1+t^2})^3}$ 
\item [
5)
]
 $\kvec {a}=\tfrac {2t}{\sqrt {1+t^2}}\kvec {T}+\tfrac {2}{\sqrt {1+t^2}}\kvec {N}$ 
\item [
7)
]
 (ب) \عددی {\cos x} 
\item [
9)
]
 (ب) \عددی {\kvec {N}=\tfrac {-2e^{2t}}{\sqrt {1+4e^{4t}}}\ai +\tfrac {1}{\sqrt {1+4e^{4t}}}\aj } ، (ج) \عددی {\kvec {N}=-\tfrac {1}{2}(\sqrt {4-t^2}\ai +t\aj )} 
\item [
11)
]
 $\kvec {T}=\tfrac {3\cos t}{5}\ai -\tfrac {3\sin t}{5}\aj +\tfrac {4}{5}\ak ,\,\kvec {N}=(-\sin t)\ai -(\cos t)\aj ,\,\kvec {B}=(\tfrac {4}{5}\cos t)\ai -(\tfrac {4}{5}\sin t)\aj -\tfrac {3}{5}\ak ,\,\kappa =\tfrac {3}{25},\,\tau =-\tfrac {4}{25}$ 
\item [
13)
]
 $\kvec {T}=(\tfrac {\cos t-\sin t}{\sqrt {2}})\ai +(\tfrac {\cos t+\sin t}{\sqrt {2}})\aj ,\,\kvec {N}=(\tfrac {-\cos t-\sin t}{\sqrt {2}})\ai +(\tfrac {-\sin t+\cos t}{\sqrt {2}})\aj ,\,\kvec {B}=\ak ,\,\kappa =\tfrac {1}{e^t\sqrt {2}},\, \tau =0$ 
\item [
15)
]
 $\kvec {T}=\tfrac {t}{\sqrt {t^2+1}}\ai +\tfrac {1}{\sqrt {t^2+1}}\aj ,\,\kvec {N}=\tfrac {\ai }{\sqrt {t^2+1}}-\tfrac {t\aj }{\sqrt {t^2+1}},\, \kvec {B}=-\ak ,\,\kappa =\tfrac {1}{t(t^2+1)^{3/2}},\,\tau =0$ 
\item [
17)
]
 $\kvec {T}=(\sech \tfrac {t}{a})\ai +(\tanh \tfrac {t}{a})\aj ,\,\kvec {N}=(-\tanh \tfrac {t}{a})\ai +(\sech \tfrac {t}{a})\aj ,\,\kvec {B}=\ak ,\,\kappa =\tfrac {1}{a}\sech ^2\tfrac {t}{a},\,\tau =0$ 
\item [
19)
]
 $\kvec {a}=\abs {a}\kvec {N}$ 
\item [
21)
]
 (ا) \عددی {\kvec {a}(1)=\tfrac {4}{3}\kvec {T}+\tfrac {2\sqrt {5}}{3}\kvec {N}} 
\item [
23)
]
 $\kvec {a}(0)=2\kvec {N}$ 
\item [
25)
]
 $\kvec {r}(\tfrac {\pi }{4})=\tfrac {\sqrt {2}}{2}\ai +\tfrac {\sqrt {2}}{2}\aj -\ak ,\,\kvec {T}(\tfrac {\pi }{4})=-\tfrac {\sqrt {2}}{2}\ai +\tfrac {\sqrt {2}}{2}\aj ,\,\kvec {N}(\tfrac {\pi }{4})=-\tfrac {\sqrt {2}}{2}\ai -\tfrac {\sqrt {2}}{2}\aj ,\,\kvec {B}(\tfrac {\pi }{4})=\ak ;$\\ مستوی دائرہ انحنا \عددی {z=-1} ہے؛ عمودی مستوی \عددی {-x+y=0} ہے؛ سمت کار مستوی \عددی {x+y=\sqrt {2}} ہے۔ 
\item [
27)
]
 جی ہاں۔اگر گاڑی مڑتی سڑک \عددی { (\kappa \ne 0)} پر چل رہی ہو تب \عددی {a_N=\kappa \abs {\kvec {v}}^2\ne 0} اور \عددی {\kvec {a}\ne \kvec {0}} ہو گا۔ 
\item [
31)
]
 $\abs {\kvec {F}}=\kappa (m(\tfrac {\dif s}{\dif t})^2)$ 
\item [
35)
]
 $\tfrac {1}{2b}$ 
\item [
39)
]
 (ا) \عددی {b-a}، (ب) \عددی {\pi } 
\item [
45)
]
 $\kappa (x)=\tfrac {2}{(1+4x^2)^{3/2}}$ 
\item [
47)
]
 $\kappa (x)=\tfrac {\abs {\sin x}}{(1+\cos ^2x)^{3/2}}$ 
\item [
57)
]
 \عددی {\kvec {v}} کے اجزاء:\عددی {-1.8701}، \عددی {0.7089}، \عددی {1.0000}؛ \عددی {\kvec {a}} کے اجزاء:\عددی {-1.6960}، \عددی {-2.0307}، \عددی {0}؛ رفتار \عددی {2.2361}؛ \عددی {\kvec {T}} کے اجزاء: \عددی {-0.8364}، \عددی {0.3170}، \عددی {0.4472}؛ \عددی {\kvec {N}} کے اجزاء: \عددی {-0.4143}، \عددی {-0.8998}، \عددی {-0.1369}؛ \عددی {\kvec {B}} کے اجزاء:\عددی {0.3590}، \عددی {-0.2998}، \عددی {0.8839}؛ انحنا \عددی {0.5060}؛ مروڑ \عددی {0.2813}؛ اسراع کا مماسی جزو:\عددی {0.7746}؛ اسراع کا عمودی جزو \عددی {2.5298}؛ 
\item [
59)
]
 \عددی {\kvec {v}} کے اجزاء:\عددی {2.0000}، \عددی {0}، \عددی {0.1629}؛ \عددی {\kvec {a}} کے اجزاء:\عددی {0}، \عددی {-1.0000}، \عددی {0.0086}؛ رفتار \عددی {2.0066}؛ \عددی {\kvec {T}} کے اجزاء: \عددی {0.9967}، \عددی {0}، \عددی {0.0812}؛ \عددی {\kvec {N}} کے اجزاء: \عددی {-0.0007}، \عددی {-1.0000}، \عددی {0.0086}؛ \عددی {\kvec {B}} کے اجزاء:\عددی {0.0812}، \عددی {-0.0086}، \عددی {-0.9967}؛ انحنا \عددی {0.2484}؛ مروڑ \عددی {-0.0411}؛ اسراع کا مماسی جزو:\عددی {0.0007}؛ اسراع کا عمودی جزو \عددی {1.0000}؛ 
\end {description}
 {\urduTechTermsfont {حصہ}} 12.5\hskip 1em\relax {\urduTechTermsfont {صفحہ}} 1506
\begin {description}\setlength {\parskip }{0pt} \setlength {\itemsep }{0pt plus 1pt}
\item [
1)
]
 $T=\SI {93.2}{\minute }$ 
\item [
3)
]
 $a=\SI {6763}{\kilo \meter }$ 
\item [
5)
]
 $T=\SI {1655}{\minute }$ 
\item [
7)
]
 $a=\SI {20430}{\kilo \meter }$ 
\item [
9)
]
 $\abs {v}=1.9966\times 10^7r^{-1/2}\,\si {\meter \per \second }$ 
\item [
11)
]
 دائرہ: \عددی {v_0=\sqrt {\tfrac {GM}{r_0}}}؛ ترخیم: \عددی {\sqrt {\tfrac {GM}{r_0}}<v_0<\sqrt {\tfrac {2GM}{r_0}}}؛ قطع مکافی : \عددی {v_0=\sqrt {\tfrac {2GM}{r_0}}}؛ قطع زائد: \عددی {v_0>\sqrt {\tfrac {2GM}{r_0}}} 
\item [
15)
]
 (ا) \عددی {x(t)=2+(3-4\cos (\pi t))\cos (\pi t)}\\ \عددی {y(t)=(3-4\cos (\pi t))\sin (\pi t)} 
\item [
17)
]
 (ج) $\kvec {v}=\dot {r}\kvec {u}_r+r\dot {\theta }\kvec {u}_{\theta }+\dot {z}\ak $\\ $\kvec {a}=(\ddot {r}-r\dot {\theta }^2)\kvec {u}_r+(r\ddot {\theta }+2\dot {r}\dot {\theta })\kvec {u}_{\theta }+\ddot {z}\ak $ 
\item [
19)
]
 (ا) $\kvec {u}_r=\sin \theta \cos \phi \ai +\sin \theta \sin \phi \aj +\cos \theta \ak $\\ $\kvec {u}_{\theta }=\cos \theta \cos \phi \ai +\cos \theta \sin \phi \aj -\sin \theta \ak $\\ $\kvec {u}_{\phi }=-\sin \phi \ai +\cos \phi \aj $ 
\end {description}
