\باب{تکمل بالکثرت}\شناخت{باب_تکمل_بالکثرت}
\جزوحصہء{جائزہ}
تکمل سے حل دو اور تین متغیری تفاعل کی نوعیت   تکمل سے حل ایک متغیری تفاعل کے مسائل  کی طرح ہوتی ہے، بس یہ زیادہ عمومی  ہوتے ہیں۔ گزشتہ ابواب کی طرح ہم  ایک متغیری تفاعل   کی معلومات استعمال کرتے ہوئے  دو اور تین متغیری تفاعل کا حساب  آگے بڑھا سکتے ہیں۔

\حصہ{دوہرا تکملات}
ہم  \عددی{xy} مستوی میں محدود  خطہ  پر استمراری تفاعل \عددی{f(x,y)}  کا تکمل حاصل کرنا سکھاتے ہیں۔یہاں متعارف کیے جانے والا  دوہرا  (دو گنّا) تکمل اور باب \حوالہ{باب_تکمل} میں متعارف کردہ  ایک گنّا تکمل میں بہت ساری یکساں خوبیاں پائی جاتی ہیں۔ ہر دوہرا  تکمل کی قیمت  ایک گنّا تکمل کی ترکیب سے مراحل میں حاصل کی  جا سکتی ہے۔

\جزوحصہء{مستطیل پر دوہرا تکملات}
فرض کریں تفاعل \عددی{f(x,y)}درج ذیل  مستطیل خطہ \عددی{R} میں معین ہے۔
\begin{align*}
R:\quad a\le x\le b,\quad c\le y\le d
\end{align*}
ہم تصور میں   \عددی{R} پر \عددی{x} اور \عددی{y} محور کے متوازی لکیروں کا  ایک جال بچھاتے ہیں جو  \عددی{R} کو چھوٹے چھوٹے رقبوں \عددی{\Delta S=\Delta x\Delta y} میں تقسیم کرتے ہیں (شکل \حوالہ{شکل_بالکثرت_مستطیل_جال})۔ ہم ان رقبوں کو کسی ترتیب \عددی{\Delta S_1}، \عددی{\Delta S_2}، \نقطے، \عددی{\Delta S_n} سے  شمار کر کے ہر چھوٹے رقبہ  \عددی{\Delta S_k} میں  ایک نقطہ \عددی{(x_k,y_k)} منتخب کر کے  درج ذیل مجموعہ  \عددی{J_n} لیتے ہیں۔
\begin{align}\label{مساوات_کثیرالمتغیر_مجموعہ_الف}
J_n=\sum_{k=1}^{n}f(x_k,y_k)\Delta S_k
\end{align}
اگر پورے \عددی{R} میں \عددی{f} استمراری ہو، تب، ہم جال  کے خانوں  کو اتنا چھوٹا کر سکتے ہیں کہ   \عددی{\Delta x} اور \عددی{\Delta y}  دونوں صفر  تک پہنچنے کی کوشش کریں۔ ایسا کرنے سے  مساوات \حوالہ{مساوات_کثیرالمتغیر_مجموعہ_الف}  میں دیا گیا   مجموعہ ایک تحدیدی قیمت تک پہنچے گا جس کو \عددی{R} پر \عددی{f} کا د\اصطلاح{دوہرا تکمل}\فرہنگ{تکمل!دوہرا}\حاشیہب{double integral}\فرہنگ{integral!double} کہتے ہیں۔  اس کو علامتی طور پر
\begin{align*}
\iint\limits_R f(x,y)\dif S \quad \text{یا}\quad \iint\limits_R f(x,y)\dif x\dif y
\end{align*}
لکھا جاتا ہے۔یوں درج ذیل ہو گا۔
\begin{align}\label{مساوات_کثیرالمتغیر_مجموعہ_ب}
\iint\limits_R f(x,y)\dif S=\lim\limits_{\Delta S\to 0}\sum_{k=1}^n f(x_k,y_k)\Delta S_k
\end{align}
واحد متغیری تفاعل کی طرح،   جب تک  خانہ بندی کے دونوں معیار صفر تک پہنچتے ہوں،  وقفات   \عددی{[a,b]} اور \عددی{[c,d]} کی طرز تقسیم کا   مجموعہ کی  حد پر کوئی اثر نہیں پایا جائے گا۔ مساوات \حوالہ{مساوات_کثیرالمتغیر_مجموعہ_ب} میں  حد کی قیمت، نا تو  رقبات \عددی{\Delta S_k} کی ترتیب شمار پر اور نا ہی ہر \عددی{\Delta S_k} میں نقطہ \عددی{(x_k,y_k)}  کے مقام پر منحصر ہو گی۔ انفرادی مجموعات \عددی{J_n} کی قیمتیں ان پر ضرور منحصر ہوں گی لیکن   ان مجموعات  کا حد آخر میں  وہی ایک  ہو گا۔استمراری \عددی{f} کے لئے  اس حد کی وجودیت اور  یکتائی کے  ثبوت اعلٰی  نصاب میں دیے   جاتے ہیں۔دوہرا تکمل کی وجودیت  کے لئے   \عددی{f} کا  استمرار کافی لیکن غیر لازمی شرط  ہے۔ یہ حد بہت سارے غیر استمراری تفاعل کے لئے بھی موجود  ہے۔

\begin{figure}
\centering
\begin{minipage}{0.55\textwidth}
\centering
\begin{tikzpicture}[font=\small]
\pgfmathsetmacro{\a}{3.75}
\pgfmathsetmacro{\b}{2.5}
\draw[-latex](0,0)--++(4.5,0)node[right]{$x$};
\draw[-latex](0,0)--++(0,3.25)node[right]{$y$};
\draw[thick](0.4,0.25) rectangle ++(\a,\b)node[right]{$R$};
\draw(0.4,0)node[below]{$a$}--++(0,0.1)  (0.4,0)++(\a,0)node[below]{$b$}--++(0,0.1); 
\draw(0,0.25)node[left]{$c$}--++(0.1,0)  (0,0.25)++(0,\b)node[left]{$d$}--++(0.1,0); 
\foreach \x in {0.125,0.2,0.3,0.45,0.6,0.8,0.9} {\draw(0.4+\a*\x,0.125)--++(0,\b+0.25);}
\foreach \y in {0.2,0.3,0.5,0.7,0.8,0.9}{\draw(0.25,0.25+\b*\y)--++(\a+0.25,0);}
\draw[fill=lgray](0.4+0.45*\a,0.25+0.3*\b) rectangle (0.4+0.6*\a,0.25+0.5*\b);
\draw[stealth-stealth](0.4+0.45*\a,-0.1)--(0.4+0.6*\a,-0.1)node[pos=0.5,below]{$\Delta x_k$};
\draw[stealth-stealth](-0.1,0.25+0.3*\b)--(-0.1,0.25+0.5*\b)node[pos=0.5,left]{$\Delta y_k$};
\draw(0.4+0.475*\a,0.25+0.4*\b)--++(-0.3,0.3)node[above,fill=white]{$\Delta S_k$};
\draw(0.4+0.55*\a,0.25+0.45*\b)node[circ]{}--++(0.3,0.2)node[above right,fill=white]{$(x_k,y_k)$};
\end{tikzpicture}
\caption{
خطہ \عددی{R} کو مستطیل جال  چھوٹے  مستطیل خانوں میں تقسیم کرتا ہے جن کے رقبے   \عددی{\Delta S_k=\Delta x_k\Delta y_k}ہوں گے۔
}
\label{شکل_بالکثرت_مستطیل_جال}
\end{minipage}\hfill
\begin{minipage}{0.4\textwidth}
\centering
\begin{tikzpicture}
\pgfmathsetmacro{\a}{2.5}
\pgfmathsetmacro{\b}{2}
\draw(0,0) rectangle (\a,\b);
\draw(0.65*\a,0)--++(0,\b);
\draw(0.325*\a,0.5*\b)node[]{$R_1$};
\draw(0.82*\a,0.5*\b)node[]{$R_2$};
\draw(0.5*\a,0)node[below,font=\scriptsize]{$\iint\limits_{R_1\cup R_2}f(x,y)\dif S=\iint\limits_{R_1}f(x,y)\dif S+\iint\limits_{R_2}f(x,y)\dif S$};
\end{tikzpicture}
\caption{
دوہرا تکملات بھی ایک گنّا تکملات کی طرح  مجموعیت دائرہ کار کی خاصیت رکھتے ہیں۔
}
\label{شکل_بالکثرت_مجموعیت_دائرہ_کار}
\end{minipage}
\end{figure}


\جزوحصہء{دوہرا تکملات کے خواص}
ایک گنّا تکملات کی طرح، دوہرا تکملات کے ایسا الجبرائی خواص پائے جاتے ہیں جو حساب اور عملی استعمال کے لئے کارآمد ثابت ہوتے ہیں۔
\begin{enumerate}[a.]
\item\quad
$\iint\limits_R kf(x,y)\dif S=k\iint\limits_R f(x,y)\dif S$\quad
جہاں \عددی{k} کوئی مستقل ہے۔
\item\quad
$\iint\limits_R (f(x,y)\mp g(x,y))\dif S=\iint\limits_R f(x,y)\dif S\mp\iint\limits_R g(x,y)\dif S$
\item\quad
اگر \عددی{R} پر \عددی{f(x,y)\ge 0} ہو تب  \عددی{\iint\limits_R f(x,y)\dif S\ge 0} ہو گا۔
\item\quad
اگر \عددی{R} پر \عددی{f(x,y)\ge g(x,y)} ہو تب  \عددی{\iint\limits_R f(x,y)\dif S\ge \iint\limits_R g(x,y)\dif S} ہو گا۔\\
یہ خواص ایک گنّا تکملات  کے خواص کی طرح ہیں (حصہ \حوالہ{حصہ_تکمل_خصوصیات_رقبہ_اوسط_قیمت_مسئلہ})۔ ان کے علاوہ  درج ذیل مجموعیت کا خواص بھی پایا جاتا ہے
\item\quad
$\iint\limits_R f(x,y)\dif S=\int\limits_{R_1}f(x,y)\dif S+\iint\limits_{R_2} f(x,y)\dif S$\\
جہاں  ایک دوسرے کو نا ڈھانپنے والے مستطیل   \عددی{R_1} اور \عددی{R_2} خطوں  کا  اشتراک    \عددی{R} ہے (شکل \حوالہ{شکل_بالکثرت_مجموعیت_دائرہ_کار})۔ یہاں بھی ہم ثبوت پیش نہیں کریں گے۔
\end{enumerate}

\جزوحصہء{دوہرا تکملات بطور حجم}
مثبت \عددی{f(x,y)} کی صورت میں ہم مستطیل خطہ \عددی{R} پر \عددی{f} کے دوہرا تکمل کو  ٹھوس  منشور نما  کا حجم تصور کر سکتے ہیں جس کی  نچلا سطح \عددی{R} اور بالائی سطح \عددی{z=f(x,y)} ہو گی (شکل \حوالہ{شکل_بالکثرت_ٹھوس_جسم_حجم_منشور})۔ مجموعہ \عددی{J_n=\sum f(x_k,y_k)\Delta S_k} میں ہر رکن \عددی{f(x_k,y_k)\Delta S_k} ایک   انتصابی مستطیلی   منشور نما  کا حجم ہو گا جو   جو    بنیاد \عددی{\Delta S_k} پر  سیدھا کھڑے ٹھوس خطے  کے حجم کی تخمینی قیمت  ہو گی۔     یوں مجموعہ \عددی{J_n}   پورے ٹھوس جسم کے حجم کی تخمین  ہو گی۔ اس حجم کی تعریف درج ذیل  ہے۔
\begin{align}\label{مساوات_دوہرا_فوبینی_الف}
\text{حجم}=\lim J_n=\iint\limits_R f(x,y)\dif S
\end{align}
جیسا ہم توقع کرتے ہیں، حجم تلاش کرنے کی  مذکورہ بالا زیادہ عمومی ترکیب  سے حاصل نتائج ، باب \حوالہ{باب_تکمل_کا_استعمال} میں پیش کی گئی ترکیب کے نتائج کے عین مطابق ہیں۔ ہم اس حقیقت کا ثبوت یہاں پیش نہیں کریں گے۔ 

\begin{figure}
\centering
\begin{tikzpicture}[font=\small,declare function={f(\x,\y)=4-(\x)^2-(\y)^2;}]
\pgfmathsetmacro{\a}{0.25}
\pgfmathsetmacro{\b}{0.8}
\pgfmathsetmacro{\c}{0.4}
\pgfmathsetmacro{\d}{0.8}
\pgfmathsetmacro{\aa}{0.5}
\pgfmathsetmacro{\bb}{0.62}
\pgfmathsetmacro{\cc}{0.5}
\pgfmathsetmacro{\dd}{0.62}
\pgfmathsetmacro{\delA}{0.1}
\pgfmathsetmacro{\delB}{0.15}
\pgfmathsetmacro{\delC}{1/2*(\delA+\delB)}
\pgfmathsetmacro{\mx}{\aa+1/3*(\bb-\aa)}
\pgfmathsetmacro{\my}{\cc+2/3*(\dd-\cc)}
\begin{axis}[clip=false,small,axis lines=center,view/h=130,colormap={}{gray(0cm)=(0.6);gray(1cm)=(0.9);},enlargelimits=true,xlabel={$x$},ylabel={$y$},zlabel={$z$},xtick={\a,\b},ytick={\c,\d},ztick={\empty},xticklabels={$a$,$b$},yticklabels={$c$,$d$},xmin=0,ymin=0,zmin=0,xlabel style={anchor=east},ylabel style={anchor=west}]
\addplot3[]coordinates{(\a,\c,{f(\a,\c)})(\a,\c,0)};
\addplot3[]coordinates{(\b,\c,{f(\b,\c)})(\b,\c,0)};
\addplot3[]coordinates{(\b,\d,{f(\b,\d)})(\b,\d,0)};
\addplot3[]coordinates{(\a,\d,{f(\a,\d)})(\a,\d,0)};
\addplot3[surf,domain=\a:\b,domain y=\c:\d]{f(x,y)};
\addplot3[fill=llgray]coordinates{(\a,\c,0)(\b,\c,0)(\b,\d,0)(\a,\d,0)(\a,\c,0)};
\addplot3[]coordinates{(\aa,\cc,{f(\aa,\cc)})(\bb,\cc,{f(\bb,\cc)})(\bb,\dd,{f(\bb,\dd)})(\aa,\dd,{f(\aa,\dd)})(\aa,\cc,{f(\aa,\cc)})};
\addplot3[]coordinates{(\aa,\cc,0)(\bb,\cc,0)(\bb,\dd,0)(\aa,\dd,0)(\aa,\cc,0)};
\addplot[dashed]coordinates{(\a,0)(\a,\c)};
\addplot[dashed]coordinates{(\b,0)(\b,\c)};
\addplot[dashed]coordinates{(\a,\c)(0,\c)};
\addplot[dashed]coordinates{(\a,\d)(0,\d)};
\addplot3[]coordinates{(\mx,\my,{f(\mx,\my)})}node[circ]{};
\addplot3[]coordinates{(\mx,\my,0)}node[circ]{}node[pin=-45:{$(x_k,y_k)$}]{}node[pin=-130:{$\Delta S_k$}]{};
\addplot3[]coordinates{(\mx,\my+\delA,{f(\mx,\my+\delA)})(\mx,\my+\delB,{f(\mx,\my+\delB)})};
\addplot3[]coordinates{(\mx,\my+\delA,0)(\mx,\my+\delB,0)};
\addplot3[stealth-stealth] coordinates{(\mx,\my+\delC,{f(\mx,\my+\delC)})(\mx,\my+\delC,0)}node[pos=0.3,pin={[pin edge=-,right,pin distance=1.25cm]30:{$z=f(x_k,y_k)$}}]{};
\addplot3[]coordinates{(\a,\c+0.1,{f(\a,\c+0.1)})}node[pin={30:{$z=f(x,y)$}}]{};
\addplot3[]coordinates {(\aa,\cc,{f(\aa,\cc)})(\aa,\cc,0)};
\addplot3[]coordinates {(\bb,\cc,{f(\bb,\cc)})(\bb,\cc,0)};
\addplot3[]coordinates {(\bb,\dd,{f(\bb,\dd)})(\bb,\dd,0)};
\addplot3[]coordinates {(\aa,\dd,{f(\aa,\dd)})(\aa,\dd,0)};
\addplot3[] coordinates{(\a,\d,0)}node[left,xshift=-2ex]{$R$};
\end{axis}
\end{tikzpicture}
\caption{
ٹھوس جسم کو تخمینی طور پر  متعدد مستطیل منشور نما  سے ظاہر کرتے ہوئے ہم  زیادہ عمومی منشور نما کے حجم کو بطور دوہرا تکمل  تعین کر سکتے ہیں۔ یہاں منشور کا حجم \عددی{R} پر \عددی{f(x,y)} کا دوہرا تکمل ہو گا۔
}
\label{شکل_بالکثرت_ٹھوس_جسم_حجم_منشور}
\end{figure}
%====================================

\begin{figure}
\centering
\begin{minipage}{0.45\textwidth}
\centering
\begin{tikzpicture}[font=\small,declare function={f(\x,\y)=4-\x-\y;}]
\pgfmathsetmacro{\a}{0}
\pgfmathsetmacro{\b}{2}
\pgfmathsetmacro{\c}{0}
\pgfmathsetmacro{\d}{1}
\pgfmathsetmacro{\del}{0.2}
\pgfmathsetmacro{\kx}{0.6}
\begin{axis}[clip=false,small,axis lines=center,view/h=130,colormap={}{gray(0cm)=(0.6);gray(1cm)=(0.9);},enlargelimits=true,xlabel={$x$},ylabel={$y$},zlabel={$z$},xtick={\kx,\b},ytick={\d},ztick={4},xticklabels={$x$,$2$},xmin=0,ymin=0,zmin=0,xlabel style={anchor=east},ylabel style={anchor=west},zlabel style={anchor=south}]
\addplot3[fill=lgray]coordinates{(\kx,\c,{f(\kx,\c)})(\kx,\d,{f(\kx,\d)})(\kx,\d,0)(\kx,\c,0)(\kx,\c,{f(\kx,\c)})};
\addplot3[]coordinates{(\a,\c,{f(\a,\c)})(\a,\c,0)};
\addplot3[]coordinates{(\b,\c,{f(\b,\c)})(\b,\c,0)};
\addplot3[]coordinates{(\b,\d,{f(\b,\d)})(\b,\d,0)};
\addplot3[]coordinates{(\a,\d,{f(\a,\d)})(\a,\d,0)};
\addplot3[]coordinates{(\a,\c,{f(\a,\c)})(\b,\c,{f(\b,\c)})(\b,\d,{f(\b,\d)})(\a,\d,{f(\a,\d)})(\a,\c,{f(\a,\c)})};
\addplot3[]coordinates{(\a,\c,0)(\b,\c,0)(\b,\d,0)(\a,\d,0)(\a,\c,0)};
\addplot3[thick] coordinates {(\kx,\c,{f(\kx,\c)})(\kx,\d,{f(\kx,\d)})}node[pos=0.5,pin={[pin distance=0.75cm]45:{$z=4-x-y$}}]{};
\addplot3[]coordinates{(\kx,\d-\del,{1/4*f(\kx,\d-\del)})}  node[pin={[below,pin distance=1cm]-110:{$S(x)=\int_{y=0}^{y=1}(4-x-y)\dif y$}}]{};
\end{axis}
\end{tikzpicture}
\caption{
رقبہ عمودی تراش \عددی{S(x)} حاصل کرنے کے لئے ہم \عددی{x} کو مستقل ٹھراتے  ہوئے \عددی{y} کے لحاظ سے  تکمل لیتے ہیں۔
}
\label{شکل_بالکثرت_فوبینی_ایکس_مستقل}
\end{minipage}\hfill
\begin{minipage}{0.45\textwidth}
\centering
\begin{tikzpicture}[font=\small,declare function={f(\x,\y)=4-\x-\y;}]
\pgfmathsetmacro{\a}{0}
\pgfmathsetmacro{\b}{2}
\pgfmathsetmacro{\c}{0}
\pgfmathsetmacro{\d}{1}
\pgfmathsetmacro{\del}{0.2}
\pgfmathsetmacro{\ky}{0.6}
\begin{axis}[clip=false,small,axis lines=center,view/h=130,colormap={}{gray(0cm)=(0.6);gray(1cm)=(0.9);},enlargelimits=true,xlabel={$x$},ylabel={$y$},zlabel={$z$},xtick={\b},ytick={\ky,\d},yticklabels={$y$,$1$},ztick={4},xmin=0,ymin=0,zmin=0,xlabel style={anchor=east},ylabel style={anchor=west},zlabel style={anchor=south}]
\addplot3[fill=lgray]coordinates{(\a,\ky,{f(\a,\ky)})(\b,\ky,{f(\b,\ky)})(\b,\ky,0)(\a,\ky,0)(\a,\ky,{f(\a,\ky)})};
\addplot3[]coordinates{(\a,\c,{f(\a,\c)})(\a,\c,0)};
\addplot3[]coordinates{(\b,\c,{f(\b,\c)})(\b,\c,0)};
\addplot3[]coordinates{(\b,\d,{f(\b,\d)})(\b,\d,0)};
\addplot3[]coordinates{(\a,\d,{f(\a,\d)})(\a,\d,0)};
\addplot3[]coordinates{(\a,\c,{f(\a,\c)})(\b,\c,{f(\b,\c)})(\b,\d,{f(\b,\d)})(\a,\d,{f(\a,\d)})(\a,\c,{f(\a,\c)})};
\addplot3[]coordinates{(\a,\c,0)(\b,\c,0)(\b,\d,0)(\a,\d,0)(\a,\c,0)};
\addplot3[thick] coordinates {(\a,\ky,{f(\a,\ky)})(\b,\ky,{f(\b,\ky)})}node[pos=0.25,pin={[]45:{$z=4-x-y$}}]{};
\addplot3[]coordinates{(\b-\del,\ky,{1/4*f(\b-\del,\ky)})}  node[pin={[below,pin distance=1cm]-45:{$S(y)=\int_{x=0}^{x=2} (4-x-y)\dif x$}}]{};
\end{axis}
\end{tikzpicture}
\caption{
رقبہ عمودی تراش \عددی{S(y)} حاصل کرنے کے لئے ہم \عددی{y} کو مستقل ٹھراتے  ہوئے \عددی{x} کے لحاظ سے  تکمل لیتے ہیں۔
}
\label{شکل_بالکثرت_فوبینی_وائے_مستقل}
\end{minipage}
\end{figure}


\جزوحصہء{دوہرا تکمل کے حصول کا مسئلہ فوبینی}
فرض کریں ہم  مستوی \عددی{xy} میں  مستطیل خطہ \عددی{R:0\le x\le 2,\, 0\le y\le 1} پر مستوی \عددی{z=4-x-y}  کے نیچے حجم تلاش کرنا چاہتے ہیں۔ اگر ہم حصہ \حوالہ{حصہ_استعمال_تکمل_ترکیب_ٹکیاں} کی ترکیب استعمال کرتے ہوئے  محور \عددی{x} کے عمودی ٹکیاں  لیں  (شکل \حوالہ{شکل_بالکثرت_فوبینی_ایکس_مستقل})    تب حجم
\begin{align}\label{مساوات_دوہرا_فوبینی_ب}
\int_{x=0}^{x=2}S(x)\dif x
\end{align}
ہو گا  جہاں \عددی{x} پر رقبہ عمودی تراش \عددی{S(x)} ہے۔ہم    \عددی{x}  کی ہر قیمت کے لئے    درج ذیل تکمل سے  \عددی{S(x)}  معلوم کر سکتے ہیں
\begin{align}\label{مساوات_دوہرا_فوبینی_پ}
S(x)=\int_{y=0}^{y=1}(4-x-y)\dif y
\end{align}
جو منحنی \عددی{z=4-x-y} کے نیچے،  \عددی{x} پر عمودی تراش کے مستوی میں،  رقبہ ہو گا۔ رقبہ \عددی{S(x)} کے حصول میں \عددی{x} کو مستقل   تصور کرتے ہوئے \عددی{y} کے لحاظ سے تکمل حاصل کیا جاتا ہے۔ مساوات \حوالہ{مساوات_دوہرا_فوبینی_الف} اور مساوات \حوالہ{مساوات_دوہرا_فوبینی_ب} کو ملا کر  پورے ٹھوس جسم کا حجم درج ذیل حاصل ہو گا۔
\begin{gather}
\begin{aligned}\label{مساوات_دوہرا_فوبینی_ت}
\text{حجم}&=\int_{x=0}^{x=2}S(x)\dif x=\int_{x=0}^{x=2}\big(\int_{y=0}^{y=1}(4-x-y)\dif y\big)\dif x\\
&=\int_{x=0}^{x=2}\left[4y-xy-\frac{y^2}{2}\right]_{y=0}^{y=1}\dif x=\int_{x=0}^{x=2}\big(\frac{7}{2}-x\big)\dif x=\left[\frac{7}{2}x-\frac{x^2}{2}\right]_0^2=5
\end{aligned}
\end{gather}

اگر ہم    حجم تلاش کرنے کی  صرف بات کرنا چاہتے ہوں تب ہم درج ذیل لکھیں گے۔
\begin{align*}
\text{حجم}=\int_0^2\int_0^1 (4-x-y)\dif y\dif x
\end{align*}
دائیں ہاتھ الجبرائی فقرہ، جسے  \اصطلاح{بارہا تکمل}\فرہنگ{تکمل!بارہا}\حاشیہب{repeated integral}\فرہنگ{integral!repeated} یا  \اصطلاح{اعادہ تکمل}\فرہنگ{تکمل!اعادہ}\حاشیہب{iterated integral}\فرہنگ{integral!iterated} کہتے ہیں، کہتا ہے  کہ حجم تلاش کرنے کی خاطر، پہلے  \عددی{x} کو مستقل  ٹھراتے  ہوئے \عددی{y} کے لحاظ سے \عددی{4-x-y} کا   تکمل \عددی{y=0} تا \عددی{y=1}   لیں اور اس کے بعد  \عددی{y} کو مستقل ٹھراتے  ہوئے، \عددی{x} کے لحاظ سے حاصل نتیجہ کا تکمل   \عددی{x=0} تا \عددی{x=2}  لیں۔

 اگر ہم  محور  \عددی{y}  کے عمودی ٹکیاں لیتے تب نتیجہ کیا ہوتا (شکل \حوالہ{شکل_بالکثرت_فوبینی_وائے_مستقل})؟    ایسی صورت میں   ایک علامتی عمودی تراش رقبہ، \عددی{y} کا تفاعل ہو گا:
\begin{align}
S(y)=\int_{x=0}^{x=2}(4-x-y)\dif x=\left[4x-\frac{x^2}{2}-xy\right]_{x=0}^{x=2}=6-2y
\end{align} 
یوں پورے جسم کا حجم 
\begin{align}
\text{حجم}=\int_{y=0}^{y=1}S(y)\dif y=\int_{y=0}^{y=1}(6-2y)\dif y=\left[6y-y^2\right]_0^1=5
\end{align}
ہو گا جو ہماری گزشتہ حساب کے عین مطابق ہے۔

ہم  اب  حجم کی بات کرتے ہوئے
\begin{align*}
\text{حجم}=\int_0^1\int_0^2(4-x-y)\dif x\dif y
\end{align*}
لکھ سکتے ہیں۔ دائیں ہاتھ الجبرائی فقرہ کہتا ہے کہ حجم تلاش کرنے کی خاطر،  پہلے \عددی{y} کو مستقل ٹھراتے  ہوئے \عددی{x} کے لحاظ سے \عددی{4-x-y} کا تکمل \عددی{x=0} تا \عددی{x=2} لیں۔ اس کے بعد \عددی{x} کو مستقل  ٹھراتے  ہوئے  \عددی{x} کے لحاظ سے  حاصل  نتیجہ کا تکمل \عددی{y=0} تا \عددی{y=1}  لیں۔ اس بار ہم بارہا تکمل کے حصول میں پہلے \عددی{x} اور بعد میں \عددی{y} کے لحاظ سے تکمل لیتے ہیں جو  مساوات \حوالہ{مساوات_دوہرا_فوبینی_ت} میں تکمل کے ترتیب کا  الٹ ہے۔

مذکورہ بالا  دو بار حجم کے حساب کا مستطیل خطہ \عددی{R:\,0\le x\le 2,\,0\le y\le 1} پر    درج ذیل دوہرا تکمل کے ساتھ کیا تعلق ہے؟
\begin{align*}
\iint\limits_R(4-x-y)\dif S
\end{align*}
اس کا جواب ہے کہ یہ دونوں تکمل اس دوہرا تکمل کی قیمت دیتے ہیں۔ مسئلہ فوبینی کہتا ہے کہ   مستطیل خطہ   پر استمراری   تفاعل  کا  دوہرا تکمل،   کسی بھی ترتیب سے،  بارہا تکمل  سے  حاصل کیا جا سکتا ہے۔ (جناب فوبینی نے اس مسئلہ کو  زیادہ عمومیت کے ساتھ ثابت کیا لیکن  فی الحال اس کو ہم درج ذیل بیان کرتے ہیں۔)

\ابتدا{مسئلہ}\موٹا{مسئلہ فوبینی (پہلا روپ)}\\
 اگر مستطیل خطہ \عددی{R:\, a\le x\le b,\, c\le y\le d} پر \عددی{f(x,y)} استمراری ہو تب درج ذیل ہوگا۔
\begin{align*}
\iint\limits_Rf(x,y)\dif S=\int_c^d\int_a^bf(x,y)\dif x\dif y=\int_a^b\int_c^d f(x,y)\dif y\dif x
\end{align*}
\انتہا{مسئلہ}
%===================

مسئلہ فوبینی کہتا ہے کہ مستطیل خطہ پر  دوہرا تکمل کی قیمت   بارہا تکمل سے حاصل کی جا سکتی ہے۔ یوں دوہرا تکمل کے حصول میں ہم باری باری   ایک ایک متغیر کے لحاظ سے تکمل لے سکتے ہیں۔

مسئلہ فوبینی مزید کہتا ہے کہ  دوہرا تکمل کی قیمت حاصل کرتے ہوئے ہم بارہا تکمل کسی بھی ترتیب سے حل کر سکتے ہیں، جو بہت کار آمد ثابت ہوتا ہے (جیسا ہم جلد ایک  مثال میں دیکھتے ہیں)۔بالخصوص حجم کی تلاش میں ہم \عددی{x} محور یا \عددی{y} محور کے عمودی سطحیں لے کر ٹکیاں کاٹ سکتے ہیں۔

\ابتدا{مثال}
خطہ \عددی{R:\, 0\le x\le 2,\, -1\le y\le 1} میں \عددی{f(x,y)=1-6x^2y} کے دوہرا تکمل \عددی{\iint_R f(x,y)\dif S}  کی قیمت تلاش کریں۔

حل:\quad
مسئلہ فوبینی کے تحت درج ذیل ہو گا:
\begin{align*}
\iint\limits_R f(x,y)\dif S&=\int_{-1}^1 \int_0^2(1-6x^2y)\dif x\dif y=\int_{-1}^1 \left[x-2x^3y\right]_{x=0}^{x=2}\dif y\\
&=\int_{-1}^1 (2-16y)\dif y=\left[2y-8y^2\right]_{-1}^1=4
\end{align*}
تکمل کی ترتیب بدلنے سے بھی یہی نتیجہ حاصل ہوتا ہے:
\begin{align*}
\int_0^2\int_{-1}^1 (1-6x^2y)\dif y\dif x&=\int_0^2 \left[y-3x^2y^2\right]_{y=-1}^{y=1}\dif x\\
&=\int_0^2\left[(1-3x^2)-(-1-3x^2)\right]\dif x=\int_0^2 2\dif x=4
\end{align*}
\انتہا{مثال}
%=================

آپ سے گزارش کی جاتی ہے کہ کمپیوٹر پر دوہرا تکملات  کا حصول سیکھیں۔ کمپیوٹر الجبرائی   پروگرام  \ترچھا{میکسما}\حاشیہب{wxMaxima} میں یہ عمل درج ذیل ہو گا۔
\begin{center}
\begin{tabular}{LL}
\text{\RL{درکار دوہرا تکمل}}&\text{\RL{میکسما  احکامات}}\\
\midrule
\iint x^2y\dif x\dif y&\textup{integrate} (\textup{integrate}(x^2*y,x),y);\\
\int_{-\pi/3}^{\pi/4}\int_0^1 x\cos y\dif x\dif y&\textup{integrate}(\textup{integrate}(x*\cos(y),x,0,1),y,-\%pi/3,\%pi/4);

\end{tabular}
\end{center}



\begin{figure}
\centering
\begin{minipage}{0.55\textwidth}
\centering
\begin{tikzpicture}[font=\small]
\pgfmathsetmacro{\a}{3.75}
\pgfmathsetmacro{\b}{2.5}
\draw[-latex](0,0)--++(4.5,0)node[right]{$x$};
\draw[-latex](0,0)--++(0,3.25)node[right]{$y$};
\draw[fill=lgray](0.4+\a*0.07,0.25+\b*0.97)
\foreach \x/\y in{0.07/0.97,0.8/0.97,0.8/0.8,0.9/0.8,0.9/0.7,0.97/0.7,0.97/0.2,0.8/0.2,0.8/0.1,0.6/0.1,
0.6/0.02,0.2/0.02,0.2/0.1,0.07/0.1,0.07/0.3,0/0.3,0/0.8,0.07/0.8,0.07/0.97} {--(0.4+\a*\x,0.25+\b*\y)};
\draw(0.4,0)node[below]{$a$}--++(0,0.1)  (0.4,0)++(\a,0)node[below]{$b$}--++(0,0.1); 
\draw(0,0.25)node[left]{$c$}--++(0.1,0)  (0,0.25)++(0,\b)node[left]{$d$}--++(0.1,0); 
\foreach \x in {0,0.07,0.125,0.2,0.3,0.45,0.6,0.8,0.9,0.97} {\draw(0.4+\a*\x,0.125)--++(0,\b+0.25);}
\foreach \y in {0.02,0.1,0.2,0.3,0.5,0.7,0.8,0.97}{\draw(0.25,0.25+\b*\y)--++(\a+0.25,0);}
\draw[fill=gray](0.4+0.45*\a,0.25+0.3*\b) rectangle (0.4+0.6*\a,0.25+0.5*\b);
\draw[stealth-stealth](0.4+0.45*\a,-0.1)--(0.4+0.6*\a,-0.1)node[pos=0.5,below]{$\Delta x_k$};
\draw[stealth-stealth](-0.1,0.25+0.3*\b)--(-0.1,0.25+0.5*\b)node[pos=0.5,left]{$\Delta y_k$};
\draw(0.4+0.475*\a,0.25+0.4*\b)--++(-0.3,0.3)node[above]{$\Delta S_k$};
\draw(0.4+0.55*\a,0.25+0.45*\b)node[circ]{}--++(0.3,0.2)node[above right]{$(x_k,y_k)$};
\draw[thick](0.3,1/2*\b) to [out=-90,in=170](1/3*\a,0.25) to [out=-10,in=-170](2/3*\a,0.265) to [out=10,in=-90] (\a+0.45,1/2*\b) to [out=90,in=-10](\a-0.5,\b+0.25) to [out=170,in=5](1/4*\a,\b+0.25) to [out=-175,in=90](0.3,1/2*\b);
\draw(0.4+\a*0.54,0.25+\b*0.87)node[]{$R$};
\end{tikzpicture}
\caption{
غیر مستطیل محدود  خطہ کو  مستطیل جال سے   خانہ بند کیا گیا ہے۔
}
\label{شکل_بالکثرت_غیر_مستطیل_دائرہ_کار}
\end{minipage}\hfill
\begin{minipage}{0.4\textwidth}
\centering
\begin{tikzpicture}
\pgfmathsetmacro{\a}{3}
\pgfmathsetmacro{\b}{2}
\draw(\a*0,\b*0.75) to [out=90,in=180] coordinate[pos=0.85](kT)(\a*0.4,\b*1)node[above right]{$R=R_1\cup R_2$} to [out=0,in=90](\a*1,\b*0.4) to [out=-90,in=0](\a*0.8,\b*0) to [out=180,in=-45]coordinate[pos=0.2](kB)(\a*0.4,\b*0.5) to [out=145,in=-90](\a*0,\b*0.75);
\draw[-latex](0.2,0.2)--++(3.5,0)node[right]{$x$};
\draw[-latex](0.2,0.2)--++(0,2.5)node[left]{$y$};
\draw(kT) to [out=-45,in=70]node[pos=0.3,left,xshift=-3ex]{$R_1$}node[pos=0.7,right,xshift=1ex]{$R_2$}(kB)node[below,yshift=-2ex,font=\scriptsize]{$\iint\limits_{R}f(x,y)\dif S=\iint\limits_{R_1}f(x,y)\dif S+\iint\limits_{R_2}f(x,y)\dif S$};
\end{tikzpicture}
\caption{
مستطیل خطہ کی مجموعیت کی خاصیت ان غیر مستطیل  خطوں کے لئے بھی کارآمد ہے جن کی پوری  سرحد استمراری منحنیات سے بنی ہو۔
}
\label{شکل_بالکثرت_مجموعیت_غیر_مستطیل_دائرہ_کار}
\end{minipage}
\end{figure}

\جزوحصہء{محدود غیر مستطیل  خطہ پر دوہرا تکملات }
محدود غیر مستطیل خطہ پر تفاعل \عددی{f(x,y)}  کا دوہرا تکمل تعین کرنے کی خاطر  ہم اب بھی \عددی{R} پر مستطیل جال بچھاتے ہیں (شکل \حوالہ{شکل_بالکثرت_غیر_مستطیل_دائرہ_کار})    لیکن جزوی مجموعہ میں صرف ان چھوٹے رقبوں  \عددی{\Delta S=\Delta x\Delta y}کو شامل کرتے ہیں جو مکمل طور پر اس خطہ میں پائے جاتے ہوں۔ ہم ان چھوٹے رقبوں کو کسی بھی  ترتیب سے شمار کرتے  ہوئے، ہر رقبہ \عددی{\Delta S_k} میں کوئی نقطہ \عددی{(x_k,y_k)} منتخب کر کے   درج ذیل مجموعہ حاصل کرتے ہیں۔
\begin{align*}
J_n=\sum_{k=1}^n f(x_k,y_k)\Delta S_k
\end{align*}
اس مجموعہ میں اور مستطیل خطے  پر مجموعہ (مساوات \حوالہ{مساوات_کثیرالمتغیر_مجموعہ_الف})  میں صرف اتنا فرق ہے کہ  اب شامل کردہ تمام \عددی{\Delta S_k} مل کر خطہ \عددی{R} کو  مکمل طور پر نہیں ڈھانپتے ہیں۔البتہ جیسے جیسے جال کے  خانوں کا رقبہ چھوٹے سے چھوٹا ہو،  \عددی{J_n} میں  اجزاء کی تعداد بڑھتی جائے گی  اور \عددی{R} کا زیادہ سے زیادہ حصہ  \عددی{J_n} میں شامل ہو گا۔ اگر \عددی{f} استمراری ہو اور \عددی{R} کی سرحد،     متغیر \عددی{x} کی  متناہی تعداد کے  استمراری تفاعل اور (یا) متغیر \عددی{y} کی  متناہی تعداد کے استمراری تفاعل کی ترسیمات،  ایک دوسرے کے ساتھ جوڑ کر حاصل کی گئی ہو، تب، بشرطیکہ  مستطیل جال کے خانوں کے   معیار  غیر  مختارانہ  طور پر صفر کو پہنچتے ہوں،  مجموعہ \عددی{J_n} کا حد موجود  ہو گا۔ ہم اس حد کو \عددی{R} پر \عددی{f} کا \اصطلاح{ دوہرا تکمل} کہتے ہیں:
\begin{align*}
\iint\limits_R f(x,y)\dif S=\lim_{\Delta S\to 0}\sum f(x_,y_k)\Delta S_k
\end{align*}
یہ حد کم  پابندی کی صورت میں بھی موجود ہو سکتا ہے۔

غیر مستطیل خطہ پر استمراری  تفاعل کے دوہرا تکملات کے وہی خواص ہوں گے جو مستطیل خطہ پر  دوہرا تکملات کے ہوتے ہیں۔ دائرہ کار کی  خواص مجموعیت  کہتی ہے کہ اگر \عددی{R} کو ایسے   دو  خطوں  \عددی{R_1} اور \عددی{R_2} میں تقسیم کیا جائے جو ایک دوسرے کو نہ ڈھانپتے   ہوں  اور  جن کی سرحدیں  متناہی تعداد کے قطعات یا ہموار منحنیات  سے بنی ہوئی  ہوں (مثال کے لئے  شکل \حوالہ{شکل_بالکثرت_مجموعیت_غیر_مستطیل_دائرہ_کار} دیکھیں)        تب  درج ذیل ہو گا۔
\begin{align*}
\iint\limits_R f(x,y)\dif S=\iint\limits_{R_1}f(x,y)\dif S+\iint\limits_{R_2}f(x,y)\dif S
\end{align*}

ہم  \عددی{R} پر استمراری اور مثبت  \عددی{f} کی صورت میں  \عددی{R} اور \عددی{z=f(x,y)} کے بیچ ٹھوس جسم کے حجم  کی تعریف پہلے کی طرح اب بھی   \عددی{\iint_R f(x,y)\dif S} کرتے ہیں۔



\begin{figure}
\centering
\begin{tikzpicture}[font=\small,declare function={fx(\x,\y)=\x;fy(\x,\y)=\y;fz(\x,\y)=1-(\y-1)^2;}]
\pgfmathsetmacro{\a}{0}
\pgfmathsetmacro{\b}{0}
\pgfmathsetmacro{\c}{1}
\pgfmathsetmacro{\d}{2}
\pgfmathsetmacro{\e}{1}
\pgfmathsetmacro{\f}{0.5}
\pgfmathsetmacro{\h}{1}
\pgfmathsetmacro{\angA}{90}
\pgfmathsetmacro{\angB}{-100}
\pgfmathsetmacro{\angC}{30}
\pgfmathsetmacro{\angD}{-135}
\pgfmathsetmacro{\angE}{20}
\pgfmathsetmacro{\angF}{130}
\pgfmathsetmacro{\angX}{-117}
\pgfmathsetmacro{\angY}{-27}
\draw(\a,\b)coordinate(ka) to [out=\angA,in=\angB] coordinate[pos=0.75](kL)++(\c,\d)coordinate(kc);
\draw(\a+\e,\b-\f)coordinate(kb) to [out=\angC,in=\angD] coordinate[pos=0.75](kR)++(\c,\d)coordinate(kd);
\draw(ka) to [out=\angE,in=\angF](kb);
\draw(kc) to [out=\angE,in=\angF](kd);
\draw[thick](kL) to [out=\angE,in=\angF]coordinate[pos=0.5](kM)(kR);
\draw[gray](kM)--++(30:1)node[black,above,xshift=2ex]{$z=f(x,y)$};
\draw(ka)--++(0,-\h)coordinate(kka);
\draw(kb)--++(0,-\h)coordinate(kkb);
\draw(kc)--++(0,-\h)coordinate(kkc);
\draw(kd)--++(0,-\h)coordinate(kkd);
\draw(kL)--++(0,-\h)coordinate(kkL);
\draw(kR)--++(0,-\h)coordinate(kkR);
\draw(kka)--(kkb);
\draw(kkc)--(kkd);
\draw(kkL)--(kkR);
\draw[thick](kka)to [out=\angA,in=\angB] coordinate[pos=0.25](kLeft)(kkc);
\draw[thick](kkb)to [out=\angC,in=\angD]coordinate[pos=0.25](kRight) (kkd);
\draw(kRight)node[pin=20:{$y=g_2(x)$}]{};
\draw[gray](kLeft)--++(135:0.75)node[above,black,xshift=-2ex]{$y=g_1(x)$};
\draw[fill=lgray,opacity=0.5](kL) to [out=\angE,in=\angF](kR)--(kkR)--(kkL)--(kL);
\draw($(kka)!0.5!(kkb)$)node[above]{$R$};
\draw[dashed](kka)--++(\angY:-2)node[left]{$b$};
\draw[dashed](kkc)--++(\angY:-2)node[left]{$a$};
\draw[dashed](kkL)--++(\angY:-2)node[left]{$x$};
\draw[-latex](kkc)++(\angY:-2)--++(\angX:2.75)node[below]{$x$};
\draw[-latex](kkc)++(\angY:-2)--++(\angX:-1)--++(\angY:1)node[right]{$y$};
\draw[-latex](kkc)++(\angY:-2)++(\angX:-1)--++(0,0.5)node[left]{$z$};
\end{tikzpicture}
\caption{
سایہ دار انتصابی ٹکیہ  کا رقبہ \عددی{S(x)=\int_{g_1(x)}^{g_2(x)}f(x,y)\dif y} ہو گا۔ اس ٹھوس  جسم کا حجم تلاش کرنے کے لئے ہم  \عددی{x=a} سے \عددی{x=b} تک \عددی{S(x)} کا تکمل لیں گے۔ 
}
\label{شکل_بالکثرت_انتصابی_ٹکیہ_کا_رقبہ}
\end{figure}


اگر شکل \حوالہ{شکل_بالکثرت_انتصابی_ٹکیہ_کا_رقبہ}  میں  مستوی \عددی{xy} میں دکھائے گئے خطہ کی طرح \عددی{R}ہو  اور حجم کی   "بالائی"   حد \عددی{y=g_2(x)}،  "زیریں"  حد \عددی{y=g_1(x)}، اور اطراف کے حدود  خط \عددی{x=a} اور خط  \عددی{x=b} ہوں تب  ہم حجم \عددی{H}  کو  ٹکیوں کی ترکیب سے حاصل کر سکتے ہیں۔ ہم پہلے رقبہ عمودی تراش تلاش کرتے ہیں
\begin{align*}
S(x)=\int_{y=g_1(x)}^{y=g_2(x)} f(x,y)\dif y
\end{align*}
اور اس کے بعد \عددی{x=a}  سے \عددی{x=b} تک \عددی{S(x)} کا تکمل لیتے   ہوئے بارہا تکمل سے حجم حاصل کرتے ہیں۔
\begin{align}\label{مساوات_بالکثرت_حجم_ٹکیاں_الف}
H=\int_a^b S(x)\dif x=\int_a^b\int_{g_1(x)}^{g_2(x)} f(x,y)\dif y\dif x
\end{align}

\begin{figure}
\centering
\begin{tikzpicture}[font=\small,declare function={fx(\x,\y)=\x;fy(\x,\y)=\y;fz(\x,\y)=1-(\y-1)^2;}]
\pgfmathsetmacro{\a}{0}
\pgfmathsetmacro{\b}{0}
\pgfmathsetmacro{\c}{2}
\pgfmathsetmacro{\d}{1}
\pgfmathsetmacro{\e}{1}
\pgfmathsetmacro{\f}{1}
\pgfmathsetmacro{\h}{1}
\pgfmathsetmacro{\angA}{10}
\pgfmathsetmacro{\angB}{170}
\pgfmathsetmacro{\angC}{-10}
\pgfmathsetmacro{\angD}{100}
\pgfmathsetmacro{\angE}{20}
\pgfmathsetmacro{\angF}{130}
\pgfmathsetmacro{\angX}{-120}
\draw[thick](\a,\b)coordinate(ka)to [out=\angA,in=\angB]coordinate[pos=0.4](kFL)coordinate[pos=0.75](kFM)++(\c,-\d)coordinate(kb)++(\angX:-2)coordinate(kd);
\draw[gray](kFM)--++(-135:0.5)node[below,black]{$x=h_2(y)$};
\draw[thick](\a,\b)++(\angX:-0.75)coordinate(kc)to [out=\angC,in=\angD]coordinate[pos=0.85](kBM)coordinate[pos=0.325](kBL)(kd);
\draw[gray](kBM)--++(20:0.75)node[right,black]{$x=h_1(y)$};
\draw(kb)--++(0,\h)coordinate(kf);
\draw(ka)--++(0,1/2*\h)coordinate(ke);
\draw(kd)--++(0,\h)coordinate(kh);
\draw(ke) to [out=-\angA,in=-\angB]coordinate[pos=0.4](kFH)(kf);
\draw(ka)--(kc)--++(0,1/2*\h)coordinate(kg) to [out=-\angA,in=-\angB]coordinate[pos=0.4](kBH)(kh);
\draw (kb)--(kd);
\draw(ke)to [out=50,in=-170](kg)    (kf)to [out=40,in=-160](kh);
\draw[fill=lgray,opacity=0.5](kFL)--(kFH) to [out=45,in=-165](kBH)--(kBL)--(kFL); 
\draw[thick](kFH) to [out=45,in=-165]coordinate[pos=0.75](kM)(kBH);
\draw[gray](kM)--++(135:0.75)node[above,xshift=-2ex,black,font=\scriptsize]{$z=f(x,y)$};
\draw[latex-latex](\a,\b)++(-1.5,0)node[below]{$x$}--++(\angX:-2.5)coordinate(kO)node[left]{$O$}--++(5,0)node[right]{$y$};
\draw[-latex](kO)--++(0,1)node[left]{$z$};
\draw[dashed](kc)--++(\angX:-1.75)node[above]{$c$};
\draw[dashed](kd)--++(\angX:-1.6)node[above]{$d$};
\draw[dashed](kBL)--++(\angX:-1.55)node[above]{$x$};
\end{tikzpicture}
\caption{
سایہ دار ٹکیہ کا رقبہ \عددی{S(y)=\int_{h_1(y)}^{h_2(y)}f(x,y)\dif x} ہے۔ \\
ٹھوس جسم کا حجم \عددی{\int_c^dS(y)\dif y=\int_c^d\int_{h_1(y)}^{h_2(y)}f(x,y)\dif x\dif y} ہو گا۔
}
\label{شکل_بالکثرت_انتصابی_ٹکیہ_ٹھوس_حجم}
\end{figure}

اسی طرح اگر  شکل \حوالہ{شکل_بالکثرت_انتصابی_ٹکیہ_ٹھوس_حجم} میں  دکھائے گئے خطہ کی طرح \عددی{R} ہو اور حجم کے حدود \عددی{x=h_2(y)}، \عددی{x=h_1(y)} اور خط \عددی{y=c} اور \عددی{y=d} ہوں تب  ٹکیوں کی ترکیب سے بارہا تکمل سے حجم تلاش کیا جا سکتا ہے:
\begin{align}\label{مساوات_بالکثرت_حجم_ٹکیاں_ب}
H=\int_c^d\int_{h_1(y)}^{h_2(y)}f(x,y)\dif x\dif y
\end{align}

ہم نے دیکھا کہ مساوات \حوالہ{مساوات_بالکثرت_حجم_ٹکیاں_الف} اور مساوات \حوالہ{مساوات_بالکثرت_حجم_ٹکیاں_الف}، جو \عددی{R} پر \عددی{f} کے دوہرا تکمل ہیں ،   دونوں حجم  دیتے ہیں ۔ اس کی وجہ مسئلہ فوبینی   کی درج ذیل  زیادہ مضبوط  صورت ہے۔

\ابتدا{مسئلہ}\موٹا{مسئلہ فوبینی (مضبوط روپ)}\\
فرض کریں خطہ \عددی{R} پر \عددی{f} استمراری ہے۔
\begin{enumerate}[a.]
\item
اگر  \عددی{R} کو  \عددی{a\le x\le b}، \عددی{g_1(x)\le y\le g_2(x)} تعین کرتے ہوں جہاں \عددی{[a,b]} پر \عددی{g_1} اور \عددی{g_2} استمراری ہوں  تب درج ذیل ہو گا۔
\begin{align*}
\iint\limits_R f(x,y)\dif S=\int_a^b\int_{g_1(x)}^{g_2(x)}f(x,y)\dif y\dif x
\end{align*} 
\item

اگر  \عددی{R} کو  \عددی{c\le y\le d}، \عددی{h_1(y)\le y\le h_2(y)} تعین کرتے ہوں جہاں \عددی{[c,d]} پر \عددی{h_1} اور \عددی{h_2} استمراری ہوں  تب درج ذیل ہو گا۔
\begin{align*}
\iint\limits_R f(x,y)\dif S=\int_c^d\int_{h_1(y)}^{h_2(y)}f(x,y)\dif x\dif y
\end{align*} 
\end{enumerate}
\انتہا{مسئلہ}
%===========

\begin{figure}
\centering
\begin{subfigure}{0.45\textwidth}
\centering
\begin{tikzpicture}[font=\small,declare function={f(\x,\y)=3-\x-\y;}]
\pgfmathsetmacro{\m}{0.6}
\pgfmathsetmacro{\n}{0.25}
\pgfmathsetmacro{\a}{0.2}
\begin{axis}[clip=false,small,axis lines=center,view={100}{15},colormap={}{gray(0cm)=(0.6);gray(1cm)=(0.9);},enlargelimits=true, xlabel={$x$}, ylabel={$y$}, zlabel={$z$}, xtick={\empty},ytick={\empty},ztick={\empty}, xmin=0,ymin=0,zmin=0,xmax=1.5,ymax=1.125,xlabel style={anchor=north}, ylabel style={anchor=west},zlabel style={anchor=south}]
\addplot3[fill=llgray,opacity=0.5]coordinates{(0,0,0)(1,1,0)(1,0,0)};
\addplot3[fill=lgray,opacity=0.5]coordinates {(0,0,3)(1,0,2)(1,1,1) (0,0,3)}node[pos=0.9,pin={[black,right]20:{$\begin{aligned}z&=f(x,y)\\ &=3-x-y\end{aligned}$}}]{};
\addplot3[]coordinates{(0,0,0)(1.25,1.25,0)}node[below]{$y=x$};
\addplot3[]coordinates {(1,0,0)(1,0,2)}node[pos=0,left,yshift=0.5ex]{$(1,0,0)$}node[left]{$(1,0,2)$};
\addplot3[]coordinates {(1,1,0)(1,1,1)}node[pos=0,right,yshift=1ex]{$(1,1,0)$}node[right,yshift=1ex]{$(1,1,1)$};
\addplot3[]coordinates{(0,0,3)}node[left]{$(0,0,3)$};
\addplot3[]coordinates{(0.6,0.6,0)}node[pin=-135:{$R$}]{};
\addplot3[]coordinates{(\m,\n,{f(\m,\n)})(\m+\a,\n,{f(\m+\a,\n)})(\m+\a,\n+\a,{f(\m+\a,\n+\a)})(\m,\n+\a,{f(\m,\n+\a)})(\m,\n,{f(\m,\n)})};
\addplot3[]coordinates{(\m,\n,0)(\m+\a,\n,0)(\m+\a,\n+\a,0)(\m,\n+\a,0)(\m,\n,0)};
\addplot3[]coordinates{(\m,\n,{f(\m,\n)})(\m,\n,0)};
\addplot3[]coordinates{(\m+\a,\n,{f(\m+\a,\n)})(\m+\a,\n,0)};
\addplot3[]coordinates{(\m+\a,\n+\a,{f(\m+\a,\n+\a)})(\m+\a,\n+\a,0)};
\addplot3[]coordinates{(\m,\n+\a,{f(\m,\n+\a)})(\m,\n+\a,0)};
\end{axis}
\end{tikzpicture}
\caption{}
\end{subfigure}\hfill
\begin{subfigure}{0.25\textwidth}
\centering
\begin{tikzpicture}[scale=2,font=\small]
\fill[lgray](0,0)--(1,0)--(1,1)--(0,0);
\draw[-latex](0,0)--(1.25,0)node[below]{$x$};
\draw[-latex](0,0)--(0,1.25)node[right]{$y$};
\draw(0,0)--(1.25,1.25)node[above,xshift=-1ex]{$y=x$};
\draw(1,0)node[below]{$1$}--(1,1.125)node[left]{$x=1$};
\draw[thick](0.6,0)node[below,xshift=-1ex]{$y=0$}--(0.6,0.6);
\draw[gray](0.6,0.6)++(0,0.02)--++(-0.1,0.125)node[above,black]{$y=x$};
\draw(0.4,0)node[above]{$R$};
\end{tikzpicture}
\caption{}
\end{subfigure}\hfill
\begin{subfigure}{0.25\textwidth}
\centering
\begin{tikzpicture}[scale=2,font=\small]
\fill[lgray](0,0)--(1,0)--(1,1)--(0,0);
\draw[-latex](0,0)--(1.25,0)node[below]{$x$};
\draw[-latex](0,0)--(0,1.25)node[right]{$y$};
\draw(0,0)--(1.25,1.25)node[above,xshift=-1ex]{$y=x$};
\draw(1,0)node[below]{$1$}--(1,1.125)node[left]{$x=1$};
\draw[thick](0.6,0.6)node[left]{$x=y$}--(1,0.6)node[right]{$x=1$};
\draw(0.5,0)node[above]{$R$};
\end{tikzpicture}
\caption{}
\end{subfigure}
\caption{منشور کا حجم (مثال \حوالہ{مثال_بالکثرت_فوبینی_مثلث})}
\label{شکل_مثال_بالکثرت_فوبینی_مثلث}
\end{figure}

\ابتدا{مثال}\شناخت{مثال_بالکثرت_فوبینی_مثلث}
ایک منشور جس کا قاعدہ  مستوی \عددی{xy} میں ایک مثلث   ہو،  جس کے اضلاع محور \عددی{x}، خط \عددی{x=1} اور خط \عددی{y=x} ہوں اور جس کا راس  درج ذیل مستوی میں پایا جاتا ہو، کا حجم تلاش کریں۔
\begin{align*}
z=f(x,y)=3-x-y
\end{align*}
حل:\quad
ہم دیکھتے ہیں (شکل \حوالہ{شکل_مثال_بالکثرت_فوبینی_مثلث}-ا)  کہ  \عددی{0} اور \عددی{1} تک کسی بھی  \عددی{x} کے لئے \عددی{y} کی قیمت \عددی{y=0} تا \عددی{y=x} ہو گی (شکل \حوالہ{شکل_مثال_بالکثرت_فوبینی_مثلث}-ب)۔ یوں درج ذیل ہو گا۔
\begin{align*}
H&=\int_0^1\int_0^x(3-x-y)\dif y\dif x=\int_0^1\big[3y-xy-\frac{y^2}{2}\big]_{y=0}^{y=x}\dif x\\
&=\int_0^1\big(3x-\frac{3x^2}{2}\big)\dif x=\big[\frac{3x^2}{2}-\frac{x^3}{2}\big]_{x=0}^{x=1}=1
\end{align*}
تکملات  کی ترتیب الٹ کرنے سے درج ذیل  ہو گا (شکل \حوالہ{شکل_مثال_بالکثرت_فوبینی_مثلث}-ج)۔
\begin{align*}
H&=\int_0^1\int_y^1(3-x-y)\dif x\dif y=\int_0^1\big[3x-\frac{x^2}{2}-xy\big]_{x=y}^{x=1}\dif y\\
&=\int_0^1\big(3-\frac{1}{2}-y-3y+\frac{y^2}{2}+y^2\big)\dif y\\
&=\int_0^1\big(\frac{5}{4}-4y+\frac{3}{2}y^2\big)\dif y=\big[\frac{5}{2}y-2y^2+\frac{y^3}{2}\big]_{y=0}^{y=1}=1
\end{align*}
دونوں تکملات کے جواب ایک جیسے ہیں۔ہمیں یہی توقع تھی۔
\انتہا{مثال}
%==============

اگرچہ مسئلہ فوبینی ہمیں یقین دھیانی  کرتا ہے کہ دوہرا تکمل کی قیمت    بارہا تکمل میں    کسی بھی ترتیب سے  تکملات لیتے ہوئے حاصل کیا جا سکتا ہے، حقیقت میں ایک تکمل کا حصول  دوسرے سے آسان ہو سکتا ہے۔ اگلی مثال میں آپ ایسی صورت حال دیکھتے ہیں۔

\begin{figure}
\centering
\begin{tikzpicture}[scale=2,font=\small]
\fill[lgray](0,0)--(1,0)--(1,1)--(0,0);
\draw[-latex](0,0)--(1.5,0)node[right]{$x$};
\draw[-latex](0,0)--(0,1.25)node[left]{$y$};
\draw(0,0)node[left]{$O$}--(1.25,1.25)node[right]{$y=x$};
\draw(1,0)node[below]{$1$}--(1,1.125)node[above,xshift=-2ex]{$x=1$};
\draw(0.5,0)node[above]{$R$};
\end{tikzpicture}
\caption{تکمل کا دائرہ کار برائے مثال \حوالہ{مثال_بالکثرت_غیر_بنیادی_تفاعل}}
\label{شکل_مثال_بالکثرت_غیر_بنیادی_تفاعل}
\end{figure}
\ابتدا{مثال}\شناخت{مثال_بالکثرت_غیر_بنیادی_تفاعل}
مستوی \عددی{xy} میں محور \عددی{x}، خط \عددی{x=1} اور خط \عددی{y=x} کے بیچ خطہ \عددی{R} ہے۔ درج ذیل کی قیمت تلاش کریں۔
\begin{align*}
\iint\limits_R \frac{\sin x}{x}\dif S
\end{align*}
حل:\quad
تکمل کا خطہ شکل \حوالہ{شکل_مثال_بالکثرت_غیر_بنیادی_تفاعل}  میں دکھایا گیا ہے۔ اگر ہم پہلے \عددی{y} اور بعد میں \عددی{x} کے لحاظ سے تکمل لیں تب
\begin{align*}
\int_0^1\big(\int_0^x \frac{\sin x}{x}\dif y\big)\dif x&=\int_0^1\big(y\frac{\sin x}{x}\big]_{y=0}^{y=x}\big)\dif x=\int_0^1\sin x\dif x\\
&=-\cos(1)+1\approx 0.46
\end{align*}
ہو گا۔اگر ہم تکمل لینے کی ترتیب الٹ کریں تب 
\begin{align*}
\int_0^1\int_y^1\frac{\sin x}{x}\dif x\dif y
\end{align*}
ہو گا اور چونکہ \عددی{\int ((\sin x)/x)\dif x} کو  بنیادی تفاعل کی صورت میں نہیں لکھا جا سکتا ہے لہٰذا  ہم اس کو حل کرنے سے قاصر ہیں۔

قبل از وقت یہ جاننا ممکن نہیں    کہ کس ترتیب سے تکمل لینے سے ہمیں آسانی ہو گی لہٰذا  اس پر زیادہ  مت سوچیں اور کسی  ایک ترتیب سے حل کرنے کی کوشش کریں اور اگر  مشکلات  پیش آئیں تب تکمل کی ترتیب الٹ کر کے دوبارہ کوشش کریں۔
\انتہا{مثال}
%================
\begin{figure}
\centering
\begin{subfigure}{0.45\textwidth}
\centering
\begin{tikzpicture}[font=\small,declare function={fx(\r,\t)=\r*cos(\t);fy(\r,\t)=\r*sin(\t);}]
\begin{axis}[axis equal,clip=false,small,axis lines=center,enlargelimits=true, xlabel={$x$}, ylabel={$y$}, zlabel={$z$}, xtick={1},ytick={1},ztick={\empty},xlabel style={anchor=west}, ylabel style={anchor=south},zlabel style={anchor=south},xmin=-0.125]
\addplot[thick,fill=lgray,domain=0:90,variable=\t]({fx(1,t)},{fy(1,t)})node[pos=0.75,above right]{$x^2+y^2=1$}--(1,0)node[pos=0.25,above right]{$R$}node[pos=0.5,sloped,below]{$x+y=1$};
\addplot[]coordinates{(0,0)}node[below left]{$0$};
\end{axis}
\end{tikzpicture}
\caption{تکمل کے خطہ کا خاکہ بنائیں اور تحدیدی منحنیات کی نشاندہی کریں۔}
\end{subfigure}\hfill
\begin{subfigure}{0.45\textwidth}
\centering
\begin{tikzpicture}[font=\small,declare function={fx(\r,\t)=\r*cos(\t);fy(\r,\t)=\r*sin(\t);}]
\begin{axis}[axis equal,clip=false,small,axis lines=center,enlargelimits=true, xlabel={$x$}, ylabel={$y$}, zlabel={$z$}, xtick={1},ytick={1},ztick={\empty},xlabel style={anchor=west}, ylabel style={anchor=south},zlabel style={anchor=south},xmin=-0.125]
\addplot[thick,fill=lgray,domain=0:90,variable=\t]({fx(1,t)},{fy(1,t)})--(1,0)node[pos=0.3,above right]{$R$};
\addplot[-latex]coordinates {(0.6,-0.125)(0.6,1)}node[pos=0.25,left]{$L$};
\addplot[]coordinates{(0.6,0)}node[below right]{$x$};
\addplot[]coordinates {(0.6,0.4)}node[pin={[align=center,pin distance=1cm]10:{\text{دخول}\\  $y=1-x$}}]{};
\addplot[]coordinates {(0.6,0.8)}node[pin={[align=center,pin distance=0.25cm]45:{\text{خروج}\\  $y=\sqrt{1-x^2}$}}]{};
\addplot[]coordinates{(0,0)}node[below left]{$0$}node[pin={[align=center]-70:{\RL{\text{کم سے کم}}}\\$x$}]{};
\addplot[]coordinates{(1,0)}node[pin={[align=center]-110:{\RL{\text{زیادہ سے زیادہ}}\\$x$}}]{};
\end{axis}
\end{tikzpicture}
\caption{خطہ \عددی{R} میں جس نقاط پر انتصابی لکیر داخل اور خارج ہوتی ہے، ان کی نشاندہی کریں۔یہی  تکمل کے \عددی{y} حد  ہوں گے۔تمام    انتصابی لکیروں کو شامل کرنے والے \عددی{x} حدود کی نشاندہی کریں۔ یہی تکمل کے \عددی{x} حد ہوں گے۔}
\end{subfigure}
\caption{تکمل کے حدوں کی تلاش۔}
\label{شکل_بالکثرت_تکمل_کے_حدوں_کی_تلاش}
\end{figure}

\جزوحصہء{تکمل کی  حدوں  کی تلاش}
دوہرا تکمل کی قیمت کے حصول میں سب سے مشکل کام تکمل کی  حدیں  تلاش کرنا ہو سکتا ہے۔ خوش قسمتی سے   ایک  اچھا  طریقہ کار موجود ہے جس پر ہم چل سکتے ہیں۔ 

\موٹا{تکمل کی  حدیں   تلاش کرنے کا طریقہ کار }\\
(ا) \quad
   خطہ \عددی{R} پر \عددی{\iint_R f(x,y)\dif S} کی قیمت حاصل کرتے ہوئے  پہلے \عددی{y} اور بعد میں \عددی{x} کے لحاظ سے تکمل لینے کے لئے درج ذیل اقدام کریں۔
\begin{enumerate}[1.]
\item
\ترچھا{خاکہ:}\quad
تکمل کے   خطہ کا خاکہ بنائیں اور اس کی سرحدی منحنیات   پر نام و نشان لگائیں (شکل \حوالہ{شکل_بالکثرت_تکمل_کے_حدوں_کی_تلاش}-ا)۔
\item
\ترچھا{تکمل کی  \عددی{y} حدیں:}\quad
بڑھتی \عددی{y} رخ خطہ \عددی{R} سے گزرتا ہوا انتصابی  خط \عددی{L} کھینچیں۔ جن مقامات  پر \عددی{L}  اس خطہ میں داخل   اور اس سے     خارج ہوتا ہے، یہ تکمل کی  \عددی{y} حدیں  ہوں گی (شکل \حوالہ{شکل_بالکثرت_تکمل_کے_حدوں_کی_تلاش}-ب)۔
\item
 \ترچھا{تکمل کی  \عددی{x} حدیں:}\quad
متغیر   \عددی{x} کی وہ قیمتیں  منتخب کریں جن میں \عددی{R} سے گزرتی ہوئی تمام انتصابی  لکیریں شامل ہوں (شکل \حوالہ{شکل_بالکثرت_تکمل_کے_حدوں_کی_تلاش}-ب)۔ یہ قیمتیں تکمل کی \عددی{x} حدیں ہوں گی۔
\end{enumerate}
 تکمل درج ذیل ہو گا۔
\begin{align*}
\iint\limits_R f(x,y)\dif S=\int_{x=0}^{x=1}\int_{y=1-x}^{y=\sqrt{1-x^2}}f(x,y)\dif y\dif x
\end{align*}

(ب) \quad 
اسی دوہرا تکمل کو   بطور  بارہا تکمل حل کرتے ہوئے، ترتیب الٹ کرنے سے، انتصابی لکیروں کی بجائے  افقی لکیریں استعمال کریں (شکل \حوالہ{شکل_بالکثرت_بار_بار_تکمل_الٹ_ترتیب})۔ تکمل درج ذیل ہو گا۔
\begin{align*}
\iint\limits_R f(x,y)\dif S=\int_0^1\int_{1-y}^{\sqrt{1-y^2}}f(x,y)\dif x\dif y
\end{align*} 

\begin{figure}
\centering
\begin{tikzpicture}[font=\small,declare function={fx(\r,\t)=\r*cos(\t);fy(\r,\t)=\r*sin(\t);}]
\begin{axis}[axis equal,clip=false,small,axis lines=center,enlargelimits=true, xlabel={$x$}, ylabel={$y$}, zlabel={$z$}, xtick={1},ytick={1},ztick={\empty},xlabel style={anchor=west}, ylabel style={anchor=south},zlabel style={anchor=south},xmin=-0.125]
\addplot[thick,fill=lgray,domain=0:90,variable=\t]({fx(1,t)},{fy(1,t)})--(1,0)node[pos=0.3,above right]{$R$};
\addplot[-latex]coordinates {(-0.125,0.6)(1.25,0.6)}node[pos=0,left]{$y$};
\addplot[]coordinates{(0.6,0)}node[below right]{$x$};
\addplot[]coordinates {(0.4,0.6)}node[pin={[align=center,pin distance=1cm]60:{\text{دخول}\\  $x=1-y$}}]{};
\addplot[]coordinates {(0.8,0.6)}node[pin={[align=center,pin distance=0.5cm]10:{\text{خروج}\\  $x=\sqrt{1-y^2}$}}]{};
\addplot[]coordinates{(0,0)}node[below left]{$0$}node[pin={[align=center]170:{\RL{\text{کم سے کم}}}\\$y$}]{};
\addplot[]coordinates{(0,1)}node[pin={[align=center]-170:{\RL{\text{زیادہ سے زیادہ}}\\$y$}}]{};
\end{axis}
\end{tikzpicture}
\caption{بارہا تکمل میں ترتیب الٹ کرنے سے  \عددی{R} پر افقی لکیریں کھینچی جائیں گی۔}
\label{شکل_بالکثرت_بار_بار_تکمل_الٹ_ترتیب}
\end{figure}
%===============
\begin{figure}
\centering
\begin{subfigure}{0.45\textwidth}
\centering
\begin{tikzpicture}[font=\small,declare function={f(\x)=2*\x;g(\x)=(\x)^2;}]
\begin{axis}[clip=false,small,axis lines=center,enlargelimits=true, xlabel={$x$}, ylabel={$y$}, zlabel={$z$}, xtick={2},ytick={4},ztick={\empty},xlabel style={anchor=west}, ylabel style={anchor=south},zlabel style={anchor=south}]
\addplot[name path=kL,domain=0:2.15](x,{f(x)})node[pos=0.5,above left]{$y=2x$};
\addplot[name path=kR,domain=0:2.15](x,{g(x)})node[pos=0.5,below right]{$y=x^2$};
\addplot[lgray] fill between[of=kL and kR,soft clip={domain=0:2}];
\addplot[]coordinates{(2,4)}node[below right]{$(2,4)$};
\addplot[]coordinates{(0,0)}node[below left]{$0$};
\end{axis}
\end{tikzpicture}
\caption{}
\end{subfigure}\hfill
\begin{subfigure}{0.45\textwidth}
\centering
\begin{tikzpicture}[font=\small,declare function={f(\x)=2*\x;g(\x)=(\x)^2;}]
\begin{axis}[clip=false,small,axis lines=center,enlargelimits=true, xlabel={$x$}, ylabel={$y$}, zlabel={$z$}, xtick={2},ytick={4},ztick={\empty},xlabel style={anchor=west}, ylabel style={anchor=south},zlabel style={anchor=south}]
\addplot[name path=kL,domain=0:2.15](x,{f(x)});
\addplot[name path=kR,domain=0:2.15](x,{g(x)});
\addplot[lgray] fill between[of=kL and kR,soft clip={domain=0:2}];
\addplot[]coordinates{(2,4)}node[below right]{$(2,4)$};
\addplot[-latex]coordinates {(-0.125,2)(2.5,2)};
\addplot[]coordinates{(1,2)}node[pin=110:{$x=\frac{y}{2}$}]{};
\addplot[]coordinates{(sqrt(2),2)}node[pin=20:{$x=\sqrt{y}$}]{};
\addplot[]coordinates{(0,0)}node[below left]{$0$};
\end{axis}
\end{tikzpicture}
\caption{}
\end{subfigure}
\caption{دو منحنیات کے بیچ خطہ (مثال \حوالہ{مثال_بالکثرت_منحنیات_بیچ_خطہ_دوہرا_تکمل})}
\label{شکل_مثال_بالکثرت_منحنیات_بیچ_خطہ_دوہرا_تکمل}
\end{figure}
\ابتدا{مثال}\شناخت{مثال_بالکثرت_منحنیات_بیچ_خطہ_دوہرا_تکمل}
درج ذیل   تکمل کے خطہ تکمل  کا خاکہ  بنائیں اور تکمل کی ترتیب الٹ کرتے ہوئے اس   کا مساوی تکمل لکھیں۔
\begin{align*}
\int_0^2\int_{x^2}^{2x}(4x+2)\dif y\dif x
\end{align*}
حل:\quad
تکمل کا خطہ،  عدم مساوات \عددی{x^2\le y\le 2x}  اور \عددی{0\le x\le 2} دیتے ہیں۔ یوں اس خطہ کی  حدیں، خط \عددی{x=0}،  خط  \عددی{x=2} اور منحنیات \عددی{y=x^2} اور \عددی{y=2x} ہوں گی (شکل \حوالہ{شکل_مثال_بالکثرت_منحنیات_بیچ_خطہ_دوہرا_تکمل}-ا)۔

تکمل کی ترتیب الٹ کرتے ہوئے ہم اس خطہ  پر افقی لکیریں کھینچتے ہیں۔ یہ لکیریں اس خطہ میں \عددی{x=\tfrac{y}{2}} پر داخلی ہوتی ہیں اور \عددی{x=\sqrt{y}} پر اس سے خارج ہوتی ہیں۔ ان تمام افقی لکیریں کو شامل کرنے کے لئے ہمیں \عددی{y=0} سے \عددی{y=4} تک لینا ہو گا (شکل \حوالہ{شکل_مثال_بالکثرت_منحنیات_بیچ_خطہ_دوہرا_تکمل}-ب)۔ یوں  متبادل تکمل درج ذیل ہو گا۔
\begin{align*}
\int_0^4\int_{y/2}^{\sqrt{y}}(4x+2)\dif x\dif y
\end{align*}
ان دونوں تکملات کے جواب \عددی{8} ہے۔
\انتہا{مثال}
%=====================

\جزوحصہء{سوالات}
\ابتدا{سوالات}
\موٹا{تکمل کے  خطہ کی تلاش اور دوہرا تکملات}\\
سوال \حوالہ{سوال_بالکثرت_خطہ_خاکہ_الف} تا سوال \حوالہ{سوال_بالکثرت_خطہ_خاکہ_ب} میں تکمل کے  خطے کا خاکہ بنائیں اور تکمل کی قیمت تلاش کریں۔ 

\ابتدا{سوال}\شناخت{سوال_بالکثرت_خطہ_خاکہ_الف}
$\int_0^3\int_0^2(4-y^2)\dif y\dif x$
\انتہا{سوال}
%=================
\ابتدا{جواب}
\wf{\unexpanded{
$16$
\begin{center}
\begin{tikzpicture}[font=\scriptsize]
\pgfmathsetmacro{\a}{2.5}
\pgfmathsetmacro{\b}{1.5}
\draw[-latex](0,0)--(3,0)node[right]{$x$};
\draw[-latex](0,0)--(0,2)node[above]{$y$};
\draw[fill=lgray](0,0) rectangle++(\a,\b)node[above right]{$(3,2)$};
\draw(0,0)node[below left]{$0$} (\a,0)node[below]{$3$}  (0,\b)node[left]{$2$};
\end{tikzpicture}
\end{center}
}}
\انتہا{جواب}
%====================
\ابتدا{سوال}
$\int_0^3\int_{-2}^0 (x^2y-2xy)\dif y\dif x$
\انتہا{سوال}
%=================
\ابتدا{سوال}
$\int_{-1}^{0}\int_{-1}^1(x+y+1)\dif x\dif y$
\انتہا{سوال}
%=================
\ابتدا{جواب}
\wf{\unexpanded{
$1$
\begin{center}
\begin{tikzpicture}[font=\scriptsize]
\pgfmathsetmacro{\a}{2}
\pgfmathsetmacro{\b}{1}
\draw[-latex](-1.25,0)--(1.5,0)node[right]{$x$};
\draw[-latex](0,-1.25)--(0,0.5)node[above]{$y$};
\draw[fill=lgray](-1,0) rectangle++(\a,-\b)node[right]{$(1,-1)$};
\draw(-1,0)node[below left]{$-1$} (-1,-\b)node[left]{$(-1,-1)$} (1,0)node[below right]{$1$};
\end{tikzpicture}
\end{center}
}}
\انتہا{جواب}
%====================
\ابتدا{سوال}
$\int_{\pi}^{2\pi}\int_0^{\pi}(\sin x+\cos y)\dif x\dif y$
\انتہا{سوال}
%=================
\ابتدا{سوال}
$\int_0^{\pi}\int_0^x x\sin y\dif y\dif x$
\انتہا{سوال}
%=================
\ابتدا{جواب}
\wf{\unexpanded{
$\frac{\pi^2}{2}+2$
\begin{center}
\begin{tikzpicture}[font=\scriptsize]
\pgfmathsetmacro{\a}{1}
\pgfmathsetmacro{\b}{1}
\draw[-latex](0,0)--(1.5,0)node[right]{$x$};
\draw[-latex](0,0)--(0,1.5)node[above]{$y$};
\draw[fill=lgray](0,0)--(\a,\b)node[right]{$(\pi,\pi)$}--(\a,0)node[below]{$\pi$}--cycle;
\draw(0,1)node[left]{$\pi$}--++(0.1,0) (0,0)node[below left]{$0$};
\end{tikzpicture}
\end{center}
}}
\انتہا{جواب}
%====================
\ابتدا{سوال}
$\int_0^{\pi}\int_0^{\sin x}y\dif y\dif x$
\انتہا{سوال}
%=================
\ابتدا{سوال}
$\int_1^{\ln 8}\int_0^{\ln y}e^{x+y}\dif x\dif y$
\انتہا{سوال}
%=================
\ابتدا{جواب}
\wf{\unexpanded{
$8\ln 8-16+e$
\begin{center}
\begin{tikzpicture}[font=\scriptsize,declare function={f(\y)=ln(\y);}]
\pgfmathsetmacro{\a}{ln(8)}
\pgfmathsetmacro{\b}{ln(ln(8))}
\begin{axis}[axis on top,small,axis lines=middle,xlabel={$x$},ylabel={$y$},xtick={\b},xticklabels={$\ln\ln 8$},ytick={1,\a},yticklabels={$1$,$\ln 8$},ymin=0,enlargelimits=true,xlabel style={at={(current axis.right of origin)},anchor=west},ylabel style={at={(current axis.above origin)},anchor=south},xmax=1.2]
\addplot[name path=kf,domain=1:ln(8)]({f(x)},{x})node[pos=0.25,below right]{$x=\ln y$}node[right]{$(\ln\ln8,\ln 8)$};
\addplot[draw=none,name path=ky]coordinates{(0,1)(0,\a)};
\addplot[]coordinates{(0,\a)(\b,\a)};
\addplot[lgray]fill between[of={kf and ky}];
\end{axis}
\end{tikzpicture}
\end{center}
}}
\انتہا{جواب}
%====================
\ابتدا{سوال}
$\int_1^2\int_y^{y^2}\dif x\dif y$
\انتہا{سوال}
%=================
\ابتدا{سوال}
$\int_0^1\int_0^{y^2}3y^3e^{xy}\dif x\dif y$
\انتہا{سوال}
%=================
\ابتدا{جواب}
\wf{\unexpanded{
$e-2$
\begin{center}
\begin{tikzpicture}[font=\scriptsize,declare function={f(\y)=(\y)^2;}]
\pgfmathsetmacro{\a}{0}
\pgfmathsetmacro{\b}{1}
\begin{axis}[axis on top,small,axis lines=middle,xlabel={$x$},ylabel={$y$},xtick={1},xticklabels={$1$},ytick={1},yticklabels={$1$}, ymin=0,enlargelimits=true,xlabel style={at={(current axis.right of origin)},anchor=west},ylabel style={at={(current axis.above origin)},anchor=south},xmax=1.1]
\addplot[name path=kf,domain=0:1]({f(x)},{x})node[pos=0.5,below right]{$x=y^2$}node[right]{$(1,1)$};
\addplot[draw=none,name path=ky]coordinates{(0,0)(0,1)};
\addplot[]coordinates{(0,1)(1,1)};
\addplot[lgray]fill between[of={kf and ky}];
\end{axis}
\end{tikzpicture}
\end{center}
}}
\انتہا{جواب}
%====================
\ابتدا{سوال}\شناخت{سوال_بالکثرت_خطہ_خاکہ_ب}
$\int_1^4\int_0^{\sqrt{x}}\frac{3}{2}e^{y/\sqrt{x}}\dif y\dif x$
\انتہا{سوال}
%=================

سوال \حوالہ{سوال_بالکثرت_دیا_خطہ_تکمل_الف} تا سوال \حوالہ{سوال_بالکثرت_دیا_خطہ_تکمل_ب} میں \عددی{f} کو دیے ہوئے خطہ پر تکمل کریں۔\\
\ابتدا{سوال}\شناخت{سوال_بالکثرت_دیا_خطہ_تکمل_الف}
ربع اول میں لکیر \عددی{y=x}، \عددی{y=2x}، \عددی{x=1} اور \عددی{x=2}  کے بیچ خطہ پر تفاعل \عددی{f(x,y)=\tfrac{x}{y}} کا تکمل۔
\انتہا{سوال}
%=================
\ابتدا{جواب}
\wf{\unexpanded{
$\frac{3}{2}\ln 2$
}}
\انتہا{جواب}
%====================
\ابتدا{سوال}
چکور \عددی{1\le x\le 2,\, 1\le y\le 2} پر تفاعل \عددی{f(x,y)=\tfrac{1}{xy}}  کا تکمل۔
\انتہا{سوال}
%==================
\ابتدا{سوال}
مثلث خطہ جس کے راس \عددی{(0,0)}، \عددی{(1,0)} اور \عددی{(0,1)} ہیں میں تفاعل \عددی{f(x,y)=x^2+y^2} کا تکمل۔
\انتہا{سوال}
%=================
\ابتدا{جواب}
\wf{\unexpanded{
$\frac{1}{6}$
}}
\انتہا{جواب}
%====================
\ابتدا{سوال}
مستطیل \عددی{0\le x\le \pi,\, 0\le y\le 1} پر تفاعل \عددی{f(x,y)=y\cos xy} کا تکمل۔
\انتہا{سوال}
%==================
\ابتدا{سوال}
مستوی \عددی{uv} کے ربع اول میں  لکیر \عددی{u+v=1} کے نیچے تفاعل \عددی{f(u,v)=v-\sqrt{u}} کا تکمل۔ 
\انتہا{سوال}
%================
\ابتدا{جواب}
\wf{\unexpanded{
$-\frac{1}{10}$
}}
\انتہا{جواب}
%====================
\ابتدا{سوال}\شناخت{سوال_بالکثرت_دیا_خطہ_تکمل_ب}
مستوی \عددی{st} کے ربع اول میں  منحنی \عددی{s=\ln t} کے اوپر جانب \عددی{t=1} سے \عددی{t=2} تک تفاعل \عددی{f(s,t)=e^s\ln t} کا تکمل۔
\انتہا{سوال}
%==============
سوال \حوالہ{سوال_بالکثرت_خاکہ_بنائیں_قیمت_حاصل_کریں_الف} تا سوال \حوالہ{سوال_بالکثرت_خاکہ_بنائیں_قیمت_حاصل_کریں_ب} میں تکملات دیے گئے ہیں۔ ان تکملات کے خطوں کا خاکہ  بنائیں اور تکمل کی قیمت حاصل کریں۔

\ابتدا{سوال}\شناخت{سوال_بالکثرت_خاکہ_بنائیں_قیمت_حاصل_کریں_الف}
$\int_{-2}^0\int_v^{-v}2\dif p\dif v$\quad
مستوی \عددی{pv}
\انتہا{سوال}
%=================
\ابتدا{جواب}
\wf{\unexpanded{
$8$
\begin{center}
\begin{tikzpicture}[font=\scriptsize,declare function={f(\y)=ln(\y);}]
\pgfmathsetmacro{\a}{ln(8)}
\pgfmathsetmacro{\b}{ln(ln(8))}
\begin{axis}[clip=false,axis on top,small,axis lines=middle,xlabel={$p$},ylabel={$v$},xtick={-2,2},xticklabels={$-2$,$2$},ytick={-2}, yticklabels={$-2$},enlargelimits=true,xlabel style={at={(current axis.right of origin)},anchor=west},ylabel style={at={(current axis.above origin)},anchor=south}]
\addplot[fill=lgray]coordinates{(0,0)(2,-2)(-2,-2)(0,0)};
\addplot[]coordinates{(-2,-2)}node[left]{$(-2,-2)$};
\addplot[]coordinates{(2,-2)}node[right]{$(2,-2)$};
\addplot[]coordinates{(-1,-1)}node[above left]{$v=p$};
\addplot[]coordinates{(1,-1)}node[above right]{$v=-p$};
\end{axis}
\end{tikzpicture}
\end{center}
}}
\انتہا{جواب}
%====================
\ابتدا{سوال}
$\int_0^1\int_0^{\sqrt{1-s^2}}8t\dif t\dif s$\quad
مستوی \عددی{st}
\انتہا{سوال}
%====================
\ابتدا{سوال}
$\int_{-\pi/3}^{\pi/3}\int_0^{\sec t}3\cos t\dif u\dif t$\quad
مستوی \عددی{tu}
\انتہا{سوال}
%====================
\ابتدا{جواب}
\wf{\unexpanded{
$2\pi$
\begin{center}
\begin{tikzpicture}[font=\scriptsize,declare function={f(\x)=1/(cos(deg(\x)));}]
\pgfmathsetmacro{\a}{-pi/3}
\pgfmathsetmacro{\b}{pi/3}
\begin{axis}[clip=false,axis on top,small,axis lines=middle,xlabel={$t$},ylabel={$u$},xtick={\a,\b},xticklabels={$-\tfrac{\pi}{3}$,$\tfrac{\pi}{3}$},ytick={1,2},yticklabels={$1$,$2$}, ymin=0,enlargelimits=true,xlabel style={at={(current axis.right of origin)},anchor=west},ylabel style={at={(current axis.above origin)},anchor=south},xmax=1.1]
\addplot[name path=kf,domain=\a:\b](x,{f(x)})node[pos=0.75,above left]{$u=\sec t$}node[above,pos=0]{$(-\tfrac{\pi}{3},2)$}node[above]{$(\tfrac{\pi}{3},2)$};
\addplot[draw=none,name path=kx]coordinates{(\a,0)(\b,0)};
\addplot[lgray]fill between[of={kf and kx}];
\addplot[]coordinates{(\a,0)(\a,2)};
\addplot[]coordinates{(\b,0)(\b,2)};
\end{axis}
\end{tikzpicture}
\end{center}
}}
\انتہا{جواب}
%====================
\ابتدا{سوال}\شناخت{سوال_بالکثرت_خاکہ_بنائیں_قیمت_حاصل_کریں_ب}
$\int_0^3\int_{-2}^{4-2u}\frac{4-2u}{v^2}\dif v\dif u$\quad
مستوی \عددی{uv}
\انتہا{سوال}
%====================

\موٹا{تکمل کی الٹ ترتیب}\\
سوال \حوالہ{سوال_بالکثرت_الٹ_لکھیں_الف} تا سوال \حوالہ{سوال_بالکثرت_الٹ_لکھیں_ب} میں تکمل کے خطہ کا خاکہ بنا کر معادل الٹ ترتیب کا تکمل لکھیں۔

\ابتدا{سوال}\شناخت{سوال_بالکثرت_الٹ_لکھیں_الف}
$\int_0^1\int_2^{4-2x}\dif y\dif x$
\انتہا{سوال}
%===============
\ابتدا{جواب}
\wf{\unexpanded{
$\int_2^4\int_0^{(4-y)/2}\dif x\dif y$
\begin{center}
\begin{tikzpicture}[font=\scriptsize,declare function={f(\x)=4-2*\x;}]
\pgfmathsetmacro{\a}{0}
\pgfmathsetmacro{\b}{1}
\begin{axis}[clip=false,axis on top,small,axis lines=middle,xlabel={$x$},ylabel={$y$},xtick={1},xticklabels={$1$},ytick={2,4},yticklabels={$2$,$4$}, ymin=0,enlargelimits=true,xlabel style={at={(current axis.right of origin)},anchor=west},ylabel style={at={(current axis.above origin)},anchor=south},xmax=1.1]
\addplot[name path=kf,domain=\a:\b](x,{f(x)})node[pos=0.5,above right]{$y=4-2x$}node[right]{$(1,2)$};
\addplot[name path=kx]coordinates{(0,2)(1,2)};
\addplot[lgray]fill between[of={kf and kx}];
\end{axis}
\end{tikzpicture}
\end{center}
}}
\انتہا{جواب}
%====================
\ابتدا{سوال}
$\int_0^2\int_{y-2}^0\dif x\dif y$
\انتہا{سوال}
%================
\ابتدا{سوال}
$\int_0^1\int_y^{\sqrt{y}}\dif x\dif y$
\انتہا{سوال}
%================
\ابتدا{جواب}
\wf{\unexpanded{
$\int_0^1\int_{x^2}^{x}\dif y\dif x$
\begin{center}
\begin{tikzpicture}[font=\scriptsize,declare function={f(\x)=\x^2;g(\x)=\x;}]
\pgfmathsetmacro{\a}{0}
\pgfmathsetmacro{\b}{1}
\begin{axis}[clip=false,axis on top,small,axis lines=middle,xlabel={$x$},ylabel={$y$},xtick={1},xticklabels={$1$},ytick={1},yticklabels={$1$}, ymin=0,enlargelimits=true,xlabel style={at={(current axis.right of origin)},anchor=west},ylabel style={at={(current axis.above origin)},anchor=south},xmax=1.1]
\addplot[name path=kf,domain=\a:\b](x,{f(x)})node[pos=0.5,below right]{$y=x^2$}node[above,right]{$(1,1)$};
\addplot[name path=kg,domain=\a:\b](x,{g(x)})node[pos=0.5,above left]{$y=x$};
\addplot[lgray]fill between[of={kf and kg}];
\end{axis}
\end{tikzpicture}
\end{center}
}}
\انتہا{جواب}
%====================
\ابتدا{سوال}
$\int_0^1\int_{1-x}^{1-x^2}\dif y\dif x$
\انتہا{سوال}
%================
\ابتدا{سوال}
$\int_0^1\int_1^{e^x}\dif y\dif x$
\انتہا{سوال}
%================
\ابتدا{جواب}
\wf{\unexpanded{
$\int_1^e\int_{\ln y}^{1}\dif x\dif y$
\begin{center}
\begin{tikzpicture}[font=\scriptsize,declare function={f(\x)=e^(\x);}]
\pgfmathsetmacro{\a}{0}
\pgfmathsetmacro{\b}{1}
\pgfmathsetmacro{\e}{2.718}
\begin{axis}[clip=false,axis on top,small,axis lines=middle,xlabel={$x$},ylabel={$y$},xtick={1},xticklabels={$1$},ytick={1,\e},yticklabels={$1$,$e$}, ymin=0,enlargelimits=true,xlabel style={at={(current axis.right of origin)},anchor=west},ylabel style={at={(current axis.above origin)},anchor=south},xmax=1.1]
\addplot[name path=kf,domain=\a:\b](x,{f(x)})node[pos=0.5,above left]{$y=e^x$}node[above,right]{$(1,e)$};
\addplot[name path=kx]coordinates{(0,1)(1,1)}node[right]{$(1,1)$};
\addplot[lgray]fill between[of={kf and kx}];
\addplot[]coordinates{(1,1)(1,\e)};
\end{axis}
\end{tikzpicture}
\end{center}
}}
\انتہا{جواب}
%====================
\ابتدا{سوال}
$\int_0^{\ln 2}\int_{e^y}^2\dif x\dif y$
\انتہا{سوال}
%================
\ابتدا{سوال}
$\int_0^{3/2}\int_0^{9-4x^2}16x\dif y\dif x$
\انتہا{سوال}
%================
\ابتدا{جواب}
\wf{\unexpanded{
$\int_0^9\int_{0}^{(\sqrt{9-y})/2}16x\dif x\dif y$
\begin{center}
\begin{tikzpicture}[font=\scriptsize,declare function={f(\x)=9-4*(\x)^2;}]
\pgfmathsetmacro{\a}{0}
\pgfmathsetmacro{\b}{3/2}
\pgfmathsetmacro{\e}{2.718}
\begin{axis}[clip=false,axis on top,small,axis lines=middle,xlabel={$x$},ylabel={$y$},xtick={1.5},xticklabels={$\tfrac{3}{2}$},ytick={9},yticklabels={$9$}, ymin=0,enlargelimits=true,xlabel style={at={(current axis.right of origin)},anchor=west},ylabel style={at={(current axis.above origin)},anchor=south}]
\addplot[name path=kf,domain=\a:\b](x,{f(x)})node[pos=0.5,above right]{$y=9-4x^2$};
\addplot[name path=kx,draw=none]coordinates{(0,0)(1.5,0)};
\addplot[lgray]fill between[of={kf and kx}];
\end{axis}
\end{tikzpicture}
\end{center}
}}
\انتہا{جواب}
%====================
\ابتدا{سوال}
$\int_0^2\int_0^{4-y^2}y\dif x\dif y$
\انتہا{سوال}
%================
\ابتدا{سوال}
$\int_0^1\int_{-\sqrt{1-y^2}}^{\sqrt{1-y^2}}3y\dif x\dif y$
\انتہا{سوال}
%================
\ابتدا{جواب}
\wf{\unexpanded{
$\int_{-1}^{1}\int_{0}^{\sqrt{1-x^2}}3y\dif y\dif x$
\begin{center}
\begin{tikzpicture}[font=\scriptsize,declare function={fx(\x)=cos(\x);fy(\x)=sin(\x);}]
\pgfmathsetmacro{\a}{0}
\pgfmathsetmacro{\b}{180}
\pgfmathsetmacro{\e}{2.718}
\begin{axis}[axis equal,clip=false,axis on top,small,axis lines=middle,xlabel={$x$},ylabel={$y$},xtick={-1,1},xticklabels={$-1$,$1$},ytick={1},yticklabels={$1$}, ymin=0,ymax=1.25,enlargelimits=true,xlabel style={at={(current axis.right of origin)},anchor=west},ylabel style={at={(current axis.above origin)},anchor=south}]
\addplot[name path=kf,domain=\a:\b]({fx(x)},{fy(x)})node[pos=0.25,above right]{$x^2+y^2=1$};
\addplot[name path=kx]coordinates{(-1,0)(1,0)};
\addplot[lgray]fill between[of={kf and kx}];
\end{axis}
\end{tikzpicture}
\end{center}
}}
\انتہا{جواب}
%====================
\ابتدا{سوال}\شناخت{سوال_بالکثرت_الٹ_لکھیں_ب}
$\int_0^2\int_{-\sqrt{4-x^2}}^{\sqrt{4-x^2}}6x\dif y\dif x$
\انتہا{سوال}
%================

\موٹا{دوہرا تکمل کی قیمت کا حصول}\\
سوال \حوالہ{سوال_بالکثرت_ترتیب_تعین_الف} تا سوال \حوالہ{سوال_بالکثرت_ترتیب_تعین_ب} میں  تکمل کے خطہ کا خاکہ بنا کر تکمل کی ترتیب تعین کرتے ہوئے  تکمل کی قیمت تلاش کریں۔

\ابتدا{سوال}\شناخت{سوال_بالکثرت_ترتیب_تعین_الف}
$\int_0^{\pi}\int_x^{\pi}\frac{\sin y}{y}\dif y\dif x$
\انتہا{سوال}
%==================
\ابتدا{جواب}
\wf{\unexpanded{
$2$
\begin{center}
\begin{tikzpicture}[font=\scriptsize,declare function={f(\x)=\x;}]
\pgfmathsetmacro{\a}{0}
\pgfmathsetmacro{\b}{pi}
\begin{axis}[clip=false,axis on top,small,axis lines=middle,xlabel={$x$},ylabel={$y$},xtick={\b},xticklabels={$\pi$},ytick={\b},yticklabels={$\pi$}, ymin=0,enlargelimits=true,xlabel style={at={(current axis.right of origin)},anchor=west},ylabel style={at={(current axis.above origin)},anchor=south},xmax=4]
\addplot[name path=kf,domain=\a:\b](x,{f(x)})node[pos=0.5,below right]{$y=x$}node[right]{$(\pi,\pi)$};
\addplot[name path=kx]coordinates{(0,\b)(\b,\b)};
\addplot[lgray]fill between[of={kf and kx}];
\addplot[]coordinates{(0,\b)(\b,\b)};
\end{axis}
\end{tikzpicture}
\end{center}
}}
\انتہا{جواب}
%====================
\ابتدا{سوال}
$\int_0^2\int_x^22y^2\sin xy\dif y\dif x$
\انتہا{سوال}
%================
\ابتدا{سوال}
$\int_0^1\int_y^1 x^2e^{xy}\dif x\dif y$
\انتہا{سوال}
%================
\ابتدا{جواب}
\wf{\unexpanded{
$\frac{e-2}{2}$
\begin{center}
\begin{tikzpicture}[font=\scriptsize,declare function={f(\x)=\x;}]
\pgfmathsetmacro{\a}{0}
\pgfmathsetmacro{\b}{1}
\begin{axis}[clip=false,axis on top,small,axis lines=middle,xlabel={$x$},ylabel={$y$},xtick={1},xticklabels={$1$},ytick={1},yticklabels={$1$}, ymin=0,enlargelimits=true,xlabel style={at={(current axis.right of origin)},anchor=west},ylabel style={at={(current axis.above origin)},anchor=south},xmax=1.25]
\addplot[name path=kf,domain=\a:\b](x,{f(x)})node[pos=0.5,above left]{$y=x$}node[above,right]{$(1,1)$};
\addplot[name path=kx]coordinates{(0,0)(1,0)};
\addplot[lgray]fill between[of={kf and kx}];
\addplot[]coordinates{(1,0)(1,1)};
\end{axis}
\end{tikzpicture}
\end{center}
}}
\انتہا{جواب}
%====================
\ابتدا{سوال}
$\int_0^2\int_0^{4-x^2}\frac{xe^{2y}}{4-y}\dif y\dif x$
\انتہا{سوال}
%================
\ابتدا{سوال}
$\int_0^{2\sqrt{\ln 3}}\int_{y/2}^{\sqrt{\ln 3}}e^{x^2}\dif x\dif y$
\انتہا{سوال}
%================
\ابتدا{جواب}
\wf{\unexpanded{
$2$
\begin{center}
\begin{tikzpicture}[font=\scriptsize,declare function={f(\x)=2*\x;}]
\pgfmathsetmacro{\a}{0}
\pgfmathsetmacro{\b}{sqrt(ln(3))}
\pgfmathsetmacro{\c}{2*\b}
\begin{axis}[clip=false,axis on top,small,axis lines=middle,xlabel={$x$},ylabel={$y$},xtick={\b},xticklabels={$\sqrt{\ln 3}$},ytick={\c},yticklabels={$2\sqrt{\ln 3}$}, ymin=0,enlargelimits=true,xlabel style={at={(current axis.right of origin)},anchor=west},ylabel style={at={(current axis.above origin)},anchor=south},xmax=1.5]
\addplot[name path=kf,domain=\a:\b](x,{f(x)})node[pos=0.5,above left]{$y=2x$}node[above,right]{$(\sqrt{\ln 3},2\sqrt{\ln 3})$};
\addplot[name path=kx]coordinates{(0,0)(\b,0)};
\addplot[lgray]fill between[of={kf and kx}];
\addplot[]coordinates{(\b,0)(\b,\c)};
\end{axis}
\end{tikzpicture}
\end{center}
}}
\انتہا{جواب}
%====================
\ابتدا{سوال}
$\int_0^3\int_{\sqrt{x/3}}^1e^{y^3}\dif y\dif x$
\انتہا{سوال}
%================
\ابتدا{سوال}
$\int_0^{1/{16}}\int_{y^{1/4}}^{1/2}\cos(16\pi x^5)\dif x\dif y$
\انتہا{سوال}
%================
\ابتدا{جواب}
\wf{\unexpanded{
$\tfrac{1}{80\pi}$
\begin{center}
\begin{tikzpicture}[font=\scriptsize,declare function={f(\x)=\x^4;}]
\pgfmathsetmacro{\a}{0}
\pgfmathsetmacro{\b}{0.5}
\pgfmathsetmacro{\c}{0.0625}
\begin{axis}[scaled y ticks=false,clip=false,axis on top,small,axis lines=middle,xlabel={$x$},ylabel={$y$},xtick={\b},xticklabels={$0.5$},ytick={\c},yticklabels={$0.0625$}, ymin=0,enlargelimits=true,xlabel style={at={(current axis.right of origin)},anchor=west},ylabel style={at={(current axis.above origin)},anchor=south},xmax=0.75]
\addplot[name path=kf,domain=\a:\b](x,{f(x)})node[pos=0.75,above left]{$y=x^4$}node[above,]{$(0.5,0.0625)$};
\addplot[name path=kx]coordinates{(0,0)(\b,0)};
\addplot[lgray]fill between[of={kf and kx}];
\addplot[]coordinates{(\b,0)(\b,\c)};
\end{axis}
\end{tikzpicture}
\end{center}
}}
\انتہا{جواب}
%====================
\ابتدا{سوال}
$\int_0^8\int_{\sqrt[3]{x}}^2\frac{\dif y\dif x}{y^4+1}$
\انتہا{سوال}
%================
\ابتدا{سوال}
$\iint\limits_R (y-2x^2)\dif S$
جہاں \عددی{R} چکور \عددی{\abs{x}+\abs{y}=1} کا اندرونی خطہ  ہے۔
\انتہا{سوال}
%================
\ابتدا{جواب}
\wf{\unexpanded{
$-\tfrac{2}{3}$
}}
\انتہا{جواب}
%====================
\ابتدا{سوال}\شناخت{سوال_بالکثرت_ترتیب_تعین_ب}
$\iint\limits_R xy\dif S$
جہاں لکیر \عددی{y=x}، \عددی{y=2x} اور \عددی{x+y=2} کے بیچ خطہ \عددی{R} ہے۔
\انتہا{سوال}
%================

\موٹا{سطح \عددی{z=f(x,y)} کے نیچے حجم}\\
\ابتدا{سوال}
مستوی \عددی{xy} میں لکیر \عددی{y=x}، \عددی{x=0} اور \عددی{x+y=2} کے بیچ مثلث کے اور  قطع مکافی سطح  \عددی{z=x^2+y^2} کے نیچے  خطہ کا حجم تلاش کریں۔
\انتہا{سوال}
%===========
\ابتدا{جواب}
\wf{\unexpanded{
$\tfrac{4}{3}$
}}
\انتہا{جواب}
%====================
\ابتدا{سوال}
ایک ٹھوس جسم  اوپر سے بیلن \عددی{z=x^2} اور نیچے سے مستوی \عددی{xy} میں لکیر \عددی{y=x} اور قطع مکافی \عددی{y=2-x^2}  کے بیچ مثلث   خطہ کے درمیان پایا جاتا ہے۔ اس جسم  کا حجم تلاش کریں۔
\انتہا{سوال}
%===============
\ابتدا{سوال}
ایک ٹھوس جسم  کا قاعدہ مستوی \عددی{xy} میں  لکیر \عددی{y=3x} اور قطع مکافی \عددی{y=4-x^2} کے بیچ خطہ ہے جبکہ اس کا بالائی سر مستوی \عددی{z=x+4} پر مشتمل ہے۔  اس جسم کا حجم تلاش کریں۔
\انتہا{سوال}
%=============
\ابتدا{جواب}
\wf{\unexpanded{
$\tfrac{625}{12}$
}}
\انتہا{جواب}
%====================
\ابتدا{سوال}
ثُمن  اول میں  محددی مستویات،  بیلن \عددی{x^2+y^2=4} اور مستوی \عددی{z+y=3} کے بیچ ٹھوس جسم کا حجم تلاش کریں۔ 
\انتہا{سوال}
%============
\ابتدا{سوال}
ثُمن اول میں  محددی مستویات، مستوی \عددی{x=3}  اور قطع مکافی بیلن \عددی{z=4-y^2} کے بیچ ٹھوس جسم کا حجم تلاش کریں۔
\انتہا{سوال}
%===========
\ابتدا{جواب}
\wf{\unexpanded{
$16$
}}
\انتہا{جواب}
%====================
\ابتدا{سوال}
ثُمن اول سے  سطح \عددی{z=4-x^2-y} ایک ٹھوس جسم کاٹتی ہے۔ اس جسم کا حجم تلاش کریں۔
\انتہا{سوال}
%=================
\ابتدا{سوال}
ثُمن اول سے بیلن \عددی{z=12-3y^2} اور مستوی \عددی{x+y=2}  ایک  پچر کاٹتے ہیں۔ اس پچر کا حجم تلاش کریں۔
\انتہا{سوال}
%===========
\ابتدا{جواب}
\wf{\unexpanded{
$20$
}}
\انتہا{جواب}
%====================
\ابتدا{سوال}
چکور ستون   \عددی{\abs{x}+\abs{y}\le 1} سے مستویات  \عددی{z=0} اور \عددی{3x+z=3}  جس ٹھوس جسم کو کاٹتے ہیں اس کا حجم تلاش کریں۔
\انتہا{سوال}
%==============
\ابتدا{سوال}
ایک ٹھوس جسم  سامنے اور پشت سے  مستویات \عددی{x=2} اور \عددی{x=1}،  اطراف سے بیلن \عددی{y=\mp\tfrac{1}{x}}، اوپر سے مستوی \عددی{z=x+1} اور نیچے سے مستوی \عددی{z=0} میں گھیرا ہوا ہے۔ اس جسم کا حجم تلاش کریں۔
\انتہا{سوال}
%=============
\ابتدا{جواب}
\wf{\unexpanded{
$2(1+\ln 2)$
}}
\انتہا{جواب}
%====================
\ابتدا{سوال}
ایک جسم سامنے اور پشت سے مستویات \عددی{x=\pm \tfrac{\pi}{3}}، اطراف سے بیلن \عددی{y=\mp\sec x} ، اوپر سے بیلن \عددی{z=1+y^2} اور نیچے سے مستوی \عددی{xy} میں گھیرا ہوا ہے۔ اس جسم کا حجم تلاش کریں۔
\انتہا{سوال}
%==========
\موٹا{غیر محدود خطوں پر تکملات}\\
سوال \حوالہ{سوال_بالکثرت_غیر_مناسب_الف} تا سوال \حوالہ{سوال_بالکثرت_غیر_مناسب_ب} میں غیر مناسب تکملات کو بارہا تکمل تصور کرتے ہوئے ان  کی قیمت تلاش کریں۔

\ابتدا{سوال}\شناخت{سوال_بالکثرت_غیر_مناسب_الف}
$\int_1^{\infty}\int_{e^{-x}}^1\frac{1}{x^3y}\dif y\dif x$
\انتہا{سوال}
%==================
\ابتدا{جواب}
\wf{\unexpanded{
$1$
}}
\انتہا{جواب}
%====================
\ابتدا{سوال}
$\int_{-1}^1\int_{-1/\sqrt{1-x^2}}^{1/\sqrt{1-x^2}}(2y+1)\dif y\dif x$
\انتہا{سوال}
%=======================
\ابتدا{سوال}
$\int_{-\infty}^{\infty}\int_{-\infty}^{\infty}\frac{1}{(x^2+1)(y^2+1)}\dif x\dif y$
\انتہا{سوال}
%=======================
\ابتدا{جواب}
\wf{\unexpanded{
$\pi^2$
}}
\انتہا{جواب}
%====================
\ابتدا{سوال}\شناخت{سوال_بالکثرت_غیر_مناسب_ب}
$\int_0^{\infty}\int_0^{\infty}xe^{-(x+2y)}\dif x\dif y$
\انتہا{سوال}
%=======================

\موٹا{دوہرا تکملات کی تخمین}\\
سوال \حوالہ{سوال_بالکثرت_تخمینی_دوہرا_الف} اور سوال \حوالہ{سوال_بالکثرت_تخمینی_دوہرا_ب} میں تفاعل  \عددی{f(x,y)} کے  دوہرا تکمل   کے خطہ \عددی{R} کو انتصابی خط \عددی{x=a} اور افقی خط \عددی{y=c} خانہ بند کرتی ہیں۔ ہر  ذیلی مستطیل میں دکھائے گئے   \عددی{(x_k,y_k)} لیتے ہوئے  درج  ذیل تخمین استعمال کر کے دوہرا تکملات کی تخمینی قیمتیں تلاش کریں۔
\begin{align*}
\iint\limits_R f(x,y)\dif S\approx \sum\limits_{k=1}^{n}f(x_k,y_k)\Delta S_k
\end{align*}

\ابتدا{سوال}\شناخت{سوال_بالکثرت_تخمینی_دوہرا_الف}
تفاعل \عددی{f(x,y)=x+y}  اور خطہ \عددی{R}،   جو   نصف دائرہ \عددی{y=\sqrt{1-x^2}} اور    محور \عددی{x} کے بیچ ہے۔ خانہ بندی \عددی{x=-1,-1/2,0,1/4,1/2,1} اور \عددی{y=0,1/2,1} لیں۔ نقطہ \عددی{(x_k,y_k) } کو \عددی{k} واں  خانے  کا نچلا بایاں کونا لیں بشرطیکہ یہ مستطیل \عددی{R} کے اندر پایا جاتا ہو ۔
\انتہا{سوال}
%================
\ابتدا{جواب}
\wf{\unexpanded{
$-\tfrac{1}{4}$
}}
\انتہا{جواب}
%====================
\ابتدا{سوال}\شناخت{سوال_بالکثرت_تخمینی_دوہرا_ب}
تفاعل \عددی{f(x,y)=x+2y} ہے جبکہ   اور دائرہ \عددی{(x-2)^2+(y-3)^2=1} کا اندرونی  خطہ \عددی{R} ہے۔ خانہ بندی \عددی{x=1,3/2,2,5/2,3} اور \عددی{y=2,5/2,3,7/2,4} لیں ۔بشرطیکہ \عددی{k} واں مستطیل \عددی{R} میں پایا جاتا ہو، \عددی{k} ویں مستطیل کے وسطانی مرکز کو   \عددی{(x_k,y_k)}  لیں۔
\انتہا{سوال}
%=======================
\موٹا{نظریہ اور مثالیں}\\
\ابتدا{سوال}
قرص \عددی{x^2+y^2\le 4}  کو  شعاع \عددی{\theta=\tfrac{\pi}{6}} اور \عددی{\theta=\tfrac{\pi}{2}}  دو ٹکڑوں میں تقسیم کرتے ہیں۔ ان میں سے چھوٹے ٹکڑے پر \عددی{f(x,y)=\sqrt{4-x^2}}  کا تکمل لیں۔ 
\انتہا{سوال}
%===========
\ابتدا{جواب}
\wf{\unexpanded{
$\tfrac{20\sqrt{3}}{9}$
}}
\انتہا{جواب}
%====================
\ابتدا{سوال}
لا متناہی مستطیل \عددی{2\le x\le \infty,\, 0\le y\le 2} پر \عددی{f(x,y)=\tfrac{1}{(x^2-x)(y-1)^{2/3}}} کا تکمل لیں۔ 
\انتہا{سوال}
%===================
\ابتدا{سوال}
ایک ٹھوس (غیر دائری) قائمہ بیلن کا قاعدہ \عددی{xy} مستوی ہے جبکہ اس کی بالائی سرحد قطع مکافی سطح  \عددی{z=x^2+y^2} ہے۔ اس بیلن کا حجم
\begin{align*}
H=\int_0^1\int_0^y (x^2+y^2)\dif x\dif y+\int_1^2\int_0^{2-y}(x^2+y^2)\dif x\dif y
\end{align*}
ہے۔ خطہ \عددی{R}  کا خاکہ بنائیں اور بیلن کے  حجم کو ،  تکمل کی ترتیب الٹ کرتے ہوئے ،  ایک بارہا تکمل کی صورت میں لکھ کر  حل کریں۔
\انتہا{سوال}
%======================
\ابتدا{جواب}
\wf{\unexpanded{
$\int_0^1\int_x^{2-x}(x^2+y^2)\dif y\dif x=\tfrac{4}{3}$
\begin{center}
\begin{tikzpicture}[font=\scriptsize,declare function={f(\x)=2-\x;g(\x)=\x;}]
\pgfmathsetmacro{\a}{0}
\pgfmathsetmacro{\b}{1}
\begin{axis}[clip=false,axis on top,small,axis lines=middle,xlabel={$x$},ylabel={$y$},xtick={\b},xticklabels={$1$},ytick={1,2},yticklabels={$1$,$2$}, ymin=0,enlargelimits=true,xlabel style={at={(current axis.right of origin)},anchor=west},ylabel style={at={(current axis.above origin)},anchor=south}]
\addplot[name path=kf,domain=\a:\b](x,{f(x)})node[pos=0.5,sloped,above]{$y=2-x$};
\addplot[name path=kg,domain=\a:\b](x,{g(x)})node[pos=0.5,sloped,below]{$y=x$};
\addplot[lgray]fill between[of={kf and kg}];
\end{axis}
\end{tikzpicture}
\end{center}
}}
\انتہا{جواب}
%====================
\ابتدا{سوال}
درج  ذیل کی قیمت تلاش کریں۔ (اشارہ: متکمل کو ایک تکمل کی صورت میں لکھیں۔)
\begin{align*}
\int_0^2 (\tan^{-1}\pi x-\tan^{-1}x)\dif x
\end{align*}
\انتہا{سوال}
%================
\ابتدا{سوال}
مستوی \عددی{xy} میں کونسا خطہ \عددی{R} درج ذیل تکمل کی قیمت کو زیادہ سے زیادہ بناتا ہے؟
\begin{align*}
\iint\limits_R (4-x^2-2y^2)\dif S
\end{align*}
اپنے جواب کی وجہ پیش کریں۔
\انتہا{سوال}
%=============
\ابتدا{سوال}
مستوی \عددی{xy} میں کونسا خطہ \عددی{R} درج ذیل تکمل کی قیمت کو کم سے کم  بناتا ہے؟
\begin{align*}
\iint\limits_R (x^2+y^2-9)\dif S
\end{align*}
اپنے جواب کی وجہ پیش کریں۔
\انتہا{سوال}
%=============
\ابتدا{سوال}
کیا استمراری تفاعل \عددی{f(x,y)} کا مستوی \عددی{xy} میں مستطیل خطہ پر  تکمل کی ترتیب بدلتے ہوئے مختلف نتائج کا حصول ٹھیک ہو گا؟ اپنے جواب کی وجہ بنائیں۔
\انتہا{سوال}
%================
\ابتدا{سوال}
ایک مثلث جس کے راس  \عددی{(0,1)}، \عددی{(2,0)} اور \عددی{(1,2)} ہوں پر استمراری تفاعل \عددی{f(x,y)} کے  دوہرا  تکمل  کی قیمت درکار ہے۔ آپ یہ قیمت کیسے حاصل کریں گے؟ اپنے جواب کی وجہ پیش کریں۔
\انتہا{سوال}
%================
\ابتدا{سوال}
درج ذیل تعلق کو ثابت کریں۔
\begin{align*}
\int_{-\infty}^{\infty}\int_{-\infty}^{\infty} e^{-x^2-y^2}\dif x\dif y=\lim_{b\to\infty}\int_{-b}^b\int_{-b}^b e^{-x^2-y^2}\dif x\dif y=4\left(\int_0^{\infty}e^{-x^2}\dif x\right)^2
\end{align*}
\انتہا{سوال}
%=============
\ابتدا{سوال}
درج ذیل غیر مناسب تکمل کی قیمت تلاش کریں۔
\begin{align*}
\int_0^1\int_0^3\frac{x^2}{(y-1)^{2/3}}\dif y\dif x
\end{align*}
\انتہا{سوال}
%===============
\موٹا{اعدادی تراکیب سے تکمل کی قیمت کی تلاش}\\
سوال \حوالہ{سوال_بالکثرت_اعدادی_تراکیب_الف} تا سوال \حوالہ{سوال_بالکثرت_اعدادی_تراکیب_ب} میں کمپیوٹر استعمال کرتے ہوئے اعدادی تراکیب سے دوہرا تکملات کی قیمتیں دریافت کریں۔

\ابتدا{سوال}\شناخت{سوال_بالکثرت_اعدادی_تراکیب_الف}
$\int_1^3\int_1^x\frac{1}{xy}\dif y\dif x$
\انتہا{سوال}
%===============
\ابتدا{جواب}
\wf{\unexpanded{
$0.603$
}}
\انتہا{جواب}
%===================
\ابتدا{سوال}
$\int_0^1\int_0^1e^{-x^2-y^2}\dif y\dif x$
\انتہا{سوال}
%==============
\ابتدا{سوال}
$\int_0^1\int_0^1\tan^{-1}xy\dif y\dif x$
\انتہا{سوال}
%==================
\ابتدا{جواب}
\wf{\unexpanded{
$0.233$
}}
\انتہا{جواب}
%===================
\ابتدا{سوال}\شناخت{سوال_بالکثرت_اعدادی_تراکیب_ب}
$\int_{-1}^1\int_0^{\sqrt{1-x^2}}3\sqrt{1-x^2-y^2}\dif y\dif x$
\انتہا{سوال}
%===============
\انتہا{سوالات}


\حصہ{رقبات،  معیار اثر، اور  مراکز کمیت}\شناخت{حصہ_بالکثرت_رقبات_معیار_اثر_مرکز_کمیت}
اس حصہ میں دوہرا تکملات استعمال کرتے ہوئے  مستوی میں محدود  خطوں کے رقبات اور ان خطوں  پر باریک چادروں کی کمیت، معیار اثر، مرکز کمیت، اور  \اصطلاح{حرکت دواری}\فرہنگ{حرکت!دواری}\فرہنگ{دوار}\حاشیہب{gyration}\فرہنگ{gyration} کے رداس معلوم کرنا دکھایا جائے گا۔ ان کا حساب باب \حوالہ{باب_تکمل_کا_استعمال}  کے حساب کی طرح ہو گا لیکن اب ہم زیادہ قسم کے اشکال کے لئے حساب کر پائیں گے۔

\جزوحصہء{مستوی میں محدود خطوں کے رقبات} 
گزشتہ حصہ میں خطہ \عددی{R} پر   دوہرا تکمل کی تعریف میں \عددی{f(x,y)=1} لینے سے جزوی مجموعات  کی تخفیف شدہ صورت
\begin{align}
J_n=\sum_{k=1}^{n}f(x_k,y_k)\Delta S_k=\sum_{k=1}^{n}\Delta S_k
\end{align}
حاصل ہو گی۔یہ  تخمینی طور پر \عددی{R} کا رقبہ ہو گا۔ جوں جوں  شکل \حوالہ{شکل_بالکثرت_خانہ_بندی_پہلے} میں \عددی{\Delta x} اور \عددی{\Delta y} صفر کے قریب تر ہوتے جاتے ہیں توں توں \عددی{R} کے زیادہ سے زیادہ حصہ کو  تمام  \عددی{\Delta S_k} مل کر  کو ڈھانپتے ہیں، اور ہم \عددی{R} کی رقبہ کی تعریف درج ذیل لیتے ہیں۔
\begin{align}
\text{رقبہ}=\lim_{n\to\infty}\sum_{k=1}^n \Delta S_k=\iint\limits_R\dif S
\end{align}

\ابتدا{تعریف}
بند محدود خطہ \عددی{R} کا رقبہ درج ذیل ہو گا۔
\begin{align}\label{مساوات_بالکثرت_رقبہ_تعریف}
S=\iint\limits_R \dif S
\end{align}
\انتہا{تعریف}
%============

اس باب  کے دیگر تعریفات کی  طرح،     رقبے کی یک متغیری  تعریف   کے  لحاظ سے، جو ہم پہلے پیش کر چکے ہیں،    موجودہ تعریف زیادہ  اقسام کے خطوں پر قابل اطلاق  ہو  گی، لیکن،  جن خطوں پر دونوں تعریفات قابل اطلاق ہوں، وہاں موجودہ تعریف گزشتہ تعریف کے عین موافق  ہو گی۔

مساوات \حوالہ{مساوات_بالکثرت_رقبہ_تعریف} میں دی گئی تکمل کی قیمت کے حصول میں  ہم \عددی{R} پر \عددی{f(x,y)=1} لیتے ہیں۔

\ابتدا{مثال}\شناخت{مثال_بالکثرت_رقبہ_بیچ_دو_تفاعل}
ربع اول میں \عددی{y=x} اور \عددی{y=x^2} کے بیچ محیط   رقبہ تلاش کریں۔

حل:\quad
ہم اس خطہ کا خاکہ (شکل \حوالہ{شکل_مثال_بالکثرت_رقبہ_بیچ_دو_تفاعل})  بنا کر رقبہ تلاش کرتے ہیں۔
\begin{align*}
S=\int_0^1\int_{x^2}^x\dif y\dif x=\int_0^1\big[y\big]_{x^2}^x\dif x=\int_0^1 (x-x^2)\dif x=\big[\frac{x^2}{2}-\frac{x^3}{3}\big]_0^1=\frac{1}{6}
\end{align*}
\انتہا{مثال}
%==================
\begin{figure}
\centering
\begin{minipage}{0.45\textwidth}
\centering
\begin{tikzpicture}[font=\small]
\pgfmathsetmacro{\a}{3.75}
\pgfmathsetmacro{\b}{2.5}
\draw[-latex](0,0)--++(4.5,0)node[right]{$x$};
\draw[-latex](0,0)--++(0,3.25)node[right]{$y$};
\draw[fill=lgray](0.4+\a*0.07,0.25+\b*0.97)
\foreach \x/\y in{0.07/0.97,0.8/0.97,0.8/0.8,0.9/0.8,0.9/0.7,0.97/0.7,0.97/0.2,0.8/0.2,0.8/0.1,0.6/0.1,
0.6/0.02,0.2/0.02,0.2/0.1,0.07/0.1,0.07/0.3,0/0.3,0/0.8,0.07/0.8,0.07/0.97} {--(0.4+\a*\x,0.25+\b*\y)};
\draw(0.4,0)node[below]{$a$}--++(0,0.1)  (0.4,0)++(\a,0)node[below]{$b$}--++(0,0.1); 
\draw(0,0.25)node[left]{$c$}--++(0.1,0)  (0,0.25)++(0,\b)node[left]{$d$}--++(0.1,0); 
\foreach \x in {0,0.07,0.125,0.2,0.3,0.45,0.6,0.8,0.9,0.97} {\draw(0.4+\a*\x,0.125)--++(0,\b+0.25);}
\foreach \y in {0.02,0.1,0.2,0.3,0.5,0.7,0.8,0.97}{\draw(0.25,0.25+\b*\y)--++(\a+0.25,0);}
\draw[fill=gray](0.4+0.45*\a,0.25+0.3*\b) rectangle (0.4+0.6*\a,0.25+0.5*\b);
\draw[stealth-stealth](0.4+0.45*\a,-0.1)--(0.4+0.6*\a,-0.1)node[pos=0.5,below]{$\Delta x_k$};
\draw[stealth-stealth](-0.1,0.25+0.3*\b)--(-0.1,0.25+0.5*\b)node[pos=0.5,left]{$\Delta y_k$};
\draw(0.4+0.475*\a,0.25+0.4*\b)--++(-0.3,0.3)node[above]{$\Delta S_k$};
\draw(0.4+0.55*\a,0.25+0.45*\b)node[circ]{}--++(0.3,0.2)node[above right]{$(x_k,y_k)$};
\draw[thick](0.3,1/2*\b) to [out=-90,in=170](1/3*\a,0.25) to [out=-10,in=-170](2/3*\a,0.265) to [out=10,in=-90] (\a+0.45,1/2*\b) to [out=90,in=-10](\a-0.5,\b+0.25) to [out=170,in=5](1/4*\a,\b+0.25) to [out=-175,in=90](0.3,1/2*\b);
\draw(0.4+\a*0.54,0.25+\b*0.87)node[]{$R$};
\end{tikzpicture}
\caption{
ایک خطہ کے رقبے کی تلاش میں پہلا قدم خطے کی اندرون کی خانہ بندی ہے۔
}
\label{شکل_بالکثرت_خانہ_بندی_پہلے}
\end{minipage}\hfill
\begin{minipage}{0.45\textwidth}
\centering
\begin{tikzpicture}[font=\small,declare function={f(\x)=\x;g(\x)=\x^2;}]
\pgfmathsetmacro{\a}{0}
\pgfmathsetmacro{\b}{1.1}
\pgfmathsetmacro{\c}{0.5}
\begin{axis}[clip=false,small,axis lines=middle,enlargelimits=true, xlabel={$x$}, ylabel={$y$}, zlabel={$z$}, xtick={1},ytick={1},ztick={\empty},xlabel style={anchor=north}, ylabel style={anchor=west},zlabel style={anchor=south}]
\addplot[name path=kf,domain=\a:\b]{f(x)}node[pos=0.8,left]{$y=x$};
\addplot[name path=kg,domain=\a:\b]{g(x)}node[pos=0.65,below right]{$y=x^2$};
\addplot[]coordinates {(1,1)}node[right,yshift=-1ex]{$(1,1)$};
\addplot[]coordinates{(\c,0)(\c,{\c*\c})};
\addplot[-latex]coordinates{(\c,\c)(\c,0.75)};
\addplot[thick]coordinates{(\c,{\c*\c})(\c,\c)}node[pos=0,pin={[right]-10:{$y=x^2$}}]{}node[pos=1,pin={135:{$y=x$}}]{};
\addplot[lgray]fill between[of={kf and kg},soft clip={domain=0:1}];
\end{axis}
\end{tikzpicture}
\caption{قطع مکافی اور لکیر کے بیچ رقبہ (مثال \حوالہ{مثال_بالکثرت_رقبہ_بیچ_دو_تفاعل})۔}
\label{شکل_مثال_بالکثرت_رقبہ_بیچ_دو_تفاعل}
\end{minipage}
\end{figure}

\ابتدا{مثال}\شناخت{مثال_بالکثرت_قطع_مکافی_لکیر_بیچ}
قطع مکافی \عددی{y=x^2} اور لکیر \عددی{y=x+2} کے بیچ محیط رقبہ تلاش کریں۔

حل:\quad
اگر ہم پہلے \عددی{x} کے لحاظ سے تکمل لیں تب ہمیں اس خطہ کو \عددی{R_1} اور \عددی{R_2} میں تقسیم کر کے  درج ذیل دو علیحدہ علیحدہ تکملات کی  ضرورت پیش آئے گی (شکل \حوالہ{شکل_مثال_بالکثرت_قطع_مکافی_لکیر_بیچ}-ا)۔
\begin{align*}
S=\iint\limits_{R_1}\dif S+\iint\limits_{R_2}\dif S=\int_0^1\int_{-\sqrt{y}}^{\sqrt{y}}\dif x\dif y+\int_1^4\int_{y-2}^{\sqrt{y}}\dif x\dif y
\end{align*}
اس کے برعکس تکمل کی ترتیب الٹ کرنے سے صرف   ایک تکمل
\begin{align*}
S=\int_{-1}^2\int_{x^2}^{x+2}\dif y\dif x
\end{align*}
 کی ضرورت پیش آئے گی (شکل \حوالہ{شکل_مثال_بالکثرت_قطع_مکافی_لکیر_بیچ}-ب)۔ہم   اسی سے  رقبہ تلاش کرتے ہیں۔
\begin{align*}
S=\int_{-1}^2\big[y\big]_{x^2}^{x+2}\dif x=\int_{-1}^2(x+2-x^2)\dif x=\big[\frac{x^2}{2}+2x-\frac{x^3}{3}\big]_{-1}^2=\frac{9}{2}
\end{align*}

\انتہا{مثال}
%=================
\begin{figure}
\centering
\begin{subfigure}{0.45\textwidth}
\centering
\begin{tikzpicture}[font=\small,declare function={f(\x)=\x+2;g(\x)=\x^2;}]
\pgfmathsetmacro{\a}{-1}
\pgfmathsetmacro{\b}{2.1}
\pgfmathsetmacro{\c}{0.5}
\begin{axis}[clip=false,small,axis lines=middle,enlargelimits=true, xlabel={$x$}, ylabel={$y$}, zlabel={$z$}, xtick={2},ytick={4},ztick={\empty},xlabel style={anchor=north}, ylabel style={anchor=west},zlabel style={anchor=south}]
\addplot[name path=kf,domain=\a:\b]{f(x)}node[pos=0.85,left,yshift=1ex]{$y=x+2$};
\addplot[name path=kg,domain=\a:\b]{g(x)}node[above]{$y=x^2$}node[pos=0,left]{$(-1,1)$};
\addplot[]coordinates {(2,4)}node[right,yshift=-1ex]{$(2,4)$};
\addplot[lgray,opacity=0.5]fill between[of={kf and kg},soft clip={domain=-1:2}];
\addplot[draw=none,name path=kx]coordinates{(-1,1)(1,1)}node[pos=0.25,below]{$R_1$}node[pos=0.75,above]{$R_2$};
\addplot[gray,opacity=0.5]fill between[of={kx and kg},soft clip={domain=-1:1}];
\addplot[-latex]coordinates {(-1,0.4)(1,0.4)};
\addplot[-latex]coordinates {(0.25,3)(2,3)};
\addplot[]coordinates{(0.7,0.8)}node[pin={[right]10:{$\int_0^1\int_{-\sqrt{y}}^{\sqrt{y}}\dif x\dif y$}}]{};
\addplot[]coordinates{(1.25,2.2)}node[pin={[right]-10:{$\int_1^4\int_{y-2}^{\sqrt{y}}\dif x\dif y$}}]{};
\end{axis}
\end{tikzpicture}
\caption{}
\end{subfigure}\hfill
\begin{subfigure}{0.45\textwidth}
\centering
\begin{tikzpicture}[font=\small,declare function={f(\x)=\x+2;g(\x)=\x^2;}]
\pgfmathsetmacro{\a}{-1}
\pgfmathsetmacro{\b}{2.1}
\pgfmathsetmacro{\c}{0.5}
\begin{axis}[clip=false,small,axis lines=middle,enlargelimits=true, xlabel={$x$}, ylabel={$y$}, zlabel={$z$}, xtick={2},ytick={4},ztick={\empty},xlabel style={anchor=north}, ylabel style={anchor=west},zlabel style={anchor=south}]
\addplot[name path=kf,domain=\a:\b]{f(x)}node[pos=0.8,left,yshift=1ex]{$y=x+2$};
\addplot[name path=kg,domain=\a:\b]{g(x)}node[above]{$y=x^2$}node[pos=0,left]{$(-1,1)$};
\addplot[]coordinates {(2,4)}node[right,yshift=-1ex]{$(2,4)$};
\addplot[lgray,opacity=0.5]fill between[of={kf and kg},soft clip={domain=-1:2}];
\addplot[-latex]coordinates {(0.5,-0.25)(0.5,3)};
\addplot[]coordinates{(1.25,2.2)}node[pin={[right]-10:{$\int_{-1}^{2}\int_{x^2}^{x+2}\dif y\dif x$}}]{};
\end{axis}
\end{tikzpicture}
\caption{}
\end{subfigure}
\caption{(ا) اگر ہم پہلے \عددی{x} کے لحاظ سے تکمل لیں تب رقبے کے حصول کے لئے دو  تکملات کا مجموعہ درکار ہو گا۔ (ب) البتہ پہلے \عددی{y} کے لحاظ سے تکمل لیتے ہوئے  صرف ایک تکمل سے حاصل ہو گا۔}
\label{شکل_مثال_بالکثرت_قطع_مکافی_لکیر_بیچ}
\end{figure}

\جزوحصہء{اوسط قیمت}
بند وقفہ پر  قابل تکمل واحد متغیر تفاعل   کی اوسط قیمت اس وقفہ پر تفاعل کا تکمل تقسیم  لمبائی وقفہ ہو گی۔ بند اور محدود خطہ پر، جس کا رقبہ قابل ناپ ہو،    معین قابل تکمل دو متغیر تفاعل کی اوسط قیمت اس خطہ پر تفاعل کا تکمل تقسیم خطہ کا رقبہ ہو گی۔ اگر خطہ \عددی{R} اور تفاعل \عددی{f} ہوں تب درج ذیل  ہو گا۔
\begin{align}
\text{\RL{\عددی{R} پر \عددی{f} کی\موٹا{ اوسط قیمت}}}=\frac{1}{\text{\RL{\عددی{R} کا رقبہ}}}\iint\limits_{R}f\dif S
\end{align}
اگر  خطہ  \عددی{R}  پر باریک (پتلی) چادر کی  کثافت رقبہ  \عددی{f} ہو تب \عددی{R} پر \عددی{f} کے دوہرا تکمل کو \عددی{R} کے رقبہ سے تقسیم کرنے سے اس چادر کی اوسط کثافت حاصل ہو گی جس کی اکائی  کمیت فی اکائی رقبہ  ہو گی۔ اگر نقطہ \عددی{(x,y)} سے مقررہ نقطہ \عددی{N} تک فاصلہ \عددی{f(x,y)} ہو تب \عددی{R} پر \عددی{f} کی اوسط قیمت، \عددی{N} سے \عددی{R} کے نقاط کا اوسط فاصلہ ہو گا۔ 

\ابتدا{مثال}
مستطیل \عددی{R:\,0\le x\le \pi,\,0\le y\le 1} پر \عددی{f(x,y)=x\cos xy} کی اوسط قیمت تلاش کریں۔

حل:\quad
خطہ \عددی{R} پر \عددی{f} کا تکمل 
\begin{align*}
\int_0^{\pi}\int_0^1 x\cos xy\dif x\dif y&=\int_0^{\pi}\big[\sin xy\big]_{y=0}^{y=1}\dif x\\
&=\int_0^{\pi}(\sin x-0)\dif x=-\cos x\big]_{0}^{\pi}1+1=2
\end{align*}
ہو گا جبکہ مستطیل \عددی{R}  کا رقبہ \عددی{\pi} ہے۔یوں \عددی{R} پر \عددی{f} کی اوسط قیمت \عددی{\tfrac{2}{\pi}} ہو گی۔
\انتہا{مثال}
%===========================

\جزوحصہء{مراکز کمیت کے  معیار اثر اول اور دوم}
باریک چادروں کی کمیت اور معیار اثر تلاش کرنے کے لئے ہم  باب \حوالہ{باب_تکمل_کا_استعمال} کے کلیات کی طرح کلیات استعمال کرتے ہیں۔ فرق صرف اتنا ہے کہ دوہرا تکمل کی بنا اب ہم زیادہ  اشکال اور  کثافتی تفاعل کو  عمل میں لا  سکتے ہیں۔   جدول میں ان کلیات درج ذیل ہیں۔

\موٹا{مستوی \عددی{xy} میں باریک چادر کی \اصطلاح{کمیت}\فرہنگ{کمیت} ،   \اصطلاح{معیار اثر اول}\فرہنگ{معیار اثر!اول}\حاشیہب{first moment}\فرہنگ{moment!first}،\اصطلاح{ معیار اثر دوم}\فرہنگ{معیار اثر!دوم}\حاشیہب{second moment}\فرہنگ{moment!second} اور\اصطلاح{ رداس دوار}\فرہنگ{دوار!رداس}\فرہنگ{رداس!دوار}\حاشیہب{radius of gyration}\فرہنگ{gyration!radius}  کے کلیات}\\
\begin{description}
\item{کثافت:}\quad
$\delta(x,y)$
\item{کمیت:}\quad
$M=\iint \delta(x,y)\dif S$
\item{معیار اثر اول:}\quad
$M_x=\iint y\delta(x,y)\dif S,\quad M_y=\iint x\delta(x,y)\dif S$
\item{مرکز کمیت:}\quad
$\bar{x}=\frac{M_y}{M},\quad \bar{y}=\frac{M_x}{M}$
\item{معیار اثر دوم (جمودی معیار اثر):}\quad
\begin{align*}
I_x&=\iint y^2\delta(x,y)\dif S&&\text{\RL{بلحاظ محور \عددی{x}}}\\
I_y&=\int x^2\delta(x,y)\dif S&&\text{\RL{بلحاظ محور \عددی{y}}}\\
I_L&=\iint r^2(x,y)\delta(x,y)\dif S,\quad \text{\small\RL{{(جہاں \عددی{L} سے \عددی{(x,y)} کا فاصلہ \عددی{r(x,y)} ہے)}}}&&\text{\RL{بلحاظ  خط  \عددی{L}}}\\
I_0&=\int (x^2+y^2)\delta(x,y)\dif S=I_x+I_y&&\text{\RL{(قطبی معیار اثر) بلحاظ مبدا}}
\end{align*}
\item{رداس دوار:}\quad
\begin{align*}
R_x&=\sqrt{\frac{I_x}{M}}&&\text{\RL{بلحاظ محور \عددی{x}}}\\
R_y&=\sqrt{\frac{I_y}{M}}&&\text{\RL{بلحاظ محور \عددی{y}}}\\
R_0&=\sqrt{\frac{I_0}{M}}&&\text{\RL{بلحاظ مبدا}}\\
\end{align*}
\end{description}
ان کلیات کا استعمال مثالوں کی مدد سے سمجھایا جائے گا۔

معیار اثر اول \عددی{M_x} اور   \عددی{M_y} اور معیار اثر دوم  (جمودی معیار اثر) \عددی{I_x} اور \عددی{I_y}  میں  ریاضیاتی فرق یہ ہے کہ معیار اثر دور  "بیرم کے بازوؤں" کے فاصلوں،      \عددی{x} اور \عددی{y}،      کا مربع لیتا ہے۔ 

معیار اثر \عددی{I_0} کو \اصطلاح{قطبی معیار اثر}\فرہنگ{معیار اثر!قطبی}\حاشیہب{polar moment}\فرہنگ{moment!polar} بھی کہتے ہیں۔ کمیتی کثافت \عددی{\delta(x,y)}  (کمیت فی اکائی رقبہ)   ضرب \عددی{x^2+y^2}، جو  نمائندہ نقطہ  \عددی{(x,y)}  سے مبدا تک فاصلہ ہے،  کا تکمل قطبی معیار اثر کہلاتا ہے۔ چونکہ \عددی{I_0=I_x+I_y}  ہے لہٰذا ان میں سے کسی دو کے حصول کے بعد تیسرے کو اس تعلق سے اخذ کیا جا سکتا ہے۔ (معیار اثر \عددی{I_0}  بعض اوقات \عددی{I_z} لکھا جاتا  اور بلحاظ محور \عددی{z} معیار اثر کہلاتا ہے۔ تب تماثل \عددی{I_z=I_x+I_y}  \اصطلاح{مسئلہ عمودی محور}\فرہنگ{مسئلہ!عمودی محور}\حاشیہب{Perpendicular Axis Theorem}\فرہنگ{theorem!perpendicular axis} کہلاتا ہے۔)

\اصطلاح{رداس دوار}  \عددی{R_x} کی تعریف درج ذیل مساوات  ہے۔
\begin{align*}
I_x=MR_x^2
\end{align*}
رداس دوار ہمیں بتاتا ہے کہ محور \عددی{x}  کتنا دور   پوری چادر کی کمیت منجمد کرتے ہوئے  وہی \عددی{I_x} حاصل ہو گا۔ رداس دوار استعمال کرتے ہوئے ہم معیار اثر کو کمیت اور لمبائی کی صورت میں لکھ پاتے ہیں۔ رداس \عددی{R_y} اور \عددی{R_0} کی تعریفات بھی اسی طرح ہیں:
\begin{align*}
I_y=MR_y^2,\quad I_0=MR_0^2
\end{align*}
ہم ان تعریفی مساوات کے جذر سے \عددی{R_x}، \عددی{R_y} اور \عددی{R_0} کے    کلیات لکھتے ہیں۔

\begin{figure}
\centering
\begin{tikzpicture}
\pgfmathsetmacro{\b}{1}
\pgfmathsetmacro{\a}{\b/2}
\pgfmathsetmacro{\bb}{3/4*\b}
\pgfmathsetmacro{\aa}{\bb/2}
\pgfmathsetmacro{\h}{4}
\pgfmathsetmacro{\m}{0.2}
\pgfmathsetmacro{\n}{0.125}
\pgfmathsetmacro{\t}{0.4}
\pgfmathsetmacro{\s}{1/2*\t}
\pgfmathsetmacro{\ang}{45}
\draw([shift={(90:\a cm and \b cm)}]0,0) arc (90:270:\a cm and \b cm);
\draw([shift={(0:\a cm and \b cm)}]\h,0) arc (0:360:\a cm and \b cm);
\draw(0,\b)--++(\h,0)  (0,-\b)--++(\h,0);
\draw[-stealth]([shift={(-155:\s cm and \t cm)}]\h+2.2,0) arc (-155:155:\s cm and \t cm);
\draw[line width=0.4cm,white](\h+\a+0.5,0)--++(2.4,0);
\draw(\h,0)--++(2.6,0)node[above,pos=0.5]{\RL{گھومنے کا محور}};
\draw([shift={(0:\aa cm and \bb cm)}]2/3*\h,0) arc (0:360:\aa cm and \bb cm)coordinate[pos=0.1](kS)coordinate[pos=0.32](k);
\draw(k)node[circ]{}node[left,xshift=-1ex]{$\Delta m_k$}++(-1/2*\m,-1/2*\m)--++(\m,0)coordinate(kFR)--++(0,\m)coordinate(kFT)--++(-\m,0)--++(0,-\m);
\draw(kFR)--++(\ang:\n)--++(0,\m)--++(-\m,0)--++(\ang:-\n)  (kFT)--++(\ang:\n);
\draw(k)--(2/3*\h,0)coordinate(kk)coordinate[pos=0.75](kkB)node[shift={(160:0.2)},yshift=0.5ex]{$r_k$};
\draw(kk)--(kS)coordinate[pos=0.75](kkA);
\draw[-stealth](kkA) to [out=135,in=60]node[pos=0.6,above,yshift=-0.5ex]{$\theta$}(kkB); 
\draw[-latex](k)--++(-0.5,-0.5)node[left]{$\kvec{v}_k$};
\draw[dashed](-\a,0)--(\h,0) ;
\draw(-\a,0)--++(-0.5,0);
\end{tikzpicture}
\caption{گھومتے ہوئے دھرے میں ذخیرہ توانائی دریافت کرنے کی خاطر ہم اس کو متعدد چھوٹے کمیتوں میں تقسیم کر کے ہر تمام چھوٹے کمیتوں کی حرکی توانائی کا مجموعہ لیتے ہیں۔}
\label{شکل_بالکثرت_دھرے_کی_حرکی_توانائی}
\end{figure}
ہمیں معیار اثر میں کیا  دلچسپی ہے؟ ایک جسم کا  پہلا معیار ا اثر  ہمیں ثقلی میدان میں اس جسم کے  توازن  اور مختلف محوروں کے لحاظ  سے اس کی  قوت مروڑ کے بارے میں معلومات فراہم کرتا ہے۔  اب اگر یہ جسم  گھومتا ہوا دھرا ہو تب ہمیں اس   میں ذخیرہ توانائی جاننے میں زیادہ دلچسپی ہو گی تا کہ ہم جان سکیں کہ اس کو روکنے کے لئے یا اس کو کسی خاص زاویاتی رفتار تک پہنچانے میں کتنی توانائی درکار ہو گی۔ایسی صورت میں معیار اثر دوم  استعمال ہو گا۔

اس دھرا کو متعدد چھوٹی کمیتوں \عددی{\Delta m_k} میں  تقسیم کریں  اور  گھومنے کے محور سے \عددی{k} ویں کمیتی ٹکڑے کے  فاصلہ کو  \عددی{r_k} سے ظاہر کریں (شکل \حوالہ{شکل_بالکثرت_دھرے_کی_حرکی_توانائی})۔ اگر دھرا کی زاویاتی سمتی  رفتار \عددی{\omega=\tfrac{\dif\theta}{\dif t}}  ریڈیئن فی سیکنڈ ہو، تب اس ٹکڑے کا کمیتی مرکز اپنے مدار میں  خطی رفتار
\begin{align*}
v_k=\frac{\dif}{\dif t}(r_k\theta)=r_k\frac{\dif \theta}{\dif t}=r_k\omega
\end{align*}
سے  حرکت کرے گا۔اس ٹکڑے کی حرکی توانائی تخمیناً
\begin{align}
\frac{1}{2}\Delta m_kv_k^2=\frac{1}{2}\Delta m_k(r_k\omega)^2=\frac{1}{2}\omega^2r_k^2\Delta m_k
\end{align}
ہو گی۔دھرا کی حرکی توانائی تخمیناً
\begin{align}
\sum \frac{1}{2}\omega^2r_k^2\Delta m_k
\end{align}
ہو گی۔دھرا کو زیادہ سے زیادہ ٹکڑوں میں تقسیم کرنے سے اس مجموعہ کی قیمت ایک حد تک پہنچتی ہے جسے تکمل
\begin{align}\label{مساوات_بالکثرت_حرکی_توانائی_تعریف_الف}
\text{\RL{دھرا کی حرکی توانائی}}=\int\frac{1}{2}\omega^2r^2\dif m=\frac{1}{2}\omega^2\int r^2\dif m
\end{align}
لکھا جا سکتا ہے۔ جزو
\begin{align}
I=\int r^2\dif m
\end{align}
درحقیقت گھومنے کے محور کے لحاظ  سے دھرے کا جمودی  معیار اثر ہے جس کو استعمال کرتے ہوئے مساوات \حوالہ{مساوات_بالکثرت_حرکی_توانائی_تعریف_الف} درج ذیل صورت اختیار کرتی ہے۔
\begin{align}
\text{\RL{دھرا کی حرکی توانائی}}=\frac{1}{2}I\omega^2
\end{align}

ایک دھرا،  جس کا جمودی معیار اثر \عددی{I} ہو،  کو \عددی{\omega} زاویاتی سمتی رفتار  تک پہنچانے  کے لئے \عددی{\tfrac{1}{2}I\omega^2} حرکی  توانائی درکار ہو گی اور اس رفتار پر چلتے ہوئے دھرا کو روکنے کے لئے  ہمیں دھرا سے اتنی ہی  حرکی توانائی   نکالنی ہو گی۔ کمیت \عددی{m} کی گاڑی کو سمتی رفتار \عددی{v} تک پہنچانے کے لئے اس کو \عددی{\tfrac{1}{2}mv^2}  حرکی توانائی درکار ہو گی اور اس کو روکنے کے لئے  اس  گاڑی سے اتنی ہی حرکی  توانائی نکالنی ہو گی۔ دھرے کا جمودی معیار اثر  گاڑی کی کمیت کا مماثل ہے۔ گاڑی کی رفتار تیز یا کم کرنے  کو   گاڑی  کی کمیت مشکل بناتی ہے۔اسی طرح دھرے کی زاویاتی رفتار تیز یا کم کرنے  کو  دھرے کا جمودی معیار اثر مشکل بناتا ہے۔ جمودی معیار اثر کمیت کے علاوہ کمیت کی تقسیم  کا بھی حساب رکھتا ہے۔

بوجھ بردار افقی  دھاتی شہتیر کے  جھکاو کو  بھی جمودی معیار اثر تعین کرتا ہے۔ شہتیر کا اکڑا پن  \عددی{I} ضرب ایک مستقل ہوتا ہے، جہاں  شہتیر کے افقی محور  کے لحاظ سے عمودی تراش کا قطبی معیار اثر \عددی{I} ہے۔ جمودی معیار اثر \عددی{I} کی قیمت جتنی زیادہ ہو، شہتیر اتنا زیادہ  اکڑ ہو گا اور اتنا کم جھکے گا۔  یہی وجہ ہے کہ ہم شکل \حوالہ{شکل_بالکثرت_اکڑ_شہتیر}-ا  میں دکھایا گیا   شہتیر استعمال کرتے ہیں نا کہ ایسے شہتیر جن کا عمودی تراش  چکور ہو (شکل \حوالہ{شکل_بالکثرت_اکڑ_شہتیر}-ب)۔ شہتیر کے بالائی اور زیریں کگر زیادہ تر کمیت کو افقی محور سے دور رکھتے ہوئے \عددی{I} کی قیمت کو زیادہ سے زیادہ بناتے ہیں۔

 جمودی معیار اثر  کو سمجھنے کے لئے ایک تجربہ کریں۔ ایک قلم کے دونوں سروں کے ساتھ  سکے    چپکا   کر  قلم کو انگلیوں میں  تیزی سے آگے پیچھے گھمائیں۔  گھومنے کا رخ تبدیل کرتے وقت آپ کو جو مزاحمت محسوس ہوتی ہے وہ جمودی معیار اثر کی بنا ہے۔ اب ان سکوں کو  قلم کے  سروں سے دور اور آپس میں  قریب کریں۔ قلم اور سکوں کی کمیت تبدیل نہیں ہوئی ہے البتہ  اس نظام کا جمودی معیار اثر کم ہو ہے۔ اب   آپ دیکھیں گے کہ انہیں آگے پیچھے گھمانا زیادہ آسان ہو گا۔


آپ کہہ سکتے ہیں کہ  معیار اثر اول  کا تعلق توازن سے ہے جبکہ معیار اثر دوم  کا تعلق گھومنے سے ہے۔

\begin{figure}
\centering
\begin{subfigure}{0.45\textwidth}
\centering
\begin{tikzpicture}
\pgfmathsetmacro{\l}{3}
\pgfmathsetmacro{\h}{0.75}
\pgfmathsetmacro{\w}{1}
\pgfmathsetmacro{\t}{0.4}
\pgfmathsetmacro{\ang}{45}
\draw(0,0)coordinate(kS)--(\l,0)--++(\ang:\w)--++(0,\t)--++(-\l,0)--++(\ang:-\w)--++(0,-\t);
\draw(kS)++(0,\t)--++(\l,0)coordinate(kL)--++(\ang:1/2*\w-1/2*\t)coordinate(kBL)++(\ang:\t)coordinate(kBR)--++(\ang:1/2*\w-1/2*\t)  (kL)--++(0,-\t);  
\draw[name path=top,fill=white](kBR)--++(0,\h)coordinate[pos=0.5](kAxis)--++(\ang:1/2*\w-1/2*\t)--++(0,\t)coordinate(kTR)--++(-\l,0)--++(\ang:-\w)--++(0,-\t)--++(\l,0)--++(\ang:1/2*\w-1/2*\t)--(kBL);
\path[name path=leftEdge](kS)++(0,\t)--++(\ang:1/2*\w-1/2*\t)coordinate(BL)--++(0,\h);
\draw[fill=white,name intersections={of={top and leftEdge}}](intersection-1)--(BL)--(kBL)--++(0,\h)--++(\ang:-1/2*\w+1/2*\t)--(intersection-1);
\draw(kTR)--++(\ang:-\w)coordinate(kTF)--++(-\l,0)  (kTF)--++(0,-\t);
\draw(kAxis)++(\ang:-1/2*\t)node[circ]{}--++(1,0)node[pos=0.75,above]{محور};
\end{tikzpicture}
\caption{}
\end{subfigure}\hfill
\begin{subfigure}{0.45\textwidth}
\centering
\begin{tikzpicture}
\pgfmathsetmacro{\l}{3}
\pgfmathsetmacro{\h}{0.75}
\pgfmathsetmacro{\w}{1}
\pgfmathsetmacro{\t}{0.4}
\pgfmathsetmacro{\ang}{45}
\pgfmathsetmacro{\kl}{sqrt(2*\w*\t+\h*\t)}
\draw(0,0)--++(\l,0)--++(\ang:\kl)--++(0,\kl)coordinate(kTR)--++(-\l,0)--++(\ang:-\w)coordinate(kkL)--(0,0);
\draw(kkL)--++(\l,0)coordinate(kkR)--(kTR)  (kkR)--++(0,-\kl)coordinate[pos=0.5](kRM)   (kRM)++(\ang:1/2*\kl)node[circ]{}--++(1,0)node[pos=0.75,above]{محور};
\end{tikzpicture}
\caption{}
\end{subfigure}
\caption{دونوں شہتیر کا رقبہ عمودی تراش ایک جیسا ہے لیکن شہتیر -ا  کا جمودی معیار اثر زیادہ ہے لہٰذا  شہتیر- ا  زیادہ اکڑ ہو گا۔}
\label{شکل_بالکثرت_اکڑ_شہتیر}
\end{figure}
\ابتدا{مثال}\شناخت{مثال_بالکثرت_تکون_رداس_دوار}
محور \عددی{x}، لکیر \عددی{x=1} اور لکیر \عددی{y=2x}  کے بیچ تکونی چادر پائی جاتی ہے۔ نقطہ \عددی{(x,y)} پر اس چادر کی کثافت \عددی{\delta(x,y)=6x+6y+6} ہے۔ اس چادر کی کمیت، پہلا معیار اثر، مرکز کمیت، جمودی معیار اثر اور محددی  محوروں کے لحاظ سے رداس دوار  تلاش کریں۔

حل:\quad
ہم اس خطہ کا خاکہ بنا کر (شکل \حوالہ{شکل_مثال_بالکثرت_تکون_رداس_دوار})   اس پر اتنی معلومات درج کرتے ہیں کہ تکمل کے حد  جان سکیں۔

چادر کی کمیت درج ذیل ہو گی۔
\begin{align*}
M&=\int_0^1\int_0^{2x}\delta(x,y)\dif y\dif x=\int_0^1\int_0^{2x}(6x+6y+6)\dif y\dif x\\
&=\int_0^1\big[6xy+3y^2+6y\big]_{y=0}^{y=2x}\dif x\\
&=\int_0^1(24x^2+12x)\dif x=\big[8x^3+6x^2\big]_0^1=14
\end{align*}
محور \عددی{x} کے لحاظ سے پہلا معیار اثر درج ذیل ہو گا۔
\begin{align*}
M_x&=\int_0^1\int_0^{2x}y\delta(x,y)\dif y\dif x=\int_0^1\int_0^{2x}(6xy+6y^2+6y)\dif y\dif x\\
&=\int_0^1\big[3xy^2+2y^3+3y^2\big]_{y=0}^{y=2x}\dif x=\int_0^1 (28x^3+12x^2)\dif x\\
&=\big[7x^4+4x^3\big]_0^1=11
\end{align*}
اسی طرح محور \عددی{y} کے  لحاظ سے پہلا معیار اثر درج ذیل حاصل ہو گا۔
\begin{align*}
M_y=\int_0^1\int_0^{2x}x\delta(x,y)\dif y\dif x=10
\end{align*}
مرکز کمیت کے محدد درج ذیل ہوں گے۔
\begin{align*}
\bar{x}=\frac{M_y}{M}=\frac{10}{14}=\frac{5}{7},\quad \bar{y}=\frac{M_x}{M}=\frac{11}{14}
\end{align*} 
محور \عددی{x} کے لحاظ  سے جمودی معیار اثر درج ذیل ہو گا۔
\begin{align*}
I_x&=\int_0^1\int_0^{2x}y^2\delta(x,y)\dif y\dif x=\int_0^1\int_0^{2x}(6xy^2+6y^3+6y^2)\dif y\dif x\\
&=\int_0^1\big[2xy^3+\frac{3}{2}y^4+2y^3\big]_{y=0}^{y=2x}\dif x=\int_0^1(40x^4+16x^3)\dif x\\
&=\big[8x^5+4x^4\big]_0^1=12
\end{align*}
اسی طرح  محور \عددی{y} کے لحاظ سے جمودی معیار اثر درج ذیل حاصل ہو گا۔
\begin{align*}
I_y=\int_0^1\int_0^{2x}x^2\delta(x,y)\dif y\dif x=\frac{39}{5}
\end{align*}
ہم \عددی{I_x} اور \عددی{I_y} کی قیمتوں سے \عددی{I_0} کی قیمت کلیہ \عددی{I_0=I_x+I_y} کی مدد  سے  حاصل کرتے ہیں۔
\begin{align*}
I_0=12+\frac{39}{5}=\frac{60+39}{5}=\frac{99}{5}
\end{align*}
تین رداس دوار درج ذیل ہوں گے۔
\begin{align*}
R_x&=\sqrt{\frac{I_x}{M}}=\sqrt{\frac{12}{14}}=\sqrt{\frac{6}{7}}\\
R_y&=\sqrt{\frac{I_y}{M}}=\sqrt{\big(\frac{39}{5}\big)/14}=\sqrt{\frac{39}{70}}\\
R_0&=\sqrt{\frac{I_0}{M}}=\sqrt{\big(\frac{99}{5}\big)/14}=\sqrt{\frac{99}{70}}
\end{align*}
\انتہا{مثال}
%================
\begin{figure}
\centering
\begin{minipage}{0.45\textwidth}
\centering
\begin{tikzpicture}[font=\small]
\begin{axis}[axis on top,clip=false,small,axis lines=center,view/h=130,colormap={}{gray(0cm)=(0.6);gray(1cm)=(0.9);},enlargelimits=true,xlabel={$x$},ylabel={$y$},zlabel={$z$},xtick={1},ytick={2},ztick={\empty},xlabel style={anchor=west},ylabel style={anchor=south}]
\addplot[fill=lgray]coordinates{(0,0)(1,0)(1,2)(0,0)};
\addplot[]coordinates{(1,1)}node[right]{$x=1$};
\addplot[]coordinates{(1,2)}node[above]{$(1,2)$};
\addplot[]coordinates{(0.5,1)}node[above left]{$y=2x$};
\end{axis}
\end{tikzpicture}
\caption{خطہ برائے مثال \حوالہ{مثال_بالکثرت_تکون_رداس_دوار}}
\label{شکل_مثال_بالکثرت_تکون_رداس_دوار}
\end{minipage}\hfill
\begin{minipage}{0.45\textwidth}
\centering
\begin{tikzpicture}[font=\small,declare function={f(\x)=x;g(\x)=\x^2;}]
\begin{axis}[axis on top,clip=false,small,axis lines=center,view/h=130,colormap={}{gray(0cm)=(0.6);gray(1cm)=(0.9);},enlargelimits=true,xlabel={$x$},ylabel={$y$},zlabel={$z$},xtick={1},ytick={1},ztick={\empty},xlabel style={anchor=west},ylabel style={anchor=south}]
\addplot[name path=kf,domain=0:1.1]{f(x)}node[pos=0.5,above left]{$y=x$};
\addplot[name path=kg,domain=0:1.1]{g(x)}node[pos=0.5,below right]{$y=x^2$};
\addplot[lgray]fill between[of={kf and kg},soft clip={domain=0:1}];
\addplot[]coordinates{(1,1)}node[below right]{$(1,1)$};
\end{axis}
\end{tikzpicture}
\caption{خطہ برائے مثال \حوالہ{مثال_بالکثرت_مربع_خطی_بیچ}}
\label{شکل_مثال_بالکثرت_مربع_خطی_بیچ}
\end{minipage}
\end{figure}

\جزوحصہء{جیومیٹریائی اشکال کے وسطانی مراکز}
مستقل کثافت کی صورت میں \عددی{\bar{x}} اور \عددی{\bar{y}} کے کلیات میں  شمارکنندہ اور نسب نما   میں موجود کثافت ایک دوسرے کو منسوخ  کرتے ہیں۔    \عددی{\bar{x}} اور \عددی{\bar{y}}  کے نقطہ نظر سے    \عددی{\delta} کی قیمت \عددی{1} ہو سکتی ہے۔ یوں مستقل \عددی{\delta} کی صورت میں مرکز کمیت کا دارومدار جسم کی شکل و صورت پر منحصر ہو گا نا کہ جسم کے مادہ پر۔ایسی صورت میں    مرکز کمیت  عموماً   شکل کا    \اصطلاح{وسطانی مرکز}\فرہنگ{مرکز!وسطانی}\حاشیہب{centroid}\فرہنگ{centroid}  پکارا جاتا ہے۔ وسطانی مرکز کی تلاش میں ہم \عددی{\delta=1} لے کر، پہلے کی طرح،  معیار اثر اول کو کمیت سے تقسیم کرتے ہوئے  \عددی{\bar{x}} اور \عددی{\bar{y}} دریافت کرتے ہیں۔

\ابتدا{مثال}\شناخت{مثال_بالکثرت_مربع_خطی_بیچ}
ربع اول میں اوپر سے لکیر \عددی{y=x}  اور نیچے سے قطع مکافی \عددی{y=x^2}  ایک خطہ کو محدود کرتے ہیں۔ اس خطہ کا وسطانی مرکز تلاش کریں۔

حل:\quad
ہم خطے کا خاکہ بنا کر تکمل کے حد جانتے ہیں (شکل \حوالہ{شکل_مثال_بالکثرت_مربع_خطی_بیچ})۔ اس کے بعد \عددی{\delta=1} لے کر آگے بڑھتے ہیں۔
\begin{align*}
M&=\int_0^1\int_{x^2}^x1\dif y\dif x=\int_0^1\big[y\big]_{y=x^2}^{y=x}\dif x=\int_0^1(x-x^2)\dif x=\big[\frac{x^2}{2}-\frac{x^3}{3}\big]_0^1=\frac{1}{6}\\
M_x&=\int_0^1\int_{x^2}^xy\dif y\dif x=\int_0^1\big[\frac{y^2}{2}\big]_{y=x^2}^{y=x}\dif x\\
&=\int_0^1\big(\frac{x^2}{2}-\frac{x^4}{2}\big)\dif x=\big[\frac{x^3}{6}-\frac{x^5}{10}\big]_0^1=\frac{1}{15}\\
M_y&=\int_0^1\int_{x^2}^xx\dif y\dif x=\int_0^1\big[xy\big]_{y=x^2}^{y=x}\dif x=\int_0^1(x^2-x^3)\dif x=\big[\frac{x^3}{3}-\frac{x^4}{4}\big]_0^1=\frac{1}{12}
\end{align*}
ان قیمتوں کو استعمال کرتے ہوئے ہم وسطانی مرکز کے محدد دریافت کرتے ہیں۔
\begin{align*}
\bar{x}=\frac{M_y}{M}=\frac{1/12}{1/6}=2,\quad \bar{y}=\frac{M_x}{M}=\frac{1/15}{1/6}=\frac{2}{5}
\end{align*}
نقطہ \عددی{(\tfrac{1}{2},\tfrac{2}{5})} اس خطے کا وسطانی مرکز ہو گا۔
\انتہا{مثال}
%==================

\جزوحصہء{سوالات}
\ابتدا{سوالات}
\موٹا{رقبہ بذریعہ دوہرا تکمل}\\
سوال \حوالہ{سوال_بالکثرت_رقبہ_بطور_دوہرا_تکمل_الف} تا سوال \حوالہ{سوال_بالکثرت_رقبہ_بطور_دوہرا_تکمل_ب} میں  منحنیات اور لکیروں کے بیچ خطے کا خاکہ بنا کر اس خطے کے رقبہ کو بطور دوہرا بارہا  تکمل لکھیں۔ اس تکمل کی قیمت دریافت کریں۔

\ابتدا{سوال}\شناخت{سوال_بالکثرت_رقبہ_بطور_دوہرا_تکمل_الف}
محددی محور اور لکیر \عددی{x+y=2}
\انتہا{سوال}
%==================
\ابتدا{جواب}
\wf{\unexpanded{
$\int_0^2\int_0^{2-x}\dif y\dif x=2,\,\, \int_0^2\int_0^{2-y}\dif x\dif y=2$
\begin{center}
\begin{tikzpicture}[font=\small]
\draw[-latex](0,0)--(3,0)node[right]{$x$};
\draw[-latex](0,0)--(0,2.5)node[left]{$y$};
\draw[fill=lgray](0,2)node[left]{$2$}--(2,0)node[pos=0.5,above right,font=\scriptsize]{$y=2-x$}node[below]{$2$}--(0,0)--(0,2);
\end{tikzpicture}
\end{center}
}}
\انتہا{جواب}
%====================
\ابتدا{سوال}
لکیر \عددی{x=0}، \عددی{y=2x} اور \عددی{y=4}
\انتہا{سوال}
%====================
\ابتدا{سوال}
قطع  مکافی \عددی{x=-y^2} اور لکیر \عددی{y=x+2}
\انتہا{سوال}
%====================
\ابتدا{جواب}
\wf{\unexpanded{
$\int_{-2}^1\int_{y-2}^{-y^2}\dif x\dif y=\frac{9}{2}$
\begin{center}
\begin{tikzpicture}[font=\small,declare function={f(\y)=-\y^2;}]
\begin{axis}[axis on top,clip=false,small,axis lines=center,view/h=130,colormap={}{gray(0cm)=(0.6);gray(1cm)=(0.9);},enlargelimits=true,xlabel={$x$},ylabel={$y$},zlabel={$z$},xtick={-2,-4},ytick={-2,1},ztick={\empty},xlabel style={anchor=west},ylabel style={anchor=south}]
\addplot[name path=kf,domain=-2:1]({f(x)},x)node[pos=0.25,below right]{$x=-y^2$};
\addplot[name path=kg]coordinates{(-1,1)(-4,-2)}node[pos=0,above]{$(-1,1)$}node[below]{$(-4,-2)$}node[pos=0.75,above left]{$y=x+2$};
\addplot[lgray]fill between[of={kf and kg}];
\end{axis}
\end{tikzpicture}
\end{center}
}}
\انتہا{جواب}
%====================
\ابتدا{سوال}
قطع مکافی \عددی{x=y-y^2} اور لکیر \عددی{y=-x}
\انتہا{سوال}
%====================
\ابتدا{سوال}
منحنی \عددی{y=e^x} اور لکیر \عددی{y=0}، \عددی{x=0} اور \عددی{x=\ln 2}
\انتہا{سوال}
%====================
\ابتدا{جواب}
\wf{\unexpanded{
$\int_0^{\ln 2}\int_0^{e^x}\dif y\dif x=1$
\begin{center}
\begin{tikzpicture}[font=\small,declare function={f(\x)=e^(\x);}]
\pgfmathsetmacro{\k}{ln(2)}
\begin{axis}[axis on top,clip=false,small,axis lines=center,view/h=130,colormap={}{gray(0cm)=(0.6);gray(1cm)=(0.9);},enlargelimits=true,xlabel={$x$},ylabel={$y$},zlabel={$z$},xtick={\k},xticklabels={$\ln 2$},ytick={1,2},ztick={\empty},xlabel style={anchor=west},ylabel style={anchor=south},ymin=0]
\addplot[name path=kf,domain=0:\k](x,{f(x)})node[pos=0.5,above left]{$y=e^x$}node[above]{$(\ln 2,2)$};
\addplot[name path=kg]coordinates{(\k,2)(\k,0)};
\addplot[draw=none,name path=kx]coordinates{(0,0)(\k,0)};
\addplot[lgray]fill between[of={kf and kx}];
\end{axis}
\end{tikzpicture}
\end{center}
}}
\انتہا{جواب}
%====================
\ابتدا{سوال}
ربع اول میں منحنیات \عددی{y=\ln x}، \عددی{y=2\ln x} اور لکیر \عددی{x=e}
\انتہا{سوال}
%====================
\ابتدا{سوال}
قطع مکافی \عددی{x=y^2} اور \عددی{x=2y-y^2}
\انتہا{سوال}
%====================
\ابتدا{جواب}
\wf{\unexpanded{
$\int_0^1\int_{y^2}^{2y-y^2}\dif x\dif y=\frac{1}{3}$
\begin{center}
\begin{tikzpicture}[font=\small,declare function={f(\y)=\y^2;g(\y)=2*\y-\y^2;}]
\pgfmathsetmacro{\k}{ln(2)}
\begin{axis}[axis on top,clip=false,small,axis lines=center,view/h=130,colormap={}{gray(0cm)=(0.6);gray(1cm)=(0.9);},enlargelimits=true,xlabel={$x$},ylabel={$y$},zlabel={$z$},xtick={1},ytick={1},ztick={\empty},xlabel style={anchor=west},ylabel style={anchor=south},ymin=0]
\addplot[name path=kf,domain=0:1]({f(x)},x)node[pos=0.5,above left]{$x=y^2$}node[above]{$(1,1)$};
\addplot[name path=kg,domain=0:1](g(x),x)node[pos=0.45,below right]{$x=2y-y^2$};
\addplot[lgray]fill between[of={kf and kg}];
\end{axis}
\end{tikzpicture}
\end{center}
}}
\انتہا{جواب}
%====================
\ابتدا{سوال}\شناخت{سوال_بالکثرت_رقبہ_بطور_دوہرا_تکمل_ب}
قطع مکافی \عددی{x=y^2-1} اور \عددی{x=2y^2-2}
\انتہا{سوال}
%====================

سوال \حوالہ{سوال_بالکثرت_رقبہ_سے_خطہ_الف} تا سوال \حوالہ{سوال_بالکثرت_رقبہ_سے_خطہ_ب} میں مستوی \عددی{xy} میں   خطوں کے رقبات کو تکمل یا تکملات کے مجموعوں کی کی صورت میں پیش کیا گیا ہے۔  ان خطوں کا خاکہ بنا کر    سرحدی منحنیات  پر ان  کی مساواتیں لکھیں اور ان نقطوں کی نشاندہی کریں جہاں منحنیات ایک دوسرے کو قطع کرتی ہیں۔ اس کے بعد ان خطہ کا رقبہ تلاش کریں۔ 

\ابتدا{سوال}\شناخت{سوال_بالکثرت_رقبہ_سے_خطہ_الف}
$\int_0^6\int_{y^2/3}^{2y}\dif x\dif y$
\انتہا{سوال}
%===================
\ابتدا{جواب}
\wf{\unexpanded{
$12$
\begin{center}
\begin{tikzpicture}[font=\small,declare function={f(\x)=sqrt(3*\x);g(\x)=1/2*\x;}]
\pgfmathsetmacro{\k}{ln(2)}
\begin{axis}[axis on top,clip=false,small,axis lines=center,view/h=130,colormap={}{gray(0cm)=(0.6);gray(1cm)=(0.9);},enlargelimits=true,xlabel={$x$},ylabel={$y$},zlabel={$z$},xtick={12},ytick={6},ztick={\empty},xlabel style={anchor=west},ylabel style={anchor=south},ymin=0]
\addplot[name path=kfa,domain=0:1](x,{f(x)});
\addplot[name path=kf,domain=1:12](x,{f(x)})node[pos=0.5,above left]{$y^2=3x$}node[above]{$(12,6)$};
\addplot[name path=kg,domain=0:12](x,{g(x)})node[pos=0.45,below right]{$y=\tfrac{x}{2}$};
\addplot[lgray]fill between[of={kfa and kg},soft clip={domain=0:1}];
\addplot[lgray]fill between[of={kf and kg},soft clip={domain=1:12}];
\end{axis}
\end{tikzpicture}
\end{center}
}}
\انتہا{جواب}
%====================
\ابتدا{سوال}
$\int_0^3\int_{-x}^{x(2-x)}\dif y\dif x$
\انتہا{سوال}
%====================
\ابتدا{سوال}
$\int_0^{\pi/4}\int_{\sin x}^{\cos x}\dif y\dif x$
\انتہا{سوال}
%===============
\ابتدا{جواب}
\wf{\unexpanded{
$\sqrt{2}-1$
\begin{center}
\begin{tikzpicture}[font=\small,declare function={f(\x)=cos(deg(\x));g(\x)=sin(deg(\x));}]
\pgfmathsetmacro{\k}{pi/4}
\pgfmathsetmacro{\kk}{1/sqrt(2)}
\begin{axis}[axis on top,clip=false,small,axis lines=center,view/h=130,colormap={}{gray(0cm)=(0.6);gray(1cm)=(0.9);},enlargelimits=true,xlabel={$x$},ylabel={$y$},zlabel={$z$},xtick={\k},xticklabels={$\tfrac{\pi}{4}$},ytick={\kk},yticklabels={$\tfrac{\sqrt{2}}{2}$},ztick={\empty},xlabel style={anchor=west},ylabel style={anchor=south},ymin=0]
\addplot[name path=kf,domain=0:\k](x,{f(x)})node[pos=0.5,above right]{$y=\cos x$}node[right]{$(\tfrac{\pi}{4},\tfrac{\sqrt{2}}{2})$};
\addplot[name path=kg,domain=0:\k](x,{g(x)})node[pos=0.45,below right]{$y=\sin x$};
\addplot[lgray]fill between[of={kf and kg}];
\end{axis}
\end{tikzpicture}
\end{center}
}}
\انتہا{جواب}
%====================
\ابتدا{سوال}
$\int_{-1}^2\int_{y^2}^{y+2}\dif x\dif y$
\انتہا{سوال}
%===============
\ابتدا{سوال}
$\int_{-1}^0\int_{-2x}^{1-x}\dif y\dif x+\int_0^2\int_{-x/2}^{1-x}\dif y\dif x$
\انتہا{سوال}
%===============
\ابتدا{جواب}
\wf{\unexpanded{
$\frac{3}{2}$
\begin{center}
\begin{tikzpicture}[font=\small,declare function={f(\x)=1-\x;g(\x)=-1/2*\x;h(\x)=-2*\x;}]
\begin{axis}[axis on top,clip=false,small,axis lines=center,view/h=130,colormap={}{gray(0cm)=(0.6);gray(1cm)=(0.9);},enlargelimits=true,xlabel={$x$},ylabel={$y$},zlabel={$z$},xtick={-1,2},ytick={2},ztick={\empty},xlabel style={anchor=west},ylabel style={anchor=south}]
\addplot[name path=kf,domain=-1:2](x,{f(x)})node[pos=0.5,above right]{$y=1-x$}node[pos=0,above]{$(-1,2)$}node[right]{$(2,-1)$};
\addplot[name path=kg,domain=0:2](x,{g(x)})node[pos=0.55,below left]{$y=-\tfrac{x}{2}$};
\addplot[name path=kh,domain=-1:0](x,{h(x)})node[pos=0.5,below left]{$y=-2x$};
\addplot[lgray]fill between[of={kf and kg},soft clip={domain=0:2}];
\addplot[lgray]fill between[of={kf and kh},soft clip={domain=-1:0}];
\end{axis}
\end{tikzpicture}
\end{center}
}}
\انتہا{جواب}
%====================
\ابتدا{سوال}\شناخت{سوال_بالکثرت_رقبہ_سے_خطہ_ب}
$\int_0^2\int_{x^2-4}^0\dif y\dif x+\int_0^4\int_0^{\sqrt{x}}\dif y\dif x$
\انتہا{سوال}
%===============

\موٹا{اوسط قیمت}\\
\ابتدا{سوال}
تفاعل \عددی{f(x,y)=\sin(x+y)} کی اوسط قیمت درج ذیل خطوں پر تلاش کریں۔
\begin{enumerate}[a.]
\item
مستطیل \عددی{0\le x\le \pi,\, 0\le y\le \pi}
\item
مستطیل \عددی{0\le x\le \pi,\, 0\le y\le \pi/2}
\end{enumerate}
\انتہا{سوال}
%==========
\ابتدا{جواب}
\wf{\unexpanded{
(ا) \عددی{0}، (ب) \عددی{\tfrac{4}{\pi^2}}
}}
\انتہا{جواب}
%====================
\ابتدا{سوال}
کیا چکور \عددی{0\le x\le 1,\, 0\le y\le 1} یا ربع اول میں دائرہ \عددی{x^2+y^2=1}  میں \عددی{f(x,y)=xy}  کی اوسط قیمت زیادہ ہو گی؟ ان دونوں خطوں میں اوسط کی قیمت تلاش کریں۔
\انتہا{سوال}
%================
\ابتدا{سوال}
چکور \عددی{0\le x\le 2,\, 0\le y\le 2} میں قطع مکافی \عددی{z=x^2+y^2} کا  اوسط قد تلاش کریں۔
\انتہا{سوال}
%===========
\ابتدا{جواب}
\wf{\unexpanded{
$\tfrac{8}{3}$
}}
\انتہا{جواب}
%====================
\ابتدا{سوال}
چکور \عددی{\ln 2\le x\le 2\ln 2,\, \ln 2\le y\le 2\ln 2} میں \عددی{f(x,y)=\tfrac{1}{xy}} کی اوسط قیمت تلاش کریں۔
\انتہا{سوال}
%================
\موٹا{مستقل کثافت}\\
\ابتدا{سوال}
ربع اول میں قطع مکافی \عددی{y=2-x^2} اور لکیر \عددی{x=0}، \عددی{y=x} کے بیچ ایک باریک چادر جس کی کثافت \عددی{\delta=3} ہو  پائی جاتی ہے۔اس کا مرکز کمیت تلاش کریں۔
\انتہا{سوال}
%================
\ابتدا{جواب}
\wf{\unexpanded{
$\bar{x}=\tfrac{5}{14},\,\bar{y}=\tfrac{38}{35}$
}}
\انتہا{جواب}
%====================
\ابتدا{سوال}
ربع اول میں محددی محور اور  لکیر \عددی{x=3} اور \عددی{y=3}  کے بیچ  مستقل کثافت کی باریک  مستطیل چادر پائی جاتی ہے۔ اس کے جمودی  معیار اثر اور رداس دوار تلاش کریں۔
\انتہا{سوال}
%============
\ابتدا{سوال}
ربع اول میں محور \عددی{x}، قطع مکافی \عددی{y^2=2x} اور لکیر \عددی{x+y=4} کے بیچ خطہ کا وسطانی مرکز تلاش کریں۔
\انتہا{سوال}
%==========
\ابتدا{جواب}
\wf{\unexpanded{
$\bar{x}=\tfrac{64}{35},\,\bar{y}=\tfrac{5}{7}$
}}
\انتہا{جواب}
%====================
\ابتدا{سوال}
ربع اول سے لکیر \عددی{x+y=3} ایک تکونی خطہ کاٹتی ہے۔ اس خطہ کا وسطانی مرکز تلاش کریں۔
\انتہا{سوال}
%=================
\ابتدا{سوال}
محور \عددی{x} اور  منحنی \عددی{y=\sqrt{1-x^2}} کے بیچ خطہ کا وسطانی مرکز تلاش کریں۔
\انتہا{سوال}
%=================
\ابتدا{جواب}
\wf{\unexpanded{
$\bar{x}=0,\,\bar{y}=\tfrac{4}{3\pi}$
}}
\انتہا{جواب}
%====================
\ابتدا{سوال}
ربع اول میں قطع مکافی \عددی{y=6x-x^2}  اور لکیر \عددی{y=x} کے بیچ خطے کا رقبہ \عددی{\tfrac{125}{6}} ہے۔ اس کا وسطانی مرکز تلاش کریں۔
\انتہا{سوال}
%===================
\ابتدا{سوال}
ربع اول سے دائرہ \عددی{x^2+y^2=a^2}  ایک خطہ کاٹتا ہے۔ اس خطہ کا وسطانی مرکز تلاش کریں۔
\انتہا{سوال}
%===========
\ابتدا{جواب}
\wf{\unexpanded{
$\bar{x}=\bar{y}=\tfrac{4a}{3\pi}$
}}
\انتہا{جواب}
%====================
\ابتدا{سوال}
دائرہ \عددی{x^2+y^2=4} کے بیچ کثافت \عددی{\delta=1}  کی باریک چادر کی محور \عددی{x} کے لحاظ سے جمودی معیار اثر تلاش کریں۔ اس نتیجہ کو استعمال کرتے ہوئے اس خطہ کی \عددی{I_y} اور \عددی{I_0} دریافت کریں۔
\انتہا{سوال}
%==================
\ابتدا{سوال}
محور \عددی{x} اور قوس \عددی{y=\sin x,\, 0\le x\le \pi} کے بیچ خطہ کا وسطانی مرکز تلاش کریں۔
\انتہا{سوال}
%=================
\ابتدا{جواب}
\wf{\unexpanded{
$\bar{x}=\tfrac{\pi}{2},\,\bar{y}=\tfrac{\pi}{8}$
}}
\انتہا{جواب}
%====================
\ابتدا{سوال}
محور \عددی{x} اور  منحنی \عددی{y=\tfrac{\sin^2x}{x^2}} کے بیچ وقفہ \عددی{\pi\le x\le 2\pi} پر کثافت \عددی{\delta=1} کی باریک چادر پائی جاتی ہے۔ محور \عددی{y} کے لحاظ سے اس کی جمودی معیار اثر تلاش کریں۔
\انتہا{سوال}
%=============
\ابتدا{سوال}\ترچھا{لامتناہی خطہ کا وسطانی مرکز}\\
ربع دوم میں محددی محور اور منحنی \عددی{y=e^x} کے بیچ خطہ کا وسطانی مرکز تلاش کریں۔ (کمیت اور معیار اثر کے کلیات میں آپ کو غیر مناسب تکملات استعمال کرنے ہوں گے۔)
\انتہا{سوال}
%============
\ابتدا{جواب}
\wf{\unexpanded{
$\bar{x}=-1,\,\bar{y}=\tfrac{1}{4}$
}}
\انتہا{جواب}
%====================
\ابتدا{سوال}\ترچھا{لامتناہی چادر کا پہلا معیار اثر}\\
ربع اول میں منحنی \عددی{y=e^{-x^2/2}} کے نیچے  کثافت \عددی{\delta=1} کے لامتناہی جسامت کی چادر   کا محور \عددی{y} کے لحاظ سے پہلا معیار اثر تلاش کریں۔
\انتہا{سوال}
%===============
\موٹا{متغیر کثافت}\\
\ابتدا{سوال}
قطع مکافی \عددی{x=y-y^2} اور لکیر \عددی{x+y=0} کے بیچ باریک چادر کی کثافت \عددی{\delta(x,y)=x+y} ہے۔ محور \عددی{x} کے لحاظ سے اس کی جمودی معیار اثر اور رداس دوار تلاش کریں۔ 
\انتہا{سوال}
%===========
\ابتدا{جواب}
\wf{\unexpanded{
$I_x=\tfrac{64}{105},\,R_x=2\sqrt{\frac{2}{7}}$
}}
\انتہا{جواب}
%====================
\ابتدا{سوال}
ترخیم \عددی{x^2+4y^2=12} سے قطع مکافی \عددی{x=4y^2}  جس چھوٹے حصہ کو کاٹتا ہے، اس کی کثافت \عددی{\delta(x,y)=5x} ہے۔ اس کی کمیت تلاش کریں۔
\انتہا{سوال}
%============
\ابتدا{سوال}
محور \عددی{y} اور  لکیر \عددی{y=x} اور \عددی{y=2-x} کے بیچ تکونی چادر کی کثافت \عددی{\delta(x,y)=6x+3y+3} ہے۔ اس چادر کا مرکز کمیت تلاش کریں۔
\انتہا{سوال}
%===================
\ابتدا{جواب}
\wf{\unexpanded{
$\bar{x}=\tfrac{3}{8},\,\bar{y}=\tfrac{17}{16}$
}}
\انتہا{جواب}
%====================
\ابتدا{سوال}\شناخت{سوال_بالکثرت_درکار_رقبہ}
منحنیات  \عددی{x=y^2} اور \عددی{x=2y-y^2} کے بیچ باریک چادر کی کثافت \عددی{\delta(x,y)=y+1} ہے۔ اس کی کمیت اور محور \عددی{x} کے لحاظ سے جمودی معیار اثر تلاش کریں۔
\انتہا{سوال}
%=============
\ابتدا{سوال}
ربع اول سے خطوط \عددی{x=6} اور \عددی{y=1} ایک مستطیل  باریک چادر کاٹتے ہیں جس کی کثافت \عددی{\delta(x,y)=x+y+1} ہے۔ اس کی مرکز کمیت اور محور \عددی{y} کے لحاظ سے جمودی معیار اثر  اور رداس دوار تلاش کریں۔
\انتہا{سوال}
%============
\ابتدا{جواب}
\wf{\unexpanded{
$\bar{x}=\tfrac{11}{3},\,\bar{y}=\tfrac{14}{27},\, I_y=432,\,R_y=4$
}}
\انتہا{جواب}
%====================
\ابتدا{سوال}
قطع مکافی \عددی{y=x^2} اور لکیر \عددی{y=1} کے بیچ باریک چادر کی کثافت \عددی{\delta(x,y)=y+1} ہے۔ اس کا مرکز کمیت اور محور \عددی{y} کے لحاظ سے جمودی معیار اثر اور رداس  دوار تلاش کریں۔
\انتہا{سوال}
%=============
\ابتدا{سوال}
قطع مکافی \عددی{y=x^2}، محور \عددی{x} اور  لکیر \عددی{x=\mp 1} کے بیچ باریک چادر کی کثافت \عددی{\delta(x,y)=7y+1} ہے۔ اس کا مرکز کمیت اور محور \عددی{y} کے لحاظ سے جمودی معیار اثر اور رداس  دوار تلاش کریں۔
\انتہا{سوال}
%================
\ابتدا{جواب}
\wf{\unexpanded{
$\bar{x}=0,\,\bar{y}=\tfrac{13}{31},\, I_y=\tfrac{7}{5},\,R_y=\sqrt{\frac{21}{31}}$
}}
\انتہا{جواب}
%====================
\ابتدا{سوال}
خطوط  \عددی{x=0}،   \عددی{x=20}، \عددی{y=-1} اور  \عددی{y=1} کے بیچ باریک چادر کی کثافت \عددی{\delta(x,y)=1+x/20} ہے۔ اس کا مرکز کمیت اور محور \عددی{x} کے لحاظ سے جمودی معیار اثر اور رداس  دوار تلاش کریں۔
\انتہا{سوال}
%================
\ابتدا{سوال}\شناخت{سوال_بالکثرت_قطبی_الف}
لکیر \عددی{y=x}، \عددی{y=-x} اور \عددی{y=1} کے بیچ تکونی چادر کی کثافت \عددی{\delta(x,y)=y+1} ہے۔ اس کا مرکز کمیت  اور محددی محوروں کے لحاظ سے جمودی معیار اثر اور رداس دوار تلاش کریں۔ اس کا قطبی جمودی معیار اثر اور  رداس دوار بھی تلاش کریں۔
\انتہا{سوال}
%=================
\ابتدا{جواب}
\wf{\unexpanded{
$\bar{x}=0,\,\bar{y}=\tfrac{7}{10},\,I_x=\frac{9}{10},\, I_y=\frac{3}{10}$\\
$I_0=\frac{6}{5},\, R_x=\frac{3\sqrt{6}}{10},\,R_y=\frac{3\sqrt{2}}{10},\,R_0=\frac{3\sqrt{2}}{5}$
}}
\انتہا{جواب}
%====================
\ابتدا{سوال}
کثافت \عددی{\delta(x,y)=3x^2+1} لیتے ہوئے سوال \حوالہ{سوال_بالکثرت_قطبی_الف} کو دوبارہ حل کریں۔
\انتہا{سوال}
%=================

\موٹا{نظریہ اور مثالیں}\\
\ابتدا{سوال}
مستوی \عددی{xy} میں جراثیم   کی تعدادی کثافت \عددی{f(x,y)=\tfrac{10000e^y}{1+\abs{x}/2}} ہے جہاں \عددی{x} اور \عددی{y} کی ناپ سنٹی میٹر میں ہے۔ مستطیل \عددی{-5\le x\le 5,\, -2\le y\le 0} میں جراثیم کی کل تعداد تلاش کریں۔
\انتہا{سوال}
%=================
\ابتدا{جواب}
\wf{\unexpanded{
$\num{40000}(1-e^{-2})\ln(\tfrac{7}{2})\approx \num{43329}$
}}
\انتہا{جواب}
%====================
\ابتدا{سوال}
سطح زمین پر  کثافت آبادی \عددی{f(x,y)=100(y+1)} ہے جہاں \عددی{x} اور \عددی{y} کلومیٹر میں ہیں۔منحنیات \عددی{x=y^2} اور \عددی{x=2y-y^2} کے بیچ کل آبادی کتنی ہو گی؟
\انتہا{سوال}
%==============
\ابتدا{سوال}
مستقل کثافت کا ایک برتن مستوی \عددی{xy} میں خطہ \عددی{0\le y\le a(1-x^2),\, -1\le x\le 1} پر واقع ہے۔ یہ برتن \عددی{45^{\circ}}   تک ٹیڑھا کرنے تک واپس اپنی جگہ پر آن گرتا ہے۔ مستقل \عددی{a} کی قیمت تلاش کریں۔
\انتہا{سوال}
%================
\ابتدا{جواب}
\wf{\unexpanded{
 \عددی{0<a\le \tfrac{5}{2}}  
}}
\انتہا{جواب}
%====================
\ابتدا{سوال}\ترچھا{جمودی معیار اثر کم سے کم کرنا}\\
ربع اول میں کثافت \عددی{\delta(x,y)=1} کی چادر  لکیر \عددی{x=4} اور \عددی{y=2} کے بیچ  پائی جاتی ہے۔ لکیر \عددی{y=a} کے لحاظ سے اس چادر کی جمودی معیار اثر \عددی{I_a} درج ذیل ہے۔
\begin{align*}
I_a=\int_0^4\int_0^2(y-a)^2\dif y\dif x
\end{align*}
مستقل \عددی{a} کی وہ قیمت تلاش کریں جو \عددی{I_a} کو کم سے کم کرتا ہو۔
\انتہا{سوال}
%================
\ابتدا{سوال}
مستوی \عددی{xy} میں لکیر \عددی{y=\tfrac{1}{\sqrt{1-x^2}}}، \عددی{y=-\tfrac{1}{\sqrt{1-x^2}}}، \عددی{x=0} اور \عددی{x=1}  کے بیچ لامتناہی خطہ کا وسطانی مرکز تلاش کریں۔
\انتہا{سوال}
%================
\ابتدا{جواب}
\wf{\unexpanded{
$(\bar{x},\bar{y})=(2/\pi,0)$
}}
\انتہا{جواب}
%====================
\ابتدا{سوال}
ایک پتلی چھڑی   کی مستقل  خطی کثافت \عددی{\delta} گرام فی  سنٹی میٹر   اور لمبائی \عددی{L} ہے۔  اس کا رداس دوار  دیے گئے محور کے لحاظ سے تلاش کریں۔
\begin{enumerate}[a.]
\item
چھڑی کے محور کو عمودی اور اس کی مرکز کمیت سے گزرتے ہوا خط۔
\item
چھڑی کے ایک سر پر چھڑی کے محور کو عمودی خط۔ 
\end{enumerate}
\انتہا{سوال}
%================
\ابتدا{سوال}
مستوی \عددی{xy} میں مستقل کثافت \عددی{\delta}  کی چادر منحنیات \عددی{x=y^2} اور \عددی{x=2y-y^2} کے بیچ پائی جاتی ہے۔
\begin{enumerate}[a.]
\item
ایسا  \عددی{\delta} دریافت کریں کہ چادر کی  کمیت   سوال \حوالہ{سوال_بالکثرت_درکار_رقبہ} کے چادر کی کمیت کے برابر ہو۔
\item
جزو-ا میں حاصل \عددی{\delta} کی قیمت کا اس خطہ پر \عددی{\delta(x,y)=y+1}  کی اوسط قیمت کے ساتھ موازنہ کریں۔
\end{enumerate}
\انتہا{سوال}
%=========
\ابتدا{جواب}
\wf{\unexpanded{
(ا) \عددی{\tfrac{3}{2}}، (ب) ایک جیسے ہیں۔
}}
\انتہا{جواب}
%====================
\ابتدا{سوال}
دائرہ \عددی{x^2+(y-1)^2=1}  کی کثافت مستقل ہے۔ محوروں کے لحاظ سے اس کے جمودی معیار اثر تلاش کریں۔ 
\انتہا{سوال}
%================
\موٹا{مسئلہ متوازی محور}\\
مستوی \عددی{xy} میں  ایک خطہ  پر  کمیت \عددی{m} کی باریک چادر پائی جاتی ہے۔ اس کے مرکز کمیت سے خط \عددی{L_{c,m}}  گزرتا ہے۔ خط \عددی{L_{c,m}} کے متوازی   \عددی{h} اکائیاں دور خط \عددی{L} پایا جاتا ہے۔ مسئلہ متوازی محور کہتا ہے  کہ \عددی{L_{c,m}} اور \عددی{L} کے لحاظ سے  بالترتیب جمودی معیار اثر \عددی{I_{c,m}} اور \عددی{I_L} درج ذیل کلیہ کو مطمئن کریں گے۔
\begin{align}
I_L=I_{c,m}+mh^2
\end{align}
اس کلیہ کو استعمال کرتے ہوئے ایک جمودی معیار اثر سے دوسرا با آسانی دریافت کیا جا سکتا ہے۔

\ابتدا{سوال}\ترچھا{مسئلہ متوازی محور کا ثبوت}\\
(ا) دکھائیں کہ باریک چادر کے مرکز کمیت سے گزرتی خط کے لحاظ سے  چادر کا جمودی معیار اثر صفر ہو گا۔ (اشارہ: مرکز کمیت کو مبدا پر رکھیں اور  خط کو محور \عددی{y} پر رکھیں۔ کلیہ \عددی{\bar{x}=\tfrac{M_y}{M}} کیا دیگا؟) (ب)  جزو-ا کے نتیجہ سے مسئلہ متوازی محور  اخذ کریں۔(اشارہ:  خط \عددی{L_{c,m}} کو محور \عددی{y} اور \عددی{L} کو \عددی{x=h} پر رکھ کر \عددی{I_L} کے تکمل کو دو حصوں میں  لکھیں۔)
\انتہا{سوال}
%==================
\ابتدا{سوال}
(ا) مسئلہ متوازی محور استعمال کرتے ہوئے مثال \حوالہ{مثال_بالکثرت_تکون_رداس_دوار} کے نتائج استعمال کرتے ہوئے  اس مثال میں چادر کے مرکز کمیت سے گزرتی افقی اور انتصابی خطوط کے لحاظ سے چادر کی جمودی معیار اثر تلاش کریں۔ (ب) جزو-ا کے نتائج استعمال کرتے ہوئے خطوط \عددی{x=1} اور \عددی{y=2} کے لحاظ سے چادر کی جمودی معیار اثر دریافت کریں۔
\انتہا{سوال}
%==============
\موٹا{کلیہ پاپس}\\
جناب پاپس نے حصہ  \حوالہ{حصہ_استعمال_تکمل_بنیادی_نقش_دیگر_نمونی_استعمال} کا مسئلہ پاپس بیان کیا۔ اس کے علاوہ وہ جانتے تھے کہ  ایک دوسرے کو نہ ڈھانپتے ہوئے دو  مستوی خطوں  کا وسطانی مرکز ان خطوں کے وسطانی مراکز سے گزرتے ہوئے خط پر پایا جاتا ہے۔مستوی \عددی{xy} میں ایک دوسرے کو نہ ڈھانپتی ہوئی   دو باریک چادر  \عددی{P_1} اور \عددی{P_2} فرض کریں،  جن کی کمیت بالترتیب \عددی{m_1} اور \عددی{m_2} ہو۔مبدا سے بالترتیب  ان چادروں کے مراکز کمیت تک سمتیات    \عددی{\kvec{c}_1} اور \عددی{\kvec{c}_2} لیں۔اب اشتراک \عددی{P_1\cup P_2} کے مرکز کمیت کا تعین گر سمتیہ  درج ذیل   دیگا۔
\begin{align}\label{مساوات_بالکثرت_کلیہ_پاپس}
\kvec{c}=\frac{m_1\kvec{c}_1+m_2\kvec{c}_2}{m_1+m_2}
\end{align}
مساوات \حوالہ{مساوات_بالکثرت_کلیہ_پاپس} کو \اصطلاح{کلیہ پاپس}\فرہنگ{کلیہ!پاپس}\حاشیہب{Pappus's formula}\فرہنگ{Pappus!formula} کہتے ہیں۔ایک دوسرے کو نہ ڈھانپتی ہوئی دو سے زیادہ (لیکن متناہی تعداد کی)  چادروں کے لئے درج ذیل کلیہ ہو گا۔
\begin{align}\label{مساوات_بالکثرت_عمومی_کلیہ_پاپس}
\kvec{c}=\frac{m_1\kvec{c}_1+m_2\kvec{c}_2+\cdots+m_n\kvec{c}_n}{m_1+m_2+\cdots+m_n}
\end{align}
یہ کلیہ بالخصوص وہاں فائدہ مند ہو گا جہاں   غیر منظم  شکل و صورت کی چادر کے حصوں کے وسطانی مراکز ہم  جیومیٹری سے علیحدہ علیحدہ طور پر جانتے ہوں اور جہاں  ہر حصہ از خود  مستقل کثافت کا ہو۔ ہم اس کلیہ کو استعمال کرتے ہوئے پوری چادر کا وسطانی مرکز معلوم کر سکتے ہیں۔ 

\ابتدا{سوال}
کلیہ پاپس (مساوات \حوالہ{مساوات_بالکثرت_کلیہ_پاپس})  اخذ کریں۔ (اشارہ:  ربع اول میں ان خطوں کو ترسیم کر کے ان کے  مراکز کمیت  \عددی{(\bar{x}_1,\bar{y}_1)} اور \عددی{(\bar{x}_2,\bar{y}_2)} کی نشاندہی کریں۔  محددی محور کے لحاظ سے \عددی{P_1\cup P_2} کے معیار اثر کیا ہوں گے؟)
\انتہا{سوال}
%==================
\ابتدا{سوال}
ریاضی (الکراجی)  ماخوذ اور مساوات \حوالہ{مساوات_بالکثرت_کلیہ_پاپس} استعمال کرتے ہوئے دکھائیں کہ کسی بھی عدد صحیح \عددی{n>2} کے لئے  مساوات \حوالہ{مساوات_بالکثرت_عمومی_کلیہ_پاپس} مطمئن ہو گا۔
\انتہا{سوال}
%===================
\ابتدا{سوال}\شناخت{سوال_بالکثرت_تین_خطے_الف}
فرض کریں \عددی{A} ، \عددی{B} اور \عددی{C} تین اشکال ہیں (شکل \حوالہ{شکل_سوال_بالکثرت_تین_خطے_الف}-ا)۔ کلیہ پاپس کی مدد سے  درج ذیل کے وسطانی مراکز دریافت کریں۔
\begin{multicols}{4}
\begin{enumerate}[a.]
\item
$A\cup B$
\item
$A\cup C$
\item
$B\cup C$
\item
$A\cup B\cup C$
\end{enumerate}
\end{multicols}
\انتہا{سوال}
%==============
\ابتدا{جواب}
\wf{\unexpanded{
(ا)  \عددی{(\tfrac{7}{5},\tfrac{31}{10})}، (ب) \عددی{(\tfrac{19}{7},\tfrac{18}{7})}، (ج) \عددی{(\tfrac{9}{2},\tfrac{19}{8})}، (د) \عددی{(\tfrac{11}{4},\tfrac{43}{16})}
}}
\انتہا{جواب}
%====================
\begin{figure}
\centering
\begin{subfigure}{0.45\textwidth}
\centering
\begin{tikzpicture}[font=\small]
\begin{axis}[axis on top,clip=false,small,axis lines=center,view/h=130,colormap={}{gray(0cm)=(0.6);gray(1cm)=(0.9);},enlargelimits=true,xlabel={$x$},ylabel={$y$},zlabel={$z$},xtick={2,4,7},ytick={1,2,3,4,5},ztick={\empty},xlabel style={anchor=west},ylabel style={anchor=south}]
\addplot[fill=lgray]coordinates{(0,1)(2,1)(2,5)(0,5)(0,1)}node[pos=0,above right]{$A$};
\addplot[fill=lgray]coordinates{(2,3)(4,3)(4,4)(2,4)(2,3)}node[pos=0,above right]{$B$};
\addplot[fill=lgray]coordinates{(4,0)(7,2)(4,4)(4,0)}node[pos=0,above right,yshift=1ex]{$C$};
\end{axis}
\end{tikzpicture}
\caption{}
\end{subfigure}\hfill
\begin{subfigure}{0.45\textwidth}
\centering
\begin{tikzpicture}[font=\small]
\begin{axis}[axis on top,clip=false,small,axis lines=center,view/h=130,colormap={}{gray(0cm)=(0.6);gray(1cm)=(0.9);},enlargelimits=true,xlabel={$x$},ylabel={$y$},zlabel={$z$},xtick={24},ytick={2,12},ztick={\empty},xlabel style={anchor=west},ylabel style={anchor=south}]
\addplot[fill=lgray]coordinates{(0,0)(0,12)(1.5,12)(1.5,2)(24,2)(24,0) (0,0)};
\addplot[]coordinates {(1.5,13)(1.5,15)};
\addplot[-stealth]coordinates{(-2,14)(0,14)};
\addplot[-stealth]coordinates{(5,14)(1.2,14)}node[pos=0,right]{$\tfrac{1}{2}$};
\end{axis}
\end{tikzpicture}
\caption{}
\end{subfigure}
\caption{اشکال برائے سوال \حوالہ{سوال_بالکثرت_تین_خطے_الف} اور سوال \حوالہ{سوال_بالکثرت_تین_خطے_ب}}
\label{شکل_سوال_بالکثرت_تین_خطے_الف}
\end{figure}
\ابتدا{سوال}\شناخت{سوال_بالکثرت_تین_خطے_ب}
وسطانی مرکز دریافت کریں (شکل \حوالہ{شکل_سوال_بالکثرت_تین_خطے_الف}-ب)۔
\انتہا{سوال}
%=================
\ابتدا{سوال}\شناخت{سوال_بالکثرت_مثلث_دائرہ}
ایک مساوی الساقین  مثلث  \عددی{T} کا قاعدہ \عددی{2a} اور قد \عددی{h} ہے۔ اس کا قاعدہ ، رداس \عددی{a} کے  نصف دائرہ \عددی{D}   کے قطر پر   پایا جاتا ہے۔ مثلث دائرہ کے باہر ہے۔  \عددی{T\cup D}  کا وسطانی مرکز (ا)  \عددی{T} اور \عددی{D} کی مشترک سرحد پر  (ب)  \عددی{T} کے اندر ہونے کے لئے \عددی{a} اور \عددی{h} کا تعلق دریافت کریں۔
\انتہا{سوال}
%==========
\ابتدا{جواب}
\wf{\unexpanded{
مشترک سرحد پر ہونے کے لئے \عددی{h=a\sqrt{2}}، مثلث کے اندر ہونے کے لئے \عددی{h>a\sqrt{2}}
}}
\انتہا{جواب}
%====================
\ابتدا{سوال}\شناخت{سوال_بالکثرت_چکور_قاعدہ}
ایک مساوی الساقین مثلث  \عددی{T} جس کا قد \عددی{h} ہے کا قاعدہ  چکور \عددی{Q}  کا ایک ضلع ہے۔ چکور کے   ضلع کی لمبائی  \عددی{s}  ہے۔ (چکور اور مثلث ایک دوسرے کو نہیں ڈھانپتے ہیں۔) \عددی{T\cup Q}  کا  وسطانی مرکز مثلث کے قاعدہ پر رکھنے کی خاطر \عددی{h} کا  \عددی{s} کے ساتھ کیا  تعلق   گا؟  اپنے جواب کا موازنہ سوال \حوالہ{سوال_بالکثرت_مثلث_دائرہ} کے جواب کے ساتھ کریں۔
\انتہا{سوال}
%==========
\انتہا{سوالات}




\حصہ{دوہرا تکملات کا قطبی   روپ}\شناخت{حصہ_بالکثرت_دوہرا_تکملات_قطبی_روپ}
بعض اوقات تکمل کو قطبی روپ میں تبدیل کرنے سے  اس کا حل آسان ہو جاتا ہے۔اس حصہ میں یہ تبدیلی دکھائی جائے گی اور ان تکملات کی قیمت کا حصول دکھایا جائے گا جن کے سرحد قطبی روپ میں دیے گئے ہوں۔

\جزوحصہء{قطبی روپ میں تکملات} 
مستوی \عددی{xy} میں دوہرا تکمل کا ذکر کرتے ہوئے  ہم نے خطہ \عددی{R} کو  مستطیلی ٹکڑوں میں اس طرح   کاٹا کہ مستطیل کے اضلاع محددی محوروں کے متوازی ہوں۔اس طرح   ان مستطیلوں کے اضلاع  مستقل \عددی{x} اور یا مستقل \عددی{y} لکھے جا سکتے  ہیں۔ کارتیسی محدد میں مستطیل قدرتی صورت ہے۔ قطبی محددی نظام میں "قطبی مستطیل" قدرتی صورت ہے جس کے اضلاع مستقل \عددی{r} اور مستقل \عددی{\theta} لکھے جا سکتے ہیں۔

فرض کریں تفاعل  \عددی{f(r,\theta)} خطہ \عددی{R} پر معین ہے  جس  کے سرحد شعاع \عددی{\theta=\alpha} اور \عددی{\theta=\beta} اور استمراری منحنیات \عددی{r=g_1(\theta)} اور \عددی{r=g_2(\theta)}  ہیں۔مزید \عددی{\alpha} اور \عددی{\beta} کے بیچ ہر قیمت کے لئے \عددی{0\le g_1(\theta)\le g_2(\theta)\le a} ہے۔ یوں  \عددی{R} پنکھا نما  خطہ \عددی{Q} میں، جس کو عدم مساوات \عددی{0\le r\le a,\, \alpha\le\theta\le\beta} ظاہر کرتی ہیں، پایا جائے گا (شکل \حوالہ{مساوات_بالکثرت_قطبی_مستطیل_خانے_الف})۔

\begin{figure}
\centering
\begin{minipage}{0.60\textwidth}
\centering
\begin{tikzpicture}[font=\small]
\pgfmathsetmacro{\r}{2.5}
\pgfmathsetmacro{\A}{30}
\pgfmathsetmacro{\B}{180-\A}
\pgfmathsetmacro{\C}{\B-\A}
\fill[lgray,opacity=0.5]([shift={(\A:0.4*\r)}]0,0) arc (\A:\B:0.4*\r)--++(\B:0.3*\r) arc (\B:\A:0.7*\r)--cycle;
\fill[lgray,opacity=0.5]([shift={(\A+0.2*\C:0.3*\r)}]0,0) arc (\A+0.2*\C:\B:0.3*\r)--++(\B:0.1*\r) arc (\B:\A+0.2*\C:0.4*\r)--cycle;
\fill[lgray,opacity=0.5]([shift={(\A+0.4*\C:0.2*\r)}]0,0) arc (\A+0.4*\C:\B-0.2*\C:0.2*\r)--++(\B-0.2*\C:0.1*\r) arc (\B-0.2*\C:\A+0.4*\C:0.3*\r)--cycle;
\fill[lgray,opacity=0.5]([shift={(\A:0.7*\r)}]0,0) arc (\A:\A+0.2*\C:0.7*\r)--++(\A+0.2*\C:0.1*\r) arc (\A+0.2*\C:\A:0.8*\r)--cycle;
\fill[gray,opacity=0.5]([shift={(\A+0.2*\C:0.5*\r)}]0,0) arc (\A+0.2*\C:\A+0.4*\C:0.5*\r)--++(\A+0.4*\C:0.1*\r) arc (\A+0.4*\C:\A+0.2*\C:0.6*\r)--cycle;
\draw[-latex](-1.1*\r,0)node[left]{$\theta=\pi$}--(1.1*\r,0)node[right]{$\theta=0$};
\draw([shift={(0:\r)}]0,0) arc (0:180:\r)node[pos=0.07,right]{$r=a$};
\draw(0,0)--++(\A:1.1*\r)coordinate[pos=0.3](kRA)coordinate[pos=0.8](kRB)node[right]{$\theta=\beta$};
\draw(0,0)--++(\B:1.1*\r)coordinate[pos=0.25](kLA)coordinate[pos=0.7](kLB)node[left]{$\theta=\alpha$};
\foreach \k in {0.1,0.2,0.3,0.4,0.5,0.6,0.7,0.8,0.9}{\draw([shift={(\A:\k*\r)}]0,0) arc (\A:\B:\k*\r);}
\foreach \kk in {0.2,0.4,0.6,0.8}{\draw(0,0)--++(\A+\kk*\C:\r);}
\draw(\A+0.2*\C:\r+0.1)--++(\A+0.2*\C:0.4);
\draw(\A+0.4*\C:\r+0.1)--++(\A+0.4*\C:0.4);
\draw[stealth-stealth]([shift={(\A+0.2*\C:\r+0.3)}]0,0) arc (\A+0.2*\C:\A+0.4*\C:\r+0.3)node[pos=0.4,above]{$\Delta \theta$};
\draw([shift={(\A-0.05*\C:0.5*\r)}]0,0) arc (\A-0.05*\C:\A-0.15*\C:0.5*\r)coordinate[pos=0.5](kL);
\draw([shift={(\A-0.05*\C:0.6*\r)}]0,0) arc (\A-0.05*\C:\A-0.15*\C:0.6*\r)coordinate[pos=0.5](kH);
\draw[stealth-](kL)--++(\A-0.1:-0.2);
\draw[stealth-](kH)--++(\A-0.1:0.3)node[right,xshift=-0.75ex]{$\Delta r$};
\draw(\A+0.3*\C:0.55*\r)node[circ]{}node[pin={[pin distance=0.25cm]170:{$(r_k,\theta_k)$}}]{}node[pin={[pin distance=0.3cm]0:{$\Delta S_k$}}]{};
\draw[thick](kLA)to[out=-10,in=160]coordinate[pos=0.35](Ga)(kRA);
\draw[thick](kLB) to [out=80,in=130]coordinate[pos=0.20](Gb)(kRB);
\draw(Ga)node[pin=-135:{$r=g_1(\theta)$}]{};
\draw(Gb)node[pin=135:{$r=g_2(\theta)$}]{};
\end{tikzpicture}
\caption{
خطہ \عددی{R:\, g_1(\theta)\le r\le g_2(\theta)}، \عددی{\alpha\le \theta\le \beta}  پنکھا نما خطہ \عددی{Q:\, 0\le r\le a}، \عددی{\alpha\le \theta\le \beta}  میں پایا جاتا ہے۔ خطہ \عددی{Q} کی خانہ بندی شعاعوں اور دائری قوسین سے کرتے ہوئے \عددی{R} کی خانہ بندی کی جاتی ہے۔
}
\label{مساوات_بالکثرت_قطبی_مستطیل_خانے_الف}
\end{minipage}\hfill
\begin{minipage}{0.30\textwidth}
\centering
\begin{tikzpicture}[font=\small]
\pgfmathsetmacro{\ra}{2.5}
\pgfmathsetmacro{\rb}{\ra+1}
\pgfmathsetmacro{\angA}{30}
\pgfmathsetmacro{\angB}{55}
\pgfmathsetmacro{\angC}{1/2*(\angA+\angB)}
\pgfmathsetmacro{\rc}{1/2*(\ra+\rb)}
\pgfmathsetmacro{\rd}{1.15*\rb}
\draw(0,0)node[below left]{$O$}--++(\angA:\ra) arc (\angA:\angB:\ra)--(0,0);
\draw[fill=lgray](\angA:\ra)--(\angA:\rb) arc (\angA:\angB:\rb)--(\angB:\ra)node[pos=0.35,shift={(\angB+90:0.25)}]{$\Delta r$} arc (\angB:\angA:\ra);
\draw[decorate,decoration={brace,amplitude=10pt,raise=3pt}](\angA:\rb)--(0,0)node[pos=0.5,shift={(\angA-90:0.75)}]{\text{\RL{بڑا خطہ}}};
\draw[decorate,decoration={brace,amplitude=10pt,raise=3pt}](0,0)--++(\angB:\ra)node[pos=0.5,shift={(\angB+90:0.80)}]{\text{\RL{چھوٹا  خطہ}}};
\draw(\angA:\rb+0.1*\rb)--++(\angA:0.1*\rb);
\draw(\angB:\rb+0.1*\rb)--++(\angB:0.1*\rb);
\draw[stealth-stealth]([shift={(\angA:\rd)}]0,0) arc (\angA:\angB:\rd)node[pos=0.5,shift={(\angC:0.3)}]{$\Delta \theta$};
\draw(\angC:\rc)node[circ]{}node[shift={(\angC+90:0.3)}]{$r_k$}node[shift={(\angC-90:0.35)}]{$\Delta S_k$};
\draw[thick](0,0)--++(\angC:\rb)node[circ]{}node[pin={[pin distance=1cm]100:{$r_k+\tfrac{\Delta r}{2}$}}]{};
\draw(\angC:\ra)node[circ]{}node[pin={[pin distance=1.25cm]115:{$r_k-\tfrac{\Delta r}{2}$}}]{};
\end{tikzpicture}
\caption{
سایہ دار خطے کا رقبہ \عددی{\Delta S_k} حاصل کرنے کے لئے بڑے خطے سے چھوٹے خطے  کا رقبہ منفی کریں۔
}
\label{مساوات_بالکثرت_قطبی_مستطیل_خانے_ب}
\end{minipage}
\end{figure}

ہم \عددی{Q} پر  دائری قوسین  اور شعاعوں  کا جال بچھاتے ہیں۔  یہ قوسین ان دائروں سے کاٹے جاتے ہیں جن کا مرکز مبدا    پر ہے اور جن کے رداس \عددی{\Delta r}، \عددی{2\Delta r}، \نقطے،\عددی{m\Delta r} ہیں جہاں \عددی{\Delta r=\tfrac{a}{m}} ہے۔ان شعاع کو درج ذیل لکھا جا سکتا ہے جہاں \عددی{\Delta\theta=\tfrac{\beta-\alpha}{m'}} ہے۔
\begin{align*}
\theta=\alpha,\, \theta=\alpha+\Delta \theta,\,\theta=\alpha+2\Delta \theta,\cdots, \theta=\alpha+m'\Delta\theta=\beta
\end{align*}
یہ شعاع اور قوسین \عددی{Q} کو  "قطبی مستطیلوں" میں تقسیم کرتے ہیں۔

ہم ان قطبی مستطیلوں کو \عددی{1} تا \عددی{n}  کی   شمار سے ظاہر کرتے ہیں جو مکمل طور پر  \عددی{R} کے اندر پائے جاتے ہوں اور ان کے رقبوں کو \عددی{\Delta S_1}، \عددی{\Delta S_2}، \نقطے، \عددی{\Delta S_n} سے ظاہر کرتے ہیں۔ شمار کرنے کی ترتیب غیر ضروری ہے۔  ہم   \عددی{\Delta S_k}  رقبے کی قطبی مستطیل  کے  مرکز کو  \عددی{(r_k,\theta_k)} سے ظاہر کرتے ہیں۔ قطبی مستطیل کے مرکز سے مراد وہ نقطہ ہے جو  دونوں دائری قوسین کی اوسط  رداس کے قوس  اور  اس شعاع پر پایا جاتا ہو جو دونوں قوسین کو درمیان سے کاٹتی ہو۔  ہم اب  درج ذیل مجموعہ لیتے ہیں۔
\begin{align}\label{مساوات_بالکثرت_قطبی_مجموعہ_الف}
J_n=\sum_{k=1}^n f(r,\theta)\Delta S_n
\end{align}
اگر  پورے \عددی{R} پر \عددی{f} استمراری ہو، تب جال کے خانے چھوٹے سے چھوٹے کر کے    \عددی{\Delta r} اور \عددی{\Delta \theta} کو صفر تک پہنچانے سے  یہ مجموعہ ایک حد تک پہنچتا ہے۔ یہ حد \عددی{R} پر \عددی{f} کا دوہرا تکمل کہلاتا ہے جس کو علامتی طور پر درج ذیل لکھا جاتا ہے۔
\begin{align*}
\lim_{n\to\infty} J_n=\iint\limits_R f(r,\theta)\dif S
\end{align*}

اس حد کی قیمت تلاش کرنے کی خاطر ہمیں  مجموعہ \عددی{J_n}  یوں لکھنا ہو گا کہ \عددی{\Delta S_k}  کی قیمت \عددی{\Delta r} اور \عددی{\Delta\theta} کی روپ میں ہو۔ رقبہ \عددی{\Delta S_k} کی  اندرونی قوسی سرحد کا رداس \عددی{r_k-\tfrac{\Delta r}{2}} جبکہ اس کی   بیرونی قوسی سرحد کا رداس \عددی{r_k+\tfrac{\Delta r}{2}} ہے (شکل \حوالہ{مساوات_بالکثرت_قطبی_مستطیل_خانے_ب})۔ ان قوسین سے مبدا تک دائری تکونی خطوں کے رقبے
\begin{gather}
\begin{aligned}
&\frac{1}{2}\big(r_k-\frac{\Delta r}{2}\big)^2\Delta \theta&&\text{\RL{اندرونی تکونی رقبہ}}\\
&\frac{1}{2}\big(r_k+\frac{\Delta r}{2}\big)^2\Delta\theta&&\text{\RL{بیرونی تکونی رقبہ}}
\end{aligned}
\end{gather}
ہوں گے۔ یوں درج  ہو گا۔
\begin{align*}
\Delta S_k&=\text{\RL{بیرونی تکونی رقبہ}}-\text{\RL{اندرونی تکونی رقبہ}}\\
&=\frac{\Delta \theta}{2}\left[\big(r_k+\frac{\Delta r}{2}\big)^2-\big(r_k-\frac{\Delta r}{2}\big)^2\right]=\frac{\Delta\theta}{2}(2r_k\Delta r)=r_k\Delta r_k\Delta \theta
\end{align*}
اس نتیجہ کو مساوات \حوالہ{مساوات_بالکثرت_قطبی_مجموعہ_الف} میں پر کرنے سے درج ذیل حاصل ہو گا۔
\begin{align}
J_n=\sum_{k=1}^nf(r_k,\theta_k)r_k\Delta r\Delta \theta
\end{align}

مسئلہ فوبینی کی ایک صورت کہتی ہے کہ اس مجموعہ  کی حد  \عددی{r} اور \عددی{\theta} کے لحاظ سے  درج ذیل  بارہا تکمل  دیگا۔
\begin{align}
\iint\limits_Rf(r,\theta)\dif S=\int_{\theta=\alpha}^{\theta=\beta}\int_{r=g_1(\theta)}^{r=g_2(\theta)}f(r,\theta)r\dif r\dif \theta
\end{align}

\جزوحصہء{تکمل کی حدیں}
کارتیسی محدد میں تکمل کی حدیں تلاش کرنے کا طریقہ کار  قطبی محدد کے لئے  بھی کارآمد ہے۔

%========================================

\begin{figure}
\centering
\begin{subfigure}{0.30\textwidth}
\centering
\begin{tikzpicture}[font=\small,declare function={f(\x)=sqrt(4-\x^2);}]
\pgfmathsetmacro{\k}{sqrt(2)}
\begin{axis}[clip=false,width=4.5cm,axis lines=middle,xtick={\k},ytick={\k,2},xticklabels={$\sqrt{2}$},yticklabels={$\sqrt{2}$,$2$},xmin=0,ymin=0,enlargelimits=true, xlabel={$x$}, ylabel={$y$}, xlabel style={at={(current axis.right of origin)},anchor=west},ylabel style={at={(current axis.above origin)},anchor=south}]
\addplot[domain=0:sqrt(2)]{f(x)}node[pos=0.5,above]{$x^2+y^2=4$};
\addplot[]coordinates{(0,\k)(\k,\k)}node[pos=0.5,below]{$y=\sqrt{2}$}node[below,xshift=1ex,font=\scriptsize]{$(\sqrt{2},\sqrt{2})$}node[pos=0.5,above]{$R$};
\end{axis}
\end{tikzpicture}
\caption{}
\end{subfigure}\hfill
\begin{subfigure}{0.30\textwidth}
\centering
\begin{tikzpicture}[font=\small,declare function={f(\x)=sqrt(4-\x^2);}]
\pgfmathsetmacro{\k}{sqrt(2)}
\pgfmathsetmacro{\ang}{60}
\pgfmathsetmacro{\ra}{sqrt(2)/(sin(\ang))}
\begin{axis}[clip=false,width=4.5cm,axis lines=middle,xtick={\k},ytick={\k,2},xticklabels={$\sqrt{2}$},yticklabels={$\sqrt{2}$,$2$},xmin=0,ymin=0,enlargelimits=true, xlabel={$x$}, ylabel={$y$}, xlabel style={at={(current axis.right of origin)},anchor=west},ylabel style={at={(current axis.above origin)},anchor=south}]
\addplot[domain=0:sqrt(2)]{f(x)};
\addplot[]coordinates{(0,\k)(\k,\k)}node[pos=0.5,above]{$R$};
\draw[-latex](0,0)--(\ang:2.5)node[right,yshift=-1ex]{$L$};
\draw(\ang:\ra)node[pin={[align=center,pin distance=0.5cm,yshift=2ex]-45:{\text{\RL{شعاع \عددی{\sqrt{2}\csc\theta} }}\\  \text{\RL{پر داخل ہوتی ہے}}}}]{};
\draw(\ang:2)node[pin={[above,yshift=2ex]100:{\text{\RL{شعاع \عددی{r=2} پر خارج ہوتی ہے}}}}]{};
\draw[-stealth]([shift={(0:0.5)}]0,0) arc (0:\ang:0.5)node[pos=0.6,right,yshift=0.5ex]{$\theta$};
\end{axis}
\end{tikzpicture}
\caption{}
\end{subfigure}\hfill
\begin{subfigure}{0.30\textwidth}
\centering
\begin{tikzpicture}[font=\small,declare function={f(\x)=sqrt(4-\x^2);}]
\pgfmathsetmacro{\k}{sqrt(2)}
\pgfmathsetmacro{\ang}{60}
\pgfmathsetmacro{\ra}{sqrt(2)/(sin(\ang))}
\begin{axis}[clip=false,width=4.5cm,axis lines=middle,xtick={\k},ytick={\k,2},xticklabels={$\sqrt{2}$},yticklabels={$\sqrt{2}$,$2$},xmin=0,ymin=0,ymax=2.25,enlargelimits=true, xlabel={$x$}, ylabel={$y$}, xlabel style={at={(current axis.right of origin)},anchor=west},ylabel style={at={(current axis.above origin)},anchor=south}]
\addplot[domain=0:sqrt(2)]{f(x)};
\addplot[]coordinates{(0,\k)(\k,\k)}node[pos=0.5,above]{$R$};
\draw[-latex](0,0)--(\ang:2.5)node[above left]{$L$};
\draw[-stealth]([shift={(0:0.5)}]0,0) arc (0:45:0.5)node[pos=0.6,right,yshift=0.5ex]{\text{\RL{کم سے کم \عددی{\theta}}}};
\draw[dashed](0,0)--(45:2.5)node[above]{$y=x$};
\draw(0,2.2)node[pin=20:{\RL{زیادہ سے زیادہ زاویہ \عددی{\tfrac{\pi}{2}} ہے}}]{};
\end{axis}
\end{tikzpicture}
\caption{}
\end{subfigure}
\caption{قطبی محدد میں تکمل کی قیمت کے قدم۔}
\label{شکل_بالکثرت_قطبی_دوہرا_اقدام}
\end{figure}


%==============
\موٹا{قطبی محدد میں تکمل حاصل کرنے  کا طریقہ}\\
قطبی محدد میں خطہ \عددی{R} پر \عددی{\iint_Rf(r,\theta)\dif S} کی قیمت حاصل کرنے کے لئے    پہلے \عددی{r} کے لحاظ سے اور بعد میں \عددی{\theta} کے لحاظ سے تکمل لیتے ہوئے ہمیں  درج ذیل اقدام کرنے ہوں گے۔
\begin{enumerate}[1.]
\item
\ترچھا{خاکہ:}\quad
تکمل کے  خطہ کا خاکہ بنائیں اور اس کی سرحدی منحنیات   پر نام و نشان لگائیں  (شکل \حوالہ{شکل_بالکثرت_قطبی_دوہرا_اقدام}-ا)۔
\item
\ترچھا{تکمل کی  \عددی{r} حدیں:}\quad
مبدا سے بڑھتی ہوئی  \عددی{r} کے رخ خطہ \عددی{R} سے گزرتا ہوا   شعاع  \عددی{L} کھینچیں۔ جن مقامات  پر \عددی{L}  اس خطہ میں داخل   اور اس سے     خارج ہوتا ہے، یہ تکمل کی  \عددی{r} حدیں  ہوں گے۔ ان کی قیمتیں عموماً \عددی{\theta} پر منحصر ہو گی (شکل \حوالہ{شکل_بالکثرت_قطبی_دوہرا_اقدام}-ب)۔
\item
\ترچھا{تکمل کی  \عددی{\theta} حدیں:}\quad
وہ \عددی{\theta} حدیں  منتخب کریں جن میں \عددی{R} سے گزرتی ہوئی تمام شعاعیں  شامل ہوں (شکل \حوالہ{شکل_بالکثرت_قطبی_دوہرا_اقدام}-ج)۔
\end{enumerate}

 تکمل درج ذیل ہو گا۔
\begin{align*}
\int\limits_R f(r,\theta)\dif S=\int_{\theta=\pi/4}^{\theta=\pi/2}\int_{r=\sqrt{2}\csc\theta}^{r=2}f(r,\theta)r\dif r\dif \theta
\end{align*}

%===================
\ابتدا{مثال}\شناخت{مثال_بالکثرت_دائرہ_قلب_نما}
دائرہ \عددی{r=1} کے باہر اور قلب نما \عددی{r=1+\cos\theta} کے اندر خطہ میں  \عددی{f(r,\theta)} کے تکمل کی  حدیں  تلاش کریں۔ 

حل:\quad
\begin{enumerate}[1.]
\item
\ترچھا{خاکہ:}\quad
ہم خطے کا خاکہ بنا کر  سرحدی منحنیات پر نام و نشان لکھتے ہیں (شکل \حوالہ{شکل_مثال_بالکثرت_دائرہ_قلب_نما})۔
\item
\ترچھا{تکمل کی  \عددی{r} حدیں:}\quad
مبدا سے  نکلتی ہوئی  علامتی شعاع خطہ \عددی{R} میں \عددی{r=1} کے مقام پر داخل اور \عددی{r=1+\cos\theta} کے مقام پر خارج ہو گی۔
\item
\ترچھا{تکمل کی \عددی{\theta} حدیں:}\quad
مبدا سے نکلتی ہوئی وہ شعاعیں جو \عددی{R} سے گزرتی ہوں، \عددی{\theta=-\tfrac{\pi}{2}} تا \عددی{\theta=\tfrac{\pi}{2}} میں پائی جاتی ہیں۔
\end{enumerate}
یوں تکمل درج ذیل ہو گا۔
\begin{align*}
\int_{-\pi/2}^{\pi/2}\int_{1}^{1+\cos\theta}f(r,\theta)r\dif r\dif \theta
\end{align*}
\انتہا{مثال}
%=====================

\begin{figure}
\centering
\begin{minipage}{0.50\textwidth}
\centering
\begin{tikzpicture}[font=\small,declare function={f(\x)=1+cos(\x);}]
\pgfmathsetmacro{\ang}{-40}
\pgfmathsetmacro{\ra}{1+cos(\ang)}
\begin{axis}[axis on top,axis equal,clip=false,small,axis lines=middle,xtick={1,2},xticklabels={\rlap{$1$},\rlap{$2$}},ytick={\empty},enlargelimits=true, xlabel={$x$}, ylabel={$y$}, xlabel style={at={(current axis.right of origin)},anchor=west},ylabel style={at={(current axis.above origin)},anchor=south}]
\addplot[opacity=0.5,fill=lgray,data cs=polar,domain=0:360,smooth]{f(x)}node[pos=0.2,above right]{$r=1+\cos\theta$};
\draw[fill=white](0,0) circle (1);
\addplot[data cs=polar,domain=90:270,smooth]{f(x)};
\draw[-latex](0,0)--(\ang:2.2);
\addplot[-stealth,domain=0:\ang] ({0.4*cos(x)},{0.4*sin(x)})node[pos=0.6,right]{$\theta$}; 
\draw(\ang:\ra)node[pin={[right,align=center]20:{\text{\RL{خارج}}\\ $r=1+\cos\theta$}}]{};
\draw(\ang:1)node[pin={[below,pin distance=1cm]-110:{\text{\RL{داخل \عددی{r=1}}}}}]{};
\draw(0,1)node[pin=135:{$\theta=\tfrac{\pi}{2}$}]{};
\draw(0,-1)node[pin=-135:{$\theta=-\tfrac{\pi}{2}$}]{};
\end{axis}
\end{tikzpicture}
\caption{دائرہ اور قلب نما (مثال \حوالہ{مثال_بالکثرت_دائرہ_قلب_نما})}
\label{شکل_مثال_بالکثرت_دائرہ_قلب_نما}
\end{minipage}\hfill
\begin{minipage}{0.40\textwidth}
\centering
\begin{tikzpicture}[font=\small,declare function={f(\x)=sqrt(4*cos(2*\x));}]
\pgfmathsetmacro{\ang}{20}
\pgfmathsetmacro{\ra}{sqrt(4*cos(2*\ang))}
\begin{axis}[axis equal,axis on top,clip=false,small,axis lines=middle,xtick={\empty},ytick={\empty},enlargelimits=true, xlabel={$x$}, ylabel={$y$}, xlabel style={at={(current axis.right of origin)},anchor=west},ylabel style={at={(current axis.above origin)},anchor=south}]
\addplot[opacity=0.5,fill=lgray,data cs=polar,domain=0:90,smooth]{f(x)};
\addplot[data cs=polar,domain=90:360,smooth]{f(x)}node[pos=0.9,below right]{$r^2=4\cos 2\theta$};
\draw[-latex](0,0)--(\ang:2.5);
\draw(\ang:\ra)node[pin={[above,pin distance=1cm]70:{\text{\RL{خارج \عددی{r=\sqrt{4\cos 2\theta}}}}}}]{};
\draw(0,0)node[pin={[below,pin distance=1cm]-130:{\text{\RL{داخل \عددی{r=0}}}}}]{};
\draw(0,0)--(45:1.25)node[above]{$\tfrac{\pi}{4}$};
\draw(0,0)--(-45:1.25)node[below]{$-\tfrac{\pi}{4}$};
\end{axis}
\end{tikzpicture}
\caption{
تکمل کی قیمت کے حصول میں ہم \عددی{r} کو \عددی{0} تا \عددی{\sqrt{4\cos 2\theta}} جبکہ \عددی{\theta} کو 0 تا \عددی{\tfrac{\pi}{4}} لیتے ہیں۔
}
\label{شکل_مثال_بالکثرت_دو_چشمہ}
\end{minipage}
\end{figure}

اگر \عددی{f(r,\theta)} ایک مستقل تفاعل ہو جس کی قیمت \عددی{1} ہو  تب \عددی{R} پر \عددی{f} کا تکمل \عددی{R} کا رقبہ ہو گا۔

\موٹا{قطبی محدد میں رقبہ}\\
قطبی محددی مستوی میں بند اور محدود خطہ \عددی{R} کا  رقبہ درج ذیل ہو گا۔
\begin{align}
S=\iint\limits_R r\dif r\dif\theta
\end{align}

جیسا آپ توقع کرتے ہوں گے یہ کلیہ، پہلے دیے گئے کلیات کے عین مطابق ہے۔ ہم اس حقیقت کا ثبوت پیش نہیں کریں گے۔

\ابتدا{مثال}\شناخت{مثال_بالکثرت_دو_چشمہ}
دو  چشمہ  \عددی{r^2=4\cos 2\theta}میں گھیرا ہوا  رقبہ تلاش کریں۔

حل:\quad
ہم دو چشمہ کا خاکہ بنا کر تکمل کی حدیں معلوم کرتے ہیں (شکل \حوالہ{شکل_مثال_بالکثرت_دو_چشمہ})۔ہم دیکھتے ہیں کہ ربع اول میں دو چشمہ کے رقبہ کو \عددی{4} سے ضرب دے کر پورا رقبہ حاصل کیا جا سکتا ہے۔
\begin{align*}
S&=4\int_{0}^{\pi/4}\int_0^{\sqrt{4\cos2\theta}}r\dif r\dif \theta=4\int_0^{\pi/4}\big[\frac{r^2}{2}\big]_{r=0}^{r=\sqrt{4\cos2\theta}}\dif\theta\\
&=4\int_0^{\pi/4}2\cos2\theta\dif\theta=4\sin2\theta\big]_0^{\pi/4}=4
\end{align*}
\انتہا{مثال}
%=============

\جزوحصہء{کارتیسی تکملات کی قطبی تکملات میں تبدیلی}
کارتیسی تکمل \عددی{\iint_Rf(x,y)\dif x\dif y} کو قطبی تکمل میں  دو قدموں میں تبدیل کیا جاتا ہے:
\begin{enumerate}[1.]
\item
کارتیسی تکمل میں  \عددی{x=r\cos\theta} اور \عددی{y=r\sin\theta} پر کرتے ہوئے  \عددی{\dif x\dif y} کی جگہ \عددی{r\dif r\dif \theta} لکھیں۔
\item
خطہ \عددی{R} کی سرحد کی قطبی حدیں مہیا کریں۔
\end{enumerate}
 
یوں کارتیسی تکمل سے درج ذیل حاصل ہو گا جہاں تکمل کے خطہ کو قطبی محدد میں \عددی{G} سے ظاہر کیا گیا ہے۔
\begin{align}\label{مساوات_بالکثرت_قطبی_محدد_میں_رقبہ_کی_عمومی}
\iint\limits_R f(x,y)\dif x\dif y=\iint\limits_G f(r\cos\theta,r\sin\theta)r\dif r\dif \theta
\end{align}
یہ  باب \حوالہ{باب_تکمل} میں ترکیب بدل کی طرح ہے البتہ یہاں ایک کی بجائے دو متغیرات ہیں۔دھیان رہے کہ \عددی{\dif x\dif y} کی جگہ \عددی{\dif r\dif\theta} نہیں بلکہ \عددی{r\dif r\dif \theta} پر کیا جاتا ہے۔ اس کی وجہ آگے پیش کی جائے گی۔

\ابتدا{مثال}\شناخت{مثال_بالکثرت_چوتھائی_دائرہ}
ربع اول میں دائرہ \عددی{x^2+y^2=1} کی ایک چوتھائی   میں کثافت \عددی{\delta(x,y)=1} کی باریک چادر کی مبدا کے لحاظ سے  قطبی  معیار اثر تلاش کریں۔

حل:\quad
ہم چادر  کا خاکہ بنا کر تکمل کی حدیں معلوم کرتے ہیں (شکل \حوالہ{شکل_مثال_بالکثرت_چوتھائی_دائرہ})۔ کارتیسی محدد میں اس خطہ  کا قطبی معیار اثر سے مراد درج ذیل تکمل ہے۔
\begin{align*}
\int_0^1\int_0^{\sqrt{1-x^2}}(x^2+y^2)\dif y\dif x
\end{align*}
ہم  \عددی{y} کے لحاظ سے تکمل لے  کر
\begin{align*}
\int_0^1 (x^2\sqrt{1-x^2}+\frac{(1-x^2)^{3/2}}{3})\dif x
\end{align*}
حاصل کرتے ہیں جس کا حل، جدول کی مدد کے بغیر، مشکل ہے۔

 اس تکمل کو قطبی تکمل میں تبدیل کرنے  سے  حالات بہتر ہوتے ہیں۔ ہم \عددی{x=r\cos\theta} اور \عددی{y=r\sin\theta}  پر کر کے \عددی{\dif x\dif y} کی جگہ \عددی{r\dif r\dif \theta} لکھ کر
\begin{align*}
\int_0^1\int_0^{\sqrt{1-x^2}}(x^2+y^2)\dif y\dif x&=\int_0^{\pi/2}\int_0^{1} (r^2)r\dif r\dif \theta\\
&=\int_0^{\pi/2}\big[\frac{r^2}{4}\big]_{r=0}^{r=1}\dif\theta=\int_{0}^{\pi/2}\frac{1}{4}\dif\theta=\frac{\pi}{8}
\end{align*}
حاصل کرتے ہیں۔قطبی محدد میں تکمل اتنا آسان کیوں ہوا۔ ایک وجہ یہ ہے کہ \عددی{x^2+y^2}  سادہ صورت \عددی{r^2} اختیار کرتا ہے۔ دوسری وجہ یہ کہ تکمل کی حدیں اب مستقل ہیں۔
\انتہا{مثال}
%=====================
\begin{figure}
\centering
\begin{minipage}{0.45\textwidth}
\centering
\begin{tikzpicture}[font=\small,declare function={fx(\x)=cos(\x);fy(\x)=sin(\x);}]
\begin{axis}[axis equal,axis on top,clip=false,small,axis lines=middle,xtick={1},ytick={1},enlargelimits=true, xlabel={$x$}, ylabel={$y$}, xlabel style={at={(current axis.right of origin)},anchor=west},ylabel style={at={(current axis.above origin)},anchor=south}]
\addplot[domain=0:90,smooth]({fx(x)},{fy(x)})node[pos=0.5,right,align=center]{$x^2+y^2=1$\\ $r=1$}node[pos=0,pin=45:{$\theta=0$}]{}node[pos=1,pin={[right]45:{$\theta=\tfrac{\pi}{2}$}}]{};
\draw(0,0)node[below left]{$O$};
\end{axis}
\end{tikzpicture}
\caption{
قطبی محدد میں یہ خطہ \عددی{0\le r\le 1}، \عددی{0\le\theta\le\tfrac{\pi}{2}} ہے ۔
}
\label{شکل_مثال_بالکثرت_چوتھائی_دائرہ}
\end{minipage}\hfill
\begin{minipage}{0.45\textwidth}
\centering
\begin{tikzpicture}[font=\small,declare function={fx(\x)=cos(\x);fy(\x)=sin(\x);}]
\begin{axis}[axis equal,axis on top,clip=false,small,axis lines=middle,xtick={-1,1},ytick={1},enlargelimits=true, xlabel={$x$}, ylabel={$y$}, xlabel style={at={(current axis.right of origin)},anchor=west},ylabel style={at={(current axis.above origin)},anchor=south}]
\addplot[domain=0:180,smooth]({fx(x)},{fy(x)})node[pos=0.4,pin={[align=center,above,pin distance=0.25cm,xshift=2ex]45:{$y=\sqrt{1-x^2}$\\  $r=1$}}]{}node[pos=0,pin={[above]45:{$\theta=0$}}]{}node[pos=1,pin={[above]135:{$\theta=\pi$}}]{};
\draw(0,0)node[below left]{$O$};
\end{axis}
\end{tikzpicture}
\caption{
نصف دائری  خطہ \عددی{0\le r\le 1}، \عددی{0\le \theta\le \pi} ہے۔
}
\label{شکل_مثال_بالکثرت_نصف_دائرہ}
\end{minipage}
\end{figure}

\ابتدا{مثال}\شناخت{مثال_بالکثرت_نصف_دائرہ}
محور \عددی{x} اور منحنی \عددی{y=\sqrt{1-x^2}}  کے بیچ نصف دائری خطہ \عددی{R} پر درج ذیل تکمل کی قیمت تلاش کریں (شکل \حوالہ{شکل_مثال_بالکثرت_نصف_دائرہ})۔
\begin{align*}
\iint\limits_R e^{x^2+y^2}\dif y\dif x
\end{align*}

حل:\quad
کارتیسی محدد میں یہ تکمل غیر بنیادی ہے اور \عددی{e^{x^2+y^2}} کا \عددی{x} یا  \عددی{y} کے لحاظ سے تکمل، سیدھا طریقے سے،   حاصل نہیں کیا جا سکتا ہے۔ اس کے باوجود یہ تکمل اور اس طرح کے دیگر تکملات ریاضیات میں اہمیت رکھتے ہیں اور ان کا حل ضروری ہے۔ قطبی محدد یہاں مدد گار ثابت ہوتے ہیں۔ہم \عددی{x=r\cos\theta} اور \عددی{y=r\sin\theta} پر کر کے \عددی{\dif y\dif x} کی جگہ \عددی{r\dif r\dif\theta} لکھ کر تکمل کی قیمت حاصل کرتے ہیں:
\begin{align*}
\iint\limits_R e^{x^2+y^2}\dif y\dif x&=\int_0^{\pi}\int_0^1 e^{r^2}r\dif r\dif \theta=\int_0^{\pi}\big[\frac{1}{2}e^{r^2}\big]_0^1\dif\theta\\
&=\int_0^{\pi}\frac{1}{2}(e-1)\dif\theta=\frac{\pi}{2}(e-1)
\end{align*}
آپ نے دیکھا کہ   \عددی{e^{r^2}}  کے تکمل میں ہمیں   \عددی{r\dif r\dif \theta} کا \عددی{r} درکار  تھا  جس کے بغیر ہم تکمل حاصل نہیں کر سکتے  تھے۔
\انتہا{مثال}
%===============

\جزوحصہء{سوالات}
\ابتدا{سوالات}
\موٹا{قطبی تکملات کی قیمت کی تلاش}\\
سوال \حوالہ{سوال_بالکثرت_کارتیسی_سے_قطبی_الف} تا سوال \حوالہ{سوال_بالکثرت_کارتیسی_سے_قطبی_ب}  میں دیے گئے تکملات کو  قطبی روپ میں تبدیل کر کے حل کریں۔

\ابتدا{سوال}\شناخت{سوال_بالکثرت_کارتیسی_سے_قطبی_الف}
$\int_{-1}^1\int_0^{\sqrt{1-x^2}}\dif y\dif x$
\انتہا{سوال}
%===================
\ابتدا{جواب}
\wf{\unexpanded{
$\tfrac{\pi}{2}$
}}
\انتہا{جواب}
%==================
\ابتدا{سوال}
$\int_{-1}^1\int_{-\sqrt{1-x^2}}^{\sqrt{1-x^2}}\dif y\dif x$
\انتہا{سوال}
%=======================
\ابتدا{سوال}
$\int_{0}^{1}\int_{0}^{\sqrt{1-y^2}}(x^2+y^2)\dif x\dif y$
\انتہا{سوال}
%=======================
\ابتدا{جواب}
\wf{\unexpanded{
$\tfrac{\pi}{8}$
}}
\انتہا{جواب}
%==================
\ابتدا{سوال}
$\int_{-1}^{1}\int_{-\sqrt{1-y^2}}^{\sqrt{1-y^2}}(x^2+y^2)\dif x\dif y$
\انتہا{سوال}
%=======================
\ابتدا{سوال}
$\int_{-a}^{a}\int_{-\sqrt{a^2-x^2}}^{\sqrt{a^2-x^2}}\dif y\dif x$
\انتہا{سوال}
%=======================
\ابتدا{جواب}
\wf{\unexpanded{
$\pi a^2$
}}
\انتہا{جواب}
%==================
\ابتدا{سوال}
$\int_{0}^{2}\int_{0}^{\sqrt{4-y^2}}(x^2+y^2)\dif x\dif y$
\انتہا{سوال}
%=======================
\ابتدا{سوال}
$\int_{0}^{6}\int_{0}^{y}x\dif x\dif y$
\انتہا{سوال}
%=======================
\ابتدا{جواب}
\wf{\unexpanded{
$36$
}}
\انتہا{جواب}
%==================
\ابتدا{سوال}
$\int_{0}^{2}\int_{0}^{x}y\dif y\dif x$
\انتہا{سوال}
%=======================
\ابتدا{سوال}
$\int_{-1}^{0}\int_{-\sqrt{1-x^2}}^{0}\frac{2}{1+\sqrt{x^2+y^2}}\dif y\dif x$
\انتہا{سوال}
%=======================
\ابتدا{جواب}
\wf{\unexpanded{
$(1-\ln 2)\pi$
}}
\انتہا{جواب}
%==================
\ابتدا{سوال}
$\int_{-1}^{1}\int_{-\sqrt{1-y^2}}^{0}\frac{4\sqrt{x^2+y^2}}{1+x^2+y^2}\dif x\dif y$
\انتہا{سوال}
%=======================
\ابتدا{سوال}
$\int_{0}^{\ln 2}\int_{0}^{\sqrt{(\ln 2)^2-y^2}}e^{\sqrt{x^2+y^2}}\dif x\dif y$
\انتہا{سوال}
%=======================
\ابتدا{جواب}
\wf{\unexpanded{
$(2\ln 2-1)(\pi/2)$
}}
\انتہا{جواب}
%==================
\ابتدا{سوال}
$\int_{0}^{1}\int_{0}^{\sqrt{1-x^2}}e^{-(x^2+y^2)}\dif y\dif x$
\انتہا{سوال}
%=======================
\ابتدا{سوال}
$\int_{0}^{2}\int_{0}^{\sqrt{1-(x-1)^2}}\frac{x+y}{x^2+y^2}\dif y\dif x$
\انتہا{سوال}
%=======================
\ابتدا{جواب}
\wf{\unexpanded{
$\tfrac{\pi}{2}+1$
}}
\انتہا{جواب}
%==================
\ابتدا{سوال}
$\int_{0}^{2}\int_{-\sqrt{1-(y-1)^2}}^{0} xy^2\dif x\dif y$
\انتہا{سوال}
%=======================
\ابتدا{سوال}
$\int_{-1}^{1}\int_{-\sqrt{1-y^2}}^{\sqrt{1-y^2}}\ln(x^2+y^2+1)\dif x\dif y$
\انتہا{سوال}
%=======================
\ابتدا{جواب}
\wf{\unexpanded{
$\pi(\ln(4)-1)$
}}
\انتہا{جواب}
%==================
\ابتدا{سوال}\شناخت{سوال_بالکثرت_کارتیسی_سے_قطبی_ب}
$\int_{-1}^{1}\int_{-\sqrt{1-x^2}}^{\sqrt{1-x^2}}\frac{2}{(1+x^2+y^2)^2}\dif y\dif x$
\انتہا{سوال}
%=======================

\موٹا{قطبی محدد میں رقبات کی تلاش}\\
\ابتدا{سوال}
ربع اول سے منحنی \عددی{r=2(2-\sin 2\theta)^{1/2}} جس خطہ کو کاٹتی ہے، اس کا رقبہ تلاش کریں۔
\انتہا{سوال}
%=================
\ابتدا{جواب}
\wf{\unexpanded{
$2(\pi-1)$
}}
\انتہا{جواب}
%==================
\ابتدا{سوال}
قلب نما \عددی{r=1+\cos\theta} کے اندر اور دائرہ \عددی{r=1} کے باہر خطہ کا رقبہ تلاش کریں۔
\انتہا{سوال}
%=================
\ابتدا{سوال}
گلاب \عددی{r=12\cos 3\theta} کے ایک پتے  کا رقبہ تلاش کریں۔
\انتہا{سوال}
%=================
\ابتدا{جواب}
\wf{\unexpanded{
$12\pi$
}}
\انتہا{جواب}
%==================
\ابتدا{سوال}
مثبت محور \عددی{x} اور پیچ دار \عددی{r=\tfrac{4\theta}{3},\, 0\le\theta\le 2\pi}  کے بیچ رقبہ تلاش کریں۔ اس  خطہ کی صورت گھونگا  کے خول سے ملتی جلتی ہے۔  
\انتہا{سوال}
%=================
\ابتدا{سوال}
ربع اول میں قلب نما \عددی{r=1+\sin\theta} جس خطہ کو کاٹتا ہے، اس کا رقبہ تلاش کریں۔
\انتہا{سوال}
%=========
\ابتدا{جواب}
\wf{\unexpanded{
$\tfrac{3\pi}{8}+1$
}}
\انتہا{جواب}
%==================
\ابتدا{سوال}
قلب نما \عددی{r=1+\cos\theta} اور \عددی{r=1-\cos\theta} کے مشترکہ خطہ کا رقبہ تلاش کریں۔ 
\انتہا{سوال}
%==========
\موٹا{کمیت  اور معیار اثر}\\
\ابتدا{سوال}
مستقل کثافت \عددی{\delta(x,y)=3} کی باریک چادر جس کی زیریں سرحد محور \عددی{x} اور بالائی سرحد قلب نما \عددی{r=1-\cos\theta} ہے، کا محور \عددی{x} کے لحاظ سے معیار اثر اول تلاش کریں۔
\انتہا{سوال}
%===============
\ابتدا{جواب}
\wf{\unexpanded{
$4$
}}
\انتہا{جواب}
%==================
\ابتدا{سوال}
دائرہ \عددی{x^2+y^2=a^2} کے اندر باریک  دائرہ قرص  کی کثافت \عددی{\delta(x,y)=k(x^2+y^2)} ہے جہاں \عددی{k} ایک مستقل ہے۔ اس قرص  کی محور \عددی{x} کے لحاظ سے جمودی معیار اثر اور مبدا کے لحاظ سے قطبی معیار اثر تلاش کریں۔
\انتہا{سوال}
%============
\ابتدا{سوال}
دائرہ \عددی{r=3} کے باہر اور دائرہ \عددی{r=6\sin\theta} کے اندر  چادر کی کثافت \عددی{\delta(r,\theta)=\tfrac{1}{r}} ہے۔ اس  چادر کی کمیت تلاش کریں۔
\انتہا{سوال}
%===============
\ابتدا{جواب}
\wf{\unexpanded{
$6\sqrt{3}-2\pi$
}}
\انتہا{جواب}
%==================
\ابتدا{سوال}
قلب نما \عددی{r=1-\cos\theta} کے اندر اور دائرہ \عددی{r=1} کے باہر باریک چادر کی کثافت \عددی{\delta(r,\theta)=\tfrac{1}{r^2}} ہے۔ مبدا کے لحاظ سے اس چادر کی قطبی معیار اثر تلاش کریں۔ 
\انتہا{سوال}
%===========
\ابتدا{سوال}
قلب نما \عددی{r=1+\cos\theta} کا وسطانی مرکز تلاش کریں۔
\انتہا{سوال}
%=================
\ابتدا{جواب}
\wf{\unexpanded{
$\bar{x}=\tfrac{5}{6},\,\bar{y}=0$
}}
\انتہا{جواب}
%==================
\ابتدا{سوال}
قلب نما \عددی{r=1+\cos\theta} کے اندر باریک چادر کی کثافت \عددی{\delta(x,y)=1} ہے۔ مبدا کے لحاظ سے اس چادر کی قطبی معیار اثر تلاش کریں۔
\انتہا{سوال}
%==================

\موٹا{اوسط قیمتیں}\\
\ابتدا{سوال}
مستوی \عددی{xy} میں قرص  \عددی{x^2+y^2\le a^2} کے اوپر نصف کرہ \عددی{z=\sqrt{a^2-x^2-y^2}}  کا اوسط قد تلاش کریں۔
\انتہا{سوال}
%============
\ابتدا{جواب}
\wf{\unexpanded{
$\tfrac{2a}{3}$
}}
\انتہا{جواب}
%==================
\ابتدا{سوال}
مستوی \عددی{xy} میں قرص \عددی{x^2+y^2\le a^2} کے اوپر (ایک)  مخروط \عددی{z=\sqrt{x^2+y^2}}  کا اوسط قد تلاش کریں۔ 
\انتہا{سوال}
%=============
\ابتدا{سوال}
قرص \عددی{x^2+y^2\le a^2} میں مبدا سے نقطہ \عددی{N(x,y)} کا  اوسط  فاصلہ تلاش کریں۔
\انتہا{سوال}
%============
\ابتدا{جواب}
\wf{\unexpanded{
$\tfrac{2a}{3}$
}}
\انتہا{جواب}
%==================
\ابتدا{سوال}
قرص \عددی{x^2+y^2\le a} میں نقطہ \عددی{N(x,y)} کا سرحدی نقطہ \عددی{A(1,0)} سے  فاصلے کے مربع   کی اوسط  قیمت تلاش کریں۔
\انتہا{سوال}
%=============

\موٹا{نظریہ اور مثالیں}\\
\ابتدا{سوال}
خطہ \عددی{1\le x^2+y^2\le e} پر \عددی{f(x,y)=\tfrac{\ln(x^2+y^2)}{\sqrt{x^2+y^2}}}   کے تکمل کی قیمت  دریافت کریں۔
\انتہا{سوال}
%===============
\ابتدا{جواب}
\wf{\unexpanded{
$2\pi$
}}
\انتہا{جواب}
%==================
\ابتدا{سوال}
خطہ \عددی{1\le x^2+y^2\le e^2} پر \عددی{f(x,y)=\tfrac{\ln(x^2+y^2)}{x^2+y^2}} کا تکمل حل کریں۔
\انتہا{سوال}
%===============
\ابتدا{سوال}
قلب نما \عددی{r=1+\cos\theta} کے اندر اور دائرہ \عددی{r=1} کے باہر خطہ ٹھوس قائمہ بیلن کا قاعدہ ہے۔ اس بیلن کی چوٹی مستوی \عددی{z=x} میں پائی جاتی  ہے۔ اس بیلن کا حجم تلاش کریں۔ 
\انتہا{سوال}
%=============
\ابتدا{جواب}
\wf{\unexpanded{
$\tfrac{4}{3}+\tfrac{5\pi}{8}$
}}
\انتہا{جواب}
%==================
\ابتدا{سوال}
دو چشمہ \عددی{r^2=2\cos 2\theta} کے اندر خطہ ٹھوس قائمہ بیلن کا قاعدہ ہے۔ اس بیلن کی چوٹی کرہ \عددی{z=\sqrt{2-r^2}} کی سطح  کو مس کرتی ہے۔اس بیلن کا حجم تلاش کریں۔ 
\انتہا{سوال}
%===============
\ابتدا{سوال}\شناخت{سوال_بالکثرت_تفاعل_خلل_قیمت_درکار}
(ا) غیر مناسب تکمل \عددی{I=\int_0^{\infty} e^{-x^2}\dif x}  کے حل کا درست طریقہ یہ ہے کہ پہلے اس کا مربع لیں:
\begin{align*}
I^2=\big(\int_0^{\infty}e^{-x^2}\dif x\big)\big(\int_0^{\infty}e^{-y^2}\dif y\big)=\int_0^{\infty}\int_0^{\infty}e^{-(x^2+y^2)}\dif x\dif y
\end{align*}
اس تکمل کو قطبی روپ میں لکھ کر حل کریں۔  (ب) درج ذیل تکمل  کی قیمت تلاش کریں۔  (حصہ \حوالہ{حصہ_تکمل_کی_ترکیب_غیر_مناسب_تکمل} کا سوال \حوالہ{سوال_تراکیب_تکمل_تفاعل_خلل_اندازاً_قیمت_درکار} جاری)۔
\begin{align*}
\lim_{x\to\infty}\erf(x)=\lim_{x\to \infty}\int_0^{x}\frac{2e^{-t^2}}{\sqrt{\pi}}\dif t
\end{align*}
\انتہا{سوال}
%=================
\ابتدا{جواب}
\wf{\unexpanded{
(ا) \عددی{\sqrt{\pi/2}}، (ب) \عددی{1}
}}
\انتہا{جواب}
%==================
\ابتدا{سوال}
درج ذیل تکمل کی قیمت   تلاش   کریں۔
\begin{align*}
\int_{0}^{\infty}\int_{0}^{\infty}\frac{1}{(1+x^2+y^2)^2}\dif x\dif y
\end{align*}
\انتہا{سوال}
%=================
\ابتدا{سوال}
قرص \عددی{x^2+y^2\le \tfrac{3}{4}} پر  تفاعل \عددی{f(x,y)=1/(1-x^2-y^2)} کا   تکمل حل کریں۔ کیا قرص \عددی{x^2+y^2\le 1} پر \عددی{f(x,y)} کا تکمل موجود ہے؟ اپنے جواب کی وجہ پیش کریں۔
\انتہا{سوال}
%================
\ابتدا{جواب}
\wf{\unexpanded{
$\pi\ln 4$\quad نہیں
}}
\انتہا{جواب}
%==================
\ابتدا{سوال}
قطبی محدد میں دوہرا تکمل استعمال کرتے ہوئے قطبی   منحنی \عددی{r=f(\theta),\, \alpha\le\theta\le\beta} اور مبدا کے بیچ پنکھا نما خطہ کے رقبہ کا درج ذیل  کلیہ اخذ کریں۔
\begin{align*}
S=\int_{\alpha}^{\beta}\frac{1}{2}r^2\dif\theta
\end{align*}
\انتہا{سوال}
%===========
\ابتدا{سوال}
رداس \عددی{a} کے دائرہ میں \عددی{N_0} ایک نقطہ  ہے اور \عددی{N_0} سے دائرہ کے مرکز تک فاصلہ \عددی{h} ہے۔ کسی بھی اختیاری نقطہ  \عددی{N} سے \عددی{N_0} تک فاصلہ  کو \عددی{d} سے ظاہر کریں۔ دائرہ میں محیط خطہ پر \عددی{d^2} کی اوسط قیمت تلاش کریں۔ (اشارہ:  دائرے کے مرکز کو مبدا پر اور \عددی{N_0} کو محور \عددی{x} پر  رکھ کر اپنے لئے آسانی پیدا کریں۔)
\انتہا{سوال}
%==============
\ابتدا{جواب}
\wf{\unexpanded{
$\tfrac{1}{2}(a^2+2h^2)$
}}
\انتہا{جواب}
%==================
\ابتدا{سوال}
فرض کریں ایک قطبی خطے کا  رقبہ درج ذیل ہے۔
\begin{align*}
S=\int_{\pi/4}^{3\pi/4}\int_{\csc\theta}^{2\sin\theta}r\dif r\dif\theta
\end{align*}
(ا) اس تکمل  کے خطہ کا خاکہ بنائیں۔ (ب) پاپس کے ایک مسئلہ اور حصہ \حوالہ{حصہ_استعمال_تکمل_بنیادی_نقش_دیگر_نمونی_استعمال} میں سوال \حوالہ{سوال_تکمل_استعمال_وسطانی_مرکز_درکار_الف} میں وسطانی مرکز کی معلومات استعمال کرتے ہوئے  اس خطہ کو محور \عددی{x} کے گرد گھمانے سے حاصل ٹھوس  جسم طواف کا حجم تلاش کریں۔
\انتہا{سوال}
%================
\موٹا{کمپیوٹر کا استعمال}\\
سوال \حوالہ{سوال_بالکثرت_کمپیوٹر_کارتیسی_سے_قطبی_الف} تا سوال \حوالہ{سوال_بالکثرت_کمپیوٹر_کارتیسی_سے_قطبی_ب} میں کمپیوٹر استعمال کرتے ہوئے  کارتیسی  تکملات کو قطبی تکملات میں تبدیل کر کے ان قطبی تکملات کی قیمتیں تلاش کریں۔ آپ کو درج ذیل اقدام کرنے ہوں گے۔
\begin{enumerate}[a.]
\item
کارتیسی تکمل کے خطہ کا خاکہ  مستوی   \عددی{xy}  پر  بنائیں۔
\item
جزو-ا میں خطہ کی ہر سرحد کی کارتیسی مساوات کو \عددی{r} اور \عددی{\theta} کے لئے حل کرتے  ان کی قطبی مساوات تلاش کریں۔
\item
جزو-ب کے نتائج استعمال کرتے ہوئے تکمل کے خطہ  کے  خاکہ کو قطبی \عددی{r\theta} مستوی میں  بنائیں۔
\item
متکمل کو کارتیسی سے قطبی روپ میں تبدیل کریں۔ جزو-ج کے خاکہ سے تکمل کی حدیں معلوم کر کے قطبی تکمل کی قیمت کمپیوٹر کی مدد سے  حاصل کریں۔
\end{enumerate} 
\ابتدا{سوال}\شناخت{سوال_بالکثرت_کمپیوٹر_کارتیسی_سے_قطبی_الف}
$\int_0^1\int_x^1\frac{y}{x^2+y^2}\dif y\dif x$
\انتہا{سوال}
%=====
\ابتدا{سوال}
$\int_{0}^{1}\int_{0}^{x/2}\frac{x}{x^2+y^2}\dif y\dif x$
\انتہا{سوال}
%==================
\ابتدا{سوال}
$\int_{0}^{1}\int_{-y/3}^{y/3}\frac{y}{\sqrt{x^2+y^2}}\dif x\dif y$
\انتہا{سوال}
%==================
\ابتدا{سوال}\شناخت{سوال_بالکثرت_کمپیوٹر_کارتیسی_سے_قطبی_ب}
$\int_{0}^{1}\int_{y}^{2-y}\sqrt{x+y}\dif x\dif y$
\انتہا{سوال}
%==================
\انتہا{سوالات}


\حصہ{کارتیسی محدد میں تہرا تکمل}
ہم تہرا تکملات کی مدد سے تین بعدی اجسام کے حجم، کمیت اور معیار اثر  اور تین متغیری تفاعل کی اوسط قیمت معلوم کرتے ہیں۔ اگلے باب میں ہم دیکھیں گے کہ سمتی میدان   اور  حرکت سیال  کے مطالعہ میں ہمیں ان تکملات سے کیسا واسطہ پڑتا ہے۔

\جزوحصہء{تہرا تکمل}   
فرض کریں فضا میں بند محدود   خطہ \عددی{D}  پر تفاعل \عددی{F(x,y,z)} معین ہے، تب \عددی{D} پر تکمل \عددی{F}  کی تعریف کچھ یوں ہو گی۔ہم ایک مستطیل خطہ  جس میں \عددی{D} پایا جاتا ہو کو محددی مستویات کے متوازی  مستویات  سے مستطیل خانوں میں تقسیم کرتے ہیں (شکل \حوالہ{شکل_بالکثرت_تہرا_تکمل_مستطیل_خانے}-ا)۔ ہم \عددی{D} کے اندر پائے جانے والے  خانوں کو  (کسی بھی ترتیب سے) \عددی{1} تا \عددی{n} کی  شمار سے ظاہر  کرتے ہیں۔   یوں ایک علامتی مستطیل خانے کے اضلاع \عددی{\Delta x_k}، \عددی{\Delta y_k} اور \عددی{\Delta z_k} جبکہ اس کا حجم \عددی{\Delta H_h} ہو گا (شکل \حوالہ{شکل_بالکثرت_تہرا_تکمل_مستطیل_خانے}-ب)۔ ہم ہر مستطیل خانے میں کوئی نقطہ \عددی{(x_k,y_k,z_k)} منتخب کر کے درج ذیل مجموعہ  لیتے  ہیں۔
\begin{align}\label{مساوات_بالکثرت_تہرا_مجموعہ_الف}
J_n=\sum_{k=1}^{n} F(x_k,y_k,z_k)\Delta H_k
\end{align}
اگر \عددی{F} استمراری ہو اور \عددی{D} کی تحدیدی سطح ہموار سطحوں پر مشتمل ہو جو ایک دوسرے کے ساتھ استمراری منحنیات میں جڑتے ہوں، تب جوں جوں \عددی{\Delta x_k}، \عددی{\Delta y_k} اور \عددی{\Delta z_k} صفر کے قریب پہنچتے ہوں  توں توں  مجموعات \عددی{J_n}   ایک حد تک پہنچتے ہیں:
\begin{align}
\lim_{n\to \infty}J_n=\iiint\limits_D F(x,y,z)\dif H
\end{align} 
ہم اس حد کو\اصطلاح{ \عددی{D} پر \عددی{F} کا تہرا تکمل}\فرہنگ{تکمل!تہرا}\حاشیہب{triple integral}\فرہنگ{integral!triple} کہتے ہیں۔ یہ حد چند غیر استمراری تفاعل کے لئے بھی موجود ہے۔
\begin{figure}
\centering
\begin{subfigure}{0.45\textwidth}
\centering
\begin{tikzpicture}[x={(-0.5cm,-0.7071cm)},y={(1cm,0)},z={(0,1cm)}]
\pgfmathsetmacro{\kkx}{2}
\pgfmathsetmacro{\kky}{3}
\pgfmathsetmacro{\kkz}{2}
\pgfmathsetmacro{\kx}{\kkx/3}
\pgfmathsetmacro{\ky}{\kky/4}
\pgfmathsetmacro{\kz}{\kkz/3}
\foreach \x in {0,1,2,3}{\draw(\x*\kx,0,0)--++(0,\kky,0)--++(0,0,\kkz)--++(0,-\kky,0)--++(0,0,-\kkz);}
\foreach \y in {0,1,2,3,4}{\draw(0,\y*\ky,0)--++(\kkx,0,0)--++(0,0,\kkz)--++(-\kkx,0,0)--++(0,0,-\kkz);}
\foreach \z in {0,1,2,3}{\draw(0,0,\z*\kz)--++(\kkx,0,0)--++(0,\kky,0)--++(-\kkx,0,0)--++(0,-\kky,0);}
\draw[-latex](\kkx,0,0)--++(0.5,0,0)node[left]{$x$};
\draw[-latex](0,\kky,0)--++(0,0.5,0)node[right]{$y$};
\draw[-latex](0,0,\kkz)--++(0,0,0.25)node[left]{$z$};
\end{tikzpicture}
\caption{
ایک حجم جس میں \عددی{D} پایا جاتا ہے کو محددی مستویات کے متوازی سطحوں سے  مستطیل  مکعب خلیوں میں تقسیم کیا جاتا ہے۔
}
\end{subfigure}\hfill
\begin{subfigure}{0.45\textwidth}
\centering
\begin{tikzpicture}[x={(-0.5cm,-0.7071cm)},y={(1cm,0)},z={(0,1cm)}]
\pgfmathsetmacro{\kkx}{2}
\pgfmathsetmacro{\kky}{3}
\pgfmathsetmacro{\kkz}{2}
\pgfmathsetmacro{\kx}{\kkx/3}
\pgfmathsetmacro{\ky}{\kky/4}
\pgfmathsetmacro{\kz}{\kkz/3}
\draw[](\kx,0,0)--++(0,\ky,0)--++(0,0,\kz)--++(0,-\ky,0)--++(0,0,-\kz);
\draw(0,0,\kz)--++(\kx,0);
\draw(0,\ky,\kz)--++(\kx,0);
\draw(0,\ky,0)--++(\kx,0);
\draw(0,0,\kz)--++(0,\ky,0)--++(0,0,-\kz);
\draw[stealth-stealth](\kx,0,-0.2)--++(0,\ky,0)node[below,pos=0.5]{$\Delta y$};
\draw[stealth-stealth](\kx,-0.2,0)--++(0,0,\kz)node[left,pos=0.5]{$\Delta z$};
\draw[stealth-stealth](0,\ky+0.2,0)--++(\kx,0,0)node[right,pos=0.5]{$\Delta x$};
\draw(\kx,0.8*\ky,0.7*\kz)node[circ]{}node[pin=135:{$(x_k,y_k)$}]{};
\draw(0,\ky,\kz)node[above right]{$\Delta H_k$};
\end{tikzpicture}
\caption{
ایک مستطیل مکعب خلیہ کا حجم۔
}
\end{subfigure}
\caption{ٹھوس جسم کو \عددی{\Delta H_k} حجم کے مستطیل خانوں میں تقسیم کیا جاتا ہے۔}
\label{شکل_بالکثرت_تہرا_تکمل_مستطیل_خانے}
\end{figure}
\جزوحصہء{تہرا تکملات کے خواص}
تہرا تکملات کے خواص وہی ہیں جو واحد تکملات اور دوہرا تکملات کے ہیں۔ اگر \عددی{F=F(x,y,z)} اور \عددی{G=G(x,y,z)} استمراری ہوں، تب
\begin{enumerate}[1.]
\item
$\iiint\limits_D kF\dif H=k\iiint\limits_D F\dif H$\quad \text{\RL{(\عددی{k} کوئی عدد ہے)}}
\item
$\iiint\limits_D (F\mp G)\dif H=\iiint\limits_D F\dif H\mp \iiint\limits_D G\dif H$
\item
$\iiint\limits_D F\dif H\ge 0\quad \text{\RL{اگر \عددی{D} پر \عددی{F\ge 0} تب}}$
\item
$\iiint\limits_D F\dif H\ge \iiint\limits_D G\dif H\quad \text{\RL{اگر \عددی{D} پر \عددی{F\ge D} ہو تب}}$
\item
تہرا تکملات مجموعیت کی  خاصیت بھی رکھتے ہیں جو طبیعیات،  انجینئری اور ریاضیات  کے میدان میں کام آتی ہے۔ اگر استمراری تفاعل \عددی{F} کے دائرہ کار \عددی{D} کو  ہموار سطحوں سے  متناہی تعداد کے   علیحدہ علیحدہ ٹکڑوں   \عددی{D_1}، \عددی{D_2}، \نقطے، \عددی{D_n} میں تقسیم کیا جائے تب درج ذیل ہو گا۔\\
$\iiint\limits_D F\dif H=\iiint\limits_{D_1}F\dif H+\iiint\limits_{D_2}F\dif H+\cdots+\iiint\limits_{D_n}F\dif H$
\end{enumerate}

\جزوحصہء{فضا میں خطے کا حجم}
اگر \عددی{F} ایک مستقل تفاعل ہو جس کی قیمت \عددی{1} ہو تب مساوات \حوالہ{مساوات_بالکثرت_تہرا_مجموعہ_الف} کے تہرا مجموعہ کی تخفیف صورت درج ذیل ہو گی۔
\begin{align}
J_n=\sum F(x_k,y_k,z_k)\Delta H_k=\sum 1\cdot \Delta H_k=\sum \Delta H_k
\end{align}

جوں جوں \عددی{\Delta x_k}، \عددی{\Delta y_k} اور \عددی{\Delta z_k} صفر تک پہنچتے ہیں  توں توں  \عددی{\Delta H_k}  جسامت میں چھوٹے  اور تعداد میں زیادہ ہوتے جاتے ہیں اور \عددی{D} کے زیادہ حصہ کو بھرتے ہیں۔اسی لئے ہم \عددی{D} کے حجم  کی تعریف درج ذیل لیتے ہیں۔
\begin{align*}
\lim\limits_{n\to\infty}\sum_{k=1}^{n}\Delta H_k=\iiint\limits_D \dif H
\end{align*}

\ابتدا{تعریف}
فضا میں بند  محدود خطہ \عددی{D}  کا \اصطلاح{حجم}\فرہنگ{حجم}\حاشیہب{volume}\فرہنگ{volume} درج ذیل ہو گا۔
\begin{align}
H=\iiint\limits_D \dif H
\end{align}
\انتہا{تعریف}
%=================

جیسا ہم جلد دیکھیں گے،   قوسی سطحوں میں  ملفوف  ٹھوس اجسام کا حجم اس تکمل سے حاصل  کیا جاتا ہے۔

\جزوحصہء{تہرا تکمل کی قیمت کا حصول}
ہم تہرا تکمل کی تعریف سے اس کی قیمت شاذ و نادر حاصل کرتے ہیں۔ اس کی بجائے ہم مسئلہ فوبینی کی  تین بعدی روپ استعمال کرتے ہوئے تین بار ایک گنّا تکملات سے اس کی قیمت معلوم کرتے ہیں۔ دہرا تکمل کی طرح، تکمل کے حدیں  معلوم کرنے کا جیومیٹریائی طریقہ کار پایا جاتا  ہے۔
 \begin{figure}
\centering
\begin{subfigure}{0.30\textwidth}
\centering
\begin{tikzpicture}[font=\scriptsize]
\pgfmathsetmacro{\angA}{-170}
\pgfmathsetmacro{\angB}{100}
\pgfmathsetmacro{\angC}{-10}
\pgfmathsetmacro{\angD}{-135}
\pgfmathsetmacro{\angE}{-100}
\pgfmathsetmacro{\angF}{-40}
\draw[-latex,name path=kx](0,0)--++(-1.5,-1.5)node[right]{$x$};
\draw[-latex](0,0)--++(2.5,0)node[right]{$y$};
\draw[-latex](0,0)--++(0,2)node[left]{$z$};
\draw[fill=lgray,draw=black,text=black,opacity=0.5](2.25,-0.25)coordinate(kA) to [out=\angA,in=\angB]node[pos=0.75,above left,yshift=-1ex,font=\scriptsize]{$y=g_1(x)$}++(-1.5,-0.75)coordinate(kB)node[above right,xshift=1ex,yshift=0.5ex]{$R$} to [out=\angC,in=\angD]node[pos=0.5,right]{$y=g_2(x)$}(kA);
\path[dashed](kA)--++(0,2.25)coordinate(kC)coordinate[pos=0.5](kE);
\path[dashed](kB)--++(0,3)coordinate(kD)coordinate[pos=0.5](kF);
\draw[dashed](kA)--(kE)  (kB)--(kF);
\draw(kC)--(kE)  (kD)--(kF);
\draw[dashed](kE) to [out=\angA,in=\angB](kF);
\draw(kF) to [out=\angC,in=\angD](kE); 
\draw(kE) to [out=\angE,in=\angF]coordinate[pos=0.25](kLower)(kF);
\draw(kC) to [out=110,in=60]coordinate[pos=0.25](kUpper)(kD);
\draw[](kC) to [out=-\angA,in=-\angB](kD);
\draw[dashed](kD) to [out=-\angC,in=-\angD](kC);
\draw(kLower)node[pin={[right,pin distance=0.25cm,font=\scriptsize]10:{$z=f_1(x,y)$}}]{};
\draw(kUpper)node[pin={[above]35:{$z=f_2(x,y)$}}]{};
\draw($(kC)!0.5!(kE)$)node[right]{$D$};
\path[name path=kLa](kA)--++(-3.5,0);
\path[name path=kLb](kB)--++(-2.5,0);
\draw[dashed,name intersections={of=kx and kLa}] (kA)--(intersection-1)node[left]{$a$};
\draw[dashed,name intersections={of=kx and kLb}] (kB)--(intersection-1)node[left]{$b$};
\end{tikzpicture}
\caption{}
\end{subfigure}\hfill
\begin{subfigure}{0.30\textwidth}
\centering
\begin{tikzpicture}[font=\scriptsize]
\pgfmathsetmacro{\angA}{-170}
\pgfmathsetmacro{\angB}{100}
\pgfmathsetmacro{\angC}{-10}
\pgfmathsetmacro{\angD}{-135}
\pgfmathsetmacro{\angE}{-100}
\pgfmathsetmacro{\angF}{-40}
\draw[-latex,name path=kx](0,0)--++(-1.5,-1.5)node[right]{$x$};
\draw[-latex](0,0)--++(2.5,0)node[right]{$y$};
\draw[-latex](0,0)--++(0,2)node[left]{$z$};
\draw[fill=lgray,draw=black,text=black,opacity=0.5](2.25,-0.25)coordinate(kA) to [out=\angA,in=\angB]node[pos=0.5, left,xshift=-2ex,yshift=-1ex,font=\scriptsize]{$y=g_1(x)$}++(-1.5,-0.75)coordinate(kB)node[above right,xshift=1ex,yshift=0.5ex]{$R$} to [out=\angC,in=\angD]node[pos=0.75,right,font=\scriptsize]{$y=g_2(x)$}(kA);
\path[dashed](kA)--++(0,2.25)coordinate(kC)coordinate[pos=0.5](kE);
\path[dashed](kB)--++(0,3)coordinate(kD)coordinate[pos=0.5](kF);
%\draw[dashed](kA)--(kE)  (kB)--(kF);
\draw(kC)--(kE)  (kD)--(kF);
\draw[dashed](kE) to [out=\angA,in=\angB](kF);
\draw(kF) to [out=\angC,in=\angD](kE); 
\draw(kE) to [out=\angE,in=\angF]coordinate[pos=0.25](kLower)(kF);
\draw(kC) to [out=110,in=60]coordinate[pos=0.25](kUpper)(kD);
\draw[](kC) to [out=-\angA,in=-\angB](kD);
\draw[dashed](kD) to [out=-\angC,in=-\angD](kC);
\draw($(kC)!0.5!(kE)$)node[right]{$D$};
\path[name path=kLa](kA)--++(-3.5,0);
\path[name path=kLb](kB)--++(-2.5,0);
\draw[dashed,name intersections={of=kx and kLa}] (kA)--(intersection-1)node[left]{$a$};
\draw[dashed,name intersections={of=kx and kLb}] (kB)--(intersection-1)node[left]{$b$};
\draw($(kA)!0.5!(kB)$)node[circ]{}node[pin={[pin distance=0.25cm]-45:{$(x,y)$}}]{}--($(kE)!0.5!(kF)+(0,-0.3)$)coordinate(kMm)node[circ]{};
\draw[dashed](kMm)node[pin={[right,pin edge={-,solid},align=center,pin distance=1cm]30:{\text{\RL{داخل}}\\ $z=f_1(x,y)$}}]{}--($(kC)!0.5!(kD)$)node[circ]{}node[pin={[align=center,pin edge={-,solid}]35:{\text{\RL{خارج}}\\ $z=f_2(x,y)$}}]{};
\draw[-latex]($(kC)!0.5!(kD)$)--++(0,1)node[left]{$M$};
\end{tikzpicture}
\caption{}
\end{subfigure}\hfill
\begin{subfigure}{0.30\textwidth}
\centering
\begin{tikzpicture}[font=\scriptsize]
\pgfmathsetmacro{\angA}{-170}
\pgfmathsetmacro{\angB}{100}
\pgfmathsetmacro{\angC}{-10}
\pgfmathsetmacro{\angD}{-135}
\pgfmathsetmacro{\angE}{-100}
\pgfmathsetmacro{\angF}{-40}
\draw[-latex,name path=kx](0,0)--++(-1.5,-1.5)node[right]{$x$};
\draw[-latex](0,0)--++(2.5,0)node[right]{$y$};
\draw[-latex](0,0)--++(0,2)node[left]{$z$};
\draw[name path=kR,fill=lgray,draw=black,text=black,opacity=0.5](2.25,-0.25)coordinate(kA) to [out=\angA,in=\angB]++(-1.5,-0.75)coordinate(kB)node[above right,xshift=1ex,yshift=-0.25ex]{$R$} to [out=\angC,in=\angD](kA);
\path[dashed](kA)--++(0,2.25)coordinate(kC)coordinate[pos=0.5](kE);
\path[dashed](kB)--++(0,3)coordinate(kD)coordinate[pos=0.5](kF);
\draw(kC)--(kE)  (kD)--(kF);
\draw[dashed](kE) to [out=\angA,in=\angB](kF);
\draw(kF) to [out=\angC,in=\angD](kE); 
\draw(kE) to [out=\angE,in=\angF]coordinate[pos=0.25](kLower)(kF);
\draw(kC) to [out=110,in=60]coordinate[pos=0.25](kUpper)(kD);
\draw[](kC) to [out=-\angA,in=-\angB](kD);
\draw[dashed](kD) to [out=-\angC,in=-\angD](kC);
\draw($(kC)!0.5!(kE)$)node[right]{$D$};
\path[name path=kLa](kA)--++(-3.5,0);
\path[name path=kLb](kB)--++(-2.5,0);
\draw[dashed,name intersections={of=kx and kLa}] (kA)--(intersection-1)node[left]{$a$};
\draw[dashed,name intersections={of=kx and kLb}] (kB)--(intersection-1)node[left]{$b$};
\draw($(kA)!0.5!(kB)$)coordinate(kLM)node[circ]{}node[pin=-120:{$(x,y)$}]{}--($(kE)!0.5!(kF)+(0,-0.3)$)coordinate(kMm)node[circ]{};
\draw[dashed](kMm)--($(kC)!0.5!(kD)$)node[circ]{};
\draw[-latex]($(kC)!0.5!(kD)$)--++(0,1)node[left]{$M$};
\path[name path=kL](kLM)--++(-3,0);
\draw[-latex,name intersections={of=kL and kx}](intersection-1)--++(3,0)node[right]{$L$};
\path[name path=kLeft](kLM)--(intersection-1);
\path[name path=kRight](kLM)--++(1.5,0);
\draw[name intersections={of=kLeft and kR}](intersection-1)node[pin={[align=center]-135:{\text{داخل}\\$y=g_1(x)$}}]{};
\draw[name intersections={of=kRight and kR}](intersection-1)node[pin={[align=center,pin distance=0.25cm,below]-70:{\text{خارج}\\  $y=g_2(x)$}}]{};
\end{tikzpicture}
\caption{}
\end{subfigure}
\caption{تہرا تکملات کی حدوں کی تلاش۔}
\label{شکل_بالکثرت_تہرا_تکمل_کی_حدوں_کی_تلاش}
\end{figure}

\موٹا{تہرا تکملات کی حدوں کی تلاش}\\
دائرہ کار \عددی{D} پر درج ذیل تکمل میں  پہلے  \عددی{z}، اس کے بعد \عددی{y} اور آخر میں \عددی{x}   کے لحاظ سے تکمل لیتے ہوئے  درج ذیل اقدام کرنے ہوں گے۔
\begin{align*}
\iiint\limits_D F(x,y,z)\dif H
\end{align*}
\begin{enumerate}[1.]
\item
\ترچھا{خاکہ:}\quad
خطہ \عددی{D} کا خاکہ بنائیں اور مستوی \عددی{xy} پر  اس کا انتصابی سایہ \عددی{R}  دکھائیں۔  خطہ \عددی{D} کی بالائی اور زیریں تحدیدی سطحوں کی نشاندہی کریں اور \عددی{R} کی بالائی اور زیریں تحدیدی منحنیات  کی نشاندہی کریں (شکل \حوالہ{شکل_بالکثرت_تہرا_تکمل_کی_حدوں_کی_تلاش}-ا)۔
\item
\ترچھا{تکمل کی \عددی{z} حدیں:}\quad
خطہ \عددی{R} میں  علامتی نقطہ \عددی{(x,y)} سے \عددی{z} محور کے متوازی  لکیر \عددی{M} کھینچیں۔  بڑھتے  \عددی{z} رک چلتے ہوئے،  یہ لکیر \عددی{z=f_1(x,y)} پر \عددی{D} میں داخل ہو گی اور  \عددی{z=f_2(x,y)} پر \عددی{D} سے خارج ہو گی۔یہی تکمل کی  \عددی{z} حدیں ہیں (شکل \حوالہ{شکل_بالکثرت_تہرا_تکمل_کی_حدوں_کی_تلاش}-ب)۔
\item
\ترچھا{تکمل کی \عددی{y} حدیں:}\quad
 نقطہ \عددی{(x,y)} سے گزرتی ہوئی  \عددی{y} محور کے متوازی  لکیر \عددی{L} کھینچیں۔ بڑھتے \عددی{y} رخ چلتے ہوئے  یہ لکیر \عددی{R} میں \عددی{y=g_1(x)} پر داخل اور \عددی{y=g_2(x)} پر خارج ہو گی۔ یہی تکمل کی \عددی{y} حدیں ہیں (شکل \حوالہ{شکل_بالکثرت_تہرا_تکمل_کی_حدوں_کی_تلاش}-ج)۔
\item
\ترچھا{تکمل کی \عددی{x} حدیں:}\quad
وہ \عددی{x} حدیں منتخب کریں جس میں  محور \عددی{y} کے متوازی،  \عددی{R} سے گزرتی ہوئی تمام لکیریں \عددی{L}   شامل ہوں۔ ہماری مثال میں  یہ حدیں  \عددی{x=a} اور \عددی{x=b} ہیں۔
\end{enumerate}
یوں تکمل درج ذیل ہو گا۔
\begin{align*}
\int_{x=a}^{x=b}\int_{y=g_1(x)}^{y=g_2(x)}\int_{z=f_1(x,y)}^{z=f_2(x,y)}F(x,y,z)\dif z\dif y\dif x
\end{align*}
تکملات کی ترتیب  تبدیل کرنے کی صورت میں اسی طرح کی طریقہ کار سے تکملات کی حدیں تلاش کریں۔بارہا تکمل میں آخری دو متغیرات،  جن کے لحاظ سے تکمل لیا  گیا ہو، کے مستوی میں \عددی{D} کا سایہ درکار ہو گا۔

\begin{figure}
\centering
\begin{tikzpicture}[font=\small]
\pgfmathsetmacro{\kx}{1.5}
\pgfmathsetmacro{\kxm}{0.4*\kx}
\pgfmathsetmacro{\kyL}{1.5}
\pgfmathsetmacro{\kyH}{1.5}
\pgfmathsetmacro{\angAL}{-70}
\pgfmathsetmacro{\angBL}{-110}
\pgfmathsetmacro{\angAH}{70}
\pgfmathsetmacro{\angBH}{110}
\pgfmathsetmacro{\angX}{-160}
\pgfmathsetmacro{\angY}{-30}
\draw(-\kx,0) to [out=\angAL,in=180](0,-\kyL) to [out=0,in=\angBL]coordinate[pos=0.75](kkL) (\kx,0);
\draw(-\kx,0) to [out=\angAH,in=180] (0,\kyH) to [out=0,in=\angBH]coordinate[pos=0.75](kkH)(\kx,0);
\draw(0,0) circle (\kx cm and \kxm cm);
\draw[name path=kc](0,-\kyL) circle (\kx cm and \kxm cm);
\draw(kkL)node[pin={[align=center]10:{\text{\RL{زیریں سطح}}\\$z=x^2+3y^2$}}]{};
\draw(kkH)node[pin={[right,align=center]45:{\text{\RL{بالائی سطح}}\\$z=8-x^2-y^2$}}]{};
\draw(120:\kx cm and \kxm cm)node[pin={[align=center,pin distance=1cm]150:{\text{\RL{منحنی تقاطع}}\\$x^2+2y^2=4$}}]{};
\draw(-110:\kx cm and \kxm cm)++(0,-\kyL)node[below right]{$R$}node[pin=-135:{$x^2+2y^2=4$}]{};
\draw(0,-\kyL-0.25)++(\angY:0.25)coordinate(N)node[circ]{}node[left,font=\scriptsize]{$(x,y)$};
\draw[-latex,name path=kxAxis](0,-\kyL)--++(\angX:2.5)node[left]{$x$};
\draw[name path=knegX](0,-\kyL)--++(\angX:-1.75);
\draw[-latex](0,-\kyL)--++(\angY:2)node[right]{$y$};
\path[name path=kLa](N)--++(\angY:-1.75);
\path[name path=kLb](N)--++(\angY:1.75);
\draw[-latex](N)++(\angY:-1.75)--++(\angY:3)node[below]{$L$};
\draw[name intersections={of={kc and kLa}}](intersection-1)node[pin={[align=center,pin distance=1cm]170:{\text{\RL{داخل}}\\$y=-\sqrt{(4-x^2)/2}$}}]{};
\draw[name intersections={of=kc and kLb}](intersection-1)node[pin={[align=center,pin distance=1cm]-100:{\text{\RL{خارج}}\\$y=\sqrt{(4-x^2)/2}$}}]{};
\draw(N)--++(0,1.15)coordinate(kEnter)node[circ]{};
\draw[dashed](N)++(0,1)--++(0,1.75);
\draw[-latex](N)++(0,1)++(0,1.75)node[circ]{}coordinate(kExit)--++(0,1)node[above]{$M$};
\draw(kExit)node[pin={[pin distance=1cm,align=center]45:{\text{\RL{خارج}}\\$8-x^2-y^2$}}]{};
\draw(kEnter)node[pin={[align=center,pin distance=3cm]165:{\text{\RL{داخل}}\\$z=x^2+3y^2$}}]{};
\draw[name intersections={of=kxAxis and kc}](intersection-1)node[left,xshift=-1ex,font=\scriptsize]{$(2,0,0)$};
\draw[name intersections={of=knegX and kc}](intersection-1)node[right,yshift=-0.5ex,xshift=1ex,font=\scriptsize]{$(-2,0,0)$};
\end{tikzpicture}
\caption{دو سطحوں کے بیچ حجم (مثال \حوالہ{مثال_بالکثرت_دو_سطحوں_کے_بیچ_خطہ})}
\label{شکل_مثال_بالکثرت_دو_سطحوں_کے_بیچ_خطہ}
\end{figure}

\ابتدا{مثال}\شناخت{مثال_بالکثرت_دو_سطحوں_کے_بیچ_خطہ}
خطہ \عددی{D} سطح \عددی{z=x^2+3y^2} اور سطح \عددی{z=8-x^2-y^2} میں  ملفوف ہے۔ اس کا حجم تلاش کریں۔

حل:\quad
ہم \عددی{F(x,y,z)=1} لیتے ہوئے حجم کے لئے درج ذیل تکمل لکھتے ہیں۔
\begin{align*}
H=\iiint\limits_D \dif z\dif y\dif x
\end{align*} 
ہم تکمل کی حدیں درج ذیل اقدام سے معلوم کرتے ہیں۔
\begin{enumerate}[1.]
\item
\ترچھا{خاکہ:}\quad
یہ سطحیں ایک دوسرے کو قطع مکافی \عددی{x^2+3y^2=8-x^2-y^2} یعنی \عددی{x^2+2y^2=4} میں قطع کرتی ہیں (شکل \حوالہ{شکل_مثال_بالکثرت_دو_سطحوں_کے_بیچ_خطہ})۔ مستوی \عددی{xy} میں  \عددی{D} کے سایہ \عددی{R}  کی سرحد کی مساوات
  یہی (\عددی{x^2+2y^2=4})  ہو گی۔ خطہ \عددی{R} کی بالائی سرحد منحنی \عددی{y=\sqrt{(4-x^2)/2}} اور اس کی زیریں سرحد منحنی \عددی{y=-\sqrt{(4-x^2)/2}} ہو گی۔ 
\item
\ترچھا{تکمل کی \عددی{z} حدیں:}\quad
خطہ \عددی{R} میں  علامتی نقطہ \عددی{(x,y)} سے گزرتی ہوئی   محور \عددی{z} کی  متوازی   لکیر \عددی{M}  خطہ \عددی{D} میں \عددی{z=x^2+3y^2} پر داخل اور \عددی{z=8-x^2-y^2} پر خارج ہوتی ہے۔
\item
\ترچھا{تکمل کی \عددی{y} حدیں:}\quad
 نقطہ \عددی{(x,y)} سے گزرتی ہوئی محور \عددی{y} کی متوازی  لکیر \عددی{L} خطہ  \عددی{R} میں  \عددی{y=-\sqrt{(4-x^2)/2}} پر داخل  اور  \عددی{y=\sqrt{(4-x^2)/2}} پر خارج ہوتی ہے۔
\item
\ترچھا{تکمل کی \عددی{x} حدیں:}\quad
   محور \عددی{y} کی متوازی لکیریں \عددی{L}  خطہ \عددی{R}   میں  \عددی{x=-2} (نقطہ (-2,0,0)) سے \عددی{x=2} (نقطہ (2,0,0))   تک گزرتی ہیں۔
\end{enumerate}
یوں حجم درج ذیل ہو گا۔
\begin{align*}
H&=\iiint\limits_D \dif z\dif y\dif x\\
&=\int_{-2}^{2}\int_{-\sqrt{(4-x^2)/2}}^{\sqrt{(4-x^2)/2}}\int_{x^2+3y^2}^{8-x^2-y^2}\dif z\dif y\dif x\\
&=\int_{-2}^{2}\int_{-\sqrt{(4-x^2)/2}}^{\sqrt{(4-x^2)/2}}(8-2x^2-4y^2)\dif y\dif x\\
&=\int_{-2}^{2}\big[(8-2x^2)y-\frac{4}{3}y^3\big]_{y=-\sqrt{(4-x^2)/2}}^{y=\sqrt{(4-x^2)/2}}\dif x\\
&=\int_{-2}^2\big(2(8-2x^2)]\sqrt{\frac{4-x^2}{2}}-\frac{8}{3}\big(\frac{4-x^2}{2}\big)^{3/2}\big)\dif x\\
&=\int_{-2}^2\left[8\big(\frac{4-x^2}{2}\big)^{3/2}-\frac{8}{3}\big(\frac{4-x^2}{2}\big)^{3/2}\right]\dif x\\
&=\frac{4\sqrt{2}}{3}\int_{-2}^2(4-x^2)^{3/2}\dif x\\
&=8\pi\sqrt{2}&&\text{\RL{\عددی{x=2\sin u} پر کر کے تکمل لیا گیا ہے}}
\end{align*}
\انتہا{مثال}
%================

اگلی مثال میں ہم  مستوی \عددی{xy} کی بجائے مستوی \عددی{xz} میں \عددی{D} کا سایہ لیتے ہیں۔

\ابتدا{مثال}\شناخت{مثال_بالکثرت_چو_سطحہ}
چو سطحہ  \عددی{D} کے راس \عددی{(0,0,0)}، \عددی{(1,1,0)}، \عددی{(0,1,0)} اور \عددی{(0,1,1)} ہیں۔ تفاعل \عددی{F(x,y,z)} کے تہرا تکمل کی حدیں معلوم کریں۔

حل:\quad
\begin{enumerate}[1.]
\item
\ترچھا{خطہ:}\quad
ہم \عددی{D} اور مستوی \عددی{xz} میں اس کے  سایہ \عددی{R} کا خاکہ بناتے ہیں (شکل \حوالہ{شکل_مثال_بالکثرت_چو_سطحہ})۔ خطہ \عددی{D} کی بالائی (دائیں ہاتھ)  تحدیدی سطح    مستوی \عددی{y=1} میں پائی جاتی ہے۔ اس کی زیریں (بائیں ہاتھ) تحدیدی سطح مستوی \عددی{y=x+z} میں پائی جاتی ہے۔ خطہ \عددی{R} کی بالائی  سرحد لکیر \عددی{z=1-x}  اور زیریں سرحد لکیر \عددی{z=0} ہیں۔
\item
\ترچھا{تکمل   کی \عددی{y} حدیں:}\quad
خطہ \عددی{R} میں علامتی نقطہ \عددی{(x,y)}  سے گزرتی لکیر  جو محور \عددی{y} کے متوازی ہو \عددی{D} میں \عددی{y=x+z} پر داخل اور \عددی{y=1} پر خارج ہوتی ہے۔
\item
\ترچھا{تکمل کی \عددی{z} حدیں:}\quad
محور \عددی{z} کے متوازی نقطہ \عددی{(x,y)} سے گزرتی لکیر \عددی{L} خطہ  \عددی{R} میں \عددی{z=0} پر داخل اور \عددی{z=1-x} پر خارج ہوتی ہے۔
\item
\ترچھا{تکمل کی \عددی{x} حدیں:}\quad
خطہ \عددی{R} میں  \عددی{x=0} سے \عددی{x=1} تک \عددی{} گزرتی ہیں۔
\end{enumerate}
یوں تکمل درج ذیل ہو گا۔
\begin{align*}
\int_0^1\int_0^{1-x}\int_{x+z}^1F(x,y,z)\dif y\dif z\dif x
\end{align*}
\انتہا{مثال}
%=================
\begin{figure}
\centering
\begin{tikzpicture}[font=\small]
\pgfmathsetmacro{\kx}{0.55}
\pgfmathsetmacro{\kz}{0.125}
\pgfmathsetmacro{\ky}{\kx+\kz}
\begin{axis}[clip=false,view/h=145,small,axis lines=middle,xtick={1},ytick={\empty},ztick={1},enlargelimits=true, xlabel={$x$}, ylabel={$y$},zlabel={$z$}, xlabel style={anchor=north},ylabel style={anchor=west},zlabel style={anchor=south},xmax=1.25,ymax=1.25,zmax=1.25,hide y axis]
\addplot3[]coordinates{(0,0,0)(1,1,0)(0,1,1)(0,0,0)};
\addplot3[]coordinates{(1,1,0)(0,1,0)(0,1,1)};
\addplot3[]coordinates{(1,0,0)(0,0,1)}node[pos=0.75,pin=120:{\text{\RL{لکیر \عددی{x+z=1}}}}]{}node[pos=0.8,below right]{$R$};
\addplot3[]coordinates{(0,1,1)}node[right]{$(0,1,1)$};
\addplot3[]coordinates{(1,1,0)}node[below]{$(1,1,0)$};
\addplot3[]coordinates{(0,1,0)}node[above right]{$(0,1,0)$};
\addplot3[dashed]coordinates{(0,0,0)(0,1,0)};
\addplot3[-latex]coordinates{(0,1,0)(0,1.5,0)}node[right]{$y$};
\addplot3[]coordinates{(0,0.65,0.5)}node[pin={[above]100:{$y=x+z$}}]{};
\addplot3[]coordinates{(\kx,0,\kz)(\kx,\ky,\kz)}node[pos=0,circ]{}node[pos=0,pin=145:{$(x,y)$}]{}node[pos=1,circ]{};
\addplot3[dashed]coordinates{(\kx,\ky,\kz)(\kx,1,\kz)}node[pos=0,pin={[align=center,pin distance=2cm,below,pin edge=solid]-160:{\text{داخل}\\$y=x+z$}}]{}node[pos=1,pin={[align=center,below,pin edge=solid]-60:{\text{خارج}\\$y=1$}}]{};
\addplot3[-latex]coordinates{(\kx,1,\kz)(\kx,1.75,\kz)}node[pos=0,circ]{}node[right]{$M$};
\addplot3[-latex]coordinates{(\kx,0,-0.25)(\kx,0,1)}node[left]{$L$};
%\addplot3[dashed]coordinates{(1,1,0)(1,0,0)};
%\addplot3[dashed]coordinates{(0,0,1)(0,1,1)};
\end{axis}
\end{tikzpicture}
\caption{چو سطحہ (مثال \حوالہ{مثال_بالکثرت_چو_سطحہ})}
\label{شکل_مثال_بالکثرت_چو_سطحہ}
\end{figure}
جیسا ہم جانتے ہیں، دہرا تکمل کا حصول عموماً (لیکن ضروری نہیں)    ایک گنّا تکملات کو دو مختلف  ترتیب سے حاصل کر کے حاصل کرنا ممکن ہوتا ہے۔ تہرا تکمل کے لئے اس طرح کے چھ  ترتیب ممکن ہو سکتے ہیں۔

\ابتدا{مثال}\شناخت{مثال_بالکثرت_چھ_اعادے}
درج  ذیل چھ تکملات شکل \حوالہ{شکل_مثال_بالکثرت_چھ_اعادے} میں دکھائے گئے منشور  کا حجم دیتے ہیں۔
\begin{align*}
&\int_0^1\int_0^{1-z}\int_0^2 \dif x\dif y\dif z&&\int_0^1\int_0^{1-y}\int_0^2\dif x\dif z\dif y\\
&\int_0^1\int_0^2\int_0^{1-z}\dif y\dif x\dif z&&\int_0^2\int_0^1\int_0^{1-z}\dif y\dif z\dif x\\
&\int_0^1\int_0^2\int_0^{1-y}\dif z\dif x\dif y&&\int_0^2\int_0^1\int_0^{1-y}\dif z\dif y\dif x
\end{align*}
\انتہا{مثال}
%==============

\جزوحصہء{فضا میں تفاعل کی اوسط قیمت}
فضا میں خطہ \عددی{D} پر تفاعل \عددی{F} کی  اوسط قیمت درج ذیل کلیہ دیتا ہے۔
\begin{align}\label{مساوات-بالکثرت_تین_بعدی_اوسط}
\text{\RL{\عددی{D} پر \عددی{F} کی \موٹا{ اوسط قیمت}}}=\frac{1}{\text{\RL{\عددی{D} کا حجم}}}\iiint\limits_D F\dif H
\end{align}
مثال کے طور پر اگر \عددی{F(x,y,z)=\sqrt{x^2+y^2+z^2}} ہو تب \عددی{D} پر \عددی{F} کی اوسط قیمت سے مراد مبدا سے  \عددی{D} میں نقطوں کا اوسط فاصلہ ہے۔ اگر \عددی{D} میں  \عددی{F(x,y,z)} ایک ٹھوس جسم کی کمیتی کثافت ہو تب \عددی{D} میں  \عددی{F} کی اوسط قیمت اس جسم کی اوسط کمیتی  کثافت ہو گی جس کی اکائی  کمیت فی اکائی حجم ہو گی۔
\begin{figure}
\centering
\begin{minipage}{0.45\textwidth}
\centering
\begin{tikzpicture}[font=\small]
\pgfmathsetmacro{\k}{sqrt(2)}
\begin{axis}[clip=false,view/h=135,small,axis lines=middle,xtick={2},ytick={1},ztick={1},enlargelimits=true, xlabel={$x$}, ylabel={$y$},zlabel={$z$}, xlabel style={anchor=north},ylabel style={anchor=west},zlabel style={anchor=east},xmax=2.2,ymax=1.5,zmax=1.5]
\addplot3[]coordinates{(2,0,0)(2,1,0)(2,0,1)(2,0,0)};
\addplot3[]coordinates{(2,0,1)(0,0,1)(0,1,0)(2,1,0)};
\addplot3[]coordinates{(0.5,0.6,0.5)}node[pin=45:{$y+z=1$}]{};
\end{axis}
\end{tikzpicture}
\caption{منشور کے حجم کی چھ بارہا تہرا تکملات مثال \حوالہ{مثال_بالکثرت_چھ_اعادے} میں دیے گئے ہیں۔}
\label{شکل_مثال_بالکثرت_چھ_اعادے}
\end{minipage}\hfill
\begin{minipage}{0.45\textwidth}
\centering
\begin{tikzpicture}[font=\small]
\begin{axis}[clip=false,view/h=145,small,axis lines=middle,xtick={2},ytick={2},ztick={2},enlargelimits=true, xlabel={$x$}, ylabel={$y$},zlabel={$z$}, xlabel style={anchor=north},ylabel style={anchor=west},zlabel style={anchor=south},xmax=2.5,ymax=2.5,zmax=2.5]
\addplot3[]coordinates{(2,0,0)(2,2,0)(2,2,2)(2,0,2)(2,0,0)};
\addplot3[]coordinates{(0,2,0)(0,2,2)(0,0,2)};
\addplot3[]coordinates{(2,0,2)(0,0,2)};
\addplot3[]coordinates{(2,2,2)(0,2,2)};
\addplot3[]coordinates{(2,2,0)(0,2,0)};
\end{axis}
\end{tikzpicture}
\caption{تکمل کا خطہ (مثال \حوالہ{مثال_بالکثرت_مکعب_حجم})}
\label{شکل_مثال_بالکثرت_مکعب_حجم}
\end{minipage}
\end{figure}
\ابتدا{مثال}\شناخت{مثال_بالکثرت_مکعب_حجم}
ثُمن  اول میں محددی مستویات  اور مستویات \عددی{x=2}،  \عددی{y=2} اور \عددی{z=2} کے بیچ   \عددی{F(x,y,z)=xyz} کی اوسط قیمت تلاش کریں۔

حل:\quad
ہم اس مکعب  کا خاکہ بنا کر اس پر  تکمل کی حدوں کی نشاندہی کرتے ہیں (شکل \حوالہ{شکل_مثال_بالکثرت_مکعب_حجم})۔ اس کے بعد مساوات \حوالہ{مساوات-بالکثرت_تین_بعدی_اوسط} سے مکعب  پر \عددی{F} کی   اوسط قیمت حاصل کرتے ہیں۔

مکعب کا حجم \عددی{(2)(2)(2)=8} ہو گا۔مکعب پر \عددی{F} کی قیمت درج ذیل ہو گا۔
\begin{align*}
\int_0^2\int_0^2\int_0^2 xyz\dif x\dif y\dif z&=\int_0^2\int_0^2\big[\frac{x^2}{2}yz\big]_{x=0}^{x=2}\dif y\dif z=\int_0^2\int_0^22yz\dif y\dif z\\
&=\int_0^2\big[y^2z\big]_{y=0}^{y=2}\dif z=\int_0^24z\dif z=\big[2z^2\big]_{0}^{2}=8
\end{align*}
ان قیمتوں کو استعمال کرتے ہوئے مساوات \حوالہ{مساوات-بالکثرت_تین_بعدی_اوسط} سے درج  ذیل اوسط قیمت حاصل ہو گی۔
\begin{align*}
\text{\RL{مکعب پر اوسط قیمت}}=\frac{1}{\text{حجم}}\iiint_{}xyz\dif H=\big(\frac{1}{8}\big)(8)=1
\end{align*}
ہم نے اس  تکمل کو  \عددی{\dif x}، \عددی{\dif y}، \عددی{\dif z} ترتیب سے حاصل کیا۔ ہم باقی پانچ ترتیب میں سے کسی ایک ترتیب کو استعمال کرتے ہوئے بھی اس تکمل کو حل کر سکتے ہیں۔
\انتہا{مثال}
%================


\جزوحصہء{سوالات}
\ابتدا{سوالات}
\موٹا{مختلف اعادوں سے تہرا تکمل کی قیمت کا   حصول}\\
\ابتدا{سوال}
چھ مختلف اعادوں سے مثال \حوالہ{مثال_بالکثرت_چھ_اعادے} میں حجم کا حل دیا گیا ہے۔ ان تمام کا مشترک جواب کیا ہے؟
\انتہا{سوال}
%=========
\ابتدا{جواب}
\wf{\unexpanded{
$1$
}}
\انتہا{جواب}
%=================
\ابتدا{سوال}
ثُمن  اول میں محددی مستویات اور  مستویات \عددی{x=1}، \عددی{y=2} اور \عددی{z=3} کے بیچ ٹھوس مستطیل  جسم کے حجم  کے چھ مختلف اعادہ   تہرا  تکملات لکھیں۔ ان میں سے ایک تکمل کی قیمت معلوم کریں۔
\انتہا{سوال}
%================
\ابتدا{سوال}
ثُمن  اول  سے مستوی \عددی{6x+3y+2z=6} ایک چو سطحہ کاٹتا ہے۔ اس کے حجم کے چھ مختلف اعادہ تہرا تکملات لکھیں۔ ان میں سے ایک تکمل کی قیمت حاصل کریں۔
\انتہا{سوال}
%===============
\ابتدا{جواب}
\wf{\unexpanded{
$\int_{0}^{1}\int_{0}^{2-2x}\int_{0}^{3-3x-3y/2}\dif z\dif y\dif x,$\\
$\int_{0}^{2}\int_{0}^{1-y/2}\int_{0}^{3-3x-3y/2}\dif z\dif x\dif y,$\\
$\int_{0}^{1}\int_{0}^{3-3x}\int_{0}^{2-2x-2z/3}\dif y\dif z\dif x,$\\
$\int_{0}^{3}\int_{0}^{1-z/3}\int_{0}^{2-2x-2z/3}\dif y\dif x\dif z,$\\
$\int_{0}^{2}\int_{0}^{3-3y/2}\int_{0}^{1-y/2-z/3}\dif x\dif z\dif y,$\\
$\int_{0}^{3}\int_{0}^{2-2z/3}\int_{0}^{1-y/2-z/3}\dif x\dif y\dif z$\quad
تمام  تکملات کا جواب  \عددی{1} ہے۔
}}
\انتہا{جواب}
%=================
\ابتدا{سوال}
ثُمن اول  سے بیلن \عددی{x^2+z^2=4} اور مستوی \عددی{y=3}  ایک خطہ کاٹتے ہیں۔ اس خطہ کے حجم کے چھ مختلف اعادہ تہرا ا تکملات  لکھیں۔ ان میں سے ایک تکمل کی قیمت تلاش کریں۔
\انتہا{سوال}
%==============
\ابتدا{سوال}
قطعات  مکافی \عددی{z=8-x^2-y^2} اور \عددی{z=x^2+y^2} میں محیط خطہ \عددی{D}  کے حجم کا چھ مختلف  تہرا اعادہ تکملات لکھیں۔ان میں سے ایک تکمل کی قیمت معلوم کریں۔
\انتہا{سوال}
%==========
\ابتدا{جواب}
\wf{\unexpanded{
$\int_{-2}^{2}\int_{-\sqrt{4-x^2}}^{\sqrt{4-x^2}}\int_{x^2+y^2}^{8-x^2-y^2}1 \dif z\dif y\dif x,$\\
$\int_{-2}^{2}\int_{-\sqrt{4-y^2}}^{\sqrt{4-y^2}}\int_{x^2+y^2}^{8-x^2-y^2} 1\dif z\dif x\dif y,$\\
$\int_{-2}^{2}\int_{4}^{8-y^2}\int_{-\sqrt{8-z-y^2}}^{\sqrt{8-z-y^2}}1\dif x\dif z\dif y+\int_{-2}^{2}\int_{y^2}^{4}\int_{-\sqrt{z-y^2}}^{\sqrt{z-y^2}}1\dif x\dif z\dif y,$\\
$\int_{4}^{8}\int_{-\sqrt{8-z}}^{\sqrt{8-z}}\int_{-\sqrt{8-z-y^2}}^{\sqrt{8-z-y^2}}1\dif x\dif y\dif z+\int_{0}^{4}\int_{-\sqrt{z}}^{\sqrt{z}}\int_{-\sqrt{z-y^2}}^{\sqrt{z-y^2}}1\dif x\dif y\dif z,$\\
$\int_{-2}^{2}\int_{4}^{8-x^2}\int_{-\sqrt{8-z-x^2}}^{\sqrt{8-z-x^2}}1\dif y\dif z\dif x+\int_{-2}^{2}\int_{x^2}^{4}\int_{-\sqrt{z-x^2}}^{\sqrt{z-x^2}}1\dif y\dif z\dif x,$\\
$\int_{4}^{8}\int_{-\sqrt{8-z}}^{\sqrt{8-z}}\int_{-\sqrt{8-z-x^2}}^{\sqrt{8-z-x^2}}1\dif y\dif x\dif z+\int_{0}^{4}\int_{-\sqrt{z}}^{\sqrt{z}}\int_{-\sqrt{z-x^2}}^{\sqrt{z-x^2}}1\dif y\dif x\dif zS$\quad
تمام تکملات کا جواب \عددی{16\pi} ہے۔
}}
\انتہا{جواب}
%=================
\ابتدا{سوال}
قطع مکافی \عددی{z=x^2+y^2} اور مستوی \عددی{z=2y}  میں  ملفوف خطہ \عددی{D} کے حجم کی تہرا اعادہ تکملات ترتیب  \عددی{\dif z\dif x\dif y} اور  \عددی{\dif z\dif y\dif x} میں لکھیں۔ ان میں سے کسی بھی تکمل کی قیمت حاصل نہ کریں۔
\انتہا{سوال}
%===================
\موٹا{تہرا اعادہ تکمل کی قیمت کی تلاش}\\
سوال \حوالہ{سوال_بالکثرت_تہرا_قیمت_الف} تا سوال \حوالہ{سوال_بالکثرت_تہرا_قیمت_ب} میں تکملات کی قیمتیں تلاش کریں۔

\ابتدا{سوال}\شناخت{سوال_بالکثرت_تہرا_قیمت_الف}
$\int_0^1\int_0^1\int_0^1 (x^2+y^2+z^2)\dif z\dif y\dif x$
\انتہا{سوال}
%====================
\ابتدا{جواب}
\wf{\unexpanded{
$1$
}}
\انتہا{جواب}
%=================
\ابتدا{سوال}
$\int_{0}^{\sqrt{2}}\int_{0}^{3y}\int_{x^2+3y^2}^{8-x^2-y^2}\dif z\dif x\dif y$
\انتہا{سوال}
%===================
\ابتدا{سوال}
$\int_{1}^{e}\int_{1}^{e}\int_{1}^{e}\tfrac{1}{xyz}\dif x\dif y\dif z$
\انتہا{سوال}
%===================
\ابتدا{جواب}
\wf{\unexpanded{
$1$
}}
\انتہا{جواب}
%=================
\ابتدا{سوال}
$\int_{0}^{1}\int_{0}^{3-3x}\int_{0}^{3-3x-y}\dif z\dif y\dif x$
\انتہا{سوال}
%===================
\ابتدا{سوال}
$\int_{0}^{1}\int_{0}^{\pi}\int_{0}^{\pi}y\sin z\dif x\dif y\dif z$
\انتہا{سوال}
%===================
\ابتدا{جواب}
\wf{\unexpanded{
$\tfrac{\pi^3}{2}(1-\cos 1)$
}}
\انتہا{جواب}
%=================
\ابتدا{سوال}
$\int_{-1}^{1}\int_{-1}^{1}\int_{-1}^{1}(x+y+z)\dif y\dif x\dif z$
\انتہا{سوال}
%===================
\ابتدا{سوال}
$\int_{0}^{3}\int_{0}^{\sqrt{9-x^2}}\int_{0}^{\sqrt{9-x^2}}\dif z\dif y\dif x$
\انتہا{سوال}
%===================
\ابتدا{جواب}
\wf{\unexpanded{
$18$
}}
\انتہا{جواب}
%=================
\ابتدا{سوال}
$\int_{0}^{2}\int_{-\sqrt{4-y^2}}^{\sqrt{4-x^2}}\int_{0}^{2x+y}\dif z\dif x\dif y$
\انتہا{سوال}
%===================
\ابتدا{سوال}
$\int_{0}^{1}\int_{0}^{2-x}\int_{0}^{2-x-y}\dif z\dif y\dif x$
\انتہا{سوال}
%===================
\ابتدا{جواب}
\wf{\unexpanded{
$\tfrac{7}{6}$
}}
\انتہا{جواب}
%=================
\ابتدا{سوال}
$\int_{0}^{1}\int_{0}^{1-x^2}\int_{3}^{4-x^2-y}x\dif z\dif y\dif x$
\انتہا{سوال}
%===================
\ابتدا{سوال}
$\int_{0}^{\pi}\int_{0}^{\pi}\int_{0}^{\pi}\cos(u+v+w)\dif u\dif v\dif w$\quad
(\عددی{uvw} فضا)
\انتہا{سوال}
%===================
\ابتدا{جواب}
\wf{\unexpanded{
$0$
}}
\انتہا{جواب}
%=================
\ابتدا{سوال}
$\int_{1}^{e}\int_{1}^{e}\int_{1}^{e}\ln r\ln s\ln t\dif t\dif r\dif s$\quad
(\عددی{rst} فضا)
\انتہا{سوال}
%===================
\ابتدا{سوال}
$\int_{0}^{\pi/4}\int_{0}^{\ln \sec v}\int_{-\infty}^{2t}e^x\dif x\dif t\dif v$\quad
(\عددی{tvx} فضا)
\انتہا{سوال}
%===================
\ابتدا{جواب}
\wf{\unexpanded{
$\tfrac{1}{2}-\tfrac{\pi}{8}$
}}
\انتہا{جواب}
%=================
\ابتدا{سوال}\شناخت{سوال_بالکثرت_تہرا_قیمت_ب}
$\int_{0}^{7}\int_{0}^{2}\int_{0}^{\sqrt{4-q^2}}\tfrac{q}{r+1}\dif p\dif q\dif r$\quad
(\عددی{pqr} فضا)
\انتہا{سوال}
%===================

\موٹا{حجم بذریعہ تہرا تکملات}\\
\ابتدا{سوال}\شناخت{سوال_بالکثرت_حجم_بالتکمل_الف}
درج  ذیل تکمل کا خطہ شکل \حوالہ{شکل_سوال_بالکثرت_حجم_بالتکمل_الف} میں  دکھایا گیا ہے۔
\begin{align*}
\int_{-1}^1\int_{x^2}^1\int_0^{1-y}\dif z\dif y\dif x
\end{align*}
اس تکمل کو درج ذیل  ترتیب کے اعادہ  معادل روپ میں لکھیں۔
\begin{multicols}{3}
\begin{enumerate}[a.]
\item
$\dif y\dif z\dif x$
\item
$\dif y\dif x\dif z$
\item
$\dif x\dif y\dif z$
\item
$\dif x\dif z\dif y$
\item
$\dif z\dif x\dif y$
\end{enumerate}
\end{multicols}
\انتہا{سوال}
%======
\ابتدا{جواب}
\wf{\unexpanded{
\begin{enumerate}[a.]
\item
$\int_{-1}^1\int_0^{1-x^2}\int_{x^2}^{1-z}\dif y\dif z\dif x$
\item
$\int_0^1\int_{-\sqrt{1-z}}^{\sqrt{1-z}}\int_{x^2}^{1-z}\dif y\dif x\dif z$
\item
$\int_0^1\int_0^{1-z}\int_{-\sqrt{y}}^{\sqrt{y}}\dif y\dif z\dif x$
\item
$\int_0^1\int_0^{1-y}\int_{-\sqrt{y}}^{\sqrt{y}}\dif x\dif z\dif y$
\item
$\int_0^1\int_{-\sqrt{y}}^{\sqrt{y}}\int_{0}^{1-y}\dif z\dif x\dif y$
\end{enumerate}
}}
\انتہا{جواب}
%=================

\begin{figure}
\centering
\begin{minipage}{0.45\textwidth}
\centering
\begin{tikzpicture}[font=\small,declare function={fx(\x,\y)=\x;fy(\x,\y)=\x^2;fz(\x,\y)=1-\x^2;}]
\pgfmathsetmacro{\kx}{0.35}
\pgfmathsetmacro{\ky}{\kx^2}
\pgfmathsetmacro{\kz}{1-\ky}
\begin{axis}[clip=false,view/h=145,small,axis lines=middle,xtick={1},ytick={1},ztick={1},enlargelimits=true, xlabel={$x$}, ylabel={$y$},zlabel={$z$}, xlabel style={anchor=north},ylabel style={anchor=west},zlabel style={anchor=south},colormap={}{gray(0cm)=(0.6);gray(1cm)=(0.9);},xmax=1.25,ymax=1.25,zmax=1.25]
\addplot3[black,domain=-1:1,domain y=0:1]({fx(x,y)},{fy(x,y)},{fz(x,y)});
\addplot3[domain=-1:1,samples y=0]({x},{x^2},{0})node[pos=0,right]{$(-1,1,0)$}node[pos=1,below]{$(1,1,0)$};
\addplot3[]coordinates{(\kx,\ky,0)(\kx,\ky,\kz)}node[pos=0.5,pin={[align=center]135:{\text{\RL{اطراف}}\\$y=x^2$}}]{};
\addplot3[]coordinates{(-0.4,0.5,0.5)}node[pos=0.8,pin={[align=center]45:{\text{\RL{بالائی سطح}}\\$y+z=1$}}]{};
\end{axis}
\end{tikzpicture}
\caption{خاکہ برائے سوال \حوالہ{سوال_بالکثرت_حجم_بالتکمل_الف}}
\label{شکل_سوال_بالکثرت_حجم_بالتکمل_الف}
\end{minipage}\hfill
\begin{minipage}{0.45\textwidth}
\centering
\begin{tikzpicture}[font=\small]
\begin{axis}[clip=false,view/h=145,small,axis lines=middle,xtick={1},ytick={1},ztick={\empty},enlargelimits=true, xlabel={$x$}, ylabel={$y$},zlabel={$z$}, xlabel style={anchor=north},ylabel style={anchor=west},zlabel style={anchor=south},colormap={}{gray(0cm)=(0.6);gray(1cm)=(0.9);},xmax=1.25,ymax=0.25,zmax=1.5]
\addplot3[black,domain y=-1:0](1,{y},{y^2});
\addplot3[black,domain y=-1:0](0,{y},{y^2});
\addplot3[]coordinates{(1,-1,0)(1,-1,1)(0,-1,1)(0,-1,0)(1,-1,0)(1,0,0)};
\addplot3[]coordinates{(1,-1,0)}node[left]{$(1,-1,0)$};
\addplot3[]coordinates{(1,-1,1)}node[left]{$(1,-1,1)$};
\addplot3[]coordinates{(0,-1,1)}node[above]{$(0,-1,1)$};
\addplot3[]coordinates{(0.3,-0.5,0.5)}node[pin=70:{$z=y^2$}]{};
\end{axis}
\end{tikzpicture}
\caption{خاکہ برائے سوال \حوالہ{سوال_بالکثرت_حجم_بالتکمل_ب}}
\label{شکل_سوال_بالکثرت_حجم_بالتکمل_ب}
\end{minipage}
\end{figure}
\ابتدا{سوال}\شناخت{سوال_بالکثرت_حجم_بالتکمل_ب}
درج ذیل تکمل  کا خطہ شکل \حوالہ{شکل_سوال_بالکثرت_حجم_بالتکمل_ب} میں  دکھایا گیا ہے۔
\begin{align*}
\int_0^1\int_{-1}^0\int_0^{y^2}\dif z\dif y\dif x
\end{align*}
اس تکمل کو درج ذیل  ترتیب کے اعادہ  معادل روپ میں لکھیں۔
\begin{multicols}{3}
\begin{enumerate}[a.]
\item
$\dif y\dif z\dif x$
\item
$\dif y\dif x\dif z$
\item
$\dif x\dif y\dif z$
\item
$\dif x\dif z\dif y$
\item
$\dif z\dif x\dif y$
\end{enumerate}
\end{multicols}
\انتہا{سوال}
%=============

\begin{figure}
\centering
\begin{minipage}{0.30\textwidth}
\centering
\begin{tikzpicture}[font=\small,declare function={fx(\x,\y)=\x;fy(\x,\y)=\y;fz(\x,\y)=(\y)^2;}]
\begin{axis}[clip=false,view/h=145,width=5cm,axis lines=middle,xtick={\empty},ytick={\empty},ztick={\empty},enlargelimits=true, xlabel={$x$}, ylabel={$y$},zlabel={$z$}, xlabel style={anchor=north},ylabel style={anchor=west},zlabel style={anchor=south},colormap={}{gray(0cm)=(0.6);gray(1cm)=(0.9);},xmax=1.25,ymax=1.25,zmax=0.85]
\addplot3[black,domain y=-1:1]({fx(0,y)},{fy(0,y)},{fz(0,y)});
\addplot3[black,domain y=-1:1]({fx(1,y)},{fy(1,y)},{fz(1,y)});
\addplot3[]coordinates{({fx(0,-1)},{fy(0,-1)},{fz(0,-1)})(0,-1,0)};
\addplot3[]coordinates{(0,1,0)({fx(0,1)},{fy(0,1)},{fz(0,1)})};
\addplot3[]coordinates{({fx(1,-1)},{fy(1,-1)},{fz(1,-1)})(1,-1,0)(1,1,0)({fx(1,1)},{fy(1,1)},{fz(1,1)})};
\addplot3[]coordinates{({fx(0,1)},{fy(0,1)},{fz(0,1)})({fx(1,1)},{fy(1,1)},{fz(1,1)})};
\addplot3[]coordinates{({fx(0,-1)},{fy(0,-1)},{fz(0,-1)})({fx(1,-1)},{fy(1,-1)},{fz(1,-1)})};
\addplot3[]coordinates{(1,-1,0)(0,-1,0)};
\addplot3[]coordinates{(1,1,0)(0,1,0)};
\end{axis}
\end{tikzpicture}
\caption{خاکہ برائے سوال \حوالہ{سوال_بالکثرت_خطہ_کا_حجم_الف}}
\label{شکل_سوال_بالکثرت_خطہ_کا_حجم_الف}
\end{minipage}\hfill
\begin{minipage}{0.30\textwidth}
\centering
\begin{tikzpicture}[font=\small]
\begin{axis}[axis equal,clip=false,view/h=145,width=5cm,axis lines=middle,xtick={\empty},ytick={\empty},ztick={\empty},enlargelimits=true, xlabel={$x$}, ylabel={$y$},zlabel={$z$}, xlabel style={anchor=north},ylabel style={anchor=west},zlabel style={anchor=south},colormap={}{gray(0cm)=(0.6);gray(1cm)=(0.9);},xmax=1.15,ymax=2.25,zmax=1.25,xmin=0]
\addplot3[]coordinates{(0,0,1)(1,0,0)(1,2,0)(0,0,1)};
\addplot3[]coordinates{(1,2,0)(0,2,0)(0,0,1)};
\end{axis}
\end{tikzpicture}
\caption{خاکہ برائے سوال \حوالہ{سوال_بالکثرت_خطہ_کا_حجم_درکار_پ}}
\label{شکل_سوال_بالکثرت_خطہ_کا_حجم_درکار_پ}
\end{minipage}\hfill
\begin{minipage}{0.30\textwidth}
\centering
\begin{tikzpicture}[font=\small,declare function={fx(\x,\y)=4-(\y)^2;fy(\x,\y)=\y;fz(\x,\y)=2-\y;}]
\begin{axis}[clip=false,view/h=145,width=5cm,axis lines=middle,xtick={\empty},ytick={\empty},ztick={\empty},enlargelimits=true, xlabel={$x$}, ylabel={$y$},zlabel={$z$}, xlabel style={anchor=north},ylabel style={anchor=west},zlabel style={anchor=south},colormap={}{gray(0cm)=(0.6);gray(1cm)=(0.9);}]
\addplot3[domain y=0:2]({fx(x,y)},{fy(x,y)},{fz(x,y)});
\addplot3[domain y=0:2]({fx(x,y)},{fy(x,y)},0);
\addplot3[]coordinates{({fx(0,0)},{fy(0,0)},{fz(0,0)})({fx(0,0)},{fy(0,0)},0)};
\addplot3[domain y=0:2]({0},{fy(0,y)},{fz(0,y)});
\addplot3[]coordinates{({fx(0,0)},{fy(0,0)},{fz(0,0)})({0},{0},{fz(0,0)})};
\end{axis}
\end{tikzpicture}
\caption{خاکہ برائے سوال \حوالہ{سوال_بالکثرت_خطہ_کا_حجم_درکار_ت}}
\label{شکل_سوال_بالکثرت_خطہ_کا_حجم_درکار_ت}
\end{minipage}
\end{figure}
سوال \حوالہ{سوال_بالکثرت_خطہ_کا_حجم_الف} تا سوال \حوالہ{سوال_بالکثرت_خطہ_کا_حجم_ب} میں خطوں کا حجم تلاش کریں۔

\ابتدا{سوال}\شناخت{سوال_بالکثرت_خطہ_کا_حجم_الف}
بیلن \عددی{z=y^2} اور مستوی \عددی{xy}  کے بیچ خطہ  جس کی سرحدیں  مستویات\عددی{x=0}، \عددی{x=1}، \عددی{y=-1} اور \عددی{y=1} ہیں (شکل \حوالہ{شکل_سوال_بالکثرت_خطہ_کا_حجم_الف})۔
\انتہا{سوال}
%================
\ابتدا{جواب}
\wf{\unexpanded{
$\tfrac{2}{3}$
}}
\انتہا{جواب}
%=================
\ابتدا{سوال}\شناخت{سوال_بالکثرت_خطہ_کا_حجم_درکار_پ}
ثُمن اول  میں محددی مستویات اور مستویات \عددی{x+z=1}، \عددی{y+2z=2}   کے بیچ خطہ (شکل \حوالہ{شکل_سوال_بالکثرت_خطہ_کا_حجم_درکار_پ})۔
\انتہا{سوال}
%=================
\ابتدا{سوال}\شناخت{سوال_بالکثرت_خطہ_کا_حجم_درکار_ت}
ثُمن اول  میں محددی مستویات اور مستوی \عددی{y+z=2} اور بیلن \عددی{x=4-y^2}   کے بیچ خطہ (شکل \حوالہ{شکل_سوال_بالکثرت_خطہ_کا_حجم_درکار_ت})۔
\انتہا{سوال}
%=================
\ابتدا{جواب}
\wf{\unexpanded{
$\tfrac{20}{3}$
}}
\انتہا{جواب}
%=================

\begin{figure}
\centering
\begin{minipage}{0.30\textwidth}
\centering
\begin{tikzpicture}[font=\small,declare function={fx(\x,\y)=sqrt(1-(\y)^2);fy(\x,\y)=\y;fz(\x,\y)=-\y;}]
\pgfmathsetmacro{\ky}{-0.9}
\begin{axis}[clip=false,view/h=135,width=5cm,axis lines=middle,xtick={\empty},ytick={\empty},ztick={\empty},enlargelimits=true, xlabel={$x$}, ylabel={$y$},zlabel={$z$}, xlabel style={anchor=north},ylabel style={anchor=west},zlabel style={anchor=south},colormap={}{gray(0cm)=(0.6);gray(1cm)=(0.9);},zmax=1]
\addplot3[domain y=0:-1,samples y=75]({fx(x,y)},{fy(x,y)},{fz(x,y)});
\addplot3[domain y=0:-1]({fx(x,y)},{fy(x,y)},0);
\addplot3[domain y=0:-1]({-fx(x,y)},{fy(x,y)},{fz(x,y)});
\addplot3[domain y=0:-1]({-fx(x,y)},{fy(x,y)},0);
\addplot3[]coordinates{({fx(0,\ky)},{fy(0,\ky)},{fz(0,\ky)})({fx(0,\ky)},{fy(0,\ky)},0)};
\end{axis}
\end{tikzpicture}
\caption{خاکہ برائے سوال \حوالہ{سوال_بالکثرت_خطہ_کا_حجم_درکار_ٹ}}
\label{شکل_سوال_بالکثرت_خطہ_کا_حجم_درکار_ٹ}
\end{minipage}\hfill
\begin{minipage}{0.3\textwidth}
\centering
\begin{tikzpicture}[font=\small,declare function={fx(\x,\y)=\y;fy(\x,\y)=\y;fz(\x,\y)=3-3*\x-3/2*\y;}]
\pgfmathsetmacro{\ky}{-0.9}
\begin{axis}[clip=false,view/h=135,width=5cm,axis lines=middle,xtick={\empty},ytick={\empty},ztick={\empty},enlargelimits=true, xlabel={$x$}, ylabel={$y$},zlabel={$z$}, xlabel style={anchor=north},ylabel style={anchor=west},zlabel style={anchor=south},colormap={}{gray(0cm)=(0.6);gray(1cm)=(0.9);}]
\addplot3[]coordinates{(1,0,0)(0,2,0)(0,0,3)(1,0,0)};
\end{axis}
\end{tikzpicture}
\caption{خاکہ برائے سوال \حوالہ{سوال_بالکثرت_خطہ_کا_حجم_درکار_ث}}
\label{شکل_سوال_بالکثرت_خطہ_کا_حجم_درکار_ث}
\end{minipage}\hfill
\begin{minipage}{0.3\textwidth}
\centering
\begin{tikzpicture}[font=\small,declare function={fx(\x)=\x;fy(\x)=1-\x;fz(\x)=cos(90*\x);}]
\begin{axis}[clip=false,view/h=135,width=5cm,axis lines=middle,xtick={\empty},ytick={\empty},ztick={\empty},enlargelimits=true, xlabel={$x$}, ylabel={$y$},zlabel={$z$}, xlabel style={anchor=north},ylabel style={anchor=west},zlabel style={anchor=south},colormap={}{gray(0cm)=(0.6);gray(1cm)=(0.9);}]
\addplot3[domain=0:1,samples y=0]({fx(x)},{fy(x)},{fz(x)});
\addplot3[domain=0:1,samples y=0]({fx(x)},{0},{fz(x)});
\addplot3[domain=0:1]({fx(x)},{fy(x)},0);
\addplot3[]coordinates{({0},{0},{fz(x)})(0,{fy(0)},{fz(0)})};
\addplot3[]coordinates{(0,{fy(0)},{fz(0)})(0,{fy(0)},0)};
\end{axis}
\end{tikzpicture}
\caption{خاکہ برائے سوال \حوالہ{سوال_بالکثرت_خطہ_کا_حجم_درکار_ج}}
\label{شکل_سوال_بالکثرت_خطہ_کا_حجم_درکار_ج}
\end{minipage}
\end{figure}

\ابتدا{سوال}\شناخت{سوال_بالکثرت_خطہ_کا_حجم_درکار_ٹ}
بیلن \عددی{x^2+y^2=1} سے مستویات \عددی{z=-y} اور \عددی{z=0} جو پچر  کاٹتے ہیں (شکل \حوالہ{شکل_سوال_بالکثرت_خطہ_کا_حجم_درکار_ٹ})۔
\انتہا{سوال}
%=================
\ابتدا{سوال}\شناخت{سوال_بالکثرت_خطہ_کا_حجم_درکار_ث}
ثُمن اول میں  محددی مستویات اور مستوی \عددی{x+\tfrac{y}{2}+\tfrac{z}{3}=1}  کے بیچ چو سطحہ  (شکل \حوالہ{شکل_سوال_بالکثرت_خطہ_کا_حجم_درکار_ث})۔
\انتہا{سوال}
%=================
\ابتدا{جواب}
\wf{\unexpanded{
$1$
}}
\انتہا{جواب}
%=================
\ابتدا{سوال}\شناخت{سوال_بالکثرت_خطہ_کا_حجم_درکار_ج}
ثُمن اول میں محددی مستویات، مستوی \عددی{y=1-x}  اور سطح \عددی{z=\cos(\pi x/2),\, 0\le x\le 1} کے بیچ خطہ  (شکل \حوالہ{شکل_سوال_بالکثرت_خطہ_کا_حجم_درکار_ج})۔
\انتہا{سوال}
%=================
\ابتدا{سوال}\شناخت{سوال_بالکثرت_خطہ_کا_حجم_درکار_چ}
بیلن \عددی{x^2+y^2=1} اور بیلن \عددی{x^2+z^2=1}  کا مشترک  اندرون (شکل \حوالہ{شکل_سوال_بالکثرت_خطہ_کا_حجم_درکار_چ})۔
\انتہا{سوال}
%=================
\ابتدا{جواب}
\wf{\unexpanded{
$\tfrac{16}{3}$
}}
\انتہا{جواب}
%=================
\begin{figure}
\centering
\begin{tikzpicture}[font=\small,declare function={fx(\x)=\x;fy(\x)=sqrt(1-(\x)^2);gx(\x)=\x;gz(\x)=sqrt(1-\x^2);}]
\begin{axis}[clip=false,view/h=135,small,axis lines=middle,xtick={\empty},ytick={\empty},ztick={\empty},enlargelimits=true, xlabel={$x$}, ylabel={$y$},zlabel={$z$}, xlabel style={anchor=north},ylabel style={anchor=west},zlabel style={anchor=south},colormap={}{gray(0cm)=(0.6);gray(1cm)=(0.9);}]
\addplot3[domain=0:1,samples y=0]({fx(x)},{fy(x)},0);
\addplot3[domain=0:1,samples y=0]({fx(x)},{fy(x)},{2});
\addplot3[]coordinates{(0,0,2)(1,0,2)(1,0,0)};
\addplot3[]coordinates{(0,0,2)(0,1,2)(0,1,0)};
\addplot3[domain=0:1,samples y=0]({gx(x)},0,{gz(x)});
\addplot3[domain=0:1,samples y=0]({gx(x)},3,{gz(x)});
\addplot3[]coordinates{(1,0,0)(1,3,0)(0,3,0)(0,3,1)(0,0,1)};
\addplot3[domain=0:1,samples y=0]({x},{sqrt(1-x^2)},{sqrt(1-x^2)});
\end{axis}
\end{tikzpicture}
\caption{خاکہ برائے سوال \حوالہ{سوال_بالکثرت_خطہ_کا_حجم_درکار_چ}}
\label{شکل_سوال_بالکثرت_خطہ_کا_حجم_درکار_چ}
\end{figure}
%=======================
\begin{figure}
\centering
\begin{minipage}{0.30\textwidth}
\centering
\begin{tikzpicture}[font=\small,declare function={fx(\x,\y)=\x;fy(\x,\y)=\y;fz(\x,\y)=4-\x^2-\y;}]
\begin{axis}[clip=false,view/h=135,width=5cm,axis lines=middle,xtick={\empty},ytick={\empty},ztick={\empty},enlargelimits=true, xlabel={$x$}, ylabel={$y$},zlabel={$z$}, xlabel style={anchor=north},ylabel style={anchor=west},zlabel style={anchor=south},colormap={}{gray(0cm)=(0.6);gray(1cm)=(0.9);}]
\addplot3[domain=0:2,samples y=0]({fx(x,0)},{fy(x,0)},{fz(x,0)});
\addplot3[domain=0:2,samples y=0]({x},{4-x^2},{0});
\addplot3[]coordinates{(0,4,0)(0,0,4)};
\end{axis}
\end{tikzpicture}
\caption{خاکہ برائے سوال \حوالہ{سوال_بالکثرت_خطہ_کا_حجم_درکار_ح}}
\label{شکل_سوال_بالکثرت_خطہ_کا_حجم_درکار_ح}
\end{minipage}\hfill
\begin{minipage}{0.30\textwidth}
\centering
\begin{tikzpicture}[font=\small,declare function={fx(\z)=4-sqrt(16-4*\z^2);fy(\z)=sqrt(16-4*\z^2);fz(\z)=\z;}]
\begin{axis}[clip=false,view/h=135,width=5cm,axis lines=middle,xtick={\empty},ytick={\empty},ztick={\empty},enlargelimits=true, xlabel={$x$}, ylabel={$y$},zlabel={$z$}, xlabel style={anchor=north},ylabel style={anchor=west},zlabel style={anchor=south},colormap={}{gray(0cm)=(0.6);gray(1cm)=(0.9);}]
\addplot3[domain=0:2,samples y=0,variable=\z]({fx(z)},{fy(z)},{fz(z)});
\addplot3[domain=0:2,samples y=0,variable=\z]({fx(z)},{fy(z)},{0});
\addplot3[domain=0:2,samples y=0,variable=\z]({0},{fy(z)},{fz(z)});
\addplot3[]coordinates{(0,0,2)(4,0,2)(4,0,0)};
\end{axis}
\end{tikzpicture}
\caption{خاکہ برائے سوال \حوالہ{سوال_بالکثرت_خطہ_کا_حجم_درکار_خ}}
\label{شکل_سوال_بالکثرت_خطہ_کا_حجم_درکار_خ}
\end{minipage}\hfill
\begin{minipage}{0.30\textwidth}
\centering
\begin{tikzpicture}[font=\small,declare function={fx(\x)=\x;fy(\x)=sqrt(4-(\x)^2);fz(\x)=3-\x;}]
\pgfmathsetmacro{\kxa}{1.5}
\pgfmathsetmacro{\kxb}{-1.5}
\begin{axis}[clip=false,view/h=135,width=5cm,axis lines=middle,xtick={\empty},ytick={\empty},ztick={\empty},enlargelimits=true, xlabel={$x$}, ylabel={$y$},zlabel={$z$}, xlabel style={anchor=north},ylabel style={anchor=west},zlabel style={anchor=south},colormap={}{gray(0cm)=(0.6);gray(1cm)=(0.9);},zmax=6]
\addplot3[domain=-2:2,samples y=0,smooth]({fx(x)},{fy(x)},{fz(x)});
\addplot3[domain=-2:2,samples y=0,smooth]({fx(x)},{-fy(x)},{fz(x)});
\addplot3[domain=-2:2,samples y=0,smooth]({fx(x)},{fy(x)},{0});
\addplot3[domain=-2:2,samples y=0,smooth]({fx(x)},{-fy(x)},{0});
\addplot3[]coordinates{(\kxa,{-fy(\kxa)},{fz(\kxa)})(\kxa,{-fy(\kxa)},0)};
\addplot3[]coordinates{(\kxb,{fy(\kxb)},{fz(\kxb)})(\kxb,{fy(\kxb)},0)};
\end{axis}
\end{tikzpicture}
\caption{خاکہ برائے سوال \حوالہ{سوال_بالکثرت_خطہ_کا_حجم_درکار_د}}
\label{شکل_سوال_بالکثرت_خطہ_کا_حجم_درکار_د}
\end{minipage}
\end{figure}
\ابتدا{سوال}\شناخت{سوال_بالکثرت_خطہ_کا_حجم_درکار_ح}
ثُمن اول میں محددی مستویات اور سطح \عددی{z=4-x^2-y} کے بیچ خطہ (شکل \حوالہ{شکل_سوال_بالکثرت_خطہ_کا_حجم_درکار_ح})۔
\انتہا{سوال}
%=================
\ابتدا{سوال}\شناخت{سوال_بالکثرت_خطہ_کا_حجم_درکار_خ}
ثُمن اول میں محددی مستویات، مستوی \عددی{x+y=4} اور بیلن \عددی{y^2+4z^2=16}  کے بیچ خطہ (شکل \حوالہ{شکل_سوال_بالکثرت_خطہ_کا_حجم_درکار_خ})۔
\انتہا{سوال}
%=================
\ابتدا{جواب}
\wf{\unexpanded{
$8\pi-\tfrac{32}{3}$
}}
\انتہا{جواب}
%=================
\ابتدا{سوال}\شناخت{سوال_بالکثرت_خطہ_کا_حجم_درکار_د}
بیلن \عددی{x^2+y^2=4} سے مستویات \عددی{z=0} اور \عددی{x+z=3} جو خطہ کاٹتے ہیں (شکل \حوالہ{شکل_سوال_بالکثرت_خطہ_کا_حجم_درکار_د})۔
\انتہا{سوال}
%=================
\ابتدا{سوال}
ثُمن اول میں مستویات \عددی{x+y+2z=2} اور \عددی{2x+2y+z=4} کے بیچ خطہ۔
\انتہا{سوال}
%=================
\ابتدا{جواب}
\wf{\unexpanded{
$2$
}}
\انتہا{جواب}
%=================
\ابتدا{سوال}
مستویات \عددی{z=x}، \عددی{x+z=8}، \عددی{z=y}،  \عددی{y=8}  اور \عددی{z=0} کے بیچ متناہی خطہ۔
\انتہا{سوال}
%=================
\ابتدا{سوال}
ٹھوس ترخیمی بیلن \عددی{x^2+4y^2\le 4}  سے \عددی{xy} مستوی اور مستوی \عددی{z=x+2} جو خطہ کاٹتے ہیں۔
\انتہا{سوال}
%=================
\ابتدا{جواب}
\wf{\unexpanded{
$4\pi$
}}
\انتہا{جواب}
%=================
\ابتدا{سوال}\شناخت{سوال_بالکثرت_خطہ_کا_حجم_ب}
وہ خطہ جس کا پشت مستوی \عددی{x=0}، سامنے اور اطراف  قطع مکافی بیلن \عددی{x=1-y^2}،  بالا  قطع مکافی  سطح  \عددی{z=x^2+y^2} اور نیچے  مستوی  \عددی{xy} ہوں۔
\انتہا{سوال}
%=================

\موٹا{اوسط قیمتیں}\\
سوال \حوالہ{سوال_بالکثرت_اوسط_تین_بعدی_الف} تا سوال \حوالہ{سوال_بالکثرت_اوسط_تین_بعدی_ب} میں دیے گئے خطہ پر \عددی{F(x,y,z)} کی اوسط قیمت تلاش کریں۔

\ابتدا{سوال}\شناخت{سوال_بالکثرت_اوسط_تین_بعدی_الف}
ثُمن اول میں محددی مستویات اور مستویات \عددی{x=2}، \عددی{y=2} اور \عددی{z=2} کے بیچ مکعب  خطہ اور  تفاعل \عددی{F(x,y,z)=x^2+9}لیں۔
\انتہا{سوال}
%====================
\ابتدا{جواب}
\wf{\unexpanded{
$\tfrac{31}{3}$
}}
\انتہا{جواب}
%=================
\ابتدا{سوال}
ثُمن اول میں محددی مستویات اور مستویات \عددی{x=1}، \عددی{y=1} اور \عددی{z=2} کے بیچ خطہ  اور  تفاعل \عددی{F(x,y,z)=x+y-z}لیں۔
\انتہا{سوال}
%=================
\ابتدا{سوال}
ثُمن اول میں محددی مستویات اور مستویات \عددی{x=1}، \عددی{y=1}اور \عددی{z=1} کے بیچ  خطہ  اور  تفاعل \عددی{F(x,y,z)=x^2+y^2+z^2}لیں۔
\انتہا{سوال}
%=================
\ابتدا{جواب}
\wf{\unexpanded{
$1$
}}
\انتہا{جواب}
%=================
\ابتدا{سوال}\شناخت{سوال_بالکثرت_اوسط_تین_بعدی_ب}
ثُمن اول میں محددی مستویات اور مستویات \عددی{x=2}، \عددی{y=2} اور \عددی{z=2} کے بیچ خطہ  اور  تفاعل \عددی{F(x,y,z)=xyz} لیں۔
\انتہا{سوال}
%=================

\موٹا{تکمل کی ترتیب بدلنا}\\
سوال \حوالہ{سوال_بالکثرت_ترتیب_بدل_کر_حل_الف} تا سوال \حوالہ{سوال_بالکثرت_ترتیب_بدل_کر_حل_ب} میں موزوں طریقہ سے تکمل کی ترتیب تبدیل کر کے تکمل کی قیمت تلاش کریں۔

\ابتدا{سوال}\شناخت{سوال_بالکثرت_ترتیب_بدل_کر_حل_الف}
$\int_0^4\int_0^1\int_{2y}^2 \frac{4\cos(x^2)}{2\sqrt{z}}\dif x\dif y\dif z$
\انتہا{سوال}
%=================
\ابتدا{جواب}
\wf{\unexpanded{
$2\sin 4$
}}
\انتہا{جواب}
%=================
\ابتدا{سوال}
$\int_{0}^{1}\int_{0}^{1}\int_{x^2}^{1}12xze^{zy^2}\dif y\dif x\dif z$
\انتہا{سوال}
%=================
\ابتدا{سوال}
$\int_{0}^{1}\int_{\sqrt[3]{z}}^{1}\int_{0}^{\ln 3}\frac{\pi e^{2x}\sin \pi y^2}{y^2}\dif x\dif y\dif z$
\انتہا{سوال}
%=================
\ابتدا{جواب}
\wf{\unexpanded{
$4$
}}
\انتہا{جواب}
%=================
\ابتدا{سوال}\شناخت{سوال_بالکثرت_ترتیب_بدل_کر_حل_ب}
$\int_{0}^{2}\int_{0}^{4-x^2}\int_{0}^{x}\frac{\sin 2z}{4-z}\dif y\dif z\dif x$
\انتہا{سوال}
%=================

\موٹا{نظریہ اور مثالیں}\\
\ابتدا{سوال}
درج ذیل کو \عددی{ a} کے لئے حل کریں۔
\begin{align*}
\int_0^1\int_0^{4-a-x^2}\int_a^{4-x^2-y}\dif z\dif y\dif x=\frac{4}{15}
\end{align*}
\انتہا{سوال}
%==========
\ابتدا{جواب}
\wf{\unexpanded{
\عددی{a=3} یا \عددی{a=\tfrac{13}{3}}
}}
\انتہا{جواب}
%=================
\ابتدا{سوال}
ترخیمی سطح \عددی{x^2+\tfrac{y^2}{4}+\tfrac{z^2}{c^2}=1} کا حجم  \عددی{c} کی کس قیمت کے لئے \عددی{8\pi} ہو گا؟
\انتہا{سوال}
%===================
\ابتدا{سوال}
فضا میں  کونسا دائرہ کار \عددی{D} درج ذیل تکمل کی قیمت کو کم سے کم بناتا ہے؟ اپنے جواب کی وجہ پیش کریں۔
\begin{align*}
\iiint\limits_D (4x^2+4y^2+z^2-4)\dif H
\end{align*}
\انتہا{سوال}
%=============
\ابتدا{سوال}
فضا میں کونسا دائرہ کار \عددی{D} درج ذیل تکمل کی قیمت کو زیادہ سے زیادہ بناتا ہے؟ اپنے جواب کی وجہ پیش کریں۔
\begin{align*}
\iiint\limits_D (1-x^2-y^2-z^2)\dif H
\end{align*}
\انتہا{سوال}
%==========
\موٹا{کمپیوٹر}\\
سوال \حوالہ{سوال_بالکثرت_کمپیوٹر_تہرا_الف} تا سوال \حوالہ{سوال_بالکثرت_کمپیوٹر_تہرا_ب} میں دیے گئے خطہ پر  تفاعل کا تہرا تکمل کمپیوٹر کی مدد سے حل کریں۔

\ابتدا{سوال}\شناخت{سوال_بالکثرت_کمپیوٹر_تہرا_الف}
مستویات \عددی{z=0} اور \عددی{z=1}  اور سطح \عددی{x^2+y^2=1} کے بیچ ٹھوس بیلن پر تفاعل \عددی{F(x,y,z)=x^2y^2z}  لیں۔
\انتہا{سوال}
%=================
\ابتدا{سوال}
ٹھوس خطہ جو نیچے سے   قطع مکافی سطح \عددی{z=x^2+y^2}  اور اوپر سے مستوی \عددی{z=1}  میں ملفوف ہو اور تفاعل \عددی{F(x,y,z)=\abs{xyz}} لیں۔
\انتہا{سوال}
%==================
\ابتدا{سوال}
ٹھوس خطہ جو نیچے سے مخروط \عددی{z=\sqrt{x^2+y^2}} اور اوپر سے مستوی \عددی{z=1} میں ملفوف ہو  اور تفاعل \عددی{ F(x,y,z)=\frac{z}{(x^2+y^2+z^2)^{3/2}}}لیں۔ 
\انتہا{سوال}
%==================
\ابتدا{سوال}\شناخت{سوال_بالکثرت_کمپیوٹر_تہرا_ب}
ٹھوس کرہ \عددی{x^2+y^2+z^2\le 1}  اور تفاعل \عددی{F(x,y,z)=x^4+y^2+z^2} لیں۔
\انتہا{سوال}
%==================
\انتہا{سوالات}

%%%%
\حصہ{تعین بعدی کمیت اور معیار اثر}
اس حصہ میں تین بعدی اجسام کی  کمیت اور معیار اثر کا حصول کارتیسی محدد میں سکھایا جائے گا۔ یہ کلیات دو بعدی  اجسام کے کلیات کی طرح ہیں۔ کروی اور نلکی محدد میں حساب کرنا  اگلے حصہ میں دکھایا جائے گا۔

\جزوحصہء{کمیت اور معیار اثر}
فضا میں خطہ \عددی{D} میں پائے جانے والے ایک جسم کی  کمیتی کثافت \عددی{\delta(x,y,z)} ہے۔ خطہ \عددی{D} پر \عددی{\delta} کا تکمل اس جسم کی کمیت دیگا۔ یہ دیکھنے کی خاطر کہ ایسا کیوں  کر ہو گا  ہم  اس جسم کو \عددی{n} ٹکڑوں  میں تقسیم کرتے ہیں (شکل \حوالہ{شکل_بالکثرت_ٹھوس_جسم_کے_ٹکڑے})۔ جسم کی کمیت درج ذیل حد ہو گی۔
\begin{align}\label{مساوات_بالکثرت_تین_بعدی_کمیت}
M=\lim_{n\to\infty}\sum_{k=1}^{n}\Delta m_k=\lim_{n\to\infty}\sum_{k=1}^{n}\delta(x_k,y_k,z_k)\Delta H_k=\iiint\limits_D \delta(x,y,z) \dif H
\end{align}

اگر \عددی{D} میں لکیر \عددی{L} سے نقطہ \عددی{(x,y,z)} کا فاصلہ \عددی{r(x,y,z)} ہو، تب  \عددی{L} کے لحاظ سے کمیت \عددی{\Delta m_k=\delta(x_,y_k,z_k)\Delta H_k} تقریباً \عددی{\Delta I_k=r^2(x_k,y_k,z_k)\Delta m_k} ہو گا۔ یوں \عددی{L} کے لحاظ سے پورے جسم کا معیار اثر درج ذیل ہو گا۔
\begin{align*}
I_L=\lim_{n\to\infty}\sum_{k=1}^{n}\Delta I_k=\lim_{n\to\infty}\sum_{k=1}^{n} r^2(x_k,y_k,z_k)\delta(x_k,y_k,z_k)\Delta H_k=\iiint\limits_D r^2\delta \dif H
\end{align*}
اگر \عددی{L} محور \عددی{x} ہو تب \عددی{r^2=y^2+z^2} ہو گا  اور
\begin{align*}
I_x=\iiint\limits_D (y^2+z^2)\delta \dif H
\end{align*}
ہو گا۔اسی طرح
\begin{align*}
I_z=\iiint\limits_D (x^2+y^2)\delta \dif H\quad  \text{اور}\quad I_y=\iiint\limits_D (x^2+z^2)\delta \dif H
\end{align*}
ہوں گے۔ان کلیات کو   دیگر کلیات کے ساتھ  یہاں  یکجا کیا گیا ہے۔
\begin{description}
\item{کمیت:}\quad
$M=\iiint\limits_D \delta \dif H$\quad
(\عددی{\delta}=کثافت)
\item{محددی مستویات کے لحاظ سے معیار اثر اول:}\\
$M_{yz}=\iiint\limits_D x\delta \dif H,\quad M_{xz}=\iiint\limits_D y\delta \dif H,\quad M_{xy}=\iiint\limits_D z\delta \dif H$
\item{مرکز کمیت:}\quad
$\bar{x}=\frac{M_{yz}}{M},\quad \bar{y}=\frac{M_{xz}}{M},\quad \bar{z}=\frac{M_{xy}}{M}$
\item{جمودی معیار اثر (معیار اثر دوم):}\quad
\begin{align*}
I_x&=\iiint(y^2+z^2)\delta \dif H\\
I_y&=\iiint(x^2+z^2)\delta \dif H\\
I_z&=\iiint(x^2+y^2)\delta \dif H
\end{align*}
\item{خط \عددی{L} کے لحاظ سے معیار اثر:}\quad
$I_L=\iiint r^2\delta \dif H$\quad
(جہاں  \عددی{L} سے نقطہ \عددی{(x,y,z)} کا فاصلہ \عددی{r(x,y,z)} ہے۔)
\item{خط \عددی{L} کے لحاظ سے رداس دوار:}\quad
$R_L=\sqrt{\frac{I_L}{M}}$
\end{description}

%================
\begin{figure}
\centering
\begin{tikzpicture}[font=\small]
\pgfmathsetmacro{\a}{0.1}
\pgfmathsetmacro{\b}{0.2}
\pgfmathsetmacro{\c}{0.2}
\pgfmathsetmacro{\kx}{1}
\pgfmathsetmacro{\ky}{2}
\pgfmathsetmacro{\kz}{1.5}
\begin{axis}[clip=false,view/h=110,small,axis lines=center,xlabel={$x$},ylabel={$y$},zlabel={$z$},xlabel style={anchor=east},ylabel style={anchor=west},zlabel style={anchor=east},enlargelimits=true,xtick={\empty},ytick={\empty},ztick={\empty}]
\addplot3[]coordinates{(\kx-\a,\ky+\b,\kz-\c)(\kx-\a,\ky+\b,\kz+\c)(\kx-\a,\ky-\b,\kz+\c)};
\addplot3[]coordinates{(\kx+\a,\ky-\b,\kz-\c)(\kx+\a,\ky+\b,\kz-\c)(\kx+\a,\ky+\b,\kz+\c)(\kx+\a,\ky-\b,\kz+\c)(\kx+\a,\ky-\b,\kz-\c)};
\addplot3[]coordinates{(\kx-\a,\ky-\b,\kz+\c)(\kx+\a,\ky-\b,\kz+\c)};
\addplot3[]coordinates{(\kx-\a,\ky+\b,\kz+\c)(\kx+\a,\ky+\b,\kz+\c)};
\addplot3[]coordinates{(\kx-\a,\ky+\b,\kz-\c)(\kx+\a,\ky+\b,\kz-\c)};
\addplot3[]coordinates{(\kx,0,0)(\kx,\ky,0)(\kx,\ky,\kz)(\kx,0,\kz)(\kx,0,0)};
\addplot3[]coordinates{(\kx,\ky,0)(0,\ky,0)};
\addplot3[]coordinates{(\kx,0,\kz)(0,0,\kz)}node[pos=0.5,left]{$x$};
\addplot3[]coordinates{(0,0,\kz)(\kx,\ky,\kz)}node[pos=0.5,above right]{$\sqrt{x^2+y^2}$};
\addplot3[]coordinates{(\kx,\ky,\kz)(0,\ky,0)}node[pos=0.5,above right]{$\sqrt{x^2+y^2}$};
\addplot3[]coordinates{(\kx,\ky,\kz)(\kx,0,0)}node[pos=0.55,sloped,below]{$\sqrt{y^2+z^2}$};
\RightAngle{(\kx,\ky,\kz)}{(\kx,0,\kz)}{(0,0,\kz)}
\addplot3[draw=none]coordinates{(\kx,0,0)(\kx,0,\kz)}node[pos=0.5,left]{$z$};
\addplot3[draw=none]coordinates{(\kx,0,0)(\kx,\ky,0)}node[pos=0.5,below]{$y$};
\addplot3[]coordinates{(\kx-\a,\ky+\b,\kz+\c)}node[above right]{$\Delta H$};
\end{axis}
\end{tikzpicture}
\caption{محددی محور اور محددی مستویات سے ایک ٹکڑے کے فاصلے۔}
\label{شکل_بالکثرت_ٹھوس_جسم_کے_ٹکڑے}
\end{figure}

\ابتدا{مثال}\شناخت{مثال_بالکثرت_تین_بعدی_ٹھوس_جسم_الف}
مستقل کثافت \عددی{\delta} کا مستطیل ٹھوس جسم  شکل \حوالہ{شکل_مثال_بالکثرت_تین_بعدی_ٹھوس_جسم_الف}  میں دکھایا گیا ہے۔ اس  کے \عددی{I_x}، \عددی{I_y} اور \عددی{I_z} دریافت کریں۔

حل:\quad
ہم مذکورہ بالا کلیات استعمال کرتے ہیں۔یوں
\begin{align}\label{مساوات_بالکثرت_جمودی_معیار_ایکس}
I_x&=\int_{-c/2}^{c/2}\int_{-b/2}^{b/2}\int_{-a/2}^{a/2} (y^2+z^2)\delta \dif x\dif y\dif z
\end{align}
ہو گا۔چونکہ \عددی{\delta (y^2+z^2)}  متغیرات \عددی{x}، \عددی{y} اور \عددی{z} کا جفت تفاعل ہے لہٰذا درج ذیل لکھا جا سکتا ہے۔
\begin{align*}
I_x&=8\int_{0}^{c/2}\int_{0}^{b/2}\int_{0}^{a/2} (y^2+z^2)\delta \dif x\dif y\dif z=4a\delta \int_{0}^{c/2}\int_{0}^{b/2}(y^2+z^2)\dif y\dif z\\
&=4a\delta \int_0^{c/2}\big[\frac{y^3}{3}+z^2y\big]_{y=0}^{y=b/2}\dif z\\
&=4a\delta \int_0^{c/2}\big(\frac{b^3}{24}+\frac{z^2b}{2}\big)\dif z\\
&=4a\delta \big(\frac{b^3c}{48}+\frac{c^3b}{48}\big)=\frac{abc\delta}{12}(b^2+c^2)=\frac{M}{12}(b^2+c^2)
\end{align*}
اسی طرح درج ذیل ہوں گے۔
\begin{align*}
I_z=\frac{M}{12}(a^2+b^2)\quad \text{اور}\quad I_y=\frac{M}{12}(a^2+c^2)
\end{align*}
\انتہا{مثال}
%=============
\begin{figure}
\centering
\begin{minipage}{0.45\textwidth}
\centering
\begin{tikzpicture}[font=\small]
\pgfmathsetmacro{\a}{1}
\pgfmathsetmacro{\b}{2}
\pgfmathsetmacro{\c}{1.5}
\pgfmathsetmacro{\kx}{0}
\pgfmathsetmacro{\ky}{0}
\pgfmathsetmacro{\kz}{0}
\begin{axis}[clip=false,view/h=110,small,axis lines=center,xlabel={$x$},ylabel={$y$},zlabel={$z$},xlabel style={anchor=north},ylabel style={anchor=west},zlabel style={anchor=east},enlargelimits=true,xtick={\empty},ytick={\empty},ztick={\empty},hide axis]
\addplot3[]coordinates{(\kx-\a,\ky+\b,\kz-\c)(\kx-\a,\ky+\b,\kz+\c)(\kx-\a,\ky-\b,\kz+\c)};
\addplot3[]coordinates{(\kx+\a,\ky-\b,\kz-\c)(\kx+\a,\ky+\b,\kz-\c)(\kx+\a,\ky+\b,\kz+\c)(\kx+\a,\ky-\b,\kz+\c)(\kx+\a,\ky-\b,\kz-\c)};
\addplot3[]coordinates{(\kx-\a,\ky-\b,\kz+\c)(\kx+\a,\ky-\b,\kz+\c)};
\addplot3[]coordinates{(\kx-\a,\ky+\b,\kz+\c)(\kx+\a,\ky+\b,\kz+\c)};
\addplot3[]coordinates{(\kx-\a,\ky+\b,\kz-\c)(\kx+\a,\ky+\b,\kz-\c)};
\addplot3[-latex]coordinates{(\a,0,0)(\a+3,0,0)}node[left]{$x$};
\addplot3[-latex]coordinates{(0,\b,0)(0,\b+2,0)}node[right]{$y$};
\addplot3[-latex]coordinates{(0,0,\c)(0,0,\c+2)}node[above]{$z$};
\addplot3[]coordinates{(\a,0,-\c)}node[below]{$b$};
\addplot3[]coordinates{(0,\b,-\c)}node[right]{$a$};
\addplot3[]coordinates{(-\a,\b,0)}node[right]{$c$};
\end{axis}
\end{tikzpicture}
\caption{ٹھوس جسم برائے مثال \حوالہ{مثال_بالکثرت_تین_بعدی_ٹھوس_جسم_الف}}
\label{شکل_مثال_بالکثرت_تین_بعدی_ٹھوس_جسم_الف}
\end{minipage}\hfill
\begin{minipage}{0.45\textwidth}
\centering
\begin{tikzpicture}[font=\small,declare function={fx(\r,\t)=\r*cos(\t);fy(\r,\t)=\r*sin(\t);fz(\t,\t)=4-(\r)^2;}]
\pgfmathsetmacro{\ra}{1.2}
\pgfmathsetmacro{\ta}{45}
\begin{axis}[clip=false,view/h=135,small,axis lines=center,xlabel={$x$},ylabel={$y$},zlabel={$z$},xlabel style={anchor=north},ylabel style={anchor=west},zlabel style={anchor=east},enlargelimits=true,xtick={\empty},ytick={\empty},ztick={\empty},colormap={}{gray(0cm)=(0.6);gray(1cm)=(0.9);}]
\addplot3[z buffer=sort,surf,domain=0:2,domain y=0:360,variable=\r,variable y=\t]({fx(r,t)},{fy(r,t)},{fz(r,t)});
\addplot3[domain y=0:360,variable=\r,variable y=\t]({fx(2,t)},{fy(2,t)},0);
\addplot3[]coordinates{(0,1,3)}node[pin=45:{$z=4-x^2-y^2$}]{};
\addplot3[]coordinates{(2,2,0)}node[pin=-45:{$x^2+y^2=4$}]{};
\end{axis}
\end{tikzpicture}
\caption{ٹھوس جسم برائے مثال \حوالہ{مثال_بالکثرت_تین_بعدی_ٹھوس_جسم_ب}}
\label{شکل_مثال_بالکثرت_تین_بعدی_ٹھوس_جسم_ب}
\end{minipage}
\end{figure}


\ابتدا{مثال}\شناخت{مثال_بالکثرت_تین_بعدی_ٹھوس_جسم_ب}
مستقل کثافت \عددی{\delta} کے جسم کی  نچلی سرحد مستوی \عددی{z=0} میں قرص \عددی{R:\, x^2+y^2\le 4} ہے جبکہ اس کی بالائی حد  قطع مکافی \عددی{z=4-x^2-y^2} ہے (شکل  \حوالہ{شکل_مثال_بالکثرت_تین_بعدی_ٹھوس_جسم_ب})۔ اس جس کا مرکز کمیت تلاش کریں۔

حل:\quad
تشاکلی کی بنا \عددی{\bar{x}=\bar{y}=0} ہو گا۔ ہمیں \عددی{\bar{z}}  معلوم  کرنے کے لئے پہلے درج ذیل دریافت  کرنے ہوں گے۔
\begin{align*}
M_{xy}&=\iint\limits_R\int_{z=0}^{z=4-x^2-y^2} z\delta \dif z\dif y\dif x=\iint\limits_R \big[\frac{z^2}{2}\big]_{z=0}^{z=4-x^2-y^2}\delta \dif y\dif x\\
&=\frac{\delta}{2}\iint\limits_R (4-x^2-y^2)^2\dif y\dif x\\
&=\frac{\delta}{2}\int_{0}^{2\pi}\int_{0}^{2}(4-r)^2r\dif r\dif\theta&&\text{\RL{قطبی محدد}}\\
&=\frac{\delta}{2}\int_0^{2\pi}\big[-\frac{1}{6}(4-r^2)^3\big]_{r=0}^{r=2}\dif\theta=\frac{16\delta}{3}\int_0^{2\pi}\dif\theta=\frac{32\pi\delta}{3}
\end{align*}
اسی طرح
\begin{align*}
M=\iint\limits_R\int_{0}^{4-x^2-y^2}\delta \dif z\dif y\dif x=8\pi\delta
\end{align*}
ہو گا۔یوں \عددی{\bar{z}=\tfrac{M_{xy}}{M}=\tfrac{4}{3}} اور مرکز کمیت \عددی{(\bar{x},\bar{y},\bar{z})=(0,0,4/3)} ہو گا۔
\انتہا{مثال}
%===========

جب جسم کی کثافت  اٹل ہو (جیسا مثال \حوالہ{مثال_بالکثرت_تین_بعدی_ٹھوس_جسم_الف}  اور مثال \حوالہ{مثال_بالکثرت_تین_بعدی_ٹھوس_جسم_ب} میں تھا)، تب (دو  بعدی اجسام کی طرح)   مرکز کمیت اس جسم کا \اصطلاح{وسطانی مرکز}\فرہنگ{مرکز!وسطانی}\حاشیہب{centroid}\فرہنگ{centroid} ہو گا۔

\جزوحصہء{سوالات}
\موٹا{مستقل کثافت}\\
سوال \حوالہ{سوال_بالکثرت_اکائی_کثافت_الف} تا سوال \حوالہ{سوال_بالکثرت_اکائی_کثافت_ب} میں کثافت \عددی{\delta=1} ہے۔

\ابتدا{سوال}\شناخت{سوال_بالکثرت_اکائی_کثافت_الف}
جمودی معیار اثر کی مساوات \حوالہ{مساوات_بالکثرت_جمودی_معیار_ایکس} کو سیدھا حل کر کے  مثال \حوالہ{مثال_بالکثرت_تین_بعدی_ٹھوس_جسم_الف} میں مستعمل چھوٹے    طریقہ کے نتیجہ کی تصدیق کریں۔   مثال \حوالہ{مثال_بالکثرت_تین_بعدی_ٹھوس_جسم_الف} کے نتائج استعمال کرتے ہوئے  تینوں محددی محوروں کے لحاظ سے اس جسم کے  رداس دوار تلاش کریں۔ 
\انتہا{سوال}
%=================
\ابتدا{سوال}\شناخت{سوال_بالکثرت_اکائی_کثافت_درکار_پ}
ایک پچر کے وسطانی مرکز  گزرتے محددی محور پچر کے کناروں کے متوازی ہیں  (شکل \حوالہ{شکل_سوال_بالکثرت_اکائی_کثافت_درکار_پ})۔اگر \عددی{a=b=6} اور \عددی{c=4} تب \عددی{I_x}، \عددی{I_y} اور \عددی{I_z} کیا ہوں گے۔
\انتہا{سوال}
%===================
\begin{figure}
\centering
\begin{minipage}{0.45\textwidth}
\centering
\begin{tikzpicture}[font=\small,]
\pgfmathsetmacro{\a}{1.5}
\pgfmathsetmacro{\b}{2}
\pgfmathsetmacro{\c}{1}
\begin{axis}[clip=false,axis lines=center,view/h=110,enlargelimits=true,xtick={\a},ytick={\b},ztick={\c},xticklabels={$a$},yticklabels={$b$},zticklabels={$c$},xlabel={$x$},ylabel={$y$},zlabel={$z$},xlabel style={anchor=east},ylabel style={anchor=west},zlabel style={anchor=south}, hide axis]
\addplot3[]coordinates{(1/2*\a,-1/3*\b,-1/3*\c)(1/2*\a,2/3*\b,-1/3*\c)(1/2*\a,-1/3*\b,2/3*\c)(1/2*\a,-1/3*\b,-1/3*\c)};
\addplot3[]coordinates{(1/2*\a,2/3*\b,-1/3*\c)(-1/2*\a,2/3*\b,-1/3*\c)(-1/2*\a,-1/3*\b,2/3*\c)(1/2*\a,-1/3*\b,2/3*\c)};
\addplot3[-latex]coordinates{(1/2*\a,0,0)(1.75*\a,0,0)}node[below]{$x$};
\addplot3[-latex]coordinates{(0,1/3*\b,0)(0,\b,0)}node[right]{$y$};
\addplot3[-latex]coordinates{(0,0,1/3*\c)(0,0,\c)}node[above]{$z$};
\addplot3[stealth-stealth]coordinates{(1/2*\a,0,-1/3*\c)(1/2*\a,0,0)}node[pos=0.5,right]{$\tfrac{c}{3}$};
\addplot3[stealth-stealth]coordinates{(1/2*\a,0,0)(1/2*\a,-1/3*\b,0)}node[pos=0.5,above]{$\tfrac{b}{3}$};
\addplot3[stealth-stealth]coordinates{(0,0,1/3*\c)(1/2*\a,0,1/3*\c)}node[pos=0.5,above left]{$\tfrac{a}{2}$};
\addplot3[]coordinates{(1/2*\a,1/3*\b,-1/3*\c)}node[below]{$b$};
\addplot3[]coordinates{(0,2/3*\b,-1/3*\c)}node[right]{$a$};
\addplot3[]coordinates{(1/2*\a,-1/3*\b,1/3*\c)}node[left]{$c$};
\addplot3[]coordinates{(-1/2*\a,1/3*\b,1/3*\c)}node[above,align=center]{\text{\RL{وسطانی مرکز}}\\  \عددی{(0,0,0)}};
\end{axis}
\end{tikzpicture}
\caption{پچر برائے سوال \حوالہ{سوال_بالکثرت_اکائی_کثافت_درکار_پ}}
\label{شکل_سوال_بالکثرت_اکائی_کثافت_درکار_پ}
\end{minipage}\hfill
\begin{minipage}{0.45\textwidth}
\centering
\begin{tikzpicture}[font=\small,]
\pgfmathsetmacro{\a}{1.5}
\pgfmathsetmacro{\b}{2}
\pgfmathsetmacro{\c}{1}
\begin{axis}[small,clip=false,axis lines=center,view/h=110,enlargelimits=true,xtick={\a},ytick={\b},ztick={\c},xticklabels={$a$},yticklabels={$\rlap{b}$},zticklabels={$c$},xlabel={$x$},ylabel={$y$},zlabel={$z$},xlabel style={anchor=east},ylabel style={anchor=west},zlabel style={anchor=south}]
\addplot3[]coordinates{(\a,0,0)(\a,\b,0)(\a,\b,\c)(\a,0,\c)(\a,0,0)};
\addplot3[]coordinates{(\a,\b,0)(0,\b,0)(0,\b,\c)(\a,\b,\c)};
\addplot3[]coordinates{(\a,0,\c)(0,0,\c)(0,\b,\c)(\a,\b,\c)};
\end{axis}
\end{tikzpicture}
\caption{مستطیل ٹھوس جسم برائے سوال \حوالہ{سوال_بالکثرت_اکائی_کثافت_درکار_ت}}
\label{شکل_سوال_بالکثرت_اکائی_کثافت_درکار_ت}
\end{minipage}
\end{figure}

\ابتدا{سوال}\شناخت{سوال_بالکثرت_اکائی_کثافت_درکار_ت}
مستطیل ٹھوس جسم کے \عددی{I_x}، \عددی{I_y} اور \عددی{I_z} دریافت کرتے ہوئے جسم کے کناروں کے لحاظ سے جمودی معیار اثر تلاش کریں (شکل \حوالہ{شکل_سوال_بالکثرت_اکائی_کثافت_درکار_ت})۔
\انتہا{سوال}
%=======================
\ابتدا{سوال}
(ا) ایک چو سطحہ جس کے راس \عددی{(0,0,0)}، \عددی{(1,0,0)}، \عددی{(0,1,0)  } اور \عددی{(0,0,1)}  ہیں کا وسطانی مرکز اور \عددی{I_x}، \عددی{I_y} اور \عددی{I_z} تلاش کریں۔ (ب) محور \عددی{x} کے لحاظ سے اس چو سطحہ کا رداس دوار معلوم کریں۔ محور \عددی{x} سے وسطانی مرکز  تک  فاصلہ کے ساتھ اس کا موازنہ کریں۔
\انتہا{سوال}
%=========================
\ابتدا{سوال}
مستقل کثافت کے ایک ٹھوس "کونڈا"  کی زیریں سرحدی سطح \عددی{z=4y^2}،  بالائی سرحدی سطح \عددی{z=4} اور اطراف  مستویات \عددی{x=1} اور \عددی{x=-1} ہیں۔ اس کی مرکز کمیت  اور تینوں محوروں کے لحاظ سے جمودی معیار اثر تلاش کریں۔
\انتہا{سوال}
%============================
\ابتدا{سوال}\شناخت{سوال_بالکثرت_چھ_سوال}
مستقل کثافت کے ایک ٹھوس جسم کی زیریں سرحد مستوی \عددی{z=0}، بالائی سرحد مستوی \عددی{z=2-x} اور اس کے اطراف ترخیمی بیلن \عددی{x^2+4y^2=4} ہے (شکل \حوالہ{شکل_سوال_بالکثرت_چھ_سوال})۔ (ا)  \عددی{\bar{x}} اور \عددی{\bar{y}} دریافت کریں۔ (ب)  درج ذیل تکمل کی قیمت حاصل کریں۔آخری تکمل میں \عددی{x} کے لحاظ سے تکمل لیتے ہوئے   آپ کو  تکملات کا جدول استعمال کرنا ہو گا۔
$M_{xy}=\int_{-2}^{2}\int_{-(1/2)\sqrt{4-x^2}}^{(1/2)\sqrt{4-x^2}}\int_{0}^{2-x}z\dif z\dif y\dif x$
اس کے بعد \عددی{M_{xy}} کو \عددی{M} سے تقسیم کر کے تصدیق کریں کہ \عددی{\bar{z}=\tfrac{5}{4}} ہو گا۔
\انتہا{سوال}
%======================
\ابتدا{سوال}
(ا) مستقل کثافت کے ایک ٹھوس جسم کی زیریں سرحد قطع مکافی \عددی{z=x^2+y^2} اور بالائی سرحد مستوی \عددی{z=4} ہے۔ اس جسم  کا مرکز کمیت تلاش کریں۔ (ب)  وہ مستوی \عددی{z=c} دریافت کریں جو اس جسم کو برابر حجم کے دو ٹکڑوں میں تقسیم کرتا ہو۔ یہ مستوی اس جسم کے  مرکز کمیت سے نہیں گزرتا ہے۔
\انتہا{سوال}
%====================
\ابتدا{سوال}
ایک ٹھوس مکعب کے اضلاع  کی لمبائیاں \عددی{2}  اکائیاں ہے۔ یہ مستویات \عددی{x=\mp1}، \عددی{z=\mp1}،  \عددی{y=3} اور \عددی{y=5}  کے بیچ واقع ہے۔ اس مکعب کا مرکز کمیت اور محددی محوروں کے لحاظ سے مکعب کے  رداس دوار تلاش کریں۔
\انتہا{سوال}
%=====================

\begin{figure}
\centering
\begin{minipage}{0.45\textwidth}
\centering
\begin{tikzpicture}[font=\small,declare function={fx(\r,\t)=2*\r*cos(\t);fy(\r,\t)=\r*sin(\t);fz(\r,\t)=2-2*\r*cos(\t);}]
\pgfmathsetmacro{\ta}{160}
\begin{axis}[small,clip=false,axis lines=center,view/h=160,enlargelimits=true,xlabel={$x$},ylabel={$y$},zlabel={$z$},xlabel style={anchor=east},ylabel style={anchor=west},zlabel style={anchor=south},xtick={\empty},ytick={\empty},ztick={\empty},hide axis]
\addplot3[smooth,domain y=0:360,variable y=\t] ({fx(1,t)},{fy(1,t)},{fz(1,t)})node[pos=0.75,pin=135:{$z=2-x$}]{};
\addplot3[smooth,domain y=0:360,variable y=\t] ({fx(1,t)},{fy(1,t)},0)node[pos=0.4,pin=-45:{$x^2+4y^2=4$}]{};
\addplot3[]coordinates{({fx(1,\ta)},{fy(1,\ta)},{fz(1,\ta)}) ({fx(1,\ta)},{fy(1,\ta)},0)};
\addplot3[dashed]coordinates{(0,0,0)(2,0,0)}node[circ]{}node[below]{$2$};
\addplot3[-latex]coordinates{(2,0,0)(2.5,0,0)}node[left]{$x$};
\addplot3[dashed]coordinates{(0,0,0)(0,1,0)}node[circ]{}node[below]{$1$};
\addplot3[-latex]coordinates{(0,1,0)(0,2,0)}node[right]{$y$};
\addplot3[dashed]coordinates{(0,0,0)(0,0,2)}node[circ]{}node[left]{$2$};
\addplot3[-latex]coordinates{(0,0,2)(0,0,4.5)}node[left]{$z$};
\end{axis}
\end{tikzpicture}
\caption{ٹھوس جسم برائے سوال \حوالہ{سوال_بالکثرت_چھ_سوال}}
\label{شکل_سوال_بالکثرت_چھ_سوال}
\end{minipage}\hfill
\begin{minipage}{0.45\textwidth}
\centering
\begin{tikzpicture}[font=\small,declare function={fx(\x)=\x;fy(\x)=sqrt(\x);fz(\x)=4-(\x)^2;}]
\pgfmathsetmacro{\xa}{0.4}
\pgfmathsetmacro{\xb}{0.1}
\begin{axis}[small,clip=false,axis lines=center,view/h=160,enlargelimits=true,xlabel={$x$},ylabel={$y$},zlabel={$z$},xlabel style={anchor=east},ylabel style={anchor=west},zlabel style={anchor=south},xtick={\empty},ytick={\empty},ztick={\empty}]
\addplot3[smooth,domain=0:\xa,samples y=0] ({fx(x)},{fy(x)},{fz(x)});
\addplot3[smooth,domain=0:\xa,samples y=0] ({fx(x)},{fy(x)},0);
\addplot3[smooth,domain=\xa:2,samples y=0] ({fx(x)},{fy(x)},{fz(x)});
\addplot3[smooth,domain=\xa:2,samples y=0] ({fx(x)},{fy(x)},0)node[pos=0.4,pin=-45:{$x=y^2$}]{};
\addplot3[smooth,domain=0:2,samples y=0] ({fx(x)},0,{fz(x)})node[pos=0.75,pin=135:{$z=4-x^2$}]{};
\addplot3[]coordinates{(2,0,0)(2,sqrt(2),0)}node[pos=0,below]{$2$}node[below]{$(2,\sqrt{2},0)$};
\addplot3[]coordinates{ ({fx(\xb)},{fy(\xb)},{fz(\xb)}) ({fx(\xb)},{fy(\xb)},0)};
\addplot3[]coordinates{(0,0,4)}node[right]{$4$};
\end{axis}
\end{tikzpicture}
\caption{ٹھوس جسم برائے سوال \حوالہ{سوال_بالکثرت_متغیر_کثافت_ب}}
\label{شکل_سوال_بالکثرت_متغیر_کثافت_ب}
\end{minipage}
\end{figure}

\ابتدا{سوال}
ایک پچر  کے \عددی{a=4}، \عددی{b=6} اور \عددی{c=3} ہیں     (سوال \حوالہ{سوال_بالکثرت_اکائی_کثافت_درکار_پ} دیکھیں)۔  اس کا خاکہ بنا کر تصدیق کریں کہ  پچر کے کسی علامتی  نقطہ \عددی{(x,y,z)}  سے  لکیر \عددی{L:\, y=6,\, z=0}  تک  فاصلے کا مربع  \عددی{ r^2=(y-6)^2+z^2} ہو گا۔ لکیر \عددی{L} کے لحاظ سے اس پچر کا جمودی معیار اثر اور رداس دوار معلوم کریں۔
\انتہا{سوال}
%================
\ابتدا{سوال}
ایک پچر  کے \عددی{a=4}، \عددی{b=6} اور \عددی{c=3} ہیں     (سوال \حوالہ{سوال_بالکثرت_اکائی_کثافت_درکار_پ} دیکھیں)۔  اس کا خاکہ بنا کر تصدیق کریں کہ  پچر کے کسی علامتی  نقطہ \عددی{(x,y,z)}  سے  لکیر \عددی{L:\, x=4,\, y=0}  تک  فاصلے کا مربع  \عددی{ r^2=(x-4)^2+y^2} ہو گا۔ لکیر \عددی{L} کے لحاظ سے اس پچر کا جمودی معیار اثر اور رداس دوار معلوم کریں۔
\انتہا{سوال}
%================
\ابتدا{سوال}
ایک مستطیل  ٹھوس جسم کے \عددی{a=4}، \عددی{b=2} اور \عددی{c=1} ہیں (سوال \حوالہ{سوال_بالکثرت_اکائی_کثافت_درکار_ت} دیکھیں)۔ اس جسم کا خاکہ بنا کر تصدیق کریں  کہ اس جسم کے کسی علامتی نقطہ \عددی{(x,y,z)} سے لکیر \عددی{L:\, y=2,\, z=0}  تک فاصلہ کا مربع \عددی{r^2=(y-2)^2+z^2} ہو گا۔ لکیر \عددی{L} کے لحاظ سے اس جسم کا جمودی معیار اثر اور رداس دوار تلاش کریں۔
\انتہا{سوال}
%============
\ابتدا{سوال}\شناخت{سوال_بالکثرت_اکائی_کثافت_ب}
ایک مستطیل  ٹھوس جسم کے \عددی{a=4}، \عددی{b=2} اور \عددی{c=1} ہیں (سوال \حوالہ{سوال_بالکثرت_اکائی_کثافت_درکار_ت} دیکھیں)۔  اس جسم کا خاکہ بنا کر تصدیق کریں  کہ اس جسم کے کسی علامتی نقطہ \عددی{(x,y,z)} سے لکیر \عددی{L:\, x=4,\, y=0}  تک فاصلہ کا مربع \عددی{r^2=(x-4)^2+y^2} ہو گا۔ لکیر \عددی{L} کے لحاظ سے اس جسم کا جمودی معیار اثر اور رداس دوار تلاش کریں۔
\انتہا{سوال}
%========================

\موٹا{متغیر کثافت}\\
سوال \حوالہ{سوال_بالکثرت_متغیر_کثافت_الف} اور سوال \حوالہ{سوال_بالکثرت_متغیر_کثافت_ب} میں (ا) جسم کی کمیت اور (ب)  اس کا مرکز کمیت تلاش کریں۔

\ابتدا{سوال}\شناخت{سوال_بالکثرت_متغیر_کثافت_الف}
ثُمن اول میں ایک ٹھوس جسم  جو محددی مستویات اور مستوی \عددی{x+y+z=2}     کے بیچ واقع ہے۔ اس جسم کی کثافت \عددی{\delta(x,y,z)=2x} ہے۔
\انتہا{سوال}
%=================
\ابتدا{سوال}\شناخت{سوال_بالکثرت_متغیر_کثافت_ب}
ثُمن اول میں مستویات \عددی{y=0} اور \عددی{z=0}  اور سطح \عددی{z=4-x^2} اور سطح \عددی{x=y^2} کے بیچ واقع  جسم کی کثافت \عددی{\delta(x,y,z)=kxy} ہے  جہاں \عددی{k} ایک مستقل ہے (شکل \حوالہ{شکل_سوال_بالکثرت_متغیر_کثافت_ب})۔
\انتہا{سوال}
%=======================
سوال \حوالہ{سوال_بالکثرت_ٹھوس_جسم_اقدام_الف} اور سوال \حوالہ{سوال_بالکثرت_ٹھوس_جسم_اقدام_ب} میں درج ذیل تلاش کریں۔
\begin{enumerate}[a.]
\item
اس جسم کی کمیت۔
\item
اس جسم کا مرکز کمیت۔
\item
محددی محوروں کے لحاظ سے جمودی معیار اثر۔
\item
محددی محوروں کے لحاظ سے رداس دور۔
\end{enumerate}

\ابتدا{سوال}\شناخت{سوال_بالکثرت_ٹھوس_جسم_اقدام_الف}
ثُمن اول میں محددی مستویات اور مستویات \عددی{x=1}،  \عددی{y=1} اور \عددی{z=1}  کے بیچ ٹھوس مکعب  جس کی کثافت \عددی{\delta(x,y,z)=x+y+z+1} ہے۔
\انتہا{سوال}
%==================
\ابتدا{سوال}\شناخت{سوال_بالکثرت_ٹھوس_جسم_اقدام_ب}
ایک مستطیل ٹھوس جسم جس کے \عددی{a=2}، \عددی{b=6} اور \عددی{c=3} ہیں (سوال \حوالہ{سوال_بالکثرت_اکائی_کثافت_درکار_پ} دیکھیں)  کی کثافت \عددی{\delta(x,y,z)=x+1} ہے۔ آپ دیکھ سکتے ہیں کہ مستقل کثافت کی صورت میں اس جسم کا مرکز  کمیت \عددی{(0,0,0)} ہو گا۔
\انتہا{سوال}
%===========
\ابتدا{سوال}
مستویات \عددی{x+z=1}، \عددی{x-z=-1}، \عددی{y=0} اور سطح \عددی{y=\sqrt{z}} کے بیچ واقع ٹھوس جسم جس کی کثافت \عددی{\delta(x,y,z)=2y+5} ہے۔
\انتہا{سوال}
%========================
\ابتدا{سوال}
قطع مکافی سطح \عددی{z=16-2x^2-2y^2}  اور \عددی{z=2x^2+2y^2} کے بیچ ٹھوس جسم کی کثافت \عددی{\delta(x,y,z)=\sqrt{x^2+y^2}} ہے۔ اس جسم کی کمیت تلاش کریں۔
\انتہا{سوال}
%==============
\موٹا{کام}\\
سوال \حوالہ{سوال_بالکثرت_کام_الف} اور  سوال \حوالہ{سوال_بالکثرت_کام_ب} میں درج ذیل معلوم کریں۔
\begin{enumerate}[a.]
\item
    مکمل بھرے ہوئے برتن سے   سیال  کو مستوی \عددی{xy} میں منتقل کرنے کے لئے مستقل  تجاذب  \عددی{g}  کتنا  کام کرے گا؟   (اشارہ: برتن میں سیال  کو چھوٹے چھوٹے حجم کے ٹکڑوں \عددی{\Delta H_k}  میں تقسیم کرتے ہوئے  ہر ٹکڑے کو منتقل کرنے کے لئے درکار کام دریافت کریں۔ ان تمام کا مجموعہ پورے سیال کو منتقل کرنے کا کام ہو گا۔ یہ مجموعہ، حد کی صورت میں، تہرا تکمل دیگا جس کی قیمت آپ کو معلوم کرنی ہو گی۔)
\item
مکمل بھرے ہوئے برتن میں  سیال کے مرکز کمیت کو مستوی \عددی{xy} میں منتقل کرنے کے لئے مستقل تجاذب \عددی{g} کتنا کام کرے گا؟
\end{enumerate}

\ابتدا{سوال}\شناخت{سوال_بالکثرت_کام_الف}
برتن ثُمن اول میں کعبی ڈبہ کی صورت کا ہے جو محددی مستویات اور مستویات \عددی{x=1}، \عددی{y=1} اور \عددی{z=1} کے بیچ پایا جاتا ہے۔ سیال کی کثافت \عددی{\delta(x,y,z)=x+y+z+1} ہے (سوال \حوالہ{سوال_بالکثرت_ٹھوس_جسم_اقدام_الف} دیکھیں)۔
\انتہا{سوال}
%====================
\ابتدا{سوال}\شناخت{سوال_بالکثرت_کام_ب}
مستویات \عددی{y=0}، \عددی{z=0} اور سطحوں \عددی{z=4-x^2}، \عددی{x=y^2} کے بیچ برتن پایا جاتا ہے۔ سیال کی کثافت \عددی{\delta(x,y,z)=kxy} ہے جہاں \عددی{k} ایک مستقل ہے (سوال \حوالہ{سوال_بالکثرت_متغیر_کثافت_ب} دیکھیں )۔
\انتہا{سوال}
%==========================

\موٹا{مسئلہ متوازی محور}\\
مسئلہ متوازی محور (حصہ \حوالہ{حصہ_بالکثرت_رقبات_معیار_اثر_مرکز_کمیت} کے سوالات دیکھیں)   دو بعدی صورت  کے ساتھ ساتھ تین بعدی صورت  کے لئے بھی کارآمد ہے۔فرض کریں ایک جسم جس کی کمیت \عددی{m} ہو کے مرکز کمیت سے خط \عددی{L_{c,m}} گزرتا ہو  جس کے متوازی  \عددی{h} فاصلہ پر خط \عددی{L} پایا جاتا ہو ۔ مسئلہ متوازی محور کہتا ہے کہ \عددی{L_{c,m}} اور \عددی{L} کے لحاظ سے اس جسم کے جمودی معیار اثر درج ذیل کلیہ کو مطمئن کرتے ہیں۔
\begin{align}\label{مساوات_بالکثرت_مسئلہ_متوازی_محور_الف}
I_L=I_{c,m}+mh^2
\end{align}
دو بعدی  صورت کی طرح اگر ہمیں ایک جمودی معیار اثر، فاصلہ \عددی{h} اور جسم کی کمیت \عددی{m} معلوم ہو تب   ہم اس  مسئلہ کی  مدد سے  دوسرا جمودی معیار اثر با آسانی  دریافت کر سکتے ہیں۔

\ابتدا{سوال}\ترچھا{مسئلہ متوازی محور کا ثبوت}\\
\begin{enumerate}[a.]
\item
پہلے دکھائیں کہ جسم کے مرکز کمیت سے گزرتے ہوئے  فضا میں کسی بھی مستوی کے لحاظ سے  معیار اثر اول صفر ہو گا۔ (اشارہ: جسم کے مرکز کمیت کو مبدا پر اور مستوی کو مستوی \عددی{yz} لیں۔ تب کلیہ \عددی{ \bar{x}=\tfrac{M_{yz}}{M}} کیا معلومات فراہم کرتا ہے؟)
\item
جسم کے مرکز کمیت کو بدا پر،  خط \عددی{L_{c,m}} کو محور \عددی{z} پر، اور نقطہ \عددی{(h,0,0)} پر  \عددی{L} کو مستوی \عددی{xy} کا متوازی رکھیں۔ فرض کریں یہ جسم  فضا میں خطہ \عددی{D} میں پایا جاتا ہے۔ تب شکل کے لحاظ سے  درج ذیل ہو گا۔
\begin{align}\label{مساوات_بالکثرت_مسئلہ_متوازی_محور_ب}
I_L=\iiint\limits_D \abs{\kvec{v}-h\ai}^2\dif m
\end{align}
اس تکمل کو   پھیلا   کر حل کر کرتے ہوئے ثبوت مکمل کریں۔
\end{enumerate}
\انتہا{سوال}
%====================
\ابتدا{سوال}
مستقل کثافت  ، رداس \عددی{a} کے کرہ قطر کے لحاظ سے جمودی معیار اثر \عددی{\tfrac{2}{5}ma^2} ہو گا جہاں کرہ کی کمیت \عددی{m} ہے۔ کرہ کو مماسی خط کے لحاظ سے کرہ کا جمودی معیار اثر تلاش کریں۔
\انتہا{سوال}
%=============
\ابتدا{سوال}
محور \عددی{z} کے لحاظ سے سوال \حوالہ{سوال_بالکثرت_اکائی_کثافت_درکار_ت} کے جسم کا جمودی معیار اثر \عددی{I_z=\tfrac{1}{3}abc(a^2+b^2)} ہے۔
\begin{enumerate}[a.]
\item
مساوات \حوالہ{مساوات_بالکثرت_مسئلہ_متوازی_محور_الف} استعمال کرتے ہوئے اس ٹھوس جسم کے مرکز کمیت سے گزرتے ہوئے، محور \عددی{z} کے متوازی  خط کے لحاظ سے جسم  کا جمودی معیار اثر اور رداس دوار تلاش کریں۔
\item
جزو-ا کے نتائج اور مساوات \حوالہ{مساوات_بالکثرت_مسئلہ_متوازی_محور_الف} استعمال کرتے ہوئے  خط \عددی{x=0,\,y=2b} کے لحاظ سے اس جسم کا جمودی معیار اثر اور رداس دوار تلاش کریں۔
\end{enumerate}
\انتہا{سوال}

%===============
\ابتدا{سوال}
اگر سوال \حوالہ{سوال_بالکثرت_اکائی_کثافت_درکار_پ} کے پچر میں \عددی{a=b=6} اور \عددی{c=4} ہوں محور \عددی{x} کے لحاظ سے \عددی{I_x=208} ہو گا۔ اس پچر کا خط \عددی{y=4,\, z=-4/3} (پچر کے  تنگ سر کے  کنارہ) کے لحاظ سے جمودی معیار اثر تلاش کریں۔
\انتہا{سوال}
%===================

\موٹا{کلیہ پاپس}\\

کلیہ پاپس  (حصہ \حوالہ{حصہ_بالکثرت_رقبات_معیار_اثر_مرکز_کمیت} کے سوالات دیکھیں)   دو بعدی صورت  کے ساتھ ساتھ تین بعدی صورت  کے لئے بھی کارآمد ہے۔ فرض کریں دو اجسام \عددی{B_1} اور \عددی{B_2} جن کی  کمیتیں   بالترتیب \عددی{m_1} اور \عددی{m_2} ہوں  فضا میں ایک دوسرے کو  نہ ڈھانپتے ہوئے خطوں  میں پائے جاتے ہیں۔ مبدا سے ان اجسام کے مراکز کمیت تک سمتیات بالترتیب \عددی{\kvec{c}_1} اور \عددی{\kvec{c}_2} ہیں۔تب ان کے اشتراک  \عددی{B_1\cup B_2}  کے مرکز کمیت کا تعین گر سمتیہ درج ذیل ہو گا۔
\begin{align}\label{مساوات_بالکثرت_کلیہ_تین_بعدی_کلیہ_پاپس_الف}
\kvec{c}=\frac{m_1\kvec{c}_1+m_2\kvec{c}_2}{m_1+m_2}
\end{align}
پہلے کی طرح، اس کو \اصطلاح{کلیہ پاپس}\فرہنگ{کلیہ!پاپس}\حاشیہب{Pappus's formula}\فرہنگ{Pappus's formula} کہتے ہیں۔ دو بعدی صورت کی طرح،  \عددی{n} عدد اجسام کے لئے   اس کلیہ کی عمومی روپ درج ذیل ہو گی۔
\begin{align}\label{مساوات_بالکثرت_کلیہ_تین_بعدی_کلیہ_پاپس_ب}
\kvec{c}=\frac{m_1\kvec{c}_1+m_2\kvec{c}_2+\cdots+m_n\kvec{c}_n}{m_1+m_2+\cdots+m_n}
\end{align}
\ابتدا{سوال}
کلیہ پاپس (مساوات  \حوالہ{مساوات_بالکثرت_کلیہ_تین_بعدی_کلیہ_پاپس_الف}) اخذ کریں۔ (اشارہ:ثُمن اول میں ایک دوسرے کو نہ ڈھانپتے ہوئے اجسام \عددی{B_1} اور \عددی{B_2} کا خاکہ بنا کر ان کے مراکز کمیت کو \عددی{(\bar{x}_1,\bar{y}_1,\bar{z}_1)} اور \عددی{(\bar{x}_2,\bar{y}_2,\bar{z}_2)} سے ظاہر کریں۔ کمیت \عددی{m_1} اور \عددی{m_2} اور ان کمیت کے مراکز کے محدد  کی صورت میں محددی مستویات کے لحاظ سے \عددی{B_1\cup B_2} کے  معیار اثر حاصل کریں۔)
\انتہا{سوال}
%===============

\begin{figure}
\centering
\begin{tikzpicture}[font=\small,declare function={fx(\x)=\x;fy(\x)=sqrt(\x);fz(\x)=4-(\x)^2;}]
\pgfmathsetmacro{\xa}{0.4}
\pgfmathsetmacro{\xb}{0.1}
\begin{axis}[small,clip=false,axis lines=center,view/h=110,enlargelimits=true,xlabel={$x$},ylabel={$y$},zlabel={$z$},xlabel style={anchor=east},ylabel style={anchor=west},zlabel style={anchor=south},xtick={\empty},ytick={\empty},ztick={\empty},hide axis]
\addplot3[]coordinates{(2,0,0)(2,3,0)(2,3,2)(2,0,2)(2,0,0)};
\addplot3[]coordinates{(2,3,2)(0,3,2)(0,0,2)(2,0,2)};
\addplot3[]coordinates{(2,3,0) (1,3,0) (1,3,1)(0,3,1) (0,3,2)};
\addplot3[]coordinates{(1,3,0)(1,5,0)};
\addplot3[]coordinates{(1,3,1)(1,5,1)};
\addplot3[]coordinates{(0,3,1)(0,5,1)};
\addplot3[fill=white]coordinates{(3,6,-2)(-1,6,-2)(-1,6,1)(-1,5,1)(3,5,1)(3,5,-2)(3,6,-2)};
\addplot3[]coordinates{(3,6,-2)(3,6,1)(-1,6,1)};
\addplot3[]coordinates{(3,5,1)(3,6,1)};
\addplot3[-latex]coordinates{(2,0,0)(3,0,0)}node[left]{$x$};
\addplot3[-latex]coordinates{(0,6,0)(0,8,0)}node[right]{$y$};
\addplot3[-latex]coordinates{(0,0,2)(0,0,2.5)}node[left]{$z$};
\addplot3[]coordinates{(2,0,2)}node[left]{$(2,0,2)$};
\addplot3[]coordinates{(2,0,0)}node[below,xshift=2ex]{$(2,0,0)$};
\addplot3[]coordinates{(3,6,-2)}node[right]{$(3,6,-2)$};
\addplot3[]coordinates{(-1,6,-2)}node[right]{$(-1,6,-2)$};
\addplot3[]coordinates{(-1,6,1)}node[right]{$(-1,6,1)$};
\addplot3[]coordinates{(0,3,2)}node[right]{$(0,3,2)$};
\addplot3[]coordinates{(-1,5.5,1)}node[above]{$1$};
\addplot3[]coordinates{(0,4,1)}node[above]{$2$};
\addplot3[]coordinates{(1,0,2)}node[left]{$2$};
\addplot3[]coordinates{(2,1.5,2)}node[above]{$3$};
\addplot3[]coordinates{(2,0,1)}node[left]{$2$};
\addplot3[]coordinates{(1,6,-2)}node[right]{$4$};
\addplot3[]coordinates{(2,1.5,1)}node[]{$A$};
\addplot3[]coordinates{(1,4,0.5)}node[]{$B$};
\addplot3[]coordinates{(3,5.5,-0.5)}node[]{$C$};
\end{axis}
\end{tikzpicture}
\caption{ٹھوس جسم برائے سوال \حوالہ{سوال_بالکثرت_تین_حصے_جسم}}
\label{شکل_سوال_بالکثرت_تین_حصے_جسم}
\end{figure}

\ابتدا{سوال}\شناخت{سوال_بالکثرت_تین_حصے_جسم}
مستقل کثافت \عددی{\delta=1} کے تین مستطیل ٹھوس اجسام سے ایک جسم حاصل کیا گیا ہے (شکل \حوالہ{شکل_سوال_بالکثرت_تین_حصے_جسم})۔ کلیہ پاپس استعمال کرتے ہوئے درج  ذیل کے مراکز کمیت تلاش کریں۔
\begin{multicols}{4}
\begin{enumerate}[a.]
\item
$A\cup B$
\item
$A\cup C$
\item
$B\cup C$
\item
$A\cup B\cup C$
\end{enumerate}
\end{multicols}
\انتہا{سوال}
%=============
\ابتدا{سوال}\شناخت{سوال_بالکثرت_قلفی}
\begin{enumerate}[a.]
\item
 قد \عددی{h} اور رداس \عددی{r} کے قاعدہ کا  دائری ٹھوس  مخروط \عددی{C}، رداس \عددی{a} کے    ٹھوس  نصف کرہ \عددی{S}  پر قلفی کی طرح جمایا گیا ہے۔ٹھوس مخروط کا وسطانی مرکز قاعدہ سے راس کے رخ  ایک 
چوتھائی (\عددی{\tfrac{1}{4}})  فاصلہ پر واقع ہے۔ نصف کرہ کے وسطانی مرکز قاعدہ سے سر کے رخ  تین آٹھواں (\عددی{\tfrac{3}{8}})  فاصلہ دور ہے۔ مشترک جسم \عددی{C\cup S} کا مرکز مشترک قاعدہ پر رکھنے کی خاطر \عددی{a} اور \عددی{h} کے بیچ تعلق معلوم کریں۔ 
\item
اگر آپ نے حصہ \حوالہ{حصہ_بالکثرت_رقبات_معیار_اثر_مرکز_کمیت} میں معادل    سوال \حوالہ{سوال_بالکثرت_مثلث_دائرہ} کو اب تک حل نہ کیا ہو، تب اس کو حل کریں۔ دونوں کے جواب ایک جیسے نہیں ہیں۔
\end{enumerate}
\انتہا{سوال}
%================
\ابتدا{سوال}
ایک ٹھوس اہرام \عددی{P}   جس کا  قد \عددی{h}   اور مماثل چار اضلاع ہیں کا قاعدہ ٹھوس    مکعب \عددی{C}   کا ایک مربعی  سطح ہے جس کے اضلاع کی  لمبائیاں \عددی{s} ہے۔  ٹھوس اہرام کا وسطانی مرکز قاعدہ سے راس کے رخ ایک چوتھائی فاصلہ پر ہے۔ ٹھوس جسم \عددی{P\cup C} کا وسطانی مرکز اہرام کے قاعدہ پر رکھنے کی خاطر \عددی{s} اور \عددی{h} کا تعلق دریافت کریں۔ سوال \حوالہ{سوال_بالکثرت_قلفی} کے نتیجہ کے ساتھ موازنہ کریں۔ حصہ \حوالہ{حصہ_بالکثرت_رقبات_معیار_اثر_مرکز_کمیت} میں سوال \حوالہ{سوال_بالکثرت_چکور_قاعدہ} کے نتیجہ کے ساتھ بھی موازنہ کریں۔
\انتہا{سوال}
%============

\حصہ{نلکی اور کروی محدد میں تہرا تکمل}
انجینئری،  طبیعیات اور جیومیٹری میں  مخروط ، بیلن یا کرہ کے ساتھ کام  نلکی اور کروی محدد میں زیادہ آسان ہوتا ہے۔


\begin{figure}
\centering
\begin{minipage}{0.45\textwidth}
\centering
\begin{tikzpicture}[font=\small,declare function={fx(\x)=cos(\x);fy(\x)=sin(\x);}]
\pgfmathsetmacro{\xa}{135}
\pgfmathsetmacro{\xb}{\xa+180}
\pgfmathsetmacro{\xc}{65}
\pgfmathsetmacro{\xd}{\xc+180}
\pgfmathsetmacro{\a}{1.5}
\pgfmathsetmacro{\b}{1.25}
\begin{axis}[clip=false,axis lines=center,view/h=135,enlargelimits=true,xlabel={$x$},ylabel={$y$},zlabel={$z$},xlabel style={anchor=east},ylabel style={anchor=west},zlabel style={anchor=south},xtick={\empty},ytick={\empty},ztick={\empty},hide  axis]
\addplot3[domain=0:360]({fx(x)},{fy(x)},2);
\addplot3[domain=0:360]({fx(x)},{fy(x)},0);
\addplot3[]coordinates{({fx(\xa)},{fy(\xa)},{2})({fx(\xa)},{fy(\xa)},0)};
\addplot3[]coordinates{({fx(\xb)},{fy(\xb)},{2})({fx(\xb)},{fy(\xb)},0)};
\addplot3[]coordinates{(-\a,-\b,2)(\a,-\b,2)(\a,\b,2)(-\a,\b,2)(-\a,-\b,2)};
\addplot3[]coordinates{({2*fx(\xc)},{2*fy(\xc)},2)({2*fx(\xc)},{2*fy(\xc)},0)({2*fx(\xd)},{2*fy(\xd)},0)({2*fx(\xd)},{2*fy(\xd)},2) ({2*fx(\xc)},{2*fy(\xc)},2)};
\addplot3[thick]coordinates{({fx(\xc)},{fy(\xc)},0)({fx(\xc)},{fy(\xc)},2)}node[circ]{}node[pin={[pin distance=1cm,right]-10:{$(\rho,\phi,z)$}}]{};
\addplot3[-stealth,domain=0:\xc,samples y=0] ({1.2*fx(x)},{1.2*fy(x)},0)node[pos=0.4,below]{$\phi$};
\addplot3[thick]coordinates{(0,0,0)({fx(\xc)},{fy(\xc)},0)}node[pos=0.6,left]{$\rho$};
\addplot3[dashed]coordinates{(0,0,0)(0,0,2)}node[circ]{}node[right]{$z$};
\addplot3[-latex]coordinates{(0,0,2)(0,0,3.5)}node[left]{$z$};
\addplot3[-latex]coordinates{(0,0,0)(2,0,0)}node[left]{$x$};
\addplot3[-latex]coordinates{(0,0,0)(0,2,0)}node[right]{$y$};
\end{axis}
\end{tikzpicture}
\caption{نلکی محدد میں مستقل محدد سطحیں۔}
\label{شکل_بالکثرت_نلکی_مستقل_سطحیں}
\end{minipage}\hfill
\begin{minipage}{0.45\textwidth}
\centering
\begin{tikzpicture}[font=\small,declare function={fx(\r,\t)=\r*cos(\t);fy(\r,\t)=\r*sin(\t);}]
\pgfmathsetmacro{\ta}{30}
\pgfmathsetmacro{\tb}{\ta+45}
\pgfmathsetmacro{\ra}{1.25}
\pgfmathsetmacro{\rb}{\ra+0.45}
\pgfmathsetmacro{\h}{0.5}
\begin{axis}[clip=false,axis lines=center,view/h=20,enlargelimits=true,xlabel={$x$},ylabel={$y$},zlabel={$z$},xlabel style={anchor=east},ylabel style={anchor=west},zlabel style={anchor=south},xtick={\empty},ytick={\empty},ztick={\empty},xmin=0,ymin=0,zmin=0,zmax=1,hide x axis, hide y axis]
\addplot3[thick,domain y=\ta:\tb,samples=0,variable=\r,variable y=\t] ({fx(\ra,t)},{fy(\ra,t)},0)node[pos=0.7,sloped,below]{$\rho\dif\phi$};
\addplot3[domain y=\ta:\tb,samples=0,variable=\r,variable y=\t] ({fx(\rb,t)},{fy(\rb,t)},0);
\addplot3[domain y=\ta:\tb,samples=0,variable=\r,variable y=\t] ({fx(\ra,t)},{fy(\ra,t)},\h);
\addplot3[domain y=\ta:\tb,samples=0,variable=\r,variable y=\t] ({fx(\rb,t)},{fy(\rb,t)},\h);
\addplot3[thick]coordinates{({fx(\ra,\ta)},{fy(\ra,\ta)},0)({fx(\rb,\ta)},{fy(\rb,\ta)},0)}node[pos=0.5,below]{$\dif\rho$};
\addplot3[]coordinates{({fx(\ra,\tb)},{fy(\ra,\tb)},0)({fx(\rb,\tb)},{fy(\rb,\tb)},0)};
\addplot3[]coordinates{({fx(\ra,\ta)},{fy(\ra,\ta)},\h)({fx(\rb,\ta)},{fy(\rb,\ta)},\h)};
\addplot3[]coordinates{({fx(\ra,\tb)},{fy(\ra,\tb)},\h)({fx(\rb,\tb)},{fy(\rb,\tb)},\h)};
\addplot3[]coordinates{({fx(\ra,\ta)},{fy(\ra,\ta)},0)({fx(\ra,\ta)},{fy(\ra,\ta)},\h)};
\addplot3[thick]coordinates{({fx(\rb,\ta)},{fy(\rb,\ta)},0)({fx(\rb,\ta)},{fy(\rb,\ta)},\h)}node[pos=0.5,right]{$\dif z$};
\addplot3[]coordinates{({fx(\ra,\tb)},{fy(\ra,\tb)},0)({fx(\ra,\tb)},{fy(\ra,\tb)},\h)};
\addplot3[]coordinates{({fx(\rb,\tb)},{fy(\rb,\tb)},0)({fx(\rb,\tb)},{fy(\rb,\tb)},\h)};
\addplot3[dashed]coordinates{(0,0,0)({fx(\ra,\ta)},{fy(\ra,\ta)},0)}node[pos=0.75,below]{$\rho$};
\addplot3[dashed]coordinates{(0,0,0)({fx(\ra,\tb)},{fy(\ra,\tb)},0)};
\addplot3[dashed]coordinates{(0,0,\h)({fx(\ra,\ta)},{fy(\ra,\ta)},\h)};
\addplot3[dashed]coordinates{(0,0,\h)({fx(\ra,\tb)},{fy(\ra,\tb)},\h)};
\addplot3[dashed,samples=0,domain y=\ta:\tb,variable y=\t]({fx(0.5,t)},{fy(0.5,t)},0)node[pos=0.7,xshift=1ex,right]{$\dif \phi$};
\end{axis}
\end{tikzpicture}
\caption{نلکی محدد میں چھوٹا حجم \عددی{\dif H=\dif z\rho\dif\rho\dif\phi} ہو گا۔}
\label{شکل_بالکثرت_نلکی_چھوٹا_حجم_تعریف}
\end{minipage}
\end{figure}

\جزوحصہء{نلکی محدد}
جن  بیلن  کا محور  \عددی{z} محدد   پر پایا جاتا ہو اور وہ مستویات جن میں \عددی{z} محدد پایا جاتا ہو یا جو \عددی{z} محدد کے عمودی ہوں، کو نلکی محدد میں بیان کرنا  نہایت  آسان ہوتا ہے (شکل \حوالہ{شکل_بالکثرت_نلکی_مستقل_سطحیں})۔


جیسا ہم دیکھ چکے ہیں ان سطحوں کی مساوات  مستقل محددی صورت رکھتی ہیں۔
\begin {align*}
\rho  & = 4\\
\phi & =\frac{\pi}{3}\\
z &=2
\end {align*}

فضا میں خطہ کی نلکی محدد  میں مستطیلی خانہ بندی کا   ایک خانہ  شکل \حوالہ{شکل_بالکثرت_نلکی_چھوٹا_حجم_تعریف} میں دکھایا گیا ہے۔ اگر \عددی{z} محدد سے اس خانے کی وسط تک رداس \عددی{\rho} ہو تب خانے کے اندرونی اور بیرونی سطحوں کے رداس  بالترتیب \عددی{\rho-\tfrac{\dif \rho}{2}} اور \عددی{\rho+\tfrac{\dif \rho}{2}} ہوں گے۔اس چھوٹے   خانے کو نقطہ دار لکیروں سے   \عددی{z} محدد تک بڑھا کر
  حجم 
\(\tfrac{1}{2}(\rho+\tfrac{\dif\rho}{2})^2\dif\phi\dif z\)
 حاصل ہو گا جس میں  سے اضافی حجم  
\(\tfrac{1}{2}(\rho-\tfrac{\dif\rho}{2})^2\dif\phi\dif z\)
 منفی کر کے چھوٹے حصہ کا حجم معلوم کیا جا سکتا ہے۔
\begin {align*}
\dif H &=\frac{1}{2}\big(\rho+\frac{\dif \rho}{2}\big)^2\dif\phi\dif z-\frac{1}{2}\big(\rho-\frac{\dif \rho}{2}\big)^2\dif\phi\dif z \\
&= \dif z \,\rho \dif \rho \dif \phi
\end {align*}
چھوٹے  مستطیل  خانے کی وسطی (یا اندرونی قوسی) چوڑائی \عددی{\rho\dif\phi}، لمبائی \عددی{\dif\rho} اور قد \عددی{\dif z} لے  کر  
\begin{align*}
\dif H=(\rho\dif\phi)(\dif\rho)(\dif z)=\dif z\rho\dif \rho\dif\phi
\end{align*}
حجم  زیادہ  آسانی سے حاصل کیا جا سکتا ہے۔اسی طرح خانے کی سامنے (یا پشت) سطح کا رقبہ \عددی{\dif \rho\dif z}، نچلی (یا بالائی) سطح   کا رقبہ \عددی{\rho\dif \phi\dif\rho} اور قوسی  (اندرونی یا بیرونی) سطح کا رقبہ  \عددی{\rho\dif\phi\dif z}  ہو   گا ۔ یوں  نلکی محدد میں تہرا تکملات کو  بطور بارہا تکملات حل کیا جائے گا۔ ایسا  اگلی مثال 
میں دکھایا گیا ہے۔


\ابتدا {مثال}\شناخت{مثال_بالترتیب_نلکی_تہرا_تکمل_الف}
خطہ \عددی{D} پر تفاعل  \عددی{f(\rho,\phi,z)} کی نلکی محدد میں تکمل  کی حدیں تلاش کریں۔ خطہ \عددی{D} نیچے سے مستوی  
\(z=0\)
اور اطراف سے دائری بیلن
\(x^2+(y-1)^2=1\)
 جبکہ اوپر سے قطع مکافی
\(z=x^2+y^2\)
 کے بیچ پایا جاتا ہے (شکل \حوالہ{شکل_مثال_بالترتیب_نلکی_تہرا_تکمل_الف})۔


حل
\begin {enumerate}[1.]
\item
\ترچھا {خاکہ بنانا:}\quad
\عددی{D}  کا  قاعدہ ہی  مستوی \عددی{xy} پر \عددی{D} کی تظلیل \عددی{R}  ہو  گی۔تظلیل \عددی{R} کی سرحد دائرہ \عددی{x^2+(y-1)^2=2} ہو گی جس کی قطبی مساوات درج ذیل ہے۔


\begin {align*}
x^2+(y-1)^2&=1\\
x^2+y^2-2y+1&=1\\
\rho^2-2\rho\sin\phi&=0\\
\rho&=2\sin\phi
\end {align*}
\item
\ترچھا {تکمل کی \عددی{z} حدیں:}\quad
خطہ \عددی{R} میں عمومی نقطہ \عددی{(\rho,\phi)} سے گزرتی ہوئی لکیر \عددی{M}، جو \عددی{z} محدد کے متوازی  ہو \عددی{D} میں   \عددی{z=0} پر داخل اور \عددی{z=x^2+y^2=\rho^2} پر خارج  ہو گی۔
\item
\ترچھا{تکمل کی \عددی{\rho} حدیں:}\quad
مبدا سے  خط \عددی{L} جو نقطہ \عددی{(\rho,\phi)}  سے گزرتا ہو، \عددی{R} میں \عددی{\rho=0} پر داخل اور \عددی{\rho=2\sin\phi} پر خارج ہو گا۔
\item
\ترچھا{تکمل کی \عددی{\phi} حدیں:}\quad
خط \عددی{L}  جھاڑو کی طرح \عددی{R} کو جھاڑتے ہوئے مثبت \عددی{x} محور کے ساتھ \عددی{\phi=0} اور \عددی{\phi=\pi}کے بیچ رہتا ہے۔
\end {enumerate}
یوں تکمل درج ذیل ہو گا۔
\begin{align*}
\iiint\limits_D f(\rho,\phi,z)\dif H=\int_{0}^{\pi}\int_0^{2\sin\phi}\int_{0}^{\rho^2}f(\rho,\phi,z)\dif z \,\rho\dif \rho\dif \phi
\end{align*}
\انتہا{مثال}


اس مثال میں ہم نے نلکی محدد میں تکمل کی حدیں تلاش کرنا سیکھا۔



\begin{figure}
\centering
\begin{minipage}{0.45\textwidth}
\centering
\begin{tikzpicture}[font=\small,declare function={fx(\x)=cos(\x);fy(\x)=1+sin(\x);fz(\x)=2+2*sin(\x);}]
\pgfmathsetmacro{\xa}{110}
\pgfmathsetmacro{\xb}{\xa+180}
\pgfmathsetmacro{\xx}{0.35}
\pgfmathsetmacro{\yy}{1.7}
\pgfmathsetmacro{\yyy}{2.5}
\pgfmathsetmacro{\zz}{4}
\pgfmathsetmacro{\angL}{atan(\yy/\xx)}
\begin{axis}[clip=false,axis lines=center,view/h=110,enlargelimits=true,xlabel={$x$},ylabel={$y$},zlabel={$z$},xlabel style={anchor=east},ylabel style={anchor=west},zlabel style={anchor=south},xtick={\empty},ytick={2},ztick={\empty},zmax=5,hide z axis]
\addplot3[domain=0:360]({fx(x)},{fy(x)},{fz(x)})node[pos=0.9,pin={[font=\scriptsize]135:{\begin{minipage}{2cm} \begin{align*} z&=x^2+y^2 &&\text{کارتیسی}\\ z&=\rho^2&&\text{نلکی}\end{align*} \end{minipage}}}]{};
\addplot3[domain=0:360]({fx(x)},{fy(x)},0)node[pos=0.195,below]{$R$}node[pos=0.2,pin={[font=\scriptsize]-45:{\begin{minipage}{2cm} \begin{align*}&x^2+(y-1)^2=1&&\text{کارتیسی}\\ &\rho=2\sin\phi &&\text{نلکی} \end{align*}\end{minipage}}}]{};
\addplot3[] coordinates{({fx(\xa)},{fy(\xa)},{fz(\xa)})({fx(\xa)},{fy(\xa)},0)}node[pos=0.5,right]{$D$};
\addplot3[] coordinates{({fx(\xb)},{fy(\xb)},{fz(\xb)})({fx(\xb)},{fy(\xb)},0)};
\addplot3[dashed]coordinates{(\xx,\yy,0)(\xx,\yy,\zz)}node[pos=0,pin=-45:{$(\rho,\phi)$}]{}node[pos=0,circ]{}node[pos=1,circ]{};
\addplot3[-latex]coordinates{(\xx,\yy,\zz)(\xx,\yy,6)}node[right]{$M$};
\addplot3[-latex]coordinates{(0,0,0)(\xx*\yyy/\yy,\yyy,0)}node[right]{$L$};
\addplot3[-latex]coordinates{(0,0,0)(0,0,2)}node[left]{$z$};
\addplot3[-stealth,domain=0:\angL,samples y=0]({0.6*cos(x)},{0.6*sin(x)},0)node[pos=0.75,below]{$\phi$};
\end{axis}
\end{tikzpicture}
\caption{جسم برائے مثال \حوالہ{مثال_بالترتیب_نلکی_تہرا_تکمل_الف}}
\label{شکل_مثال_بالترتیب_نلکی_تہرا_تکمل_الف}
\end{minipage}\hfill
\begin{minipage}{0.45\textwidth}
\centering
\begin{tikzpicture}[font=\small]
\pgfmathsetmacro{\ang}{atan(0.25/0.5)}
\pgfmathsetmacro{\a}{1}
\pgfmathsetmacro{\b}{0.75*\a}
\pgfmathsetmacro{\h}{2}
\draw(0,0) circle (\a cm and \b cm);
\draw(0,\h) circle (\a cm and \b cm);
\draw[]plot [domain=-1:1] (\x,{2*(\x)^2});
\draw(-0.75,{2*(0.75)^2})node[pin={-135:{$\begin{aligned}z&=x^2+y^2\\  z&=\rho^2\end{aligned}$}}]{};
\draw(-\a,0)--++(0,\h);
\draw(\a,0)--++(0,\h)node[pos=0.5,pin=45:{$\begin{aligned}x^2+y^2&=4\\  \rho&=2  \end{aligned}$}]{};
\draw(0.5,-0.25)node[circ]{}node[pin=-80:{$(\rho,\phi)$}]{}--++(0,1.25)node[circ]{};
\path[name path=klid]([shift={(180:\a cm and \b cm)}]0,\h) arc (180:360:\a cm and \b cm);
\path[name path=kM] (0.5,-0.25)++(0,1.25)--++(0,1);
\draw[-latex,name intersections={of={kM and klid}}](intersection-1)--++(0,0.5)node[above]{$M$};
\draw[-latex](0,0)--(1.25,-1.25*0.25/0.5)node[right]{$L$};
\draw[-latex](0,0)--++(-135:1.2)node[left]{$x$};
\draw[-latex](0,0)--++(2,0)node[right]{$y$};
\draw[-latex](0,\h-\b)--++(0,2)node[left]{$z$};
\draw(0,\h)node[left]{$4$}--++(0.1,0);
\draw[-stealth]([shift={(-135:0.3)}]0,0) arc (-135:-\ang:0.3)node[pos=0.6,below]{$\phi$};
\end{tikzpicture}
\caption{جسم برائے مثال \حوالہ{مثال_بالترتیب_نلکی_تہرا_تکمل_ب}}
\label{شکل_مثال_بالترتیب_نلکی_تہرا_تکمل_ب}
\end{minipage}
\end{figure}

\ابتدا{مثال}\شناخت{مثال_بالترتیب_نلکی_تہرا_تکمل_ب}
بیلن
\(x^2+y^2=4\)
 میں بند ٹھوس جسم  جو اوپر سے  قطع مکافی سطح 
\(z=x^2+y^2\)
اور نیچے سے مستوی
\(xy\)
 کے بیچ پایا جاتا ہو، کا وسطانی مرکز تلاش  کریں (شکل \حوالہ{مثال_بالترتیب_نلکی_تہرا_تکمل_ب})۔ ٹھوس جسم کی کثافت \عددی{\delta=1} ہے۔


حل


ہم اوپر سے قطع کافی
\(z = \rho^2\)
 اور نیچے سے مستوی
\(z=0\)
 میں ملفوف ٹھوس جسم  کا خاکہ بناتے ہیں۔اس کا قاعدہ  
\(R\)
 مستوی
\(xy\)
 میں قرص
\(\abs{\rho}\le 2\)
 ہو گا۔


ٹھوس جسم کا وسطانی مرکز  تشاکلی محور پر ہو گا جو محور \عددی{z} ہے۔ یوں \عددی{\bar{x}=\bar{y}=0} ہوگا۔ ہم معیار اثر \عددی{M_{xy}} کو کمیت \عددی{M} سے تقسیم کر کے \عددی{\bar{z}}  تلاش کرتے ہیں۔


کمیت اور معیار اثر کے تکملات کی حدیں تلاش کرنے کی خاطر  ہم وہی چار مخصوص قدم لیتے ہیں۔  خاکہ بنا  کر ہم پہلا قدم مکمل کر چکے ہیں۔باقی اقدام  درج ذیل  ہیں۔
\begin{enumerate}
 \setcounter{enumi}{1}
\item
\ترچھا{تکمل کی \عددی{z} حدیں:}\quad
علامتی نقطہ \عددی{(\rho,\phi)} سے گزرتی ہوئی، محدد \عددی{z} کی متوازی  لکیر \عددی{M}،  ٹھوس جسم میں  \عددی{z=0} سے  داخل اور  \عددی{z=\rho^2} سے  خارج ہو گی۔ 
\item
\ترچھا{تکمل کی \عددی{\rho} حدیں:}\quad
مبدا سے شروع نقطہ \عددی{\rho,\phi} سے گزرتی ہوئی لکیر \عددی{L} خطہ \عددی{R} میں \عددی{\rho=0} سے داخل اور \عددی{\rho=2} سے خارج ہو گی۔
\item
\ترچھا{تکمل کی \عددی{\phi} حدیں:}\quad
لکیر \عددی{L} قاعدہ پر گھڑی کی سوئی کی طرح گھومتی  ہوئی    \عددی{\phi=0} سے  \عددی{\phi=2\pi} تک طے کرتی ہے۔
\end{enumerate}
یوں  \عددی{M_{xy}}کی قیمت
\begin{align*} 
 M_{xy}&=\int_0^{2\pi}\int_0^{2}\int_0^{\rho^2}z\dif z \,\rho\dif \rho\dif \phi =\int_0^{2\pi}\int_0^2\big[\frac{z^2}{2}\big]_{0}^{\rho^2} \rho\dif \rho\dif \phi\\
&=\int_0^{2\pi}\int_0^2\frac{\rho^5}{2}\dif \rho\dif \phi=\int_0^{2\pi}\big[\frac{\rho^6}{12}\big]_0^{\rho^2}\dif \phi=\int_0^{2\pi}\frac{16}{3}\dif\phi=\frac{32\pi}{3}
\end{align*}
اور \عددی{M} کی قیمت
\begin{align*}
M&=\int_0^{2\pi}\int_0^2\int_0^{\rho^2}\dif z \,\rho\dif \rho\dif \phi=\int_0^{2\pi}\int_0^2\big[z\big]_{0}^{\rho^2} \rho\dif \rho\dif \phi\\
&=\int_0^{2\pi}\int_0^2 \rho^3\dif \rho\dif \phi=\int_0^{2\pi}\big[\frac{\rho^4}{4}\big]_0^2\dif\phi=\int_0^{2\pi}4\dif\phi=8\pi
\end{align*}
ہو گی لہٰذا
\begin {align*}
\bar {z}=\frac  {M_{xy}} {M}=\frac{32\pi}{3}\frac{1}{8\pi}=\frac{4}{3}
\end {align*}
ہو گا۔ وسطانی مرکز \عددی{(0,0,4/3)} ہو گا جو  ٹھوس جسم سے باہر ہے۔
\انتہا {مثال}
%===================

\جزوحصہء{نلکی محدد میں تکمل کی قیمت کا حصول}
فضا میں خطہ \عددی{D} پر  تکمل
\[\iiint\limits_D f(\rho,\phi,z)\dif H\]
کی قیمت  حاصل کرتے ہوئے نلکی محدد میں پہلے \عددی{z}، اس کے بعد \عددی{\rho} اور آخر میں \عددی{\phi} کے لحاظ سے تکمل لیتے ہوئے  درج ذیل اقدام کرنے ہوں گے۔
\begin{enumerate}[1.]
\item
\ترچھا{خاکہ:}\quad
خطہ \عددی{D} اور مستوی \عددی{xy} پر اس کی تظلیل \عددی{R}  کا خاکہ بنائیں ۔ \عددی{D} اور \عددی{R} کی سرحدی  سطحوں  اور منحنیات کی نشاندہی کریں (شکل \حوالہ{شکل_بالکثرت_نلکی_محدد_تکمل_کا_حصول}-ا)۔
\item
\ترچھا{تکمل کی \عددی{z} حدیں:}\quad
\عددی{R} میں علامتی نقطہ \عددی{(\rho,\phi)} پر محور \عددی{z} کے متوازی  ایک علامتی لکیر \عددی{M} کھینچیں جو بڑھ کر  \عددی{D} میں \عددی{z=g_1(\rho,\phi)} سے داخل اور \عددی{z=g_2(\rho,\phi)} سے خارج ہو گی (شکل \حوالہ{شکل_بالکثرت_نلکی_محدد_تکمل_کا_حصول}-ب)۔یہی تکمل کی \عددی{z} حدیں ہوں گی۔
\item
\ترچھا{تکمل کی \عددی{\rho} حدیں:}\quad
مبدا سے ایک لکیر \عددی{L} کھینچیں جو نقطہ \عددی{(\rho,\phi)} سے گزرتی ہو۔ یہ شعاع  خطہ \عددی{R} میں \عددی{\rho=h_1(\phi)} سے داخل اور \عددی{\rho=h_2(\phi)} سے خارج ہو گی (شکل \حوالہ{شکل_بالکثرت_نلکی_محدد_تکمل_کا_حصول}-ج)۔  یہی تکمل کی \عددی{\rho} حدیں ہوں گی۔
\item
\ترچھا{تکمل کی \عددی{\phi} حدیں:}\quad
لکیر \عددی{L} خطہ \عددی{R} کو جھاڑتے ہوئے  مثبت \عددی{x} محور کے ساتھ زاویہ \عددی{\phi=\alpha} اور \عددی{\phi=\beta}  کے بیچ رہتی ہے (شکل \حوالہ{شکل_بالکثرت_نلکی_محدد_تکمل_کا_حصول}-ج)۔ یہی تکمل کی \عددی{\phi} حدیں ہوں گی۔
\end{enumerate}
یوں تکمل درج ذیل ہو گا۔
\begin{align}
\iiint\limits_D f(\rho,\phi,z)\dif H=\int_{\phi=\alpha}^{\phi=\beta}\int_{\rho=h_1(\phi)}^{\rho=h_2(\phi)}\int_{z=g_1(\rho,\phi)}^{z=g_2(\rho,\phi)}f(\rho,\phi,z)\dif z \,\rho\dif \rho \dif\phi
\end{align}
\begin{figure}
\centering
\begin{subfigure}{0.30\textwidth}
\centering
\begin{tikzpicture}[font=\scriptsize]
\pgfmathsetmacro{\angA}{-170}
\pgfmathsetmacro{\angB}{100}
\pgfmathsetmacro{\angC}{-10}
\pgfmathsetmacro{\angD}{-135}
\pgfmathsetmacro{\angE}{-100}
\pgfmathsetmacro{\angF}{-40}
\draw[-latex,name path=kx](0,0)--++(-1.5,-1.5)node[right]{$x$};
\draw[-latex](0,0)--++(2.5,0)node[right]{$y$};
\draw[-latex](0,0)--++(0,2)node[left]{$z$};
\draw[fill=lgray,draw=black,text=black,opacity=0.5](2.25,-0.25)coordinate(kA) to [out=\angA,in=\angB]node[pos=0.75,above left,yshift=-1ex,font=\scriptsize]{$\rho=h_1(\phi)$}++(-1.5,-0.75)coordinate(kB)node[above right,xshift=1ex,yshift=0.5ex]{$R$} to [out=\angC,in=\angD]node[pos=0.5,right]{$\rho=h_2(\phi)$}(kA);
\path[dashed](kA)--++(0,2.25)coordinate(kC)coordinate[pos=0.5](kE);
\path[dashed](kB)--++(0,3)coordinate(kD)coordinate[pos=0.5](kF);
\draw[dashed](kA)--(kE)  (kB)--(kF);
\draw(kC)--(kE)  (kD)--(kF);
\draw[dashed](kE) to [out=\angA,in=\angB](kF);
\draw(kF) to [out=\angC,in=\angD](kE); 
\draw(kE) to [out=\angE,in=\angF]coordinate[pos=0.25](kLower)(kF);
\draw(kC) to [out=110,in=60]coordinate[pos=0.25](kUpper)(kD);
\draw[](kC) to [out=-\angA,in=-\angB](kD);
\draw[dashed](kD) to [out=-\angC,in=-\angD](kC);
\draw(kLower)node[pin={[right,pin distance=0.25cm,font=\scriptsize]10:{$z=g_1(\rho,\phi)$}}]{};
\draw(kUpper)node[pin={[above]35:{$z=g_2(\rho,\phi)$}}]{};
\draw($(kC)!0.5!(kE)$)node[right]{$D$};
\end{tikzpicture}
\caption{}
\end{subfigure}\hfill
\begin{subfigure}{0.30\textwidth}
\centering
\begin{tikzpicture}[font=\scriptsize]
\pgfmathsetmacro{\angA}{-170}
\pgfmathsetmacro{\angB}{100}
\pgfmathsetmacro{\angC}{-10}
\pgfmathsetmacro{\angD}{-135}
\pgfmathsetmacro{\angE}{-100}
\pgfmathsetmacro{\angF}{-40}
\draw[-latex,name path=kx](0,0)--++(-1.5,-1.5)node[right]{$x$};
\draw[-latex](0,0)--++(2.5,0)node[right]{$y$};
\draw[-latex](0,0)--++(0,2)node[left]{$z$};
\draw[fill=lgray,draw=black,text=black,opacity=0.5](2.25,-0.25)coordinate(kA) to [out=\angA,in=\angB]node[pos=0.5, left,xshift=-2ex,yshift=-1ex,font=\scriptsize]{$\rho=h_1(\phi)$}++(-1.5,-0.75)coordinate(kB)node[above right,xshift=1ex,yshift=0.5ex]{$R$} to [out=\angC,in=\angD]node[pos=0.75,right,font=\scriptsize]{$\rho=h_2(\phi)$}(kA);
\path[dashed](kA)--++(0,2.25)coordinate(kC)coordinate[pos=0.5](kE);
\path[dashed](kB)--++(0,3)coordinate(kD)coordinate[pos=0.5](kF);
%\draw[dashed](kA)--(kE)  (kB)--(kF);
\draw(kC)--(kE)  (kD)--(kF);
\draw[dashed](kE) to [out=\angA,in=\angB](kF);
\draw(kF) to [out=\angC,in=\angD](kE); 
\draw(kE) to [out=\angE,in=\angF]coordinate[pos=0.25](kLower)(kF);
\draw(kC) to [out=110,in=60]coordinate[pos=0.25](kUpper)(kD);
\draw[](kC) to [out=-\angA,in=-\angB](kD);
\draw[dashed](kD) to [out=-\angC,in=-\angD](kC);
\draw($(kC)!0.5!(kE)$)node[right]{$D$};
\draw($(kA)!0.5!(kB)$)node[circ]{}node[pin={[pin distance=0.25cm]-45:{$(\rho,\phi)$}}]{}--($(kE)!0.5!(kF)+(0,-0.3)$)coordinate(kMm)node[circ]{};
\draw[dashed](kMm)node[pin={[right,pin edge={-,solid},align=center,pin distance=1cm]30:{\text{\RL{داخل}}\\ $z=g_1(\rho,\phi)$}}]{}--($(kC)!0.5!(kD)$)node[circ]{}node[pin={[align=center,pin edge={-,solid}]35:{\text{\RL{خارج}}\\ $z=g_2(\rho,\phi)$}}]{};
\draw[-latex]($(kC)!0.5!(kD)$)--++(0,1)node[left]{$M$};
\end{tikzpicture}
\caption{}
\end{subfigure}\hfill
\begin{subfigure}{0.30\textwidth}
\centering
\begin{tikzpicture}[font=\scriptsize]
\pgfmathsetmacro{\angA}{-170}
\pgfmathsetmacro{\angB}{100}
\pgfmathsetmacro{\angC}{-10}
\pgfmathsetmacro{\angD}{-135}
\pgfmathsetmacro{\angE}{-100}
\pgfmathsetmacro{\angF}{-40}
\pgfmathsetmacro{\angMM}{atan(-0.625/1.55)}
\pgfmathsetmacro{\angEE}{atan(-0.25/2.25)}
\pgfmathsetmacro{\angSS}{atan(-1/0.75)}
\draw[-latex,name path=kx](0,0)--++(-1.5,-1.5)node[right]{$x$};
\draw[-latex](0,0)--++(2.25,0)node[right]{$y$};
\draw[-latex](0,0)--++(0,2)node[left]{$z$};
\draw[fill=lgray,draw=black,text=black,opacity=0.5](2.25,-0.25)coordinate(kA)node[circ]{} to [out=\angA,in=\angB]++(-1.5,-0.75)coordinate(kB)node[circ]{}node[right,xshift=2ex,yshift=-1ex]{$R$} to [out=\angC,in=\angD](kA);
\path[name path=kR](kA) to [out=\angA,in=\angB](kB);
\path[name path=kL](kB) to [out=\angC,in=\angD](kA);
\path[dashed](kA)--++(0,2.25)coordinate(kC)coordinate[pos=0.5](kE);
\path[dashed](kB)--++(0,3)coordinate(kD)coordinate[pos=0.5](kF);
\draw(kC)--(kE)  (kD)--(kF);
\draw[dashed](kE) to [out=\angA,in=\angB](kF);
\draw(kF) to [out=\angC,in=\angD](kE); 
\draw(kE) to [out=\angE,in=\angF]coordinate[pos=0.25](kLower)(kF);
\draw(kC) to [out=110,in=60]coordinate[pos=0.25](kUpper)(kD);
\draw[](kC) to [out=-\angA,in=-\angB](kD);
\draw[dashed](kD) to [out=-\angC,in=-\angD](kC);
\draw($(kC)!0.5!(kE)$)node[right]{$D$};
\draw($(kA)!0.5!(kB)$)coordinate(kLM)node[circ]{}node[xshift=-2ex,yshift=-1ex]{$(\rho,\phi)$}--($(kE)!0.5!(kF)+(0,-0.3)$)coordinate(kMm)node[circ]{};
\draw[dashed](kMm)--($(kC)!0.5!(kD)$)node[circ]{};
\draw[-latex]($(kC)!0.5!(kD)$)--++(0,1)node[left]{$M$};
\draw[-latex,name path=kray](0,0)--++(\angMM:2.25)node[right]{$L$};
\draw[name intersections={of=kR and kray}](intersection-1)node[circ]{}node[pin={[align=center,pin distance=0.75cm]-170:{\text{داخل}\\$\rho=h_1(\phi)$}}]{};
\draw[name intersections={of=kL and kray}](intersection-1)node[circ]{}node[pin={[align=center,pin distance=0.25cm,below]-70:{\text{خارج}\\  $\rho=h_2(\phi)$}}]{};
\draw[-latex](0,0)--++(\angEE:2.75)node[right]{$\phi=\beta$};
\draw[-latex](0,0)--++(\angSS:1.75)node[left,yshift=-1ex]{$\phi=\alpha$};
\draw[-stealth]([shift={(-135:0.3)}]0,0) arc (-135:\angSS:0.3)node[pos=0.5,below]{$\alpha$};
\end{tikzpicture}
\caption{}
\end{subfigure}
\caption{نلکی محدد میں تہرا تکمل کی حدوں کا تعین۔}
\label{شکل_بالکثرت_نلکی_محدد_تکمل_کا_حصول}
\end{figure}


\جزوحصہء{کروی محدد}
ایسے کرہ جن   کے مراکز مبدا پر ہوں، وہ نصف چادر جن کا چول محور \عددی{z} ہو، اور وہ مخروط جن کا راس  مبدا پر  اور محور محددی نظام کے محور \عددی{z} پر ہو، کو کروی محدد میں بیان کرنا آسان ہوتا ہے۔ ان سطحوں کی مساواتیں درج ذیل ہوں گی۔
\begin{align*}
r&=4&&\text{\RL{کرہ ،جس کا رداس \عددی{4} اور مرکز مبدا پر ہے}}\\
\phi&=\frac{\pi}{3}&&\text{\RL{مبدا سے اوپر رخ کھلتا ہوا مخروط جو مثبت \عددی{z} محور کے ساتھ \عددی{\pi/3} زاویہ بناتا ہے}}\\
\phi&=\frac{\pi}{3}&&\text{\RL{نصف چادر جس کا چول محور \عددی{z} ہے اور جو مثبت \عددی{x} محور کے ساتھ \عددی{\pi/3} زاویہ بناتا ہے}}
\end{align*}

کروی محدد میں  چھوٹے مستطیل حجم  سے  مراد  وہ \اصطلاح{کروی  پچر}\فرہنگ{کروی!پچر}\حاشیہب{spherical wedge}\فرہنگ{spherical!wedge}   ہے جس کو \عددی{\dif r}، \عددی{\dif \phi} اور \عددی{\dif \phi} تعین کرتے ہیں۔ یہ پچر تقریباً مستطیلی ہو گا جس کے ایک اطراف  کی  قوسی لمبائی  \عددی{r\dif\phi}، دوسرے طرف کی قوسی لمبائی  \عددی{r\sin\phi\dif\phi} اور موٹائی  \عددی{\dif r} ہو گی۔ یوں   کروی محدد میں چھوٹے ٹکڑے کا حجم 
\begin{align}
\dif H=r^2\sin\phi \dif r\dif \phi\dif \phi
\end{align}
ہو گا اور تہرا تکمل کی صورت درج ذیل ہو گی۔
\begin{align*}
\iiint F(r,\phi,\phi)\dif H=\iiint F(r,\phi,\phi) r^2\sin\phi\dif r \dif \phi\dif \phi
\end{align*}
ہم عموماً پہلے \عددی{r} کے لحاظ سے تکمل لیتے ہیں۔ ہم  صرف ان تکملات  پر غور کریں گے  جو \عددی{z} محور کے لحاظ سے اجسام طواف  (یا ان کا حصہ) ہوں اور جن کے \عددی{\phi} اور \عددی{\phi} حدیں مستقل ہوں۔

\جزوحصہء{کروی محدد میں تکمل کی قیمت کا حصول}
فضا میں خطہ \عددی{D} پر تکمل
\begin{align*}
\iiint\limits_D F(r,\phi,\phi)\dif H
\end{align*}
کی قیمت حاصل کرتے ہوئے  پہلے \عددی{r}، اس کے بعد \عددی{\phi}، اور آخر میں \عددی{\phi} کے لحاظ سے تکمل لیتے ہوئے ہمیں درج ذیل اقدام کرنے ہوں گے۔
\begin{enumerate}[1.]
\item
\ترچھا{خاکہ:}\quad
خطہ \عددی{D} اور مستوی \عددی{xy} میں \عددی{D} کی تظلیل \عددی{R}  کا خاکہ بنا کر \عددی{D} کی سرحدی سطحوں کی نشاندہی کریں۔
\item
\ترچھا{تکمل کی \عددی{r} حدیں:}\quad
مبدا سے ایک لکیر \عددی{M} کھینچیں جو مثبت محور \عددی{z} کے ساتھ زاویہ \عددی{\phi} بناتی ہو۔ ساتھ ہی \عددی{R} پر \عددی{M} کی تظلیل \عددی{L} کا خاکہ بنائیں جو مثبت \عددی{x} محور کے ساتھ زاویہ \عددی{\phi} بنائی گی۔ جیسے جیسے \عددی{r} بڑھے گا \عددی{M} خطہ \عددی{D} میں \عددی{r=g_1(\phi,\phi)} سے داخل اور \عددی{r=g_2(\phi,\phi)} سے خارج ہو گی۔ یہی تکمل کی \عددی{r} حدیں ہوں گی۔
\item
\ترچھا{تکمل کی \عددی{\phi} حدیں:}\quad
کسی بھی مخصوص \عددی{\phi} کے لئے  \عددی{M} مثبت محور \عددی{z} کے ساتھ \عددی{\phi=\phi_1} سے \عددی{\phi=\phi_2} تک زاویہ بنائے گی۔یہی تکمل کی \عددی{\phi} حدیں ہوں گی۔
\item
\ترچھا{تکمل کی \عددی{\phi} حدیں:}\quad
کسی بھی مخصوص \عددی{\phi} کے لئے \عددی{L}  خطہ \عددی{R} پر جھاڑو کی طرح چلتے ہوئے \عددی{\phi=\alpha} سے \عددی{\phi=\beta} تک چلتی ہے۔  یہی تکمل کی \عددی{\phi} حدیں ہوں گی۔
\end{enumerate}
یوں تکمل درج ذیل ہو گا۔
\begin{align*}
\iiint\limits_D F(r,\phi,\phi)\dif H=\int_{\phi=\alpha}^{\phi=\beta}\int_{\phi_1}^{\phi_2}\int_{r=g_1(\phi,\phi)}^{r=g_2(\phi,\phi)} F(r,\phi,\phi) r^2\sin\phi \dif r\dif \phi \dif \phi
\end{align*} 

\ابتدا{مثال}\شناخت{مثال_بالکثرت_کروی_کرہ_مخروط}
ٹھوس کرہ \عددی{r\le 1} سے مخروط \عددی{\theta=\pi/3} بالائی     خطہ \عددی{D} کاٹتا ہے۔ اس خطہ کا حجم تلاش کریں۔

حل:\quad
اس خطے کا حجم \عددی{\iiint\limits_D r^2\sin\phi\dif r\dif \phi \dif \phi} ہو گا۔تکمل کی قیمت معلوم کرنے کے لئے درج ذیل اقدام کرنے ہوں گے۔
\begin{enumerate}[1.]
\item
\ترچھا{خاکہ:}\quad
ہم \عددی{D} اور مستوی \عددی{xy} میں اس کی تظلیل  \عددی{R} کا خاکہ بناتے ہیں۔
\item
\ترچھا{تکمل کی \عددی{r} حدیں:}\quad
ہم مثبت \عددی{z} محور کے ساتھ \عددی{\phi} زاویہ پر  مبدا سے شعاع \عددی{M}  کھینچتے ہیں اور ساتھ ہی  \عددی{xy} مستوی میں اس کی تظلیل \عددی{L}  کھینچتے ہیں جو مثبت \عددی{x} محور کے ساتھ زاویہ \عددی{\phi} بناتا ہے۔ شعاع \عددی{M} خطہ \عددی{D}میں \عددی{r=0)} سے داخل اور \عددی{r=1} سے خارج ہو گا۔
\item
\ترچھا{تکمل کی \عددی{\phi} حدیں:}\quad
 مخروط \عددی{\theta=\pi/3}  مثبت \عددی{z} محور کے ساتھ  زاویہ \عددی{\pi/3} بناتا ہے۔ یوں شعاع \عددی{M}  زاویہ \عددی{\phi=0} سے \عددی{\phi=\pi/3} تک چل سکتی ہے۔
\item
\ترچھا{تکمل کی \عددی{\phi} حدیں:}\quad
شعاع \عددی{L} خطہ \عددی{R} پر   \عددی{\phi=0} سے \عددی{2\pi} تک چلتی ہے۔
\end{enumerate}
یوں تکمل درج ذیل ہو گا۔
\begin{align*}
H&=\iiint\limits_D r^2\sin\phi\dif r\dif \phi\dif \phi=\int_{0}^{2\pi}\int_{0}^{\pi/3}\int_{0}^{1} r^2\sin\phi\dif r\dif \phi\dif \phi\\
&=\int_{0}^{2\pi}\int_{0}^{\pi/3}\big[\frac{r^3}{3}\big]_0^1 \sin\phi\dif \phi\dif\phi=\int_{0}^{2\pi}\int_{0}^{\pi/3}\frac{1}{3} \sin\phi\dif\phi\dif\phi\\
&=\int_{0}^{2\pi}\big[-\frac{1}{3}\cos\phi\big]_{0}^{\pi/3} \dif\phi=\int_{0}^{2\pi} \big(-\frac{1}{6}+\frac{1}{3}\big)\dif\phi=\frac{1}{6}(2\pi)=\frac{\pi}{3}
\end{align*}
\انتہا{مثال}
%======
\ابتدا{مثال}
مستقل کثافت \عددی{\delta=1} کا ایک ٹھوس جسم مثال \حوالہ{مثال_بالکثرت_کروی_کرہ_مخروط} کے خطہ \عددی{D} میں پایا جاتا ہے۔ محور \عددی{z} کے لحاظ سے اس جسم کا جمودی معیار اثر تلاش کریں۔

حل:\quad
کارتیسی محدد  میں جمودی معیار اثر
\begin{align*}
I_z=\iiint (x^2+y^2)\dif H
\end{align*}
ہو گا۔کروی محدد میں \عددی{x^2+y^2=(r\sin\phi\cos\phi)^2+(r\sin\phi\sin\phi)^2=r^2\sin^2\phi} کی بنا  جمودی معیار اثر
\begin{align*}
I_z&=\iiint (r^2\sin^2\phi)r^2\sin\phi\dif r\dif \phi\dif\phi=\iiint r^4\sin^3\phi\dif r\dif\phi\dif\phi
\end{align*}
ہو گا جس کی قیمت  مثال \حوالہ{مثال_بالکثرت_کروی_کرہ_مخروط} کے خطہ کے لئے درج ذیل ہو گی۔
\begin{align*}
I_z&=\int_{0}^{2\pi}\int_{0}^{\pi/3}\int_{0}^{1}r^4\sin^3\phi\dif r\dif \phi\dif\phi=\int_{0}^{2\pi}\int_{0}^{\pi/3}\big[\frac{r^5}{5}\big]_0^1 \sin^3\phi\dif\phi\dif\phi\\
&=\frac{1}{5}\int_{0}^{2\pi}\int_{0}^{\pi/3}(1-\cos^2\phi)\sin\phi\dif\phi\dif\phi=\frac{1}{5}\int_0^{2\pi} \big[-\cos\phi+\frac{\cos^3\phi}{3}\big]_0^{\pi/3}\dif\phi\\
&=\frac{1}{5}\int_0^{2\pi}\big(-\frac{1}{2}+1+\frac{1}{24}-\frac{1}{3}\big)\dif\phi=\frac{1}{5}\int_0^{2\pi}\frac{5}{24}\dif\phi=\frac{1}{24}(2\pi)=\frac{\pi}{12}
\end{align*}
\انتہا{مثال}
%=====================

\موٹا{محددی بدل کے کلیات}\\
\begin{center}
\begin{tabular}{LLL}
\toprule
\multicolumn{1}{C}{\text{\RL{نلکی سے کارتیسی}}}&\multicolumn{1}{C}{\text{\RL{کروی سے کارتیسی}}}&\multicolumn{1}{C}{\text{\RL{کروی سے نلکی}}}\\
\midrule
x=\rho\cos\phi&x=r\sin\phi\cos\phi&\rho=r\sin\phi\\
y=\rho\sin\phi&y=r\sin\phi\sin\phi&z=r\cos\phi\\
z=z&z=r\cos\phi&\phi=\phi\\
\bottomrule
\end{tabular}
\end{center}

مطابقتی چھوٹے حجم درج ذیل ہیں۔
\begin{align*}
\dif H&=\dif x\dif y\dif z\\
&=\dif z\, \rho \dif \rho \dif \phi\\
&=r^2\sin\phi \dif r\dif \phi\dif \phi
\end{align*}

%===========================
\جزوحصہء{سوالات}
\ابتدا{سوالات}
\موٹا{نلکی محدد}\\
سوال \حوالہ{سوال_بالکثرت_نلکی_تکمل_الف} تا سوال \حوالہ{سوال_بالکثرت_نلکی_تکمل_ب} میں تکمل کی قیمت نلکی محدد استعمال کرتے ہوئے تلاش کریں۔

\ابتدا{سوال}\شناخت{سوال_بالکثرت_نلکی_تکمل_الف}
$\int_{0}^{2\pi}\int_{0}^{1}\int_{\rho}^{\sqrt{2-\rho^2}} \dif z\,\rho \dif\rho\dif\phi $
\انتہا{سوال}
%======================
\ابتدا{جواب}
\wf{\unexpanded{
\(\tfrac{4\pi(\sqrt{2}-1)}{3}\)
}}
\انتہا{جواب}
%======================
\ابتدا{سوال}
$\int_{0}^{2\pi}\int_{0}^{3}\int_{\rho^2/3}^{\sqrt{18-\rho^2}} \dif z\,\rho \dif\rho\dif\phi $
\انتہا{سوال}
%======================
\ابتدا{سوال}
$\int_{0}^{2\pi}\int_{0}^{\phi/2\pi}\int_{0}^{3+24\rho^2} \dif z\,\rho \dif\rho\dif\phi $
\انتہا{سوال}
%======================
\ابتدا{جواب}
\wf{\unexpanded{
\(\tfrac{17\pi}{5}\)
}}
\انتہا{جواب}
%======================
\ابتدا{سوال}
$\int_{0}^{\pi}\int_{0}^{\phi/\pi}\int_{-\sqrt{4-\rho^2}}^{3\sqrt{4-\rho^2}}z \dif z\,\rho \dif\rho\dif\phi $
\انتہا{سوال}
%================
\ابتدا{سوال}
$\int_{0}^{2\pi}\int_{0}^{1}\int_{\rho}^{1/\sqrt{2-\rho^2}}3 \dif z\,\rho \dif\rho\dif\phi $
\انتہا{سوال}
%================
\ابتدا{جواب}
\wf{\unexpanded{
\(\pi(6\sqrt{2}-8)\)
}}
\انتہا{جواب}
%======================
\ابتدا{سوال}\شناخت{سوال_بالکثرت_نلکی_تکمل_ب}
$\int_{0}^{2\pi}\int_{0}^{1}\int_{-1/2}^{1/2}(\rho^2\sin^2\phi+z^2) \dif z\,\rho \dif\rho\dif\phi $
\انتہا{سوال}
%================
اب تک ہم  نلکی محدد کی تکملات  کو پسندیدہ  ترتیب  \عددی{z}، \عددی{\rho}،  \عددی{\phi}   سے حل کرتے آ رہے ہیں۔ بعض اوقات دیگر ترتیبات سے تکمل کا حل زیادہ آسان ہوتا ہے۔ سوال \حوالہ{سوال_بالکثرت_دیگر_نلکی_الف} تا سوال \حوالہ{سوال_بالکثرت_دیگر_نلکی_ب} کے تکملات کی قیمت تلاش کریں۔

\ابتدا{سوال}\شناخت{سوال_بالکثرت_دیگر_نلکی_الف}
$\int_{0}^{2\pi}\int_{0}^{3}\int_{0}^{z/3} \rho^3\dif\rho \dif z\dif \phi$
\انتہا{سوال}
%==================
\ابتدا{جواب}
\wf{\unexpanded{
\(\tfrac{3\pi}{10}\)
}}
\انتہا{جواب}
%======================
\ابتدا{سوال}
$\int_{-1}^{1}\int_{0}^{2\pi}\int_{0}^{1+\cos\phi} 4\rho\dif\rho \dif \rho\dif z$
\انتہا{سوال}
%==================
\ابتدا{سوال}
$\int_{0}^{1}\int_{0}^{\sqrt{z}}\int_{0}^{2\pi}(\rho^2\cos^2\phi+z^2)\rho \dif\phi\dif\rho \dif z$
\انتہا{سوال}
%=====
\ابتدا{جواب}
\wf{\unexpanded{
\(\tfrac{\pi}{3}\)
}}
\انتہا{جواب}
%======================
\ابتدا{سوال}\شناخت{سوال_بالکثرت_دیگر_نلکی_ب}
$\int_{0}^{2}\int_{\rho-2}^{\sqrt{4-\rho^2}}\int_{0}^{2\pi} (\rho\sin\phi+1)\rho \dif \phi \dif z\dif \rho$
\انتہا{سوال}
%===============================
\ابتدا{سوال}\شناخت{سوال_بالکثرت_درکار_خطہ_ملفوف}
نیچے سے مستوی \عددی{z=0}، اوپر سے کرہ \عددی{x^2+y^2+z^2=4}، اور اطراف سے بیلن \عددی{x^2+y^2=1} میں  خطہ \عددی{D}  ملفوف ہے۔نلکی محدد میں  خطہ \عددی{D} کا حجم معلوم کرنے کے لئے تہرا تکمل درج ذیل  تکمل کی ترتیب کے لئے لکھیں۔
\begin{multicols}{3}
\begin{enumerate}[a.]
\item
$\dif z\dif \rho\dif\phi$
\item
$\dif \rho\dif z\dif\phi$
\item
$\dif \phi\dif z\dif\rho$
\end{enumerate}
\end{multicols}
\انتہا{سوال}
%====================
\ابتدا{جواب}
\wf{\unexpanded{
(ا) 
\(\int_{0}^{2\pi}\int_{0}^{1}\int_{0}^{\sqrt{4-\rho^2}}\rho\dif z\dif \rho\dif\phi\)\\
(ب)
\(\int_{0}^{2\pi}\int_{0}^{\sqrt{3}}\int_{0}^{1}\rho\dif\rho\dif z\dif\phi+\int_{0}^{2\pi}\int_{\sqrt{3}}^{2}\int_{0}^{\sqrt{4-z^2}}\rho\dif\rho\dif z\dif\phi\)\\
(ج) 
\(\int_{0}^{1}\int_{0}^{\sqrt{4-\rho^2}}\int_{0}^{2\pi}\rho\dif\phi\dif z\dif \rho\)
}}
\انتہا{جواب}
%======================
\ابتدا{سوال}
نیچے سے مخروط  \عددی{z=\sqrt{x^2+y^2}}، اوپر سے قطع مکافی  \عددی{z=2-x^2-y^2}  میں  خطہ \عددی{D}  ملفوف ہے۔نلکی محدد میں  خطہ \عددی{D} کا حجم معلوم کرنے کے لئے تہرا تکمل درج ذیل  تکمل کی ترتیب کے لئے لکھیں۔
\begin{multicols}{3}
\begin{enumerate}[a.]
\item
$\dif z\dif \rho\dif\phi$
\item
$\dif \rho\dif z\dif\phi$
\item
$\dif \phi\dif z\dif\rho$
\end{enumerate}
\end{multicols}
\انتہا{سوال}
%====================
\ابتدا{سوال}
نیچے سے مستوی \عددی{z=0}،  اطراف سے بیلن \عددی{\rho=\cos\phi}، اور اوپر سے قطع مکافی سطح \عددی{z=3\rho^2} میں ملفوف خطہ \عددی{D} کے لئے درج ذیل تکمل کی قیمت تلاش کرنے کے لئے  کے تکمل کی  حدیں معلوم کریں۔
\begin{align*}
\iiint F(\rho,\phi,z)\dif z\,\rho \dif \rho\dif\phi
\end{align*}
\انتہا{سوال}
%====================
\ابتدا{جواب}
\wf{\unexpanded{
\(\int_{-\pi/2}^{\pi/2}\int_{0}^{\cos\phi}\int_{0}^{3\rho^2}F(\rho,\phi,z) \rho \dif z\dif \rho\dif \phi\)
}}
\انتہا{جواب}
%======================
\ابتدا{سوال}
درج ذیل تکمل کو معادل نلکی محدد کے تکمل میں تبدیل کر  اس کی قیمت تلاش کریں۔
\begin{align*}
\int_{-1}^1 \int_0^{\sqrt{1-y^2}}\int_0^x (x^2+y^2)\dif z\dif x\dif y
\end{align*}
\انتہا{سوال}
%====================

سوال \حوالہ{سوال_بالکثرت_تہرا_نلکی_درکار_الف} تا سوال \حوالہ{سوال_بالکثرت_تہرا_نلکی_درکار_ب}  میں دیے گئے خطہ \عددی{D} پر تکمل \عددی{\iiint_D F(\rho,\phi,z)\dif z\,\rho\dif \rho\dif\phi} کی قیمت حاصل کرنے کے لئے تہرا تکمل لکھیں۔

\ابتدا{سوال}\شناخت{سوال_بالکثرت_تہرا_نلکی_درکار_الف}
وہ     قائمہ دائری بیلن جس کا  قاعدہ  مستوی \عددی{xy} میں  دائرہ \عددی{\rho=2\sin\phi}  اور  سر   مستوی \عددی{z=4-y}میں  ہو، خطہ \عددی{D} ہے۔
\انتہا{سوال}
%===============
\ابتدا{جواب}
\wf{\unexpanded{
\(\int_{0}^{\pi}\int_{0}^{2\sin\phi}\int_{0}^{4-\rho\sin\phi}F(\rho,\phi,z) \dif z\,\rho\dif \rho\dif \phi\)
}}
\انتہا{جواب}
%======================
\ابتدا{سوال}
وہ   قائمہ دائری بیلن جس کا  قاعدہ  مستوی \عددی{xy} میں  دائرہ \عددی{\rho=3\cos \phi}  اور  سر   مستوی \عددی{z=5-x} میں   ہو، خطہ \عددی{D} ہے۔
\انتہا{سوال}
%============
\ابتدا{سوال}
وہ  قائمہ دائری بیلن جس کا  قاعدہ  مستوی \عددی{xy} میں  قلب نما \عددی{\rho=1+\cos\phi}  کے اندر  اور دائرہ \عددی{\rho=1} کے باہر  اور سر  مستوی \عددی{z=4} میں   ہو، خطہ \عددی{D} ہے۔
\انتہا{سوال}
%============
\ابتدا{جواب}
\wf{\unexpanded{
\(\int_{-\pi/2}^{\pi/2}\int_{1}^{1+\cos\phi}\int_{0}^{4}F(\rho,\phi,z) \dif z\,\rho\dif\rho \dif \phi\)
}}
\انتہا{جواب}
%======================
\ابتدا{سوال}
وہ ٹھوس قائمہ بیلن جس کا  قاعدہ دائرہ \عددی{\rho=\cos\phi} اور دائرہ \عددی{\rho=2\cos\phi} کے بیچ اور سر مستوی \عددی{z=3-y} میں ہو،  خطہ \عددی{D} ہے۔
\انتہا{سوال}
%================
\ابتدا{سوال}
وہ منشور جس کا قاعدہ مستوی \عددی{xy} میں  محور \عددی{x}،  لکیر \عددی{y=x} اور لکیر \عددی{x=1} کے بیچ  مثلث   اور  سر مستوی \عددی{z=2-y} میں ہو، خطہ \عددی{D} ہے۔
\انتہا{سوال}
%===============
\ابتدا{جواب}
\wf{\unexpanded{
\(\int_{0}^{\pi/4}\int_{0}^{\sec\phi}\int_{0}^{2-\rho\sin\phi}F(\rho,\phi,z) \dif z\,\rho\dif \rho\dif \phi\)
}}
\انتہا{جواب}
%======================
\ابتدا{سوال}\شناخت{سوال_بالکثرت_تہرا_نلکی_درکار_ب}
وہ منشور جس کا قاعدہ مستوی \عددی{xy} میں  محور \عددی{y}،  لکیر \عددی{y=x} اور لکیر \عددی{y=1} کے بیچ  مثلث  اور  سر مستوی \عددی{z=2-x} میں ہو، خطہ \عددی{D} ہے۔
\انتہا{سوال}
%====================

\موٹا{کروی محدد}\\
سوال \حوالہ{سوال_بالکثرت_کروی_الف} تا سوال \حوالہ{سوال_بالکثرت_کروی_ب} میں کروی تکملات کی قیمت تلاش کریں۔

\ابتدا{سوال}\شناخت{سوال_بالکثرت_کروی_الف}
$\int_{0}^{\pi}\int_{0}^{\pi}\int_{0}^{2\sin\theta} r^2\sin\theta \dif r\dif\theta\dif\phi$
\انتہا{سوال}
%=================
\ابتدا{جواب}
\wf{\unexpanded{
\(\pi^2 \)
}}
\انتہا{جواب}
%======================
\ابتدا{سوال}
$\int_{0}^{2\pi}\int_{0}^{\pi/4}\int_{0}^{2} (r\cos\theta) r^2\sin\theta\dif r\dif\theta\dif\phi$
\انتہا{سوال}
%=================
\ابتدا{سوال}
$\int_{0}^{2\pi}\int_{0}^{\pi}\int_{0}^{(1-\cos\theta)/2} r^2\sin\theta\dif r\dif\theta\dif\phi$
\انتہا{سوال}
%=================
\ابتدا{جواب}
\wf{\unexpanded{
\(\pi/3 \)
}}
\انتہا{جواب}
%======================
\ابتدا{سوال}
$\int_{0}^{3\pi/2}\int_{0}^{\pi}\int_{0}^{1} 5 r^3\sin^3\theta\dif r\dif\theta\dif\phi$
\انتہا{سوال}
%=================
\ابتدا{سوال}
$\int_{0}^{2\pi}\int_{0}^{\pi/3}\int_{\sec\theta}^{2} 3 r^2\sin\theta\dif r\dif\theta\dif\phi$
\انتہا{سوال}
%=================
\ابتدا{جواب}
\wf{\unexpanded{
\(5\pi \)
}}
\انتہا{جواب}
%======================
\ابتدا{سوال}\شناخت{سوال_بالکثرت_کروی_ب}
$\int_{0}^{2\pi}\int_{0}^{\pi/4}\int_{0}^{\sec\theta} (r\cos\theta) r^2\sin\theta\dif r\dif\theta\dif\phi$
\انتہا{سوال}
%=================

اب تک ہم   کروی  محدد کی تکملات  کو پسندیدہ  ترتیب   سے حل کرتے آ رہے ہیں۔ بعض اوقات دیگر ترتیبات سے تکمل کا حل زیادہ آسان ہوتا ہے۔سوال \حوالہ{سوال_بالکثرت_کروی_دیگر_الف} تا سوال \حوالہ{سوال_بالکثرت_کروی_دیگر_ب} میں تکملات کی قیمت تلاش کریں۔

\ابتدا{سوال}\شناخت{سوال_بالکثرت_کروی_دیگر_الف}
$\int_{0}^{2}\int_{-\pi}^{0}\int_{\pi/4}^{\pi/2} r^3 \sin 2\theta \dif \theta\dif \phi\dif r$
\انتہا{سوال}
%================
\ابتدا{جواب}
\wf{\unexpanded{
\(2\pi \)
}}
\انتہا{جواب}
%======================
\ابتدا{سوال}
$\int_{\pi/6}^{\pi/3}\int_{\csc\theta}^{2\csc\theta}\int_{0}^{2\pi} r^2\sin\theta\dif\phi \dif r\dif \theta$
\انتہا{سوال}
%================
\ابتدا{سوال}
$\int_{0}^{1}\int_{0}^{\pi}\int_{0}^{\pi/4} 12r\sin^3\theta\dif \theta\dif \phi\dif r$
\انتہا{سوال}
%================
\ابتدا{جواب}
\wf{\unexpanded{
\((\tfrac{8-5\sqrt{2}}{2})\pi\)
}}
\انتہا{جواب}
%======================
\ابتدا{سوال}\شناخت{سوال_بالکثرت_کروی_دیگر_ب}
$\int_{\pi/6}^{\pi/2}\int_{-\pi/2}^{\pi/2}\int_{\csc\theta}^{2}5r^4\sin^3\theta \dif r\dif \phi\dif \theta$
\انتہا{سوال}
%================
\ابتدا{سوال}
کروی محدد میں (ا) \عددی{\dif r\dif\theta\dif\phi}، (ب) \عددی{\dif\theta\dif r\dif \phi} ترتیب سے سوال \حوالہ{سوال_بالکثرت_درکار_خطہ_ملفوف}  کے خطہ کے حجم  کے  تہرا  تکمل لکھیں۔ 
\انتہا{سوال}
%==================
\ابتدا{جواب}
\wf{\unexpanded{
(ا) 
\(\int_{0}^{2\pi}\int_{0}^{\pi/6}\int_{0}^{2}r^2\sin\theta \dif r\dif \theta\dif \phi+\int_{0}^{2\pi}\int_{\pi/6}^{\pi/2}\int_{0}^{\csc\theta} r^2\sin\theta\dif r\dif \theta\dif \phi\)\\
(ب) 
\(\int_{0}^{2\pi}\int_{1}^{2}\int_{\pi/6}^{\sin^{-1}(1/r)} r^2\sin\theta\dif \theta\dif r\dif\phi+\int_{0}^{2\pi}\int_{0}^{2}\int_{0}^{\pi/6}r^2\sin\theta\dif \theta\dif r\dif\phi \)
}}
\انتہا{جواب}
%======================
\ابتدا{سوال}
نیچے سے مخروط   \عددی{z=\sqrt{x^2+y^2}} اور اوپر سے مستوی \عددی{z=1}  کے بیچ خطہ \عددی{D}  کے حجم کا تکمل کروی محدد میں(ا) \عددی{\dif r\dif\theta\dif\phi} ، (ب) \عددی{\dif\theta\dif r\dif \phi} ترتیب    کے لئے  لکھیں۔
\انتہا{سوال}
%=================
سوال \حوالہ{سوال_بالکثرت_کروی_حجم_تلاش_الف} تا سوال \حوالہ{سوال_بالکثرت_کروی_حجم_تلاش_ب} میں  دئے گئے ٹھوس جس کے حجم کے کروی تکمل (ا) کی حدیں تلاش کریں۔ (ب) کروی تکمل حل کرتے  ہوئے جسم کا حجم معلوم کریں۔

\ابتدا{سوال}\شناخت{سوال_بالکثرت_کروی_حجم_تلاش_الف}
کرہ \عددی{r=\cos\theta} اور نصف کرہ \عددی{r=2,\,z\ge 0} کے بیچ ٹھوس جسم۔
\انتہا{سوال}
%==================
\ابتدا{جواب}
\wf{\unexpanded{
\(\int_{0}^{2\pi}\int_{0}^{\pi/2}\int_{\cos\theta}^{2} r^2\sin\theta \dif r\dif \theta\dif\phi =\tfrac{31\pi}{6}\)
}}
\انتہا{جواب}
%======================
\ابتدا{سوال}
نیچے سے نصف کرہ \عددی{r=1,\,z\ge 0} اور اوپر سے سطح طواف قلب نما \عددی{r=1+\cos\theta}  میں ملفوف  ٹھوس جسم۔
\انتہا{سوال}
%===================
\ابتدا{سوال}\شناخت{سوال_بالکثرت_کروی_حجم_درکار_پ}
جسم طواف قلب نما \عددی{r=1-\cos\theta}   میں ملفوف۔
\انتہا{سوال}
%====================
\ابتدا{جواب}
\wf{\unexpanded{
\(\int_{0}^{2\pi}\int_{0}^{\pi}\int_{0}^{1-\cos\theta}r^2\sin\theta \dif r\dif \theta\dif \phi=\tfrac{8\pi}{3}\)
}}
\انتہا{جواب}
%======================
\ابتدا{سوال}
وہ  بالائی خطہ جو   سوال \حوالہ{سوال_بالکثرت_کروی_حجم_درکار_پ} کے جسم سے مستوی \عددی{xy}   کاٹتا ہے۔
\انتہا{سوال}
%======================
\ابتدا{سوال}
نیچے سے کرہ \عددی{r=2\cos\theta}  اور اوپر سے مخروط \عددی{z=\sqrt{x^2+y^2}} میں ملفوف جسم۔
\انتہا{سوال}
%================
\ابتدا{جواب}
\wf{\unexpanded{
\(\int_{0}^{2\pi}\int_{\pi/4}^{\pi/2}\int_{0}^{2\cos\theta}r^2\sin\theta \dif r\dif \theta\dif \phi=\tfrac{\pi}{3}\)
}}
\انتہا{جواب}
%======================
\ابتدا{سوال}\شناخت{سوال_بالکثرت_کروی_حجم_تلاش_ب}
نیچے سے مستوی \عددی{xy}، اوپر سے  مخروط \عددی{\theta=\tfrac{\pi}{3}} اور اطراف سے کرہ \عددی{r=2} میں ملفوف
\انتہا{سوال}
%===========

\موٹا{کارتیسی، نلکی اور کروی محدد}\\
\ابتدا{سوال}
کرہ \عددی{r=2} کے حجم کا تہرا تکمل (ا) کروی ، (ب) نلکی، اور (ج) کارتیسی محدد میں لکھیں۔
\انتہا{سوال}
%===========
\ابتدا{جواب}
\wf{\unexpanded{
(ا)
\(8\int_{0}^{\pi/2}\int_{0}^{\pi/2}\int_{0}^{2}r^2\sin\theta \dif r\dif \theta\dif \phi\)\\
(ب)
\(8\int_{0}^{\pi/2}\int_{0}^{2}\int_{0}^{\sqrt{4-\rho^2}}\rho\dif z\dif \rho\dif \phi\)\\
(ج)
\(8\int_{0}^{2}\int_{0}^{\sqrt{4-x^2}}\int_{0}^{\sqrt{4-x^2-y^2}}\dif z\dif y\dif x\)
}}
\انتہا{جواب}
%======================
\ابتدا{سوال}
ثُمن اول میں نیچے سے مخروط \عددی{\theta=\tfrac{\pi}{4}} اور اوپر سے کرہ \عددی{r=3} میں ملفوف خطہ \عددی{D} کے حجم کا تہرا تکمل (ا) نلکی اور  (ب) کروی محدد میں لکھیں۔ (ج)  اس کے بعد اس جسم کا حجم تلاش کریں۔
\انتہا{سوال}
%================
\ابتدا{سوال}
رداس \عددی{2} اکائیاں   کے کرہ کو ،کرہ سے مرکز سے \عددی{1} اکائی دور، مستوی دو ٹکڑوں میں کاٹتی ہے۔  چھوٹے ٹکڑے  کے حجم کا تہرا تکمل (ا) کروی، (ب) نلکی، اور (ج) کارتیسی محدد میں لکھیں۔ (د) اس ٹکڑے کا حجم کسی ایک تہرا تکمل کو حل کرتے ہوئے   معلوم کریں۔
\انتہا{سوال}
%==================
\ابتدا{جواب}
\wf{\unexpanded{
(ا)
\(\int_{0}^{2\pi}\int_{0}^{\pi/3}\int_{\sec\theta}^{2}r^2\sin\theta \dif r\dif \theta\dif \phi\)\\
(ب)
\(\int_{0}^{2\pi}\int_{0}^{\sqrt{3}}\int_{1}^{\sqrt{4-\rho^2}}\rho\dif z\dif \rho\dif \phi\)\\
(ج)
\(\int_{-\sqrt{3}}^{\sqrt{3}}\int_{-\sqrt{3-x^2}}^{\sqrt{3-x^2}}\int_{1}^{\sqrt{4-x^2-y^2}}\dif z\dif y\dif x\)\\
(د)
\(\tfrac{5\pi}{3}\)
}}
\انتہا{جواب}
%======================
\ابتدا{سوال}
ٹھوس نصف کرہ \عددی{x^2+y^2+z^2\le 1,\, z\ge 0} کے جمودی معیار اثر \عددی{I_z} کو (ا) نلکی اور (ب) کروی محدد میں لکھیں۔ (ج) \عددی{I_z} کی قیمت تلاش کریں۔
\انتہا{سوال}
%==================

\موٹا{حجم}\\
سوال \حوالہ{سوال_بالکثرت_ٹھوس_جسم_حجم_الف} تا سوال \حوالہ{سوال_بالکثرت_ٹھوس_جسم_حجم_ب} میں ٹھوس اجسام کے حجم تلاش کریں۔

\ابتدا{سوال}\شناخت{سوال_بالکثرت_ٹھوس_جسم_حجم_الف}
\انتہا{سوال}
%============================
\ابتدا{جواب}
\wf{\unexpanded{
\(8\pi/3\)
}}
\انتہا{جواب}
%======================
\ابتدا{سوال}
\انتہا{سوال}
%============================
\ابتدا{سوال}
\انتہا{سوال}
%============================
\ابتدا{جواب}
\wf{\unexpanded{
\(9/4\)
}}
\انتہا{جواب}
%======================
\ابتدا{سوال}
\انتہا{سوال}
%============================
\ابتدا{سوال}
\انتہا{سوال}
%============================
\ابتدا{جواب}
\wf{\unexpanded{
\((3\pi-4)/18\)
}}
\انتہا{جواب}
%======================
\ابتدا{سوال}\شناخت{سوال_بالکثرت_ٹھوس_جسم_حجم_ب}
\انتہا{سوال}
%============================
\ابتدا{سوال}
مخروط \عددی{\theta=\tfrac{\pi}{3}} اور \عددی{\theta=\tfrac{2\pi}{3}} کے بیچ ٹھوس  کرہ \عددی{r\le a} کے حصہ  کا حجم تلاش کریں۔
\انتہا{سوال}
%===================
\ابتدا{جواب}
\wf{\unexpanded{
\(\tfrac{2\pi a^3}{3}\)
}}
\انتہا{جواب}
%======================
\ابتدا{سوال}
ثُمن اول میں نصف مستویات \عددی{\phi=0} اور \عددی{\phi=\tfrac{\pi}{6}} کے بیچ  ٹھوس کرہ  \عددی{r\le a}  کے حصہ کا حجم تلاش کریں۔
\انتہا{سوال}
%=================
\ابتدا{سوال}
ٹھوس کرہ \عددی{r\le 2} سے مستوی \عددی{z=1} جو چھوٹا ٹکڑا کاٹتا ہے، اس کا حجم تلاش کریں۔
\انتہا{سوال}
%===================
\ابتدا{جواب}
\wf{\unexpanded{
\(\tfrac{5\pi}{3}\)
}}
\انتہا{جواب}
%======================
\ابتدا{سوال}
مستویات \عددی{z=1} اور \عددی{z=2} کے بیچ مخروط \عددی{z=\sqrt{x^2+y^2}} کے حصہ کا حجم تلاش کریں۔
\انتہا{سوال}
%=============
\ابتدا{سوال}
نیچے سے مستوی \عددی{z=0}، اوپر سے سطح قطع مکافی \عددی{z=x^2+y^2} اور اطراف سے بیلن \عددی{x^2+y^2=1} میں ملفوف خطے کا حجم تلاش کریں۔
\انتہا{سوال}
%====================
\ابتدا{جواب}
\wf{\unexpanded{
\(\pi/2\)
}}
\انتہا{جواب}
%======================
\ابتدا{سوال}
نیچے سے سطح قطع مکافی  \عددی{z=x^2+y^2}، اوپر سے سطح قطع مکافی \عددی{z=1+x^2+y^2} اور اطراف سے بیلن \عددی{x^2+y^2=1} میں ملفوف خطے کا حجم تلاش کریں۔
\انتہا{سوال}
%====================
\ابتدا{سوال}
موٹی  دیوار کے بیلن \عددی{1\le x^2+y^2\le 2}  سے مخروط \عددی{z=\mp\sqrt{x^2+y^2}}  جتنا حصہ کاٹتے ہیں، اس کا حجم تلاش کریں۔
\انتہا{سوال}
%=============
\ابتدا{جواب}
\wf{\unexpanded{
\(\tfrac{4(2\sqrt{2}-1)\pi}{3}\)
}}
\انتہا{جواب}
%======================
\ابتدا{سوال}
کرہ \عددی{x^2+y^2+z^2=2} کے اندر اور بیلن \عددی{x^2+y^2=1} کے باہر خطے کا حجم تلاش کریں۔
\انتہا{سوال}
%==========
\ابتدا{سوال}
بیلن \عددی{x^2+y^2=4} اور مستویات \عددی{z=0} اور \عددی{y+z=4} میں ملفوف خطے کا حجم تلاش کریں۔
\انتہا{سوال}
%=========
\ابتدا{جواب}
\wf{\unexpanded{
\(16\pi\)
}}
\انتہا{جواب}
%======================
\ابتدا{سوال}
بیلن \عددی{x^2+y^2=4} اور مستویات \عددی{z=0} اور \عددی{x+y+z=4} میں ملفوف خطے کا حجم تلاش کریں۔
\انتہا{سوال}
%=========
\ابتدا{سوال}
اوپر سے سطح قطع مکافی \عددی{z=5-x^2-y^2} اور نیچے سے سطح قطع مکافی \عددی{z=4x^2+4y^2} میں ملفوف خطے کا حجم تلاش کریں۔
\انتہا{سوال}
%================
\ابتدا{جواب}
\wf{\unexpanded{
\(5\pi/2\)
}}
\انتہا{جواب}
%======================
\ابتدا{سوال}
 بیلن \عددی{x^2+y^2=1} سے باہر،  اوپر سے سطح قطع مکافی \عددی{z=9-x^2-y^2} اور  نیچے سے مستوی \عددی{xy} میں ملفوف    خطے  کا حجم تلاش کریں۔
\انتہا{سوال}
%================
\ابتدا{سوال}
اس خطے کا حجم تلاش کریں جسے ٹھوس بیلن \عددی{x^2+y^2\le 1}  کرہ \عددی{x^2+y^2+z^2=4} سے کاٹتا ہے۔
\انتہا{سوال}
%=========
\ابتدا{جواب}
\wf{\unexpanded{
\(\tfrac{4\pi(8-3\sqrt{3})}{3}\)
}}
\انتہا{جواب}
%======================
\ابتدا{سوال}
اوپر سے کرہ \عددی{x^2+y^2+z^2=2} اور نیچے سے سطح قطع مکافی \عددی{z=x^2+y^2} میں ملفوف خطے کا حجم تلاش کریں۔
\انتہا{سوال}
%================

\موٹا{اوسط قیمت}\\

\ابتدا{سوال}
مستویات \عددی{z=-1} اور \عددی{z=1} کے بیچ بیلن \عددی{\rho=1}  میں تفاعل \عددی{F(\rho,\phi,z)=\rho} کی اوسط قیمت    تلاش کریں۔
\انتہا{سوال}
%================
\ابتدا{جواب}
\wf{\unexpanded{
\(2/3\)
}}
\انتہا{جواب}
%======================
\ابتدا{سوال}
کرہ \عددی{\rho^2+z^2=1} (یعنی کرہ \عددی{x^2+y^2+z^2=1})  کے اندر  تفاعل \عددی{F(\rho,\phi,z)=\rho} کی اوسط قیمت تلاش کریں۔
\انتہا{سوال}
%===============
\ابتدا{سوال}
ٹھوس گیند \عددی{r\le 1}  میں تفاعل \عددی{F(r,\theta,\phi)=r} کی اوسط قیمت تلاش کریں۔
\انتہا{سوال}
%========
\ابتدا{جواب}
\wf{\unexpanded{
\(3/4\)
}}
\انتہا{جواب}
%======================
\ابتدا{سوال}
بالائی نصف ٹھوس کرہ \عددی{r\le 1,\, 0\le\theta\le\pi/2}  میں تفاعل \عددی{F(r,\theta,\phi)=r\cos\theta} کی اوسط قیمت تلاش کریں۔
\انتہا{سوال}
%==========

\موٹا{کمیت، معیار اثر، اور وسطانی مراکز}\\
\ابتدا{سوال}
نیچے سے مستوی \عددی{z=0}، اوپر سے مخروط \عددی{z=\rho,\,\rho\ge 0}، اور اطراف سے بیلن \عددی{\rho=1} میں ملفوف  مستقل کثافت کے ٹھوس جسم کا  مرکز کمیت تلاش کریں۔
\انتہا{سوال}
%================
\ابتدا{جواب}
\wf{\unexpanded{
\(\bar{x}=\bar{y}=0,\,\bar{z}=3/8\)
}}
\انتہا{جواب}
%======================
\ابتدا{سوال}
ثُمن اول میں اوپر سے مخروط \عددی{z=\sqrt{x^2+y^2}}، نیچے سے مستوی \عددی{z=0}، اور اطراف سے بیلن \عددی{x^2+y^2=4} اور مستویات \عددی{x=0} اور \عددی{y=0} میں ملفوف  خطے کا وسطانی مرکز تلاش کریں۔
\انتہا{سوال}
%===============
\ابتدا{سوال}
اس ٹھوس جسم کا وسطانی مرکز تلاش کریں جو سوال \حوالہ{سوال_بالکثرت_کروی_حجم_تلاش_ب} میں دیا گیا ہے۔ 
\انتہا{سوال}
%============
\ابتدا{جواب}
\wf{\unexpanded{
\((\bar{x},\bar{y},\bar{z})=(0,0,3/8)\)
}}
\انتہا{جواب}
%======================
\ابتدا{سوال}
اوپر سے کرہ \عددی{r=a} اور نیچے سے  مخروط \عددی{\theta=\tfrac{\pi}{4}} کے بیچ ٹھوس جسم کا وسطانی مرکز تلاش کریں۔
\انتہا{سوال}
%================
\ابتدا{سوال}
اوپر سے سطح \عددی{z=\sqrt{\rho}}، نیچے سے مستوی \عددی{xy}، اور اطراف سے  بیلن \عددی{\rho=4} میں ملفوف ٹھوس جسم کا وسطانی مرکز تلاش کریں۔
\انتہا{سوال}
%==============
\ابتدا{جواب}
\wf{\unexpanded{
\(\bar{x}=\bar{y}=0,\,\bar{z}=5/6\)
}}
\انتہا{جواب}
%======================
\ابتدا{سوال}
اس خطے کا وسطانی مرکز تلاش کریں جو نصف مستویات \عددی{\phi=-\pi/3,\,\rho\ge 0} اور \عددی{\phi=\pi/3,\,\rho\ge 0} ٹھوس گیند \عددی{\rho^2+z^2\le 1}  سے  کاٹتے ہیں۔
\انتہا{سوال}
%==========
\ابتدا{سوال}
قائمہ دائری موٹی دیوار کے  بیلن   کی اندرونی سطح بیلن \عددی{\rho=1} اور بیرونی سطح بیلن \عددی{\rho=2} ہیں۔اس کا نچلا سر مستوی \عددی{z=0} اور بالائی سر  مستوی \عددی{z=4}  میں پایا جاتا ہے۔ محور \عددی{z} کے لحاظ سے  اس  کا جمودی معیار اثر اور رداس دوار تلاش کریں (\عددی{\delta=1} لیں)۔
\انتہا{سوال}
%======================
\ابتدا{جواب}
\wf{\unexpanded{
\(I_z=30\pi,\, R_z=\sqrt{\tfrac{5}{2}}\)
}}
\انتہا{جواب}
%======================
\ابتدا{سوال}
ایک قائمہ دائری بیلن کا رداس \عددی{1} اور قد \عددی{2} ہے۔ (ا) بیلن کے محور،  (ب) بیلن کے وسطانی مرکز سے گزرتی ہوئے لکیر جو بیلن کے محور کو عمودی ہو،  کے لحاظ سے بیلن کا جمودی معیار اثر تلاش کریں (\عددی{\delta=1} لیں)۔
\انتہا{سوال}
%============================
\ابتدا{سوال}
ایک قائمہ دائری  مخروط کا   رداس قاعدہ  \عددی{1} اور قد \عددی{1} ہے۔مخروط کے راس سے گزرتی ہوئی لکیر جو مخروط کے محور کو عمودی ہے کے لحاظ سے مخروط کا جمودی معیار اثر تلاش کریں  (\عددی{\delta=1} لیں)۔
\انتہا{سوال}
%===========================
\ابتدا{جواب}
\wf{\unexpanded{
\(I_x=\pi/4\)
}}
\انتہا{جواب}
%======================
\ابتدا{سوال}
رداس \عددی{a} کے کرہ کا جمودی معیار اثر کرہ کے قطر کے لحاظ سے تلاش کریں  (\عددی{\delta=1} لیں)۔
\انتہا{سوال}
%====================
\ابتدا{سوال}
ایک قائمہ دائری مخروط کا رداس قاعدہ \عددی{a} اور قد \عددی{h} ہے۔ اس کا جمودی معیار اثر مخروط کے محور کے لحاظ سے تلاش کریں۔ (اشارہ: مخروط کے محور کو محور \عددی{z} اور راس کو مبدا پر رکھیں ۔)
\انتہا{سوال}
%=====================
\ابتدا{جواب}
\wf{\unexpanded{
\(\tfrac{a^4h\pi}{10}\)
}}
\انتہا{جواب}
%======================
\ابتدا{سوال}
ایک ٹھوس جسم اوپر سے قطع مکافی سطح \عددی{z=\rho^2}، نیچے سے مستوی \عددی{z=0}،  اور اطراف سے بیلن \عددی{\rho=1} میں ملفوف ہے۔ اس کا مرکز کمیت اور محور \عددی{z} کے لحاظ سے جمودی معیار اثر اور رداس دوار تلاش کریں جہاں جسم کی کثافت  (ا) \عددی{\delta(\rho,\phi,z)=z}، (ب) \عددی{\delta(\rho,\phi,z)=\rho} ہے۔
\انتہا{سوال}
%=======================
\ابتدا{سوال}
ایک ٹھوس جسم نیچے سے مخروط \عددی{z=\sqrt{x^2+y^2}} اور اوپر سے مستوی \عددی{z=1}  میں ملفوف ہے۔ اس کا مرکز کمیت اور محور \عددی{z} کے لحاظ سے جمودی معیار اثر اور رداس دوار تلاش کریں  جہاں جسم کی کثافت (ا) \عددی{\delta(\rho,\phi,z)=z}، (ب) \عددی{\delta(\rho,\phi,z)=z^2} ہے۔
\انتہا{سوال}
%==================
\ابتدا{جواب}
\wf{\unexpanded{
(ا)
\((\bar{x},\bar{y},\bar{z})=(0,0,4/5),\, I_z=\pi/12\)\\
\(R_z=\sqrt{1/3}\)
(ب)
\((\bar{x},\bar{y},\bar{z})=(0,0,5/6)\)\\
\( I_z=\pi/14,\,R_z=\sqrt{5/14}\)
}}
\انتہا{جواب}
%======================
\ابتدا{سوال}
ایک ٹھوس گیند کا رداس \عددی{r=a} ہے اور  کثافت (ا) \عددی{\delta(r,\theta,\phi)=r^2}، (ب) \عددی{\delta(r,\theta,\phi)=\rho=r\sin\theta} ہے۔ محور \عددی{z} کے لحاظ سے اس گیند کا جمودی معیار اثر تلاش کریں۔
\انتہا{سوال}
%===================
\ابتدا{سوال}
دکھائیں کہ ایک نیم  ترخیمی سطح طواف \عددی{\tfrac{\rho^2}{a^2}+\tfrac{z^2}{h^2}\le 1,\, z\ge 0}  کا وسطانی مرکز محور \عددی{z} پر  قاعدہ سے  سر جانب  تین آٹھواں فاصلے پر ہے۔ بالخصوص \عددی{h=a} ایک  ٹھوس نصف کرہ دیتا ہے۔یوں ٹھوس نصف کرہ کا وسطانی مرکز قاعدہ سے سر جانب تین آٹھواں فاصلے پر ہو گا۔
\انتہا{سوال}
%======================
\ابتدا{سوال}
دکھائیں کہ ایک قائمہ دائری  ٹھوس مخروط کا وسطانی مرکز محور پر قاعدہ سے راس جانب ایک  چوتھائی فاصلے  پر ہو گا۔ (عمومی طور پر مخروط اور  اہرام کا وسطانی مرکز قاعدہ سے راس جانب ایک چوتھائی فاصلے  پر ہو گا)۔
\انتہا{سوال}
%====================
\ابتدا{سوال}
رداس \عددی{\rho=a} کا ایک قائمہ دائری بیلن  مستویات \عددی{z=0} اور \عددی{z=h,\, h>0} کے بیچ پایا جاتا ہے۔ اس کی کثافت \عددی{\delta(\rho,\phi,z)=z+1} ہے۔ اس کا مرکز کمیت اور محور \عددی{z} کے لحاظ سے جمودی معیار اثر اور رداس دوار تلاش کریں۔
\انتہا{سوال}
%================
\ابتدا{جواب}
\wf{\unexpanded{
\((\bar{x},\bar{y},\bar{z})=(0,0,\tfrac{2h^2+3h}{3h+6})\)\\
\( I_z=\tfrac{\pi a^4(h^2+2h)}{4},\, R_z=\tfrac{a}{\sqrt{2}}\)
}}
\انتہا{جواب}
%======================
\ابتدا{سوال}
 رداس \عددی{R} کے ایک سیارہ پر  ہوا کی کثافت \عددی{\mu=\mu_0e^{-ch}} ہے جہاں سیارہ کی سطح سے بلندی  \عددی{h} ہے جبکہ سیارہ کی سطح پر ہوا کی کثافت \عددی{\mu_0} ہے اور \عددی{c} ایک مثبت  مستقل ہے۔ سیارہ میں ہوا کی کمیت تلاش کریں۔
\انتہا{سوال}
%=================
\ابتدا{سوال}
ایک سیارہ کا رداس \عددی{R} اور کمیت \عددی{M} ہے۔ اس کی کثافت کروی تشاکلی ہے جو  سطح سے مرکز تک خطی بڑھتی ہے۔ سیارہ کی سطحی کثافت صفر لیتے ہوئے  اس  کے مرکز پر کثافت تلاش کریں۔
\انتہا{سوال}
%===========
\ابتدا{جواب}
\wf{\unexpanded{
\(\tfrac{3M}{\pi R^3}\)
}}
\انتہا{جواب}
%======================
\انتہا{سوالات}
%=================================

\حصہ{تکملات بالکثرت میں بدل}\شناخت{حصہ_دہرا_تکملات_بالکثرت_میں_بدل}
اس حصہ میں بارہا تکمل کی قیمت کا حصول بذریعہ بدل سکھایا جائے گا۔ واحد تکمل کی طرح یہاں بھی پیچیدہ تکمل کو سادہ تکمل سے بدلا  جاتا  ہے۔ بدل سے متکمل  یا تکمل کی حدوں یا ان دونوں  کی سادہ  روپ استعمال کی جاتی ہے۔


\جزوحصہء{دوہرا تکملات  میں بدل} 


ہم قطبی محدد د کی بدل کا استعمال حصہ \حوالہ{حصہ_بالکثرت_دوہرا_تکملات_قطبی_روپ} میں  دیکھ چکے ہیں جو دہرا تکملات کی بدل، جس میں متغیرات کی تبدیلی کو خطے کی تبدیلی تصور کیا جاتا ہے، کی ایک  مخصوص شکل ہے۔ 


فرض کریں مستوی \عددی{uv} کے خطہ \عددی{G} کو ایک ایک مطابقت کے ساتھ مساوات
\[x=g(u,v),\quad y=h(u,v)\]
 کے ذریعہ مستوی \عددی{xy} کے خطہ \عددی{R} میں بدلا جاتا ہے۔ ہم \عددی{R} کو اس بدل میں \عددی{G} کا \اصطلاح{عکس}\فرہنگ{عکس}\حاشیہب{image}\فرہنگ{image} اور \عددی{G} کو \عددی{R} کا  \اصطلاح{قبل عکس}\فرہنگ{عکس!قبل}\حاشیہب{preimage}\فرہنگ{image!pre} کہتے ہیں۔خطہ \عددی{R} کسی بھی تفاعل \عددی{f(x,y)} کو خطہ \عددی{G} میں معین تفاعل \عددی{f(g(u,v),h(u,v))} بھی تصور کیا جا سکتا ہے۔ خطہ \عددی{R} میں \عددی{f(x,y)} کے تکمل کا خطہ \عددی{G} میں \عددی{f(g(u,v),h(u,v))} کے تکمل کے ساتھ کیا تعلق ہو گا؟


اس کا جواب: اگر \عددی{g}، \عددی{h} اور \عددی{f} کے جزوی تفرقات استمراری ہوں اور \عددی{J(u,v)} (جس پر جلد تبصرہ کیا جائے گا) صرف  تنہا نقطوں پر صفر ہو (اگر صفر ہو بھی)  تب درج ذیل ہو گا۔ 
\begin{align}\label{مساوات_بالکثرت_یعقوبی_بدل}
\iint\limits_{R} f(x,y)\dif x\dif y=\iint\limits_G f(g(u,v),h(u,v))\abs{J(u,v)}\dif u\dif v
\end{align}
مذکورہ بالا مساوات میں \عددی{J(u,v)}، جو \ترچھا{ یعقوبی} کہلاتا ہے، کی مطلق قیمت استعمال کی گئی۔


\ابتدا{تعریف}
  \اصطلاح{یعقوبی مقطع}\فرہنگ{یہ ریاضی دان کارل   گستاف یعقوب یعقوبی کے نام سے منسوب ہے۔} یا محددی بدل \عددی{x=g(u,v)}، \عددی{y=h(u,v)} کے \اصطلاح{یعقوبی}\فرہنگ{یعقوبی}\حاشیہب{Jacobian}\فرہنگ{Jacobian} سے مراد درج ذیل ہے:
\begin{align}\label{مساوات_بالکثرت_یعقوبی_تعریف}
\renewcommand{\arraystretch}{1.5}
J(u,v)=\begin{vmatrix}
\frac{\partial x}{\partial u}&\frac{\partial x}{\partial v}\\
\frac{\partial y}{\partial u}&\frac{\partial y}{\partial v}
\end{vmatrix}=
\frac{\partial x}{\partial u}\frac{\partial y}{\partial v}-\frac{\partial y}{\partial u}\frac{\partial x}{\partial v}
\end{align}
\انتہا{تعریف}


یعقوبی کو 
\begin{align*}
J(u,v)=\frac{\partial(x,y)}{\partial(u,v)}
\end{align*}
سے بھی ظاہر کیا جاتا ہے جو ہمیں یاد دلاتا ہے کہ \عددی{x} اور \عددی{y} کی جزوی تفرقات سے یعقوبی (مساوات \حوالہ{مساوات_بالکثرت_یعقوبی_تعریف})  حاصل ہوتا ہے۔مساوات \حوالہ{مساوات_بالکثرت_یعقوبی_بدل} کی استخراج آپ کو اعلٰی احصاء کے نصاب میں ملے گی جس کو یہاں پیش نہیں کیا جائے گا۔






قطبی محدد میں میں \عددی{u} اور \عددی{v} کی جگہ \عددی{r} اور \عددی{\theta} ہوں گے لہٰذا \عددی{x=r\cos\theta} اور \عددی{y=r\sin\theta} لیتے ہوئے یعقوبی
\begin{align*}
\renewcommand{\arraystretch}{1.5}
J(r,\theta)=\begin{vmatrix}
\frac{\partial x}{\partial r}&\frac{\partial x}{\partial \theta}\\
\frac{\partial y}{\partial r}&\frac{\partial y}{\partial \theta}
\end{vmatrix}=\begin{vmatrix}
\cos\theta&-r\sin\theta\\
\sin\theta&r\cos\theta
\end{vmatrix}=
r(\cos^2\theta+\sin^2\theta)=r
\end{align*}
ہو گا اور مساوات \حوالہ{مساوات_بالکثرت_یعقوبی_بدل} درج ذیل صورت اختیار کرے گی جو حصہ  \حوالہ{حصہ_بالکثرت_دوہرا_تکملات_قطبی_روپ}کی مساوات \حوالہ{مساوات_بالکثرت_قطبی_محدد_میں_رقبہ_کی_عمومی} ہے۔
\begin{gather}
\begin{aligned}\label{مساوات_بالکثرت_قطبی_مستطیل_تکمل}
\iint\limits_R f(x,y)\dif x\dif y&=\iint\limits_G f(r\cos\theta,r\sin\theta)\abs{r}\dif r \dif \theta\\
&=\iint\limits_G f(r\cos\theta,r\sin\theta)r\dif r\dif\theta&&\text{\RL{اگر \عددی{r\ge 0} ہو}}
\end{aligned}
\end{gather}


شکل میں دکھایا گیا ہے کہ کس طرح  مستطیل \عددی{G: 0\le r\le 1,\,0\le \theta\le \pi/2} کو مساوات \عددی{x=r\cos\theta} اور \عددی{y=r\sin\theta}  ایک چوتھائی دائرہ \عددی{R}، جس کی سرحد  ربع اول میں مستوی \عددی{xy} پر \عددی{x^2+y^2=1} ہے، میں بدلتے ہیں۔

دھیان رہے کہ مساوات \حوالہ{مساوات_بالکثرت_قطبی_مستطیل_تکمل} کی دائیں ہاتھ میں قطبی محددی مستوی میں کسی خطہ پر \عددی{f(r\cos\theta,r\sin\theta)} کا تکمل نہیں  بلکہ  کارتیسی \عددی{r,\theta} مستوی کے خطہ \عددی{G} میں   \عددی{f(r\cos\theta,r\sin\theta)} اور \عددی{r} کے حاصل ضرب کا تکمل ہے۔

آئیں بدل کی دوسری مثال دیکھیں۔

 \ابتدا{مثال}\شناخت{مثال_بالکثرت_بدل_خطہ}
  مستوی  \عددی{uv} میں موزوں خطہ پر   بدل  
\begin{align}\label{مساوات_بالکثرت_بدل_مثال}
u=\frac{2x-y}{2},\quad v=\frac{y}{2}
\end{align}
 استعمال کرتے  ہوئے درج ذیل  تکمل کی قیمت تلاش کریں۔
\begin{align*}
\int_0^4\int_{x=y/2}^{x=y/2+1}\frac{2x-y}{2}\dif x\dif y
\end{align*}
حل:\quad
ہم مستوی  \عددی{xy} میں تکمل کے خطے کا خاکہ بنا کر اس کی سرحدوں کی نشاندہی کرتے ہیں۔

مساوات \حوالہ{مساوات_بالکثرت_یعقوبی_بدل} استعمال کرنے کی خاطر ہمیں مستوی \عددی{uv} میں مطابقتی خطہ \عددی{G} اور بدل کا یعقوبی  معلوم کرنے ہوں  گے۔ انہیں دریافت کرنے کے لئے  ہم مساوات \حوالہ{مساوات_بالکثرت_بدل_مثال} کو \عددی{x} اور \عددی{y} کے لئے \عددی{u}  اور \عددی{v} کی صورت میں حل    کرتے ہیں۔یوں درج ذیل حاصل ہو گا۔
\begin{align}\label{مساوات_بالکثرت_یو_وی_سرحد}
x=u+v,\quad y=2v
\end{align}
اس کے بعد ہم  \عددی{R} کی سرحدوں کی مساوات میں انہیں پر کر کے \عددی{G} کی سرحدیں دریافت کرتے ہیں۔
\begin{center}
\renewcommand{\arraystretch}{1.5}
\begin{tabular}{LCL}
\toprule
\begin{minipage}{2cm} \text{\RL{خطہ \عددی{R} کی سرحد}}\\  \text{\RL{کی \عددی{xy} مساواتیں}}  \end{minipage} & 
\begin{minipage}{2cm}\text{\RL{خطہ \عددی{G} کی مطابقتی}}\\ \text{\RL{سرحد کی  \عددی{uv} مساواتیں}} \end{minipage}&
\begin{minipage}{2cm} \text{\RL{\عددی{uv} مساواتوں}}\\  \text{\RL{کی سادہ صورت}} \end{minipage}\\
\midrule
x=\frac{y}{2}& u+v=\frac{2v}{2}=v& u=0\\
x=\frac{y}{2}+1& u+v=\frac{2v}{2}+1=v+1&u=1\\
y=0&2v=0&v=0\\
y=4&2v=4&v=2\\
\bottomrule
\end{tabular}
\end{center}
بدل کا یعقوبی  (مساوات \حوالہ{مساوات_بالکثرت_یو_وی_سرحد} سے) درج ذیل ہو گا۔
\begin{align*}
\renewcommand{\arraystretch}{1.5}
J(u,v)=\begin{vmatrix}
\frac{\partial x}{\partial u}&\frac{\partial x}{\partial v}\\
\frac{\partial y}{\partial u}&\frac{\partial y}{\partial v}
\end{vmatrix}=\begin{vmatrix}
\frac{\partial}{\partial u}(u+v)&\frac{\partial}{\partial v}(u+v)\\
\frac{\partial}{\partial u}(2v)&\frac{\partial}{\partial v}(2v)
\end{vmatrix}=\begin{vmatrix}
1&1\\
0&2
\end{vmatrix}=2
\end{align*}
ہم اب مساوات \حوالہ{مساوات_بالکثرت_یعقوبی_بدل} استعمال کرنے کی تمام معلومات جانتے ہیں:
\begin{align*}
\int_0^4\int_{x=y/2}^{x=y/2+1}\frac{2x-y}{2}\dif x\dif y&=\int_{v=0}^{v=2}\int_{u=0}^{u=1} u\abs{J(u,v)}\dif u\dif v\\
&=\int_0^2\int_0^1 (u)(2)\dif u\dif v=\int_0^2\big[u^2\big]_{0}^{1}\dif v=\int_0^2\dif v=2
\end{align*}
\انتہا{مثال}
%======================
\ابتدا{مثال}
درج ذیل تکمل کی قیمت تلاش کریں۔
\begin{align*}
\int_0^1\int_0^{1-x}\sqrt{x+y}(y-2x)^2\dif y\dif x
\end{align*}
حل:\quad
ہم  مستوی{xy} میں تکمل کے خطہ \عددی{R} کا خاکہ بنا  کر اس کی سرحدوں کی نشاندہی کرتے ہیں۔متکمل کو دیکھ کر ہمیں خیال آتا ہے کہ بدل \عددی{u=x+y} اور \عددی{v=y-2x}استعمال کیا جائے جنہیں \عددی{u} اور \عددی{v} کی صورت میں \عددی{x} اور \عددی{y} کے لئے حل کرتے ہوئے درج ذیل حاصل ہو گا۔
\begin{align}\label{مساوات_بالکثرت_سرحدیں_دوبارہ}
x=\frac{u}{3}-\frac{v}{3},\quad y=\frac{2u}{3}+\frac{v}{3}
\end{align}
ہم مساوات \حوالہ{مساوات_بالکثرت_سرحدیں_دوبارہ} سے مستوی \عددی{uv} میں خطہ \عددی{G} کی سرحدیں معلوم کرتے ہیں۔
\begin{center}
\renewcommand{\arraystretch}{1.5}
\begin{tabular}{LCL}
\toprule
\begin{minipage}{2cm}\text{\RL{\عددی{R} کی سرحد}}\\  \text{\RL{کی \عددی{xy} مساواتیں}}  \end{minipage}&
\begin{minipage}{2cm}\text{\RL{\عددی{G} کی مطابقتی سرحد}}\\  \text{\RL{کی \عددی{uv} مساواتیں}}  \end{minipage}&
\begin{minipage}{2cm}\text{\RL{\عددی{uv}مساواتوں}}\\  \text{\RL{کی سادہ صورت}}  \end{minipage}\\
\midrule
x+y=1&\big(\frac{u}{3}-\frac{v}{3}\big)+\big(\frac{2u}{3}+\frac{v}{3}\big)=1&u=1\\
x=0&\frac{u}{3}-\frac{v}{3}=0&v=u\\
y=0&\frac{2u}{3}+\frac{v}{3}=0&v=-2u\\
\bottomrule
\end{tabular}
\end{center}
مساوات \حوالہ{مساوات_بالکثرت_سرحدیں_دوبارہ} میں دیے بدل کا یعقوبی درج ذیل ہو گا۔
\begin{align*}
\renewcommand{\arraystretch}{1.5}
J(u,v)=\begin{vmatrix}
\frac{\partial x}{\partial u}&\frac{\partial x}{\partial v}\\
\frac{\partial y}{\partial u}&\frac{\partial y}{\partial v}
\end{vmatrix}=\begin{vmatrix}
\frac{1}{3} &-\frac{1}{3}\\
\frac{2}{3}&\frac{1}{3}
\end{vmatrix}=\frac{1}{3}
\end{align*}
 ہم مساوات \حوالہ{مساوات_بالکثرت_یعقوبی_بدل} سے تکمل کی قیمت حاصل کرتے ہیں:
\begin{align*}
\int_0^1\int_0^{1-x}&\sqrt{x+y}(y-2x)^2\dif y\dif x=\int_{u=0}^{u=1}\int_{v=-2u}^{v=u} u^{1/2}\,v^2\abs{J(u,v)}\dif v\dif u\\
&=\int_0^1\int_{-2u}^{u}u^{1/2}\,v^2\big(\frac{1}{3}\big)\dif v\dif u=\frac{1}{3}\int_0^1 u^{1/2}\big(\frac{1}{3}v^3\big)_{v=-2u}^{v=u}\dif u\\
&=\frac{1}{9}\int_0^1 u^{1/2}(u^3+8u^3)\dif u=\int_{0}^{1} u^{7/2}\dif u=\left. \frac{2}{9}u^{9/2}\right\vert_{0}^{1}=\frac{2}{9}
\end{align*}
\انتہا{مثال}
%======================

\جزوحصہء{تہرا تکملات میں بدل}
تہرا تکملات کے متغیرات کی تبدیلی کو   تین بعدی خطہ کا   بدل  تصور کرنے والے ترکیب کی خصوصی صورتیں  نلکی اور کروی محددی بدل ہیں۔ یہ ترکیب دوہرا  تکملات کی ترکیب کی طرح ہے، بس اب ہم دو کی بجائے تین بعد میں کام  کرتے ہیں۔ 

فرض کریں \عددی{uvw} فضا میں خطہ \عددی{G} کو  ایک ایک مطابقت کے  ساتھ \عددی{xyz} فضا کے خطہ \عددی{D} میں درج ذیل  روپ کی مساواتوں  سے بدلا جاتا ہے۔
\begin{align*}
x=g(u,v,w),\quad y=h(u,v,w),\quad z=k(u,v,w)
\end{align*}
تب \عددی{D} میں کسی بھی تفاعل \عددی{F(x,y,z)} کو \عددی{G} میں   معین  تفاعل
\begin{align*}
F(g(u,v,w),h(u,v,w),k(u,v,w))=H(u,v,w)
\end{align*}
تصور کیا جا سکتا ہے۔اگر \عددی{g}، \عددی{h}  اور \عددی{k} کے اول جزوی تفرقات استمراری ہوں تب \عددی{D} پر \عددی{F} کے تکمل کا \عددی{G} پر \عددی{H(u,v,w)} کے تکمل کے ساتھ تعلق درج ذیل مساوات دیگی۔
  \begin{align}\label{مساوات_بالکثرت_یعقوبی_تہرا_تکمل}
\iiint\limits_D F(x,y,z)\dif x\dif y\dif z=\iiint\limits_G H(u,v,w)\abs{J(u,v,w)}\dif u\dif v\dif w
\end{align}
اس مساوات میں \عددی{J(u,v,w)} کی مطلق قیمت استعمال کی گئی ہے جو درج ذیل \اصطلاح{یعقوبی مقطع}\فرہنگ{یعقوبی!مقطع}\حاشیہب{Jacobian determinant}\فرہنگ{Jacobian!determinant} ہے۔
\begin{align}
\renewcommand{\arraystretch}{1.5}
J(u,v,w)=\begin{vmatrix}
\frac{\partial x}{\partial u}&\frac{\partial x}{\partial v}&\frac{\partial x}{\partial w}\\
\frac{\partial y}{\partial u}&\frac{\partial y}{\partial v}&\frac{\partial y}{\partial w}\\
\frac{\partial z}{\partial u}&\frac{\partial z}{\partial v}&\frac{\partial z}{\partial w}
\end{vmatrix}=\frac{\partial(x,y,z)}{\partial(u,v,w)}
\end{align} 
متغیرات کی تبدیلی کا    کلیہ، جس کو مساوات \حوالہ{مساوات_بالکثرت_یعقوبی_تہرا_تکمل} میں پیش کیا گیا ہے، پیچیدہ ہے اور دو بعدی صورت کی طرح ، اس کی اشتقاق کو یہاں پیش نہیں کیا جائے گا۔

نلکی محدد میں \عددی{u}، \عددی{v} اور \عددی{w}  کی جگہ \عددی{\rho}، \عددی{\phi} اور \عددی{z} ہوں گے۔ کارتیسی \عددی{\rho\phi z} فضا سے کارتیسی \عددی{xyz} فضا میں بدل درج ذیل مساوات دیں گی۔
\begin{align*}
x=\rho\cos\phi,\quad y=\rho\sin\phi,\quad z=z
\end{align*}   
اس بدل کا یعقوبی
\begin{align*}
\renewcommand{\arraystretch}{1.5}
J(\rho,\phi,z)=&\begin{vmatrix}
\frac{\partial x}{\partial \rho}&\frac{\partial x}{\partial \phi}&\frac{\partial x}{\partial z}\\
\frac{\partial y}{\partial \rho}&\frac{\partial y}{\partial \phi}&\frac{\partial y}{\partial z}\\
\frac{\partial z}{\partial \rho}&\frac{\partial z}{\partial \phi}&\frac{\partial z}{\partial z}
\end{vmatrix}=\begin{vmatrix} 
\cos\phi&-\rho\sin\phi&0\\
\sin\phi&\rho\cos\phi&0\\
0&0&1
 \end{vmatrix}\\
&=\rho\cos^2\phi+\rho\sin^2\phi=\rho
\end{align*}
ہو گا۔یوں مساوات \حوالہ{مساوات_بالکثرت_یعقوبی_تہرا_تکمل} درج ذیل صورت  اختیار کریگی۔
 \begin{align}
\iiint\limits_D F(x,y,z)\dif x\dif y\dif z=\iiint\limits_G H(\rho,\phi,z)\abs{\rho}\dif \rho\dif \phi\dif z
\end{align}
    جب بھی \عددی{\rho\ge 0} ہو، ہم  مطلق کی علامت سے چھٹکارا حاصل کر سکتے ہیں۔ 


کروی  محدد میں \عددی{u}، \عددی{v} اور \عددی{w}  کی جگہ \عددی{r}، \عددی{\theta} اور \عددی{\phi} ہوں گے۔ کارتیسی \عددی{r \theta\phi} فضا سے کارتیسی \عددی{xyz} فضا میں بدل درج ذیل مساوات دیں گی۔
\begin{align*}
x=r\sin\theta\cos\phi,\quad y=r\sin\theta\sin\phi,\quad z=r\sin\theta
\end{align*}   
اس بدل کا یعقوبی
\begin{align}\label{مساوات_بالکثرت_کروی_یعقوبی}
\renewcommand{\arraystretch}{1.5}
J(\rho,\phi,z)=&\begin{vmatrix}
\frac{\partial x}{\partial r}&\frac{\partial x}{\partial \theta}&\frac{\partial x}{\partial \phi}\\
\frac{\partial y}{\partial r}&\frac{\partial y}{\partial \theta}&\frac{\partial y}{\partial \phi}\\
\frac{\partial z}{\partial r}&\frac{\partial z}{\partial \theta}&\frac{\partial z}{\partial \phi}
\end{vmatrix}=r^2\sin\theta
\end{align}
ہو گا (سوال  \حوالہ{سوال_بالکثرت_کروی_یعقوبی})۔یوں مساوات \حوالہ{مساوات_بالکثرت_یعقوبی_تہرا_تکمل} درج ذیل صورت  اختیار کریگی۔
 \begin{align}
\iiint\limits_D F(x,y,z)\dif x\dif y\dif z=\iiint\limits_G H(r,\theta,\phi)\abs{r^2\sin\theta}\dif r\dif \theta\dif \phi
\end{align}
 کروی محدد میں \عددی{0\le \theta\le \pi} کی بنا   \عددی{\sin\theta} کبھی بھی منفی نہیں ہو  سکتا   ہے  لہٰذا مطلق کی علامت لکھنے کی ضرورت نہیں ہے۔ 

آئیں بدل کی ایک مثال دیکھتے ہیں۔

\ابتدا{مثال}\شناخت{مثال_بالکثرت_بدل_تکمل}
درج ذیل بدل 
\begin{align}\label{مساوات_بالکثرت_کروی_بدل}
u=(2x-y)/2,\quad v=y/2,\quad w=z/3
\end{align}
 استعمال  کرتے ہوئے \عددی{uvw}  فضا میں موزوں خطہ پر  تکمل لے کر درج ذیل تکمل کی قیمت دریافت کریں۔ 
\begin{align*}
\int_0^3\int_0^4\int_{x=y/2}^{x=y/2+1} \big(\frac{2x-y}{2}+\frac{z}{3}\big)\dif x\dif y\dif z
\end{align*}
حل:\quad
ہم \عددی{xyz} فضا میں تکمل کے خطہ \عددی{D} کا خاکہ بنا کر اس کی سرحدوں کی نشاندہی کرتے ہیں۔ یہاں سرحدی سطحیں مستویات ہیں۔

مساوات \حوالہ{مساوات_بالکثرت_یعقوبی_تہرا_تکمل}  استعمال کرنے کے لئے ہمیں  \عددی{uvw}  فضا میں  مطابقتی خطہ \عددی{G} اور بدل کا یعقوبی جاننا ہو گا۔ ہم مساوات \حوالہ{مساوات_بالکثرت_کروی_بدل} کو \عددی{x}، \عددی{y} اور \عددی{z} کے لئے \عددی{u}، \عددی{v} اور \عددی{w} کی صورت میں حل کر کے 
\begin{align}\label{مساوات_بالکثرت_نئے_متغیرات}
x=u+v,\quad y=2v,\quad z=3w
\end{align}
حاصل کرتے ہیں۔ ہم \عددی{D} کی سرحدوں کی مساوات میں  یہ قیمتیں پر کر کے  \عددی{G} کی سرحدوں کی مساواتیں دریافت کرتے ہیں:
\begin{center}
\begin{tabular}{LCL}
\toprule
\begin{minipage}{2cm}\text{\RL{\عددی{D} کی سرحدوں}}\\ \text{\RL{کی \عددی{xyz} مساواتیں}}  \end{minipage}&
\begin{minipage}{2cm}\text{\RL{\عددی{G} کی سرحدوں کی}}\\ \text{\RL{مطابقتی \عددی{uvw} مساواتیں}}  \end{minipage}&
\begin{minipage}{2cm}\text{\RL{\عددی{uvw} مساواتوں}}\\ \text{\RL{کی سادہ صورتیں}}  \end{minipage}\\
\midrule
x=y/2&u+v=2v/2=v&u=0\\
x=y/2+1&u+v=2v/2+1=v+1&u=1\\
y=0&2v=0&v=0\\
y=4&2v=4&v=2\\
z=0&3w=0&w=0\\
z=3&3w=3&w=1\\
\bottomrule
\end{tabular}
\end{center}
ہم مساوات \حوالہ{مساوات_بالکثرت_نئے_متغیرات} استعمال کرتے ہوئے یعقوبی تلاش کرتے ہیں۔
\begin{align*}
\renewcommand{\arraystretch}{1.5}
J(u,v,w)=\begin{vmatrix}
\frac{\partial x}{\partial u} &\frac{\partial x}{\partial v}&\frac{\partial x}{\partial w}\\
\frac{\partial y}{\partial u}&\frac{\partial y}{\partial v}&\frac{\partial y}{\partial w}\\
\frac{\partial z}{\partial u}&\frac{\partial z}{\partial v}&\frac{\partial z}{\partial w}
\end{vmatrix}=\begin{vmatrix}
1&1&0\\
0&2&0\\
0&0&3
\end{vmatrix}=6
\end{align*}
ہم  مساوات \حوالہ{مساوات_بالکثرت_یعقوبی_تہرا_تکمل} استعمال کرنے کے لئے درکار تمام معلوم جان چکے ہیں۔ یوں درج ذیل ہو گا۔
\begin{align*}
\int_0^3\int_0^4\int_{x=y/2}^{x=y/2+1}&\big(\frac{2x-y}{2}+\frac{z}{3}\big)\dif x\dif y\dif z\\
&=\int_0^1\int_0^2\int_0^1(u+w)\abs{J(u,v,w)}\dif u\dif v\dif w\\
&=\int_0^1\int_0^2\int_0^1(u+w)(6)\dif u\dif v\dif w=6\int_0^1\int_0^2\big[\frac{u^2}{2}+uw\big]_0^1\dif v\dif w\\
&=6\int_0^1\int_0^2\big(\frac{1}{2}+w\big)\dif v\dif w=6\int_0^1\big[\frac{v}{2}+vw\big]_0^2\dif w=6\int_0^1(1+2w)\dif w\\
&=6\big[w+w^2\big]_0^1=6(2)=12
\end{align*}
\انتہا{مثال}
%===========================

\جزوحصہء{سوالات}
\ابتدا{سوالات}
\موٹا{بدل محدد}\\
\ابتدا{سوال}\شناخت{سوال_بالکثرت_بدل_الف}
\begin{enumerate}[a.]
\item
درج ذیل نظام کو \عددی{x} اور \عددی{y} کے لئے \عددی{u} اور \عددی{v} کی صورت میں حل کریں۔ اس کے بعد یعقوبی \عددی{\tfrac{\partial(x,y)}{\partial(u,v)}} کی قیمت تلاش کریں۔
\[u=x-y,\quad v=2x+y\]
\item
مستوی \عددی{xy} میں تکونی خطہ جس کے راس \عددی{(0,0)}، \عددی{(1,1)} اور \عددی{(1,-2)} ہیں  کا عکس بدل \عددی{u=x-y}، \عددی{v=2x+y} میں تلاش کریں۔ مستوی \عددی{uv} میں تبدیل شدہ  خطے کا خاکہ بنائیں۔
\end{enumerate}
\انتہا{سوال}
%======================
\ابتدا{جواب}
\wf{\unexpanded{
(ا) 
\(x=\tfrac{u+v}{3},\, y=\tfrac{v-2u}{3};\, \tfrac{1}{3}\)\\
(ب)
تکونی خطہ جس کی سرحدیں  \عددی{u=0}، \عددی{v=0} اور \عددی{u+v=3} ہیں۔
}}
\انتہا{جواب}
%===============================
\ابتدا{سوال}\شناخت{سوال_بالکثرت_بدل_ب}
\begin{enumerate}[a.]
\item
درج ذیل نظام کو \عددی{x} اور \عددی{y} کے لئے \عددی{u} اور \عددی{v} کی صورت میں حل کریں۔ اس کے بعد یعقوبی \عددی{\tfrac{\partial(x,y)}{\partial(u,v)}} کی قیمت تلاش کریں۔
\[u=x+2y,\quad v=x-y\]
\item
مستوی \عددی{xy} میں  لکیر  \عددی{y=0}، \عددی{y=x} اور \عددی{x+2y=2} کے بیچ  تکونی خطے   کا عکس بدل \عددی{u=x+2y}، \عددی{v=x-y} میں تلاش کریں۔ مستوی \عددی{uv} میں تبدیل شدہ  خطے کا خاکہ بنائیں۔

\end{enumerate}
\انتہا{سوال}
%========================
\ابتدا{سوال}\شناخت{سوال_بالکثرت_بدل_پ}
\begin{enumerate}[a.]
\item
درج ذیل نظام کو \عددی{x} اور \عددی{y} کے لئے \عددی{u} اور \عددی{v} کی صورت میں حل کریں۔ اس کے بعد یعقوبی \عددی{\tfrac{\partial(x,y)}{\partial(u,v)}} کی قیمت تلاش کریں۔
\[u=3x+2y,\quad v=x+4y\]
\item
مستوی \عددی{xy} میں محور   \عددی{x}، محور  \عددی{y} اور لکیر  \عددی{x+y=1} کے بیچ  تکونی خطے   کا عکس بدل \عددی{u=3x+2y}، \عددی{v=x+4y} میں تلاش کریں۔ مستوی \عددی{uv} میں تبدیل شدہ  خطے کا خاکہ بنائیں۔
\end{enumerate}
\انتہا{سوال}
%========================
\ابتدا{جواب}
\wf{\unexpanded{
(ا) 
\(x=\tfrac{1}{5}(2u-v),\,y=\tfrac{1}{10}(3v-u);\, \tfrac{1}{10}\)\\
(ب) تکونی خطہ جس کی سرحدیں \عددی{3v=u}، \عددی{v=2u} اور \عددی{3u+v=10} ہیں۔
}}
\انتہا{جواب}
%===============================
\ابتدا{سوال}\شناخت{سوال_بالکثرت_بدل_ت}
\begin{enumerate}[a.]
\item
درج ذیل نظام کو \عددی{x} اور \عددی{y} کے لئے \عددی{u} اور \عددی{v} کی صورت میں حل کریں۔ اس کے بعد یعقوبی \عددی{\tfrac{\partial(x,y)}{\partial(u,v)}} کی قیمت تلاش کریں۔
\[u=2x-3y,\quad v=-x+y\]
\item
مستوی \عددی{xy} میں    \عددی{x=-3}،   \عددی{x=0}، \عددی{y=x} اور   \عددی{y=x+1} کے بیچ   متوازی  الاضلاع کا عکس بدل \عددی{u=2x-3y}، \عددی{v=-x+y} میں تلاش کریں۔ مستوی \عددی{uv} میں تبدیل شدہ  خطے کا خاکہ بنائیں۔
\end{enumerate}
\انتہا{سوال}
%========================
\ابتدا{سوال}
درج ذیل بدل کے یعقوبی تلاش کریں۔
\begin{multicols}{2}
\begin{enumerate}[a.]
\item
\(x=u\cos v,\quad y=u\sin v\)
\item
\(x=u\sin v,\quad y=u\cos v\)
\end{enumerate}
\end{multicols}
\انتہا{سوال}
%==================
\ابتدا{جواب}
\wf{\unexpanded{
(ا) 
\(\begin{vmatrix} \cos v&-u\sin v\\ \sin v&u\cos v \end{vmatrix}=u\)\\
(ب) 
\(\begin{vmatrix}\sin v &u\cos v\\ \cos v&-u\sin v  \end{vmatrix}=-u\)
}}
\انتہا{جواب}
%===============================
\ابتدا{سوال}
درج ذیل بدل کے یعقوبی تلاش کریں۔
\begin{enumerate}[a.]
\item
\(x=u\cos v,\quad y=u\sin v,\quad z=w\)
\item
\(x=2u-1,\quad y=3v-4,\quad z=\frac{1}{2}(w-4)\)
\end{enumerate}

\انتہا{سوال}
%==================
\موٹا{دوہرا تکملات}\\
\ابتدا{سوال}
درج ذیل تکمل کی قیمت  \عددی{x}  اور \عددی{y} کے لحاظ سے تکمل  لے کر حاصل کرتے ہوئے  مثال \حوالہ{مثال_بالکثرت_بدل_خطہ} میں حاصل قیمت \عددی{(2)} کی تصدیق  کریں۔
 \begin{align*}
\int_0^4\int_{x=y/2}^{x=y/2+1}\frac{2x-y}{2}\dif x\dif y
\end{align*}
\انتہا{سوال}
%================
\ابتدا{سوال}
ربع اول میں  لکیر \عددی{y=-2x+4}، \عددی{y=-2x+7}، \عددی{y=x-2} اور \عددی{y=x+1} کے بیچ  خطہ \عددی{R}  پر درج ذیل تکمل کی قیمت سوال \حوالہ{سوال_بالکثرت_بدل_الف} کا بدل استعمال کرتے ہوئے حاصل کریں۔
\[\iint\limits_R(2x^2-xy-y^2)\dif x\dif y\]
\انتہا{سوال}
%======================
\ابتدا{سوال}
ربع اول میں  لکیر \عددی{y=-\tfrac{3}{2}x+1}،   \عددی{y=-\tfrac{3}{2}x+3} ، \عددی{y=-\tfrac{1}{4}x} اور \عددی{y=-\tfrac{1}{4}x+1} کے بیچ  خطہ  \عددی{R}  پر درج ذیل تکمل کی قیمت سوال \حوالہ{سوال_بالکثرت_بدل_پ} کا بدل استعمال کرتے ہوئے حاصل کریں۔
\[\iint\limits_R(3x^2+14xy+8y^2)\dif x\dif y\]
\انتہا{سوال}
%=====================
\ابتدا{جواب}
\wf{\unexpanded{
\(\tfrac{64}{5}\)
}}
\انتہا{جواب}
%===============================
\ابتدا{سوال}
درج ذیل تکمل کی قیمت  سوال \حوالہ{سوال_بالکثرت_بدل_ت}  کا  خطہ \عددی{R}  اور بدل استعمال کرتے ہوئے دریافت کریں۔
\[\iint\limits_R 2(x-y)\dif x\dif y\] 
\انتہا{سوال}
%==================
\ابتدا{سوال}
ربع اول میں مستوی \عددی{xy} میں  قطع زائد \عددی{xy=1}،  \عددی{xy=9} اور لکیر \عددی{y=x}، \عددی{y=4x} کے بیچ خطہ \عددی{R} ہے۔ بدل \عددی{x=u/v}، \عددی{y=uv} جہاں  \عددی{u>0}، \عددی{v>0} ہیں استعمال  کرتے ہوئے درج ذیل تکمل کو  مستوی \عددی{uv}   میں موزوں خطہ \عددی{G} پر ایک  تکمل کی صورت میں  لکھیں۔
\[\iint\limits_R \big(\sqrt{\frac{y}{x}}+\sqrt{xy}\big)\dif x\dif y\]
خطہ \عددی{G} پر اس \عددی{uv} تکمل کی قیمت تلاش کریں۔
\انتہا{سوال}
%====================
\ابتدا{جواب}
\wf{\unexpanded{
\(\int_1^2\int_1^3(u+v)\tfrac{2u}{v}\dif u\dif v=8+\tfrac{52}{3}\ln 2\)
}}
\انتہا{جواب}
%===============================
\ابتدا{سوال}
(ا)  بدل \عددی{x=u}، \عددی{y=uv} کا یعقوبی تلاش کریں اور خطہ \عددی{G:1\le u\le 2,\,1\le uv\le 2} کا  مستوی \عددی{uv} میں  خاکہ بنائیں۔ (ب) اس کے بعد مساوات \حوالہ{مساوات_بالکثرت_یعقوبی_بدل} استعمال کرتے ہوئے درج ذیل تکمل  کو \عددی{G}  پر ایک تکمل کی صورت میں لکھیں۔ دونوں تکملات کو حل کرتے ہوئے تکمل کی قیمتیں حاصل کریں۔
\[\int_1^2\int_1^2\frac{y}{x}\dif y\dif x\]
\انتہا{سوال}
%======================
\ابتدا{سوال}
مستوی \عددی{xy} میں ترخیم \عددی{\tfrac{x^2}{a^2}+\tfrac{y^2}{b^2}=1,\,a>0,b>0} کے بیچ خطہ پر مستقل کثافت کی پتلی چادر  کی مبدا کے لحاظ سے  معیار  اثر اول  تلاش کریں۔ (اشارہ: بدل \عددی{x=ar\cos\theta}، \عددی{y=br\sin\theta} استعمال کریں۔)
\انتہا{سوال}
%=======================
\ابتدا{جواب}
\wf{\unexpanded{
\(\tfrac{\pi ab(a^2+b^2)}{4}\)
}}
\انتہا{جواب}
%===============================
\ابتدا{سوال}
مستوی \عددی{xy} میں   \عددی{\tfrac{x^2}{a^2}+\tfrac{y^2}{b^2}=1}  پر تفاعل \عددی{f(x,y)=1} کا تکمل لے کر ترخیم کا رقبہ \عددی{\pi ab} حاصل کیا جا سکتا ہے۔ اس تکمل کو سیدھا حل کرنے کے لئے اس میں تکونیاتی تفاعل  پر کرنا ہو گا۔ اس سے آسان طریقہ بدل \عددی{x=au}، \عددی{y=bv} استعمال کرتے ہوئے تبدیل شدہ تکمل کی    قیمت کا مستوی \عددی{uv}     میں  قرص \عددی{G:u^2+v^2\le 1} پر حصول ہے۔
\انتہا{سوال}
%=========================
\ابتدا{سوال}
درج ذیل تکمل کو پہلے مستوی \عددی{uv}  میں خطہ \عددی{G} پر سوال \حوالہ{سوال_بالکثرت_بدل_ب} کے بدل سے تبدیل کرتے ہوئے  حل کریں۔
\[\int_0^{2/3}\int_y^{2-2y}(x+2y)e^{(y-x)}\dif x\dif y\]
\انتہا{سوال}
%======================
\ابتدا{جواب}
\wf{\unexpanded{
\(\tfrac{1}{3}(1+\tfrac{3}{e^2})\approx 0.4687\)
}}
\انتہا{جواب}
%===============================
\ابتدا{سوال}
مستوی \عددی{uv} میں درج ذیل تکمل  کو بدل \عددی{x=u+\tfrac{v}{2}}، \عددی{y=v} کی مدد سے منتقل کریں۔ تبدیل شدہ تکمل کی قیمت تلاش کریں۔
\[\int_0^2\int_{y/2}^{(y+4)/2}y^3(2x-y)e^{(2x-y)^2}\dif x\dif y\]
\انتہا{سوال}
%=======================
\موٹا{تہرا تکملات}\\
\ابتدا{سوال}\شناخت{سوال_بالکثرت_کروی_یعقوبی}
کارتیسی \عددی{r\theta\phi} فضا  سے کارتیسی \عددی{xyz} فضا کے بدل کا یعقوبی مساوات \حوالہ{مساوات_بالکثرت_کروی_یعقوبی} کا مقطع  دیتا ہے۔ اس مقطع  کو حل کر کے اس کی قیمت \عددی{r^2\sin\theta} حاصل کریں۔
\انتہا{سوال}
%=====================
\ابتدا{سوال}
متغیرات \عددی{x}، \عددی{y} اور \عددی{z} کے لحاظ سے تکمل لیتے ہوئے مثال \حوالہ{مثال_بالکثرت_بدل_تکمل} کا تکمل حل کریں۔
\انتہا{سوال}
%==========================
\ابتدا{سوال}
ترخیم نما 
\[\frac{x^2}{a^2}+\frac{y^2}{b^2}+\frac{z^2}{c^2}=1\]
کا حجم تلاش کریں۔(اشارہ: بدل \عددی{x=au}، \عددی{y=bv}، \عددی{z=cw}  لے کر فضا \عددی{uvw} میں موزوں خطہ پر تکمل  کی قیمت تلاش کریں۔)
\انتہا{سوال}
%========================
\ابتدا{جواب}
\wf{\unexpanded{
\(\tfrac{4\pi abc}{3}\)
}}
\انتہا{جواب}
%===============================
\ابتدا{سوال}
ٹھوس ترخیم نما 
\[\frac{x^2}{a^2}+\frac{y^2}{b^2}+\frac{z^2}{c^2}\le 1\]
پر درج ذیل تکمل کی قیمت تلاش کریں۔(اشارہ: بدل \عددی{x=au}، \عددی{y=bv}، \عددی{z=cw} لے کر فضا \عددی{uvw} میں موزوں خطہ پر تکمل  کی قیمت تلاش کریں۔)
\[\iiint \abs{xyz}\dif x\dif y\dif z\]
\انتہا{سوال}
%========================
\ابتدا{سوال}
فضا \عددی{xyz} میں  خطہ \عددی{D} درج           ذیل ہے۔
\[1\le x\le 2,\quad 0\le xy\le 2,\quad 0\le z\le 1\]
بدل
\[u=x,\quad v=xy,\quad w=3z\]
استعمال کر کے فضا  \عددی{uvw}  میں موزوں خطہ \عددی{G}  پر درج ذیل  تکمل کی قیمت تلاش کریں۔
\[\iiint\limits_D (x^2y+3xyz)\dif x\dif y\dif z\]
\انتہا{سوال}
%==================
\ابتدا{جواب}
\wf{\unexpanded{
\(\int_0^3\int_0^2\int_1^2(\tfrac{v}{3}+\tfrac{vw}{3u})\dif u\dif v\dif w=2+\ln 8\)
}}
\انتہا{جواب}
%===============================
\ابتدا{سوال}
یہ جانتے ہوئے کہ  نصف کرہ کا مرکز کمیت کرہ  کے قاعدہ سے   سر  جانب محور تشاکلی  پر   تین آٹھواں فاصلہ پر ہے،  موزوں تکملات کو بدل کر دکھائیں کہ نصف  ترخیم نما 
\[\frac{x^2}{a^2}+\frac{y^2}{b^2}+\frac{z^2}{c^2}\le 1,\, z\ge 0\]
کا مرکز کمیت  محور \عددی{z} پر قاعدہ سے راس جانب تین آٹھواں فاصلہ پر ہو گا۔ آپ کو تکمل حل کرنے کی ضرورت پیش نہیں آنی چاہیے۔
\انتہا{سوال}
%=========================
\ابتدا{سوال}
واحد متغیر کے تکملات  میں بدل کو کس طرح ترکیب بدل کی ایک خصوصی روپ تصور کیا جا سکتا ہے؟ ان میں یعقوبی کی قیمت کیا ہو گی؟ ایک مثال کی مدد سے وضاحت کریں۔
\انتہا{سوال}
%======================
\انتہا{سوالات}

