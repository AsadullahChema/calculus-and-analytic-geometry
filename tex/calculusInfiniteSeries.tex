\باب{لامتناہی تسلسل}
اس باب میں ہم ایک حیران کن کلیہ اخذ کرتے ہیں جس کی مدد سے بہت سارے تفاعل کو "لامتناہی کثیر رکنی" کی صورت میں لکھنا ممکن ہو گا اور ساتھ ہی کثیر رکنی کے  ارکان حذف  کر کے کثیر رکنی کو متناہی بنانے سے پیدا خلل بھی جان پائیں گے۔ ان تسلسل کو طاقتی تسلسل کہتے ہیں۔ قابل تفرق تفاعل کو تخمینی طور پر کثیر رکنی سے ظاہر کرنے میں مدد دینے کے علاوہ طاقتی تسلسل دیگر مواقع پر بھی کار آمد ثابت ہوتے ہیں۔ غیر بنیادی تکمل کی قیمت کے حصول کے علاوہ حراری توانائی کی منتقلی، ارتعاش، کیمیائی نفوذ اور ترسیل اشارات کے تفرقی مساوات  کے حل میں یہ موثر کردار ادا کرتے ہیں۔ آپ یہاں وہ ان تفاعل کے بارے میں سیکھ پائیں گے جو سائن اور انجینئری میں بہت زیادہ استعمال ہوتے ہیں۔

\حصہ{اعداد کی ترتیب کی حد}
غیر رسمی طور پر ترتیب سے مراد مرتب چیزوں کا سلسلہ ہے۔ اس باب میں ہمیں اعداد کی ترتیب سے غرض ہو گا۔ ترکیب نیوٹن سے حاصل اعداد کی ترتیب   \عددی{x_0,x_1,\cdots,x_n,\cdots} یا (ہلگے ون کوچ کے) برفانی روئی کے کثیر الاضلاع کی ترتیب \عددی{c_1,c_2,\cdots,c_n,\cdots} ہم دیکھ چکے ہیں۔ ان ترتیبوں کی حد پائی جاتی ہے، البتہ بہت سارے اہم ترتیبوں کے حد نہیں پائے جاتے ہیں۔

\جزوحصہء{تعریف اور علامتیت}
ہم \عددی{3} کے  ہر عدد صحیح مضرب کو ایک مقام مختص کر کے ایک فہرست بنا سکتے ہیں:
\begin{align*}
\begin{array}{rcccccc}
\text{\RL{دائرہ کار}} &1&2&3&\cdots&n&\cdots\\
\text{\RL{سعت}} &3&6&9&&3n&
\end{array}
\end{align*}
پہلا عدد \عددی{3}، دوسرا \عددی{6}، تیسرا \عددی{9}، وغیرہ، وغیرہ ہیں۔ مختص کرنے کا عمل ایک تفاعل ہے جو \عددی{n} ویں مقام کو \عددی{3n} مختص کرتا ہے۔ ترتیب کی بناوٹ کا بنیادی تصور یہی ہے۔ ایک تفاعل ہمیں بتاتا ہے کہ کس مقام پر کونسا عدد ہو گا۔

\ابتدا{تعریف}
ایک تفاعل جس کا دائرہ کار کسی عدد صحیح \عددی{n_0} کے برابر یا اس سے بڑے عدد صحیح پر مشتمل اعداد کا سلسلہ ہو \اصطلاح{لامتناہی ترتیب}\فرہنگ{ترتیب!لامتناہی}\حاشیہب{infinite sequence}\فرہنگ{sequence!infinite} (یا \اصطلاح{ترتیب}\فرہنگ{ترتیب}\حاشیہب{sequence}\فرہنگ{sequence})  کہلاتا ہے۔
\انتہا{تعریف}
%================

عموماً \عددی{n_0=1} ہوتا ہے اور ترتیب کا دائرہ کار مثبت اعداد صحیح پر مشتمل ہو گا۔ البتہ بعض اوقات ہم تسلسل کو کسی دوسرے عدد صحیح سے شروع کرنا چاہتے ہیں۔ ترکیب نیوٹن میں ہم \عددی{n_0=0} لیتے ہیں۔ اگر ہم \عددی{n} اضلاع پر مشتمل کثیر الاضلاع کی ترتیب کی بات کریں تب ہم \عددی{n_0=3} منتخب کرنا چاہیں گے۔

ترتیب کی تعریف کسی بھی تفاعل کی طرح کی جاتی ہے (مثال \حوالہ{مثال_تسلسل_متعدد_ترتیبات} اور شکل \حوالہ{شکل_تسلسل_مرتکز_منفرج_الف} تا شکل \حوالہ{شکل_تسلسل_مرتکز_منفرج_ث})، مثلاً:
\begin{align*}
a(n)=\sqrt{n},\quad a(n)=(-1)^{n+1}\frac{1}{n},\quad a(n)=\frac{n-1}{n}
\end{align*}

یہ ظاہر کرنے کی خاطر کہ دائرہ کار عدد صحیح ہے، ہم حرف \عددی{n} استعمال کرتے ہیں نا کہ دیگر غیر تابع متغیر کے لئے مستعمل حروف \عددی{x}، \عددی{y}، \عددی{z}، \عددی{t}، وغیرہ۔  مذکورہ بالا کی طرح تعریفی قاعدہ میں کلیات عموماً مثبت عدد صحیح سے زیادہ بڑے دائرہ کار کے لئے درست ہوتے ہیں۔ جیسا ہم دیکھیں گے یہ بعض اوقات سود مند ثابت ہوتا ہے۔ 

عدد \عددی{a(n)} ترتیب کا \اصطلاح{\عددی{n} واں جزو} یا اشاریہ \عددی{n} والا جزو ہو گا۔ اگر \عددی{a(n)=\tfrac{n-1}{n}} ہو تب درج ذیل ہو گا۔
\begin{align*}
\begin{array}{ccccc}
\text{\RL{پہلا جزو}}&\text{\RL{دوسرا جزو}}&\text{\RL{تیسرا جزو}}&&\text{\RL{$n$ واں جزو}}\\
\toprule
a(1)=0&a(2)=\frac{1}{2}&a(3)=\frac{2}{3}&\cdots&a(n)=\frac{n-1}{n}
\end{array}
\end{align*}
اشاریہ علامت استعمال کرتے ہوئے ہم \عددی{a(n)} کو \عددی{a_n} لکھتے ہیں۔ اشاریہ علامتی روپ میں یہی ترتیب درج ذیل لکھی جائے گی۔
\begin{align*}
\begin{array}{ccccc}
\text{\RL{پہلا جزو}}&\text{\RL{دوسرا جزو}}&\text{\RL{تیسرا جزو}}&&\text{\RL{$n$ واں جزو}}\\
\toprule
a_1=0&a_2=\frac{1}{2}&a_3=\frac{2}{3}&\cdots&a_n=\frac{n-1}{n}
\end{array}
\end{align*}
ترتیب پر تبصرہ کرتے ہوئے ہم عموماً \عددی{n} ویں جزو کے کلیہ کے ساتھ ساتھ چند ابتدائی اجزاء  لکھتے ہیں۔

\ابتدا{مثال}\شناخت{مثال_تسلسل_متعدد_ترتیبات}
\begin{align*}
\renewcommand{\arraystretch}{2}
\begin{array}{lll}
\text{\RL{اس کے لئے ہم درج ذیل لکھتے ہیں}}&\phantom{kkkk}&\text{\RL{جس ترتیب کا تعریفی کلیہ درج ذیل ہو}}\\
\toprule
1,\sqrt{2},\sqrt{3},\sqrt{4},\cdots,\sqrt{n},\cdots&&a_n=\sqrt{n}\\
1,\frac{1}{2},\frac{1}{3},\cdots,\frac{1}{n},\cdots&&a_n=\frac{1}{n}\\
1,-\frac{1}{2},\frac{1}{3},-\frac{1}{4},\cdots (-1)^{n+1}\frac{1}{n},\cdots&&a_n=(-1)^{n+1}\frac{1}{n}\\
0,\frac{1}{2},\frac{2}{3},\frac{3}{4},\cdots,\frac{n-1}{n},\cdots&&a_n=\frac{n-1}{n}\\
0,-\frac{1}{2},\frac{2}{3},-\frac{3}{4},\cdots,(-1)^{n+1}\big(\frac{n-1}{n}\big),\cdots&&a_n=(-1)^{n+1}\big(\frac{n-1}{n}\big)\\
3,3,3,\cdots,3,\cdots&&a_n=3
\end{array}
\end{align*}
ان تمام ترتیبوں کو دو مختلف انداز میں شکل \حوالہ{شکل_تسلسل_مرتکز_منفرج_الف} تا شکل \حوالہ{شکل_تسلسل_مرتکز_منفرج_ث} میں دکھایا گیا ہے۔
\انتہا{مثال}
%=================== 

\جزوحصہء{علامتیت}
جس ترتیب کا \عددی{n} واں جزو \عددی{a_n} ہو اس ترتیب کو ہم \عددی{\{a_n\}} سے ظاہر کرتے ہیں جو  ترتیب \عددی{a} اشاریہ \عددی{n} پڑھا جاتا ہے۔ مثال \حوالہ{مثال_تسلسل_متعدد_ترتیبات} میں دوسری ترتیب \عددی{\{\tfrac{1}{n}\}} ہے جو ترتیب ایک بٹہ تین پڑھا جاتا ہے۔ آخری ترتیب \عددی{\{3\}} ہے جو مستقل ترتیب \عددی{3} کہلائے گی۔

\begin{figure}
\centering
\begin{subfigure}{0.45\textwidth}
\centering
\begin{tikzpicture}[font=\small,xscale=2]
\draw[-latex](-0.25,0)--(2.5,0);
\foreach \n in {1,2,3,4,5}{\pgfmathsetmacro{\k}{sqrt(\n)} \draw(\k,0)node[circ]{}node[above]{$a_{\n}$};}
\foreach \x in {0,1,2}{\draw(\x,0)++(0,-0.15)node[below]{$\x$}--++(0,0.3);}
\draw(1.25,-0.5)node[below]{$a_n=\sqrt{n}$};
\end{tikzpicture}
\end{subfigure}\hfill
\begin{subfigure}{0.45\textwidth}
\centering
\begin{tikzpicture}[font=\small,xscale=1]
\draw[-latex](-0.25,0)--(5.5,0)node[right]{$n$};
\draw[-latex](0,-0.2)--(0,3.5)node[above]{$a_n$};
\foreach \n in {1,2,3,4,5}{\pgfmathsetmacro{\k}{sqrt(\n)} \draw(\n,\k)node[circ]{}node[above]{$(\n,\sqrt{\n})$};}
\foreach \x in {1,2,3,4,5}{\draw(\x,0)node[below]{$\x$}--++(0,0.2);}
\foreach \y in {1,2,3,}{\draw(0,\y)node[left]{$\y$}--++(0.2,0);}
\draw(3,3)node[above]{منفرج};
\end{tikzpicture}
\end{subfigure}
\caption{جزو $a_n$ آخر کار ہر عدد صحیح سے بڑھتا ہے لہٰذا ترتیب $\{a_n\}$ منفرج ہے۔}
\label{شکل_تسلسل_مرتکز_منفرج_الف}
\end{figure}
%%%%%%%%%%%%%%%%
\begin{figure}
\centering
\begin{subfigure}{0.45\textwidth}
\centering
\begin{tikzpicture}[font=\small,xscale=4]
\draw[-latex](-0.125,0)--(1.125,0);
\foreach \n in {1,2,3,4}{\pgfmathsetmacro{\k}{1/\n} \draw(\k,0)node[circ]{}node[above]{$a_{\n}$};}
\foreach \x in {0,1}{\draw(\x,0)++(0,-0.15)node[below]{$\x$}--++(0,0.3);}
\draw(0.5,-0.5)node[below]{$a_n=\frac{1}{n}$};
\end{tikzpicture}
\end{subfigure}\hfill
\begin{subfigure}{0.45\textwidth}
\centering
\begin{tikzpicture}[font=\small,xscale=1]
\draw[-latex](-0.25,0)--(5.5,0)node[right]{$n$};
\draw[-latex](0,-0.2)--(0,1.5)node[above]{$a_n$};
\foreach \n in {1,2,3,4,5}{\pgfmathsetmacro{\k}{1/\n} \draw(\n,\k)node[circ]{}node[above]{$(\n,\tfrac{1}{\n})$};}
\foreach \x in {1,2,3,4,5}{\draw(\x,0)node[below]{$\x$}--++(0,0.2);}
\foreach \y in {1}{\draw(0,\y)node[left]{$\y$}--++(0.12,0);}
\draw(3,1)node[above]{\RL{$0$ پر مرتکز}};
\end{tikzpicture}
\end{subfigure}
\caption{جزو $a_n=\tfrac{1}{n}$  بتدریج $n$ بڑھنے سے گھٹتے ہوئے $0$ کے قریب پہنچتے ہیں لہٰذا ترتیب $\{a_n\}$ صفر کو مرتکز ہے۔}
\label{شکل_تسلسل_مرتکز_منفرج_ب}
\end{figure}
%%%%%%%%%%%%%%%%%%%%%%%
\begin{figure}
\centering
\begin{subfigure}{0.45\textwidth}
\centering
\begin{tikzpicture}[font=\small,xscale=2]
\draw[-latex](-1.125,0)--(1.125,0);
\foreach \n in {1,2,3,4,5}{\pgfmathsetmacro{\k}{(-1)^(\n+1)/\n} \draw(\k,0)node[circ]{}node[above]{$a_{\n}$};}
\foreach \x in {0,1,-1}{\draw(\x,0)++(0,-0.15)node[below]{$\x$}--++(0,0.3);}
\draw(0,-0.5)node[below]{$a_n=\frac{(-1)^{n+1}}{n}$};
\end{tikzpicture}
\end{subfigure}\hfill
\begin{subfigure}{0.45\textwidth}
\centering
\begin{tikzpicture}[font=\small,xscale=1]
\draw[-latex](-0.25,0)--(5.5,0)node[right]{$n$};
\draw[-latex](0,-0.2)--(0,1.5)node[above]{$a_n$};
\foreach \n in {1,3,5}{\pgfmathsetmacro{\k}{(-1)^(\n+1)/\n} \draw(\n,\k)node[circ]{}node[above]{$(\n,\frac{1}{\n})$};}
\foreach \n in {2,4}{\pgfmathsetmacro{\k}{(-1)^(\n+1)/\n} \draw(\n,\k)node[circ]{}node[below]{$(\n,-\frac{1}{\n})$};}
\foreach \x in {1,2,3,4,5}{\draw(\x,0)--++(0,0.2);}
\foreach \y in {1}{\draw(0,\y)node[left]{$\y$}--++(0.12,0);}
\draw(3,1)node[above]{\RL{$0$ پر مرتکز}};
\end{tikzpicture}
\end{subfigure}
\caption{جزو $\tfrac{(-1)^{n+1}}{n}$ کی علامت ہر مرتبہ تبدیل ہوتی ہے لیکن اس کی قیمت  $0$ پر مرتکز ہے۔}
\label{شکل_تسلسل_مرتکز_منفرج_پ}
\end{figure}
%%%%%%%%%%%%%%%%%%%%%%%
\begin{figure}
\centering
\begin{subfigure}{0.45\textwidth}
\centering
\begin{tikzpicture}[font=\small,xscale=4]
\draw[-latex](-0.125,0)--(1.125,0);
\foreach \n in {1,2,3,4}{\pgfmathsetmacro{\k}{(\n-1)/\n} \draw(\k,0)node[circ]{}node[above]{$a_{\n}$};}
\foreach \x in {0,1}{\draw(\x,0)++(0,-0.15)node[below]{$\x$}--++(0,0.3);}
\draw(0.75,-0.5)node[below]{$a_n=\frac{n-1}{n}$};
\end{tikzpicture}
\end{subfigure}\hfill
\begin{subfigure}{0.45\textwidth}
\centering
\begin{tikzpicture}[font=\small,xscale=1]
\draw[-latex](-0.25,0)--(5.5,0)node[right]{$n$};
\draw[-latex](0,-0.2)--(0,1.5)node[above]{$a_n$};
\draw[gray](0,1)--(5.5,1);
\foreach \n/\nn in {1}{\draw(1,0)node[circ]{}node[above]{$(1,0)$};}
\foreach \n/\nn in {2/1,3/2,4/3,5/4}{\pgfmathsetmacro{\k}{(\n-1)/\n}; \draw(\n,\k)node[circ]{}node[above]{$(\n,\frac{\nn}{\n})$};}
\foreach \x in {1,2,3,4,5}{\draw(\x,0)--++(0,0.2);}
\foreach \y in {1}{\draw(0,\y)node[left]{$\y$}--++(0.12,0);}
\draw(1,1)node[above]{\RL{$1$ پر مرتکز}};
\end{tikzpicture}
\end{subfigure}
\caption{جیسے جیسے $n$ بڑھتا ہے جزو $a_n=\tfrac{n-1}{n}$ بتدریج $1$ تک پہنچتا ہے لہٰذا ترتیب $\{a_n\}$ مرتکز ہے $1$ پر۔}
\label{شکل_تسلسل_مرتکز_منفرج_ت}
\end{figure}
%%%%%%%%%%%%%%%%%%%%%%%%%%%
\begin{figure}
\centering
\begin{subfigure}{0.45\textwidth}
\centering
\begin{tikzpicture}[font=\small,xscale=2]
\draw[-latex](-1.125,0)--(1.125,0);
\foreach \n in {1,2,3,4,5}{\pgfmathsetmacro{\k}{(-1)^(\n+1)*(\n-1)/\n} \draw(\k,0)node[circ]{}node[above]{$a_{\n}$};}
\foreach \x in {-1,0,1}{\draw(\x,0)++(0,-0.15)node[below]{$\x$}--++(0,0.3);}
\draw(0,-0.5)node[below]{$a_n=(-1)^{n+1}\big(\frac{n-1}{n}\big)$};
\end{tikzpicture}
\end{subfigure}\hfill
\begin{subfigure}{0.45\textwidth}
\centering
\begin{tikzpicture}[font=\small,xscale=1]
\draw[-latex](-0.25,0)--(5.5,0)node[right]{$n$};
\draw[-latex](0,-1.25)--(0,1.25)node[above]{$a_n$};
\draw[gray](0,1)node[left,black]{$1$}--(5.5,1)  (0,-1)node[left,black]{$-1$}--(5.5,-1);
\foreach \n/\nn in {1}{\draw(1,0)node[circ]{}node[above]{$(1,0)$};}
\foreach \n/\nn in {3/2,5/4}{\pgfmathsetmacro{\k}{(-1)^(\n+1)*(\n-1)/\n}; \draw(\n,\k)node[circ]{}node[above]{$(\n,\tfrac{\nn}{\n})$};}
\foreach \n/\nn in {2/1,4/3}{\pgfmathsetmacro{\k}{(-1)^(\n+1)*(\n-1)/\n}; \draw(\n,\k)node[circ]{}node[below]{$(\n,-\tfrac{\nn}{\n})$};}
\foreach \x in {1,2,3,4,5}{\draw(\x,0)--++(0,0.2);}
\foreach \y in {3}{\draw(0,\y)node[left]{$\y$}--++(0.12,0);}
\draw(4,1)node[above]{\RL{منفرج}};
\end{tikzpicture}
\end{subfigure}
\caption{جزو $a_n=(-1)^{n+1}[\tfrac{n-1}{n}]$ کی علامت ہر قدم پر تبدیل ہوتی ہے۔ مثبت اجزاء $1$ کو پہنچتے ہیں جبکہ منفی اجزاء $-1$ کو پہنچتے ہیں لہٰذا ترتیب $\{a_n\}$ منفرج ہے۔}
\label{شکل_تسلسل_مرتکز_منفرج_ٹ}
\end{figure}
%%%%%%%%%%%%%%%%%%%%%%%
\begin{figure}
\centering
\begin{subfigure}{0.45\textwidth}
\centering
\begin{tikzpicture}[font=\small,xscale=0.75]
\draw[-latex](-0.25,0)--(5.5,0);
\foreach \n in {3}{\pgfmathsetmacro{\k}{\n} \draw(\k,0)node[circ]{}node[above]{$a_n$};}
\foreach \x in {0,1,2,3,4,5}{\draw(\x,0)++(0,-0.15)node[below]{$\x$}--++(0,0.3);}
\draw(2.5,-0.5)node[below]{$a_n=3$};
\end{tikzpicture}
\end{subfigure}\hfill
\begin{subfigure}{0.45\textwidth}
\centering
\begin{tikzpicture}[font=\small,xscale=1,yscale=0.3]
\draw[-latex](-0.25,0)--(5.5,0)node[right]{$n$};
\draw[-latex](0,-0.2)--(0,4)node[above]{$a_n$};
\foreach \n in {1,2,3,4,5}{\pgfmathsetmacro{\k}{3}; \draw(\n,\k)node[circ]{};}
\foreach \x in {1,2,3,4,5}{\draw(\x,0)--++(0,0.2);}
\foreach \y in {3}{\draw(0,\y)node[left]{$\y$}--++(0.12,0);}
\draw(4.5,1)node[above]{\RL{$3$ پر مرتکز}};
\end{tikzpicture}
\end{subfigure}
\caption{مستقل اجزاء $a_n=3$ کی قیمت $3$ ہی رہتی ہے لہٰذا ترتیب $\{a_n\}$ کی قیمت $3$ پر مرتکز ہے۔}
\label{شکل_تسلسل_مرتکز_منفرج_ث}
\end{figure}
%%%%%%%%%%%%%%%%%%
\جزوحصہء{ارتکاز اور انفراج}
آپ نے شکل \حوالہ{شکل_تسلسل_مرتکز_منفرج_الف} تا شکل \حوالہ{شکل_تسلسل_مرتکز_منفرج_ث} میں دیکھا کہ مثال \حوالہ{مثال_تسلسل_متعدد_ترتیبات} میں دیے گئے ترتیبات ایک جیسا رویہ نہیں رکھتے ہیں۔  متغیر \عددی{n} کی قیمت بڑھانے سے ترتیبات \عددی{\{\tfrac{1}{n}\}}، \عددی{\{\tfrac{(-1)^{n+1}}{n}\}} اور \عددی{\{\tfrac{n-1}{n}\}} میں ہر ایک کی قیمت کسی ایک منفرد تحدیدی قیمت تک پہنچتی ہے جبکہ ترتیب \عددی{\{3\}} ابتدا سے تحدیدی قیمت پر ہے۔ اس کے برعکس \عددی{\{(-1)^{n+1}\tfrac{(n-1)}{n}\}} کے اجزاء دو مختلف قیمتوں، \عددی{-1} اور \عددی{1}، پر جمع ہوتے ہیں جبکہ \عددی{\{\sqrt{n}\}} کے اجزاء بتدریج بڑھتے جاتے ہیں۔ 

ان ترتیبات میں امتیاز کرنے کی خاطر جو \عددی{n} بڑھانے سے کسی ایک منفرد قیمت \عددی{L} تک پہنچتی ہیں اور جو کسی منفرد قیمت تک نہیں پہنچتی ہیں، ہم ان ترتیبات کو جو \عددی{n} بڑھانے سے کسی ایک منفرد قیمت \عددی{L} تک پہنچتی ہو کو \ترچھا{مرتکز} کہتے ہیں۔ \ترچھا{ارتکاز} کی با ضابطہ تعریف درج ذیل ہے۔ 

\ابتدا{تعریف}
اگر ہر مثبت عدد \عددی{\epsilon} کے لئے ایسا  مطابقتی عدد صحیح \عددی{N} پایا جاتا ہو کہ ہر \عددی{n} کے لئے
\begin{align*}
n>N,\quad \implies \quad \abs{a_n-L}<\epsilon
\end{align*}
ہو تب ترتیب \عددی{\{a_n\}} عدد \عددی{L} پر \اصطلاح{مرتکز}\فرہنگ{مرتکز}\حاشیہب{convergent}\فرہنگ{convergent} ہو گی۔ اگر ایسا کوئی عدد \عددی{L} موجود نہ ہو تب ہم کہتے ہیں کہ \عددی{\{a_n\}} \اصطلاح{منفرج}\فرہنگ{منفرج}\حاشیہب{divergent}\فرہنگ{divergent} ہے۔

اگر \عددی{\{a_n\}} عدد \عددی{L} پر مرتکز ہو تب ہم \عددی{\lim_{n\to\infty}a_n=L} یا مختصراً \عددی{a_n\to L} لکھتے ہیں اور \عددی{L} کو اس ترتیب کا \اصطلاح{حد}\فرہنگ{حد}\حاشیہب{limit}\فرہنگ{limit} کہتے ہیں (شکل \حوالہ{شکل_تسلسل_ارتکاز_انفراج_تعریف})۔
\انتہا{تعریف}
%====================
\begin{figure}
\centering
\begin{subfigure}{0.45\textwidth}
\centering
\begin{tikzpicture}[font=\small,xscale=4/5,yscale=1]
\draw[-latex](-0.25,0)--(5,0);
\draw(2,0)node[circ]{}node[below]{$a_1$};
\draw(1,0)node[circ]{}node[below]{$a_2$};
\draw(1.5,0)node[circ]{}node[below]{$a_3$};
\draw(2.5,0)node[circ]{};
\draw(2.75,0)node[circ]{};
\draw(3.2,0)node[circ]{}node[below]{$a_N$};
\draw(3.8,0)node[circ]{}node[below]{$a_n$};
\draw(3.6,0)node[circ]{};
\draw(4.3,0)node[circ]{};
\draw(3.9,0)node[circ]{};
\draw(3.4,0)node[]{$($}node[above,xshift=-2ex,yshift=1ex]{$L-\epsilon$};
\draw(4.6,0)node[]{$)$}node[above,xshift=2ex,yshift=1ex]{$L+\epsilon$};
\draw(4,-0.1)--++(0,0.2)node[above]{$L$};
\draw(0,-0.1)node[below]{$0$}--++(0,0.2);
\end{tikzpicture}
\end{subfigure}\hfill
\begin{subfigure}{0.45\textwidth}
\centering
\begin{tikzpicture}[font=\small,xscale=0.5,yscale=0.5]
\draw[-latex](-0.25,0)--(11,0)node[right]{$n$};
\draw[-latex](0,-0.2)--(0,5)node[above]{$a_n$};
\draw[dashed](-0.25,4)node[left]{$L$}--(11,4);
\draw[](-0.25,3.4)--(11,3.4)node[right]{$L-\epsilon$};
\draw(8,3.6)node[above,fill=white]{$(n,a_n)$};
\draw[](-0.25,4.6)--(11,4.6)node[right]{$L+\epsilon$};
\draw(1,2)node[circ]{};
\draw(2,1)node[circ]{};
\draw(3,1.5)node[circ]{};
\draw(4,2.5)node[circ]{};
\draw(5,2.75)node[circ]{};
\draw(6,3.2)node[circ]{}node[below,xshift=1ex]{$(N,a_N)$};
\draw(7,3.8)node[circ]{};
\draw(8,3.6)node[circ]{};
\draw(9,4.3)node[circ]{};
\draw(10,3.9)node[circ]{};
\foreach \x/\s in {1/1,2/2,3/3,6/N,7/{},8/n}{\draw(\x,-0.1)node[below]{$\s$}--++(0,0.2);} 
\end{tikzpicture}
\end{subfigure}
\caption{
اگر نقاط \عددی{(n,a_n)} کی لکیر \عددی{y=L} افقی متقارب ہو تب \عددی{a_n\to L} ہو گا۔ اس شکل میں \عددی{a_N} کے بعد تمام \عددی{a_n} کا خط \عددی{L} سے فاصلہ \عددی{\epsilon} سے کم ہے۔
}
\label{شکل_تسلسل_ارتکاز_انفراج_تعریف}
\end{figure}

\ابتدا{مثال}\ترچھا{تعریف کی پرکھ}\\
درج ذیل دکھائیں۔
\begin{align*}
\text{\RL{(الف)}}\quad \lim_{n\to\infty}\frac{1}{n}&=0\\
\text{\RL{(ب)}}\quad\lim_{n\to\infty}k&=k&&\text{\RL{($k$ مستقل)}}
\end{align*}
حل:\quad
(الف) \quad
فرض کریں ہمیں \عددی{\epsilon>0} دیا گیا ہے۔ ہم نے دکھانا ہو گا کہ ایک ایسا عدد صحیح \عددی{N} پایا جاتا ہے کہ ہر \عددی{n} کے لئے 
\begin{align*}
n>N\quad \implies\quad \abs{\frac{1}{n}-0}<\epsilon
\end{align*}
ہو گا۔ یہ اس صورت ممکن ہو گا اگر \عددی{\tfrac{1}{n}<\epsilon} یا \عددی{n>\tfrac{1}{\epsilon}} ہو۔ اگر \عددی{\tfrac{1}{\epsilon}} سے \عددی{N} کوئی بھی بڑا عدد صحیح ہو تب کسی بھی \عددی{n>N} کے لئے درج بالا درست ہو گا۔ یوں ثابت ہوا کہ \عددی{\lim_{n\to\infty}(1/n)=0} ہے۔\\
(ب)\quad
فرض کریں ہمیں \عددی{\epsilon>0} دیا گیا ہے۔ ہم نے دکھانا ہو گا کہ ایک ایسا عدد صحیح \عددی{N} پایا جاتا ہے کہ ہر \عددی{n} کے لئے 
\begin{align*}
n>N\quad \implies\quad \abs{k-k}<\epsilon
\end{align*}
ہو گا۔ چونکہ \عددی{k-k=0} ہوتا ہے لہٰذا درج بالا کسی بھی مثبت عدد صحیح \عددی{N} کے لئے درست ہو گا۔یوں ثابت ہوا کہ کسی بھی مستقل \عددی{k} کے لئے \عددی{\lim_{n\to\infty}k=k} ہو گا۔
\انتہا{مثال}
%==================
\ابتدا{مثال}\شناخت{مثال_تسلسل_انفراج}
دکھائیں کہ \عددی{\{(-1)^{n+1}[\tfrac{n-1}{n}]\}} ہے۔

حل:\quad
ہم مثبت عدد \عددی{\epsilon} کو \عددی{1} سے کم چنتے ہیں تا کہ شکل \حوالہ{شکل_مثال_تسلسل_انفراج} میں \عددی{y=-1} اور \عددی{y=1} پر پٹیاں ایک دوسرے کو نہ ڈھانپیں۔ اگر کسی مخصوص \عددی{N} سے کسی بھی بڑے \عددی{n} کے لئے شکل \حوالہ{شکل_مثال_تسلسل_انفراج} میں نقطے بالائی پٹی میں پائے جاتے ہوں تب یہ ترتیب \عددی{1} پر مرتکز ہو گی۔ حقیقت میں جیسا ہی کوئی پہلا نقطہ \عددی{(n,a_n)} بالائی پٹی کے اندر آتا ہے، اس کے بعد \عددی{(n+1,a_{n+1})} سے شروع کرتے ہوئے  ہر متبادل نقطہ نچلی پٹی میں پایا جاتا ہے۔ یوں ترتیب کسی صورت \عددی{1} پر مرتکز نہیں ہو سکتی ہے۔ اسی طرح یہ ترتیب \عددی{-1} پر بھی مرتکز نہیں ہو سکتی ہے۔ ساتھ ہی ساتھ چونکہ ترتیب کے اجزاء \عددی{-1} یا \عددی{1} کے قریب تر ہوتے جاتے ہیں لہٰذا یہ کسی دوسرے نقطے کے قریب نہیں ہو سکتے ہیں لہٰذا یہ ترتیب منفرج ہے۔
\انتہا{مثال}
%===================
\begin{figure}
\centering
\begin{subfigure}{0.60\textwidth}
\centering
\begin{tikzpicture}[font=\small,xscale=2,declare function={f(\x)=(\x-1)/\x;}]
\draw[-latex](-1.5,0)--(1.5,0);
\foreach \n in {1,3,5}{\draw({f(\n)},0)node[circ]{}node[above]{$a_{\n}$};}
\foreach \n in {2,4}{\draw({-f(\n)},0)node[circ]{}node[above]{$a_{\n}$};}
\foreach \n in {6}{\draw({-f(\n)},0)node[circ]{}node[above]{\llap{$a_{\n}$}};}
\foreach \x in {-1,0,1}{\draw(\x,0)++(0,-0.15)node[below]{$\x$}--++(0,0.3);}
\draw(0,-0.5)node[below]{$a_n=(-1)^{n+1}\big(\frac{n-1}{n}\big)$};
\draw(-1-0.4,0)node[]{$($};
\draw(-1+0.4,0)node[]{$)$};
\draw(1-0.4,0)node[]{$($};
\draw(1+0.4,0)node[]{$)$};
\end{tikzpicture}
\end{subfigure}\hfill
\begin{subfigure}{0.30\textwidth}
\centering
\begin{tikzpicture}[font=\small,xscale=1/2,yscale=1,declare function={f(\x)=(\x-1)/\x;}]
\draw[-latex](-0.25,0)--(7,0);
\draw[-latex](0,-1.5)--(0,1.5)node[above]{$a_n$};
\draw(0,1+0.4)--(7,1+0.4)node[right]{$1+\epsilon$};
\draw[dashed](-0.25,1)--(7,1);
\draw(0,1-0.4)--(7,1-0.4)node[right]{$1-\epsilon$};
\draw(0,-1+0.4)--(7,-1+0.4)node[right]{$-1+\epsilon$};
\draw[dashed](-0.25,-1)--(7,-1);
\draw(0,-1-0.4)--(7,-1-0.4)node[right]{$-1-\epsilon$};
\draw(1,{f(1)})node[circ]{}node[above]{$(1,0)$};
\foreach \n/\s in {3/2,5/4}{\draw(\n,{f(\n)})node[circ]{}node[above]{$(\n,\tfrac{\s}{\n})$};}
\foreach \n/\s in {2/1,4/3,6/5}{\draw(\n,{-f(\n)})node[circ]{}node[below]{$(\n,-\tfrac{\s}{\n})$};}
\foreach \x in {1,2,3,4,5}{\draw(\x,0)--++(0,0.2);}
\foreach \y in {-1,1}{\draw(0,\y)node[left]{$\y$}--++(0.12,0);}
\end{tikzpicture}
\end{subfigure}
\caption{
تسلسل \عددی{\{(-1)^{n+1}[\tfrac{n-1}{n}]\}} منفرج ہے (مثال \حوالہ{مثال_تسلسل_انفراج})
}
\label{شکل_مثال_تسلسل_انفراج}
\end{figure}

ترتیب \عددی{\{(-1)^{n+1}[\tfrac{n-1}{n}]\}} کا رویہ \عددی{\{\sqrt{n}\}} کے رویے سے مختلف ہے۔ ترتیب \عددی{\{\sqrt{n}\}} کے منفرج ہونے کی وجہ یہ ہے کہ یہ ہر حقیقی عدد \عددی{L} سے تجاوز کرتا ہے۔اس رویے کو ہم
\begin{align*}
\lim_{n\to\infty}\sqrt{n}=\infty
\end{align*}
لکھتے ہیں۔لامتناہی حد سے یہاں ہمارا ہرگز یہ مطلب نہیں ہے کہ \عددی{n} بڑھانے سے \عددی{a_n} اور لامتناہی کے بیچ فرق کم ہوتا ہے۔ کہنے کا مطلب صرف اتنا ہے کہ  \عددی{n} بڑھانے سے \عددی{a_n} بہت بڑا ہو جاتا ہے۔ 

\جزوحصہء{تکراری تعریف}
اب تک ہم  \عددی{n} سے بلا واسطہ \عددی{a_n} تلاش کرتے آ رہے ہیں اگرچہ ترتیب کی عموماً تکراری تعریف  پیش کی جاتی ہے جہاں
\begin{enumerate}[a.]
\item
ابتدائی جزو یا اجزاء کی قیمتیں دی جاتی ہیں اور
\item
\اصطلاح{کلیہ توالی}\فرہنگ{کلیہ!توالی}\فرہنگ{توالی!کلیہ}\حاشیہب{recursion formula}\فرہنگ{recursion!formula} سے ہر جزو کو گزشتہ اجزاء کی قیمتوں سے حاصل کیا جاتا ہے۔ 
\end{enumerate}

کمپیوٹر پروگرام اور تفرقی مساوات کے اعدادی حل کے طریقوں میں توالی کلیات عموماً پائے جاتے ہیں۔

\ابتدا{مثال}\ترچھا{تواتر سے ترتیب کی بناوٹ}\\
\begin{enumerate}[a.]
\item
\عددی{a_1=1} اور \عددی{a_n=a_{n-1}+1} کا فقرہ مثبت اعداد کی ترتیب \عددی{1,2,3,\cdots,n,\cdots} کی تعریف پیش کرتا ہے۔ یوں \عددی{a_1=1} لیتے ہوئے \عددی{a_2=a_1+1=1+1=2}، \عددی{a_3=a_2+1=2+1=3}، وغیرہ، ہو گا۔
\item
 \عددی{a_1} اور \عددی{a_n=n\cdot a_{n-1}} کا فقرہ \اصطلاح{اعداد ضربیہ}\فرہنگ{اعداد ضربیہ}\حاشیہب{factorials}\فرہنگ{factorials} کی ترتیب \عددی{1,2,6,24,\cdots,n!,\cdots} کی تعریف پیش کرتا ہے۔ یوں \عددی{a_1=1} لیتے ہوئے \عددی{a_2=2\cdot a_1=2}، \عددی{a_3=3\cdot a_2=6}، \عددی{a_4=4\cdot a_3=24} وغیرہ، ہو گا۔
\item
\عددی{a_1=1}، \عددی{a_2=1} اور \عددی{a_{n+1}=a_n+a_{n-1}} کا فقرہ \اصطلاح{فبونیکی اعداد}\فرہنگ{فبونیکی اعداد}\حاشیہب{Fibonacci numbers}\فرہنگ{Fibonacci numbers} کی ترتیب  \عددی{1,1,2,3,5,\cdots} کی تعریف پیش کرتا ہے۔ یوں \عددی{a_1=1} اور \عددی{a_2=1} لیتے ہوئے \عددی{a_3=a_2+a_1=1+1=2}، \عددی{a_4=a_3+a_2=2+1=3}، \عددی{a_5=a_4+a_3=3+2=5}، وغیرہ، ہو گا۔
\item
جیسا ہم ترکیب نیوٹن کی اطلاق سے جانتے ہیں کہ \عددی{x_0=1} اور \عددی{x_{n+1}=x_n-[\tfrac{\sin x_n-x_n^2}{\cos x_n-2x_n}]} کا فقرہ ایسی ترتیب کی تعریف پیش کرتا ہے جو مساوات \عددی{\sin x-x^2=0} کے حل پر مرتکز ہوتی ہے۔
\end{enumerate}
\انتہا{مثال}
%===================

علامت \عددی{n!} (جس کو \عددی{n} کا \اصطلاح{ضربیہ عدد} کہتے ہیں) سے مراد \عددی{1} سے \عددی{n} تک اعداد صحیح کا حاصل ضرب \عددی{1\cdot 2\cdot 3\cdot\cdots \cdot n} ہے۔ آپ دیکھ سکتے ہیں کہ \عددی{(n+1)!=(n+1)\cdot n!} ہو گا لہٰذا
\begin{align*}
4!&=1\cdot 2\cdot 2\cdot 3\cdot 4=24,\\
5!&=1\cdot 2\cdot 3\cdot 4\cdot 5=5\cdot 4!=120
\end{align*}
ہوں گے۔ہم \عددی{0!} کی تعریف \عددی{1} لیتے ہیں۔

 جیسا جدول \حوالہ{جدول_تسلسل_فبونیکی_قوت_نما} میں دکھایا گیا ہے  قوت نما سے بھی زیادہ تیزی سے فبونیکی اعداد  بڑھتے ہیں۔
\begin{table}
\caption{قوت نما سے فبونیکی اعداد زیادہ تیزی سے بڑھتے ہیں۔}
\label{جدول_تسلسل_فبونیکی_قوت_نما}
\centering
\begin{tabular}{RRR}
n&e^n&n!\\
\toprule
1&3&1\\
5&148&120\\
10&\num{22026}&\num{3628800}\\
20&\num{4.9e8}&\num{2.4e18}\\
\bottomrule
\end{tabular}
\end{table}

\جزوحصہء{ذیلی ترتیبات}
اگر ایک ترتیب کے  اجزاء اسی ترتیب سے دوسری ترتیب میں پائے جاتے ہوں تب ہم پہلی ترتیب کو دوسری ترتیب کی \اصطلاح{ذیلی ترتیب}\فرہنگ{ترتیب!ذیلی}\حاشیہب{subsequence}\فرہنگ{sequence!sub} کہتے ہیں۔ 

\ابتدا{مثال}\ترچھا{مثبت اعداد صحیح کی ترتیب کی  ذیلی ترتیبات}\\
\begin{enumerate}[a.]
\item
جفت اعداد صحیح کی ذیلی ترتیب \عددی{2,4,6,\cdots,2n,\cdots}
\item
طاق اعداد صحیح کی ذیلی ترتیب \عددی{1,3,5,\cdots 2n-1,\cdots}
\item
اعداد مفرد کی ذیلی ترتیب \عددی{2,3,5,7,11,\cdots}
\end{enumerate}
\انتہا{مثال}
%===================

ذیلی ترتیبات کی اہمیت کے دو وجوہات ہیں۔
\begin{enumerate}[a.] 
\item
اگر تسلسل \عددی{\{a_n\}} مستقل \عددی{L} کو مرتکز ہو تب اس کے تمام ذیلی ترتیبات بھی \عددی{L} پر مرکوز ہوں گی۔ اگر ہم جانتے ہوں کہ ایک تسلسل مرتکز ہے تب اس کے کسی مخصوص ذیلی تسلسل سے حد کی تلاش یا اس کا تخمینہ لگانا زیادہ آسان ثابت ہو سکتا ہے۔
\item
اگر \عددی{\{a_n\}} کا کوئی بھی ذیلی تسلسل منفرج ہو یا اس کے کسی دو ذیلی ترتیبات کے حد ایک دوسرے سے مختلف ہوں تب \عددی{\{a_n\}} منفرج ہو گا۔ مثال کے طور پر تسلسل \عددی{\{(-1)^n\}} منفرج ہو گا چونکہ طاق اجزاء کی ذیلی تسلسل \عددی{-1,-1,-1,\cdots} کی حد \عددی{-1} ہے جبکہ جفت اجزاء کی ذیلی تسلسل \عددی{1,1,1,\cdots} کی حد \عددی{1} ہے جو ایک مختلف حد ہے۔
\end{enumerate}

ذیلی تسلسل کی مدد سے ارتکاز کو ایک نئی نظر سے دیکھا جا سکتا ہے۔ کسی اشاریہ \عددی{N} کے بعد تمام اجزاء کو تسلسل کی \اصطلاح{دم}\فرہنگ{تسلسل!دم}\حاشیہب{tail}\فرہنگ{sequence!tail} کہتے ہیں جو ایک ذیلی تسلسل ہو گی۔ یوں سلسلہ \عددی{\{a_n|n\ge N\}} میں سے کسی ایک کو دم کہا جا سکتا ہے۔ یوں \عددی{a_n\to L} کی جگہ ہم کہہ سکتے ہیں کہ \عددی{L} کے ارد گرد \عددی{\epsilon} وقفہ میں تسلسل کی دم پائی جائے گی۔

کسی بھی تسلسل کی ارتکاز یا انفراج کا تسلسل کی ابتدا کے ساتھ کوئی تعلق نہیں پایا جاتا ہے۔ تسلسل کی ارتکاز یا انفراج صرف تسلسل کی دم پر منحصر ہو گی۔

\جزوحصہء{محدود غیر گھٹتا تسلسل}
\ابتدا{تعریف}
ایسا تسلسل جو تمام \عددی{n} کے لئے \عددی{a_n\le a_{n+1}} خاصیت رکھتا ہو \اصطلاح{غیر گھٹتا تسلسل}\فرہنگ{تسلسل!غیر گھٹتا}\حاشیہب{nondecreasing sequence}\فرہنگ{sequence!nondecreasing} کہلاتا ہے۔ 
\انتہا{تعریف}
%=======================

\ابتدا{مثال}\ترچھا{غیر گھٹتا تسلسل}\\
\begin{enumerate}[a.]
\item
قدرتی اعداد کا تسلسل \عددی{1,2,3,\cdots,n,\cdots}
\item
تسلسل \عددی{\frac{1}{2},\frac{2}{3},\frac{3}{4},\cdots,\frac{n}{n+1},\cdots}
\item
مستقل تسلسل \عددی{\{3\}}
\end{enumerate}
\انتہا{مثال}
%======================

غیر گھٹتا تسلسل کی دو قسمیں ہیں۔ پہلی قسم کے اجزاء آخر کار ہر متناہی حد بندی سے بڑھ جاتے ہیں جبکہ دوسری قسم کے اجزاء کسی مخصوص حد بندی سے تجاوز نہیں کرتے ہیں۔

\ابتدا{تعریف}
اگر ایک ایسا عدد \عددی{M} موجود ہو کہ تمام \عددی{n} کے لئے \عددی{a_n\le M} ہوں تب تسلسل \عددی{\{a_n\}} کی \اصطلاح{بالائی حد بندی}\فرہنگ{حد بندی!بالائی}\حاشیہب{upper bound}\فرہنگ{bound!upper} \عددی{M} ہو گی۔ ہم کہتے ہیں کہ تسلسل \عددی{\{a_n\}} \اصطلاح{اوپر سے محدود}\فرہنگ{محدود!اوپر سے}\حاشیہب{bounded from above}\فرہنگ{bounded!from above} ہے۔ اگر \عددی{M} سے کم کوئی بھی عدد، \عددی{\{a_n\}} کی بالائی حد بندی نہ ہو، تب \عددی{M} کو \عددی{\{a_n\}} کی \اصطلاح{کم سے کم بالائی حد بندی}\فرہنگ{حد بندی!کم سے کم بالائی}\حاشیہب{least upper bound}\فرہنگ{bound!least upper} کہتے ہیں۔
\انتہا{تعریف}
%==================

\ابتدا{مثال}
\begin{enumerate}[a.]
\item
تسلسل \عددی{1,2,3,\cdots,n,\cdots} کی کوئی بالائی حد بندی نہیں پائی جاتی ہے۔
\item
تسلسل \عددی{\frac{1}{2},\frac{2}{3},\frac{3}{4},\cdots,\frac{n}{n+1},\cdots} اوپر سے محدود ہے اور اس کی بالائی حد بندی \عددی{M=1} ہے۔کوئی بھی عدد جو \عددی{1} سے چھوٹا ہو اس تسلسل کی بالائی حد بندی نہیں ہو سکتی ہے لہٰذا اس تسلسل کی کم سے کم بالائی حد بندی \عددی{1} ہے۔
\end{enumerate}
\انتہا{مثال}
%========================

ایسے غیر گھٹتا تسلسل کی کم سے کم بالائی حد بندی ضرور پایا جائے گا جو اوپر سے محدود ہو۔ یہ حقیقت، جس کو ہم یہاں ثابت نہیں کریں گے، حقیقی اعداد کی مکملیت کی خاصیت کی بنا ہے۔ البتہ ہم یہ ثابت کرتے ہیں کہ اگر \عددی{L} کم سے کم بالائی حد بندی ہو تب  تسلسل \عددی{L} پر مرتکز ہو گا۔

فرض کریں ہم \عددی{(1,s_1),(2,s_2),\cdots,(n,s_n),\cdots}  نقطوں کو \عددی{xy} مستوی میں ترسیم کرتے ہیں۔ اگر اس تسلسل کی بالائی حد بندی  \عددی{M} ہو تب یہ تمام نقطے لکیر \عددی{y=M} کے نیچے پائے جائیں گے (شکل \حوالہ{شکل_تسلسل_بالائی_حد_بندی_اور_حد})۔ لکیر \عددی{y=L} سب سے نچلی ایسی لکیر ہو گی۔ نقاط \عددی{(n,s_n)} میں سے کوئی بھی اس لکیر سے اوپر نہیں ہو گا اگرچہ اس سے نیچے لکیر \عددی{y=L-\epsilon} سے چند نقطے  ضرور اوپر  ہوں گے، جہاں \عددی{\epsilon} مثبت عدد ہے۔ یہ ترتیب درج ذیل وجوہات کی بنا \عددی{L} پر مرتکز ہو گی:
\begin{enumerate}[a.]
\item
تمام \عددی{n} کے لئے \عددی{s_n\le L} ہو گا اور
\item
کسی بھی دیے گئے عدد \عددی{\epsilon>0} کے لئے کم سے کم ایک ایسا عدد \عددی{N} موجود ہو گا جس کے لئے \عددی{s-N>L-\epsilon} ہو گا۔
\end{enumerate}
مزید \عددی{\{s_n\}} غیر گھٹتا ہے لہٰذا
\begin{align*} 
s_n\ge s_N>L-\epsilon\quad\quad  \text{\RL{تمام $n\ge N$ کے لئے}}
\end{align*}
ہو گا۔ یوں \عددی{N} کے بعد تمام اعداد \عددی{s_n} کا \عددی{L} سے فاصلہ عدد \عددی{\epsilon} سے کم ہو گا۔یہی وہ شرط ہے جس کی بنا تسلسل \عددی{s_n} کی حد \عددی{L} ہو گی۔
\begin{figure}
\centering
\begin{tikzpicture}[yscale=2,declare function={f(\x)=(\x-1)/\x;}]
\draw[-latex](-0.25,0)--(9,0)node[right]{$x$};
\draw[-latex](0,-0.2)--(0,1.4)node[above]{$y$};
\foreach \a in {2,3,4,5,6,7,8}{\draw(\a,{f(\a)})node[circ]{};}
\foreach \x in {1,2,3,4,5,6,7,8}{\draw(\x,-0.1)node[below]{$\x$}--++(0,0.2);}
\draw(1,0.3)node[circ]{}node[above]{$(1,s_1)$};
\draw(5,{4/5})node[below]{$(5,s_5)$};
\draw(8,{7/8})node[below]{$(8,s_8)$};
\draw(0,1)node[left]{$L$}--++(9,0)node[right]{$y=L$};
\draw(0,1.2)node[left]{$M$}--++(9,0)node[right]{$y=M$};
\end{tikzpicture}
\caption{اگر غیر گھٹتے تسلسل کی بالائی حد بندی \عددی{M} ہو تب اس کے حد \عددی{L\le M} بھی ہوں گے۔}
\label{شکل_تسلسل_بالائی_حد_بندی_اور_حد}
\end{figure}

\ابتدا{مسئلہ}\شناخت{مسئلہ_تسلسل_غیر_گھٹتا_تسلسل}\ترچھا{غیر گھٹتا تسلسل کا مسئلہ}\\
حقیقی اعداد کا ایک غیر گھٹتا تسلسل صرف اور صرف اس صورت مرتکز ہو گا جب یہ تسلسل اوپر سے محدود ہو۔ اگر ایک غیر گھٹتا تسلسل مرتکز ہو، یہ اپنے کم سے کم بالائی حد بندی پر مرتکز ہو گا۔
\انتہا{مسئلہ}
%=======================

