\حصہء{سوالات}
\موٹا{محددی لکیر پر حرکت}\\
سوال \حوالہ{سوال_تفرق_محددی_لکیر_الف} تا سوال \حوالہ{سوال_تفرق_محددی_لکیر_ب} میں \عددی{a\le t\le b} کے لئے \عددی{s=f(t)} محددی لکیر پر ایک جسم کا مقام دیتی ہے جہاں \عددی{t} کی اکائی سیکنڈ اور \عددی{s} کی اکائی میٹر ہے۔
\begin{enumerate}[a.]
\item
دیے گئے وقفے پر جسم کا ہٹاو اور سمتی رفتار حاصل کریں۔
\item
اس وقفے کے  آخری سروں پر جسم کی رفتار اور اسراع تلاش کریں۔
\item
جسم کب حرکت کی سمت تبدیل کرتا ہے (اگر ایسا کرتا ہو)؟   
\end{enumerate}

\ابتدا{سوال}\شناخت{سوال_تفرق_محددی_لکیر_الف}
$s=0.8t^2,\quad 0\le t\le 10 \quad \text{\RL{چاند پر آزادانہ گرنا}}$
\انتہا{سوال}
%======================
\ابتدا{سوال}
$s=1.86t^2,\quad 0\le t\le 0.5\quad \text{\RL{مریخ پر آزادانہ گرنا}}$
\انتہا{سوال}
%========================
\ابتدا{سوال}
$s=-t^3+3t^2-3t,\quad 0\le t\le 3$
\انتہا{سوال}
%=========================
\ابتدا{سوال}
$s=\tfrac{t^4}{4}-t^3+t^2,\quad 0\le t\le 2$
\انتہا{سوال}
%=========================
\ابتدا{سوال}
$s=\tfrac{25}{t^2}-\tfrac{5}{t},\quad 1\le t\le 5$
\انتہا{سوال}
%========================
\ابتدا{سوال}\شناخت{سوال_تفرق_محددی_لکیر_ب}
$s=\tfrac{25}{t+5},\quad -4\le t\le 0$
\انتہا{سوال}
%========================
\ابتدا{سوال}
\عددی{s} محور پر لمحہ \عددی{t} پر  ایک جسم کا مقام \عددی{s=t^3-6t^2+9t} ہے۔ (ا) ان نقطوں پر اس جسم کی اسراع تلاش کریں جن پر جسم کی سمتی رفتار صفر ہو گی۔ (ب) جب جسم کی اسراع صفر ہو اس لمحے پر اس جسم کی رفتار کیا ہو گی؟ (ج) لمحہ \عددی{t=0} تا \عددی{t=2} کے دوران یہ جسم کل کتنا فاصلہ طے کرتی ہے۔
\انتہا{سوال}
%==========================
\ابتدا{سوال}
وقت \عددی{t\ge 0} پر \عددی{s} محور پر حرکت کرتے ہوئے جسم کی سمتی رفتار \عددی{v=t^2-4t+3} ہے۔ (ا) جسم کی اسراع وہاں تلاش کریں جہاں جسم کی سمتی رفتار صفر ہے۔ (ب)  جسم کب آگے رخ اور کب پیچھے رخ حرکت کرتی ہے؟ (ج) جسم کی سمتی رفتار کب بڑھتی اور کب گھٹتی ہے؟
\انتہا{سوال}
%============================
\موٹا{آزادانہ گرنا}

\ابتدا{سوال}
مریخ اور مشتری کی سطح کے قریب آزادانہ گرنے کے مساوات بالترتیب \عددی{s=1.86t^2} اور \عددی{s=11.44t^2} ہیں جہاں \عددی{t} کی اکائی سیکنڈ اور \عددی{s} کی اکائی میٹر ہے۔ ساکن حال سے گرتے ہوئے کتنے وقت میں (مریخ اور مشتری میں) ایک جسم کی رفتار \عددی{\SI{27.8}{\meter\per\second}} یعنی تقریباً \عددی{\SI{100}{\kilo\meter\per\hour}} ہو گی؟
\انتہا{سوال}
%========================
\ابتدا{سوال}\شناخت{سوال_تفرق_پتھر_مریخ}
سطح چاند سے  انتصابی رخ \عددی{\SI{25}{\meter\per\second}} کی رفتار سے  پھینکا گیا پتھر \عددی{t} سیکنڈوں میں \عددی{s=24t-0.8t^2} میٹر بلندی پر پہنچے گا۔
\begin{enumerate}[a.]
\item
لمحہ \عددی{t} پر پتھر کی اسراع کیا ہو گی؟ (یہ اسراع چاند پر کشش ثقل کی اسراع ہو گی۔)
\item
پتھر بلند ترین مقام تک کتنے دورانیے میں پہنچے گا؟
\item
پتھر کتنی بلندی تک پہنچ پائے گا؟
\item
بلند ترین مقام کی نصف تک پتھر کتنی دیر میں پہنچے گا؟
\item
پتھر  کتنے وقت میں سطح چاند پر گرے گا؟ 
\end{enumerate}   
\انتہا{سوال}
%=========================
\ابتدا{سوال}
سطح زمین پر ہوا کی  غیر موجودگی میں سوال \حوالہ{سوال_تفرق_پتھر_مریخ} کا پتھر \عددی{t} سیکنڈوں میں \عددی{s=24t-4.9t^2} بلندی پر ہو گا۔
\begin{enumerate}[a.]
\item
لمحہ \عددی{t} پر پتھر کی اسراع کیا ہو گی؟ (یہ اسراع چاند پر کشش ثقل کی اسراع ہو گی۔)
\item
پتھر بلند ترین مقام تک کتنے دورانیے میں پہنچے گا؟
\item
پتھر کتنی بلندی تک پہنچ پائے گا؟
\item
بلند ترین مقام کی نصف تک پتھر کتنی دیر میں پہنچے گا؟
\item
پتھر  کتنے وقت میں سطح چاند پر گرے گا؟ 
\end{enumerate}   
\انتہا{سوال}
%========================
\ابتدا{سوال}
ہوا سے خالی ایک دنیا پر ایک ٹھوس جسم کو انتصابی رخ \عددی{\SI{15}{\meter\per\second}} کی ابتدائی رفتار سے پھینکا گیا۔ اس دنیا کے سطح پر ثقلی اسراع  \عددی{g_s \,\si{\meter\per\second\squared}} ہونے کی بنا \عددی{t} سیکنڈوں میں جسم \عددی{s=15t-\tfrac{1}{2}g_st^2}  میٹر بلندی تک پہنچے گا۔یہ جسم بلند ترین مقام تک \عددی{20} سیکنڈوں میں پہنچتا ہے۔ اس دنیا میں ثقلی اسراع کتنی ہے؟
\انتہا{سوال}
%==========================
\ابتدا{سوال}
چاند پر ایک بندوق کو انتصابی رخ چلایا گیا۔بندوق کی گولی \عددی{t} سیکنڈوں میں \عددی{s=300t-4.9t^2} میٹر بلندی پر ہو گی۔چاند پر یہی گولی \عددی{t} سیکنڈ بعد \عددی{s=300t-0.8t^2} میٹر بلندی پر ہو گی۔دونوں صورتوں میں گولی کتنی دیر بعد سطح پر گرے گی؟
\انتہا{سوال}
%=============================
\ابتدا{سوال}
مشتری پر ہوا کی غیر موجودگی میں یہی گولی \عددی{t} سیکنڈ بعد \عددی{s=300t-11.44t^2} میٹر بلندی پر ہو گی جبکہ مریخ پر یہ \عددی{s=300t-1.86t^2} میٹر کی بلندی پر ہو گی۔دونوں صورتوں میں گولی کتنے بلندی تک پہنچے گی؟
\انتہا{سوال}
%=================
\ابتدا{سوال}

\انتہا{سوال}
%=======================
