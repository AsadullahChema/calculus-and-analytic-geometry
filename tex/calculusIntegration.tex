\باب{تکمل}
اس باب میں دو اعمال اور ان کا ایک دوسرے کے ساتھ تعلق پر غور کیا جائے گا۔ پہلے عمل میں  ہم تفرق سے تفاعل  حاصل کرتے ہیں۔ دوسرے عمل میں ہم حجم، رقبہ، وغیرہ کے بالکل درست کلیات، بذریعہ یک بعد دیگرے تخمین، دریافت کرتے ہیں۔ ان دونوں اعمال کو تکمل کہتے ہیں۔

تکمل اور تفرق کا گہرا تعلق ہے۔ یہ تعلق تمام ریاضیات میں اہم ترین حقائق میں سے ایک ہے۔ لیبنٹز اور نیوٹن نے علیحدہ علیحدہ اس تعلق کو دریافت کیا۔

\حصہ{غیر قطعی تکملات}
کسی جسم کے موجودہ مقام اور سمتی رفتار سے اس کے مستقبل کے  مقام کی پیش گوئی کرنا احصاء کی اولین کامیابیوں میں سے ایک تھی۔ آج کل تفاعل کی کسی  ایک معلوم قیمت اور شرح تبدیلی سے تفاعل کے دیگر قیمتوں کا حصول معمول کی بات ہے۔ہم احصاء کی مدد سے  کشش زمین سے نکلنے کے لئے درکار رفتار یا تابکار مادہ کی موجودہ عملیت اور شرح تابکاری تحلیل سے اس کی قابل استعمال زندگی کا حساب لگا سکتے ہیں۔

تفاعل کی معلوم قیمتوں میں سے کسی ایک قیمت اور تفاعل کے تفرق \عددی{f(x)} سے تفاعل کا حصول دو قدموں میں ممکن ہے۔ پہلے قدم میں وہ تمام تفاعل حاصل کیے جاتے ہیں جن کا تفرق \عددی{f} ہے۔ ان تفاعل کو \عددی{f} کے الٹ تفرقات کہتے ہیں اور جس کلیہ سے انہیں اخذ کیا جاتا ہے اس کو \عددی{f} کا غیر قطعی  تکمل  کہتے ہیں۔ دوسرے قدم میں تفاعل کی معلوم قیمت استعمال کرتے ہوئے الٹ تفرقات میں سے موزوں تفاعل منتخب کیا جاتا ہے۔ اس حصہ میں پہلے قدم پر غور کیا جائے گا جبکہ دوسرے قدم پر اگلے حصہ میں غور کیا جائے گا۔

اگرچہ تفاعل کے تمام الٹ تفرقات حاصل کرنے والا کلیہ دریافت کرنا  ناممکن نظر آتا ہے، حقیقت میں ایسا نہیں ہے۔ مسئلہ اوسط قیمت (مسئلہ \حوالہ{مسئلہ_استعمال_اوسط_قیمت}) کے  پہلا اور دوسرا ضمنی نتائج کی مدد سے تفاعل کے  ایک الٹ تفرق سے اس کے تمام الٹ تفرقات حاصل کیے جا سکتے ہیں۔ 

\جزوحصہء{الٹ تفرق کا حصول۔ غیر قطعی تکمل}
\ابتدا{تعریف}
تفاعل \عددی{f(x)} کا  الٹ تفرق تب \عددی{F(x)} ہو گا جب \عددی{f} کے دائرہ کار میں تمام \عددی{x} کے لئے درج ذیل مطمئن ہوتا ہو۔
\begin{align*}
F'(x)=f(x)
\end{align*}
\عددی{f} کے تمام الٹ تفرقات کا سلسلہ \عددی{x} کے لحاظ سے \عددی{f} کا \اصطلاح{غیر قطعی تکمل}\فرہنگ{تکمل!غیر قطعی}\حاشیہب{indefinite integral}\فرہنگ{integral!indefinite} ہو گا جس کو درج ذیل سے ظاہر کیا جاتا ہے۔
\begin{align*}
\int f(x)\dif x
\end{align*}
علامت \عددی{\int} کو \اصطلاح{علامت تکمل} کہتے ہیں۔ تفاعل \عددی{f} کو \اصطلاح{متکمل}\فرہنگ{متکمل}\حاشیہب{integrand}\فرہنگ{integrand} اور \عددی{x} کو \اصطلاح{تکمل کا متغیر}\فرہنگ{تکمل!متغیر}\حاشیہب{variable of integration}\فرہنگ{integration!variable} کہتے ہیں۔
\انتہا{تعریف}
%======================

مسئلہ اوسط قیمت (مسئلہ \حوالہ{مسئلہ_استعمال_اوسط_قیمت}) کے  دوسرے ضمنی نتیجہ کے تحت تفاعل \عددی{f} کے  حاصل کردہ الٹ تفرق \عددی{F} اور اس کے  کسی دوسرے الٹ تفرق  میں صرف مستقل کا فرق پایا جائے گا۔ اس حقیقت کو تکملی علامتیت میں ظاہر کرتے ہیں:
\begin{align}\label{مساوات_تکمل_غیر_قطعی_الف}
\int f(x)\dif x=F(x)+C
\end{align}
مستقل \عددی{C} کو \اصطلاح{تکمل کا مستقل}\فرہنگ{تکمل!کا مستقل}\حاشیہب{constant of integration}\فرہنگ{integration!constant of} یا  \اصطلاح{اختیاری مستقل}\فرہنگ{مستقل!اختیاری}\حاشیہب{arbitrary constant}\فرہنگ{constant!arbitrary} کہتے ہیں۔ ہم مساوات \حوالہ{مساوات_تکمل_غیر_قطعی_الف} کو یوں پڑھتے ہیں: "\عددی{x} کے لحاظ سے تفاعل \عددی{f} کا غیر قطعی تکمل \عددی{F(x)+C} ہے۔"  \عددی{F(x)+C} کے حصول کو \عددی{f} کے \اصطلاح{تکمل} کا حصول کہتے ہیں۔

\ابتدا{مثال}
\عددی{\int 2x\dif x} تلاش کریں۔\\
حل:
\begin{align*}
\int 2x\dif x=x^2+C
\end{align*}
\عددی{2x} کا الٹ تفرق \عددی{x^2} ہے اور \عددی{C} تکمل کا مستقل ہے۔کلیہ \عددی{x^2+C} تفاعل \عددی{2x} کے تمام تفرقات دیتا ہے۔یوں \عددی{x^2+1}، \عددی{x^2-\pi} اور \عددی{x^2+\sqrt{2}} تفاعل \عددی{2x} کے ممکنہ الٹ تفرق ہیں۔ آپ ان کا تفرق لے کر تصدیق کر سکتے ہیں۔
\انتہا{مثال}
%======================

ہم عموماً تفرق کے کلیات سے الٹ تفرقات کے کلیات اخذ کرتے ہیں۔جدول \حوالہ{جدول_تکمل_کلیات_الف} میں غیر قطعی تکملات کے سامنے موزوں تفرقی کلیات کو الٹ لکھا گیا ہے۔ 
\begin{table}
\caption{تکمل کے کلیات}
\label{جدول_تکمل_کلیات_الف}
\centering
\renewcommand{\arraystretch}{1.5} 
\begin{tabular}{@{}LLL@{}}
\toprule
&\text{\RL{غیر قطعی تکمل}}&\text{\RL{تفرقی کلیات کو الٹ لکھا گیا ہے}}\\ 
\midrule
1.&\int x^n\dif x=\frac{x^{n+1}}{n+1}+C, \quad n\ne -1, \,n\text{ناطق} &\frac{\dif}{\dif x}\big(\frac{x^{n+1}}{n+1}\big)=x^n\\ 
&\int \dif x=\int 1\dif x=x+C \quad \text{\RL{(خصوصی صورت)}}&\frac{\dif}{\dif x}(x)=1\\ 
2.&\int\sin kx\dif x=-\frac{\cos kx}{k}+C&\frac{\dif}{\dif x}(-\frac{\cos kx}{k})=\sin kx\\ 
3.&\int\cos kx\dif x=\frac{\sin kx}{k}+C&\frac{\dif}{\dif x}(\frac{\sin kx}{k})=\cos kx\\ 
4.&\int\sec^2x\dif x=\tan x+C&\frac{\dif}{\dif x}\tan x=\sec^2x\\ 
5.&\int\csc^2x\dif x=-\cot x+C&\frac{\dif}{\dif x}(-\cot x)=\csc^2x \\ 
6.&\int\sec x\tan x\dif x=\sec x+C&\frac{\dif}{\dif x}\sec x=\sec x \tan x\\ 
7.&\int\csc x\cot x\dif x=-\csc x+C&\frac{\dif}{\dif x}(-\csc x)=\csc x\cot x\\
\bottomrule
\end{tabular}
\end{table}

\ابتدا{مثال}
\begin{enumerate}[a.]
\item
 جدول \حوالہ{جدول_تکمل_کلیات_الف} کے کلیہ 1 میں $n=5$ لیتے ہوئے:
\begin{align*}\int x^5\dif x=\frac{x^6}{6}+C\end{align*}

\item
کلیہ 1 میں \عددی{n=-\tfrac{1}{2}} لیتے ہوئے:
\begin{align*}\int \frac{1}{\sqrt{x}}\dif x=\int x^{-\tfrac{1}{2}}\dif x=2x^{\tfrac{1}{2}}+C\end{align*}

\item
کلیہ 2 میں \عددی{k=2} لیتے ہوئے:
\begin{align*}\int\sin 2x\dif x=-\frac{\cos 2x}{2}+C\end{align*}

\item
کلیہ 3 میں \عددی{k=\tfrac{1}{2}} لیتے ہوئے:
\begin{align*}\int\cos \frac{x}{2}\dif x=\int\frac{1}{2}x\dif x=\frac{\sin \tfrac{1}{2}x}{\tfrac{1}{2}}+C=2\sin\frac{x}{2}+C\end{align*}
\end{enumerate}
\انتہا{مثال}
%================

بعض اوقات کلیہ تکمل کا حصول مشکل ثابت ہوتا ہے البتہ  اخذ کردہ کلیہ کو پرکھنا مشکل نہیں ہے۔ کلیہ کا تفرق متکمل ہو گا۔

