\باب{تکمل}
اس باب میں دو اعمال اور ان کا ایک دوسرے کے ساتھ تعلق پر غور کیا جائے گا۔ پہلے عمل میں  ہم تفرق سے تفاعل  حاصل کرتے ہیں۔ دوسرے عمل میں ہم حجم، رقبہ، وغیرہ کے بالکل درست کلیات، بذریعہ یک بعد دیگرے تخمین، دریافت کرتے ہیں۔ ان دونوں اعمال کو تکمل کہتے ہیں۔

تکمل اور تفرق کا گہرا تعلق ہے۔ یہ تعلق تمام ریاضیات میں اہم ترین حقائق میں سے ایک ہے۔ لیبنٹز اور نیوٹن نے علیحدہ علیحدہ اس تعلق کو دریافت کیا۔

\حصہ{غیر قطعی تکملات}
کسی جسم کے موجودہ مقام اور سمتی رفتار سے اس کے مستقبل کے  مقام کی پیش گوئی کرنا احصاء کی اولین کامیابیوں میں سے ایک تھی۔ آج کل تفاعل کی کسی  ایک معلوم قیمت اور شرح تبدیلی سے تفاعل کے دیگر قیمتوں کا حصول معمول کی بات ہے۔ہم احصاء کی مدد سے  کشش زمین سے نکلنے کے لئے درکار رفتار یا تابکار مادہ کی موجودہ عملیت اور شرح تابکاری تحلیل سے اس کی قابل استعمال زندگی کا حساب لگا سکتے ہیں۔

تفاعل کی معلوم قیمتوں میں سے کسی ایک قیمت اور تفاعل کے تفرق \عددی{f(x)} سے تفاعل کا حصول دو قدموں میں ممکن ہے۔ پہلے قدم میں وہ تمام تفاعل حاصل کیے جاتے ہیں جن کا تفرق \عددی{f} ہے۔ ان تفاعل کو \عددی{f} کے الٹ تفرقات کہتے ہیں اور جس کلیہ سے انہیں اخذ کیا جاتا ہے اس کو \عددی{f} کا غیر قطعی  تکمل  کہتے ہیں۔ دوسرے قدم میں تفاعل کی معلوم قیمت استعمال کرتے ہوئے الٹ تفرقات میں سے موزوں تفاعل منتخب کیا جاتا ہے۔ اس حصہ میں پہلے قدم پر غور کیا جائے گا جبکہ دوسرے قدم پر اگلے حصہ میں غور کیا جائے گا۔

اگرچہ تفاعل کے تمام الٹ تفرقات حاصل کرنے والا کلیہ دریافت کرنا  ناممکن نظر آتا ہے، حقیقت میں ایسا نہیں ہے۔ مسئلہ اوسط قیمت (مسئلہ \حوالہ{مسئلہ_استعمال_اوسط_قیمت}) کے  پہلا اور دوسرا ضمنی نتائج کی مدد سے تفاعل کے  ایک الٹ تفرق سے اس کے تمام الٹ تفرقات حاصل کیے جا سکتے ہیں۔ 

\جزوحصہء{الٹ تفرق کا حصول۔ غیر قطعی تکمل}
\ابتدا{تعریف}
تفاعل \عددی{f(x)} کا  الٹ تفرق تب \عددی{F(x)} ہو گا جب \عددی{f} کے دائرہ کار میں تمام \عددی{x} کے لئے درج ذیل مطمئن ہوتا ہو۔
\begin{align*}
F'(x)=f(x)
\end{align*}
\عددی{f} کے تمام الٹ تفرقات کا سلسلہ \عددی{x} کے لحاظ سے \عددی{f} کا \اصطلاح{غیر قطعی تکمل}\فرہنگ{تکمل!غیر قطعی}\حاشیہب{indefinite integral}\فرہنگ{integral!indefinite} ہو گا جس کو درج ذیل سے ظاہر کیا جاتا ہے۔
\begin{align*}
\int f(x)\dif x
\end{align*}
علامت \عددی{\int} کو \اصطلاح{علامت تکمل} کہتے ہیں۔ تفاعل \عددی{f} کو \اصطلاح{متکمل}\فرہنگ{متکمل}\حاشیہب{integrand}\فرہنگ{integrand} اور \عددی{x} کو \اصطلاح{تکمل کا متغیر}\فرہنگ{تکمل!متغیر}\حاشیہب{variable of integration}\فرہنگ{integration!variable} کہتے ہیں۔
\انتہا{تعریف}
%======================

مسئلہ اوسط قیمت (مسئلہ \حوالہ{مسئلہ_استعمال_اوسط_قیمت}) کے  دوسرے ضمنی نتیجہ کے تحت تفاعل \عددی{f} کے  حاصل کردہ الٹ تفرق \عددی{F} اور اس کے  کسی دوسرے الٹ تفرق  میں صرف مستقل کا فرق پایا جائے گا۔ اس حقیقت کو تکملی علامتیت میں ظاہر کرتے ہیں:
\begin{align}\label{مساوات_تکمل_غیر_قطعی_الف}
\int f(x)\dif x=F(x)+C
\end{align}
مستقل \عددی{C} کو \اصطلاح{تکمل کا مستقل}\فرہنگ{تکمل!کا مستقل}\حاشیہب{constant of integration}\فرہنگ{integration!constant of} یا  \اصطلاح{اختیاری مستقل}\فرہنگ{مستقل!اختیاری}\حاشیہب{arbitrary constant}\فرہنگ{constant!arbitrary} کہتے ہیں۔ ہم مساوات \حوالہ{مساوات_تکمل_غیر_قطعی_الف} کو یوں پڑھتے ہیں: "\عددی{x} کے لحاظ سے تفاعل \عددی{f} کا غیر قطعی تکمل \عددی{F(x)+C} ہے۔"  \عددی{F(x)+C} کے حصول کو \عددی{f} کے \اصطلاح{تکمل} کا حصول کہتے ہیں۔

\ابتدا{مثال}
\عددی{\int 2x\dif x} تلاش کریں۔\\
حل:
\begin{align*}
\int 2x\dif x=x^2+C
\end{align*}
\عددی{2x} کا الٹ تفرق \عددی{x^2} ہے اور \عددی{C} تکمل کا مستقل ہے۔کلیہ \عددی{x^2+C} تفاعل \عددی{2x} کے تمام تفرقات دیتا ہے۔یوں \عددی{x^2+1}، \عددی{x^2-\pi} اور \عددی{x^2+\sqrt{2}} تفاعل \عددی{2x} کے ممکنہ الٹ تفرق ہیں۔ آپ ان کا تفرق لے کر تصدیق کر سکتے ہیں۔
\انتہا{مثال}
%======================

ہم عموماً تفرق کے کلیات سے الٹ تفرقات کے کلیات اخذ کرتے ہیں۔جدول \حوالہ{جدول_تکمل_کلیات_الف} میں غیر قطعی تکملات کے سامنے موزوں تفرقی کلیات کو الٹ لکھا گیا ہے۔ 
\begin{table}
\caption{تکمل کے کلیات}
\label{جدول_تکمل_کلیات_الف}
\centering
\renewcommand{\arraystretch}{2} 
\begin{tabular}{@{}LLL@{}}
\toprule
&\text{\RL{غیر قطعی تکمل}}&\text{\RL{تفرقی کلیات کو الٹ لکھا گیا ہے}}\\ 
\midrule
1.&{\displaystyle \int x^n\dif x=\frac{x^{n+1}}{n+1}+C}, \quad n\ne -1, \,n\text{ناطق} &\frac{\dif}{\dif x}\big(\frac{x^{n+1}}{n+1}\big)=x^n\\ 
&{\displaystyle \int \dif x=\int 1\dif x=x+C} \quad \text{\RL{(خصوصی صورت)}}&\frac{\dif}{\dif x}(x)=1\\ 
2.&{\displaystyle \int\sin kx\dif x=-\frac{\cos kx}{k}+C}&\frac{\dif}{\dif x}(-\frac{\cos kx}{k})=\sin kx\\ 
3.&{\displaystyle \int\cos kx\dif x=\frac{\sin kx}{k}+C}&\frac{\dif}{\dif x}(\frac{\sin kx}{k})=\cos kx\\ 
4.&{\displaystyle\int\sec^2x\dif x=\tan x+C}&\frac{\dif}{\dif x}\tan x=\sec^2x\\ 
5.&{\displaystyle\int\csc^2x\dif x=-\cot x+C}&\frac{\dif}{\dif x}(-\cot x)=\csc^2x \\ 
6.&{\displaystyle\int\sec x\tan x\dif x=\sec x+C}&\frac{\dif}{\dif x}\sec x=\sec x \tan x\\ 
7.&{\displaystyle\int\csc x\cot x\dif x=-\csc x+C}&\frac{\dif}{\dif x}(-\csc x)=\csc x\cot x\\
\bottomrule
\end{tabular}
\end{table}

\ابتدا{مثال}
\begin{enumerate}[a.]
\item
 جدول \حوالہ{جدول_تکمل_کلیات_الف} کے کلیہ 1 میں $n=5$ لیتے ہوئے:
\begin{align*}\int x^5\dif x=\frac{x^6}{6}+C\end{align*}

\item
کلیہ 1 میں \عددی{n=-\tfrac{1}{2}} لیتے ہوئے:
\begin{align*}\int \frac{1}{\sqrt{x}}\dif x=\int x^{-\tfrac{1}{2}}\dif x=2x^{\tfrac{1}{2}}+C\end{align*}

\item
کلیہ 2 میں \عددی{k=2} لیتے ہوئے:
\begin{align*}\int\sin 2x\dif x=-\frac{\cos 2x}{2}+C\end{align*}

\item
کلیہ 3 میں \عددی{k=\tfrac{1}{2}} لیتے ہوئے:
\begin{align*}\int\cos \frac{x}{2}\dif x=\int\frac{1}{2}x\dif x=\frac{\sin \tfrac{1}{2}x}{\tfrac{1}{2}}+C=2\sin\frac{x}{2}+C\end{align*}
\end{enumerate}
\انتہا{مثال}
%================

بعض اوقات کلیہ تکمل کا حصول مشکل ثابت ہوتا ہے البتہ  اخذ کردہ کلیہ کو پرکھنا مشکل نہیں ہے۔ کلیہ کا تفرق متکمل ہو گا۔

\ابتدا{مثال}
درج ذیل کی بنا
\begin{align*}
\frac{\dif}{\dif x}(x\sin x+\cos x+C)=x\cos x+\sin x-\sin x+0=x\cos x
\end{align*}
درج ذیل ہو گا۔
\begin{align*}
\int x\cos x\dif x=x\sin x+\cos x+C
\end{align*}
\انتہا{مثال}
%================

اس مثال میں تکمل کا کلیہ اخذ کرنا جلد سکھایا جائے گا۔

\جزوحصہء{الٹ تفرقات کے قواعد}
ہم الٹ تفرقات کے بارے میں درج ذیل جانتے ہیں۔
\begin{enumerate}[a.]
\item
ایک تفاعل اس صورت مستقل مضرب \عددی{kf} کا الٹ تفرق ہو گا جب یہ \عددی{f} کے الٹ تفرق ضرب \عددی{k} کے برابر ہو۔
\item
بالخصوص ایک تفاعل اس صورت \عددی{-f} کا الٹ تفرق ہو گا جب یہ \عددی{f} کے الٹ تفرق کا نفی ہو۔ 
\item
ایک تفاعل اس صورت مجموعہ یا فرق \عددی{f\mp g} کا الٹ تفرق ہو گا جب یہ \عددی{f} کے الٹ تفرق اور \عددی{g} کے الٹ تفرق کا مجموعہ یا فرق ہو۔
\end{enumerate}
ان حقائق کو تکملی علامتیت میں لکھنے سے غیر قطعی تکمل کے معیاری ریاضیاتی قواعد حاصل ہوتے ہیں (جدول \حوالہ{جدول_تکمل_غیر_قطعی_قواعد})۔ 
\begin{table}
\caption{غیر قطعی تکمل کے قواعد}
\label{جدول_تکمل_غیر_قطعی_قواعد}
\renewcommand{\arraystretch}{1.5} 
\centering
\begin{tabular}{@{}rrl@{}}
\toprule
1.&مستقل مضرب قاعدہ:& \عددی{{\displaystyle \int kf(x)\dif x=k\int f(x)\dif x}}\\
& (\عددی{k} کی قیمت \عددی{x} کے ساتھ تبدیل نہیں ہوتی)&\\
2.& منفی کے لئے قاعدہ:&\عددی{{\displaystyle \int -f(x)\dif x=-\int f(x)\dif x}}\\
& (قاعدہ 1 میں \عددی{k=-1} لیا گیا ہے۔)&\\
3.& مجموعہ اور فرق کا قاعدہ: & \عددی{{\displaystyle \int [f(x)\mp g(x)]\dif x=\int f(x)\dif x+\int g(x)\dif x}}\\
\bottomrule
\end{tabular}
\end{table}

\ابتدا{مثال}\شناخت{مثال_تکمل_مختلف_اشکال}\ترچھا{تکمل  کا مستقل}\\
\begin{align*}
\int 5\sec x\tan x\dif x&=5\int \sec x\tan x\dif x&&\text{\RL{جدول \حوالہ{جدول_تکمل_غیر_قطعی_قواعد}، قاعدہ 1}}\\
&=5(\sec x+C)&&\text{\RL{جدول \حوالہ{جدول_تکمل_کلیات_الف}، کلیہ 6}}\\
&=5\sec x+5C&&\text{\RL{غیر قطعی الٹ تفرق کی پہلی صورت}}\\
&=5\sec x+C'&&\text{\RL{مستقل $5C$ کو مستقل $C'$ لکھا گیا ہے}}\\
&=5\sec x+C&&\text{\RL{$C'$ ایک مستقل ہے جس کو ہم اب $C$ سے ظاہر کرتے ہیں}}
\end{align*}
\انتہا{مثال}
%=========================

اس مثال کے آخری قدم پر مستقل \عددی{C'} کو بغیر علامت (') لکھا گیا ہے۔

مثال \حوالہ{مثال_تکمل_مختلف_اشکال} میں حاصل چاروں جوابات صحیح ہیں البتہ آخری لکیر پر غیر قطعی الٹ تفرق کی سادہ ترین اور پسندیدہ صورت لکھی گئی ہے  لہٰذا عموماً درج ذیل لکھا جاتا ہے۔
\begin{align*}
\int 5\sec x\tan x\dif x=5\sec x+C
\end{align*}

جیسا مجموعہ اور فرق کے تفرق کا قاعدہ ہمیں اجزاء کو علیحدہ علیحدہ تفرق کی اجازت دیتا ہے، اسی طرح مجموعہ اور فرق کا تکملی قاعدہ ہمیں اجزاء کا علیحدہ علیحدہ تکمل لینے کی اجازت دیتا ہے۔ ایسا کرتے ہوئے ہم انفرادی مستقل تکمل کا مجموعہ یا فرق کو ایک مستقل سے ظاہر کرتے ہیں۔

\ابتدا{مثال}\ترچھا{جزو در جزو تکمل۔}\\
درج ذیل حاصل کریں۔
\begin{align*}
\int(x^2-2x+5)\dif x
\end{align*}
اگر ہم دیکھ کر بتلا سکیں کہ \عددی{x^2-2x+5} کا الٹ تفرق \عددی{\tfrac{x^3}{3}-x^2+5x} ہے تب ہم درج ذیل لکھ سکتے ہیں۔ 
\begin{align*}
\int(x^2-2x+5)\dif x=\underbrace{\frac{x^3}{3}-x^2+5x}_{\text{\RL{الٹ تفرق}}}+\underbrace{C}_{\text{\RL{اختیاری مستقل}}}
\end{align*}
اگر ہم الٹ تفرق پہچان نہ سکیں تب ہم مجموعہ اور فرق کے قاعدہ سے جزو در جزو تکمل لے کر درج ذیل لکھ سکتے ہیں۔
\begin{align*}
\int(x^2-2x+5)\dif x&=\int x^2\dif x-\int 2x\dif x+\int 5\dif x\\
&=\frac{x^3}{3}+C_1-x^2+C_2+5x+C_3
\end{align*}
اس کلیہ میں تین مستقلوں کا مجموعہ از خود ایک مستقل ہو گا جس کو \عددی{C} لکھا جا سکتا ہے یعنی \عددی{C_1+C_2+C_3=C}  جس سے کلیہ کی درج ذیل سادہ صورت حاصل ہوتی ہے۔
 \begin{align*}
\frac{x^3}{3}-x^2+5x+C
\end{align*}
جزو در جزو تکمل لیتے ہوئے ہم علیحدہ علیحدہ مستقل لکھ کر آخر میں انہیں جمع کر کے \عددی{C} لکھنے کی بجائے پہلے قدم پر ہی صرف ایک مستقل \عددی{C} لکھتے ہیں یعنی:
 \begin{align*}
\int(x^2-2x+5)\dif x&=\int x^2\dif x-\int 2x\dif x+\int 5\dif x\\
&=\frac{x^3}{3}-x^2+5x+C
\end{align*}
\انتہا{مثال}
%================
\جزوحصہء{\عددی{\sin^2x} اور \عددی{\cos^2x} کے تکملات}
بعض اوقات جن تکملات کا حصول ہم نہیں جانتے کو تکونیاتی تماثل کی مدد سے ان تکملات میں تبدیل کرنا ممکن ہوتا ہے جن کا حصول ہم جانتے ہیں۔\عددی{\sin^2x} اور \عددی{\cos^2x} کے تکمل عموماً استعمال میں درپیش آتے ہیں۔ آئیں تماثل کی مدد سے انہیں حل کرتے ہیں۔

\ابتدا{مثال}
\begin{enumerate}[a.]
\item
\begin{align*}
\int\sin^2x\dif x&=\int\frac{1-\cos 2x}{2}\dif x&&\sin^2x=\frac{1-\cos 2x}{2}\\
&=\frac{1}{2}\int(1-\cos 2x)\dif x\\
&=\frac{1}{2}\int \dif x-\frac{1}{2}\int \cos 2x\dif x\\
&=\frac{1}{2}x-\frac{1}{2}\frac{\sin 2x}{2}+C\\
&=\frac{x}{2}-\frac{\sin 2x}{4}+C
\end{align*}
\item
\begin{align*}
\int\cos^2x\dif x&=\int\frac{1+\cos 2x}{2}\dif x&&\cos^2x=\frac{1+\cos 2x}{2}\\
&=\frac{x}{2}+\frac{\sin 2x}{4}+C
\end{align*}
\end{enumerate}
\انتہا{مثال} 
%=====================

\حصہء{سوالات}
\موٹا{الٹ تفرق کا حصول}\\
سوال \حوالہ{سوال_تکمل_اور_تصدیق_الف} تا سوال \حوالہ{سوال_تکمل_اور_تصدیق_ب} میں دیے  ہر تفاعل کا الٹ تفرق زبانی (بغیر کسی جدول کی مدد کے) لکھیں۔ جواب کی تصدیق کی خطر جواب کا تفرق لیں۔

\ابتدا{سوال}\شناخت{سوال_تکمل_اور_تصدیق_الف}
(ا) \عددی{2x}، (ب) \عددی{x^2}، (ج) \عددی{x^2-2x+1}\\
جواب:\quad
(ا) \عددی{x^2}، (ب) \عددی{\tfrac{x^3}{3}}، (ج) \عددی{\tfrac{x^3}{3}-x^2+x}
\انتہا{سوال}
%======================
\ابتدا{سوال}
(ا) \عددی{6x}، (ب) \عددی{x^7}، (ج) \عددی{x^7-6x+8}
\انتہا{سوال}
%======================
\ابتدا{سوال}
(ا) \عددی{-3x^{-4}}، (ب) \عددی{x^{-4}}، (ج) \عددی{x^{-4}+2x+3}\\
جواب:\quad
(ا) \عددی{x^{-3}}، (ب) \عددی{-\tfrac{1}{3}x^{-3}}، (ج) \عددی{-\tfrac{1}{3}x^{-3}+x^2+3x}
\انتہا{سوال}
%======================
\ابتدا{سوال}
(ا) \عددی{2x^{-3}}، (ب) \عددی{\tfrac{x^{-3}}{2}+x^2}، (ج) \عددی{-x^{-3}+x-1}
\انتہا{سوال}
%======================
\ابتدا{سوال}
(ا) \عددی{\tfrac{1}{x^2}}، (ب) \عددی{\tfrac{5}{x^2}}، (ج) \عددی{2-\tfrac{5}{x^2}}\\
جواب:\quad
(ا) \عددی{-\tfrac{1}{x}}، (ب) \عددی{-\tfrac{5}{x}}، (ج) \عددی{2x+\tfrac{5}{x}}
\انتہا{سوال}
%======================
\ابتدا{سوال}
(ا) \عددی{-\tfrac{2}{x^3}}، (ب) \عددی{\tfrac{1}{2x^3}}، (ج) \عددی{x^3-\tfrac{1}{x^3}}
\انتہا{سوال}
%======================
\ابتدا{سوال}
(ا) \عددی{\tfrac{3}{2}\sqrt{x}}، (ب) \عددی{\tfrac{1}{2\sqrt{x}}}، (ج) \عددی{\sqrt{x}+\tfrac{1}{\sqrt{x}}}\\
جواب:\quad
(ا) \عددی{\sqrt{x^3}}، (ب) \عددی{\sqrt{x}}، (ج) \عددی{\tfrac{2\sqrt{x^3}}{3}+2\sqrt{x}}
\انتہا{سوال}
%======================
\ابتدا{سوال}
(ا) \عددی{\tfrac{4}{3}\sqrt[3]{x}}، (ب) \عددی{\tfrac{1}{3\sqrt[3]{x}}}، (ج) \عددی{\sqrt[3]{x}+\tfrac{1}{\sqrt[3]{x}}}
\انتہا{سوال}
%======================
\ابتدا{سوال}
(ا) \عددی{\tfrac{2}{3}x^{-\tfrac{1}{3}}}، (ب) \عددی{\tfrac{1}{3}x^{-\tfrac{2}{3}}}،
 (ج) \عددی{-\tfrac{1}{3}x^{-\tfrac{4}{3}}}\\
جواب:\quad
(ا) \عددی{x^{2/3}}، (ب) \عددی{x^{1/3}}، (ج) \عددی{x^{-1/3}}
\انتہا{سوال}
%======================
\ابتدا{سوال}
(ا) \عددی{\tfrac{1}{2}x^{-\tfrac{1}{2}}}، (ب) \عددی{-\tfrac{1}{2}x^{-\tfrac{3}{2}}}، (ج) \عددی{-\tfrac{3}{2}x^{-\tfrac{5}{2}}}
\انتہا{سوال}
%======================
\ابتدا{سوال}
(ا) \عددی{-\pi\sin \pi x}، (ب) \عددی{3\sin x}، (ج) \عددی{\sin \pi x-3\sin 3x}\\
جواب:\quad
(ا) \عددی{\cos (\pi x)}، (ب) \عددی{-3\cos x}، (ج) \عددی{-\tfrac{1}{\pi}\cos (\pi x)+\cos (3x)}
\انتہا{سوال}
%======================
\ابتدا{سوال}
(ا) \عددی{\pi \cos \pi x}، (ب) \عددی{\tfrac{\pi}{2}\cos \tfrac{\pi x}{2}}، (ج) \عددی{\cos\tfrac{\pi x}{2}+\pi\cos x}
\انتہا{سوال}
%======================
\ابتدا{سوال}
(ا) \عددی{\sec^2x}، (ب) \عددی{\tfrac{2}{3}\sec^2\tfrac{x}{3}}، (ج) \عددی{-\sec^2\tfrac{3x}{2}}\\
جواب:\quad
(ا) \عددی{\tan x}، (ب) \عددی{2\tan(\tfrac{x}{3})}، (ج) \عددی{-\tfrac{2}{3}\tan(\tfrac{3x}{2})}
\انتہا{سوال}
%======================
\ابتدا{سوال}
(ا) \عددی{\csc^2x}، (ب) \عددی{-\tfrac{3}{2}\csc^2\tfrac{3x}{2}}، (ج) \عددی{1-8\csc^22x}
\انتہا{سوال}
%======================
\ابتدا{سوال}
(ا) \عددی{\csc x\cot x}، (ب) \عددی{-\csc 5x\cot 5x}، (ج) \عددی{-\pi\csc\tfrac{\pi x}{2}\cot \tfrac{\pi x}{2}}\\
جواب:\quad
(ا) \عددی{-\csc x}، (ب) \عددی{\tfrac{1}{5}\csc (5x)}، (ج) \عددی{2\csc(\tfrac{\pi x}{2})}
\انتہا{سوال}
%======================
\ابتدا{سوال}
(ا) \عددی{\sec x\tan x}، (ب) \عددی{4\sec 3x\tan 3x}، (ج) \عددی{\sec\tfrac{\pi x}{2}\tan\tfrac{\pi x}{2}}
\انتہا{سوال}
%======================
\ابتدا{سوال}
\عددی{(\sin x-\cos x)^2}\\
جواب:\quad
\عددی{x+\tfrac{\cos (2x)}{2}}
\انتہا{سوال}
%======================
\ابتدا{سوال}\شناخت{سوال_تکمل_اور_تصدیق_ب}
\عددی{(1+2\cos x)^2}
\انتہا{سوال}
%======================
\موٹا{تکمل کا حصول}\\
سوال \حوالہ{سوال_تکمل_حاصل_تفرق_تصدیق_الف} تا سوال \حوالہ{سوال_تکمل_حاصل_تفرق_تصدیق_ب} میں تکمل حاصل کریں۔ تکمل کا تفرق لے کر جواب کی تصدیق کریں۔

\ابتدا{سوال}\شناخت{سوال_تکمل_حاصل_تفرق_تصدیق_الف}
$\int (x+1)\dif x$\\
جواب:\quad
$\tfrac{x^2}{2}+x+C$
\انتہا{سوال}
%===========================
\ابتدا{سوال}
$\int(5-6x)\dif x$
\انتہا{سوال}
%============================
\ابتدا{سوال}
$\int(3t^2+\tfrac{t}{2})\dif t$\\
جواب:\quad
$t^3+\tfrac{t^2}{4}+C$
\انتہا{سوال}
%============================
\ابتدا{سوال}
$(\tfrac{t^2}{2}+4t^3)\dif t$
\انتہا{سوال}
%============================
\ابتدا{سوال}
$(2x^3-5x+7)\dif x$\\
جواب:\quad
$\tfrac{x^4}{2}-\tfrac{5x^2}{2}+7x+C$
\انتہا{سوال}
%============================
\ابتدا{سوال}
$\int(1-x^2-3x^5)\dif x$
\انتہا{سوال}
%============================
\ابتدا{سوال}
$\int(\tfrac{1}{x^2}-x^2-\tfrac{1}{3})\dif x$\\
جواب:\quad
$-\tfrac{1}{x}-\tfrac{x^3}{3}-\tfrac{x}{3}+C$
\انتہا{سوال}
%============================
\ابتدا{سوال}
$\int(\tfrac{1}{5}-\tfrac{2}{x^3}+2x)\dif x$
\انتہا{سوال}
%============================
\ابتدا{سوال}
$\int x^{-\tfrac{1}{3}}\dif x$\\
جواب:\quad
$\tfrac{3}{2}x^{2/3}+C$
\انتہا{سوال}
%============================
\ابتدا{سوال}
$\int x^{-\tfrac{5}{4}}\dif x$
\انتہا{سوال}
%============================
\ابتدا{سوال}
$\int(\sqrt{x}+\sqrt[3]{x})\dif x$\\
جواب:\quad
$\tfrac{2}{3}x^{3/2}+\tfrac{3}{4}x^{4/3}+C$
\انتہا{سوال}
%============================
\ابتدا{سوال}
$\int(\tfrac{\sqrt{x}}{2}+\tfrac{2}{\sqrt{x}})\dif x$
\انتہا{سوال}
%============================
\ابتدا{سوال}
$\int(8y-\tfrac{2}{y^{1/4}})\dif y$\\
جواب:\quad
$4y^2-\tfrac{8}{3}y^{3/4}+C$
\انتہا{سوال}
%============================
\ابتدا{سوال}
$\int(\tfrac{1}{7}-\tfrac{1}{y^{5/4}})\dif y$
\انتہا{سوال}
%============================
\ابتدا{سوال}
$\int 2x(1-x^{-3})\dif x$\\
جواب:\quad
$x^2+\tfrac{2}{x}+C$
\انتہا{سوال}
%============================
\ابتدا{سوال}
$\int x^{-3}(x+1)\dif x$
\انتہا{سوال}
%============================
\ابتدا{سوال}
$\int\tfrac{t\sqrt{t}+\sqrt{t}}{t^2}\dif t$\\
جواب:\quad
$2\sqrt{t}-\tfrac{2}{\sqrt{t}}+C$
\انتہا{سوال}
%============================
\ابتدا{سوال}
$\int\tfrac{4+\sqrt{t}}{t^3}\dif t$
\انتہا{سوال}
%============================
\ابتدا{سوال}
$\int(-2\cos t)\dif t$\\
جواب:\quad
$-2\sin t+C$
\انتہا{سوال}
%============================
\ابتدا{سوال}
$\int (-5\sin t)\dif t$
\انتہا{سوال}
%============================
\ابتدا{سوال}
$7\sin\tfrac{\theta}{3}\dif \theta$\\
جواب:\quad
$-21\cos\tfrac{\theta}{3}+C$
\انتہا{سوال}
%============================
\ابتدا{سوال}
$\int3\cos 5\theta\dif\theta$
\انتہا{سوال}
%============================
\ابتدا{سوال}
$\int(-3\csc^2x)\dif x$\\
جواب:\quad
$3\cot x+C$
\انتہا{سوال}
%============================
\ابتدا{سوال}
$\int(-\tfrac{\sec^2x}{3})\dif x$
\انتہا{سوال}
%============================
\ابتدا{سوال}
$\int\tfrac{\csc \theta\cot\theta}{2}\dif\theta$\\
جواب:\quad
$-\tfrac{1}{2}\csc\theta+C$
\انتہا{سوال}
%============================
\ابتدا{سوال}
$\tfrac{2}{5}\sec\theta\tan\theta\dif\theta$
\انتہا{سوال}
%============================
\ابتدا{سوال}
$\int(4\sec x\tan x-2\sec^2x)\dif x$\\
جواب:\quad
$4\sec x-2\tan x+C$
\انتہا{سوال}
%============================
\ابتدا{سوال}
$\int\tfrac{1}{2}(\csc^2x-\csc x\cot x)\dif x$
\انتہا{سوال}
%============================
\ابتدا{سوال}
$\int(\sin 2x-\csc^2 x)\dif x$\\
جواب:\quad
$-\tfrac{1}{2}\cos 2x+\cot x+C$
\انتہا{سوال}
%============================
\ابتدا{سوال}
$\int(2\cos 2x-3\sin 3x)\dif x$
\انتہا{سوال}
%============================
\ابتدا{سوال}
$\int 4\sin^2y\dif y$\\
جواب:\quad
$2y-\sin 2y+C$
\انتہا{سوال}
%============================
\ابتدا{سوال}
$\int\tfrac{\cos^2y}{7}\dif y$
\انتہا{سوال}
%============================
\ابتدا{سوال}
$\int\tfrac{1+\cos 4t}{2}\dif t$\\
جواب:\quad
$\tfrac{t}{2}+\tfrac{\sin 4t}{8}+C$
\انتہا{سوال}
%============================
\ابتدا{سوال}
$\int\tfrac{1-\cos 6t}{2}\dif t$
\انتہا{سوال}
%============================
\ابتدا{سوال}
$\int(1+\tan^2\theta)\dif\theta$\quad
 اشارہ۔ \عددی{1+\tan^2\theta=\sec^2\theta}\\
جواب:\quad
$\tan \theta+C$
\انتہا{سوال}
%============================
\ابتدا{سوال}
$\int (2+\tan^2\theta)\dif\theta$
\انتہا{سوال}
%============================
\ابتدا{سوال}
$\int\cot^2 x\dif x$\\
جواب:\quad
$-\cot x-x+C$
\انتہا{سوال}
%============================
\ابتدا{سوال}
$\int(1-\cot^2x)\dif x$
\انتہا{سوال}
%============================
\ابتدا{سوال}
$\int\cos\theta(\tan\theta+\sec\theta)\dif \theta$\\
جواب:\quad
$-\cos \theta+\theta+C$
\انتہا{سوال}
%============================
\ابتدا{سوال}\شناخت{سوال_تکمل_حاصل_تفرق_تصدیق_ب}
$\int\tfrac{\csc\theta}{\csc\theta-\sin\theta}\dif\theta$
\انتہا{سوال}
%============================
\موٹا{تکملی کلیہ کی تصدیق}\\
سوال \حوالہ{سوال_تکمل_کلیات_تصدیق_الف} تا سوال \حوالہ{سوال_تکمل_کلیات_تصدیق_ب} میں دیے تکملی کلیات کی تصدیق بذریعہ تفرق کریں۔ ان کلیات کا حصول جلد دکھایا جائے گا۔

\ابتدا{سوال}\شناخت{سوال_تکمل_کلیات_تصدیق_الف}
$\int(7x-2)^3\dif x=\tfrac{(7x-2)^4}{28}+C$
\انتہا{سوال}
%========================
\ابتدا{سوال}
$\int (3x+5)^{-2}\dif x=-\tfrac{(3x+5)^{-1}}{3}+C$
\انتہا{سوال}
%=======================
\ابتدا{سوال}
$\int\sec^2(5x-1)\dif x=\tfrac{1}{5}\tan(5x-1)+C$
\انتہا{سوال}
%=======================
\ابتدا{سوال}
$\int \csc^2(\tfrac{x-1}{3})\dif x=-3\cot(\tfrac{x-1}{3})+C$
\انتہا{سوال}
%=======================
\ابتدا{سوال}
$\int \tfrac{1}{(x+1)^2}\dif x=-\tfrac{1}{x+1}+C$
\انتہا{سوال}
%=======================
\ابتدا{سوال}\شناخت{سوال_تکمل_کلیات_تصدیق_ب}
$\int \tfrac{1}{(x+1)^2}\dif x=\tfrac{x}{x+1}+C$
\انتہا{سوال}
%=======================
\ابتدا{سوال}
درج ذیل کلیات میں سے درست اور غلط کی نشاندہی کریں۔ اپنے جوابات کی وجہ پیش کریں۔
\begin{align*}
\int x\sin x\dif x&=\frac{x^2}{2}\sin x+C&& \text{ا۔}\\
\int x\sin x\dif x&=-x\cos x+C&&\text{ب۔}\\
\int x\sin x\dif x&=-x\cos x+\sin x+C &&\text{ج۔}
\end{align*}
جواب:\quad
(ا) غلط، (ب) غلط، (ج) درست
\انتہا{سوال}
%=======================
\ابتدا{سوال}
درج ذیل کلیات میں سے درست اور غلط کی نشاندہی کریں۔ اپنے جوابات کی وجہ پیش کریں۔
\begin{align*}
\int\tan\theta\sec^2\theta\dif\theta&=\frac{\sec^3\theta}{3}+C&&\text{ا۔}\\
\int\tan\theta\sec^2\theta\dif\theta&=\frac{1}{2}\tan^2\theta+C&&\text{ب۔}\\
\int\tan\theta\sec^2\theta\dif\theta&=\frac{1}{2}\sec^2\theta&&\text{ج۔}
\end{align*}
\انتہا{سوال}
%=====================
\ابتدا{سوال}
درج ذیل کلیات میں سے درست اور غلط کی نشاندہی کریں۔ اپنے جوابات کی وجہ پیش کریں۔
\begin{align*}
\int(2x+1)^2\dif x&=\frac{(2x+1)^3}{3}+C&&\text{ا۔}\\
\int 3(2x+1)^2\dif x&=(2x+1)^3+C&&\text{ب۔}\\
\int 6(2x+1)^2\dif x&=(2x+1)^3+C&&\text{ج۔}
\end{align*}
جواب:\quad
(ا) غلط، (ب) غلط، (ج) درست
\انتہا{سوال}
%==========================
\ابتدا{سوال}
درج ذیل کلیات میں سے درست اور غلط کی نشاندہی کریں۔ اپنے جوابات کی وجہ پیش کریں۔
\begin{align*}
\int\sqrt{2x+1}\dif x&=\sqrt{x^2+x+C}&&\text{ا۔}\\
\int\sqrt{2x+1}\dif x&=\sqrt{x^2+x}+C&&\text{ب۔}\\
\int\sqrt{2x+1}\dif x&=\frac{1}{3}(\sqrt{2x+1})^3+C&&\text{ج۔}
\end{align*}
\انتہا{سوال}
%===================
\موٹا{نظریہ اور مثالیں}\\
\ابتدا{سوال}\شناخت{سوال_تکمل_مجموعہ+فرق}
درج ذیل فرض کرتے ہوئے
\begin{align*}
f(x)=\frac{\dif}{\dif x}(1-\sqrt{x}),\quad g(x)=\frac{\dif}{\dif x}(x+2)
\end{align*}
درج ذیل تلاش کریں۔
\begin{gather*}
\begin{aligned}
&\int f(x)\dif x\quad \text{ا۔}\\
&\int g(x)\dif x\quad \text{ب۔}\\
&\int [-f(x)]\dif x\quad \text{ج۔}\\
&\int[-g(x)]\dif x\quad \text{د۔}
\end{aligned}\quad
\begin{aligned}
&\int[f(x)+g(x)]\dif x\quad \text{ہ۔}\\
&\int[f(x)-g(x)]\dif x\quad \text{و۔}\\
&\int[x+f(x)]\dif x\quad \text{ز۔}\\
&\int[g(x)-4]\dif x\quad \text{ح۔}
\end{aligned}
\end{gather*}
جواب:\quad
(ا) \عددی{-\sqrt{x}+C}، (ب) \عددی{x+C}، (ج) \عددی{\sqrt{x}+C}، (د) \عددی{-x+C}،\\
 (ہ) \عددی{x-\sqrt{x}+C}، (و) \عددی{-x-\sqrt{x}+C}، (ز) \عددی{\tfrac{x^2}{2}-\sqrt{x}+C}، (ح) \عددی{-3x+C}
\انتہا{سوال}
%==================
\ابتدا{سوال}
درج ذیل فرض کرتے ہوئے سوال \حوالہ{سوال_تکمل_مجموعہ+فرق} دوبارہ حل کریں۔
\begin{align*}
f(x)=\frac{\dif}{\dif x}e^x,\quad g(x)=\frac{\dif}{\dif x}(x\sin x)
\end{align*}
\انتہا{سوال}
%=====================
\حصہ{تفرقی مساوات، ابتدائی قیمت مسئلے، اور ریاضیاتی نمونہ کشی}
