\حصہ{سطح طواف کا رقبہ}
بچپن میں آپ نے دوستوں کے ساتھ مل کر رسی گھماتے ہوئے رسی کے اوپر سے چھلانگیں ضرور لگائی ہوں گی۔ یہ رسی فضا میں پھیر  کر ایک سطح بناتی ہے جس کو \اصطلاح{سطح طواف}\فرہنگ{طواف!سطح}\فرہنگ{سطح!طواف}\حاشیہب{surface of revolution}\فرہنگ{revolution!surface} کہتے ہیں۔ سطح طواف کا رقبہ رسی کی لمبائی اور رسی کے ہر حصے کی جھول پر منحصر ہو گا۔ اس حصہ میں سطح طواف کا رقبہ اور سطح کو پیدا کرنے والی منحنی کی لمبائی اور جھول کے تعلق پر غور کیا جائے گا۔زیادہ پیچیدہ سطحوں پر بعد کے باب میں غور کیا جائے گا۔

\جزوحصہء{بنیادی کلیہ}
فرض کریں ہم غیر منفی تفاعل \عددی{y=f(x),\, a\le x\le b} کو \عددی{x}محور کے گرد گھما کر پیدا سطح طواف کا سطحی رقبہ جاننا چاہتے ہیں۔ ہم \عددی{[a,b]} کی خانہ بندی کر کے نقاط خانہ بندی استعمال کرتے ہوئے ترسیم کو چھوٹے حصوں میں تقسیم کرتے ہیں۔ شکل میں نمائندہ حصہ \عددی{PQ}  اور اس کی پیدا کردہ پٹی دکھائی گئی ہے۔

قوس \عددی{PQ} محور \عددی{x} کے گرد گھومتے ہوئے مخروط سطح پیدا کرتی ہے۔محور \عددی{x} اس مخروط سطح کا محور ہو گا۔ مخروط کے ایسے حصے کو \اصطلاح{مخروط مقطوع}\فرہنگ{مخروط مقطوع}\حاشیہب{frustum}\فرہنگ{frustum} کہتے ہیں۔ مخروط مقطوع کا سطحی رقبہ، {PQ} کی پیدا کردہ پٹی کے رقبہ کا تخمین ہو گا۔

مخروط مقطوع کا سطحی رقبہ \عددی{2\pi} ضرب دونوں سروں کے رداس کا اوسط ضرب ترچھا قد کے برابر ہو گا۔
\begin{align*}
\text{\RL{مخروط مقطوع کا سطحی رقبہ}}=2\pi\cdot \frac{r_1+r_2}{2}\cdot L=\pi(r_1+r_2)L
\end{align*} 
قطع \عددی{PQ} کے پیدا کردہ مخروط مقطوع کے لئے اس سے درج ذیل حاصل ہوتا ہے۔
\begin{align*}
\text{\RL{مخروط مقطوع کا سطحی رقبہ}}=
\pi(f(x_{k-1})+f(x_k))\sqrt{(\Delta x_k)^2+(\Delta y_k)^2}
\end{align*}
پوری سطح طواف کا رقبہ تخمیناً ایسے تمام چھوٹے قطعات کی پیدا کردہ مخروط مقطوع کے سطحی رقبوں کا مجموعہ کے ہو گا۔
\begin{align}\label{مساوات_تکمل_استعمال_سطحی_رقبہ_الف}
\sum_{k=1}^n \pi(f(x_{k-1})+f(x_k))\sqrt{(\Delta x_k)^2+(\Delta y_k)^2}
\end{align}
ہم توقع کرتے ہیں کہ \عددی{[a,b]} کی زیادہ باریک خانہ بندی سے تخمین بہتر ہو گی۔ ہم دکھانا چاہتے ہیں کہ خانہ بندی کا معیار صفر تک پہنچنے سے مساوات \حوالہ{مساوات_تکمل_استعمال_سطحی_رقبہ_الف} میں دیا گیا مجموعہ قابل حل حد دیگا۔ 

یہ دکھانے کی خاطر ہم مساوات \حوالہ{مساوات_تکمل_استعمال_سطحی_رقبہ_الف} کو وقفہ \عددی{[a,b]} پر کسی تفاعل  کا ریمان مجموعہ لکھتے ہیں۔لمبائی قوس کے حصول کی طرح ہم تفرقات کے مسئلہ اوسط قیمت کی طرف دیکھتے ہیں۔

اگر \عددی{f} ہموار ہو تب مسئلہ اوسط قیمت کے تحت \عددی{P} اور \عددی{Q} کے بیچ ایسا نقطہ \عددی{(c_k,f(c_k))} ضرور پایا جائے گا جہاں مماس قطع \عددی{PQ} کے متوازی ہو گا۔اس نقطہ پر درج ذیل ہو گا۔
\begin{align*}
f'(c_k)&=\frac{\Delta y_k}{\Delta x_k}\\
\Delta y_k&=f'(c_k)\Delta x_k
\end{align*} 
مساوات \حوالہ{مساوات_تکمل_استعمال_سطحی_رقبہ_الف} میں درج بالا \عددی{\Delta y_k} پر کرتے ہیں۔
\begin{multline}\label{مساوات_تکمل_استعمال_سطحی_رقبہ_ب}
\sum_{k=1}^n \pi(f(x_{k-1})+f(x_k))\sqrt{(\Delta x_k)^2+(\Delta y_k)^2}\\
=\sum_{k=1}^n\pi(f(x_{k-1})+f(x_k))\sqrt{1+(f'(c_k))^2}\Delta x_k
\end{multline}
اب یہاں ایک بری خبر اور ایک اچھی خبر ہے۔

بری خبر یہ ہے کہ مساوات \حوالہ{مساوات_تکمل_استعمال_سطحی_رقبہ_ب} میں  \عددی{x_{k-1}}، \عددی{x_k} اور \عددی{c_k} ایک دوسرے سے مختلف ہیں اور انہیں ایک دوسرے جیسا کسی صورت نہیں بنایا جا سکتا ہے لہٰذا مساوات \حوالہ{مساوات_تکمل_استعمال_سطحی_رقبہ_ب} میں دیا گیا مجموعہ  ریمان مجموعہ نہیں ہے۔ اچھی خبر یہ ہے کہ اس سے کوئی فرق نہیں پڑتا ہے۔ اعلٰی احصاء کا مسئلہ بلس کہتا ہے کہ وقفہ \عددی{[a,b]} کی خانہ بندی کا معیار صفر تک پہچانے سے مساوات \حوالہ{مساوات_تکمل_استعمال_سطحی_رقبہ_ب} میں دیا گیا مجموعہ درج ذیل کو مرکوز ہو گا
\begin{align*}
\int_a^b2\pi f(x)\sqrt{1+(f'(x))^2}\dif x
\end{align*} 
جو ہم چاہتے ہیں۔یوں  \عددی{a} تا \عددی{b} تفاعل \عددی{f} کی ترسیم کو \عددی{x} محور کے گرد گھمانے سے حاصل سطح طواف کے رقبہ کی تعریف ہم اسی تکمل کو لیتے ہیں۔

\ابتدا{تعریف}\موٹا{محور \عددی{x} کے گرد سطح طواف کے رقبہ کا کلیہ}\\
اگر \عددی{[a,b]} پر تفاعل \عددی{f(x)\ge 0} ہموار ہو تب تفاعل \عددی{y=f(x)} کو \عددی{x} محور کے گرد گھمانے سے حاصل سطح طواف کا رقبہ درج ذیل ہو گا۔
\begin{align}\label{مساوات_تکمل_استعمال_سطحی_رقبہ_پ}                                
S=\int_a^b2\pi y\sqrt{1+\big(\frac{\dif y}{\dif x}\big)^2}\dif x=\int_a^b2\pi f(x)\sqrt{1+(f'(x))^2}\dif x
\end{align}
\انتہا{تعریف}
%=========================

مساوات \حوالہ{مساوات_تکمل_استعمال_سطحی_رقبہ_پ} میں جذر وہی ہے جو پیداکار منحنی کی لمبائی قوس کے کلیہ میں پایا جاتا ہے۔

\ابتدا{مثال}
محور \عددی{x} کے گرد منحنی \عددی{y=2\sqrt{x},\, 1\le x\le 2} گھما کر سطح طواف پیدا کیا جاتا ہے۔اس سطح طواف کا رقبہ تلاش کریں۔

حل:\quad
ہم درج ذیل لیتے ہوئے
\begin{align*}
a&=1,\, b=2,\, y=2\sqrt{x},\, \frac{\dif y}{\dif x}=\frac{1}{\sqrt{2}}\\
\sqrt{1+\big(\frac{\dif y}{\dif x}\big)^2}&=\sqrt{1+\big(\frac{1}{\sqrt{x}}\big)^2}\\
&=\sqrt{1+\frac{1}{x}}=\sqrt{\frac{x+1}{x}}=\frac{\sqrt{x+1}}{\sqrt{x}}
\end{align*}
 مساوات \حوالہ{مساوات_تکمل_استعمال_سطحی_رقبہ_پ} استعمال کرتے ہیں۔
\begin{align*}
S&=\int_1^22\pi \cdot2\sqrt{x}\frac{\sqrt{x+1}}{\sqrt{x}}\dif x=4\pi\int_1^2\sqrt{x+1}\dif x\\
&=\left.4\pi\cdot\frac{2}{3}(x+1)^{3/2}\right]_1^2=\frac{8\pi}{3}(3\sqrt{3}-2\sqrt{2})
\end{align*}

\انتہا{مثال}
%=========================

\جزوحصہء{محور \عددی{y} کے گرد سطح طواف}
محور \عددی{y} کے گرد سطح طواف کے لئے ہم مساوات \حوالہ{مساوات_تکمل_استعمال_سطحی_رقبہ_پ} میں \عددی{x} اور \عددی{y} کی جگہیں تبدیل کرتے  ہیں۔

\موٹا{محور \عددی{y} کے گرد سطح طواف کے رقبہ کا کلیہ}\\
اگر \عددی{[c,d]} پر \عددی{x=g(y)\ge 0} ہموار ہو تب منحنی \عددی{x=g(y)} کو محور \عددی{y} کے گرد گھمانے سے حاصل سطح طواف کا رقبہ درج ذیل ہو گا۔
\begin{align}\label{مساوات_تکمل_استعمال_سطحی_رقبہ_ت}
S=\int_c^d2\pi x\sqrt{1+\big(\frac{\dif x}{\dif y}\big)^2}\dif y=\int_c^d2\pi g(y)\sqrt{1+(g'(y))^2}\dif y
\end{align}

\ابتدا{مثال}
لکیری قطع \عددی{x=1-y,\, 0\le y\le 1} کو  محور \عددی{y} کے گرد گھما کر مخروط حاصل کیا جاتا ہے۔ اس کا رقبہ پہلو تلاش کریں۔

حل:\quad
اس رقبہ کو جیومیٹری سے حاصل کیا جا سکتا ہے۔
\begin{align*}
\text{\RL{رقبہ پہلو}}=\frac{\text{\RL{قاعدے کا محیط}}}{2}\times \text{\RL{ترچھا قد}}=\pi \sqrt{2}
\end{align*}
آئیں درج ذیل لے کر  
\begin{align*}
c&=0,\, d=1,\, x=1-y,\, \frac{\dif x}{\dif y}=-1\\
\sqrt{1+\big(\frac{\dif x}{\dif y}\big)^2}&=\sqrt{1+(-1)^2}=\sqrt{2}
\end{align*}
مساوات \حوالہ{مساوات_تکمل_استعمال_سطحی_رقبہ_ت} سے اس رقبہ کا حاصل کریں۔
\begin{align*}
S&=\int_c^d2\pi x\sqrt{1+\big(\frac{\dif x}{\dif y}\big)^2}\dif y=\int_0^12\pi (1-y)\sqrt{2}\dif y\\
&=2\pi \sqrt{2}\left[y-\frac{y^2}{2}\right]_0^1=2\pi \sqrt{2}\big(1-\frac{1}{2}\big)=\pi \sqrt{2}
\end{align*}
دونوں نتائج ایک جیسے ہیں جیسا کہ ہونا چاہیے۔
\انتہا{مثال}
%========================

\جزوحصہء{مختصر تفریقی روپ}
درج ذیل مساواتوں
\begin{align*}
S=\int_a^b2\pi y\sqrt{1+\big(\frac{\dif y}{\dif x}\big)^2}\dif x\quad \text{اور}
\quad S=\int_c^d2\pi x \sqrt{1+\big(\frac{\dif x}{\dif y}\big)^2}\dif y
\end{align*}
کو عموماً تفریقی لمبائی قوس \عددی{\dif s=\sqrt{\dif x^2+\dif y^2}} کی صورت میں لکھا جاتا ہے:
\begin{align*}
S=\int_a^b2\pi y\dif s\quad \text{اور}\quad S=\int_c^d2\pi x\dif s
\end{align*}
بایاں مساوات میں \عددی{x} محور سے قطع \عددی{\dif s} تک فاصلہ \عددی{y} ہے۔ دایاں مساوات میں \عددی{y} محور سے قطع \عددی{\dif s} کا فاصلہ \عددی{x} ہے۔ان دونوں کلیوں کو
\begin{align*}
S=\int 2\pi(\text{\RL{رداس}})(\text{\RL{چوڑائی پٹی}})= \int 2\pi \rho \dif s
\end{align*}
لکھا جا سکتا ہے جہاں رکن لمبائی قوس \عددی{\dif s} تک محور طواف سے فاصلہ \عددی{\rho} ہے۔

\موٹا{مختصر تفریقی روپ}\\
\begin{align*}
S=\int 2\pi \rho \dif s
\end{align*}
کسی مخصوص مسئلے میں آپ رکن لمبائی قوس \عددی{\dif s} اور رداس \عددی{\rho} کو کسی مشترکہ متغیر کی صورت میں لکھ کر تکمل کے حدود بھی اسی متغیر کی روپ میں مہیا کریں گے۔ 

\ابتدا{مثال}
منحنی \عددی{y=x^3,\, 0\le x\le \tfrac{1}{2}} کو محور \عددی{x} کے گرد گھما کر سطح طواف پیدا کیا جاتا ہے۔ اس کا سطحی رقبہ معلوم کریں۔

حل:\quad
ہم مختصر تفریقی روپ سے شروع کرتے ہیں۔
\begin{align*}
S&=\int 2\pi \rho \dif s\\
&=\int 2\pi y\dif s\\
&=\int 2\pi y\sqrt{\dif x^2+\dif y^2}&&\dif s=\sqrt{\dif x^2+\dif y^2}
\end{align*} 
ہم نے یہاں فیصلہ کرنا ہو گا کہ آیا \عددی{\dif s} کو \عددی{\dif x} یا \عددی{\dif y} کی روپ میں لکھیں۔منحنی کی مساوات \عددی{y=x^3} سے \عددی{\dif y} کو \عددی{\dif x} کی صورت میں لکھنا زیادہ آسان ہے لہٰذا ہم درج ذیل استعمال کریں گے۔
\begin{align*}
y=x^3,\, \dif y=3x^2\dif x,\, \sqrt{\dif x^2+\dif y^2}=\sqrt{\dif x^2+(3x^2\dif x)^2}=\sqrt{1+9x^4}\dif x
\end{align*}
انہیں استعمال کرتے ہوئے تکمل کا متغیر \عددی{x} ہو گا۔
\begin{align*}
S&=\int_{x=0}^{x=1/2}2\pi y\sqrt{\dif x^2+\dif y^2}\\
&=\int_0^{1/2}2\pi x^3\sqrt{1+9x^4}\dif x\\
&=\left.2\pi(\tfrac{1}{36})(\tfrac{2}{3})(1+9x^4)^{3/2}\right]_0^{1/2}\\
&=\tfrac{\pi}{27}[(1+\tfrac{9}{16})^{3/2}-1]\\
&=\tfrac{\pi}{27}[(\tfrac{25}{16})^{3/2}-1]\\
&=\tfrac{\pi}{27}(\tfrac{125}{64}-1)\\
&=\tfrac{61\pi}{1728}
\end{align*}
\انتہا{مثال}
%=====================

\حصہء{سوالات}
\موٹا{سطحی رقبہ کے تکمل}\\
سوال \حوالہ{سوال_تکمل_استعمال_سطحی_الف} تا سوال \حوالہ{سوال_تکمل_استعمال_سطحی_ب} میں درج ذیل اقدام کریں۔
\begin{enumerate}[a.]
\item
دیے گئے منحنی کو دیے گئے محور کے گرد گھما کر سطح طواف حاصل کیا جاتا ہے۔ اس کے سطحی رقبے کا تکمل لکھیں۔
\item
منحنی کو ترسیم کر کے اس کی صورت دیکھیں۔ سطحی رقبہ کو بھی ترسیم کریں۔
\item
کمپیوٹر کی مدد سے اس تکمل کو اعدادی طریقہ سے حل کریں۔  
\end{enumerate}

\ابتدا{سوال}\شناخت{سوال_تکمل_استعمال_سطحی_الف}
$y=\tan x,\quad 0\le x \le \tfrac{\pi}{4};\quad \text{\RL{محور \عددی{x}}}$\\
جواب:\quad
(ا) \عددی{2\pi\int_0^{\pi/4}\tan x\sqrt{1+\sec^4x}\dif x}، (ج) \عددی{\approx 3.84}
\انتہا{سوال}
%====================
\ابتدا{سوال}
$y=x^2,\quad 0\le x \le 2;\quad \text{\RL{محور \عددی{x}}}$
\انتہا{سوال}
%====================
\ابتدا{سوال}
$xy=1,\quad 1\le y \le 2;\quad \text{\RL{محور \عددی{y}}}$\\
جواب:\quad
(ا) \عددی{2\pi\int_1^2\tfrac{1}{y}\sqrt{1+y^{-4}}\dif y}، (ج) \عددی{\approx 5.02}
\انتہا{سوال}
%====================
\ابتدا{سوال}
$x=\sin y,\quad 0\le y\le \pi;\quad \text{\RL{محور \عددی{y}}}$
\انتہا{سوال}
%====================
\ابتدا{سوال}
$x^{1/2}+y^{1/2}= 3,\quad \text{\RL{نقطہ $(4,1)$ سے $(1,4)$ تک}};\quad \text{\RL{محور \عددی{x}}}$\\
جواب:\quad
(ا) \عددی{2\pi\int_0^4(3-\sqrt{x})^2\sqrt{1+(1-3x^{-1/2})^2}\dif x}، (ج) \عددی{\approx 63.37}
\انتہا{سوال}
%====================
\ابتدا{سوال}
$y+2\sqrt{y}=x,\quad 1\le y \le 2;\quad \text{\RL{محور \عددی{y}}}$
\انتہا{سوال}
%====================
\ابتدا{سوال}
$x=\int_0^y\tan t\dif t,\quad 0\le y \le\tfrac{\pi}{3};\quad \text{\RL{محور \عددی{y}}}$\\
جواب:\quad
(ا) \عددی{2\pi\int_0^{\pi/3}(\int_0^y\tan t\dif t)\sec y\dif y}، (ج) \عددی{\approx 2.08}
\انتہا{سوال}
%====================
\ابتدا{سوال}\شناخت{سوال_تکمل_استعمال_سطحی_ب}
$y=\int_1^x\sqrt{t^2-1}\dif t,\quad 1\le x \le\sqrt{5};\quad \text{\RL{محور \عددی{x}}}$
\انتہا{سوال}
%====================
\موٹا{سطحی رقبہ کا حصول}\\
\ابتدا{سوال}
لکیری قطع \عددی{y=\tfrac{x}{2},\, 0\le x\le 4} کو \عددی{x} محور کے گرد گھما کر مخروط پیدا کیا جاتا ہے۔ اس کے پہلو کا رقبہ تکمل سے تلاش کریں۔ جیومیٹری کے کلیہ (پہلو کا رقبہ=\عددی{\tfrac{1}{2}}(محیط قاعدہ)(ترچھا قد)) سے اپنے جواب کی تصدیق کریں۔\\
جواب:\quad
$4\pi \sqrt{5}$
\انتہا{سوال}
%==================
\ابتدا{سوال}
لکیری قطع \عددی{y=\tfrac{x}{2},\, 0\le x\le 4} کو \عددی{y} محور کے گرد گھما کر مخروط پیدا کیا جاتا ہے۔ اس کے پہلو کا رقبہ تکمل سے تلاش کریں۔ جیومیٹری کے کلیہ سے اپنے جواب کی تصدیق کریں۔
\انتہا{سوال}
%======================
\ابتدا{سوال}
لکیری قطع \عددی{y=\tfrac{x}{2}+\tfrac{1}{2},\, 1\le x\le 3} کو \عددی{x} محور کے گرد گھما کر مخروط مقطوع پیدا کیا جاتا ہے۔ اس کے پہلو کا رقبہ تکمل سے تلاش کریں۔ جیومیٹری کے کلیہ (رقبہ مخروط مقطوع=\عددی{\pi}(\عددی{r_1+r_2})(ترچھا قد)) سے اپنے جواب کی تصدیق کریں۔\\
جواب:\quad
$3\pi \sqrt{5}$
\انتہا{سوال}
%======================
\ابتدا{سوال}
لکیری قطع \عددی{y=\tfrac{x}{2}+\tfrac{1}{2},\, 1\le x\le 3} کو \عددی{y} محور کے گرد گھما کر مخروط مقطوع پیدا کیا جاتا ہے۔ اس کے پہلو کا رقبہ تکمل سے تلاش کریں۔ جیومیٹری کے کلیہ (رقبہ مخروط مقطوع=\عددی{\pi}(\عددی{r_1+r_2})(ترچھا قد)) سے اپنے جواب کی تصدیق کریں۔
\انتہا{سوال}
%======================
سوال \حوالہ{سوال_تکمل_استعمال_ترسیم_تکمل_الف} تا سوال \حوالہ{سوال_تکمل_استعمال_ترسیم_تکمل_ب} میں منحنی کو دیے گئے محور کے گرد گھما کر سطح  طواف پیدا کیا جاتا ہے۔ اس سطح کا رقبہ معلوم کریں۔ بہتر ہو گا کہ آپ دیے گئے منحنی کو کمپیوٹر پر ترسیم کر کے منحنی کی صورت سیکھیں۔ 

\ابتدا{سوال}\شناخت{سوال_تکمل_استعمال_ترسیم_تکمل_الف}
$y=\tfrac{x^3}{9},\quad 0\le x\le 2,\quad x\,\text{محور}$\\
جواب:\quad
$\tfrac{98\pi}{81}$
\انتہا{سوال}
%=====================
\ابتدا{سوال}
$y=\sqrt{x},\quad \tfrac{3}{4}\le x\le \tfrac{15}{4},\quad x\,\text{محور}$
\انتہا{سوال}
%=====================
\ابتدا{سوال}
$y=\sqrt{2x-x^2},\quad 0.5\le x\le 1.5,\quad x\,\text{محور}$\\
جواب:\quad
$2\pi$
\انتہا{سوال}
%=====================
\ابتدا{سوال}
$y=\sqrt{x+1},\quad 1\le x\le 5,\quad x\,\text{محور}$
\انتہا{سوال}
%=====================
\ابتدا{سوال}
$x=\tfrac{y^3}{3},\quad 0\le y\le 1,\quad y\,\text{محور}$\\
جواب:\quad
$\tfrac{\pi(\sqrt{8}-1)}{9}$
\انتہا{سوال}
%=====================
\ابتدا{سوال}
$x=\tfrac{1}{3}y^{3/2}-y^{1/2},\quad 1\le y\le 3,\quad y\,\text{محور}$
\انتہا{سوال}
%=====================
\ابتدا{سوال}
$x=2\sqrt{4-y},\quad 0\le y\le \tfrac{15}{4},\quad y\,\text{محور}$\\
جواب:\quad
$\tfrac{35\pi\sqrt{5}}{3}$
\انتہا{سوال}
%=====================
\ابتدا{سوال}
$x=\sqrt{2y-1},\quad \tfrac{5}{8}\le y\le 1,\quad y\,\text{محور}$
\انتہا{سوال}
%=====================
\ابتدا{سوال}
$x=\tfrac{y^4}{4}+\tfrac{1}{8y^2},\quad 1\le y\le 2,\quad x\,\text{محور}$
\quad
(اشارہ۔ تکمل میں \عددی{\dif s=\sqrt{\dif x^2+\dif y^2}} کو \عددی{\dif y} کی صورت میں لکھ کر \عددی{S=\int 2\pi y \dif s} میں موزوں حد لیتے ہوئے حل کریں۔)\\
جواب:\quad
$\tfrac{253\pi}{20}$
\انتہا{سوال}
%=====================
\ابتدا{سوال}\شناخت{سوال_تکمل_استعمال_ترسیم_تکمل_ب}
$y=\tfrac{1}{3}(x^2+2)^{3/2},\quad 0\le x\le \sqrt{2},\quad y\,\text{محور}$
\quad
(اشارہ۔ تکمل میں \عددی{\dif s=\sqrt{\dif x^2+\dif y^2}} کو \عددی{\dif x} کی صورت میں لکھ کر \عددی{S=\int 2\pi x \dif s} میں موزوں حد لیتے ہوئے حل کریں۔)
\انتہا{سوال}
%=====================
\ابتدا{سوال}\ترچھا{نئی تعریف کی پرکھ}\\
تفاعل \عددی{y=\sqrt{a^2-x^2},\, -a\le x\le x} کو \عددی{x} محور کے گرد گھمانے سے کروی سطح حاصل ہوتا ہے۔ دکھائیں کہ مساوات \حوالہ{مساوات_تکمل_استعمال_سطحی_رقبہ_پ} سے بھی رداس \عددی{a} کرہ کا سطحی رقبہ \عددی{4\pi a^2} حاصل ہوتا ہے۔
\انتہا{سوال}
%=======================
\ابتدا{سوال}\ترچھا{نئی تعریف کی پرکھ}\\
لکیری قطع \عددی{y=\tfrac{r}{h}x,\, 0\le x\le h} کو \عددی{x} محور کے گرد گھمانے سے مخروط پیدا ہوتا ہے جس کے پہلو کا رقبہ \عددی{\pi r\sqrt{r^2+h^2}} ہوتا ہے جہاں مخروط کا قد \عددی{h} اور اس کے قاعدہ کا رداس \عددی{r} ہے لہٰذا اس کے ترچھا قد \عددی{\sqrt{r^2+h^2}} ہو گا۔ تکمل سے مخروط کے پہلو کا رقبہ دریافت کر کے اس کلیہ کی تصدیق کریں۔ 
\انتہا{سوال}
%=================
\ابتدا{سوال}
   (ا) منحنی \عددی{y=\cos x,\, -\tfrac{\pi}{2}\le x\le \tfrac{\pi}{2}} کو \عددی{x} محور کے گرد گھما کر سطح طواف پیدا ہوتا ہے۔ اس سطح طواف کے رقبہ کا تکمل لکھیں جس کو حل کرنا بعد میں سکھایا جائے گا۔ (ب) اس سطحی رقبے کو اعدادی طریقہ سے دریافت کریں۔\\
جواب:\quad
(ا) \عددی{2\pi\int_{-\pi/2}^{\pi/2}(\cos x)\sqrt{1+\sin^2x}\dif x}، (ب) \عددی{\approx 14.4236}
\انتہا{سوال}
%===============
\ابتدا{سوال}\ترچھا{ستارہ نما کا سطحی رقبہ}\\
ستارہ نما \عددی{x^{2/3}+y^{2/3}=1} کا وہ حصہ جو \عددی{x} محور سے اوپر پایا جاتا ہے کو \عددی{x} محور کے گرد گھما کر سطح طواف پیدا کیا جاتا ہے۔ اس سطح طواف کا رقبہ معلوم کریں۔ (اشارہ۔ ربع اول میں منحنی کے حصہ \عددی{y=(1-x^{2/3})^{3/2},\, 0\le x\le 1}) کو \عددی{x} محور کے گرد گھما کر نتیجہ  کو دگنا کریں۔)
\انتہا{سوال}
%================
\ابتدا{سوال}\ترچھا{رنگ}\\
ایک برتن کو رداس \عددی{\SI{16}{\centi\meter}} کے کرہ کا حصہ تصور کیا جا سکتا ہے۔برتن کی گہرائی \عددی{\SI{9}{\centi\meter}} ہے۔برتن کو اندر اور باہر سے رنگ کرنا مطلوب ہے۔ کچے رنگ کی \عددی{\SI{0.5}{\milli\meter}} موٹی تہہ برتن پر چھڑک کر پکائی جاتی ہے۔ پانچ ہزار برتن کے لئے درکار کچے رنگ کا حجم معلوم کریں۔ رنگ کے ضیاع کو نظر انداز کریں۔\\
جواب:\quad
$\SI{452.4}{\liter}$
\انتہا{سوال}
%======================
\ابتدا{سوال}\ترچھا{ڈبل روٹی کا کرارا حصہ}\\
ڈبل روٹی اندر سے نرم اور باہر سے کرارا ہوتی ہے۔کیا آپ جانتے ہیں کہ کروی ڈبل روٹی کے ایک جتنی موٹے ٹکڑوں میں  ایک جتنا کرارا حصہ پایا جاتا ہے؟ یہ دیکھنے کی خاطر نصف دائرہ \عددی{y=\sqrt{r^2-x^2}} کو \عددی{x} محور کے گرد گھما کر کرہ بنائیں۔فرض کریں محور \عددی{x} پر وقفہ \عددی{h} کے اوپر نصف دائرے کا قوس \عددی{AB} ہے۔ دکھائیں کہ نصف دائرے کو \عددی{x} محور کے گرد گھمانے سے \عددی{AB} سے حاصل رقبہ  کی قیمت \عددی{h} کے مقام پر منحصر نہیں ہے۔ (کرارا رقبہ کی قیمت \عددی{h} پر منحصر ہو گی۔)
\انتہا{سوال}
%==================
\ابتدا{سوال}
دو متوازی سطحیں جن کے مابین فاصلہ \عددی{h} ہے رداس \عددی{R} کے کروی سطح سے ایک پٹی کاٹتے ہیں۔ دکھائیں کہ اس پٹی کا رقبہ \عددی{2\pi Rh} ہو گا۔   
\انتہا{سوال}
%===================
\ابتدا{سوال}
موسمیاتی ریڈار کو شکل میں دکھائے گنبد میں رکھا گیا ہے۔گنبد کا بیرونی رقبہ کتنا ہو گا؟ (قاعدہ کو شامل نہ کریں۔)
\انتہا{سوال}
%====================
\ابتدا{سوال}\ترچھا{محور طواف کو قطع کرنے والے منحنیات سے حاصل سطح طواف}\\
وقفہ \عددی{[a,b]} پر تفاعل \عددی{f} کو غیر منفی تصور کرتے ہوئے  مساوات \حوالہ{مساوات_تکمل_استعمال_سطحی_رقبہ_پ} اخذ کی گئی۔ جہاں تفاعل محور طواف کو قطع کرتا ہو وہاں ہم مساوات \حوالہ{مساوات_تکمل_استعمال_سطحی_رقبہ_پ} کی جگہ درج ذیل مطلق قیمت کلیہ استعمال کرتے ہیں۔
\begin{align}\label{مساوات_تکمل_استعمال_سطحی_رقبہ_ٹ}
S=\int 2\pi \rho \dif s=\int 2\pi \abs{f(x)}\dif s
\end{align}
تفاعل \عددی{y=\tfrac{x^3}{9}-\sqrt{3},\, -\sqrt{3}\le x\le \sqrt{3}}  کو محور \عددی{x} کے گرد گھمانے سے حاصل دوہرا مخروط کا سطحی رقبہ مساوات \حوالہ{مساوات_تکمل_استعمال_سطحی_رقبہ_ٹ} استعمال کرتے ہوئے  دریافت کریں۔\\
جواب:\quad
$5\sqrt{2}\pi$
\انتہا{سوال}
%=================
\ابتدا{سوال}
قوس \عددی{y=\tfrac{x^3}{9}-\sqrt{3},\, -\sqrt{3}\le x\le \sqrt{3}} کو محور \عددی{x} کے گرد گھما کر سطح طواف پیدا کیا جاتا ہے۔ مساوات \حوالہ{مساوات_تکمل_استعمال_سطحی_رقبہ_ٹ} میں مطلق کی علامت ہٹا کر سطحی رقبہ تلاش کرنے سے کیا ہو گا؟
\انتہا{سوال}
%====================
\موٹا{اعدادی تکمل}\\
سوال \حوالہ{سوال_تکمل_استعمال_اعدادی_درستگی_الف} تا سوال \حوالہ{سوال_تکمل_استعمال_اعدادی_درستگی_الف} میں محور \عددی{x} کے گرد دیے گئے منحنیات گھمانے سے سطح طواف پیدا ہوں گے۔ ان سطح طواف کے رقبے اعدادی تراکیب سے \عددی{2} اعشاریہ درستگی تک معلوم کریں۔

\ابتدا{سوال}\شناخت{سوال_تکمل_استعمال_اعدادی_درستگی_الف}
$y=\sin x,\quad 0\le x\le \pi$\\
جواب:\quad
$14.4$
\انتہا{سوال}
%=======================
\ابتدا{سوال}
$y=\tfrac{x^2}{4},\quad 0\le x\le 2$
\انتہا{سوال}
%=======================
\ابتدا{سوال}
$y=x+\sin 2x,\quad -\tfrac{2\pi}{3}\le x\le \tfrac{2\pi}{3}$\\
جواب:\quad
$54.9$
\انتہا{سوال}
%=======================
\ابتدا{سوال}
$y=\tfrac{x}{12}\sqrt{36-x^2},\quad 0\le x\le 6$
\انتہا{سوال}
%=======================
\ابتدا{سوال}\ترچھا{سطحی رقبہ کا متبادل کلیہ}\\
فرض کریں \عددی{[a,b]} پر \عددی{f} ہموار ہے۔ وقفہ \عددی{[a,b]} کی خانہ بندی کریں اور \عددی{k} ویں ذیلی وقفہ \عددی{[x_{k-1},x_k]} کے وسطی نقطہ \عددی{m_k=(\tfrac{x_{k-1}+x_k}{2})} پر منحنی کی مماس لکیر بنائیں۔
\begin{enumerate}[a.]
\item
درج ذیل دکھائیں۔
\begin{align*}
r_1=f(m_k)-f'(m_k)\frac{\Delta x_k}{2},\quad r_2=f(m_k)+f'(m_k)\frac{\Delta x_k}{2}
\end{align*}
\item
دکھائیں کہ \عددی{k} ویں ذیلی وقفہ میں مماسی قطع کی لمبائی \عددی{ L_k=\sqrt{(\Delta x_k)^2+(f'(m_k)\Delta x_k)^2}} ہے۔
\item
دکھائیں کہ مماسی قطع کو محور \عددی{x} کے گرد گھمانے سے حاصل سطح طواف کا رقبہ پہلو \عددی{2\pi f(m_k)\sqrt{1+(f'(m_k))^2}\Delta x_k} ہو گا۔
\item
دکھائیں کہ وقفہ \عددی{[a,b]} پر \عددی{y=f(x)} کو محور \عددی{x} گھمانے سے حاصل سطح طواف کا رقبہ درج ذیل ہو گا۔
\begin{align*}
\lim_{n\to \infty}\sum_{k=1}^n(\text{\RL{$k$ ویں مخروط مقطوع کا رقبہ پہلو}})=\int_a^b2\pi f(x)\sqrt{1+(f'(x))^2}\dif x
\end{align*}
\end{enumerate}
\انتہا{سوال}
%=======================

\حصہ{معیار اثر اور مرکز کمیت}
بہت سارے ساخت اور میکانی نظام کا رویہ ایسا ہوتا ہے جیسا ان کی کمیت ایک نقطہ میں سموئی   ہو جس کو مرکز کمیت کہتے ہیں۔ اس نقطہ کا مقام جاننا اہم ہے جسے ریاضی کی مدد سے معلوم کیا جا سکتا ہے۔ اس باب میں یک بعدی اور دو بعد چیزوں پر توجہ دی جائے گی۔ تین بعدی چیزوں پر بعد کے باب میں غور کیا جائے گا۔

\جزوحصہء{لکیر پر کمیت}
ہم اپنا ریاضی نمونہ بتدریج تیار کرتے ہیں۔ ابتدائی منزل میں ہم  محور \عددی{x} جس کا مبدا اس کا چول ہو، پر کمیت \عددی{m_1}، \عددی{m_2} اور \عددی{m_3} تصور کرتے ہیں۔یہ نظام متوازن یا غیر متوازن ہو گا۔ توازن کا دارومدار کمیتوں کی مقدار اور ان کے مقامات پر منحصر ہے۔ 

ہر کمیت \عددی{m_k} پر نیچے رخ قوت \عددی{m_kg} عمل کرتا ہے جہاں \عددی{g} ثقلی اسراع ہے۔ہر ایسی قوت محور کو مبدا کے گرد گھمانے کی کوشش کرتی ہے۔ گھومنے کے اس اثر کو \اصطلاح{قوت مروڑ}\فرہنگ{قوت مروڑ}\حاشیہب{torque}\فرہنگ{torque} کہتے ہیں۔ قوت \عددی{m_kg} کو مبدا سے فاصلہ \عددی{x_k} سے ضرب دینے سے قوت مروڑ کی مقدار حاصل ہوتی ہے جہاں فاصلہ مثبت یا منفی ممکن ہے۔مبدا سے بائیں جانب کمیت منفی (گھڑی مخالف) قوت مروڑ پیدا کرتا ہے جبکہ مبدا سے دائیں جانب کمیت مثبت (گھڑی رخ) قوت مروڑ پیدا کرتا ہے۔

قوت مروڑ کا مجموعہ، مبدا کے گرد نظام گھومنے کے رجحان کا ناپ ہے۔ اس مجموعہ کو \اصطلاح{نظام کی قوت مروڑ}\فرہنگ{قوت مروڑ!نظام}\حاشیہب{system torque}\فرہنگ{torque!system} کہتے ہیں۔
\begin{align}
\text{\RL{نظام کی قوت مروڑ}}=m_1gx_1+m_2gx_2+m_3gx_3
\end{align}
نظام صرف اور صرف اس صورت متوازن ہو گا جب نظام کی قوت مروڑ صفر ہو۔

نظام کی قوت مروڑ کو
\begin{align*}
\underbrace{g}_{\text{\RL{خاصیت ماحول}}}\underbrace{(m_1x_1+m_2x_2+m_3x_3)}_{\text{\RL{خاصیت نظام}}}
\end{align*}
لکھا جا سکتا ہے جہاں \عددی{g} اس ماحول کی خاصیت ہے جس میں نظام پایا جاتا ہے جبکہ  عدد \عددی{(m_1x_1+m_2x_2+m_3x_3)} نظام کی خاصیت ہے جو ایک مستقل ہے اور نظام کو ایک ماحول سے دوسرے ماحول میں منتقل کرنے سے تبدیل نہیں ہوتا۔

عدد \عددی{(m_1x_1+m_2x_2+m_3x_3)} کو \اصطلاح{مبدا کے لحاظ سے نظام کا معیار اثر} کہتے ہیں جو انفرادی کمیت کے معیار اثر \عددی{m_1x_1}، \عددی{m_2x_2} اور \عددی{m_3x_3} کا مجموعہ ہے۔
\begin{align*}
M_0=\text{\RL{مبدا کے لحاظ سے نظام کا معیار اثر}}=\sum m_kx_k
\end{align*}

ہم نظام کو متوازن بنانے کی خاطر نظام کے چول کا مقام جاننا چاہتے ہیں، یعنی چول کو کس نقطہ \عددی{\bar{x}} پر رکھنے سے نظام کا قوت مروڑ صفر ہو گا۔

اس مخصوص مقام پر چول رکھنے سے ہر کمیت کا قوت مروڑ درج ذیل لکھا جا سکتا ہے جہاں فاصلہ مثبت یا منفی ہو سکتا ہے۔
\begin{align*}
\text{\RL{$\bar{x}$کے لحاظ سے $m_k$ کا معیار اثر}}&=(\text{\RL{$\bar{x}$ سے $m_k$ کا فاصلہ}})(\text{\RL{نیچے رخ قوت}})\\
&=(x_k-\bar{x})m_kg
\end{align*} 
ان معیار اثر کے مجموعہ کو صفر کے برابر پر کرنے سے ہمیں ایسی مساوات ملتی ہے جسے ہم \عددی{\bar{x}} کے لئے حل کر سکتے ہیں:
\begin{align*}
\sum(x_k-\bar{x})m_kg&=0&&\text{\RL{معیار اثر کا مجموعہ صفر ہے}}\\
g\sum(x_k-\bar{x})m_k&=0&&\text{\RL{مجموعہ کا قاعدہ مستقل مضرب}}\\
\sum(m_kx_k-\bar{x}m_k)&=0&&\text{\RL{$g$ سے تقسیم اور $m_k$ پھیلایا گیا ہے}}\\
\sum m_kx_k-\sum\bar{x}m_k&=0&&\text{\RL{مجموعہ کا قاعدہ فرق}}\\
\sum m_kx_k&=\bar{x}\sum m_k&&\text{\RL{مستقل مضرب قاعدہ اور منتقلی}}\\
\bar{x}&=\frac{\sum m_kx_k}{\sum m_k}&&\text{\RL{$\bar{x}$ کے لئے حل}}
\end{align*} 
یہ آخری مساوات کہتی ہے کہ \عددی{\bar{x}} معلوم کرنے کے لئے مبدا کے لحاظ سے نظام کے معیار اثر کو نظام کی کل کمیت سے تقسیم کریں۔
\begin{align*}
\bar{x}=\frac{\sum x_km_k}{\sum m_k}=\frac{\text{\RL{مبدا کے لحاظ سے نظام کا معیار اثر}}}{\text{\RL{نظام کی کمیت}}}
\end{align*}
نقطہ \عددی{\bar{x}} کو نظام کا \اصطلاح{مرکز کمیت}\فرہنگ{کمیت!مرکز}\حاشیہب{center of mass}\فرہنگ{mass!center of} کہتے ہیں۔

\جزوحصہء{تار اور پتلے سلاخ}

