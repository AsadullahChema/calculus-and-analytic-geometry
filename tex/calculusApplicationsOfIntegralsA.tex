\حصہ{کام}
روزمرہ زندگی میں کام سے مراد وہ عمل ہے جو جسمانی یا ذہنی قوت سے سر انجام دیا جائے۔ سائنس میں کام کی تعریف اس سے مختلف ہے۔ اس حصہ میں کام کی سائنسی  تعریف پیش کی جائے گی اور کام کی قیمت کا حصول سکھایا جائے گا۔

\جزوحصہء{مستقل قوت اور کام}
جب کوئی جسم جس پر مستقل قوت \عددی{F} عمل کرتی ہو، قوت کی سمت میں سیدھی لکیر پر فاصل \عددی{d} حرکت کرے تب ہم (سائنسی طور پر) کہتے ہیں کہ قوت \عددی{F} اس جسم پر کام \عددی{W} کرتی ہے:
\begin{align}\label{مساوات_تکمل_استعمال_کام_مستقل_قوت}
W=Fd
\end{align}
آپ دیکھ سکتے ہیں کہ سائنس میں لفظ کام کی معنی روزمرہ زندگی میں استعمال معنی سے مختلف ہے۔ اگر آپ کسی گاڑی کو سڑک پر دکھا لگا کر ایک جگہ سے دوسری جگہ منتقل کریں تب آپ کی روزمرہ خیال کے مطابق آپ نے کام کیا اور مساوات \حوالہ{مساوات_تکمل_استعمال_کام_مستقل_قوت} کے تحت بھی آپ نے کام کیا۔ اس کے برعکس اگر آپ پورا دن گاڑی کو دکھا لگاتے رہیں لیکن گاڑی اپنی جگہ سے حرکت نہ کرے تب اگرچہ آپ کا خیال ہو گا کہ آپ نے بہت کام کیا لیکن مساوات \حوالہ{مساوات_تکمل_استعمال_کام_مستقل_قوت} کے تحت آپ نے کوئی کام نہیں کیا۔

مساوات \حوالہ{مساوات_تکمل_استعمال_کام_مستقل_قوت} سے واضح ہے کہ قوت کی اکائی کو فاصلہ کی اکائی سے ضرب دینے سے کام کی اکائی حاصل ہو گی۔ بین الاقوامی نظام اکائی میں قوت کی اکائی نیوٹن \عددی{\si{\newton}} اور فاصلہ کی اکائی میٹر \عددی{\si{\meter}} ہے لہٰذا اس نظام میں کام کی اکائی نیوٹن میٹر  \عددی{\si{\newton}\cdot\si{\meter}} ہو گی جس کو خصوصی نام \اصطلاح{جاول}\فرہنگ{جاول}\حاشیہب{joule}\فرہنگ{joule} دیا گیا ہے اور جس کو \عددی{\si{\joule}} سے ظاہر کیا جاتا ہے۔

\ابتدا{مثال}
فرض کریں  آپ  \عددی{\SI{80}{\kilo\gram}} کمیت کو \عددی{\SI{30}{\centi\meter}} بلندی تک اٹھاتے ہیں۔ایسا کرتے ہوئے آپ درج ذیل کام کرتے ہیں۔
\begin{align*}
W=Fd=(80)(9.8)(0.3)=\SI{235.2}{\joule}
\end{align*}
\انتہا{مثال}
%======================
\جزوحصہء{متغیر قوت اور کام}
اگر آپ پانی کی ایسی بالٹی کو اٹھائیں جس سے پانی ٹپکتا ہو تب لاگو قوت کی قیمت بلندی کے ساتھ تبدیل ہو گی۔ایسی صورت میں قوت کا کلیہ \عددی{W=Fd} تبدیل کرتے ہوئے تکمل کا استعمال ضروری ہو گا جو قوت  کی تبدیلی کا حساب رکھ سکے۔ 

فرض کریں  کہ محور \عددی{x} سے اس لکیر کو ظاہر کرنا ممکن ہے جس پر قوت عمل کرتی ہے اور قوت کی مقدار \عددی{F} کو  فاصلہ \عددی{x} کا استمراری تفاعل تصور کیا جا سکتا ہے۔ ہم وقفہ \عددی{x=a} تا \عددی{x=b} پر قوت کے کام کو معلوم کرنا چاہتے ہیں۔  ہم وقفہ \عددی{[a,b]} کی خانہ بندی کرتے ہوئے ہر ذیلی وقفہ \عددی{[x_{k-1},x_k]} میں کوئی نقطہ \عددی{c_k} منتخب کرتے ہیں۔ اگر ذیلی وقفہ چھوٹا ہو تب \عددی{x_{k-1}} سے \عددی{x_k} تک کے  فاصلہ میں  استمراری  قوت \عددی{F} کی تبدیلی (استمراری ہونے کی بنا) بہت کم ہو گی جس کو رد کیا جا سکتا ہے۔یوں \عددی{x_{k-1}} سے \عددی{x_k} تک حرکت کے دوران کام کی قیمت تخمیناً \عددی{F(c_k)\Delta x_k} ہو گی۔یوں درج ذیل ریمان مجموعہ \عددی{x=a} سے \عددی{x=b} تک قوت \عددی{F} کا کام دے گا۔
\begin{align}
\sum_{k=1}^n F(c_k)\Delta x_k
\end{align}
ہم توقع کرتے ہیں کہ  جیسے جیسے خانہ بندی کا معیار صفر تک پہنچتا ہو ویسے ویسے یہ تخمین مزید بہتر ہو گی لہٰذا ہم \عددی{x=a} سے \عددی{x=b} تک \عددی{F} کے تکمل کو \عددی{a} سے \عددی{b} تک قوت \عددی{F} کے کام کی تعریف لیتے ہیں۔ 

\ابتدا{تعریف}
محور \عددی{x} پر \عددی{x=a} سے \عددی{x=b} تک لاگو متغیر قوت \عددی{F(x)} درج ذیل \اصطلاح{کام} کرتی ہے۔
\begin{align}
W=\int_a^bF(x)\dif x
\end{align}
\انتہا{تعریف}
%=====================

کام کی اکائی \اصطلاح{جاول} \عددی{\si{\joule}} ہے۔

\ابتدا{مثال}
قوت \عددی{F(x)=\tfrac{1}{x^2}\,\si{\newton}} محور \عددی{x} پر \عددی{x=\SI{1}{\meter}} تا \عددی{x=\SI{10}{\meter}} عمل کرتی ہے۔ یہ قوت درج ذیل کام کرتی ہے۔
\begin{align*}
W=\int_1^{10}\frac{1}{x^2}\dif x=\left.-\frac{1}{x}\right]_1^{10}=-\frac{1}{10}+1=\SI{0.9}{\joule}
\end{align*}
\انتہا{مثال}
%===================
\ابتدا{مثال}
گاؤں میں کنواں سے پانی نکالنے کے لئے بوکا استعمال کیا جاتا ہے۔ کھوہ کی گہرائی \عددی{\SI{20}{\meter}}، خالی بوکا کی کمیت \عددی{\SI{2}{\kilo\gram}} اور رسی کی کمیت \عددی{\SI{0.1}{\kilo\gram\per\meter}} ہے۔ بوکا میں ابتدائی طور پر \عددی{\SI{10}{\liter}} پانی ہوتا ہے۔چونکہ بوکا سے پانی رستا ہے لہٰذا جتنی دیر میں بوکے کو نیچے سے اوپر کھینچا جاتا ہے اتنی دیر میں بوکا خالی ہو جاتا ہے۔ بوکا سے پانی کے اخراج کو مستقل تصور کریں۔ درج ذیل کام معلوم کریں۔ 
\begin{enumerate}[a.]
\item
صرف پانی بلند کرنے کا کام۔
\item
پانی اور بوکا بلند کرنے کا کام۔
\item
پانی، بوکا اور رسی بلند کرنا کا کام۔
\end{enumerate}
حل:\quad
\begin{enumerate}[a.]
\item
\ترچھا{صرف پانی:}\quad
پانی اٹھانے کے لئے درکار قوت پانی کے وزن جتنا ہو گا جو ابتدا میں \عددی{(10)(9.8)=\SI{98}{\newton}} اور آخر میں صفر ہے۔یوں مبدا کو کنواں کی تہہ میں رکھتے ہوئے قوت کو
\begin{align*}
F(x)=98\big(\frac{20-x}{20}\big)=98\big(1-\frac{x}{20}\big)=98-4.9x\,\si{\newton}
\end{align*}
لکھا جا سکتا ہے لہٰذا  کام درج ذیل ہو گا۔
\begin{align*}
W&=\int_a^bF(x)\dif x\\
&=\int_0^{20}(98-4.9x)\dif x=\big[98x-\frac{4.9x^2}{2}\big]_0^{20}=1960-980=\SI{980}{\joule}
\end{align*}
\item
\ترچھا{صرف بوکا:}\quad
صرف بوکا اٹھانے کے لئے درکار کام مساوات \حوالہ{مساوات_تکمل_استعمال_کام_مستقل_قوت} کے تحت \عددی{(2)(9.8)(20)=\SI{392}{\joule}} ہو گا۔یوں پانی اور بوکا دونوں کے لئے درکار کام درج ذیل ہو گا۔
\begin{align*}
W=980+392=\SI{1372}{\joule}
\end{align*}
\item
\ترچھا{پانی، بوکا اور رسی:}\quad
مبدا سے \عددی{x} بلندی پر پانی، بوکا اور رسی کی کمیت کو \عددی{g=\SI{9.8}{\meter\per\second\squared}} سے ضرب دینے سے درج ذیل درکار قوت حاصل ہوتی ہے۔
\begin{align*}
F(x)=\underbrace{(98-4.9x)}_{\text{\RL{پانی کا متغیر وزن}}}+\underbrace{(19.6)}_{\text{\RL{بوکا کا مستقل وزن}}}+\underbrace{(0.1)(9.8)(20-x)}_{\text{\RL{رسی کا متغیر وزن}}}
\end{align*}
صرف رسی کو اوپر کھینچنے کا کام درج ذیل ہو گا۔
\begin{align*}
W&=\int_0^{20}(0.1)(9.8)(20-x)\dif x=\int_0^{20}(19.6-0.98x)\dif x\\
&=\big[19.6x-\frac{0.98x^2}{2}\big]_0^{20}=392-196=\SI{196}{\joule}
\end{align*}
یوں پانی، بوکا اور رسی تینوں کو کھینچنے کے لئے درکار کام درج ذیل ہو گا۔
\begin{align*}
W=980+392+196=\SI{1568}{\joule}
\end{align*}
\end{enumerate}
\انتہا{مثال}
%===================

\جزوحصہء{قانون ہک برائے اسپرنگ}
\اصطلاح{قانون ہک}\فرہنگ{قانون!ہک}\حاشیہب{Hooke's law}\فرہنگ{law!Hooke's} کے تحت کسی بھی اسپرنگ کی قدرتی لمبائی کو تان کر یا دبا کر \عددی{x}اکائیاں تبدیل کرنے کے لئے درکار قوت لمبائی \عددی{x} کے راست متناسب ہو گی:
\begin{align}\label{مساوات_تکمل_استعمال_قانون_ہک}
F=kx
\end{align} 
\اصطلاح{مستقلہ اسپرنگ}\فرہنگ{مستقلہ اسپرنگ} \عددی{k} جو اسپرنگ کی خاصیت ہے کو \اصطلاح{مقیاس لچک}\فرہنگ{مقیاس لچک}\حاشیہب{spring constant}\فرہنگ{spring constant} کہتے ہیں۔ مقیاس لچک کو قوت فی اکائی لمبائی میں ناپا جاتا ہے۔ جب تک لاگو قوت اسپرنگ کی دھاتی تار کو بگاڑ نہ دے قانون ہک (مساوات \حوالہ{مساوات_تکمل_استعمال_قانون_ہک}) بہترین نتائج دیتا ہے۔ اس حصہ میں ہم فرض کرتے ہیں کہ لاگو قوت اسپرنگ کو خراب نہیں کرتی ہے۔

\ابتدا{مثال}\شناخت{مثال_تکمل_استعمال_تان_اسپرنگ_الف}
ایک اسپرنگ جس کا مقیاس لچک \عددی{k=\SI{8}{\newton\per\meter}} ہے کی لمبائی کو \عددی{\SI{1}{\meter}} سے تبدیل کر کے  \عددی{\SI{0.8}{\meter}}  کیا جاتا ہے۔ درکار کام تلاش کریں۔

حل:\quad
ہم اسپرنگ کو محور \عددی{x} پر پڑا ہوا تصور کرتے ہیں (شکل \حوالہ{شکل_مثال_تکمل_استعمال_تان_اسپرنگ_الف})۔ اسپرنگ کا ایک سر مبدا پر ہے جبکہ اس کا دوسرا سر \عددی{x=1} پر  باندھا ہوا ہے۔ یوں ہم قوت کو \عددی{F=8x} لکھ سکتے ہیں جہاں \عددی{x} کی قیمت \عددی{0} تا \عددی{\SI{0.2}{\meter}} ہو گی۔درکار کام درج ذیل ہو گا۔
\begin{align*}
W=\int_0^{0.2}8x\dif x=\big[\frac{8x^2}{2}\big]_0^{0.2}=\SI{0.16}{\joule}
\end{align*}
\انتہا{مثال}
%========================
\begin{figure}
\centering
\begin{subfigure}{0.45\textwidth}
\centering
\begin{tikzpicture}
\coordinate (a) at (-4,0.5) {};
\coordinate (b) at (-3,1.75) {};
\draw(-3,1)--(-3,2.2);
\draw[-stealth](-4,1.25)--(-3,1.25)node[pos=0.5,fill=white]{$x$};
\draw[thick,stealth-](b)--++(-0.25,0)node[left]{$F$};
\draw[decoration={aspect=0.3, segment length=3mm, amplitude=3mm,coil},decorate] (0,0.5) -- (a); 
\draw[decoration={aspect=0.3, segment length=1.5mm, amplitude=3mm,coil},decorate] (0,1.75) -- (b); 
\fill [pattern = north east lines] (0,0) rectangle (0.25,2.2);
\draw[thick] (0,0) -- (0,2.2);
\draw(0,0)--(-4.5,0);
\draw(-4,0)node[below]{$0$}--++(0,0.1);
\draw(-3,0)node[below]{$0.2$}--++(0,0.1);
\draw[-latex](-4,0)--++(0,2.2);
\draw[-latex](0,0)--(0.75,0)node[right]{$x$};
\end{tikzpicture}
\caption{}
\end{subfigure}\hfill
\begin{subfigure}{0.45\textwidth}
\centering
\begin{tikzpicture}[]
\path[name path=fun](0,0)--(4,2);
\path[name path=k](2.5,0)--++(0,2);
\fill[lgray,name intersections={of=fun and k}](2.5,0)node[below,black]{$0.2$}--(intersection-1)--(0,0)--(2.5,0);
\path(intersection-1)--($(0,0)!(intersection-1)!(0,2)$)coordinate(ka);
\draw(ka)node[left]{$1.6$}--++(0.1,0);
\draw[-latex](-0.25,0)--(4,0)node[right]{$x$};
\draw[-latex](0,-0.2)--(0,2)node[above]{$F$};
\draw[name path=fun](0,0)--(4,2)node[left,xshift=-1ex]{$F=8x$};
\draw(1.75,0.5)node[]{کام};
\end{tikzpicture}
\caption{}
\end{subfigure}
\caption{اسپرنگ کی لمبائی میں تبدیلی اور قوت راست تناسب ہیں۔}
\label{شکل_مثال_تکمل_استعمال_تان_اسپرنگ_الف}
\end{figure}
\ابتدا{مثال}\شناخت{مثال_تکمل_استعمال_تان_اسپرنگ_ب}
ایک اسپرنگ جس کی قدرتی لمبائی \عددی{\SI{1}{\meter}} ہے کو \عددی{\SI{24}{\newton}} قوت سے تان کر \عددی{\SI{1.8}{\meter}} لمبا کیا جاتا ہے۔
\begin{enumerate}[a.]
\item
مقیاس لچک \عددی{k} تلاش کریں۔
\item
اسپرنگ کی لمبائی کو \عددی{\SI{2}{\meter}} تبدیل کرنے کے لئے درکار کام تلاش کریں۔
\item
اسپرنگ کی لمبائی میں \عددی{\SI{45}{\newton}} کی قوت کتنی تبدیلی پیدا کرے گی؟ 
\end{enumerate}
حل:
\begin{enumerate}[a.]
\item
\ترچھا{مقیاس لچک:}\quad
قیاس لچک کو مساوات \حوالہ{مساوات_تکمل_استعمال_قانون_ہک} سے حاصل کرتے ہیں۔ اسپرنگ کی لمبائی میں تبدیلی \عددی{\SI{0.8}{\meter}} ہے۔
\begin{align*}
24&=k(0.8)\quad \implies k=\frac{24}{0.8}=\SI{30}{\newton\per\meter}
\end{align*}
\item
\ترچھا{کام:}\quad
ہم اسپرنگ کو چھت سے یوں آویزاں تصور کرتے ہیں کہ اس کا آزاد سر \عددی{x=0} پر ہو (\حوالہ{شکل_مثال_تکمل_استعمال_تان_اسپرنگ_ب})۔اسپرنگ کی لمبائی کو اس کی  قدرتی لمبائی سے \عددی{x} میٹر زیادہ کرنے کے لئے درکار قوت \عددی{F=kx} ہو گی جو اسپرنگ کو نیچے رخ کھنچے گی۔یوں \عددی{x=0} سے \عددی{x=\SI{2}{\meter}} تک کھینچنے کے لئے کام درج ذیل ہو گا۔
\begin{align*}
W=\int_0^2 30x\dif x=\left.\frac{30x^2}{2}\right]_0^2=\SI{60}{\joule}
\end{align*}
\item
\ترچھا{لمبائی میں تبدیلی:}\quad
ہم مساوات \عددی{F=30x} میں \عددی{F=45} ڈال کر \عددی{x} تلاش کرتے ہیں۔
\begin{align*}
45=30x\quad\implies x=\frac{45}{30}=\SI{1.5}{\meter} 
\end{align*}
یوں اسپرنگ کی کل لمبائی \عددی{1+1.5=\SI{2.5}{\meter}} ہو گی۔
\end{enumerate}
\انتہا{مثال}
%================
\begin{figure}
\centering
\begin{tikzpicture}
\node[circle,fill=lgray,inner sep=2mm] (a) at (1.75,0) {};
\node[inner sep=2.5mm] (b) at (0,0.8) {};
\draw[decoration={aspect=0.3, segment length=3mm, amplitude=3mm,coil},decorate] (1.75,2) -- (a)node[right,xshift=2ex]{$\SI{24}{\newton}$}; 
\draw[decoration={aspect=0.3, segment length=1.5mm, amplitude=3mm,coil},decorate] (0,2) -- (b); 
\fill [pattern = north east lines] (-1,2) rectangle (2.5,2.2);
\draw[thick] (-1,2) -- (2.5,2);
\draw(-1,2)--(-1,-0.2);
\draw(-1,1.05)node[left]{$x=\SI{0}{\meter}$}--++(0.1,0);
\draw(-1,0.25)node[left]{$x=\SI{0.8}{\meter}$}--++(0.1,0);
\end{tikzpicture}
\caption{قوت نے اسپرنگ کی لمبائی کو بڑھایا ہے۔}
\label{شکل_مثال_تکمل_استعمال_تان_اسپرنگ_ب}
\end{figure}

\جزوحصہء{پانی کی نکاسی}
کسی برتن یا ٹینکی سے پانی کی نکاسی کے لئے کتنا کام درکار ہو گا؟ ہم پانی کو افقی تہوں میں تقسیم کرتے ہوئے ایک ایک تہہ کو برتن سے باہر نکالتے ہیں۔یوں اگر تہہ کی موٹائی \عددی{\dif y} اور اس کے سطحی رقبہ \عددی{S} ہو تب اس کی کمیت \عددی{\rho S\dif y} اور وزن \عددی{\rho S g \dif y} ہو گا جہاں پانی کی کثافت کو \عددی{\rho} اور کشش ثقل کو \عددی{g} سے ظاہر کیا گیا ہے۔ اس تہہ کو بلندی \عددی{h} تک منتقل کرنے کے لئے \عددی{\dif W=Fh=\rho S g  h \dif y} کام کرنا ہو گا۔یوں تمام تہوں کو نکالنے کے لئے تکمل حل کرنا ہو گا۔  اگلے مثال میں ایک ٹھوس مثال پیش کی گئی ہے۔

\ابتدا{مثال}\شناخت{مثال_تکمل_استعمال_نکاسی_پانی}
پانی سے بھری ہوئی ایک بیلنی ٹینکی کا رداس \عددی{\SI{5}{\meter}} اور قد \عددی{h=\SI{10}{\meter}} ہے۔ پانی کو \عددی{\SI{14}{\meter}} بلندی پر منتقل کرنے کے لئے کتنا کام کرنا ہو گا؟

حل:\quad
ہم ٹینکی کو کارتیسی محدد پر تصور کرتے ہوئے وقفہ \عددی{[0,10]} کی خانہ بندی کر کے  پانی کو تہہ در تہہ تقسیم کرتے ہیں (شکل \حوالہ{شکل_مثال_تکمل_استعمال_نکاسی_پانی})۔ سطح \عددی{y} اور سطح \عددی{y+\dif y} کے بیچ پانی کا حجم 
\begin{align*}
\Delta H=\pi(\text{رداس})^2(\text{موٹائی})=\pi (5)^2\Delta y=25\pi\Delta y\,\si{\meter\cubed}
\end{align*}
اور کمیت
\begin{align*}
\dif M=(\rho) (\Delta H)=(1000)(25\pi\Delta y)=\num{25000}\pi \Delta y\,\si{\kilo\gram}
\end{align*}
ہو گی جہاں پانی کی کثافت \عددی{\rho=\SI{1000}{\kilo\gram\per\meter\cubed}} ہے۔ اس تہہ پر کشش ثقل کی وجہ سے نیچے رخ  قوت عمل کرے گی لہٰذا اس تہہ کو اٹھانے کی خاطر تہہ کی وزن کے برابر قوت \عددی{F} درکار ہو گی:
\begin{align*}
F=(g)(\dif M) =(9.8)(\num{25000}\pi \Delta y)=\num{245000}\pi\Delta y \,\si{\newton}
\end{align*}
یوں اس تہہ کو \عددی{y} کی بلندی سے \عددی{\SI{14}{\meter}} کی بلندی تک اٹھانے کے لئے درج ذیل کام کرنا ہو گا۔
\begin{align*}
\dif W=(\text{قوت})(\text{فاصلہ})=(\num{245000}\pi)(14-y)\Delta y\,\si{\joule}
\end{align*}
تمام پانی کو اس بلندی تک اٹھانے کے لئے تخمیناً 
\begin{align*}
W\approx \sum_0^{10}\Delta W=\sum_0^{10}\Delta y\,\si{\joule}
\end{align*}
کام کرنا ہو گا جو وقفہ \عددی{0\le y\le 10} پر تفاعل \عددی{\num{245000}\pi(14-y)} کا ریمان مجموعہ ہے۔ ٹینکی خالی کرنے کے لئے درکار کام  \عددی{\norm{P}\to 0} کی صورت میں اس ریمان مجموعے کا حد ہو گا:
\begin{align*}
W&=\int_0^{10}\num{245000}\pi(14-y)\dif y=\num{245000}\pi\int_0^{10}(14-y)\dif y\\
&=\num{245000}\pi\big[14y-\frac{y^2}{2}\big]_0^{10}=\num{245000}\pi[90]\approx \SI{69.3e6}{\joule}
\end{align*}
ایک کلو واٹ طاقت کا بجلی کا پمپ ایک سیکنڈ میں \عددی{\SI{1000}{\joule}} کام کرتا ہے۔اس پمپ کو یہ ٹینکی خالی کرنے کے لئے تقریباً \عددی{19} گھنٹے اور \عددی{15} منٹ کا وقت درکار ہو گا۔
\انتہا{مثال}
%========================
\begin{figure}
\centering
\begin{minipage}{0.45\textwidth}
\centering
\begin{tikzpicture}
\pgfmathsetmacro{\h}{2}
\pgfmathsetmacro{\r}{1.25}
\pgfmathsetmacro{\dh}{1}
\draw[-latex](\r,0)++(0.1,0)--++(0.5,0)node[right]{$x$};
\draw[-latex](0,\h)node[circ]{}node[left]{$10$}--++(0,\dh)node[above]{$y$};
\draw([shift={(180:\r cm and 1/4*\r cm)}]0,0) arc (180:360:\r cm and 1/4*\r cm);
\draw(0,\h) circle (\r cm and 1/4*\r cm);
\draw(-\r,0)--(-\r,\h)  (\r,0)--(\r,\h);
\draw[gray](0,1/2*\h) circle (\r cm and 1/4*\r cm);
\draw(0,1/2*\h)node[left]{$y$}--++(\r,0)node[pos=0.5,above]{$5$};
\draw(0,\h+3/4*\dh)node[left]{$14$}--++(1.25*\r,0);
\draw(\r,1/2*\h)++(0.2,0)--++(0.2,0)coordinate[pos=0.5](ka);
\draw(\r,1/2*\h)++(0.2,0.1)--++(0.2,0)coordinate[pos=0.5](kka);
\draw[stealth-](ka)--++(0,-0.2);
\draw[stealth-](kka)--++(0,0.2)--++(0.2,0)node[right]{$\dif y$};
\draw(\r,\h)++(0.1,0)--++(0.2,0)coordinate[pos=0.5](kb);
\draw[stealth-stealth](kb)--++(0,3/4*\dh)node[pos=0.5,right]{$\SI{4}{\meter}$};
\draw(0,\h+3/4*\dh)++(-4ex,0)--(-\r-0.3,\h+3/4*\dh);
\draw(-\r-0.1,1/2*\h)--++(-0.2,0)coordinate[pos=0.5](kc);
\draw[stealth-stealth](kc)--++(0,1/2*\h+3/4*\dh)node[pos=0.5,left]{$14-y$};
\end{tikzpicture}
\caption{بیلنی ٹینکی (مثال \حوالہ{مثال_تکمل_استعمال_نکاسی_پانی})}
\label{شکل_مثال_تکمل_استعمال_نکاسی_پانی}
\end{minipage}\hfill
\begin{minipage}{0.45\textwidth}
\centering
\begin{tikzpicture}[font=\small]
\pgfmathsetmacro{\r}{1}
\pgfmathsetmacro{\ra}{\r-0.25}
\pgfmathsetmacro{\rb}{3/4*(\r-0.25)}
\draw[-latex](-0.25,0)--(2.5,0)node[right]{$x$};
\draw(0,0)--(1.25,2.5)node[above]{$x=\tfrac{y}{2}$};
\draw(0,2*\r) circle (\r cm and 1/4*\r cm);
\draw[gray](0,2*\ra) circle (\ra cm and 1/4*\ra cm);
\draw(0,0)--(-1,2);
\draw[-latex](0,2)-++(0,1)node[above]{$y$};
\draw(0,2)node[left]{$3.5$} (1,2)node[circ]{}node[right]{$(1.75,3.5)$};
\draw[gray](0,2*\ra)node[circ]{}node[left]{$3$};
\draw[gray](0,2*\rb) circle (\rb cm and 1/4*\rb cm);
\draw[gray](0,2*\rb)node[left]{$y$}--++(\rb,0);
\draw(\rb,2*\rb)++(0.2,0)--++(0.2,0)coordinate[pos=0.5](ka);
\draw(\rb,2*\rb)++(0.2,0)++(0.2,0.1)--++(-0.2,0)coordinate[pos=0.5](kb);
\draw[stealth-](ka)--++(0,-0.2)--++(0.2,0)node[right]{$\Delta y$};
\draw[stealth-](kb)--++(0,0.2);
\draw(-\r,2*\r)++(-0.1,0)--++(-0.4,0)coordinate[pos=0.75](ka);
\draw(-\r,2*\rb)++(-0.1,0)--++(-0.4,0)coordinate[pos=0.75](kb);
\draw[stealth-stealth](ka)--(kb)node[pos=0.5,left]{$3.5-y$};
\end{tikzpicture}
\caption{زیتون تیل کی مخروط ٹینکی (مثال \حوالہ{مثال_تکمل_استعمال_مخروط_ٹینکی})}
\label{شکل_مثال_تکمل_استعمال_مخروط_ٹینکی}
\end{minipage}
\end{figure}
\ابتدا{مثال}\شناخت{مثال_تکمل_استعمال_مخروط_ٹینکی}
ایک مخروط ٹینکی جس کو شکل \حوالہ{شکل_مثال_تکمل_استعمال_مخروط_ٹینکی} میں دکھایا گیا ہے کنارے سے \عددی{\SI{0.5}{\meter}} نیچے تک  زیتون کی تیل  سے بھری ہے۔ زیتون کی تیل کی کثافت \عددی{\rho=\SI{930}{\kilo\gram\per\meter\cubed}} ہے۔ تیل کو ٹینکی کے کنارے تک پمپ کرنے کے لئے کتنا کام درکار ہو گا؟

حل:\quad
ہم وقفہ \عددی{[0,3]} کی خانہ بندی کرتے ہوئے خانہ بندی کے نقطوں پر افقی سطحیں تصور کرتے ہوئے تیل کو باریک تہوں میں تقسیم کرتے ہیں۔ سطح \عددی{y} اور سطح \عددی{y+\Delta y} کے بیچ تہہ کا حجم درج ذیل ہو گا۔
\begin{align*}
\Delta H=\pi(\text{رداس})^2(\text{موٹائی})=\pi\big(\frac{y}{2}\big)^2\Delta y=\frac{\pi}{4}y^2\Delta y\,\si{\meter\cubed}
\end{align*}
اس تہہ کو اٹھانے کے لئے اس تہہ کی وزن کے برابر قوت \عددی{F(y)} درکار ہو گا:
\begin{align*}
F(y)=\rho g \Delta H=(930)(9.8)\big(\frac{\pi}{4}y^2\Delta y\big)=\frac{9114\pi}{4}y^2\Delta y\,\si{\newton}
\end{align*}
ٹینکی کے کنارے سے اس تہہ تک کا فاصلہ \عددی{3.5-y} ہے  لہٰذا اس تہہ کو ٹینکی کے کنارے تک اٹھانے کے لئے درج ذیل کام درکار ہو گا۔
\begin{align*}
\Delta W=\frac{9114\pi}{4}(3.5-y)y^2\Delta y\si{\joule}
\end{align*}
\عددی{y=0} سے \عددی{y=3} تک تمام تہوں کو ٹینکی کے کنارے تک اٹھانے کے لئے تخمیناً
\begin{align*}
W\approx\sum_0^{3}\frac{9114\pi}{4}(3.5-y)y^2\Delta y\si{\joule}
\end{align*}
کام درکار ہو گا جو وقفہ \عددی{[0,3]} پر تفاعل \عددی{\frac{9114\pi}{4}(3.5-y)y^2} کا ریمان مجموعہ ہے۔ تیل کو ٹینکی کے کنارے تک پمپ کرنے کے لئے درکار کام، خانہ بندی کا معیار صفر تک کرنے سے حاصل، ریمان مجموعے کا حد ہو گا:
\begin{align*}
W&=\int_0^{3}\frac{9114\pi}{4}(3.5-y)y^2\dif y\\
&=\frac{9114\pi}{4}\int_0^{3}(3.5y^2-y^3)\dif y\\
&=\frac{9114\pi}{4}\big[\frac{3.5y^3}{3}-\frac{y^4}{4}\big]_0^3\approx \SI{80529}{\joule}
\end{align*}
\انتہا{مثال}
%==========================

\حصہء{سوالات}

