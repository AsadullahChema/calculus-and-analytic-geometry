\حصہ{سطح طواف کا رقبہ}
بچپن میں آپ نے دوستوں کے ساتھ مل کر رسی گھماتے ہوئے رسی کے اوپر سے چھلانگیں ضرور لگائی ہوں گی۔ یہ رسی فضا میں پھیر  کر ایک سطح بناتی ہے جس کو \اصطلاح{سطح طواف}\فرہنگ{طواف!سطح}\فرہنگ{سطح!طواف}\حاشیہب{surface of revolution}\فرہنگ{revolution!surface} کہتے ہیں۔ سطح طواف کا رقبہ رسی کی لمبائی اور رسی کے ہر حصے کی جھول پر منحصر ہو گا۔ اس حصہ میں سطح طواف کا رقبہ اور سطح کو پیدا کرنے والی منحنی کی لمبائی اور جھول کے تعلق پر غور کیا جائے گا۔زیادہ پیچیدہ سطحوں پر بعد کے باب میں غور کیا جائے گا۔

\جزوحصہء{بنیادی کلیہ}
فرض کریں ہم غیر منفی تفاعل \عددی{y=f(x),\, a\le x\le b} کو \عددی{x}محور کے گرد گھما کر پیدا سطح طواف کا سطحی رقبہ جاننا چاہتے ہیں۔ ہم \عددی{[a,b]} کی خانہ بندی کر کے نقاط خانہ بندی استعمال کرتے ہوئے ترسیم کو چھوٹے حصوں میں تقسیم کرتے ہیں۔ شکل میں نمائندہ حصہ \عددی{PQ}  اور اس کی پیدا کردہ پٹی دکھائی گئی ہے۔

قوس \عددی{PQ} محور \عددی{x} کے گرد گھومتے ہوئے مخروط سطح پیدا کرتی ہے۔محور \عددی{x} اس مخروط سطح کا محور ہو گا۔ مخروط کے ایسے حصے کو \اصطلاح{مخروط مقطوع}\فرہنگ{مخروط مقطوع}\حاشیہب{frustum}\فرہنگ{frustum} کہتے ہیں۔ مخروط مقطوع کا سطحی رقبہ، {PQ} کی پیدا کردہ پٹی کے رقبہ کا تخمین ہو گا۔

مخروط مقطوع کا سطحی رقبہ \عددی{2\pi} ضرب دونوں سروں کے رداس کا اوسط ضرب ترچھا قد کے برابر ہو گا۔
\begin{align*}
\text{\RL{مخروط مقطوع کا سطحی رقبہ}}=2\pi\cdot \frac{r_1+r_2}{2}\cdot L=\pi(r_1+r_2)L
\end{align*} 
قطع \عددی{PQ} کے پیدا کردہ مخروط مقطوع کے لئے اس سے درج ذیل حاصل ہوتا ہے۔
\begin{align*}
\text{\RL{مخروط مقطوع کا سطحی رقبہ}}=
\pi(f(x_{k-1})+f(x_k))\sqrt{(\Delta x_k)^2+(\Delta y_k)^2}
\end{align*}
پوری سطح طواف کا رقبہ تخمیناً ایسے تمام چھوٹے قطعات کی پیدا کردہ مخروط مقطوع کے سطحی رقبوں کا مجموعہ کے ہو گا۔
\begin{align}\label{مساوات_تکمل_استعمال_سطحی_رقبہ_الف}
\sum_{k=1}^n \pi(f(x_{k-1})+f(x_k))\sqrt{(\Delta x_k)^2+(\Delta y_k)^2}
\end{align}
ہم توقع کرتے ہیں کہ \عددی{[a,b]} کی زیادہ باریک خانہ بندی سے تخمین بہتر ہو گی۔ ہم دکھانا چاہتے ہیں کہ خانہ بندی کا معیار صفر تک پہنچنے سے مساوات \حوالہ{مساوات_تکمل_استعمال_سطحی_رقبہ_الف} میں دیا گیا مجموعہ قابل حل حد دیگا۔ 

یہ دکھانے کی خاطر ہم مساوات \حوالہ{مساوات_تکمل_استعمال_سطحی_رقبہ_الف} کو وقفہ \عددی{[a,b]} پر کسی تفاعل  کا ریمان مجموعہ لکھتے ہیں۔لمبائی قوس کے حصول کی طرح ہم تفرقات کے مسئلہ اوسط قیمت کی طرف دیکھتے ہیں۔

اگر \عددی{f} ہموار ہو تب مسئلہ اوسط قیمت کے تحت \عددی{P} اور \عددی{Q} کے بیچ ایسا نقطہ \عددی{(c_k,f(c_k))} ضرور پایا جائے گا جہاں مماس قطع \عددی{PQ} کے متوازی ہو گا۔اس نقطہ پر درج ذیل ہو گا۔
\begin{align*}
f'(c_k)&=\frac{\Delta y_k}{\Delta x_k}\\
\Delta y_k&=f'(c_k)\Delta x_k
\end{align*} 
مساوات \حوالہ{مساوات_تکمل_استعمال_سطحی_رقبہ_الف} میں درج بالا \عددی{\Delta y_k} پر کرتے ہیں۔
\begin{multline}\label{مساوات_تکمل_استعمال_سطحی_رقبہ_ب}
\sum_{k=1}^n \pi(f(x_{k-1})+f(x_k))\sqrt{(\Delta x_k)^2+(\Delta y_k)^2}\\
=\sum_{k=1}^n\pi(f(x_{k-1})+f(x_k))\sqrt{1+(f'(c_k))^2}\Delta x_k
\end{multline}
اب یہاں ایک بری خبر اور ایک اچھی خبر ہے۔

بری خبر یہ ہے کہ مساوات \حوالہ{مساوات_تکمل_استعمال_سطحی_رقبہ_ب} میں  \عددی{x_{k-1}}، \عددی{x_k} اور \عددی{c_k} ایک دوسرے سے مختلف ہیں اور انہیں ایک دوسرے جیسا کسی صورت نہیں بنایا جا سکتا ہے لہٰذا مساوات \حوالہ{مساوات_تکمل_استعمال_سطحی_رقبہ_ب} میں دیا گیا مجموعہ  ریمان مجموعہ نہیں ہے۔ اچھی خبر یہ ہے کہ اس سے کوئی فرق نہیں پڑتا ہے۔ اعلٰی احصاء کا مسئلہ بلس کہتا ہے کہ وقفہ \عددی{[a,b]} کی خانہ بندی کا معیار صفر تک پہچانے سے مساوات \حوالہ{مساوات_تکمل_استعمال_سطحی_رقبہ_ب} میں دیا گیا مجموعہ درج ذیل کو مرکوز ہو گا
\begin{align*}
\int_a^b2\pi f(x)\sqrt{1+(f'(x))^2}\dif x
\end{align*} 
جو ہم چاہتے ہیں۔یوں  \عددی{a} تا \عددی{b} تفاعل \عددی{f} کی ترسیم کو \عددی{x} محور کے گرد گھمانے سے حاصل سطح طواف کے رقبہ کی تعریف ہم اسی تکمل کو لیتے ہیں۔

\ابتدا{تعریف}\موٹا{محور \عددی{x} کے گرد سطح طواف کے رقبہ کا کلیہ}\\
اگر \عددی{[a,b]} پر تفاعل \عددی{f(x)\ge 0} ہموار ہو تب تفاعل \عددی{y=f(x)} کو \عددی{x} محور کے گرد گھمانے سے حاصل سطح طواف کا رقبہ درج ذیل ہو گا۔
\begin{align}\label{مساوات_تکمل_استعمال_سطحی_رقبہ_پ}                                
S=\int_a^b2\pi y\sqrt{1+\big(\frac{\dif y}{\dif x}\big)^2}\dif x=\int_a^b2\pi f(x)\sqrt{1+(f'(x))^2}\dif x
\end{align}
\انتہا{تعریف}
%=========================

مساوات \حوالہ{مساوات_تکمل_استعمال_سطحی_رقبہ_پ} میں جذر وہی ہے جو پیداکار منحنی کی لمبائی قوس کے کلیہ میں پایا جاتا ہے۔

\ابتدا{مثال}
محور \عددی{x} کے گرد منحنی \عددی{y=2\sqrt{x},\, 1\le x\le 2} گھما کر سطح طواف پیدا کیا جاتا ہے۔اس سطح طواف کا رقبہ تلاش کریں۔

حل:\quad
ہم درج ذیل لیتے ہوئے
\begin{align*}
a&=1,\, b=2,\, y=2\sqrt{x},\, \frac{\dif y}{\dif x}=\frac{1}{\sqrt{2}}\\
\sqrt{1+\big(\frac{\dif y}{\dif x}\big)^2}&=\sqrt{1+\big(\frac{1}{\sqrt{x}}\big)^2}\\
&=\sqrt{1+\frac{1}{x}}=\sqrt{\frac{x+1}{x}}=\frac{\sqrt{x+1}}{\sqrt{x}}
\end{align*}
 مساوات \حوالہ{مساوات_تکمل_استعمال_سطحی_رقبہ_پ} استعمال کرتے ہیں۔
\begin{align*}
S&=\int_1^22\pi \cdot2\sqrt{x}\frac{\sqrt{x+1}}{\sqrt{x}}\dif x=4\pi\int_1^2\sqrt{x+1}\dif x\\
&=\left.4\pi\cdot\frac{2}{3}(x+1)^{3/2}\right]_1^2=\frac{8\pi}{3}(3\sqrt{3}-2\sqrt{2})
\end{align*}

\انتہا{مثال}
%=========================

\جزوحصہء{محور \عددی{y} کے گرد سطح طواف}
محور \عددی{y} کے گرد سطح طواف کے لئے ہم مساوات \حوالہ{مساوات_تکمل_استعمال_سطحی_رقبہ_پ} میں \عددی{x} اور \عددی{y} کی جگہیں تبدیل کرتے  ہیں۔

\موٹا{محور \عددی{y} کے گرد سطح طواف کے رقبہ کا کلیہ}\\
اگر \عددی{[c,d]} پر \عددی{x=g(y)\ge 0} ہموار ہو تب منحنی \عددی{x=g(y)} کو محور \عددی{y} کے گرد گھمانے سے حاصل سطح طواف کا رقبہ درج ذیل ہو گا۔
\begin{align}\label{مساوات_تکمل_استعمال_سطحی_رقبہ_ت}
S=\int_c^d2\pi x\sqrt{1+\big(\frac{\dif x}{\dif y}\big)^2}\dif y=\int_c^d2\pi g(y)\sqrt{1+(g'(y))^2}\dif y
\end{align}

\ابتدا{مثال}
لکیری قطع \عددی{x=1-y,\, 0\le y\le 1} کو  محور \عددی{y} کے گرد گھما کر مخروط حاصل کیا جاتا ہے۔ اس کا رقبہ پہلو تلاش کریں۔

حل:\quad
اس رقبہ کو جیومیٹری سے حاصل کیا جا سکتا ہے۔
\begin{align*}
\text{\RL{رقبہ پہلو}}=\frac{\text{\RL{قاعدے کا محیط}}}{2}\times \text{\RL{ترچھا قد}}=\pi \sqrt{2}
\end{align*}
آئیں درج ذیل لے کر  
\begin{align*}
c&=0,\, d=1,\, x=1-y,\, \frac{\dif x}{\dif y}=-1\\
\sqrt{1+\big(\frac{\dif x}{\dif y}\big)^2}&=\sqrt{1+(-1)^2}=\sqrt{2}
\end{align*}
مساوات \حوالہ{مساوات_تکمل_استعمال_سطحی_رقبہ_ت} سے اس رقبہ کا حاصل کریں۔
\begin{align*}
S&=\int_c^d2\pi x\sqrt{1+\big(\frac{\dif x}{\dif y}\big)^2}\dif y=\int_0^12\pi (1-y)\sqrt{2}\dif y\\
&=2\pi \sqrt{2}\left[y-\frac{y^2}{2}\right]_0^1=2\pi \sqrt{2}\big(1-\frac{1}{2}\big)=\pi \sqrt{2}
\end{align*}
دونوں نتائج ایک جیسے ہیں جیسا کہ ہونا چاہیے۔
\انتہا{مثال}
%========================

\جزوحصہء{مختصر تفریقی روپ}
درج ذیل مساواتوں
\begin{align*}
S=\int_a^b2\pi y\sqrt{1+\big(\frac{\dif y}{\dif x}\big)^2}\dif x\quad \text{اور}
\quad S=\int_c^d2\pi x \sqrt{1+\big(\frac{\dif x}{\dif y}\big)^2}\dif y
\end{align*}
کو عموماً تفریقی لمبائی قوس \عددی{\dif s=\sqrt{\dif x^2+\dif y^2}} کی صورت میں لکھا جاتا ہے:
\begin{align*}
S=\int_a^b2\pi y\dif s\quad \text{اور}\quad S=\int_c^d2\pi x\dif s
\end{align*}
بایاں مساوات میں \عددی{x} محور سے قطع \عددی{\dif s} تک فاصلہ \عددی{y} ہے۔ دایاں مساوات میں \عددی{y} محور سے قطع \عددی{\dif s} کا فاصلہ \عددی{x} ہے۔ان دونوں کلیوں کو
\begin{align*}
S=\int 2\pi(\text{\RL{رداس}})(\text{\RL{چوڑائی پٹی}})= \int 2\pi \rho \dif s
\end{align*}
لکھا جا سکتا ہے جہاں رکن لمبائی قوس \عددی{\dif s} تک محور طواف سے فاصلہ \عددی{\rho} ہے۔

\موٹا{مختصر تفریقی روپ}\\
\begin{align*}
S=\int 2\pi \rho \dif s
\end{align*}
کسی مخصوص مسئلے میں آپ رکن لمبائی قوس \عددی{\dif s} اور رداس \عددی{\rho} کو کسی مشترکہ متغیر کی صورت میں لکھ کر تکمل کے حدود بھی اسی متغیر کی روپ میں مہیا کریں گے۔ 

\ابتدا{مثال}
منحنی \عددی{y=x^3,\, 0\le x\le \tfrac{1}{2}} کو محور \عددی{x} کے گرد گھما کر سطح طواف پیدا کیا جاتا ہے۔ اس کا سطحی رقبہ معلوم کریں۔

حل:\quad
ہم مختصر تفریقی روپ سے شروع کرتے ہیں۔
\begin{align*}
S&=\int 2\pi \rho \dif s\\
&=\int 2\pi y\dif s\\
&=\int 2\pi y\sqrt{\dif x^2+\dif y^2}&&\dif s=\sqrt{\dif x^2+\dif y^2}
\end{align*} 
ہم نے یہاں فیصلہ کرنا ہو گا کہ آیا \عددی{\dif s} کو \عددی{\dif x} یا \عددی{\dif y} کی روپ میں لکھیں۔منحنی کی مساوات \عددی{y=x^3} سے \عددی{\dif y} کو \عددی{\dif x} کی صورت میں لکھنا زیادہ آسان ہے لہٰذا ہم درج ذیل استعمال کریں گے۔
\begin{align*}
y=x^3,\, \dif y=3x^2\dif x,\, \sqrt{\dif x^2+\dif y^2}=\sqrt{\dif x^2+(3x^2\dif x)^2}=\sqrt{1+9x^4}\dif x
\end{align*}
انہیں استعمال کرتے ہوئے تکمل کا متغیر \عددی{x} ہو گا۔
\begin{align*}
S&=\int_{x=0}^{x=1/2}2\pi y\sqrt{\dif x^2+\dif y^2}\\
&=\int_0^{1/2}2\pi x^3\sqrt{1+9x^4}\dif x\\
&=\left.2\pi(\tfrac{1}{36})(\tfrac{2}{3})(1+9x^4)^{3/2}\right]_0^{1/2}\\
&=\tfrac{\pi}{27}[(1+\tfrac{9}{16})^{3/2}-1]\\
&=\tfrac{\pi}{27}[(\tfrac{25}{16})^{3/2}-1]\\
&=\tfrac{\pi}{27}(\tfrac{125}{64}-1)\\
&=\tfrac{61\pi}{1728}
\end{align*}
\انتہا{مثال}
%=====================

\حصہء{سوالات}

