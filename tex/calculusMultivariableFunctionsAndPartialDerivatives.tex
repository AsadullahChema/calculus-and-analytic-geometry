\باب{کثیر المتغیر تفاعل اور جزوی تفرقات}
\موٹا{جائزہ}\\
سائنس میں دو یا دو سے زائد غیر تابع  متغیرات کے تفاعل    ایک متغیر کے تفاعل سے زیادہ کثرت سے پائے جاتے ہیں اور ان کی علم   احصاء  زیادہ  عمدہ   ہوتی  ہے۔زیادہ متغیرات ایک دوسرے پر زیادہ طریقوں سے اثر انداز ہو سکتے ہیں جس کی بنا ان کے تفرقات    مختلف اور زیادہ دلچسپ  صورتیں اختیار  کر سکتے ہیں۔ ان کے تکملات  زیادہ اقسام کے عملی  مسائل میں کام آتے ہیں۔ احتمال،  سیالی حرکیات، اور برقیات، وغیرہ،    پر غور کے دوران  ایک سے زائد متغیرات  کے تفاعل قدرتی طور پر رونما ہوتے ہیں۔ان تفاعل کی  ریاضیات، سائنس کی عظیم  کامیابیوں میں  سے ایک ہے۔

\حصہ{کثیر متغیرات کے تفاعل}
کئی تفاعل ایک سے زائد متغیرات کے تابع  ہوتے ہیں۔دائری نلکی کا حجم،  اس    کے رداس اور قد سے،   تفاعل  \عددی{H=\pi r^2h}   دیتا ہے۔ مستوی \عددی{xy} میں نقطہ \عددی{N(x,y)} کے دو محدد سے،   قطع مکافی   \عددی{z=x^2+y^2}  کا قد  تفاعل \عددی{f(x,y)=x^2+y^2} دیتا ہے۔اس حصہ میں ہم ایک سے زیادہ متغیرات کے تابع تفاعل متعارف کرتے ہیں  اور ان کو ترسیم کرنے کے طریقوں پر غور کرتے ہیں۔

\جزوحصہء{تفاعل اور متغیرات} 
کثیر غیر تابع حقیقی متغیرات  کے حقیقی قیمت تفاعل کی تعریف  بالکل واحد متغیر کے تفاعل کی طرح کی جاتی ہے۔ان کے وقفے  حقیقی   (تین، چار، وغیرہ)    اعداد کے  مرتب جوڑی  کے سلسلے ہوں گے اور ان  کی  سعت  ، اس طرح  کے حقیقی اعداد کے سلسلے ہوں گے جن کے ساتھ ہم کام کرتے آ رہے ہیں۔

\ابتدا{تعریفات}
فرض کریں \عددی{n} عدد حقیقی اعداد \عددی{x_1,x_2,\cdots,x_n} کا سلسلہ \عددی{D} ہے۔  تب \عددی{D} پر \اصطلاح{حقیقی قیمت تفاعل}\فرہنگ{تفاعل!حقیقی قیمت}\حاشیہب{real valued function}\فرہنگ{function!real valued} \عددی{f} سے مراد وہ قاعدہ ہے جو \عددی{D} کے ہر رکن کو حقیقی عدد
\begin{align*}
w=f(x_1,x_2,\cdots,x_n)
\end{align*}
مختص کرتا ہو۔سلسلہ \عددی{D} اس تفاعل کا \اصطلاح{دائرہ کار}\فرہنگ{دائرہ کار}\حاشیہب{domain}\فرہنگ{domain} ہو گا۔  تفاعل  \عددی{f} کی  \عددی{w} قیمتوں کا سلسلہ  \عددی{f} کی \اصطلاح{سعت}\فرہنگ{سعت}\حاشیہب{range}\فرہنگ{range} ہو گی۔ علامت \عددی{w} تفاعل \عددی{f} کا \اصطلاح{تابع متغیر}\فرہنگ{متغیر!تابع}\حاشیہب{dependent variable}\فرہنگ{variable!dependent} ہو گا اور \عددی{f} کو  \عددی{n}  \اصطلاح{غیر تابع متغیرات}\فرہنگ{متغیر!غیر تابع}\حاشیہب{independent variable}\فرہنگ{variable!independent} \عددی{x_1} تا \عددی{x_n}   کا تفاعل کہتے ہیں۔ ہم  ان \عددی{x} کو تفاعل کے \اصطلاح{داخلی متغیرات}\فرہنگ{متغیر!داخلی}\حاشیہب{input variable}\فرہنگ{variable!input} اور \عددی{w} کو تفاعل کا \اصطلاح{خارجی متغیر}\فرہنگ{متغیر!خارجی}\حاشیہب{output variable}\فرہنگ{variable!output}  بھی کہتے ہیں۔
\انتہا{تعریفات}
%==================

اگر \عددی{f} دو غیر تابع متغیرات کا تفاعل ہو تب عموماً ہم  ان غیر تابع متغیرات کو \عددی{x} اور \عددی{y} کہتے ہیں اور \عددی{f} کے دائرہ کار  کو مستوی \عددی{xy} میں ایک خطہ تصور کرتے ہیں۔ اگر \عددی{f} تین غیر تابع متغیرات کا تفاعل ہو تب ہم  ان متغیرات کو \عددی{x}، \عددی{y} اور \عددی{z} کہتے ہیں اور تفاعل کے  دائرہ کار کو فضا میں ایک خطہ تصور کرتے ہیں۔

عملی استعمال میں ہم  وہ حروف استعمال کرتے ہیں جو ہمیں ان چیزوں  کی یاد  دلا سکیں جن کے لئے یہ متغیرات استعمال  کیے گئے ہوں۔ یہ  کہنے  کی خاطر کہ دائری نلکی کا حجم اس کے رداس \عددی{r}  اور قد \عددی{h}   کا تفاعل ہو گا، ہم \عددی{H=f(r,h)} لکھ سکتے ہیں۔بالخصوص  ہم \عددی{f(r,h)} کی جگہ وہ کلیہ استعمال کر سکتے ہیں جو \عددی{r} اور \عددی{h} کی قیمتوں سے \عددی{H} کی  قیمت دیتا ہو، یعنی ہم  \عددی{H=\pi r^2h} لکھ سکتے ہیں۔دونوں صورتوں میں \عددی{r} اور \عددی{h} غیر تابع متغیرات ہوں گے اور \عددی{H} تابع متغیر ہو گا۔

ہمیشہ کی طرح،ہم  تفاعل کی تعریفی کلیہ میں غیر تابع متغیرات کی قیمتیں پر کر  کے مطابقتی تابع متغیر کی قیمت حاصل کرتے ہیں۔

\ابتدا{مثال}
نقطہ \عددی{(3,0,4)} پر تفاعل \عددی{f(x,y,z)=\sqrt{x^2+y^2+z^2}} کی قیمت درج ذیل ہو گی۔
\begin{align*}
f(3,0,4)=\sqrt{(3)^2+(0)^2+(4)^2}=\sqrt{25}=5
\end{align*}
\انتہا{مثال}
%===============
\جزوحصہء{وقفے}
ایک سے زیادہ متغیرات کے تفاعل  کی تعریف  میں، ہمیشہ کی طرح، ہم  ان مداخل کو  شامل نہیں  کرتے ہیں  جو مخلوط اعداد دیتے ہوں یا جن کی وجہ سے   تقسیم صفر  کا عمل  پیدا ہوتا ہو۔یوں \عددی{f(x,y)=\sqrt{y-x^2}} میں \عددی{y} کی قیمت \عددی{x^2} کی قیمت سے کم نہیں ہو سکتی ہے اور \عددی{f(x,y)=\tfrac{1}{xy}} میں \عددی{xy} کی قیمت صفر نہیں ہو سکتی ہے۔ان شرائط کو مطمئن کرتے ہوئے، تفاعل کے  دائرہ کار سے مراد  وہ بڑے سے بڑا سلسلہ ہو گا جس پر تفاعل کا  تعریفی قاعدہ حقیقی اعداد  پیدا کرتا ہو۔ 

\ابتدا{مثال}\ترچھا{دو متغیرات کے تفاعل}\\
\begin{center}
\renewcommand{\arraystretch}{1.2} 
\begin{tabular}{LLL}
\multicolumn{1}{C}{\text{\RL{تفاعل}}}&\text{\RL{دائرہ کار}}&\multicolumn{1}{C}{\text{\RL{سعت}}}\\
\midrule
w=\sqrt{y-x^2}&y\ge x^2&[0,\infty)\\
w=\tfrac{1}{xy}&xy\ne 0&(-\infty,0)\cup(0,\infty)\\
w=\sin xy&\text{\RL{پورا مستوی}}&[-1,1]
\end{tabular}
\end{center}
\انتہا{مثال}
%===============
\ابتدا{مثال}\ترچھا{تین متغیرات کے  تفاعل}\\
\begin{center}
\renewcommand{\arraystretch}{1.2} 
\begin{tabular}{LLL}
\multicolumn{1}{C}{\text{\RL{تفاعل}}}&\multicolumn{1}{C}{\text{\RL{دائرہ کار}}}&\multicolumn{1}{C}{\text{\RL{سعت}}}\\
\midrule
w=\sqrt{x^2+y^2+z^2}&\text{\RL{پوری فضا}}&[0,\infty)\\
w=\dfrac{1}{x^2+y^2+z^2}&(x,y,z)\ne (0,0,0)&(0,\infty)\\
w=xy\ln z&\text{\RL{نصف فضا z>0}}&(-\infty,\infty)
\end{tabular}
\end{center}
\انتہا{مثال}
%===========
بالکل حقیقی لکیر کے وقفوں  پر معین تفاعل کے دائرہ کار کی طرح، مستوی  کے حصوں پر معین تفاعل کے دائرہ کار کے اندرونی نقطے اور  سرحدی نقطے  ہو سکتے ہیں۔

\ابتدا{تعریفات}
مستوی \عددی{xy} میں خطہ (سلسلہ) \عددی{R}  میں نقطہ \عددی{(x_0,y_0)}  تب \عددی{R} کا \اصطلاح{اندرونی نقطہ}\فرہنگ{نقطہ!اندرونی}\حاشیہب{interior point}\فرہنگ{point!interior} ہو گا جب  یہ اس قرص کا مرکز  ہو جو مکمل طور پر \عددی{R} میں پایا جاتا ہو۔  نقطہ \عددی{(x_0,y_0)} تب \عددی{R} کا  \اصطلاح{سرحدی نقطہ}\فرہنگ{نقطہ!سرحدی}\حاشیہب{boundary point}\فرہنگ{point!boundary} ہو گا جب ہر اس  قرص  میں، جس کا مرکز \عددی{(x_0,y_0)} ہو ،  \عددی{R} کے بیرونی  اور \عددی{R} کے اندرونی نقطے پائے جاتے ہوں۔(ضروری نہیں کہ سرحدی نقطہ ازخود \عددی{R} میں شامل  ہو۔ )

ایک خطہ کے اندرونی نقطے، بطور ایک سلسلہ، اس خطہ  کی\اصطلاح{ اندرون}\فرہنگ{اندرون}\حاشیہب{interior}\فرہنگ{interior} ہوں گے۔ اس خطہ کے سرحدی نقطے اس کی \اصطلاح{سرحد}\فرہنگ{سرحد}\حاشیہب{boundary}\فرہنگ{boundary}  ہیں۔ایسا خطہ  جو مکمل طور پر اندرونی نقطوں پر مشتمل ہو \اصطلاح{کھلا}\فرہنگ{کھلا}\حاشیہب{open}\فرہنگ{open} خطہ کہلاتا ہے۔ ایسا خطہ جس میں  اس کے تمام سرحدی نقطے شامل ہوں \اصطلاح{بند}\فرہنگ{بند}\حاشیہب{closed}\فرہنگ{closed} خطہ کہلاتا ہے۔ 
\انتہا{تعریفات}
%===================

حقیقی اعداد کے وقفوں کی طرح، مستوی میں بعض خطے نا کھلا اور نا ہی بند ہوتے ہیں۔ شکل کے کھلا قرص   میں چند،   نا کہ  تمام،    سرحدی نقطے شامل کرنے سے ایسا خطہ حاصل ہو گا جو نا کھلا ہو گا اور نا ہی بند ہو گا۔اس میں شامل سرحدی نقطے اس کو کھلا وقفہ بننے  سے روکتے  ہیں جبکہ اس میں نا  شامل سرحدی نقطے اس کو  بند  خطہ بننے سے روکتے ہیں۔

\ابتدا{تعریف}
مستوی میں  مقررہ رداس کے قرص  میں پائے جانے والا خطہ  \اصطلاح{محدود}\فرہنگ{محدود}\حاشیہب{bonded}\فرہنگ{bounded} ہو گا۔  ایسا خطہ جو محدود نا ہو \اصطلاح{غیر محدود}\فرہنگ{غیر محدود}\حاشیہب{unbounded}\فرہنگ{unbounded}  ہو گا۔
\انتہا{تعریف}
%================

\ابتدا{مثال}
\begin{description}
\item{مستوی میں محدود سلسلے:}
خطی قطعات؛ مثلثیں؛ مثلثوں کی اندرون؛  مستطیلیں؛ ا قراص۔
\item{مستوی میں غیر محدود سلسلے:}
خطوط،؛ محددی محور؛  لا متناہی وقفہ پر معین تفاعل کی ترسیم؛   ربعات،  نصف مستوی؛ مستوی از خود۔
\end{description}
\انتہا{مثال}
%================
\ابتدا{مثال}
تفاعل \عددی{f(x,y)=\sqrt{y-x^2}} کا دائرہ کار بند اور غیر محدود ہے۔ قطع مکافی \عددی{y=x^2} اس دائرہ کار کی سرحد ہے۔ قطع مکافی سے اوپر نقطے دائرہ کار کی اندرون ہیں۔
\انتہا{مثال}
%=================

فضا میں  اندرون، سرحد ، کھلا، بند، محدود  اور غیر محدود کی تعریفیں عین مستوی میں انہیں تعریفوں کی طرح ہیں۔ اضافی بعد کی بنا ہم قرص کی  بجائے گیند لیتے ہیں۔ \اصطلاح{بند گیند}\فرہنگ{بند!گیند}\حاشیہب{closed ball}\فرہنگ{closed!ball} میں کرہ کی اندرونی نقطوں کے ساتھ  گیند بھی شامل ہو گا۔ \اصطلاح{کھلا گیند}\فرہنگ{کھلا!گیند}\حاشیہب{open ball}\فرہنگ{open!ball} میں گیند کی اندرونی نقطے شامل ہوں گے جبکہ گیند از خود اس میں شامل نہیں ہو گا۔ 

\ابتدا{تعریفات}
فضا میں خطہ \عددی{D} میں نقطہ \عددی{(x_0,y_0,z_0)}  اس صورت \عددی{D} کا \اصطلاح{اندرونی نقطہ}\فرہنگ{اندرونی!نقطہ}\حاشیہب{interior point}\فرہنگ{interior!point} ہو گا جب  یہ نقطہ  ایسے گیند کا مرکز ہو جو مکمل طور پر \عددی{D} میں پایا جاتا ہو۔اگر ہر  گیند، جس کا مرکز   \عددی{(x_0,y_0,z_0) }  ہو، میں شامل نقطوں میں کچھ  نقطے     \عددی{D} کے اندرونی    اور کچھ  اس کے بیرونی نقطے   ہوں تب یہ نقطہ \عددی{D} کا \اصطلاح{سرحدی نقطہ}\فرہنگ{سرحدی!نقطہ}\حاشیہب{boundary point}\فرہنگ{boundary!point} ہو گا۔خطہ \عددی{D} کے اندرونی نقطوں کا  سلسلہ  \عددی{D} کا  \اصطلاح{اندرون}\فرہنگ{اندرون}\حاشیہب{interior}\فرہنگ{interior} ہو گا۔ خطہ \عددی{D} کے سرحدی نقطوں کا سلسلہ \عددی{D} کا \اصطلاح{سرحد}\فرہنگ{سرحد}\حاشیہب{boundary}\فرہنگ{boundary} ہو گا۔

ایک ایسا خطہ جو صرف اندرونی نقطوں پر مشتمل ہو  \اصطلاح{کھلا}\فرہنگ{کھلا}\حاشیہب{open}\فرہنگ{open} خطہ کہلائے  گا۔ ایک  خطہ جس میں خطے کا پورا سرحد شامل ہو\اصطلاح{ بند}\فرہنگ{بند}\حاشیہب{closed}\فرہنگ{closed} خطہ کہلائے گا۔
\انتہا{تعریفات}
%===============

\ابتدا{مثال}
\begin{description}
\item{فضا میں کھلا سلسلے}
کھلا گیند؛کھلا نصف فضا \عددی{z>0}؛ ربع اول (بغیر تحدیدی  سطحیں)  ؛ فضا  ا زخود
\item{فضا میں بند سلسلے}  
خطوط؛ مستوی؛ بند گیند؛ بند نصف فضا \عددی{z\ge 0}؛ ربع اول بمع اس کے تحدیدی  سطحیں؛ فضا از خود
\item{نا کھلا اور نا بند}
بند گیند جس میں تحدیدی کرہ کا کچھ حصہ شامل نہ ہو؛ ٹھوس مربع جس میں ایک  تحدیدی سطح یا کنارہ   یا کونا شامل نہ ہو 
\end{description}
\انتہا{مثال}
%=========

\جزوحصہء{دو متغیرات کے تفاعل کی ترسیمات اور  ہموار منحنیات}

