\باب{سمتی  قیمت تفاعل اور فضا میں حرکت}

\موٹا{سر سری جائزہ}\quad
جب کوئی جسم فضا میں حرکت کرتا ہو،   مساوات   \عددی{x=f(t)}، \عددی{y=g(t)} اور \عددی{z=h(t)}  جو اس جسم کے محدد  کو بطور وقت کا تفاعل  دیتی ہیں،  اس جسم کی راہ اور حرکت کی مقدار معلوم مساوات ہوں گی۔ سمتیہ   علامتیت  کی مدد سے ہم انہیں ایک مساوات \عددی{\kvec{r}(t)=f(t)\ai+g(t)\aj+h(t)\ak} کی صورت میں لکھ سکتے ہیں جو اس جسم کا مقام بطور وقت کا سمتی تفاعل دیتی ہے۔

اس باب میں  ہم احصاء استعمال کرتے ہوئے حرکت پذیر اجسام کی راہ، سمتی رفتار اور اسراع پر غور کریں گے۔ ہم  گولا،  سیارہ  اور مصنوعی سیارہ کی راہ اور حرکت کے عمومی سوالات کے جوابات جان سکے گے۔آخر حصہ میں ہم نیوٹن کے قوانین اور    تجاذب کی مدد سے سیاروں کی مدار کے  قوانین کپلر دریافت کریں گے۔ 


\حصہ{سمتی قیمت تفاعل اور فضائی منحنیات}
فضا میں متحرک ذرہ    کی حرکت جاننے کی خاطر ہم  مبدا سے  اس ذرہ  تک سمتیہ \عددی{\kvec{r}} لے کر \عددی{\kvec{r}} میں تبدیلی پر غور کرتے ہیں۔اگر اس ذرہ  کے محدد مقام  وقت کے ساتھ دو بار قابل تفرق ہوں،  تب  \عددی{\kvec{r}} بھی ایسا ہو گا، اور ہم کسی بھی لمحہ پر وقت کے لحاظ سے  \عددی{\kvec{r}}   کے  تفرق لے کر اس ذرہ  کی سمتی رفتار اور اسراع جان سکتے ہیں۔اگر ہمیں اس ذرہ  کی سمتیہ  سمتی رفتار یا سمتیہ  اسراع   بطور  وقت کے استمراری تفاعل معلوم ہو اور ہمیں ذرے کی ابتدائی  مقام اور سمتیہ رفتار کے بارے میں معقول معلومات ہو، تب ہم تکمل کی مدد سے، وقت کا تفاعل  \عددی{\kvec{r}} جان سکتے ہیں۔

\جزوحصہء{تعریف}
جب وقفہ \عددی{I} کے دوران ایک ذرہ فضا میں حرکت  کرتا ہو، ہم اس ذرہ کے محدد جو وقت کے تفاعل ہو گے کی تعریف درج ذیل کرتے ہیں۔
\begin{align}\label{مساوات_سمتی_تفاعل_مقدار_معلوم_راہ}
x=f(t),\quad y=g(t),\quad z=h(t),\quad t\in I
\end{align}
نقاط \عددی{(x,y,z)=(f(t),g(t),h(t)),\,t\in I}  فضا میں وہ  \اصطلاح{منحنی} دیتے ہیں جنہیں ہم اس ذرے کی \اصطلاح{راہ}\فرہنگ{راہ}\حاشیہب{path}\فرہنگ{path} کہتے ہیں۔ مساوات \حوالہ{مساوات_سمتی_تفاعل_مقدار_معلوم_راہ} اس منحنی  کی \اصطلاح{مقدار معلوم روپ }\فرہنگ{مقدار معلوم!روپ}  ہے۔ مبدا سے ذرے  کے \اصطلاح{ مقام}  \عددی{N(f(t),g(t),h(t))}  تک لمحہ \عددی{t} پر   سمتیہ 
\begin{align*}
\kvec{r}(t)=\krightharpoonup{ON}=f(t)\ai+g(t)\aj+h(t)\ak
\end{align*} 
اس ذرے کا\اصطلاح{   تعین گر سمتیہ}\فرہنگ{تعین گر!سمتیہ}\حاشیہب{position vector}\فرہنگ{vector!position}  ہے۔تفاعل \عددی{f}، \عددی{g} اور \عددی{h}  تعین گر سمتیہ کے\اصطلاح{ اجزاء}  ہیں۔
 ذرے کی راہ سے مراد وقفہ \عددی{t} کے دوران   \عددی{\kvec{r}}  کی  پیداکردہ منحنی ہے۔


مساوات \حوالہ{مساوات_سمتی_تفاعل_مقدار_معلوم_راہ}  سمتیہ \عددی{\kvec{r}} کی تعریف وقفہ \عددی{I} پر  حقیقی متغیر \عددی{t} کی صورت میں دیتی ہے۔ زیادہ عمومی طور پر  دائرہ کار،   سلسلہ \عددی{D}،  پر  \اصطلاح{سمتی تفاعل}\فرہنگ{سمتی!تفاعل}\حاشیہب{vector function}\فرہنگ{vector!function} یا  \اصطلاح{سمتی  قیمت تفاعل}\فرہنگ{سمتی قیمت تفاعل}\حاشیہب{vector-valued function}\فرہنگ{vector-valued!function}   سے مراد وہ قاعدہ ہو گا   جو \عددی{D} کے  ہر رکن کو فضا میں ایک سمتیہ مختص کرتا ہو۔موجودہ استعمال میں دائرہ کار حقیقی اعداد  کے وقفوں    پر مشتمل ہوں  گے۔ بعد کے ایک باب میں دائرہ کار، مستوی یا فضا میں خطوں پر مشتمل ہوں گے   جہاں ہم  سمتی تفاعل کو سمتی میدان  کہیں گے۔

ہم حقیقی قیمت تفاعل کو \اصطلاح{غیر سمتی تفاعل}\فرہنگ{غیر سمتی! تفاعل}\حاشیہب{scalar functions}\فرہنگ{scalar!functions} کہتے ہیں تا کہ ان میں اور سمتی تفاعل میں فرق کرنا ممکن ہو۔  سمتیہ \عددی{\kvec{r}} کے اجزاء \عددی{t}  کے غیر سمتی تفاعل ہیں۔سمتی تفاعل کی تعریف  اس کے  ارکان تفاعل کی صورت میں دیتے وقت ہم فرض کرتے ہیں کہ   سمتی تفاعل کا دائرہ کار ہی  ارکان کے دائرہ کار   ہیں۔

\ابتدا{مثال}\ترچھا{پیچ دار تفاعل}\\
تمام حقیقی متغیر  \عددی{t} کے لئے سمتی تفاعل
\begin{align*}
\kvec{r}(t)=(\cos t)\ai+(\sin t)\aj+t\ak
\end{align*}
معین ہے اور  \عددی{\kvec{r}}   دائری نلکی \عددی{x^2+y^2=1} کے گرد  لپٹ کر چلتا ہے۔  سمتی تفاعل \عددی{\kvec{r}} کے \عددی{\ai} اور \عددی{\aj} اجزاء  جو \عددی{\kvec{r}} کے سر  کے  \عددی{x} اور \عددی{y} محدد ہیں   دائری نلکی  کی مساوات
\begin{align*}
x^2+y^2=(\cos t)^2+(\sin t)^2=1
\end{align*}
کو مطمئن کرتے ہیں لہٰذا \عددی{\kvec{r}} اس نلکی پر پایا جاتا ہے۔ متغیر \عددی{t} بڑھنے  \عددی{\ak}  جزو بڑھتا ہے  جس کی بنا   منحنی   اوپر بلند ہو گی۔ نلکی کے گرد ایک دائرہ \عددی{t=2\pi}  پر مکمل ہو گا۔  درج ذیل مساوات  پیچ دار تفاعل  کی مقدار معلوم  مساوات ہے، جہاں وقفہ \عددی{-\infty\le t\le \infty} ہے۔
\begin{align*}
x=\cos t,\quad y=\sin t,\quad z=t
\end{align*}
شکل میں دیگر پیچ دار تفاعل دیے گئے ہیں۔
\انتہا{مثال}
%==============

\جزوحصہء{حد اور استمرار}
