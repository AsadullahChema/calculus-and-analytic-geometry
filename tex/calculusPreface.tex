\باب{دیباچہ} 
یہ کتاب اس امید سے لکھی گئی ہے کہ ایک دن اردو زبان میں انجینئری پڑھائی جائے گی۔اس کتاب کا مکمل ہونا اس سمت میں ایک اہم قدم ہے۔ طبیعیات کے طلبہ کے لئے بھی یہ کتاب مفید ثابت ہو گی۔ 

اس کتاب کو \تحریر{Ubuntu} استعمال کرتے ہوئے \تحریر{XeLatex} میں تشکیل دیا گیا ہے۔اشکال \تحریر{pgfplots} اور \تحریر{gnuplots} کی مدد سے بنائے گئے ہیں۔

درج ذیل کتاب کو سامنے رکھتے اس کو لکھا گیا ہے

{
\begin{otherlanguage}{english}
Calculus and Analytic Geometry\\
George B. Thomas, Jr\\
Ross L. Finney
\end{otherlanguage}
}

جبکہ اردو اصطلاحات چننے میں درج ذیل لغت سے استفادہ  کیا گیا۔
{
\begin{otherlanguage}{english}
\begin{itemize}
\item
http:/\!\!/www.urduenglishdictionary.org
\item
http:/\!\!/www.nlpd.gov.pk/lughat/
\end{itemize}
\end{otherlanguage}
}
آپ سے گزارش ہے کہ اس کتاب کو زیادہ سے زیادہ طلبہ و طالبات تک پہنچائیں اور کتاب میں غلطیوں کی نشاندہی میرے  برقی پتہ پر کریں۔میری تمام کتابوں کی مکمل \تحریر{XeLatex} معلومات

{
\begin{otherlanguage}{english}
https:/\!\!/www.github.com/khalidyousafzai
\end{otherlanguage}
}

سے حاصل کی جا سکتی ہیں جنہیں آپ مکمل اختیار کے ساتھ استعمال کر سکتے ہیں۔میں امید کرتا ہوں کہ طلبہ و طالبات اس کتاب سے استفادہ ہوں گے۔
\vspace{5mm}

{\raggedleft{
خالد خان یوسفزئی

30 مارچ \سن{2020}}}


