\حصہء{سوالات}
\موٹا{مستوی منحنیات}\\
سوال \حوالہ{سوال_سمتی_تفاعل_انحنا_مروڑ_اور_عمودی_سمتیہ_الف} تا سوال \حوالہ{سوال_سمتی_تفاعل_انحنا_مروڑ_اور_عمودی_سمتیہ_ب} میں مستوی منحنیات کا \عددی{\kvec{T}}، \عددی{\kvec{N}} اور \عددی{\kappa} تلاش کریں۔

\ابتدا{سوال}\شناخت{سوال_سمتی_تفاعل_انحنا_مروڑ_اور_عمودی_سمتیہ_الف}
$\kvec{r}(t)=t\ai+(\ln \cos t)\aj,\quad -\tfrac{\pi}{2}<t<\tfrac{\pi}{2}$
\انتہا{سوال}
%================
\ابتدا{سوال}
$\kvec{r}(t)=(\ln \sec t)\ai+t\aj,\quad -\tfrac{\pi}{2}<t<\tfrac{\pi}{2}$
\انتہا{سوال}
%================
\ابتدا{سوال}
$\kvec{r}(t)=(2t+3)\ai+(5-t^2)\aj$
\انتہا{سوال}
%================
\ابتدا{سوال}\شناخت{سوال_سمتی_تفاعل_انحنا_مروڑ_اور_عمودی_سمتیہ_ب}
$\kvec{r}(t)=(\cos t+t\sin t)\ai+(\sin t-t\cos t)\aj,\quad t>0$
\انتہا{سوال}
%================

سوال \حوالہ{سوال_سمتی_تفاعل_اسراع_روپ_الف} اور سوال \حوالہ{سوال_سمتی_تفاعل_اسراع_روپ_ب} میں \عددی{\kvec{T}} اور \عددی{\kvec{N}} معلوم کیے بغیر  \عددی{\kvec{a}} کو \عددی{\kvec{a}=a_T\kvec{T}+a_N\kvec{N}} روپ میں لکھیں۔

\ابتدا{سوال}\شناخت{سوال_سمتی_تفاعل_اسراع_روپ_الف}
$\kvec{r}(t)=(2t+3)\ai+(t^2-1)\aj$
\انتہا{سوال}
%==================
\ابتدا{سوال}\شناخت{سوال_سمتی_تفاعل_اسراع_روپ_ب}
$\kvec{r}(t)=\ln (t^2+1)\ai+(t-2\tan^{-1}t)\aj$
\انتہا{سوال}
%================
\ابتدا{سوال}
مستوی \عددی{xy} میں تفاعل کی ترسیم کی انحنا کا کلیہ۔
\begin{enumerate}[a.]
\item
مستوی \عددی{xy} میں ترسیم \عددی{y=f(x)}  کی مقدار معلوم روپ \عددی{x=x,\, y=f(x)} ہے اور سمتی کلیہ \عددی{\kvec{r}(t)=x\ai+f(x)\aj} ہو گا۔ اگر \عددی{f} دو بار قابل تفرق ہو تب اس کلیہ کو استعمال کرتے ہوئے  درج ذیل دکھائیں۔
\begin{align*}
\kappa(x)=\frac{\abs{f''(x)}}{[a+(f'(x))^2]^{3/2}}
\end{align*} 
\item
جزو-ا میں \عددی{\kappa} کا کلیہ  استعمال کرتے ہوئے  \عددی{y=\ln(\cos x),\,-\tfrac{\pi}{2}<x<\tfrac{\pi}{2}} کی انحنا تلاش کریں۔اپنے جواب کا سوال \حوالہ{سوال_سمتی_تفاعل_انحنا_مروڑ_اور_عمودی_سمتیہ_الف} کے جواب کے ساتھ موازنہ کریں۔
\item
دکھائیں کہ نقطہ  تصریف پر انحنا صفر ہو گی۔ 
\end{enumerate}
\انتہا{سوال}
%=========
\ابتدا{سوال}\ترچھا{مستوی  میں مقدار معلوم   روپ میں  دی گئی منحنی کی انحنا کا کلیہ}\\
\begin{enumerate}[a.]
\item
دکھائیں کہ دو بار قابل تفرق تفاعل \عددی{x=f(t)}، \عددی{y=g(t)} پر مبنی ہموار منحنی \عددی{\kvec{r}(t)=f(t)\ai+g(t)\aj}  کی انحنا درج ذیل کلیہ دیتا ہے۔
\begin{align*}
\kappa=\frac{\abs{\dot{x}\ddot{y}-\dot{y}\ddot{x}}}{(\dot{x}^2+\dot{y}^2)^{3/2}}
\end{align*}
اس کلیہ کو استعمال کرتے ہوئے درج ذیل منحنیات کے انحنا تلاش کریں۔
\item
$\kvec{r}(t)=t\ai+(\ln\sin t)\aj,\quad 0<t<\pi$
\item
$\kvec{r}(t)=[\tan^{-1}(\sinh t)]\ai+(\ln \cosh t)\aj$
\end{enumerate}
\انتہا{سوال}
%===============
\ابتدا{سوال}\شناخت{سوال_سمتی_ترسیم_عمود_الف}\ترچھا{مستوی منحنیات کے عمود}\\
\begin{enumerate}[a.]
\item
دکھائیں کہ نقطہ \عددی{(f(t),g(t))}  پر منحنی \عددی{\kvec{r}(t)=f(t)\ai+g(t)\aj}  کے عمودی سمتیات \عددی{\kvec{n}(t)=-g'(t)\ai+f'(t)\aj} اور \عددی{-n(t)=g'(t)\ai-f'(t)\aj} ہیں۔
کسی مخصوص مستوی کا    \عددی{\kvec{N}}  تلاش کرنے کی خاطر ہم   \عددی{\kvec{n}} اور \عددی{-\kvec{n}} میں جو   مقعر رخ ہو کو منتخب کر کے اس سے اکائی سمتیہ حاصل کرتے ہیں  (شکل \حوالہ{شکل_سمتی_تفاعل_اکائی_مماسی_اور_عمودی_سمتیات})۔ یہ طریقہ استعمال کرتے ہوئے درج ذیل کا \عددی{\kvec{N}} تلاش کریں۔
\item
$\kvec{r}(t)=t\ai+e^{2t}\aj$
\item
$\kvec{r}(t)=\sqrt{4-t^2}\ai+t\aj,\quad -2\le t\le 2$
\end{enumerate}
\انتہا{سوال}
%===================
\ابتدا{سوال}
\begin{enumerate}[a.]
\item
منحنی \عددی{\kvec{r}(t)=t\ai+\tfrac{t^3}{3}\aj} کا \عددی{\kvec{N}}  وقفہ \عددی{t<0} اور وقفہ  \عددی{t>0} پر  سوال \حوالہ{سوال_سمتی_ترسیم_عمود_الف} کے کلیہ سے حاصل کریں۔
\item
جزو-ا میں منحنی کے لئے 
\begin{align*}
\kvec{N}=\frac{\dif\kvec{T}/\dif t}{\abs{\dif\kvec{T}/\dif t}},\quad t\ne 0
\end{align*}
حاصل کریں۔ کیا \عددی{t=0} پر \عددی{\kvec{N}} موجود ہے؟ اس منحنی کو ترسیم کریں اور منفی سے مثبت جانب  گزرتے  ہوئے    \عددی{\kvec{N}}  کے رویہ پر تبصرہ کریں۔
\end{enumerate}
\انتہا{سوال}
%================

\موٹا{فضائی  منحنیات}\\
سوال \حوالہ{سوال_سمتی_تفاعل_فضائی_منحنیات_خصوصیات_الف} تا سوال \حوالہ{سوال_سمتی_تفاعل_فضائی_منحنیات_خصوصیات_ب} میں فضائی منحنیات کا \عددی{\kvec{T}}، \عددی{\kvec{N}}، \عددی{\kvec{B}}، \عددی{\kappa} اور \عددی{\tau} دریافت کریں۔

\ابتدا{سوال}\شناخت{سوال_سمتی_تفاعل_فضائی_منحنیات_خصوصیات_الف}
$\kvec{r}(t)=(3\sin t)\ai+(3\cos t)\aj+4t\ak$
\انتہا{سوال}
%==================
\ابتدا{سوال}
$\kvec{r}(t)=(\cos t+t\sin t)\ai+(\sin t-t\cos t)\aj+3\ak$
\انتہا{سوال}
%================
\ابتدا{سوال}
$\kvec{r}(t)=(e^t\cos t)\ai+(e^t\sin t)\aj+2\ak$
\انتہا{سوال}
%================
\ابتدا{سوال}
$\kvec{r}(t)=(6\sin 2t)\ai+(6\cos 2t)\aj+5t\ak$
\انتہا{سوال}
%================
\ابتدا{سوال}
$\kvec{r}(t)=\tfrac{t^3}{3}\ai+\tfrac{t^2}{2}\aj,\quad t>0$
\انتہا{سوال}
%================
\ابتدا{سوال}
$\kvec{r}(t)=(\cos^3t)\ai+(\sin^3t)\aj,\quad 0<t<\tfrac{\pi}{2}$
\انتہا{سوال}
%================
\ابتدا{سوال}
$\kvec{r}(t)=t\ai+(a\cosh \tfrac{t}{a})\aj,\quad a>0$
\انتہا{سوال}
%================
\ابتدا{سوال}\شناخت{سوال_سمتی_تفاعل_فضائی_منحنیات_خصوصیات_ب}
$\kvec{r}(t)=(\cosh t)\ai-(\sinh t)\aj+t\ak$
\انتہا{سوال}
%================

سوال \حوالہ{سوال_سمتی_تفاعل_فضائی_اسراع_الف} اور سوال \حوالہ{سوال_سمتی_تفاعل_فضائی_اسراع_ب}  میں \عددی{\kvec{T}} اور \عددی{\kvec{N}} تلاش کیے بغیر  \عددی{\kvec{a}} کو \عددی{\kvec{a}=A_T\kvec{T}+a_N\kvec{N}} روپ میں لکھیں۔

\ابتدا{سوال}\شناخت{سوال_سمتی_تفاعل_فضائی_اسراع_الف}
$\kvec{r}(t)=(a\cos t)\ai+(a\sin t)\aj+bt\ak$
\انتہا{سوال}
%====================
\ابتدا{سوال}\شناخت{سوال_سمتی_تفاعل_فضائی_اسراع_ب}
$\kvec{r}(t)=(1+3t)\ai+(t-2)\aj-3t\ak$
\انتہا{سوال}
%====================


سوال \حوالہ{سوال_سمتی_تفاعل_فضائی_اسراع_نقطہ_الف} اور سوال \حوالہ{سوال_سمتی_تفاعل_فضائی_اسراع_نقطہ_ب}  میں \عددی{\kvec{T}} اور \عددی{\kvec{N}} تلاش کیے بغیر، دیے گئے  \عددی{t} پر   \عددی{\kvec{a}} کو \عددی{\kvec{a}=A_T\kvec{T}+a_N\kvec{N}} روپ میں لکھیں۔

\ابتدا{سوال}\شناخت{سوال_سمتی_تفاعل_فضائی_اسراع_نقطہ_الف}
$\kvec{r}(t)=(t+1)\ai+2t\aj+t^2\ak,\quad t=1$
\انتہا{سوال}
%================
\ابتدا{سوال}
$\kvec{r}(t)=(t\cos t)\ai+(t\sin t)\aj+t^2\ak,\quad t=0$
\انتہا{سوال}
%===============
\ابتدا{سوال}
$\kvec{r}(t)=t^2\ai+(t+\tfrac{t^3}{3})\aj+(t-\tfrac{t^3}{3})\ak,\quad t=0$
\انتہا{سوال}
%===============
\ابتدا{سوال}\شناخت{سوال_سمتی_تفاعل_فضائی_اسراع_نقطہ_ب}
$\kvec{r}(t)=(e^t\cos t)\ai+(e^t\sin t)\aj+\sqrt{2}e^t\ak,\quad t=0$
\انتہا{سوال}
%===============

سوال \حوالہ{سوال_سمتی_تفاعل_عمودی_مستوی_الف} اور سوال \حوالہ{سوال_سمتی_تفاعل_عمودی_مستوی_ب} میں دیے گئے \عددی{\kvec{r}} پر \عددی{\kvec{T}}، \عددی{\kvec{N}}  اور \عددی{\kvec{B}}  معلوم کریں۔اس کے بعد در پیچیدہ مستوی، عمودی مستوی اور سمت کار مستوی کی مساوات اس \عددی{t} پر حاصل کریں۔

\ابتدا{سوال}\شناخت{سوال_سمتی_تفاعل_عمودی_مستوی_الف}
$\kvec{r}(t)=(\cos t)\ai+(\sin t)\aj-\ak,\quad t=\tfrac{\pi}{4}$
\انتہا{سوال}
%===============
\ابتدا{سوال}\شناخت{سوال_سمتی_تفاعل_عمودی_مستوی_ب}
$\kvec{r}(t)=(\cos t)\ai+(\sin t)\aj+t\ak,\quad t=0$
\انتہا{سوال}
%===============

\موٹا{طبعی استعمال}\\
\ابتدا{سوال}
آپ کی گاڑی کا  رفتار پیما  برقرار \عددی{\SI{60}{\kilo\meter\per\hour}} دکھا رہا ہے۔ کیا آپ کی اسراع ممکن ہے؟ جواب کی وجہ پیش کریں۔
\انتہا{سوال}
%==============
\ابتدا{سوال}
کیا مستقل رفتار سے  چلتے ہوئے ذرہ کی اسراع کے بارے میں کچھ کہنا ممکن ہو گا؟ اپنے جواب کی وجہ پیش کریں۔ 
\انتہا{سوال}
%=================
\ابتدا{سوال}
ایک ذرہ کی اسراع پر لمحہ  اس کی  سمتی رفتار کے عمودی ہے۔ اس کی رفتار کے بارے میں کیا کہنا ممکن ہے؟ اپنے جواب کی وجہ پیش کریں۔
\انتہا{سوال}
%==============
\ابتدا{سوال}
ایک جسم جس کی کمیت \عددی{m} ہے قطع مکافی \عددی{y=x^2} پر مستقل رفتار \عددی{\SI{10}{\meter\per\second}} سے حرکت کرتا ہے۔ نقطہ \عددی{(0,0)} اور نقطہ \عددی{(\sqrt{2},2)}  پر اسراع کی بدولت اس پر کتنی قوت ہو گی؟  اپنا جواب \عددی{\ai} اور \عددی{\aj} کی روپ میں لکھیں۔ (نیوٹن کا کلیہ \عددی{\kvec{F}=m\kvec{a}} ذہن میں رکھیں۔)
\انتہا{سوال}
%==============
\ابتدا{سوال}
ایک منحنی پر کسی جسم کو مستقل رفتار  سے حرکت دینے کے لئے درکار قوت، قوانین نیوٹن کے تحت، حرکت کی انحنا کی مستقل مضرب ہو گی۔ حساب  سے دکھائیں کہ یہ فقرہ کیوں درست ہے۔
\انتہا{سوال}
%================
\ابتدا{سوال}
دکھائیں اگر ایک ذرے کی  اسراع کا عمودی جزو صفر ہو  تب یہ متحرک ذرہ سیدھا  حرکت کرے  گا۔
\انتہا{سوال}
%============

\موٹا{انحنا پر مزید سوالات}\\
\ابتدا{سوال}
دکھائیں کہ قطع مکافی \عددی{y=ax^2,\, a\ne 0} کی زیادہ سے   زیادہ   انحنا راس کی  راس پر ہو گی جبکہ کسی بھی نقطہ پر کم سے کم انحنا نہیں ہو گی۔ (چونکہ منحنی کو فضا میں ایک جگہ سے دوسری جگہ منتقل کرنے یا گھمانے سے انحنا تبدیل نہیں ہوتی لہٰذا یہ حقیقت کسی بھی قطع مکافی کے لئے درست ہو گا۔)
\انتہا{سوال}
%=========
\ابتدا{سوال}
دکھائیں کہ ترخیم \عددی{x=a\cos t,\,y=b\sin t,\, a>b>0} کی زیادہ سے زیادہ انحنا اس کی  محور اکبر پر اور کم سے کم انحنا اس کی محور اصغر  پر ہو گی۔(گزشتہ سوال کی طرح یہ حقیقت بھی ہر ترخیم کے لئے درست ہو گا۔)
\انتہا{سوال}
%===========
\ابتدا{سوال}\ترچھا{پیچ دار منحنی کی   زیادہ سے زیادہ انحنا}\\
ہم نے مثال \حوالہ{مثال_سمتی_تفاعل_پیچ_دار_مثبت_الف_ب} میں دیکھا کہ پیچ دار \عددی{\kvec{r}(t)=(a\cos t)\ai+(a\sin t)\aj+bt\ak,\,(a,b\ge 0)}  کی انحنا \عددی{\kappa=\tfrac{a}{a^2+b^2}} ہو گی۔کسی بھی \عددی{b} کے لئے زیادہ سے زیادہ انحنا کتنی ہو گی؟ اپنی جواب کی وجہ پیش کریں۔ 
\انتہا{سوال}
%================
\ابتدا{سوال}\شناخت{سوال_سمتی_تفاعل_انحنا_آسان}
اگر آپ کو \عددی{\abs{a_N}} اور \عددی{\abs{\kvec{v}}} معلوم ہوں تب کلیہ \عددی{a_N=\kappa\abs{\kvec{v}}^2} سے انحنا حاصل کی جا سکتی ہے۔اس کلیہ کو استعمال کرتے ہوئے درج ذیل منحنی کی انحنا اور رداس انحنا دریافت کریں۔(\عددی{a_N} اور \عددی{\abs{\kvec{v}}} کی قیمتیں مثال \حوالہ{مثال_سمتی_تفاعل_در_پیچیدہ_اسراع} سے لیں۔)
\begin{align*}
\kvec{r}(t)=(\cos t+t\sin t)\ai+(\sin t-t\cos t)\aj,\quad t>0
\end{align*}
\انتہا{سوال}
%===============
\ابتدا{سوال}
دکھائیں کہ درج ذیل خط کے لئے \عددی{\kappa} اور \عددی{\tau} صفر ہوں گے۔
\begin{align*}
\kvec{r}(t)=(x_0+At)\ai+(y_0+Bt)\aj+(z_0+Ct)\ak
\end{align*}
\انتہا{سوال}
%==================
\ابتدا{سوال}\ترچھا{مکمل انحنا}\\
ہم ایک منحنی پر   \عددی{s=s_0} سے \عددی{s=s_1\,>s_0}  تک حصہ کی مکمل انحنا حاصل کرنے کی خاطر \عددی{s_0} تا \عددی{s_1} انحنا \عددی{\kappa} کا تکمل لیتے ہیں۔ اگر منحنی کا متغیر  \عددی{s} کی بجائے \عددی{t} ہو تب مکمل انحنا درج ذیل ہو گی، جہاں \عددی{s_0} اور \عددی{s_1} کے مطابقتی قیمتیں \عددی{t_0} اور \عددی{t_1} ہیں۔
\begin{align*}
K=\int_{s_0}^{s_1}\kappa\dif s=\int_{t_0}^{t_1}\kappa\frac{\dif s}{\dif t}\dif t=\int_{t_0}^{t_1}\kappa\abs{\kvec{v}}\dif t
\end{align*}
وقفہ \عددی{0\le t\le 4\pi} پر پیچ دار منحنی \عددی{\kvec{r}(t)=(3\cos t)\ai+(3\sin t)\aj+t\ak} کی مکمل انحنا معلوم کریں۔
\انتہا{سوال}
%================
\ابتدا{سوال}\ترچھا{گزشتہ سوال جاری}\\
درج ذیل  منحنیات کی مکمل انحنا دریافت کریں۔
\begin{enumerate}[a.]
\item
$\kvec{r}(t)=(\cos t+t\sin t)\ai+(\sin t-t\cos t)\aj,\, a\le t\le b\, (a>0)$
(انحنا معلوم کرنے کا آسان طریقہ سوال \حوالہ{سوال_سمتی_تفاعل_انحنا_آسان} میں پیش کیا گیا ہے۔ مثال \حوالہ{مثال_سمتی_تفاعل_در_پیچیدہ_اسراع} کی قیمتیں استعمال کریں۔)
\item
$y=x^2,\quad -\infty<x<\infty=$
\end{enumerate}
\انتہا{سوال}
%====================
\ابتدا{سوال}
\begin{enumerate}[a.]
\item
نقطہ \عددی{(\tfrac{\pi}{2},1)} پر منحنی \عددی{\kvec{r}(t)=t\ai+(\sin t)\aj} کے دائرہ انحنا کی مساوات  تلاش کریں۔ (یہ مستوی \عددی{xy} میں \عددی{y=\sin x} کی مقدار معلوم روپ ہے۔)
\item
نقطہ \عددی{(0,-2)} جہاں \عددی{t=1} ہے پر منحنی \عددی{\kvec{r}(t)=(2\ln t)\ai-(t+\tfrac{1}{t})\aj,\, e^{-2}\le t\le e^2}  کے دائرہ انحنا کی مساوات تلاش کریں۔
\end{enumerate}
\انتہا{سوال}
%===============

\موٹا{نظریہ اور مثالیں}\\
