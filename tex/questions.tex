
\حصہ{تکونیاتی بدل}
ہم \عددی{a^2+x^2}، \عددی{a^2-x^2} اور \عددی{x^2-a^2} میں تکونیاتی بدل پر کر کے  ایک مربع جزو حاصل کرتے ہیں جو ایسے تکمل، جن میں ان کا جذر پایا جاتا ہو، کو سادہ صورت میں بدل دیتا ہے۔ ان سادہ تکمل کا حل نسبتاً آسان  ہوتا ہے۔

\جزوحصہء{تین بنیادی بدل}
تین عمومی بدل \عددی{x=a\tan \theta}، \عددی{x=a\sin\theta} اور \عددی{x=a\sec\theta} ہیں جو شکل \حوالہ{شکل_طریقہ_حوالہ_مثلث} میں قائمہ مثلثوں سے حاصل ہوتے ہیں۔

\عددی{x=a\tan\theta} لیتے ہوئے درج ذیل حاصل ہوتا ہے۔
\begin{align}
a^2+x^2=a^2+a^2\tan^2\theta=a^2(1+\tan^2\theta)=a^2\sec^2\theta
\end{align}

\عددی{x=a\sin\theta} لیتے ہوئے درج ذیل حاصل ہوتا ہے۔
\begin{align}
a^2-x^2=a^2-a^2\sin^2\theta=a^2(1-\sin^2\theta)=a^2\cos^2\theta
\end{align}

\عددی{x=a\sec\theta} لیتے ہوئے درج ذیل حاصل ہوتا ہے۔
\begin{align}
x^2-a^2=a^2\sec^2\theta-a^2=a^2(\sec^2\theta-1)=a^2\tan^2\theta
\end{align}
\موٹا{تکونیاتی بدل}\\
\begin{enumerate}[a.]
\item
\عددی{x=a\tan\theta} لے کر \عددی{a^2+x^2} کی جگہ \عددی{a^2\sec^2\theta} پر کریں۔
\item
\عددی{x=a\sin\theta} لے کر \عددی{a^2-x^2} کی جگہ \عددی{a^2\cos^2\theta} پر کریں۔
\item
\عددی{x=a\sec\theta} لے کر \عددی{x^2-a^2} کی جگہ \عددی{a^2\tan^2\theta} پر کریں۔
\end{enumerate}


\begin{figure}
\centering
\begin{subfigure}{0.3\textwidth}
\centering
\begin{tikzpicture}[font=\small]
\pgfmathsetmacro{\len}{2}
\pgfmathsetmacro{\ang}{30}
\pgfmathsetmacro{\kx}{\len*cos(\ang)}
\pgfmathsetmacro{\ky}{\len*sin(\ang)}
\draw(0,0)--++(\ang:\len)node[pos=0.7,left,yshift=0.5ex]{$\sqrt{a^2+x^2}$}--(\kx,0)node[pos=0.5,right]{$x$}--(0,0)node[pos=0.5,below]{$a$};
\draw([shift={(0:0.5)}]0,0) arc (0:\ang:0.5);
\draw(1/2*\ang:0.7)node[]{$\theta$};
\draw(\x/2,\ky+0.25)node[above]{$\begin{aligned} x&=a\tan\theta\\ \sqrt{a^2+x^2}&=a\abs{\sec\theta} \end{aligned}$};
\end{tikzpicture}
\end{subfigure}\hfill
\begin{subfigure}{0.3\textwidth}
\centering
\begin{tikzpicture}[font=\small]
\pgfmathsetmacro{\len}{2}
\pgfmathsetmacro{\ang}{30}
\pgfmathsetmacro{\kx}{\len*cos(\ang)}
\pgfmathsetmacro{\ky}{\len*sin(\ang)}
\draw(0,0)--++(\ang:\len)node[pos=0.7,left,yshift=0.5ex]{$a$}--(\kx,0)node[pos=0.5,right]{$x$}--(0,0)node[pos=0.5,below]{$\sqrt{a^2-x^2}$};
\draw([shift={(0:0.5)}]0,0) arc (0:\ang:0.5);
\draw(1/2*\ang:0.7)node[]{$\theta$};
\draw(\x/2,\ky+0.25)node[above]{$\begin{aligned} x&=a\sin\theta\\ \sqrt{a^2-x^2}&=a\abs{\cos\theta} \end{aligned}$};
\end{tikzpicture}
\end{subfigure}\hfill
\begin{subfigure}{0.3\textwidth}
\centering
\begin{tikzpicture}[font=\small]
\pgfmathsetmacro{\len}{2}
\pgfmathsetmacro{\ang}{30}
\pgfmathsetmacro{\kx}{\len*cos(\ang)}
\pgfmathsetmacro{\ky}{\len*sin(\ang)}
\draw(0,0)--++(\ang:\len)node[pos=0.7,left,yshift=0.5ex]{$x$}--(\kx,0)node[pos=0.5,right]{$\sqrt{x^2-a^2}$}--(0,0)node[pos=0.5,below]{$a$};
\draw([shift={(0:0.5)}]0,0) arc (0:\ang:0.5);
\draw(1/2*\ang:0.7)node[]{$\theta$};
\draw(\x/2,\ky+0.25)node[above]{$\begin{aligned} x&=a\sec\theta\\ \sqrt{x^2-a^2}&=a\abs{\tan\theta} \end{aligned}$};
\end{tikzpicture}
\end{subfigure}
\caption{تکونیاتی بدل کو حوالہ مثلث۔}
\label{شکل_طریقہ_حوالہ_مثلث}
\end{figure}

ہم ایسا بدل استعمال کرنا چاہیں گے جو قابل واپسی ہو تا کہ آخری قدم پر اس کو واپس کرتے ہوئے اصل متغیرات میں نتیجہ لکھ سکیں۔ مثال کے طور پر اگر \عددی{x=a\tan\theta} کی صورت میں ہم چاہیں گے کہ تکمل لینے کے بعد آخری قدم پر \عددی{\theta=\tan^{-1}\tfrac{x}{a}} لکھنا ممکن ہو۔ اسی طرح \عددی{x=a\sin\theta} کی صورت میں ہم تکمل کے بعد \عددی{\theta=\sin^{-1}\tfrac{x}{a}} پر کرنا چاہیں گے۔

جیسا ہم حصہ \حوالہ{حصہ_ماورائی_الٹ_تکونیاتی_تفاعل} سے جانتے ہیں ان تفاعل کے الٹ صرف مخصوص وقفہ پر پائے جاتے ہیں۔
