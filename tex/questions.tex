
\حصہء{سوالات}
\موٹا{الجبرائی حساب}

سوال \حوالہ{سوال_ماورائی_سادہ_صورت_فقرہ_الف} تا سوال \حوالہ{سوال_ماورائی_سادہ_صورت_فقرہ_ب} میں ریاضی فقرے کی سادہ صورت تلاش کریں۔

\ابتدا{سوال}\شناخت{سوال_ماورائی_سادہ_صورت_فقرہ_الف}
\begin{multicols}{3}
\begin{enumerate}[a.]
\item
$5^{\log_57}$
\item
$8^{\log_8\sqrt{2}}$
\item
$1.3^{\log_{1.3}75}$
\item
$\log_416$
\item
$\log_3\sqrt{3}$
\item
$\log_4\big(\frac{1}{4}\big)$
\end{enumerate}
\end{multicols}
\انتہا{سوال}
%========================
\ابتدا{سوال}
\begin{multicols}{3}
\begin{enumerate}[a.]
\item
$2^{\log_2 3}$
\item
$10^{\log_{10}(1/2)}$
\item
$\pi^{\log_{\pi}7}$
\item
$\log_{11}121$
\item
$\log_{121}11$
\item
$\log_3\big(\frac{1}{9}\big)$
\end{enumerate}
\end{multicols}
\انتہا{سوال}
%========================
\ابتدا{سوال}
\begin{multicols}{3}
\begin{enumerate}[a.]
\item
$2^{\log_4 x}$
\item
$9^{\log 3 x}$
\item
$\log_2(e^{(\ln 2)(\sin x)})$
\end{enumerate}
\end{multicols}
\انتہا{سوال}
%========================
\ابتدا{سوال}\شناخت{سوال_ماورائی_سادہ_صورت_فقرہ_ب}
\begin{multicols}{3}
\begin{enumerate}[a.]
\item
$25^{\log_5(3x^2)}$
\item
$\log_e(e^x)$
\item
$\log_4(2^{e^x\sin x})$
\end{enumerate}
\end{multicols}
\انتہا{سوال}
%========================
سوال \حوالہ{سوال_ماورائی_قدرتی_لوگاتمی_سادہ_الف} اور سوال \حوالہ{سوال_ماورائی_قدرتی_لوگاتمی_سادہ_ب} میں نسبت کو قدرتی لوگارتھمی صورت میں لکھ کر سادہ صورت حاصل کریں۔

\ابتدا{سوال}\شناخت{سوال_ماورائی_قدرتی_لوگاتمی_سادہ_الف}
\begin{multicols}{3}
\begin{enumerate}[a.]
\item
$\tfrac{\log_2x}{\log_3x}$
\item
$\tfrac{\log_2x}{\log_8x}$
\item
$\tfrac{\log_xa}{\log_{x^2}a}$
\end{enumerate}
\end{multicols}
\انتہا{سوال}
%====================
\ابتدا{سوال}\شناخت{سوال_ماورائی_قدرتی_لوگاتمی_سادہ_ب}
\begin{multicols}{3}
\begin{enumerate}[a.]
\item
$\tfrac{\log_9x}{\log_3x}$
\item
$\tfrac{\log_{\sqrt{10}}x}{\log_{\sqrt{2}}x}$
\item
$\tfrac{\log_ab}{\log_ba}$
\end{enumerate}
\end{multicols}
\انتہا{سوال}
%====================
سوال \حوالہ{سوال_ماورائی_مساوات_حل_الف} تا سوال \حوالہ{سوال_ماورائی_مساوات_حل_ب} میں دی گئی مساوات حل کریں۔

\ابتدا{سوال}\شناخت{سوال_ماورائی_مساوات_حل_الف}
$3^{\log_3(7)+2^{\log_2(5)}}=5^{\log_5(x)}$
\انتہا{سوال}
%==================
\ابتدا{سوال}
$8^{\log_8(3)}-e^{\ln 5}=x^2-7^{\log_7(3x)}$
\انتہا{سوال}
%=====================
\ابتدا{سوال}
$3^{\log_3(x^2)=5e^{\ln x}}-3\cdot 10^{\log_{10}(2)}$
\انتہا{سوال}
%====================
\ابتدا{سوال}\شناخت{سوال_ماورائی_مساوات_حل_ب}
 $\ln e+4^{-2\log_4(x)}=\frac{1}{x}\log_{10}(100)$
\انتہا{سوال}
%==================
سوال \حوالہ{سوال_ماورائی_تفرق_بالحظ_تلاش_الف} تا سوال \حوالہ{سوال_ماورائی_تفرق_بالحظ_تلاش_ب} میں دیے گئے غیر تابع متغیر کے لحاظ سے \عددی{y} کا تفرق تلاش کریں۔

\ابتدا{سوال}\شناخت{سوال_ماورائی_تفرق_بالحظ_تلاش_الف}
$y=2^x$
\انتہا{سوال}
%==========================
\ابتدا{سوال}
$y=3^{-x}$
\انتہا{سوال}
%==========================
\ابتدا{سوال}
$y=5^{\sqrt{s}}$
\انتہا{سوال}
%==========================
\ابتدا{سوال}
$y=2^{s^2}$
\انتہا{سوال}
%==========================
\ابتدا{سوال}
$y=x^{\pi}$
\انتہا{سوال}
%==========================
\ابتدا{سوال}
$y=t^{1-e}$
\انتہا{سوال}
%==========================
\ابتدا{سوال}
$y=(\cos\theta)^{\sqrt{2}}$
\انتہا{سوال}
%==========================
\ابتدا{سوال}
$y=(\ln\theta)^{\pi}$
\انتہا{سوال}
%==========================
\ابتدا{سوال}
$y=7{\sec\theta}\ln 7$
\انتہا{سوال}
%==========================
\ابتدا{سوال}
$y=3^{\tan\theta}\ln 3$
\انتہا{سوال}
%==========================
\ابتدا{سوال}
$y=2^{\sin 3t}$
\انتہا{سوال}
%==========================
\ابتدا{سوال}
$y=5^{-\cos 2t}$
\انتہا{سوال}
%==========================
\ابتدا{سوال}
$y=\log_25\theta$
\انتہا{سوال}
%==========================
\ابتدا{سوال}
$y=\log_3(1+\theta\ln 3)$
\انتہا{سوال}
%==========================
\ابتدا{سوال}
$y=\log_4x+\log_4x^2$
\انتہا{سوال}
%==========================
\ابتدا{سوال}
$y=\log_{25}e^x-\log_5\sqrt{x}$
\انتہا{سوال}
%==========================
\ابتدا{سوال}
$y=\log_2r\cdot\log_4r$
\انتہا{سوال}
%==========================
\ابتدا{سوال}
$y=\log_3r\cdot\log_9r$
\انتہا{سوال}
%==========================
\ابتدا{سوال}
$y=\log_3\big((\tfrac{x+1}{x-1})^{\ln 3}\big)$
\انتہا{سوال}
%==========================
\ابتدا{سوال}
$y=\log_5\sqrt{(\tfrac{7x}{3x+2})^{\ln 5}}$
\انتہا{سوال}
%==========================
\ابتدا{سوال}
$y=\theta\sin(\log_7\theta)$
\انتہا{سوال}
%==========================
\ابتدا{سوال}
$y=\log_7(\tfrac{\sin\theta\cos\theta}{e^{\theta}2^{\theta}})$
\انتہا{سوال}
%==========================
\ابتدا{سوال}
$y=\log_5e^x$
\انتہا{سوال}
%==========================
\ابتدا{سوال}
$y=\log_2(\tfrac{x^2e^2}{2\sqrt{x+1}})$
\انتہا{سوال}
%==========================
\ابتدا{سوال}
$y=3^{\log_2 t}$
\انتہا{سوال}
%==========================
\ابتدا{سوال}
$y=3\log_8(\log_2t)$
\انتہا{سوال}
%==========================
\ابتدا{سوال}
$y=\log_2(8t^{\ln 2})$
\انتہا{سوال}
%==========================
\ابتدا{سوال}\شناخت{سوال_ماورائی_تفرق_بالحظ_تلاش_ب}
$y=t\log_3(e^{(\sin t)(\ln 3)})$
\انتہا{سوال}
%==========================
\موٹا{لوگارتھمی تفرق}

سوال \حوالہ{سوال_ماورائی_لوگارتھمی_تفرق_الف} تا سوال \حوالہ{سوال_ماورائی_لوگارتھمی_تفرق_ب} میں \عددی{y} کا لوگارتھمی تفرق دیے گئے غیر تابع متغیر کے لحاظ سے معلوم کریں۔ 

\ابتدا{سوال}\شناخت{سوال_ماورائی_لوگارتھمی_تفرق_الف}
$y=(x+1)^x$
\انتہا{سوال}
%======================
\ابتدا{سوال}
$y=x^{(x+1)}$
\انتہا{سوال}
%======================
\ابتدا{سوال}
$y=(\sqrt{t})^t$
\انتہا{سوال}
%======================
\ابتدا{سوال}
$y=t^{\sqrt{t}}$
\انتہا{سوال}
%======================
\ابتدا{سوال}
$y=(\sin x)^x$
\انتہا{سوال}
%======================
\ابتدا{سوال}
$y=x^{\sin x}$
\انتہا{سوال}
%======================
\ابتدا{سوال}
$y=x^{\ln x}$
\انتہا{سوال}
%======================
\ابتدا{سوال}\شناخت{سوال_ماورائی_لوگارتھمی_تفرق_ب}
$y=(\ln x)^{\ln x}$
\انتہا{سوال}
%======================
\موٹا{تکمل}\\
سوال \حوالہ{سوال_ماورائی_تکمل_تلاش_کریں-الف} تا سوال \حوالہ{سوال_ماورائی_تکمل_تلاش_کریں-ب} میں تکمل تلاش کریں۔

\ابتدا{سوال}\شناخت{سوال_ماورائی_تکمل_تلاش_کریں-الف}
$\int5^x\dif x$
\انتہا{سوال}
%======================
\ابتدا{سوال}
$\int(1.3)^x\dif x$
\انتہا{سوال}
%======================
\ابتدا{سوال}
$\int_0^12^{-\theta}\dif \theta$
\انتہا{سوال}
%======================
\ابتدا{سوال}
$\int_{-2}^05^{-\theta}\dif \theta$
\انتہا{سوال}
%======================
\ابتدا{سوال}
$\int_1^{\sqrt{2}}x2^{(x^2)}\dif x$
\انتہا{سوال}
%======================
\ابتدا{سوال}
$\int_1^4 \frac{2^{\sqrt{x}}}{\sqrt{x}}\dif x$
\انتہا{سوال}
%======================
\ابتدا{سوال}
$\int_0^{\pi/2}7^{\cos t}\sin t\dif t$
\انتہا{سوال}
%======================
\ابتدا{سوال}
$\int_0^{\pi/4}\big(\frac{1}{3}\big)^{\tan t}\sec^2t\dif t$
\انتہا{سوال}
%======================
\ابتدا{سوال}
$\int_2^4x^{2x}(1+\ln x)\dif x$
\انتہا{سوال}
%======================
\ابتدا{سوال}\شناخت{سوال_ماورائی_تکمل_تلاش_کریں-ب}
$\int_1^2\frac{2^{\ln x}}{x}\dif x$
\انتہا{سوال}
%======================
سوال \حوالہ{سوال_ماورائی_تکمل_حل_کریں_الف} تا سوال \حوالہ{سوال_ماورائی_تکمل_حل_کریں_ب} میں دیے گئے تکمل حل کریں۔

\ابتدا{سوال}\شناخت{سوال_ماورائی_تکمل_حل_کریں_الف}
$\int 3x^{\sqrt{3}}\dif x$
\انتہا{سوال}
%====================
\ابتدا{سوال}
$\int x^{\sqrt{2}-1}\dif x$
\انتہا{سوال}
%====================
\ابتدا{سوال}
$\int_0^3(\sqrt{2}+1)x^{\sqrt{2}}\dif x$
\انتہا{سوال}
%====================
\ابتدا{سوال}\شناخت{سوال_ماورائی_تکمل_حل_کریں_ب}
$\int_1^ex^{(\ln 2)-1}$
\انتہا{سوال}
%====================
سوال \حوالہ{سوال_ماورائی_دیا_تکمل_الف} تا سوال \حوالہ{سوال_ماورائی_دیا_تکمل_ب} میں دیے تکمل کو حل کریں۔

\ابتدا{سوال}\شناخت{سوال_ماورائی_دیا_تکمل_الف}
$\int\frac{\log_{10}x}{x}\dif x$
\انتہا{سوال}
%=======================
\ابتدا{سوال}
$\int_1^4\frac{\log_2 x}{x}\dif x$
\انتہا{سوال}
%=====================
\ابتدا{سوال}
$\int_1^4\frac{\ln 2\log_2 x}{x}\dif x$
\انتہا{سوال}
%=====================
\ابتدا{سوال}
$\int_1^e\frac{2\ln 10 \log_{10}x}{x}\dif x$
\انتہا{سوال}
%=================
\ابتدا{سوال}
$\int_0^2\frac{\log_2(x+2)}{x+2}\dif x$
\انتہا{سوال}
%===================
\ابتدا{سوال}
$\int_{1/10}^{10}\frac{\log_{10}(10x)}{x}\dif x$
\انتہا{سوال}
%===================
\ابتدا{سوال}
$\int_0^9\frac{2\log_{10}(x+1)}{x+1}\dif x$
\انتہا{سوال}
%===================
\ابتدا{سوال}
$\int_2^3\frac{2\log_2(x-1)}{x-1}\dif x$
\انتہا{سوال}
%===================
\ابتدا{سوال}
$\int\frac{\dif x}{x\log_{10}x}$
\انتہا{سوال}
%===================
\ابتدا{سوال}\شناخت{سوال_ماورائی_دیا_تکمل_ب}
$\int\frac{\dif x}{x(\log_8x)^2}$
\انتہا{سوال}
%===================
سوال \حوالہ{سوال_ماورائی_تکمل_کی_قیمت_الف} تا سوال \حوالہ{سوال_ماورائی_تکمل_کی_قیمت_ب} میں تکمل کی قیمت تلاش کریں۔

\ابتدا{سوال}\شناخت{سوال_ماورائی_تکمل_کی_قیمت_الف}
$\int_1^{\ln x}\frac{1}{t}\dif t,\quad x>1$
\انتہا{سوال}
%=====================
\ابتدا{سوال}
$\int_1^{e^x}\frac{1}{t}\dif t$
\انتہا{سوال}
%=====================
\ابتدا{سوال}
$\int_1^{1/x}\frac{1}{t}\dif t,\quad x>0$
\انتہا{سوال}
%=====================
\ابتدا{سوال}\شناخت{سوال_ماورائی_تکمل_کی_قیمت_ب}
$\frac{1}{\ln a}\int_1^x\frac{1}{t}\dif t,\quad x>0$
\انتہا{سوال}
%=====================
\موٹا{نظریہ اور استعمال}

\ابتدا{سوال}
منحنی \عددی{y=\tfrac{2x}{1+x^2}} اور محور \عددی{x} پر \عددی{-2\le x\le 2}  کے بیچ خطے کا رقبہ معلوم کریں۔
\انتہا{سوال}
%=====================
\ابتدا{سوال}
منحنی \عددی{y=2^{1-x}} اور محور \عددی{x} پر \عددی{-1\le x\le 1}  کے بیچ خطے کا رقبہ معلوم کریں۔
\انتہا{سوال}
%==========================
\ابتدا{سوال}\ترچھا{انسانی خون کا $\pH$}\\
انسانی خون کے \عددی{\pH} کی قیمت \عددی{7.37} سے \عددی{7.44} تک ہوتی ہے۔ انسانی خون میں برق پارہ \عددی{[\ce{H3O^+}]} کے مطابقتی حدود تلاش کریں۔ 
\انتہا{سوال}
%========================
\ابتدا{سوال}\ترچھا{دماغی سیال کا $\pH$}\\
دماغی سیال میں \عددی{[\ce{H3O^+}]} کا گاڑھا پن تقریباً \عددی{\SI{4.8e-8}{\mole\per\liter}} ہے۔ اس سیال کا \عددی{\pH} تلاش کریں۔
\انتہا{سوال}
%====================
\ابتدا{سوال}
افزائش کار (ایمپلی فائر) سے حاصل صدا کو جزو \عددی{k} سے ضرب دے کر اس سطح صدا کو \عددی{\SI{10}{\deci\bel}} مزید بلن کیا جاتا ہے۔ جزو \عددی{k} کی قیمت تلاش کریں۔
\انتہا{سوال}
%===================
\ابتدا{سوال}
ایک افزائش کار صدا کی شدت کو \عددی{10} سے ضرب دیتا ہے۔ صدا میں کتنے \عددی{\si{\deci\bel}} کا اضافہ پیدا ہو گا؟ 
\انتہا{سوال}
%===================
\ابتدا{سوال}
کسی بھی محلول میں \عددی{[\ce{H3O^+}]}  اور \عددی{[OH^-]} کی گاڑھا پن کا حاصل ضرب \عددی{10^{-14}} ہوتا ہے۔
\begin{enumerate}[a.]
\item
\عددی{[\ce{H3O^+}]} کی کیا قیمت گاڑھا پن کی مجموعی \عددی{S=[\ce{H3O^+}]+[\ce{OH^-}]} کو کم سے کم  کرتی ہے؟
\item
اس محلول کی \عددی{\pH} تلاش کریں  جس میں \عددی{S} کی قیمت کم سے کم ہو۔
\item
\عددی{[\ce{H3O^+}]} اور \عددی{[OH^-]} کی کون سی نسبت \عددی{S} کو کم سے کم بناتی ہے؟
\end{enumerate}
\انتہا{سوال}
%==================
\ابتدا{سوال}
کیا \عددی{\log_ab} کی قیمت \عددی{\tfrac{1}{\log_ba}} کے برابر ہو سکتی ہے؟ اپنے جواب کی وجہ   پیش کریں۔
\انتہا{سوال}
%===================
\موٹا{کمپیوٹر کا استعمال}

\ابتدا{سوال}
مساوات \عددی{x^2=2^x} کے دو حل \عددی{x=2} اور  \عددی{x=4} ہیں جبکہ اس کا تیسرا حل بھی پایا جاتا ہے۔ ترسیم کی مدد سے تیسرا حل تلاش کریں۔
\انتہا{سوال}
%===================
\ابتدا{سوال}
کیا \عددی{x>0} کے لئے  \عددی{x^{\ln 2}} اور \عددی{2^{\ln x}} ایک دوسرے کے برابر ہو سکتے ہیں؟دونوں تفاعل ترسیم کرتے ہوئے بتائیں کیا ہوتا ہے۔
\انتہا{سوال}
%====================
\ابتدا{سوال}\ترچھا{$2^x$ کی خط بندی}\\
(ا) نقطہ \عددی{x=0} پر \عددی{f(x)=2^x} کی خط بندی دریافت کریں۔ اس کے بعد عددی سروں کو \عددی{2} اعشاریہ  پور و پور کریں۔ (ب) وقفہ \عددی{-3\le x\le 3} اور وقفہ \عددی{-1\le x\le 1} کے لئے تفاعل اور خط بندی کو ایک ساتھ ترسیم کریں۔
\انتہا{سوال}
%===================
\ابتدا{سوال}\ترچھا{$f(x)=\log_3 x$ کی خط بندی}\\
(ا) نقطہ \عددی{x=3} پر \عددی{f(x)=\log_3 x} کی خط بندی تلاش کریں۔ اس کے بعد عددی سروں کو \عددی{2} اعشاریہ تک پور و پور کریں۔    (ب) وقفہ \عددی{0\le x\le 8} اور \عددی{2\le x\le} کے لئے تفاعل اور خط بندی کو ایک ساتھ ترسیم کریں۔
\انتہا{سوال}
%====================
\موٹا{دیگر اساس کے ساتھ حساب کتاب}

\ابتدا{سوال}
عموماً کیلکولیٹروں میں \عددی{\log_{10}x} اور \عددی{\ln x} پائے جاتے ہیں۔ دیگر اساس کے لوگارتھم تلاش کرنے کی خاطر ہم درج ذیل مساوات استعمال کرتے ہیں۔
\begin{align*}
\log_ax=\frac{\ln x}{\ln a}
\end{align*}
یوں درج ذیل ہو گا۔
\begin{align*}
\log_25=\frac{\ln 5}{\ln 2}\approx 2.3219
\end{align*}
کیلکولیٹر استعمال کرتے ہوئے  \عددی{5} اعشاریہ درستگی تک 
(ا) \عددی{\log_38}، (ب) \عددی{\log_70.5}، (ج) \عددی{\log_{2-}17}، (د) \عددی{\log_{0.5}7} تلاش کریں۔ درج ذیل معلومات استعمال کرتے ہوئے \عددی{\ln x} تلاش کریں۔
(ہ) \عددی{\log_{10}x=2.3}،  (و) \عددی{\log_2x=1.4}،  (ز) \عددی{\log_2x=-1.5}،  (ح) \عددی{\log_{10}x=-0.7}
\انتہا{سوال}
%=====================
\ابتدا{سوال}\ترچھا{تبدیلی پیمانہ}\\
(ا) دکھائیں کہ اساس \عددی{10} لوگارتھم کو اساس \عددی{2} لوگارتھم میں تبدیل کرنے کی مساوات درج ذیل ہے۔
\begin{align*}
\log_2x=\frac{\ln 10}{\ln 2}\log_{10}x
\end{align*}
(ب) دکھائیں کہ اساس  \عددی{a} لوگارتھم کو اساس \عددی{b} لوگارتھم میں تبدیل کرنے  کی مساوات درج ذیل ہے۔
\begin{align*}
\log_bx=\frac{\ln a}{\ln b}\log_ax
\end{align*}
\انتہا{سوال}
%=======================

\حصہ{افزائش اور تنزل}\شناخت{حصہ_ماورائی_افزائش_تنزل}
اس حصہ میں ہم قوت نما تبدیلی کے قاعدہ کو حاصل کریں گے۔ اس کے علاوہ ان عملی استعمال پر غور کیا جائے گا جن کی بنا لوگارتھمی اور قوت نمائی تفاعل اہمیت کے حامل ہیں۔

\جزوحصہء{قوت نما تبدیلی کا قاعدہ}
فرض کریں ہم کسی مقدار \عددی{y} (جو سمتی رفتار، درجہ حرارت، برقی رو، یا کچھ اور ہو سکتا ہے) میں دلچسپی رکھتے ہیں جس میں کسی بھی لمحہ \عددی{t} پر اضافہ یا کمی اس لمحہ موجود مقدار کے راست متناسب  ہے۔ اگر ہمیں لمحہ \عددی{t=0} پر مقدار کی قیمت \عددی{y_0} بھی معلوم ہو تب ہم متغیر \عددی{t} کے تفاعل \عددی{y} کو درج ذیل ابتدائی قیمت مسئلہ حل کر کے حاصل کر سکتے ہیں۔
\begin{gather}
\begin{aligned}\label{مساوات_ماورائی_قوت_نمائی_مساوات_الف}
\frac{\dif y}{\dif t}&=ky&&\text{\RL{تفرقی مساوات}}\\
y&=y_0,\quad t=0&&\text{\RL{ابتدائی معلومات}}
\end{aligned}
\end{gather}
اگر \عددی{y} مثبت ہو اور بڑھ رہا ہو تب \عددی{k} مثبت ہو گا اور مساوات \حوالہ{مساوات_ماورائی_قوت_نمائی_مساوات_الف} کہتی ہے کہ اضافہ کی شرح جمع کیے گئے مقدار کے راست متناسب ہے۔ اگر \عددی{y} منفی ہو اور گھٹ رہا ہو تب \عددی{k} منفی ہو گا اور مساوات \حوالہ{مساوات_ماورائی_قوت_نمائی_مساوات_الف} کہتی ہے کہ تنزل کی شرح، رہ گئی مقدار کے راست متناسب ہے۔

ہم دیکھ سکتے ہیں کہ مساوات \حوالہ{مساوات_ماورائی_قوت_نمائی_مساوات_الف} کا ایک حل \عددی{y=0}  ہے۔ غیر صفر حل حاصل کرنے کے لئے ہم مساوات \حوالہ{مساوات_ماورائی_قوت_نمائی_مساوات_الف} کے دونوں اطراف کو \عددی{y} سے تقسیم کر کے حل کرتے ہیں:
\begin{align*}
\frac{1}{y}\cdot \frac{\dif y}{\dif t}&=k\\
\ln\abs{y}&=kt+C&&\text{\RL{$t$ کے لحاظ سے تکمل}}\\
\abs{y}&=e^{kt+C}&&\text{\RL{قوت نما صورت}}\\
\abs{y}&=e^C\cdot e^{kt}&& e^{a+b}=e^a\cdot e^b\\
y&=\mp e^Ce^{kt}&&\text{\RL{اگر $\abs{y}=r$ ہو تب $y=\mp r$ ہو گا}}\\
y&=Ae^{kt}&&\text{\RL{مستقل $\mp e^{C}$ کو سادہ علامت $A$ سے ظاہر کرتے ہیں}}
\end{align*}
ہم \عددی{\mp e^C} کی تمام ممکنہ قیمتوں کے علاوہ  \عددی{0} کو بھی \عددی{A} کی قیمت لے کر  حل \عددی{y=0} کو بھی اس کلیہ میں شامل کرتے ہیں۔

ہم ابتدائی قیمت مسئلہ کے لئے \عددی{A} کی قیمت حاصل کرنے کی خاطر \عددی{t=0} پر \عددی{y=y_0} کو پر کرتے ہیں۔ 
\begin{align*}
y_0=Ae^{k\cdot 0}=A
\end{align*}
یوں اس ابتدائی قیمت مسئلے کا حل \عددی{y=y_0e^{kt}} ہو گا۔

درج ذیل \اصطلاح{قوت نما تبدیلی کا قاعدہ} ہے جس میں \عددی{k} کو \اصطلاح{شرحی مستقل}\فرہنگ{مستقل!شرحی}\حاشیہب{rate constant}\فرہنگ{constant!rate} کہتے ہیں۔
\begin{align}\label{مساوات_ماورائی_قوت_نما_تبدیلی_قاعدہ}
y&=y_0e^{kt},\quad k>0\,\text{اضافہ} ,\quad k<0\, \text{تنزل}&&\text{\RL{قوت نما تبدیلی کا قاعدہ}}
\end{align}
مساوات \حوالہ{مساوات_ماورائی_قوت_نما_تبدیلی_قاعدہ} کا حصول ہمیں دکھاتا ہے کہ صرف قوت نما تفاعل کا مستقل مضرب اپنے آپ کا تفرق ہو سکتا ہے۔ 

\جزوحصہء{نمو آبادی}
کوئی بھی آبادی (انسانی، نباتاتی، جراثیمی، وغیرہ) غیر استمراری تفاعل ہو گا چونکہ یہ صرف غیر مسلسل قیمتیں اختیار کرتی ہے۔ اس کے باوجود جب آبادی میں فردی تعداد بہت زیادہ ہو تب اس آبادی کو نا صرف استمراری بلکہ قابل تفرق تفاعل  سے ظاہر کرنا ممکن ہوتا ہے۔ اگر ہم فرض کریں کہ آبادی میں بچے پیدا کرنے والوں کی تناسب برقرار رہتی ہے تب کسی بھی لمحہ \عددی{t} پر بچوں کی پیدائشی شرح اس لمحے پر افراد کی تعداد \عددی{y(t)} کے راست تناسب ہو گی۔ اگر ہم  باہر سے آنے اور جانے والوں کو رد کریں اور ساتھ ہی مرنے والوں کی تعداد کو بھی رد کریں تب نمو آبادی کی شرح \عددی{\tfrac{\dif y}{\dif t}} پیدائشی شرح \عددی{ky} کے برابر ہو گی۔یوں \عددی{\frac{\dif y}{\dif t}=ky} لہٰذا \عددی{y=y_0e^{kt}} ہو گا۔ حقیقت میں کسی بھی آبادی پر دیگر عوامل بھی اثر انداز ہوں گے جن پر یہاں غور نہیں کیا جائے گا۔

\ابتدا{مثال}
بیماری کی پھیلاو کا ایک نمونہ فرض کرتا ہے کہ بیمار ہونے والوں کی شرح \عددی{\tfrac{\dif y}{\dif t}} اس وقت  کی تعداد \عددی{y} کے راست تناسب ہے۔ یوں جتنے زیادہ افراد کو بیماری لاحق ہو، بیماری اتنی زیادہ تیزی سے پھیلے گی۔

فرض کریں کہ ایک سال کے عرصہ میں کسی بیماری میں مبتلا افراد کی تعداد میں \عددی{\SI{20}{\percent}} کمی رونما ہوتی ہے۔ اگر آج \عددی{\num{10000}} افراد بیمار ہوں تب کتنے سالوں میں بیمار افراد کی تعداد \عددی{1000} ہو گی؟

حل:\quad
ہم مساوات \عددی{y=y_0e^{kt}} استعمال کرتے ہیں۔ہمیں تین چیزیں معلوم کرنی ہیں۔
\begin{enumerate}[a.]
\item
\عددی{y_0} کی قیمت،
\item
\عددی{k} کی قیمت،
\item
\عددی{y=1000} کرنے کے لئے درکار \عددی{t} کی قیمت۔
\end{enumerate}
\موٹا{پہلا قدم:}\quad \ترچھا{$y_0$ کی قیمت:}\quad
ہم آج کو لمحہ \عددی{t=0} لیتے ہیں۔ یوں \عددی{t=0} پر \عددی{y=\num{10000}} ہے۔یوں ہماری مساوات درج ذیل ہے۔
\begin{align*}
y=\num{10000}e^{kt}
\end{align*}
\موٹا{دوسرا قدم:}\quad \ترچھا{$k$ کی قیمت:}\quad
ایک سال کے بعد بیماروں کی تعداد، آج کی تعداد کے  \عددی{\SI{80}{\percent}} یعنی \عددی{8000} ہو گی۔آئیں \عددی{k} حاصل کریں۔
\begin{align*}
8000&=\num{10000}e^{k(1)}\\
e^k&=0.8\\
\ln(e^k)&=\ln 0.8\\
k=\ln 0.8
\end{align*} 
یوں لمحہ \عددی{t} پر درج ذیل ہو گا۔
\begin{align}\label{مساوات_ماورائی_مثال_بیماری}
y=\num{10000}e^{(\ln 0.8)T}
\end{align}
\موٹا{تیسرا قدم:}\quad \ترچھا{$t$ کی وہ قیمت جو $y=1000$ دیتی ہے:}\quad
ہم مساوات \حوالہ{مساوات_ماورائی_مثال_بیماری} میں \عددی{y=1000} پر کر کے \عددی{t} حاصل کرتے ہیں۔
\begin{align*}
1000&=\num{10000}e^{(\ln 0.8)t}\\
e^{(\ln 0.8)t}&=0.1\\
(\ln 0.8)t&=\ln 0.1\\
t&=\frac{\ln 0.1}{\ln 0.8}\approx 10.32
\end{align*}
یوں بیماروں کی تعداد \عددی{1000} کرنے کے لئے ہمیں دس سال سے کچھ زیادہ انتظار کرنا ہو گا۔
\انتہا{مثال}
%======================

\جزوحصہء{مسلسل سود در سود}
اگر آپ \عددی{A_0} روپیہ کاروبار میں ڈالیں اور ایک سال میں اس سے \عددی{r'} روپیہ کمانے کی امید رکھتے ہوں، جہاں \عددی{r'=r\times A_0} ہے، تب ایک سال کے آخر میں آپ کے پاس \عددی{A_0+r'=A_0(1+r)} روپیہ ہوں گے۔ 

ربا پر کاروبار کرنے والا بینک ایک شخص کو \عددی{A_0} روپیہ سود پر دیتا ہے۔ایک سال بعد اس شخص پر \عددی{r\times A_0} کا سود واجب الادا ہو گا لہٰذا ایک سال بعد  اس شخص پر کل \عددی{A_0+rA_0=A_0(1+r)} قرضہ ہو گا۔ ہم کہتے ہیں کہ سالانہ سود کی شرح \عددی{r} ہے۔ فرض کریں کہ یہ شخص سالانہ سود ادا نہیں کرتا ہے۔ یوں دوسرے سال کی ابتدا میں اس شخص پر \عددی{A_0(1+r)} قرضہ ہو گا اور بینک اگلے سال اس مقدار پر سود حاصل کرے گا۔ چونکہ سود کی شرح \عددی{r} ہے لہٰذا دوسرے سال اس شخص پر سود \عددی{r\times A_0(1+r)} ہو گا اور دوسرے سال کے آخر میں اس پر کل قرضہ
\begin{align*}
A_0(1+r)+rA_0(1+r)=A_0(1+r)(1+r)=A_0(1+r)^2
\end{align*}
 ہو گا۔ اسی طرح تین سال بعد قرضہ \عددی{A_0(1+r)^2+rA_0(1+r)^2=A_0(1+r)^3} اور  \عددی{t} سال بعد قرضہ
\begin{align*}
A_0(1+r)^t
\end{align*}
ہو گا۔

اب بینک کہہ سکتا ہے کہ سال میں ایک بار کی بجائے وہ ماہوار \عددی{\tfrac{r}{12}} شرح سے سود وصول کرے  گا (جو ظاہری طور پر ربا کی وہی شرح معلوم ہوتی ہے)۔یوں پہلے مہینے کی آخر میں واجب الادا ربا کی مقدار \عددی{\tfrac{r}{12}A_0} اور قرضہ \عددی{A_t=A_0(1+\tfrac{r}{12})} ہو گا۔ اسی طرح دوسرے مہینے کی آخر میں قرضہ \عددی{A_t=A_0(1+\tfrac{r}{12})^2} ہو گا۔ایک سال بعد قرضہ \عددی{A_t=A_0(1+\tfrac{r}{12})^{12}} اور \عددی{t} سال بعد قرضہ \عددی{A_t=A_0(1+\tfrac{r}{12})^{12t}} ہو گا جس کو \عددی{A_t=A_0(1+\tfrac{r}{k})^{kt}} لکھا جا سکتا ہے جہاں \عددی{k=12} ہو گا۔

یہ بینک ماہوار کی بجائے ہفتہ وار سود بھی وصول کر سکتا ہے۔چونکہ سال میں \عددی{52} ہفتے ہوتے ہیں لہٰذا ایسی صورت میں \عددی{k=52} ہو گا اور \عددی{t} سال بعد قرضہ درج ذیل ہو گا۔
\begin{align*}
A_t=A_0\big(1+\frac{r}{k}\big)^{kt}
\end{align*}
سود پر چلنے والا بینک زیادہ سے زیادہ ربا حاصل کرنے کی خاطر، سال میں  زیادہ سے زیادہ مرتبہ ربا حاصل کرنا چاہے گا۔ آئیں دیکھیں کہ \عددی{k\to \infty} کرنے سے \عددی{t} سال بعد قرضہ کتنا ہو گا؟
\begin{align*}
\lim_{k\to \infty}A_t&=\lim_{k\to \infty} A_0\big(1+\frac{r}{k}\big)^{kt}\\
&=A_0e^{rt}
\end{align*}
درج بالا حد کا حصول اگلے حصہ میں سکھایا جائے گا۔ یوں \عددی{t} سال بعد اس شخص پر قرضہ درج ذیل ہو گا۔
\begin{align}\label{مساوات_ماورائی_سود_در_سود}
A(t)=A_0e^{rt}
\end{align}
اس کلیہ کے تحت ربا کو \اصطلاح{مسلسل سود در سود}\فرہنگ{سود در سود!مسلسل}\حاشیہب{compound continuous interest}\فرہنگ{interest!continuous compound} کہتے ہیں۔

\ابتدا{مثال}
آپ آج بینک سے مسلسل سود در سود کی سالانہ \عددی{\SI{15}{\percent}} شرح پر \عددی{\num{100000}} روپیہ حاصل کرتے ہیں۔ پانچ سال بعد آپ کو کتنی مقدار واپس کرنی ہو گی؟ اگر بینک سالانہ سود وصول کرتا ہو تب پانچ سال بعد قرضہ کتنا ہو گا؟

حل:\quad
ہم \عددی{A_0=\num{100000}}، \عددی{r=0.15} اور \عددی{t=5} لیتے ہوئے مساوات \حوالہ{مساوات_ماورائی_سود_در_سود} استعمال کرتے ہیں۔
\begin{align*}
A(5)=\num{100000}e^{(0.15)(5)}=\num{211700}
\end{align*}
اگر بینک سال میں ایک بار ربا وصول کرے تب پانچ سال بعد آپ کو درج ذیل قرضہ دینا ہو گا۔
\begin{align*}
A(5)=\num{100000}(1+0.15)^5=\num{201136}
\end{align*}
\انتہا{مثال}
%=========================
\ابتدا{سوال}
سالانہ \اصطلاح{افراط زر}\فرہنگ{افراط زر}\حاشیہب{inflation}\فرہنگ{inflation} سے مراد ایک سال میں روپیہ کی قدر میں کمی ہے۔ یوں \عددی{\SI{10}{\percent}} افراط زر کا مطلب ہے کہ ایک سال بعد روپیہ کی قیمت \عددی{\SI{90}{\percent}} ہو گی۔

 ایک شخص \عددی{\num{5000000}} روپیہ بینک میں پانچ سال کے لئے جمع کرتا ہے۔بینک ہر مہینہ اس شخص کو \عددی{\num{40000}} روپیہ دیگا اور پانچ سال کے آخر میں اس کو پورے \عددی{5000000} روپیہ واپس کرے گا۔ اگر سالانہ افراط زر \عددی{\SI{12}{\percent}} ہو تب اس شخص نے کیا پایا اور کیا کھویا؟

حل:\quad
پانچ سالوں میں بینک اس شخص کو
\begin{align*}
\num{40000}\times 12 \times 5=\num{2400000}
\end{align*}
روپیہ دیتا ہے۔پانچ سال بعد شخص کو \عددی{5000000} روپیہ دیے جاتے ہیں جن کی اصل قدر
\begin{align*}
\num{5000000}\times 0.88^5=\num{2638660}
\end{align*}
ہو گی۔ یاد رہے کہ ہر مہینہ روپیہ کا قدر کم ہو گا لہٰذا پہلے مہینہ کے \عددی{40000} اور آخری مہینہ کے \عددی{40000} روپیہ کے قدر ایک جیسے نہیں ہوں گے۔ہم حساب کو آسان بنانے کی خاطر تصور کرتے ہیں کہ اس شخص کو ماہوار کی بجائے ہر سال \عددی{\num{40000}\times 12=\num{480000}} روپیہ ملتے ہیں جن کی اصل قدر
\begin{align*}
\num{480000}\times 0.88^1&=\num{422400}\\
\num{480000}\times 0.88^2&=\num{371712}\\
\num{480000}\times 0.88^3&=\num{327107}\\
\num{480000}\times 0.88^4&=\num{287854}\\
\num{480000}\times 0.88^5&=\num{253311}
\end{align*}
ہو گی لہٰذا پانچ سال میں اس کو ماہوار دیے گئے رقم کی اصل قدر درج بالا کا مجموعہ \عددی{\num{1434404}} ہو گا۔

اس شخص کو کل \عددی{\num{2638660}+\num{1434404}=\num{4073064}} قدر کے روپیہ واپس ہوتے ہیں۔ 
\انتہا{سوال}
%======================
\جزوحصہء{تابکاری}
ایک ایٹم اپنی کمیت کا کچھ حصہ خارج کر کے دوسرے ایٹم میں تبدیل ہوتا ہے۔ اس عمل کو \اصطلاح{تابکاری تحلیل}\فرہنگ{تابکاری تحلیل}\حاشیہب{radioactive decay}\فرہنگ{radioactive decay} کہتے ہیں اور جس ایٹم نے مادہ خارج کیا ہو اس کو \اصطلاح{تابکار}\فرہنگ{تابکار}\حاشیہب{radioactive}\فرہنگ{radioactive} کہتے ہیں۔ تابکار کاربن 14 مادہ خارج کر کے نائٹروجن میں تبدیل ہوتا ہے، ریڈیم کئی درمیانی عمل تابکاری  سے گزر کر آخر کار سیسہ میں تبدیل ہوتا ہے۔  

تجربہ سے دیکھا گیا ہے کہ اکائی وقت میں خارج ذرات کی تعداد،  اس وقت تابکار ایٹموں کی تعداد کے تقریباً  راست تناسب ہوتا ہے۔ یوں تابکار تحلیل کو مساوات  \عددی{\tfrac{\dif y}{\dif t}=-ky,\, k>0} ظاہر کرتی ہے۔ اگر لمحہ \عددی{t=0} پر تابکار ایٹموں کی تعداد \عددی{y_0} ہو تب لمحہ \عددی{t} پر درج ذیل ہو گا۔
\begin{align}\label{مساوات_ماورائی_تابکاری_تحلیل}
y&=y_0e^{-kt},\quad k>0&&\text{\RL{مساوات تابکاری}}
\end{align}


\ابتدا{مثال}\ترچھا{نصف زندگی}\\
کسی  عنصر کے آدھے ایٹموں کو تابکاری کے ذریعہ تبدیل ہونے کے لئے درکار وقت کو اس عنصر کی \اصطلاح{نصف زندگی}\فرہنگ{نصف زندگی}\حاشیہب{half life}\فرہنگ{half life} کہتے ہیں۔ کسی بھی عنصر کی نصف زندگی، ابتدائی ایٹموں کی تعداد پر نہیں  بلکہ عنصر پر منحصر ہوتی ہے۔

یہ دیکھنے کی خاطر کہ ایسا کیوں ہوتا ہے ہم ایک عنصر کو لیتے ہیں جس میں لمحہ \عددی{t=0} پر \عددی{y_0} ایٹم پائے جاتے ہوں۔ ہم جاننا چاہتے ہیں کہ کتنے وقت کے بعد اس میں نصف یعنی \عددی{\tfrac{y_0}{2}} ایٹم پائے جائیں گے۔ ہم مساوات \حوالہ{مساوات_ماورائی_تابکاری_تحلیل} استعمال کرتے ہیں۔
\begin{align*}
y\frac{y_0}{2}&=y_0e^{-kt}\\
e^{-kt}&=\frac{1}{2}\\
-kt&=\ln \frac{1}{2}=-\ln 2\\
t&=\frac{\ln 2}{k}
\end{align*}
اس قیمت \عددی{(t=\tfrac{\ln 2}{k})} کو نصف زندگی کہتے ہیں جو صرف \عددی{k} پر منحصر ہے  نا کہ ابتدائی ایٹموں کی تعداد پر۔
\انتہا{مثال}
%=====================
\begin{align}
\text{\RL{نصف زندگی}}=\frac{\ln 2}{k}
\end{align}

ریڈان 222 گیس کے لئے \عددی{k=0.18} دن ہے لہٰذا اس کی نصف زندگی \عددی{3.8} دن ہو گی جبکہ رات کی تاریکی میں  نظر آنے کی خاطر گھڑیوں میں استعمال ہونے والے ریڈیم 226 کا \عددی{k=4.3\times 10^{-4}} سال ہے لہٰذا اس کی نصف زندگی \عددی{1600} سال ہو گی۔

\ابتدا{مثال}\ترچھا{پولونیم 210}\\
پولونیم 210 کی نصف زندگی کو دنوں میں ناپا جاتا ہے۔ اگر \عددی{t=0} پر پولونیم 210  کے \عددی{} ایٹم پائے جاتے ہوں تب \عددی{t} دنوں بعد اس کے \عددی{y=y_0e^{-5\times 10^{-3}t}} ایٹم ہوں گے۔ اس عنصر کی نصف زندگی تلاش کریں۔

حل:\quad
\begin{align*}
\text{\RL{نصف زندگی}}&=\frac{\ln 2}{k}\\
&=\frac{\ln 2}{5\times 10^{-3}}\\
&\approx \text{\RL{139 دن}}
\end{align*}
\انتہا{مثال}
%===================
\ابتدا{مثال}\ترچھا{کاربن 14}\\
کاربن 14 جس کی نصف زندگی \عددی{5700} سال ہے، کو عموماً قدیم چیزوں کی عمر معلوم کرنے کے لئے استعمال کیا جاتا ہے۔ ایک نمونہ میں \عددی{\SI{10}{\percent}} تابکار کاربن کے ایٹم تبدیل تبدیل ہو چکے ہیں۔ اس نمونے کی عمر تلاش کریں۔

حل:\quad
ہمیں پہلے \عددی{k} تلاش کرنا ہے۔اس کے بعد ہم درکار وقت معلوم کریں گے۔ ہم مساوات \حوالہ{مساوات_ماورائی_تابکاری_تحلیل} استعمال کرتے ہیں۔
\موٹا{پہلا قدم:}\quad \ترچھا{$k$ کی تلاش۔}
\begin{align*}
k=\frac{\ln 2}{\text{\RL{}}}=\frac{\ln 2}{5700}\approx 1.2\times 10^{-4}
\end{align*}
\موٹا{دوسرا قدم:}\quad \ترچھا{درکار وقت جس میں $\SI{90}{\percent}$ ایٹم باقی رہ جائے۔}
\begin{align*}
0.9y_0&=y_0e^{-\tfrac{\ln 2}{5700}t}\\
-\frac{\ln 2}{5700}t&=\ln 0.9\\
t&=-\frac{5700(\ln 0.9)}{\ln 2}\approx \text{\RL{$866$ سال}}
\end{align*}
نمونہ \عددی{866} سال پرانا ہے۔
\انتہا{مثال}
%=======================

\جزوحصہء{منتقلی حرارت: نیوٹن کا قانون ٹھنڈک}
کوئی بھی گرم جسم کچھ دیر میں ٹھنڈا ہو کر ارد گرد ماحول کے درجہ حرارت پر آن پہنچتا ہے۔ جسم کے درجہ حرارت میں تبدیلی کی شرح، جسم اور ماحول کے درجہ حرارت میں فرق کے راست متناسب ہوتا ہے۔ اس حقیقت کو \اصطلاح{نیوٹن کا قانون ٹھنڈک} کہتے ہیں۔    

گر لمحہ \عددی{t} پر جسم کا درجہ حرارت متغیر \عددی{T} ہو اور ارد گرد ماحول کا درجہ حرارت مستقل \عددی{T_S} ہو تب
\begin{align}\label{مساوات_ماورائی_قانون_ٹھنڈک}
\frac{\dif T}{\dif t}=-k(T-T_S)
\end{align}
ہو گا۔اگر ہم \عددی{(T-T_S)} کی جگہ \عددی{y} پر کریں تب 
\begin{align*}
\frac{\dif y}{\dif t}&=\frac{\dif}{\dif t}(T-T_S)=\frac{\dif T}{\dif t}-\frac{\dif T_S}{\dif t}\\
&=\frac{\dif T}{\dif t}-0&&\text{\RL{$T_S$ مستقل}}\\
&=\frac{\dif T}{\dif t}
\end{align*}
ہو گا۔یوں \عددی{y} کے لحاظ سے مساوات \حوالہ{مساوات_ماورائی_قانون_ٹھنڈک} درج ذیل ہو گا
\begin{align*}
\frac{\dif y}{\dif t}=-ky
\end{align*} 
جس کا حل \عددی{y=y_0e^{-kt}} ہے۔یوں \اصطلاح{نیوٹن کا قانون ٹھنڈک}\فرہنگ{قانون ٹھنڈک!نیوٹن}\حاشیہب{newton's law of cooling}\فرہنگ{newton!law of cooling} 
\begin{align}\label{مساوات_ماورائی_قانون_ٹھنڈک_ب}
T-T_S&=(T_0-T_S)e^{-kt}&&\text{\RL{نیوٹن کا قانون ٹھنڈک}}
\end{align}
ہو گا جہاں لمحہ \عددی{t=0} پر جسم کا درجہ حرارت \عددی{T_S} ہے۔

\ابتدا{مثال}
ایک انڈے کو \عددی{\SI{98}{\celsius}} پر ابالنے کے بعد  \عددی{\SI{18}{\celsius}} گرم پانی سے بھرے ہوئے بالٹی میں ڈالا جاتا ہے۔پانچ منٹ گزرنے کے بعد انڈے کا درجہ حرارت \عددی{\SI{38}{\celsius}} ہوتا ہے۔ بالٹی میں پانی کے درجہ حرارت میں تبدیلی کو رد کریں۔ انڈا کتنی دیر میں \عددی{\SI{20}{\celsius}} تک پہنچے گا؟

حل:\quad
ہم پانچ منٹ بعد کی معلومات استعمال کرتے ہوئے پہلے \عددی{k} تلاش کرتے ہیں۔ مساوات \حوالہ{مساوات_ماورائی_قانون_ٹھنڈک_ب} کے تحت درج ذیل ہو گا۔
\begin{align*}
T=18+(98-18)e^{-kt}=18+80e^{-kt}
\end{align*} 
پانچ منٹ بعد \عددی{T=38} ہو گا جس سے
\begin{align*}
38&=18+80e^{-5k}\\
e^{-5k}&=\frac{1}{4}\\
-5k&=\ln\frac{1}{4}=-\ln 4\\
k&=\frac{\ln 4}{5}=0.2\ln 4\approx 0.28
\end{align*}
یوں لمحہ \عددی{t} پر \عددی{T=18+80^{-(0.2\ln 4)t}} ہو گا۔ ہمیں وہ \عددی{t} درکار ہے جس پر \عددی{T=20}  ہو گا۔
\begin{align*}
20&=18=80e^{-(0.2\ln 4)t}\\
80e^{-(0.2\ln 4)t}&=2\\
e^{-(0.2\ln 4)t}&=\frac{1}{40}\\
-(0.2\ln 4)t&=\ln \frac{1}{40}=-\ln 40\\
t&=\frac{\ln 40}{0.2\ln 4}\approx \text{\RL{$13$ منٹ}}
\end{align*}
بالٹی میں ڈالنے کے  تقریباً \عددی{13} منٹ بعد انڈے کا درجہ حرارت \عددی{\SI{20}{\celsius}} ہو گا۔
\انتہا{مثال}
%==================

\حصہء{سوالات}

