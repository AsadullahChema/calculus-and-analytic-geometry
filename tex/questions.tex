
\جزوحصہء{سوالات}
\موٹا{بقائی میدان کی پرکھ}\\
سوال \حوالہ{سوال_سمتی_تکمل_کونسا_بقائی_الف} تا سوال \حوالہ{سوال_سمتی_تکمل_کونسا_بقائی_ب} میں کون سے میدان بقائی اور کون سے غیر بقائی ہیں؟

\ابتدا{سوال}\شناخت{سوال_سمتی_تکمل_کونسا_بقائی_الف}
\(\kvec{F}=yz\ai+xz\aj+xy\ak\)
\انتہا{سوال}
%=================
\ابتدا{سوال}
\(\kvec{F}=(y\sin z)\ai+(x\sin z)\aj+(xy\cos z)\ak\)
\انتہا{سوال}
%=================
\ابتدا{سوال}
\(\kvec{F}=y\ai+(x+z)\aj-y\ak\)
\انتہا{سوال}
%=================
\ابتدا{سوال}
\(\kvec{F}=-y\ai+x\aj\)
\انتہا{سوال}
%=================
\ابتدا{سوال}
\(\kvec{F}=(y+z+\ai+z\aj+(x+y)\ak\)
\انتہا{سوال}
%=================
\ابتدا{سوال}\شناخت{سوال_سمتی_تکمل_کونسا_بقائی_ب}
\(\kvec{F}=(e^x \cos y)\ai-(e^x\sin y)\aj+z\ak\)
\انتہا{سوال}
%=================
\موٹا{مخفی قوہ تفاعل کی تلاش}\\
سوال \حوالہ{سوال_سمتی_تکمل_مخفی_قوہ_تفاعل_تلاش_الف} تا سوال \حوالہ{سوال_سمتی_تکمل_مخفی_قوہ_تفاعل_تلاش_ب} میں میدان \عددی{\kvec{F}} کا مخفی قوہ تفاعل \عددی{f} تلاش کریں۔

\ابتدا{سوال}\شناخت{سوال_سمتی_تکمل_مخفی_قوہ_تفاعل_تلاش_الف}
\(\kvec{F}=2x\ai+3y\aj+4z\ak\)
\انتہا{سوال}
%=====================
\ابتدا{سوال}
\(\kvec{F}=(y+z)\ai+(x+z)\aj+(x+y)\ak\)
\انتہا{سوال}
%=====================
\ابتدا{سوال}
\(\kvec{F}=e^{y+2z}(\ai+x\aj+2x\ak)\)
\انتہا{سوال}
%=====================
\ابتدا{سوال}
\(\kvec{F}=(y\sin z)\ai+(x\sin z)\aj+(xy\cos z)\ak\)
\انتہا{سوال}
%=====================
\ابتدا{سوال}\(\kvec{F}=(\ln x+\sec^2(x+y))\ai+\big(\sec^2(x+y)+\frac{y}{y^2+z^2}\big)\aj+\frac{z}{y^2+z^2}\ak\)
\انتہا{سوال}
%=====================
\ابتدا{سوال}\شناخت{سوال_سمتی_تکمل_مخفی_قوہ_تفاعل_تلاش_ب}
{\small{\(\kvec{F}=\frac{y}{1+x^2y^2}\ai+\big(\frac{x}{1+x^2y^2}+\frac{z}{\sqrt{1-y^2z^2}}\big)\aj+\big(\frac{y}{\sqrt{1-y^2z^2}}+\frac{1}{z}\big)\ak\)}}
\انتہا{سوال}
%============================
\موٹا{لکیری تکملات کی قیمتوں کا حصول}\\
سوال \حوالہ{سوال_سمتی_تکمل_لکیری_تکمل_کی_قیمت_الف} تا سوال \حوالہ{سوال_سمتی_تکمل_لکیری_تکمل_کی_قیمت_ب} میں دکھائیں کہ متکمل قطعی تفرقی ہے۔اس کے بعد لکیری تکمل کی قیمت حاصل کریں۔

\ابتدا{سوال}\شناخت{سوال_سمتی_تکمل_لکیری_تکمل_کی_قیمت_الف}
\(\int_{(0,0,0)}^{(2,3,-6)}2x\dif x+2y\dif y+2z\dif z\)
\انتہا{سوال}
%===========================
\ابتدا{سوال}
\(\int_{(1,1,2)}^{(3,5,0)}yz\dif x+xz\dif y+xy\dif z\)
\انتہا{سوال}
%===========================
\ابتدا{سوال}
\(\int_{(0,0,0)}^{(1,2,3)}2xy\dif x+(x^2-z^2)\dif y-2yz\dif z\)
\انتہا{سوال}
%===========================
\ابتدا{سوال}
\(\int_{(0,0,0)}^{(3,3,1)}2x\dif x-y^2\dif y-\frac{4}{1+z^2}\dif z\)
\انتہا{سوال}
%===========================
\ابتدا{سوال}
\(\int_{(1,0,0)}^{(0,1,1)}\sin y\cos x\dif x+\cos y\sin x\dif y+\dif z\)
\انتہا{سوال}
%===========================
\ابتدا{سوال}
\(\int_{(0,2,1)}^{(1,\pi/2,2)}2\cos y\dif x+\big(\frac{1}{y}-2x\sin y\big)\dif y+\frac{1}{z}\dif z\)
\انتہا{سوال}
%===========================
\ابتدا{سوال}
\(\int_{(1,1,1)}^{(1,2,3)}3x^2\dif x+\frac{z^2}{y}\dif y+2z\ln y\dif z\)
\انتہا{سوال}
%===========================
\ابتدا{سوال}
\(\int_{(1,2,1)}^{(2,1,1)}(2x\ln y-yz)\dif x+\big(\frac{x^2}{y}-xz\big)\dif y-xy\dif z\)
\انتہا{سوال}
%===========================
\ابتدا{سوال}\شناخت{سوال_سمتی_تکمل_لکیری_تکمل_کی_قیمت_ب}
\(\int_{(1,1,1)}^{(2,2,2)}\frac{2x\dif x+2y\dif y+2z\dif z}{x^2+y^2+z^2}\)
\انتہا{سوال}
%===========================
\ابتدا{سوال}
نقطہ \عددی{(1,1,1)} سے \عددی{(2,3,-1)} تک قطع کی مقدار معلوم مساواتیں دریافت کر کے اس پر میدان \عددی{\kvec{F}=y\ai+x\aj+4\ak} کی لکیری تکمل
\begin{align*}
\int_{(1,1,1)}^{(2,3,-1)} y\dif x+x\dif y+4\dif z
\end{align*}
 کی قیمت تلاش کریں۔چونکہ \عددی{\kvec{F}} بقائی میدان ہے لہٰذا تکمل کی قیمت راہ سے بے نیاز ہو گی۔
درج ذیل لکیری تکمل مثال \حوالہ{مثال_سمتی_تکمل_لکیری_تکمل}
\انتہا{سوال}
%======================
\ابتدا{سوال}
نقطہ \عددی{(0,0,0)} سے \عددی{(0,3,4)} تک  قطع \عددی{C} پر درج ذیل تکمل کی قیمت تلاش کریں۔
\begin{align*}
\int_C x^2\dif x+yz\dif y+\frac{y^2}{2}\dif z
\end{align*}
\انتہا{سوال}
%==================
\موٹا{نظریہ، عملی استعمال اور مثالیں}\\
