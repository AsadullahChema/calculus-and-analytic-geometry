\حصہ{پابند متغیرات کے تفاعل کے جزوی تفرقات}
اب تک تفاعل، مثلاً  \عددی{w=f(x,y)}،    کے جزوی تفرقات تلاش کرتے ہوئے  ہم \عددی{x} اور \عددی{y} کو بالکل آزاد  غیر تابع متغیرات تصور کرتے رہے ہیں، اگرچہ  عملی زندگی میں ضروری نہیں کہ ایسا ہو۔ مثال کے طور پر ہم گیس کی اندرونی  توانائی  \عددی{U} کو دباو \عددی{P}، حجم \عددی{H} اور حرارت \عددی{T} کا تفاعل \عددی{U=f(P,H,T)} لکھ سکتے ہیں۔ اگر گیس کے انفرادی مالیکیول ایک دوسرے  پر اثر انداز نہ ہوں تب  \عددی{P}، \عددی{H} اور \عددی{T}  مثالی گیس کے    قانون
\begin{align*}
PH&=nRT&&\text{\RL{\عددی{n,R} مستقل ہیں}}
\end{align*}
 کو مطمئن کریں گے لہٰذا یہ متغیرات بالکل آزاد ہرگز نہیں ہوں گے۔ایسی صورت میں جزوی تفرقات کی تلاش  پیچیدہ ثابت ہوتے ہیں۔بہر حال ان سے نمٹنا ضروری ہے۔ 

\جزوحصہء{فیصلہ کریں کہ کون سے متغیرات غیر تابع اور کون سے تابع ہیں}
اگر تفاعل \عددی{w=f(x,y,z)} کے متغیرات کسی تعلق، مثلاً \عددی{z=x^2+y^2}، کے پابند ہوں تب \عددی{f} کے جزوی تفرقات کی جیومیٹریائی  معنی اور عددی قیمت  اس پر منحصر ہوں گے کہ کن متغیرات کو غیر تابع اور کن کو تابع متغیرات تصور کیا جاتا ہے۔ اس انتخاب کے اثرات کو دیکھنے کی خاطر آئیں \عددی{w=x^2+y^2+z^2}  اور \عددی{z=x^2+y^2} کی صورت میں \عددی{\tfrac{\partial w}{\partial x}} تلاش کریں۔ 
 
\ابتدا{مثال}\شناخت{مثال_کثیرالمتغیر_سطح_پابند_جزوی_تفرق}
تفاعل \عددی{w=x^2+y^2+z^2} اور متغیرات کو پابند کرنے والی مساوات \عددی{z=x^2+y^2} کی صورت میں \عددی{\tfrac{\partial w}{\partial x}} تلاش کریں۔

حل:\quad
ہمیں چار متغیرات کی دو مساوات دی گئی ہیں جنہیں ہم دو ( تابع)  متغیرات کے لئے  باقی (غیر تابع ) متغیرات کی صورت میں  حل کر سکتے ہیں۔ جب ہمیں \عددی{\tfrac{\partial w}{\partial x}} تلاش کرنے کو کہا جائے،  اس کا مطلب ہے کہ \عددی{w} تابع متغیر اور \عددی{x} تابع متغیر ہے۔ یوں ہمارے پاس تابع اور غیر تابع متغیرات منتخب کرنے کے درج ذیل ممکنات ہیں۔
\begin{center}
\begin{tabular}{CC}
\text{\RL{تابع}}&\text{\RL{غیر تابع}}\\
w,z&x,y\\
w,y&x,z
\end{tabular}
\end{center}
ہم دونوں صورتوں میں   \عددی{w} کو منتخب غیر تابع متغیرات کی صورت میں  صریحاً لکھ سکتے ہیں۔ایسا کرنے کی خاطر ہم دوسری مساوات   استعمال کرتے ہوئے پہلی مساوات کا    دوسرا تابع متغیر   خارج کرتے ہیں۔ 

پہلی انتخاب میں \عددی{z} دوسرا تابع متغیر ہو گا۔ ہم پہلی مساوات میں اس کی جگہ \عددی{x^2+y^2} پر کر کے اس کو خارج کرتے ہیں۔یوں
\begin{align*}
w&=x^2+y^2+z^2=x^2+y^2+(x^+y^2)^2\\
&=x^2+y^2+x^4+2x^2y^2+y^4
\end{align*}
حاصل ہو گا جس سے
\begin{align}\label{مساوات_کثیرالمتغیر_پہلا_جواب}
\frac{\partial w}{\partial x}=2x+4x^3+4xy^2
\end{align}
 حاصل ہو گا جو  \عددی{x} اور \عددی{y} غیر تابع متغیرات لیتے ہوئے \عددی{\tfrac{\partial w}{\partial x}} کی مساوات ہے۔

دوسری  انتخاب میں غیر تابع متغیرات \عددی{x} اور \عددی{z} ہیں جبکہ دوسرا تابع تغیر \عددی{y} ہے۔یوں \عددی{y} خارج کرنے کی خاطر ہم پہلی مساوات میں \عددی{y^2} کی جگہ \عددی{z-x^2} پر کر کے
\begin{align}
w&=x^2+y^2+z^2=x^2+(z-x^2)+z^2=z+z^2\nonumber\\
\tfrac{\partial w}{\partial x}&=0\label{مساوات_کثیرالمتغیر_دوسرا_جواب}
\end{align}
حاصل کرتے ہیں۔ یوں  غیر تابع متغیرات \عددی{x} اور \عددی{z} منتخب کرنے سے \عددی{\tfrac{\partial w}{\partial x}=0} حاصل ہوتا ہے۔

مساوات \حوالہ{مساوات_کثیرالمتغیر_پہلا_جواب} اور مساوات \حوالہ{مساوات_کثیرالمتغیر_دوسرا_جواب} ایک دوسرے سے بالکل مختلف ہیں۔ ہم \عددی{z=x^2+y^2} استعمال کرتے ہوئے ایک سے دوسری مساوات حاصل نہیں کر سکتے ہیں۔یوں ہمارے پاس  ایک \عددی{\tfrac{\partial w}{\partial x}} کی بجائے  دو نتائج موجود ہیں۔ہم دیکھتے ہیں کہ ہمیں   پوری معلومات فراہم کیے بغیر  جزوی تفرق حاصل کرنے کو کہا گیا۔ہمیں پوچھنا ہو گا کہ کونسا \عددی{\tfrac{\partial w}{\partial x}} درکار ہے؟

ہم مساوات \حوالہ{مساوات_کثیرالمتغیر_پہلا_جواب} اور مساوات \حوالہ{مساوات_کثیرالمتغیر_دوسرا_جواب} کی جیومیٹریائی (شکل \حوالہ{شکل_مثال_کثیرالمتغیر_سطح_پابند_جزوی_تفرق})  مطلب کو دیکھ کر جان سکتے ہیں کہ یہ جوابات ایک دوسرے سے مختلف  کیوں ہیں۔ تفاعل \عددی{w=x^2+y^2+z^2} مبدا  سے نقطہ \عددی{(x,y,z)} کا فاصلہ ناپتا ہے۔ شرط \عددی{z=x^2+y^2} کہتا  ہے کہ نقطہ \عددی{(x,y,z)}  قطع مکافی  کے جسم طواف پر پایا جاتا ہے۔ صرف اس سطح پر چلتے ہوئے نقطہ \عددی{N(x,y,z)} پر  \عددی{\tfrac{\partial w}{\partial x}} سے کیا مراد ہے؟مثال کے طور پر نقطہ \عددی{(1,0,1)} پر \عددی{\tfrac{\partial w}{\partial x}} کی کیا قیمت ہو گی؟

اگر ہم \عددی{x} اور \عددی{y} کو غیر تابع متغیرات لیں تب ہم  \عددی{y} کو مستقل (موجودہ صورت میں \عددی{y=0})  تصور کرتے ہوئے \عددی{x} تبدیل کرتے ہوئے \عددی{\tfrac{\partial w}{\partial x}} تلاش کرتے ہیں۔اس کا مطلب ہے کہ \عددی{N} مستوی \عددی{xz} میں قطع مکافی \عددی{z=x^2} پر حرکت کرے گا۔اس قطع مکافی پر چلتے ہوئے ، \عددی{w} جو مبدا سے \عددی{N} تک فاصلے کا مربع ہے تبدیل ہو گا۔ہم ایسی صورت میں  درج ذیل حاصل کرتے ہیں (جو مذکورہ بالا  ا پہلا نتیجہ ہے۔ )
\begin{align*}
\frac{\partial w}{\partial x}=2x+4x^3+4xy^2
\end{align*}
نقطہ \عددی{(1,0,1)} پر اس کی قیمت درج ذیل ہو گی۔
\begin{align*}
\frac{\partial w}{\partial x}=2+4+0=6
\end{align*}

اگر ہم \عددی{x} اور \عددی{z} کو غیر تابع متغیرات منتخب کریں تب ہم \عددی{z} کو مستقل رکھتے ہوئے \عددی{x} تبدیل کر کے \عددی{\tfrac{\partial w}{\partial x}} تلاش کرتے ہیں۔ چونکہ \عددی{N} کا \عددی{z} محدد \عددی{1} ہے لہٰذا \عددی{x} تبدیل کرنے سے  \عددی{N}  مستوی \عددی{z=1} میں ایک دائرہ پر حرکت کرے گا۔اس دائرہ پر چلتے ہوئے مبدا سے \عددی{N} تک  فاصلہ تبدیل نہیں ہوتا ہے لہٰذا \عددی{w} جو اس فاصلے کا مربع ہے بھی تبدیل نہیں ہو گا۔یوں
\begin{align*}
\frac{\partial w}{\partial x}=0
\end{align*}
ہو گا جو دوسرا انتخاب  کرتے ہوئے ہم حاصل کر چکے ہیں۔
\انتہا{مثال}

%=====================
\begin{figure}
\centering
\begin{tikzpicture}[declare function={f(\x,\y)=(\x)^2+(\y)^2;fx(\r,\t)=\r*cos(\t);fy(\r,\t)=\r*sin(\t);fz(\r,\t)=(\r)^2;}]
\begin{axis}[small,clip=false,axis lines=center,colormap={}{gray(0cm)=(0.6);gray(1cm)=(0.8);},view/h=110,xtick={\empty},ytick={\empty},ztick={\empty},xlabel={$x$},ylabel={$y$},zlabel={$z$},xlabel style={anchor=west},zlabel style={anchor={south}},enlargelimits=true,hide z axis]
\addplot3[z buffer=sort,surf,domain=0:1.2,domain y=-45:360]({fx(x,y)},{fy(x,y)},{fz(x,y)});
\addplot3[domain y=-70:110]({fx(1,\y)},{fy(1,\y)},{fz(1,\y)})node[pin={[align=center,right]20:{\RL{مستوی \عددی{z=1} میں دائرہ}\\ $x^2+y^2=1$}}]{};
\addplot3[dashed,domain y=110:290]({fx(1,\y)},{fy(1,\y)},{fz(1,\y)});
\addplot3[dashed,thick,domain=-1.25:0,samples y=0](\x,0,{(\x)^2})node[pos=0,above,align=center,xshift=1ex]{$z=x^2$\\ $y=0$};
\addplot3[thick,domain=0:1.25,samples y=0](\x,0,{(\x)^2});
\addplot3[]coordinates{(1,0,1)}node[circ]{}node[pin={[fill=white]-145:{$N(1,0,1)$}}]{};
\addplot3[]coordinates{(0,-1.25,{f(0,1.25)})}node[left]{$z=x^2+y^2$};
\end{axis}
\end{tikzpicture}
\caption{نقطہ \عددی{N} کو پابند کرنے سے جزوی تفرقات کے مختلف نتائج حاصل ہوں گے۔}
\label{شکل_مثال_کثیرالمتغیر_سطح_پابند_جزوی_تفرق}
\end{figure}
