\حصہ{قطبی محدد میں تکمل}
اس حصہ میں قطبی محدد استعمال کرتے ہوئے مستوی خطوں کا رقبہ، منحنیات کی لمبائی، اور سطح طواف کا رقبہ حاصل کرنا سکھایا جائے گا۔

\جزوحصہء{مستوی میں رقبہ}
شکل میں خطہ \عددی{OTS} کے حدود  شعاع \عددی{\theta=\alpha}، شعاع \عددی{\theta=\beta} اور منحنی \عددی{r=f(\theta)} ہیں۔  ہم اس خطہ کو \عددی{n} عدد پنکھا نما پٹیوں میں تقسیم کرتے ہیں جو زاویہ \عددی{TOS} کی خانہ بندی \عددی{P} پر مبنی ہے۔ ایک علامتی پٹی کا رداس \عددی{r_k=f(\theta_k)} اور وسطی زاویہ \عددی{\Delta \theta_k} ہو گا جبکہ اس کا رقبہ درج ذیل ہو گا۔
\begin{align*}
S_k=\frac{1}{2}r_k^2\Delta \theta_k=\frac{1}{2}(f(\theta_k))^2\Delta \theta_k
\end{align*}
یوں مکمل خطے کا رقبہ تخمیناً
\begin{align*}
\sum_{k=1}^{n}S_k=\sum_{k=1}^{n}\frac{1}{2}(f(\theta_k))^2\Delta \theta_k
\end{align*}
ہو گا۔ اگر \عددی{f} استمراری ہو تب ہم توقع کرتے ہیں کہ \عددی{\norm{P}\to  0} کرنے سے یہ تخمین بہتر سے بہتر ہو گی لہٰذا خطے کا رقبہ درج ذیل ہو گا۔
\begin{align*}
S&=\lim_{\norm{P}\to 0}\sum_{k=1}^{n}\frac{1}{2}(f(\theta_k))^2\Delta\theta_k\\
&=\int_{\alpha}^{\beta}\frac{1}{2}(f(\theta))^2\dif\theta
\end{align*}

\موٹا{مبدا اور منحنی \عددی{r=f(\theta),\, \alpha\le\theta\le\beta} کے بیچ پنکھا نما خطہ کا رقبہ}\\
\begin{align*}
S=\int_{\alpha}^{\beta}\frac{1}{2}r^2\dif\theta
\end{align*}
یہ درج ذیل تفرقی رقبے کا تکمل ہے۔
\begin{align*}
\dif S=\frac{1}{2}r^2\dif\theta
\end{align*}

\ابتدا{مثال}\شناخت{مثال_مخروط_قلب_نما_رقبہ}
قلب نما \عددی{r=2(1+\cos\theta)} کا رقبہ تلاش کریں۔

حل:\quad
ہم قلب نما کو ترسیم (شکل \حوالہ{شکل_مثال_مخروط_قلب_نما_رقبہ}) کر کے رداس \عددی{OP}  کی نشاندہی کرتے ہیں جو \عددی{\theta=0} تا \عددی{\theta=2\pi} کرنے سے قلب نما پر ٹھیک ایک بار چلتا ہے۔ یہ رقبہ درج ذیل ہو گا۔
\begin{align*}
\int_{\theta=0}^{\theta=2\pi}\frac{1}{2}r^2\dif\theta&=\int_{0}^{2\pi}\frac{1}{2}\cdot 4(1+\cos\theta)^2\dif \theta\\
&=\int_0^{2\pi}2(1+2\cos\theta+\cos^2\theta)\dif \theta\\
&=\int_0^{2\pi}\big(2+4\cos\theta+2\frac{1+\cos2\theta}{2}\big)\dif\theta\\
&=\int_0^{2\pi}(3+4\cos\theta+\cos2\theta)\dif\theta\\
&=\big[3\theta+4\sin\theta+\frac{\sin2\theta}{2}\big]_0^{2\pi}=6\pi-0=6\pi
\end{align*}
\انتہا{مثال}
%============
\begin{figure}
\centering
\begin{minipage}{0.45\textwidth}
\centering
\begin{tikzpicture}[font=\small,declare function={f(\x)=2*(1+cos(\x));}]
\begin{axis}[clip=false,small,axis lines=middle,xlabel={$x$},ylabel={$y$},xlabel style={at={(current axis.right of origin)},anchor=west},ylabel style={at={(current axis.above origin)},anchor=south},enlargelimits=true,xtick={4},xticklabels={\rlap{$4$}},ytick={2}]
\addplot[->-=0.25,data cs=polar,samples=100,domain=0:360]{f(x)}node[pos=0.2,above]{$r=2(1+\cos\theta)$};
\addplot[data cs=polar]plot coordinates {(0,0)(30,{f(30)})}node[pos=0,below right]{$O$}node[pos=0.5,above]{$r$}node[circ]{}node[right]{$N(r,\theta)$};
\addplot[data cs=polar] plot coordinates {(0,4)}node[pin=-45:{$\theta=2\pi$}]{};
\end{axis}
\end{tikzpicture}
\caption{قلب نما کی ترسیم برائے مثال \حوالہ{مثال_مخروط_قلب_نما_رقبہ}}
\label{شکل_مثال_مخروط_قلب_نما_رقبہ}
\end{minipage}\hfill
\begin{minipage}{0.45\textwidth}
\centering
\begin{tikzpicture}[font=\small,declare function={f(\x)=1+2*cos(\x);}]
\begin{axis}[axis on top,clip=false,small,axis lines=middle,xlabel={$x$},ylabel={$y$},xlabel style={at={(current axis.right of origin)},anchor=west},ylabel style={at={(current axis.above origin)},anchor=south},enlargelimits=true,xtick={4},xticklabels={\rlap{$4$}},ytick={2}]
\addplot[data cs=polar,samples=100,domain=0:360]{f(x)}node[pos=0.2,above]{$r=1+2\cos\theta$};
\addplot[data cs=polar,smooth,fill=lgray,domain=120:240]{f(x)};
\addplot[data cs=polar]plot coordinates {(-60,1)(120,1.5)}node[above left]{$\theta=\tfrac{2}{3}\pi$};
\addplot[data cs=polar]plot coordinates {(240,1.5)(60,1)}node[pos=0,below left]{$\theta=\tfrac{4}{3}\pi$};
\addplot[data cs=polar]plot coordinates {(180,{f(180)})}node[pin=45:{$\theta=\pi$}]{};
\end{axis}
\end{tikzpicture}
\caption{رقبہ گھونگا (مثال \حوالہ{مثال_مخروط_رقبہ_چھوٹا_گھیرا})}
\label{شکل_مثال_مخروط_رقبہ_چھوٹا_گھیرا}
\end{minipage}
\end{figure}
\ابتدا{مثال}\شناخت{مثال_مخروط_رقبہ_چھوٹا_گھیرا}
درج ذیل گھونگا کے چھوٹے گھیرے کا رقبہ تلاش کریں۔
\begin{align*}
r=1+2\cos\theta
\end{align*}
حل:\quad
ہم اس گھونگے کو ترسیم کرتے ہیں (شکل \حوالہ{شکل_مثال_مخروط_رقبہ_چھوٹا_گھیرا})۔ ہم دیکھتے ہیں کہ چھوٹے گھیرا \عددی{\theta=\tfrac{2}{3}\pi} اور \عددی{\theta=\tfrac{4}{3}\pi} کے بیچ پایا جاتا ہے۔ ہم نصف رقبہ  \عددی{\theta=\tfrac{2}{3}\pi} تا \عددی{\theta=\pi} تلاش کر کے  اس کو \عددی{2} سے ضرب دیتے ہیں۔
\begin{align*}
S=2\int\limits_{\tfrac{2}{3}\pi}^{\pi}\frac{1}{2}r^2\dif\theta=\int\limits_{\tfrac{2}{3}\pi}^{\pi}r^2\dif \theta
\end{align*}
متکمل \عددی{r^2} کی سادہ صورت حاصل کرتے ہیں۔ 
\begin{align*}
r^2&=(2\cos\theta+1)^2=4\cos^2\theta+4\cos\theta+1\\
&=4\cdot\frac{1+\cos2\theta}{2}+4\cos\theta+1\\
&=2+2\cos2\theta+4\cos\theta+1\\
&=3+2\cos2\theta+4\cos\theta
\end{align*}
یوں رقبہ درج ذیل ہو گا۔
\begin{align*}
S&=\int_{\tfrac{2}{3}\pi}^{\pi} (3+2\cos2\theta+4\cos\theta)\dif \theta\\
&=\big[3\theta+\sin2\theta+4\sin\theta\big]_{\tfrac{2}{3}\pi}^{\pi}\\
&=(3\pi)-\big(2\pi-\frac{\sqrt{3}}{2}+4\cdot\frac{\sqrt{3}}{2}\big)\\
&=\pi-\frac{3\sqrt{3}}{2}
\end{align*}
\انتہا{مثال}
%=====================

ہم شکل \حوالہ{شکل_مخروط_عمومی_رداس_بیچ_رقبہ} طرز خط جو منحنیات \عددی{r_1=r_1(\theta)} اور \عددی{r_2=r_2(\theta)} کے بیچ \عددی{\theta=\alpha} تا \عددی{\theta=\beta}  پایا جاتا ہو، کا رقبہ تلاش کرنے کی خاطر تکمل \عددی{\tfrac{1}{2}r_2^2\dif\theta} سے تکمل \عددی{\tfrac{1}{2}r_1^2\dif\theta} منفی کرتے ہیں۔ اس سے درج ذیل کلیہ حاصل ہوتا ہے۔

\موٹا{خطہ \عددی{0\le r_1(\theta)\le r\le r_2(\theta),\quad \alpha\le\theta\le\beta} کا رقبہ}\\
\begin{align}\label{مساوات_مخروط_رقبہ_قطبی_رداس_کے_بیچ}
S=\int_{\alpha}^{\beta}\frac{1}{2}r_2^2\dif\theta-\int_{\alpha}^{\beta}\frac{1}{2}r_1^2\dif\theta=\int_{\alpha}^{\beta}\frac{1}{2}(r_2^2-r_1^2)\dif\theta
\end{align}

\begin{figure}
\centering
\begin{minipage}{0.45\textwidth}
\centering
\begin{tikzpicture}
\draw[-latex](-0.25,0)--(4,0)node[right]{$x$};
\draw[-latex](0,-0.2)--(0,1.5)node[above]{$y$};
\draw(0,0)node[below right]{$O$}--++(20:3.5)coordinate[pos=0.7](kaa)coordinate(ka)node[pos=0.95,below,yshift=-0.5ex]{$\theta=\alpha$};
\draw(0,0)--++(50:3)coordinate[pos=0.6](kbb)coordinate(kb)node[pos=0.9,left]{$\theta=\beta$};
\draw[thick](ka) to [out=60,in=-20] node[pos=0.5,right]{$r_2$}(kb);
\draw[thick](kaa) to [out=110,in=-60] node[pos=0.5,below left]{$r_1$}(kbb);
\draw[fill=lgray](kaa)--(ka)  to [out=60,in=-20](kb)--(kbb) to [out=-60,in=110](kaa);
\end{tikzpicture}
\caption{سایہ دار رقبہ تلاش کرنے کی خاطر \عددی{r_2} اور مبدا کے بیچ رقبہ سے \عددی{r_1} اور مبدا \عددی{O} کے بیچ رقبہ منفی کیا جاتا ہے۔}
\label{شکل_مخروط_عمومی_رداس_بیچ_رقبہ}
\end{minipage}\hfill
\begin{minipage}{0.45\textwidth}
\centering
\begin{tikzpicture}[font=\small,declare function={f(\x)=1;g(\x)=1-cos(\x);}]
\begin{axis}[clip=false,axis on top,small,axis lines=middle,xlabel={$x$},ylabel={$y$},enlargelimits=true,xlabel style={at={(current axis.right of origin)},anchor=west},ylabel style={at={(current axis.above origin)},anchor=south},xtick={\empty},ytick={\empty}]
\addplot[data cs=polar,smooth,domain=0:360]{f(x)};
\addplot[data cs=polar,smooth,domain=0:360]{g(x)}node[pos=0.25,above,xshift=-2ex]{$r_1=1-\cos\theta$};
\addplot[name path=a,data cs=polar,smooth,domain=-90:90]{f(x)}node[pos=0.85,right,yshift=0.5ex]{$r_2=1$};
\addplot[name path=b,data cs=polar,smooth,domain=-90:90]{g(x)};
\addplot[lgray] fill between [of=a and b];
\addplot[data cs=polar]plot coordinates {(-90,1)}node[pin={[pin distance=0.15cm]-45:{$\theta=-\tfrac{\pi}{2}$\,\, \text{\RL{نچلی حد}}}}]{};
\addplot[data cs=polar]plot coordinates {(90,1)}node[pin=45:{$\theta=\tfrac{\pi}{2}$\,\,\text{\RL{بالائی حد}}}]{};
\addplot[data cs=polar]plot coordinates {(30,0)(30,1)};
\addplot[-stealth,domain=0:30] ({0.4*cos(x)},{0.4*sin(x)})node[pos=0.6,right]{$\theta$};
\end{axis}
\end{tikzpicture}
\caption{دائرہ اور قلب نما کے بیچ رقبہ (مثال \حوالہ{مثال_مخروط_دائرہ_قلب_نما_بیچ_رقبہ})}
\label{شکل_مثال_مخروط_دائرہ_قلب_نما_بیچ_رقبہ}
\end{minipage}
\end{figure}

\ابتدا{مثال}\شناخت{مثال_مخروط_دائرہ_قلب_نما_بیچ_رقبہ}
اس خطے کا رقبہ تلاش کریں جو دائرہ \عددی{r=1} کے اندر اور قلب نما \عددی{r=1-\cos\theta} کے باہر پایا جاتا ہے۔

حل:\quad
ہم رقبہ ترسیم کر کے خطے کے حدود اور  تکمل کے حدود معلوم کرتے ہیں (شکل \حوالہ{شکل_مثال_مخروط_دائرہ_قلب_نما_بیچ_رقبہ})۔ بیرونی منحنی \عددی{r_2=1} جبکہ اندرونی منحنی \عددی{r_1=1-\cos\theta} ہے جبکہ \عددی{\theta} کی قیمت \عددی{-\tfrac{\pi}{2}} تا \عددی{\tfrac{\pi}{2}} ہے۔ مساوات \حوالہ{مساوات_مخروط_رقبہ_قطبی_رداس_کے_بیچ} سے درج ذیل رقبہ حاصل ہو گا۔
\begin{align*}
S&=\int\limits_{-\tfrac{\pi}{2}}^{\tfrac{\pi}{2}}\frac{1}{2}(r_2^2-r_1^2)\dif\theta\\
&=2\int_0^{\pi/2}\frac{1}{2}(r_2^2-r_1^2)\dif\theta&&\text{تشاکلی}\\
&=\int_0^{\pi/2}(1-(1-2\cos\theta)+\cos^2\theta)\dif\theta\\
&=\int_0^{\pi/2}(2\cos\theta-\cos^2\theta)\dif\theta=\int_0^{\pi/2}\big(2\cos\theta-\frac{1+\cos2\theta}{2}\big)\\
&=\big[2\sin\theta-\frac{\theta}{2}-\frac{\sin2\theta}{4}\big]_0^{\pi/2}=2-\frac{\pi}{4}
\end{align*}
\انتہا{مثال}
%===================

\جزوحصہء{منحنی کی لمبائی}
ہم منحنی \عددی{r=f(\theta),\, \alpha\le\theta\le\beta} کی لمبائی کا قطبی کلیہ اخذ کرنے کی خاطر اس منحنی کی درج ذیل مقدار معلوم مساوات لکھتے ہیں۔
\begin{align}\label{مساوات_مخروط_مقدار_معلوم_ایکس_وائے}
x=r\cos\theta=f(\theta)\cos\theta,\quad y=r\sin\theta=f(\theta\sin\theta),\quad \alpha\le\theta\le\beta
\end{align} 
یوں مقدار معلوم لمبائی کے کلیہ (مساوات \حوالہ{مساوات_مخروط_لمبائی_منحنی})
\begin{align*}
L=\int_{\alpha}^{\beta}\sqrt{\big(\frac{\dif x}{\dif\theta}\big)^2+\big(\frac{\dif y}{\dif \theta}\big)^2}\dif\theta
\end{align*}
میں \عددی{x} اور \عددی{y} کی قیمتیں مساوات \حوالہ{مساوات_مخروط_مقدار_معلوم_ایکس_وائے} سے پر کر کے درج ذیل حاصل ہو گا ۔
\begin{align*}
L=\int_{\alpha}^{\beta}\sqrt{r^2+\big(\frac{\dif r}{\dif \theta}\big)^2}\dif\theta &&
\end{align*}

\موٹا{لمبائی قوس}\\
اگر \عددی{\alpha\le\theta\le\beta} پر \عددی{r=f(\theta)} کا پہلا استمراری تفرق پایا جاتا ہو اور اگر \عددی{\theta} کی قیمت \عددی{\alpha} سے \عددی{\beta} کرنے سے  نقطہ \عددی{N(r,\theta)} پوری منحنی \عددی{r=f(\theta)} پر ٹھیک ایک بار چلتا ہو تب اس منحنی کی لمبائی درج ذیل ہو گی۔
\begin{align}\label{مساوات_مخروط_لمبائی_قوس_قطبی_کلیہ}
L=\int_{\alpha}^{\beta}\sqrt{r^2+\big(\frac{\dif r}{\dif \theta}\big)^2}\dif\theta
\end{align} 

\ابتدا{مثال}\شناخت{مثال_مخروط_لمبائی_قوس_قلب_نما}
قلب نما \عددی{r=1-\cos\theta} کی لمبائی دریافت کریں۔

حل:\quad
ہم قلب نما کا خاکہ کھینچتے ہیں تا کہ تکمل کے حدود معلوم کر سکیں (شکل \حوالہ{شکل_مثال_مخروط_لمبائی_قوس_قلب_نما})۔ زاویہ \عددی{\theta} کو \عددی{0} سے \عددی{2\pi} کرنے سے نقطہ \عددی{N(r,\theta)} ٹھیک ایک بار پوری  قلب نما پر گھڑی کے الٹ رخ چلتا ہے لہٰذا \عددی{\alpha=0} اور \عددی{\beta=2\pi} ہوں گے۔ اب 
\begin{align*}
r=1-\cos\theta,\quad \frac{\dif r}{\dif \theta}=\sin\theta
\end{align*}
سے
\begin{align*}
r^2+\big(\frac{\dif r}{\dif \theta}\big)^2&=(1-\cos\theta)^2+(\sin\theta)^2\\
&=1-2\cos\theta+\underbrace{\cos^2\theta+\sin^2\theta}_{1}=2-2\cos\theta
\end{align*}
حاصل ہو گا لہٰذا لمبائی قوس درج ذیل ہو گی۔
\begin{align*}
L&=\int_{\alpha}^{\beta}\sqrt{r^2+\big(\frac{\dif r}{\dif\theta}\big)^2}\dif\theta=\int_0^{2\pi}\sqrt{2-2\cos\theta}\dif\theta\\
&=\int_0^{2\pi}\sqrt{4\sin^2\frac{\theta}{2}}\dif\theta\quad\quad (1-\cos\theta=2\sin^2\tfrac{\theta}{2})\\
&=\int_0^{2\pi}2\abs{\sin\frac{\theta}{2}}\dif\theta\\
&=\int_0^{2\pi}2\sin\frac{\theta}{2}\dif\theta\quad \quad (\text{\RL{\عددی{0\le\theta\le2\pi} کے لئے \عددی{\sin\tfrac{\theta}{2}\ge 0} ہو گا}})\\
&=\big[-4\cos\frac{\theta}{2}\big]_0^{2\pi}=4+4=8
\end{align*} 
\انتہا{مثال}
%====================
\begin{figure}
\centering
\begin{minipage}{0.45\textwidth}
\centering
\begin{tikzpicture}[font=\small,declare function={f(\x)=1-cos(\x);}]
\begin{axis}[axis equal,clip=false,axis on top,small,axis lines=middle,xlabel={$x$},ylabel={$y$},enlargelimits=true,xlabel style={at={(current axis.right of origin)},anchor=west},ylabel style={at={(current axis.above origin)},anchor=south},xtick={-2},ytick={1}]
\addplot[data cs=polar,smooth,domain=0:360]{f(x)}node[pos=0.25,above,xshift=-2ex,yshift=1ex]{$r=1-\cos\theta$};
\addplot[data cs=polar] plot coordinates {(140,0)(140,{f(140)})}node[circ]{}node[pos=0.5,above]{$r$}node[left,yshift=1ex]{$N(r,\theta)$};
\addplot[-stealth,domain=0:140] ({0.4*cos(x)},{0.4*sin(x)})node[pos=0.75,above left]{$\theta$};
\end{axis}
\end{tikzpicture}
\caption{قلب نما کی لمبائی (مثال \حوالہ{مثال_مخروط_لمبائی_قوس_قلب_نما})}
\label{شکل_مثال_مخروط_لمبائی_قوس_قلب_نما}
\end{minipage}\hfill
\begin{minipage}{0.45\textwidth}
\centering
\begin{tikzpicture}[font=\small,declare function={f(\x)=sqrt(cos(2*\x));}]
\begin{axis}[axis equal,clip=false,axis on top,small,axis lines=middle,xlabel={$x$},ylabel={$y$},enlargelimits=true,xlabel style={at={(current axis.right of origin)},anchor=west},ylabel style={at={(current axis.above origin)},anchor=south},xtick={-2},ytick={1}]
\addplot[data cs=polar,smooth,domain=0:90]{f(x)};
\addplot[data cs=polar,smooth,domain=0:90](-x,{f(x)})node[pos=0.75,below right,xshift=4ex,yshift=-1ex]{$r^2=\cos2\theta$};
\addplot[data cs=polar,smooth,domain=90:180]{f(x)};
\addplot[data cs=polar,smooth,domain=90:180](-x,{f(x)});
\addplot[data cs=polar]plot coordinates {(20,0)(20,{f(20)})}node[circ]{}node[right]{$N(r,\theta)$}node[pos=0.6,above]{$r$};
\addplot[-stealth,domain=0:20]({0.3*cos(x)},{0.3*sin(x)})node[pos=0.6,right]{$\theta$};
\addplot[data cs=polar]plot coordinates {(135,0.5)(-45,0.75)}node[below]{$\theta=-\tfrac{\pi}{4}$};
\addplot[data cs=polar]plot coordinates {(-135,0.5)(45,0.75)}node[above]{$\theta=\tfrac{\pi}{4}$};
\end{axis}
\end{tikzpicture}
\caption{قلب نما کی لمبائی (مثال \حوالہ{مثال_سطح_طواف_قطبی})}
\label{شکل_مثال_سطح_طواف_قطبی}
\end{minipage}
\end{figure}


\جزوحصہء{سطح طواف کا رقبہ}
سطح طواف کے رقبہ کا قطبی کلیہ اخذ کرنے کی خاطر ہم مساوات \حوالہ{مساوات_مخروط_مقدار_معلوم_ایکس_وائے} کی مدد سے  منحنی \عددی{r=f(\theta),\,\alpha\le\theta\le\beta} کی مقدار معلوم مساوات  لکھ کر حصہ \حوالہ{حصہ_مخروط_مقدار_معلوم_منحنیات} میں دی گئی سطحی رقبے کا کلیہ استعمال کرتے ہیں۔

\موٹا{سطح طواف کا رقبہ}\\
اگر \عددی{\alpha\le\theta\le\beta} پر \عددی{r=f(\theta)} کا استمراری پہلا تفرق پایا جاتا ہو اور اگر \عددی{\theta} کی قیمت \عددی{\alpha} تا \عددی{\beta} کرنے سے نقطہ \عددی{N(r,\theta)}  منحنی \عددی{r=f(\theta)} پر ٹھیک ایک بار چلتا ہو تب اس منحنی کو محور \عددی{x} اور محور \عددی{y} کے گرد گھما کر حاصل سطح طواف کے رقبے درج ذیل کلیات دیں گے۔ 
\begin{align}
S&=\int_{\alpha}^{\beta}2\pi r\sin\theta\sqrt{r^2+\big(\frac{\dif r}{\dif \theta}\big)^2}\dif\theta&&\text{\RL{محور \عددی{x} کے گرد (\عددی{y\ge 0})}}\label{مساوات_مخروط_سطح_طواف_قطبی_الف}\\
S&=\int_{\alpha}^{\beta}2\pi r\cos\theta\sqrt{r^2+\big(\frac{\dif r}{\dif \theta}\big)^2}\dif\theta&&\text{\RL{محور \عددی{y} کے گرد (\عددی{x\ge 0})}}\label{مساوات_مخروط_سطح_طواف_قطبی_ب}
\end{align}

\ابتدا{مثال}\شناخت{مثال_سطح_طواف_قطبی}
گھونگا \عددی{r^2=\cos2\theta} کے دائیں گھیر کو محور \عددی{y} کے گرد گھما کر سطح طواف پیدا کیا جاتا ہے۔اس سطح کا رقبہ معلوم کریں۔

حل:\quad
ہم اس گھیر کا خاکہ بنا کر تکمل کے حد تلاش کرتے ہیں (شکل \حوالہ{شکل_مثال_سطح_طواف_قطبی})۔ زاویہ \عددی{\theta} کی قیمت \عددی{-\tfrac{\pi}{4}} تا \عددی{\tfrac{\pi}{4}} کرنے سے نقطہ \عددی{N(r,\theta)} منحنی پر ٹھیک ایک بار گھڑی کے الٹ رخ چلتا ہے لہٰذا \عددی{\alpha=-\tfrac{\pi}{4}} اور \عددی{\beta=\tfrac{\pi}{4}} ہوں گے۔

ہم مساوات \حوالہ{مساوات_مخروط_سطح_طواف_قطبی_الف} کا تکمل مرحلوں میں حل کرتے ہیں۔ پہلے مرحلہ میں درج ذیل حاصل کرتے ہیں۔
\begin{align}\label{مساوات_مخروط_ضمنی_سطحہ_طواف}
2\pi r\cos\theta\sqrt{r^2+\big(\frac{\dif r}{\dif \theta}\big)^2}=2\pi \cos\theta\sqrt{r^4+\big(r\frac{\dif r}{\dif \theta}\big)^2}
\end{align}
اس کے بعد \عددی{r^2=\cos2\theta}  لیتے ہوئے درج ذیل حاصل کرتے ہیں۔
\begin{align*}
2r\frac{\dif r}{\dif\theta}&=-2\sin2\theta\\
r\frac{\dif r}{\dif\theta}&=-\sin2\theta\\
\big(r\frac{\dif r}{\dif\theta}\big)^2&=\sin^22\theta
\end{align*}
آخر میں \عددی{r^4=(r^2)^2=\cos^22\theta} کی بنا مساوات \حوالہ{مساوات_مخروط_ضمنی_سطحہ_طواف} میں دائیں ہاتھ جذر درج ذیل صورت اختیار کرے گا۔
\begin{align*}
\sqrt{r^4+\big(r\frac{\dif r}{\dif\theta}\big)^2}=\sqrt{\cos^22\theta+\sin^22\theta}=1
\end{align*}
ان تمام نتائج کو مل کر ہم رقبہ حاصل کرتے ہیں۔
\begin{align*}
S&=\int_{\alpha}^{\beta}2\pi r\cos\theta\sqrt{r^2+\big(\frac{\dif r}{\dif\theta}\big)^2}\dif\theta&&\text{\RL{مساوات \حوالہ{مساوات_مخروط_سطح_طواف_قطبی_الف}}}\\
&=\int_{-\tfrac{\pi}{4}}^{\tfrac{\pi}{4}}2\pi\cos\theta\cdot(1)\dif \theta\\
&=2\pi\big[\sin\theta\big]_{-\pi/4}^{\pi/4}\\
&=2\pi\big[\frac{\sqrt{2}}{2}+\frac{\sqrt{2}}{2}\big]=2\pi\sqrt{2}
\end{align*}
\انتہا{مثال}
%====================

\حصہء{سوالات}
\موٹا{قطبی منحنیات کے اندر رقبہ}\\
سوال \حوالہ{سوال_مخروط_منحنیات_اندر_رقبہ_الف} تا سوال \حوالہ{سوال_مخروط_منحنیات_اندر_رقبہ_ب} میں منحنیات کے اندر رقبہ تلاش کریں۔

\ابتدا{سوال}\شناخت{سوال_مخروط_منحنیات_اندر_رقبہ_الف}
چپٹا گھونگا\عددی{r=4+2\cos\theta} کے اندر۔
\انتہا{سوال}
%===================
\ابتدا{سوال}
قلب نما \عددی{r=a(1+\cos\theta),\, a>0} کے اندر۔
\انتہا{سوال}
%=====================
\ابتدا{سوال}
چار  گل \عددی{r=\cos2\theta} کے ایک پتا کے اندر۔
\انتہا{سوال}
%==================
\ابتدا{سوال}
دو چشمہ \عددی{r^2=2a^2\cos2\theta,\, a>0} کے اندر۔
\انتہا{سوال}
%=============
\ابتدا{سوال}
دو چشمہ \عددی{r^2=4\sin 2\theta} کے ایک گھیر کے اندر۔
\انتہا{سوال}
%===================
\ابتدا{سوال}\شناخت{سوال_مخروط_منحنیات_اندر_رقبہ_ب}
شش گل \عددی{r^2=2\sin3\theta} کے اندر۔
\انتہا{سوال}
%===================

\موٹا{مشترکہ قطبی خطے کا رقبہ}\\
سوال \حوالہ{سوال_مخروط_مشترک_قطبی_رقبہ_الف} تا سوال \حوالہ{سوال_مخروط_مشترک_قطبی_رقبہ_ب} میں مشترکہ قطبی خطے کا رقبہ تلاش کریں۔

\ابتدا{سوال}\شناخت{سوال_مخروط_مشترک_قطبی_رقبہ_الف}
دائرہ \عددی{r=2\cos\theta} اور \عددی{r=2\sin\theta} کا مشترکہ رقبہ۔
\انتہا{سوال}
%==================
\ابتدا{سوال}
دائرہ \عددی{r=1,} اور \عددی{ r=2\sin\theta} کا مشترکہ رقبہ۔
\انتہا{سوال}
%==================
\ابتدا{سوال}
دائرہ \عددی{r=2} اور قلب نما \عددی{r=2(1-\cos\theta)} کا مشترکہ رقبہ۔
\انتہا{سوال}
%==================
\ابتدا{سوال}
قلب نما \عددی{r=2(1+\cos\theta)} اور \عددی{r=2(1-\cos\theta)} کا مشترکہ رقبہ۔
\انتہا{سوال}
%==================
\ابتدا{سوال}
دائرہ \عددی{r=\sqrt{3}} کے باہر اور \عددی{r^2=6\cos2\theta} کے اندر۔
\انتہا{سوال}
%==================
\ابتدا{سوال}
دائرہ \عددی{r=3a\cos\theta} کے اندر اور قلب نما \عددی{r=a(1+\cos\theta),\, a>0} کے باہر رقبہ۔ 
\انتہا{سوال}
%=================
\ابتدا{سوال}
دائرہ \عددی{r=-2\cos\theta} کے اندر اور دائرہ \عددی{r=1} کے باہر رقبہ۔
\انتہا{سوال}
%==================
\ابتدا{سوال}
(ا) گھونگا \عددی{r=2\cos\theta+1} کے بیرونی گھیر کا رقبہ (شکل \حوالہ{شکل_مثال_مخروط_رقبہ_چھوٹا_گھیرا})۔  (ب) گھونگا \عددی{r=2\cos\theta+1} کے اندرونی گھیر کے باہر اور بیرونی گھیر کے اندر رقبہ۔
\انتہا{سوال}
%==================
\ابتدا{سوال}
دائرہ \عددی{r=6} کے اندر خط \عددی{r=3\csc\theta} سے اوپر  رقبہ۔
\انتہا{سوال}
%=================
\ابتدا{سوال}\شناخت{سوال_مخروط_مشترک_قطبی_رقبہ_ب}
گھونگا\عددی{r^2=6\cos2\theta} کے اندر اور خط \عددی{r=\tfrac{3}{2}\sec\theta} کے دائیں جانب رقبہ۔
\انتہا{سوال}
%======================
\ابتدا{سوال}\شناخت{سوال_مخروط_سایہ_دار_خطہ_قطبی}
(ا) سایہ دار خطہ شکل \حوالہ{شکل_سوال_مخروط_سایہ_دار_خطہ_قطبی} میں دکھایا گیا ہے۔ اس  کا رقبہ تلاش کریں۔ (ب) ایسا معلوم ہوتا ہے کہ \عددی{r=\tan\theta,\, -\tfrac{\pi}{2}<\theta<\tfrac{\pi}{2}} لکیر \عددی{x=1} اور لکیر \عددی{x=-1} کا متقاربی خط ہو سکتا ہے۔ کیا ایسا ہے؟ اپنے جواب کی وجہ پیش کریں۔
\انتہا{سوال}
%================
\begin{figure}
\centering
\begin{tikzpicture}[font=\scriptsize,declare function={f(\x)=tan(\x);g(\x)=1/sqrt(2)*cosec(\x);}]
\begin{axis}[small,axis on top,clip=false,axis lines=middle,xtick={-1,1},ytick={\empty},xlabel={$x$},ylabel={$y$},xlabel style={at={(current axis.right of origin)},anchor=west},ylabel style={at={(current axis.above origin)},anchor=south},enlargelimits=true]
\addplot[data cs=polar,domain=-50:50]{f(x)}node[left]{$\begin{aligned} r&=\tan\theta\\ -\tfrac{\pi}{2}&<\theta<\tfrac{\pi}{2}\end{aligned}$};
\addplot[data cs=polar,domain=40:90]{g(x)}node[pos=0.6,below]{$r=\tfrac{\sqrt{2}}{2}\csc\theta$};
\addplot[data cs=polar,domain=-40:-90]{g(x)};
\addplot[data cs=polar]plot coordinates {(45,1)}node[circ]{}node[pin=-70:{$(1,\tfrac{\pi}{4})$}]{};
\addplot[name path=a,data cs=polar,draw=none,domain=-45:0]{f(x)};
\addplot[name path=b,data cs=polar,draw=none,domain=0:45]{f(x)};
\addplot[name path=c,data cs=polar,domain=45:90]{g(x)};
\addplot[name path=d,data cs=polar,domain=-45:-90]{g(x)};
\addplot[lgray] fill between [of ={b and c}];
\addplot[lgray] fill between [of ={a and d}];
\end{axis}
\end{tikzpicture}
\caption{خطہ سوال \حوالہ{سوال_مخروط_سایہ_دار_خطہ_قطبی}}
\label{شکل_سوال_مخروط_سایہ_دار_خطہ_قطبی}
\end{figure}

\ابتدا{سوال}
قلب نما \عددی{r=\cos\theta+1} کے اندر اور دائرہ \عددی{r=\cos\theta} کے باہر خطہ درج ذیل نہیں ہے۔
\begin{align*}
\frac{1}{2}\int_{0}^{2\pi}[(\cos\theta+1)^2-\cos^2\theta]\dif\theta=\pi
\end{align*}
ایسا کیوں ہے؟ اپنے جواب کی وجہ پیش کریں۔
\انتہا{سوال}
%=====================
