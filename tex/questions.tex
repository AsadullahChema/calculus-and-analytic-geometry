\جزوحصہء{دو  شرائط  کے ساتھ لیگرینج ضاربین}
متعدد مسائل میں ہمیں قابل تفرق تفاعل \عددی{f(x,y,z)}  کی انتہائی قیمتیں  اس صورت درکار ہوتی ہیں جہاں تفاعل کے متغیرات دو شرائط کو مطمئن کرتے ہوں۔ اگر یہ شرائط
\begin{align*}
g_2(x,y,z)=0,\quad \text{اور}\quad g_1(x,y,z)=0
\end{align*}
ہوں اور \عددی{g_1}، \عددی{g_2} قابل تفرق ہوں اور ساتھ ہی \عددی{\nabla g_1} اور \عددی{\nabla g_2}  آپس میں متوازی نہ ہوں تب ہم \عددی{f} کی  مشروط مقامی زیادہ سے زیادہ اور مقامی کم سے کم قیمت نقاط تلاش کرنے کی خاطر دو لیگرینج مستقل \عددی{\lambda} اور \عددی{\mu} (جس کا تلفظ  "میو" ہے)  متعارف کرتے ہیں۔ اس طرح \عددی{f} کی انتہائی قیمت نقاط تلاش کرنے کی خاطر ہم  \عددی{x}، \عددی{y}، \عددی{z}، \عددی{\lambda} اور \عددی{\mu} کی وہ قیمتیں دریافت کرتے ہیں جو درج ذیل مساوات کو بیکوقت مطمئن کرتے ہوں۔
\begin{align}\label{مساوات_کثیرالمتغیر_دو_مستقل_لیگرینج_الف}
\nabla f=\lambda\nabla g_1+\mu\nabla g_2,\quad g_1(x,y,z)=0,\quad g_2(x,y,z)=0
\end{align}

درج بالا (مساوات \حوالہ{مساوات_کثیرالمتغیر_دو_مستقل_لیگرینج_الف}) کا  ایک خوبصورت جیومیٹریائی مطلب ہے۔  سطح \عددی{g_1=0} اور سطح \عددی{g_2=0} (عموماً)  ایک ہموار منحنی  میں، جسے ہم \عددی{C} کہیں گے،  ایک دوسرے کو قطع کرتی ہیں اور اس منحنی پر چلتے ہوئے ہم وہ نقاط تلاش کرنا چاہتے ہیں جہاں \عددی{C} پر \عددی{f} کی دیگر قیمتوں کے لحاظ سے   \عددی{f} کی زیادہ سے زیادہ اور کم سے کم قیمت  پائی جاتی ہوں۔ جیسا ہم مسئلہ \حوالہ{مسئلہ_کثیرالمتغیر_عمودی_ڈھلوان_کا_مسئلہ} سے جانتے ہیں، ان نقاط پر \عددی{\nabla f} منحنی \عددی{C} کو عمودی ہو گا۔ لیکن ان نقاط پر چونکہ  منحنی \عددی{C}، سطح  \عددی{g_1=0} اور  سطح \عددی{g_2=0} میں پائی جاتی ہے لہٰذا     \عددی{\nabla g_1} اور \عددی{\nabla g_2} بھی \عددی{C} کو عمودی ہوں گے۔ یوں \عددی{\nabla f} اس مستوی میں پایا جائے گا جسے \عددی{\nabla g_1} اور \عددی{\nabla g_2} تعین کرتے ہیں لہٰذا \عددی{\nabla f=\lambda \nabla g_1+\mu\nabla g_2} ہو گا جہاں \عددی{\lambda} اور \عددی{\mu} کوئی مستقل ہوں گے۔ چونکہ  مطلوبہ نقاط بھی ان دونوں سطحوں میں پائے جاتے ہیں لہٰذا  ان کے محدد مساوات \عددی{g_1(x,y,z)=0} اور \عددی{g_2(x,y,z)=0} کو لازماً مطمئن کریں گے جو مساوات \حوالہ{مساوات_کثیرالمتغیر_دو_مستقل_لیگرینج_الف} کے باقی شرائط ہیں۔

\ابتدا{مثال}
مستوی \عددی{x+y+z=1} بیلن \عددی{x^2+y^2=1} کو ایک ترخیم میں قطع کرتا ہے۔ اس ترخیم پر وہ نقاط  تلاش کریں جو مبدا سے دور تر  اور نزدیک تر ہوں۔

حل:\quad
ہم (مبدا سے نقطہ \عددی{(x,y,z)} کے  فاصلے کے مربع) 
\begin{align*}
f(x,y,z)=x^2+y^2+z^2
\end{align*}
کی وہ انتہائی قیمتیں معلوم کرتے ہیں جو درج ذیل شرائط پر پورا اترتی ہوں۔
\begin{align}
g_1(x,y,z)&=x^2+y^2-1=0\label{مساوات_کثیر_المتغیر_مثال_شرط_الف}\\
g_2(x,y,z)&=x+y+z-1=0\label{مساوات_کثیر_المتغیر_مثال_شرط_ب}
\end{align}
یوں مساوات \حوالہ{مساوات_کثیرالمتغیر_دو_مستقل_لیگرینج_الف} میں  ڈھلوان  کی مساوات
\begin{align*}
\nabla f&=\lambda \nabla g_1+\mu\nabla g_2&&\text{\RL{مساوات \حوالہ{مساوات_کثیرالمتغیر_دو_مستقل_لیگرینج_الف}}}\\
2x\ai+2y\aj+2z\ak&=\lambda(2x\ai+2y\aj)+\mu(\ai+\aj+\ak)\\
2x\ai+2y\aj+2z\ak&=(2\lambda x+\mu)\ai+(2\lambda y+\mu)\aj+\mu \ak
\end{align*}
یعنی
\begin{align}\label{مساوات_کثیر_المتغیر_مثال_شرط_پ}
2x=2\lambda x+\mu,\quad 2y=2\lambda y+\mu,\quad 2z=\mu
\end{align}
دے گی۔غیر سمتی  مساوات \حوالہ{مساوات_کثیر_المتغیر_مثال_شرط_پ}ہمیں درج ذیل دیتی ہیں۔
\begin{gather}
\begin{aligned}\label{مساوات_کثیر_المتغیر_مثال_شرط_ت}
2x&=2\lambda x+2z\quad \implies \quad (1-\lambda)x=z\\
2y&=2\lambda y+2z\quad \implies\quad (1-\lambda)y=z
\end{aligned}
\end{gather}
مساوات \حوالہ{مساوات_کثیر_المتغیر_مثال_شرط_ت} بیکوقت اس صورت مطمئن ہوں گی جب یا \عددی{\lambda=1} اور \عددی{z=0} ہوں   یا \عددی{\lambda\ne 1} اور \عددی{x=y=\tfrac{z}{1-\lambda}} ہوں۔

اگر \عددی{z=0} ہو تب مساوات \حوالہ{مساوات_کثیر_المتغیر_مثال_شرط_الف} اور مساوات \حوالہ{مساوات_کثیر_المتغیر_مثال_شرط_ب}  کو ایک ساتھ حل کرتے ہوئے  ترخیم پر مطابقتی نقاط  \عددی{(1,0,0)} اور \عددی{(0,1,0)} حاصل ہوتے ہیں جو شکل کو دیکھ کر معنی خیز نظر آتے ہیں۔

اگر \عددی{x=y} ہو  تب  مساوات \حوالہ{مساوات_کثیر_المتغیر_مثال_شرط_الف} اور مساوات \حوالہ{مساوات_کثیر_المتغیر_مثال_شرط_ب} درج ذیل دیں گے۔
\begin{align*}
x^2+x^2-1&=0&&x+x+z-1=0\\
2x^2&=1&&z=1-2x\\
x&=\mp\frac{\sqrt{2}}{2}&&z=1\mp\sqrt{2}
\end{align*}
ترخیم پر مطابقتی نقاط
\begin{align*}
N_2=\big(-\frac{\sqrt{2}}{2},-\frac{\sqrt{2}}{2},1+\sqrt{2}\big)\quad \text{اور}\quad N_1=\big(\frac{\sqrt{2}}{2},\frac{\sqrt{2}}{2},1-\sqrt{2}\big)
\end{align*}
ہوں گے۔ یہاں  احتیاط کی ضرورت ہے۔ اگرچہ \عددی{N_1} اور \عددی{N_2} دونوں ترخیم پر \عددی{f} کی  مقامی زیادہ سے زیادہ قیمتیں  دیتے ہیں، نقطہ \عددی{N_2} مبدا سے  زیادہ دور ہے۔

ترخیم پر مبدا کے قریب تر  نقاط \عددی{(1,0,0)} اور \عددی{(0,1,0)} ہیں۔ ترخیم پر مبدا سے دور تر نقطہ \عددی{N_2} ہے۔
\انتہا{مثال}
%================
