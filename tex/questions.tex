\حصہء{سوالات}
\موٹا{ترخیم}\\
سوال \حوالہ{سوال_مخروط_ترخیم_ماسکے_ناظمہ_الف} تا سوال \حوالہ{سوال_مخروط_ترخیم_ماسکے_ناظمہ_ب} میں ترخیم کی سنک تلاش کریں۔ اس کے بعد ترخیم کے ماسکے اور ناظمہ تلاش کر کے ترسیم کریں۔

\ابتدا{سوال}\شناخت{سوال_مخروط_ترخیم_ماسکے_ناظمہ_الف}
$16x^2+25y^2=400$
\انتہا{سوال}
%====================
\ابتدا{سوال}
$7x^2+16y^2=112$
\انتہا{سوال}
%====================
\ابتدا{سوال}
$2x^2+y^2=2$
\انتہا{سوال}
%====================
\ابتدا{سوال}
$2x^2+y^2=4$
\انتہا{سوال}
%====================
\ابتدا{سوال}
$3x^2+2y^2=6$
\انتہا{سوال}
%====================
\ابتدا{سوال}
$9x^2+10y^2=90$
\انتہا{سوال}
%====================
\ابتدا{سوال}
$6x^2+9y^2=54$
\انتہا{سوال}
%====================
\ابتدا{سوال}\شناخت{سوال_مخروط_ترخیم_ماسکے_ناظمہ_ب}
$169x^2+25y^2=4225$
\انتہا{سوال}
%====================

سوال \حوالہ{سوال_مخروط_ترخیم_مساوات_حاصل_الف} تا سوال \حوالہ{سوال_مخروط_ترخیم_مساوات_حاصل_ب} میں ترخیم کے ماسکے یا راس اور سنک دیا گیا ہے۔ ترخیم \عددی{xy} مستوی میں پایا جاتا ہے اور اس کا  مرکز مبدا پر ہے۔ان میں ترخیم کی معیاری مساوات حاصل کریں۔

\ابتدا{سوال}\شناخت{سوال_مخروط_ترخیم_مساوات_حاصل_الف}
ماسکے \عددی{(0,\pm 3)} اور سنک \عددی{0.5}
\انتہا{سوال}
%======================
\ابتدا{سوال}
ماسکے \عددی{(\pm 8,0)} اور سنک \عددی{0.2}
\انتہا{سوال}
%======================
\ابتدا{سوال}
راس \عددی{(0,\pm 70)} اور سنک \عددی{0.1}
\انتہا{سوال}
%======================
\ابتدا{سوال}\شناخت{سوال_مخروط_ترخیم_مساوات_حاصل_ب}
س \عددی{(\pm 10,0)} اور سنک \عددی{0.24}
\انتہا{سوال}
%======================

سوال \حوالہ{سوال_مخروط_ترخیم_ماسکہ_ناظمہ_دیا_الف} تا سوال \حوالہ{سوال_مخروط_ترخیم_ماسکہ_ناظمہ_دیا_ب} میں ترخیم کے ماسکے اور مطابقتی ناظمہ دیے گئے ہیں۔ ترخیم \عددی{xy} مستوی میں پایا جاتا ہے اور اس کا مرکز مبدا پر ہے۔ شکل \حوالہ{شکل_مخروط_مطابقتی_ناظمہ_ماسکہ} کو دیکھ کر ترخیم کی سنک معلوم کریں۔ اس کے بعد ترخیم کی معیاری مساوات حاصل کریں۔

\ابتدا{سوال}\شناخت{سوال_مخروط_ترخیم_ماسکہ_ناظمہ_دیا_الف}
ماسکہ \عددی{(\sqrt{5},0)}، ناظمہ \عددی{x=\tfrac{9}{\sqrt{5}}}
\انتہا{سوال}
%=====================
\ابتدا{سوال}
ماسکہ \عددی{(4,0)}، ناظمہ \عددی{x=\tfrac{16}{3}}
\انتہا{سوال}
%=====================
\ابتدا{سوال}
ماسکہ \عددی{(-4,0)}، ناظمہ \عددی{x=-16}
\انتہا{سوال}
%=====================
\ابتدا{سوال}\شناخت{سوال_مخروط_ترخیم_ماسکہ_ناظمہ_دیا_ب}
ماسکہ \عددی{(\sqrt{2},0)}، ناظمہ \عددی{x=-2\sqrt{2}}
\انتہا{سوال}
%=====================

\ابتدا{سوال}
ایک ترخیم جس کی سنک \عددی{\tfrac{4}{5}} ہو کو ترسیم کریں۔ اپنی حکمت عملی کی وضاحت  کریں۔
\انتہا{سوال}
%====================
\ابتدا{سوال}
سیارہ پلوٹو (جس کی سنک \عددی{0.25} ہے) کا مدار ترسیم کریں۔  اپنی حکمت عملی کی وضاحت  کریں۔
\انتہا{سوال}
%==================
\ابتدا{سوال}
ایک ترخیم کے آخری سر \عددی{(1,1)}، \عددی{(3,4)}، \عددی{(1,7)} اور \عددی{(-1,4)} ہیں۔ اس ترخیم کا خاکہ کھینچیں، اس کی معیاری مساوات، ماسکے، سنک اور ناظمہ تلاش کریں۔
\انتہا{سوال}
%==================
\ابتدا{سوال}
ایک ترخیم کی سنک \عددی{\tfrac{2}{3}}  جبکہ لکیر \عددی{x=9} اس کی ناظمہ اور \عددی{(4,0)} مطابقتی ماسکہ ہے۔ اس ترخیم کی مساوات تلاش کریں۔
\انتہا{سوال}
%=======================
\ابتدا{سوال}
درج ذیل ترخیم \عددی{a}، \عددی{b} اور \عددی{c} کی کن قیمتوں کے لئے مبدا پر \عددی{x} محور کے متوازی ہو گا اور نقطہ \عددی{(-1,2)} سے گزرے گا؟
\begin{align*}
4x^2+y^2+ax+by+c=0
\end{align*}
اس ترخیم کی سنک  کیا ہے؟
\انتہا{سوال}
%======================
\ابتدا{سوال}\شناخت{سوال_مخروط_کمرہ_سرگوشی}\ترچھا{ترخیم کی خاصیت انعکاس}\\
 ایک ترخیم کو اس کے محور اکبر کے گرد گھما کر جسم طواف پیدا کیا جاتا ہے۔ اس ترخیمی جسم کی اندرونی سطح پر چاندی لگا کر ترخیمی آئینہ بنایا جاتا ہے۔ دکھائیں کہ اس ترخیمی آئینہ کے ایک ماسکہ سے خارج شعاع انعکاس کے بعد دوسرے ماسکہ پر پہنچتا ہے۔ صدا بھی اسی راہ کو اپناتا ہے۔ اس حقیقت کو بروئے کار لاتے ہوئے کمرہ سرگوشی بنایا جاتا ہے۔ (اشارہ: ترخیم کو \عددی{xy} مستوی پر معیاری مقام پر رکھ کر دکھائیں کہ کسی بھی نقطہ \عددی{N} سے دونوں ماسکوں تک لکیر، \عددی{N} پر ترخیم کے مماس کے ساتھ ایک جیسے زاویے بناتے ہیں۔)
\انتہا{سوال}
%======================

\موٹا{قطع زائد}\\
سوال \حوالہ{سوال_مخروط_قطع_زائد_ماسکہ_ناظمہ_الف} تا سوال \حوالہ{سوال_مخروط_قطع_زائد_ماسکہ_ناظمہ_ب} میں قطع زائد کی سنک تلاش کریں۔ اس کے بعد قطع زائد کے ماسکہ اور ناظمہ معلوم کر کے ترسیم کریں۔

\ابتدا{سوال}\شناخت{سوال_مخروط_قطع_زائد_ماسکہ_ناظمہ_الف}
$x^2-y^2=1$
\انتہا{سوال}
%=====================
\ابتدا{سوال}
$9x^2-16y^2=144$
\انتہا{سوال}
%=====================
\ابتدا{سوال}
$y^2-x^2=8$
\انتہا{سوال}
%=====================
\ابتدا{سوال}
$y^2-x^2=4$
\انتہا{سوال}
%=====================
\ابتدا{سوال}
$8x^2-2y^2=16$
\انتہا{سوال}
%=====================
\ابتدا{سوال}
$y^2-3x^2=3$
\انتہا{سوال}
%=====================
\ابتدا{سوال}
$8y^2-2x^2=16$
\انتہا{سوال}
%=====================
\ابتدا{سوال}\شناخت{سوال_مخروط_قطع_زائد_ماسکہ_ناظمہ_ب}
$64x^2-36y^2=2304$
\انتہا{سوال}
%=====================

سوال\حوالہ{سوال_مخروط_قطع_زائد_سنک_راس_الف} تا سوال \حوالہ{سوال_مخروط_قطع_زائد_سنک_راس_ب} میں قطع زائد کی سنک اور راس یا ماسکے دیے گئے ہیں۔ یہ قطع زائد \عددی{xy} مستوی میں پائے جاتے ہیں جن کا مرکز مبدا پر ہے۔ ان قطع زائد کی  معیاری مساوات تلاش کریں۔

\ابتدا{سوال}\شناخت{سوال_مخروط_قطع_زائد_سنک_راس_الف}
سنک \عددی{3} راس \عددی{(0,\pm 1)}
\انتہا{سوال}
%====================
\ابتدا{سوال}
سنک \عددی{2} راس \عددی{(\pm 2,0)}
\انتہا{سوال}
%====================
\ابتدا{سوال}
سنک \عددی{3} ماسکے \عددی{(\pm 3,0)}
\انتہا{سوال}
%====================
\ابتدا{سوال}\شناخت{سوال_مخروط_قطع_زائد_سنک_راس_ب}
سنک \عددی{1.25} ماسکے \عددی{(0,\pm 5)}
\انتہا{سوال}
%====================

سوال \حوالہ{سوال_مخروط_سنک_معیاری_الف} تا سوال \حوالہ{سوال_مخروط_سنک_معیاری_ب} میں قطع زائد کے ماسکے اور مطابقتی ناظمہ دیے گئے ہیں۔ یہ قطع زائد \عددی{xy} مستوی میں پائے جاتے ہیں اور ان کا مرکز مبدا پر پایا جاتا ہے۔ قطع زائد کی سنک اور معیاری مساوات تلاش کریں۔

\ابتدا{سوال}\شناخت{سوال_مخروط_سنک_معیاری_الف}
ماسکہ \عددی{(4,0)}، ناظمہ \عددی{x=2}
\انتہا{سوال}
%======================
\ابتدا{سوال}
ماسکہ \عددی{(\sqrt{10},0)}، ناظمہ \عددی{x=\sqrt{2}}
\انتہا{سوال}
%======================
\ابتدا{سوال}
ماسکہ \عددی{(-2,0)}، ناظمہ \عددی{x=-\tfrac{1}{2}}
\انتہا{سوال}
%======================
\ابتدا{سوال}\شناخت{سوال_مخروط_سنک_معیاری_ب}
ماسکہ \عددی{(-6,0)}، ناظمہ \عددی{x=-2}
\انتہا{سوال}
%======================

\ابتدا{سوال}
ایک قطع زائد کی سنک \عددی{\tfrac{3}{2}}  اور ایک ماسکہ \عددی{(1,-3)} ہے۔ اس کا مطابقتی ناظمہ لکیر \عددی{y=2} ہے۔ اس قطع زائد کی مساوات تلاش کریں۔
\انتہا{سوال}
%=================
\ابتدا{سوال}\ترچھا{سنک کی قطع زائد کی صورت پر اثر}\\
سنک بڑھانے سے قطع زائد کی صورت پر کیا اثر ہوتا ہے؟ یہ جاننے کے لئے مساوات \عددی{\tfrac{x^2}{a^2}-\tfrac{y^2}{b^2}=1} کو \عددی{a} اور \عددی{b} کی بجائے \عددی{a} اور \عددی{e} کی صورت میں لکھیں۔ مختلف \عددی{e} کی قیمتوں کے لئے قطع زائد ترسیم کریں (\عددی{a} مستقل لیں)۔
\انتہا{سوال}
%=======================
\ابتدا{سوال}\شناخت{سوال_مخروط_قطع_زائد_آئینہ_الف}\ترچھا{قطع زائد کی خاصیت انعکاس}\\
دکھائیں کہ قطع زائد کے ایک ماسکہ کی طرف گامزن شعاع، قطع زائد سے انعکاس کے بعد  دوسرے ماسکہ پر پہنچتا ہے (شکل\حوالہ{شکل_سوال_مخروط_قطع_زائد_آئینہ_الف})۔ (اشارہ: دکھائیں کہ نقطہ \عددی{N} پر قطع زائد کا مماس  قطع \عددی{NF_1} اور \عددی{NF_2} کے بیچ زاویہ کو نصف میں تقسیم کرتا ہے۔)
\انتہا{سوال}
%===================
\ابتدا{سوال}\شناخت{سوال_مخروط_قطع_زائد_آئینہ_ب}\ترچھا{ہم ماسکہ ترخیم اور قطع زائد}\\
دکھائیں کہ ایک ترخیم اور قطع زائد جن کے  ایک جیسے ماسکے \عددی{A} اور \عددی{B} ہوں، ایک دوسرے کو \عددی{90} درجہ پر قطع کرتے ہیں (شکل \حوالہ{شکل_سوال_مخروط_قطع_زائد_آئینہ_ب})۔ (اشارہ: ماسکہ \عددی{A} سے خارج شعاع قطع زائد کے نقطہ \عددی{N} پر پہنچ کر قطع زائد سے یوں منعکس ہو گا جیسا یہ شعاع ماسکہ \عددی{B} سے خارج ہوا ہو (سوال \حوالہ{سوال_مخروط_قطع_زائد_آئینہ_الف})۔ یہی شعاع ترخیم سے منعکس ہو کر ماسکہ \عددی{B} پر پہنچتا ہے (سوال \حوالہ{سوال_مخروط_کمرہ_سرگوشی})۔)
\انتہا{سوال}
%====================
\begin{figure}
\centering
\begin{minipage}{0.45\textwidth}
\centering
\begin{tikzpicture}[font=\scriptsize,declare function={f(\x)=sqrt(3/6)*sqrt(6+(\x)^2);}]
\pgfmathsetmacro{\a}{sqrt(3)}
\pgfmathsetmacro{\b}{sqrt(6)}
\pgfmathsetmacro{\c}{sqrt(\a^2+\b^2)}
\pgfmathsetmacro{\e}{\c/\a}
\pgfmathsetmacro{\k}{2}
\pgfmathsetmacro{\kk}{5}
\pgfmathsetmacro{\kN}{f(2)}
\begin{axis}[clip=false,small,axis lines=middle,xlabel={$x$},ylabel={$y$},xlabel style={at={(current axis.right of origin)},anchor=west},ylabel style={at={(current axis.above origin)},anchor=south},enlargelimits=true,xtick={\empty},xticklabels={\llap{$1$}},ytick={\empty}]
\addplot[thick,domain=-\kk:\kk]({f(x)},x);
\addplot[thick,domain=-\kk:\kk]({-f(x)},x);
\addplot[]plot coordinates {(-\c,0)}node[circ]{}node[below]{$F_1(-c,0)$};
\addplot[]plot coordinates {(\c,0)}node[circ]{}node[below]{$F_2(c,0)$};
\addplot[]plot coordinates {({f(\k)},\k)}node[circ]{}node[right]{$N(x,y)$};
\addplot[-<-=0.5]plot coordinates {(-\c,0) ({f(\k)},\k)};
\addplot[->-=0.25](1,5)--(\c,0);
\end{axis}
\end{tikzpicture}
\caption{قطع زائد برائے سوال \حوالہ{سوال_مخروط_قطع_زائد_آئینہ_الف}}
\label{شکل_سوال_مخروط_قطع_زائد_آئینہ_الف}
\end{minipage}\hfill
\begin{minipage}{0.45\textwidth}
\centering
\begin{tikzpicture}[font=\scriptsize,declare function={f(\x)=sqrt(3/6)*sqrt(6+(\x)^2);fp(\x)=sqrt(4/13)*sqrt((13-(\x)^2));}]
\pgfmathsetmacro{\a}{sqrt(3)}
\pgfmathsetmacro{\b}{sqrt(6)}
\pgfmathsetmacro{\c}{sqrt(\a^2+\b^2)}
\pgfmathsetmacro{\e}{\c/\a}
\pgfmathsetmacro{\k}{2}
\pgfmathsetmacro{\kk}{5}
\pgfmathsetmacro{\kN}{f(2)}
\begin{axis}[clip=false,small,axis lines=middle,xlabel={$x$},ylabel={$y$},xlabel style={at={(current axis.right of origin)},anchor=west},ylabel style={at={(current axis.above origin)},anchor=south},enlargelimits=true,xtick={\empty},xticklabels={\llap{$1$}},ytick={\empty}]
\addplot[thick,domain=-\kk:\kk]({f(x)},x);
\addplot[thick,domain=-\kk:\kk]({-f(x)},x);
\addplot[]plot coordinates {(-\c,0)}node[circ]{}node[below]{$A$};
\addplot[]plot coordinates {(\c,0)}node[circ]{}node[below]{$B$};
\addplot[]plot coordinates {({f(\k)},\k)}node[circ]{}node[right]{$N(x,y)$};
\addplot[->-=0.5]plot coordinates {(-\c,0) ({f(\k)},\k)};
\addplot[-<-=0.25](1,5)--(\c,0);
\addplot[thick,domain=-sqrt(13)+0.2:sqrt(13)-0.2]{fp(x)};
\addplot[thick,domain=-sqrt(13)+0.2:sqrt(13)-0.2]{-fp(x)};
\addplot[thick,domain=-sqrt(13):-sqrt(13)+0.2]{fp(x)};
\addplot[thick,domain=sqrt(13)-0.2:sqrt(13)]{fp(x)};
\addplot[thick,domain=-sqrt(13):-sqrt(13)+0.2]{-fp(x)};
\addplot[thick,domain=sqrt(13)-0.2:sqrt(13)]{-fp(x)};
\end{axis}
\end{tikzpicture}
\caption{قطع زائد اور ترخیم برائے سوال \حوالہ{سوال_مخروط_قطع_زائد_آئینہ_ب}}
\label{شکل_سوال_مخروط_قطع_زائد_آئینہ_ب}
\end{minipage}
\end{figure}

