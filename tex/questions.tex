\حصہ{انحنا، مروڑ اور \عددی{TNB} چھوکٹ}
اس حصہ میں ہم   تین آپس میں عمودی اکائی سمتیات پر مبنی ایسا چھوکٹ متعارف کرتے ہیں جو  فضا میں منحنی پر جسم کے ساتھ ساتھ چلتا ہو (شکل \حوالہ{شکل_سمتی_تفاعل_چھوکٹ}) ۔ اس چھوکٹ کے تین سمتیات  ہیں۔ پہلا  اکائی مماسی سمتیہ \عددی{\kvec{T}} ہے۔ دوسرا \عددی{\kvec{N}} ہے جو  \عددی{\tfrac{\dif \kvec{R}}{\dif s}} کے رخ اکائی سمتیہ ہے۔تیسرا اکائی سمتیہ  \عددی{\kvec{B}=\kvec{T}\times\kvec{N}} ہے ۔ یہ سمتیات اور ان کے تفرقات  اگر معلوم ہوں، فضا  میں سواری  کی  سمت بندی اور اس کی راہ  میں موڑ  اور بل  کے بارے میں مفید معلومات   مہیا کرتے ہیں۔

مثال کے طور پر  \عددی{\abs{\tfrac{\dif \kvec{R}}{\dif s}}} ہمیں بتاتا ہے  کہ راہ پر  آگے چلتے ہوئے، سواری  کی   راہ کتنی  دائیں یا بائیں  مڑتی  ہے؛ اسی لئے اس کو سواری کی راہ کی\اصطلاح{ انحنا}\فرہنگ{انحنا}\حاشیہب{curvature}\فرہنگ{curvature} کہتے ہیں۔ عدد \عددی{-(\dif \kvec{B}/\dif s)\cdot\kvec{N}} ہمیں بتاتا ہے کہ راہ پر آگے چلتے ہوئے،  سواری کی راہ مستوی حرکت سے کتنی باہر مڑتی ہے یا بل کھاتی ہے؛ اس کو سواری کی راہ کی \اصطلاح{مروڑ}\فرہنگ{مروڑ}\حاشیہب{torsion}\فرہنگ{torsion} کہتے ہیں۔ دوبارہ شکل \حوالہ{شکل_سمتی_تفاعل_چھوکٹ} پر نظر ڈالیں۔ اگر    قوسی راہ پر ایک  ریل گاڑی،   \عددی{P}،   اوپر    چڑھ رہی ہو تب فی اکائی فاصلہ اس   کی   سر بتی جتنی دائیں یا بائیں مڑتی ہو، یہ اس کی انحنا ہو گی۔سمتیات \عددی{\kvec{T}} اور \عددی{\kvec{N}} کے  مستوی سے ریل گاڑی  کا انجن  جس شرح سے باہر  نکلتا  ہو، یہ اس کی مروڑ ہو گی۔ 
\begin{figure}
\centering
\begin{minipage}{0.45\textwidth}
\centering
\pgfmathsetmacro{\ang}{20}
\pgfmathsetmacro{\kang}{20}
\pgfmathsetmacro{\s}{-120}
\pgfmathsetmacro{\e}{65}
\pgfmathsetmacro{\a}{2}
\pgfmathsetmacro{\b}{1.3}
\pgfmathsetmacro{\ka}{2.5}
\pgfmathsetmacro{\kb}{0.5}
\begin{tikzpicture}[font=\small,declare function={fp(\x)=(\x/\a)^2;fn(\x)=(\x/(1.2*\a))^2;}]
\draw[rotate=\ang](0,0) circle (\a cm and \b cm);
\draw[rotate=\ang,->-=0.9,domain=0:\a] plot (\x,{fp(\x)});
\draw[rotate=\ang,domain=0:1.2*\a] plot (-\x,{fn(\x)});
\draw[rotate=\ang](-0.9*1.2*\a,{(fn(0.9*1.2*\a)})++(0,0.1)--++(-90-\ang:0.2)node[below,xshift=-1ex]{$s=0$};
\draw[-latex,thick](0,0)node[below,xshift=-0.5ex]{$P$}--++(\ang:\a)node[pos=0.75,below]{$\kvec{T}$};
\draw[-latex,thick](0,0)--++(90+\ang:\b)node[pos=0.4,left]{$\kvec{N}$};
\draw[rotate=90]([shift={(\s:\ka cm and \kb cm)}]0,0) arc (\s:\e:\ka cm and \kb cm);
\draw(0,0)--++(-90+\ang:\b);
\draw[-latex,thick](0,0)--++(0,\ka)node[pos=0.7,xshift=-1.25ex]{$\kvec{B}$};
\draw[-latex,thick,gray](\ang:\a)--++(90+\ang:0.75)node[pos=0.5,pin={[black,align=center,pin edge=-]45:{\RL{\عددی{P} پر انحنا}\\  $\abs{\dif\kvec{T}/\dif s}$}}]{};
\draw[-latex,thick,gray](0,\ka)--++(90+\ang:0.5)node[right,black]{$\tfrac{\dif \kvec{B}}{\dif s}$};
\draw(90+\ang:2)node[left,align=center]{\RL{\عددی{P} پر مروڑ}\\  $-(\tfrac{\dif \kvec{B}}{\dif s})\cdot \kvec{N}$};
\end{tikzpicture}
\caption{ہر متحرک جسم کے ساتھ  ایک \عددی{\kvec{}}TNB چھوکٹ سفر کرتا ہے جو اس کی راہ کا کردار بیان کرتا ہے۔}
\label{شکل_سمتی_تفاعل_چھوکٹ}
\end{minipage}\hfill
\begin{minipage}{0.45\textwidth}
\centering
\begin{tikzpicture}[font=\small,declare function={f(\x)=0.6+(\x/3)^2;}]
\pgfmathsetmacro{\a}{20}
\draw[-latex](0,0)--(3,0)node[right]{$x$};
\draw[-latex](0,0)--(0,2)node[left]{$y$};
\draw[->-=0.25,domain=0.2:3]plot ({\x},{f(\x)});
\draw(0.3,{f(0.3)})++(0,0.1)--++(0,-0.2)node[below]{$P_0$};
\draw[-latex](1.5,{f(1.5)})node[circ]{}node[above]{$P$}--++(\a:1.25)node[pos=0.75,below]{$\kvec{T}$};
\draw(1,{f(1)})node[below]{$s$};
\end{tikzpicture}
\caption{بڑھتی لمبائی قوس کے رخ چلتے ہوئے اکائی مماسی سمتیہ \عددی{\kvec{T}} مڑتا ہے۔نقطہ \عددی{P} پر \عددی{\abs{\dif \kvec{T}/\dif s}} کی قیمت کو \عددی{P} پر منحنی کی انحنا کہتے ہیں۔}
\label{شکل_سمتی_تفاعل_ریل_گاڑی}
\end{minipage}
\end{figure}

\جزوحصہء{مستوی منحنی کی انحنا}
جیسے جیسے ایک ذرہ مستوی منحنی  میں حرکت کرتا ہے، منحنی کے مڑنے سے  \عددی{\kvec{T}=\tfrac{\dif \kvec{r}}{\dif s}} بھی مڑتا ہے۔ چونکہ \عددی{\kvec{T}} اکائی سمتیہ ہے لہٰذا اس کی لمبائی تبدیل نہیں ہوتی  اور راہ پر چلتے ہوئے  صرف اس کا رخ تبدیل ہوتا ہے۔ منحنی پر چلتے ہوئے اکائی فاصلہ پر \عددی{\kvec{T}}   کی شرح تبدیلی کو انحنا کہتے ہیں (شکل \حوالہ{شکل_سمتی_تفاعل_ریل_گاڑی})۔ انحنا کو روایتی طور پر یونانی حرف  \عددی{\kappa} سے ظاہر کیا جاتا ہے۔

\ابتدا{تعریف}
 ایک  ہموار منحنی جس  کا اکائی مماسی سمتیہ \عددی{\kvec{T}}  ہو، کا تفاعل انحنا درج ذیل ہو گا۔
\begin{align*}
\kappa=\abs{\frac{\dif \kvec{T}}{\dif s}}
\end{align*}
\انتہا{تعریف}
%==========

اگر \عددی{\abs{\dif\kvec{T}/\dif s}} بڑی قیمت ہو تب نقطہ \عددی{P} سے گزرتے ہوئے ذرہ بہت تیزی سے مڑے گا اور \عددی{P} پر انحنا زیادہ ہو گی۔ اگر \عددی{\abs{\dif\kvec{T}/\dif s}} صفر کے قریب ہو تب \عددی{\kvec{T}} کا رخ آہستہ تبدیل ہو گا اور \عددی{P} پر انحنا کم ہو گی۔ اس تعریف کو  پرکھتے ہوئے  ہم درج ذیل دو مثالوں میں دیکھتے ہیں کہ سیدھے خط اور دائروں کی انحنا  مستقل ہو گی۔

\ابتدا{مثال}\ترچھا{سیدھے لکیر کی انحنا صفر ہو گی}\\
سیدھے لکیر پر اکائی مماسی سمتیہ \عددی{\kvec{T}}  کا رخ تبدیل نہیں ہوتا ہے لہٰذا اس کے اجزاء مستقل ہوں گے۔یوں \عددی{\abs{\dif\kvec{T}/\dif s}=\abs{\kvec{0}}=0} ہو گا (شکل \حوالہ{شکل_مثال_سمتی_تفاعل_سیدھی_لکیر_مماسی_سمتیہ})۔
\انتہا{مثال}
%================= 
\ابتدا{مثال}\شناخت{مثال_سمتی_تفاعل_دائرہ_انحنا}\ترچھا{رداس \عددی{a} کے دائرے کی انحنا \عددی{\tfrac{1}{a}} ہو گی}\\
ہم دائرہ کی مقدار معلوم مساوات
\begin{align*}
\kvec{r}(\theta)=(a\cos\theta)\ai+(a\sin\theta)\aj
\end{align*}
میں \عددی{\theta=\tfrac{s}{a}} پر کر کے اس کی لمبائی قوس \عددی{s} کے لحاض سے مقدار معلوم روپ حاصل کرتے ہیں (شکل \حوالہ{شکل_مثال_سمتی_تفاعل_دائرہ_انحنا})۔
\begin{align*}
\kvec{r}=(a\cos\frac{s}{a})\ai+(a\sin\frac{s}{a})\aj
\end{align*}
یوں
\begin{align*}
\kvec{T}=\frac{\dif\kvec{r}}{\dif s}=(-\sin\frac{s}{a})\ai+(\cos\frac{s}{a})\aj
\end{align*}
اور
\begin{align*}
\frac{\dif \kvec{T}}{\dif s}=(-\frac{1}{a}\cos\frac{s}{a})\ai-(\frac{1}{a}\sin\frac{s}{a})\aj
\end{align*}
ہوں گے۔اس طرح کسی بھی \عددی{س} کے لئے درج ذیل ہو گا۔
\begin{align*}
\kappa&=\abs{\frac{\dif \kvec{T}}{\dif s}}\\
&=\sqrt{\frac{1}{a^2}\cos^2\frac{s}{a}+\frac{1}{a^2}\sin^2\frac{s}{a}}\\
&=\frac{1}{\sqrt{a^2}}=\frac{1}{\abs{a}}=\frac{1}{a}\quad\quad \text{\small\RL{\عددی{a>0} کی بنا \عددی{\abs{a}=a} ہو گا}}
\end{align*}
\انتہا{مثال}
%================
\begin{figure}
\centering
\begin{minipage}{0.45\textwidth}
\centering
\begin{tikzpicture}
\pgfmathsetmacro{\a}{20}
\pgfmathsetmacro{\b}{1}
\draw[](0,0)--++(\a:0.25*\b);
\draw[->-=0.5](\a:0.25*\b)node[circ]{}--++(\a:\b)node[pos=0.5,below]{$\kvec{T}$};
\draw[->-=0.5](0,0)++(\a:1.25*\b)node[circ]{}--++(\a:\b);
\draw[->-=0.5](0,0)++(\a:2.25*\b)node[circ]{}--++(\a:\b);
\end{tikzpicture}
\caption{سیدھے لکیر پر \عددی{\kvec{T}} کا رخ تبدیل نہیں ہوتا ہے لہٰذا اس کی انحنا \عددی{\abs{\dif\kvec{T}/\dif s}} صفر ہو گی۔}
\label{شکل_مثال_سمتی_تفاعل_سیدھی_لکیر_مماسی_سمتیہ}
\end{minipage}\hfill
\begin{minipage}{0.45\textwidth}
\centering
\begin{tikzpicture}
\draw[-latex](0,0)--(2,0)node[right]{$x$};
\draw[-latex](0,0)--(0,1.5)node[left]{$y$};
\draw(0,0)node[left]{$O$} circle (1.25);
\draw(0,0)--++(45:1.25)node[circ]{}node[right,yshift=1ex]{$P(a\cos\tfrac{s}{a},a\sin\tfrac{s}{a})$}node[pos=0.7,shift={(90+45:1ex)}]{$a$};
\draw[-stealth]([shift={(0:0.5)}]0,0) arc (0:45:0.5)node[pos=0.6,right]{$\theta$};
\draw(1.25,0)node[circ]{}node[pin={[align=center]-45:{\RL{جڑ}}\\  $(a,0)$}]{};
\draw[thick]([shift={(0:1.25)}]0,0) arc (0:45:1.25)node[pos=0.5,right]{$s=a\theta$};
\end{tikzpicture}
\caption{دائرہ برائے مثال \حوالہ{مثال_سمتی_تفاعل_دائرہ_انحنا}}
\label{شکل_مثال_سمتی_تفاعل_دائرہ_انحنا}
\end{minipage}
\end{figure}

\جزوحصہء{صدر  اکائی عمودی سمتیہ}
چونکہ \عددی{\kvec{T}} کی لمبائی اکائی ہے لہٰذا \عددی{\tfrac{\dif \kvec{T}}{\dif s}} اور \عددی{\kvec{T}} آپس میں عمودی ہوں گے (حصہ \حوالہ{حصہ_سمتی_تفاعل_سمتی_قیمت_تفاعل_اور_فضائی_منحنیات})۔ یوں \عددی{\tfrac{\dif\kvec{T}}{\dif s}} کو لمبائی \عددی{\kappa} سے تقسیم کرنے سے ایسا اکائی  سمتیہ حاصل ہو گا جو   \عددی{\kvec{T}} کو عمودی  ہو گا (شکل \حوالہ{شکل_سمتی_تفاعل_اکائی_مماسی_اور_عمودی_سمتیات})۔ 
\begin{figure}
\centering
\pgfmathsetmacro{\a}{0}
\pgfmathsetmacro{\b}{1.5}
\pgfmathsetmacro{\c}{2.75}
\pgfmathsetmacro{\d}{0.65}
\pgfmathsetmacro{\e}{2.3}
\begin{tikzpicture}[scale=1.5,font=\small,declare function={f(\x)=(\x-\a)*(\x-\b)*(\x-\c);df(\x)=3*\x^2-2*(\b+\c)*\x+\b*\c;}]
\draw[->-=0.2,domain=-0.1:3,smooth]plot ({\x},{f(\x)});
\draw(0.1,{f(0.1)})++(-10:0.1)--++(170:0.2)node[left]{$P_0$};
\draw(0.21,{f(0.21)})node[left]{$s$};
\draw[-latex](\d,{f(\d)})node[circ]{}node[above]{$P_1$}--++(1,{df(\d)})node[above]{$\kvec{T}$};
\draw[-latex](\d,{f(\d)})--++({df(\d)},-1)node[below,xshift=1ex]{$\kvec{N}=\frac{1}{\kappa}\frac{\dif \kvec{T}}{\dif s}$};
\draw[-latex](\e,{f(\e)})node[circ]{}node[below]{$P_2$}--++(1,{df(\e)})node[above]{$\kvec{T}$};
\draw[-latex](\e,{f(\e)})--++({-df(\e)},1)node[above,xshift=1ex]{$\kvec{N}=\frac{1}{\kappa}\frac{\dif\kvec{T}}{\dif s}$};
\end{tikzpicture}
\caption{منحنی کا عمودی سمتیہ \عددی{\tfrac{\dif\kvec{T}}{\dif s}} ہر وقت اس رخ ہوتا ہے جس رخ \عددی{\kvec{T}} مڑتا ہو۔ سمتیہ \عددی{\kvec{N}} کا رخ سمتیہ \عددی{\tfrac{\dif\kvec{T}}{\dif s}} کا رخ ہے۔}
\label{شکل_سمتی_تفاعل_اکائی_مماسی_اور_عمودی_سمتیات}
\end{figure}
\ابتدا{تعریف}
جس نقطہ پر \عددی{\kappa\ne 0} ہو وہاں مستوی میں منحنی کا صدر اکائی سمتیہ \عددی{\kvec{N}}  درج ذیل ہو گا۔
\begin{align*}
\kvec{N}=\frac{1}{\kappa}\frac{\dif\kvec{T}}{\dif s}
\end{align*} 
\انتہا{تعریف}
%===========

موڑ پر سمتیہ \عددی{\tfrac{\dif\kvec{T}}{\dif s}} کا رخ اس جانب ہو گا جس جانب منحنی مڑتی ہو۔یوں اگر بڑھتے فاصلہ کے رخ منہ کرتے ہوئے، اگر \عددی{\kvec{T}}  گھڑی کے رخ مڑے   تب سمتیہ  \عددی{\tfrac{\dif\kvec{T}}{\dif s}}  کا رخ دائیں  ہو گا اور اگر \عددی{\kvec{T}}  گھڑی کے مخالف رخ مڑتی ہو تب اس کا رخ بائیں ہو گا۔  دوسرے لفظوں میں صدر عمودی سمتیہ \عددی{\kvec{N}}منحنی کے   مقعر  رخ ہو گا (شکل \حوالہ{شکل_سمتی_تفاعل_اکائی_مماسی_اور_عمودی_سمتیات})۔  جس  نقطہ پر  \عددی{\kappa=0} ہو، وہاں کے بارے میں سوالات  میں غور کیا گیا ہے۔


 تعریف کی رو سے  منحنی \عددی{\kvec{r}(t)=f(t)\ai+g(t)\aj}  کی لمبائی قوس،   مثبت \عددی{\tfrac{\dif s}{\dif t}} کے لئے ہو گی لہٰذا \عددی{\tfrac{\dif s}{\dif t}=\abs{\tfrac{\dif s}{\dif t}}} ہو گا   اور زنجیری قاعدہ درج ذیل دے گا۔
\begin{align}
\kvec{N}&=\frac{\dif\kvec{T}/\dif s}{\abs{\dif\kvec{T}/\dif s}}\nonumber\\
&=\frac{(\dif\kvec{T}/\dif t)(\dif t/\dif s)}{\abs{\dif\kvec{T}/\dif t}\abs{\dif t/\dif s}}\nonumber\\
&=\frac{\dif\kvec{T}/\dif t}{\abs{\dif\kvec{T}/\dif t}}
\end{align}

اس طرح ہم \عددی{\kappa} اور \عددی{s} حاصل کیے بغیر \عددی{\kvec{N}} حاصل کر سکتے ہیں۔


\ابتدا{مثال}

\انتہا{مثال}
%===============
