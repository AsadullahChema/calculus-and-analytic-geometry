
\حصہء{سوالات}
\موٹا{الجبرائی حساب}

سوال \حوالہ{سوال_ماورائی_سادہ_صورت_فقرہ_الف} تا سوال \حوالہ{سوال_ماورائی_سادہ_صورت_فقرہ_ب} میں ریاضی فقرے کی سادہ صورت تلاش کریں۔

\ابتدا{سوال}\شناخت{سوال_ماورائی_سادہ_صورت_فقرہ_الف}
\begin{multicols}{3}
\begin{enumerate}[a.]
\item
$5^{\log_57}$
\item
$8^{\log_8\sqrt{2}}$
\item
$1.3^{\log_{1.3}75}$
\item
$\log_416$
\item
$\log_3\sqrt{3}$
\item
$\log_4\big(\frac{1}{4}\big)$
\end{enumerate}
\end{multicols}
\انتہا{سوال}
%========================
\ابتدا{سوال}
\begin{multicols}{3}
\begin{enumerate}[a.]
\item
$2^{\log_2 3}$
\item
$10^{\log_{10}(1/2)}$
\item
$\pi^{\log_{\pi}7}$
\item
$\log_{11}121$
\item
$\log_{121}11$
\item
$\log_3\big(\frac{1}{9}\big)$
\end{enumerate}
\end{multicols}
\انتہا{سوال}
%========================
\ابتدا{سوال}
\begin{multicols}{3}
\begin{enumerate}[a.]
\item
$2^{\log_4 x}$
\item
$9^{\log 3 x}$
\item
$\log_2(e^{(\ln 2)(\sin x)})$
\end{enumerate}
\end{multicols}
\انتہا{سوال}
%========================
\ابتدا{سوال}\شناخت{سوال_ماورائی_سادہ_صورت_فقرہ_ب}
\begin{multicols}{3}
\begin{enumerate}[a.]
\item
$25^{\log_5(3x^2)}$
\item
$\log_e(e^x)$
\item
$\log_4(2^{e^x\sin x})$
\end{enumerate}
\end{multicols}
\انتہا{سوال}
%========================
سوال \حوالہ{سوال_ماورائی_قدرتی_لوگاتمی_سادہ_الف} اور سوال \حوالہ{سوال_ماورائی_قدرتی_لوگاتمی_سادہ_ب} میں نسبت کو قدرتی لوگارتھمی صورت میں لکھ کر سادہ صورت حاصل کریں۔

\ابتدا{سوال}\شناخت{سوال_ماورائی_قدرتی_لوگاتمی_سادہ_الف}
\begin{multicols}{3}
\begin{enumerate}[a.]
\item
$\tfrac{\log_2x}{\log_3x}$
\item
$\tfrac{\log_2x}{\log_8x}$
\item
$\tfrac{\log_xa}{\log_{x^2}a}$
\end{enumerate}
\end{multicols}
\انتہا{سوال}
%====================
\ابتدا{سوال}\شناخت{سوال_ماورائی_قدرتی_لوگاتمی_سادہ_ب}
\begin{multicols}{3}
\begin{enumerate}[a.]
\item
$\tfrac{\log_9x}{\log_3x}$
\item
$\tfrac{\log_{\sqrt{10}}x}{\log_{\sqrt{2}}x}$
\item
$\tfrac{\log_ab}{\log_ba}$
\end{enumerate}
\end{multicols}
\انتہا{سوال}
%====================
سوال \حوالہ{سوال_ماورائی_مساوات_حل_الف} تا سوال \حوالہ{سوال_ماورائی_مساوات_حل_ب} میں دی گئی مساوات حل کریں۔

\ابتدا{سوال}\شناخت{سوال_ماورائی_مساوات_حل_الف}
$3^{\log_3(7)+2^{\log_2(5)}}=5^{\log_5(x)}$
\انتہا{سوال}
%==================
\ابتدا{سوال}
$8^{\log_8(3)}-e^{\ln 5}=x^2-7^{\log_7(3x)}$
\انتہا{سوال}
%=====================
\ابتدا{سوال}
$3^{\log_3(x^2)=5e^{\ln x}}-3\cdot 10^{\log_{10}(2)}$
\انتہا{سوال}
%====================
\ابتدا{سوال}\شناخت{سوال_ماورائی_مساوات_حل_ب}
 $\ln e+4^{-2\log_4(x)}=\frac{1}{x}\log_{10}(100)$
\انتہا{سوال}
%==================
سوال \حوالہ{سوال_ماورائی_تفرق_بالحظ_تلاش_الف} تا سوال \حوالہ{سوال_ماورائی_تفرق_بالحظ_تلاش_ب} میں دیے گئے غیر تابع متغیر کے لحاظ سے \عددی{y} کا تفرق تلاش کریں۔

\ابتدا{سوال}\شناخت{سوال_ماورائی_تفرق_بالحظ_تلاش_الف}
$y=2^x$
\انتہا{سوال}
%==========================
\ابتدا{سوال}
$y=3^{-x}$
\انتہا{سوال}
%==========================
\ابتدا{سوال}
$y=5^{\sqrt{s}}$
\انتہا{سوال}
%==========================
\ابتدا{سوال}
$y=2^{s^2}$
\انتہا{سوال}
%==========================
\ابتدا{سوال}
$y=x^{\pi}$
\انتہا{سوال}
%==========================
\ابتدا{سوال}
$y=t^{1-e}$
\انتہا{سوال}
%==========================
\ابتدا{سوال}
$y=(\cos\theta)^{\sqrt{2}}$
\انتہا{سوال}
%==========================
\ابتدا{سوال}
$y=(\ln\theta)^{\pi}$
\انتہا{سوال}
%==========================
\ابتدا{سوال}
$y=7{\sec\theta}\ln 7$
\انتہا{سوال}
%==========================
\ابتدا{سوال}
$y=3^{\tan\theta}\ln 3$
\انتہا{سوال}
%==========================
\ابتدا{سوال}
$y=2^{\sin 3t}$
\انتہا{سوال}
%==========================
\ابتدا{سوال}
$y=5^{-\cos 2t}$
\انتہا{سوال}
%==========================
\ابتدا{سوال}
$y=\log_25\theta$
\انتہا{سوال}
%==========================
\ابتدا{سوال}
$y=\log_3(1+\theta\ln 3)$
\انتہا{سوال}
%==========================
\ابتدا{سوال}
$y=\log_4x+\log_4x^2$
\انتہا{سوال}
%==========================
\ابتدا{سوال}
$y=\log_{25}e^x-\log_5\sqrt{x}$
\انتہا{سوال}
%==========================
\ابتدا{سوال}
$y=\log_2r\cdot\log_4r$
\انتہا{سوال}
%==========================
\ابتدا{سوال}
$y=\log_3r\cdot\log_9r$
\انتہا{سوال}
%==========================
\ابتدا{سوال}
$y=\log_3\big((\tfrac{x+1}{x-1})^{\ln 3}\big)$
\انتہا{سوال}
%==========================
\ابتدا{سوال}
$y=\log_5\sqrt{(\tfrac{7x}{3x+2})^{\ln 5}}$
\انتہا{سوال}
%==========================
\ابتدا{سوال}
$y=\theta\sin(\log_7\theta)$
\انتہا{سوال}
%==========================
\ابتدا{سوال}
$y=\log_7(\tfrac{\sin\theta\cos\theta}{e^{\theta}2^{\theta}})$
\انتہا{سوال}
%==========================
\ابتدا{سوال}
$y=\log_5e^x$
\انتہا{سوال}
%==========================
\ابتدا{سوال}
$y=\log_2(\tfrac{x^2e^2}{2\sqrt{x+1}})$
\انتہا{سوال}
%==========================
\ابتدا{سوال}
$y=3^{\log_2 t}$
\انتہا{سوال}
%==========================
\ابتدا{سوال}
$y=3\log_8(\log_2t)$
\انتہا{سوال}
%==========================
\ابتدا{سوال}
$y=\log_2(8t^{\ln 2})$
\انتہا{سوال}
%==========================
\ابتدا{سوال}\شناخت{سوال_ماورائی_تفرق_بالحظ_تلاش_ب}
$y=t\log_3(e^{(\sin t)(\ln 3)})$
\انتہا{سوال}
%==========================
\موٹا{لوگارتھمی تفرق}

سوال \حوالہ{سوال_ماورائی_لوگارتھمی_تفرق_الف} تا سوال \حوالہ{سوال_ماورائی_لوگارتھمی_تفرق_ب} میں \عددی{y} کا لوگارتھمی تفرق دیے گئے غیر تابع متغیر کے لحاظ سے معلوم کریں۔ 

\ابتدا{سوال}\شناخت{سوال_ماورائی_لوگارتھمی_تفرق_الف}
$y=(x+1)^x$
\انتہا{سوال}
%======================
\ابتدا{سوال}
$y=x^{(x+1)}$
\انتہا{سوال}
%======================
\ابتدا{سوال}
$y=(\sqrt{t})^t$
\انتہا{سوال}
%======================
\ابتدا{سوال}
$y=t^{\sqrt{t}}$
\انتہا{سوال}
%======================
\ابتدا{سوال}
$y=(\sin x)^x$
\انتہا{سوال}
%======================
\ابتدا{سوال}
$y=x^{\sin x}$
\انتہا{سوال}
%======================
\ابتدا{سوال}
$y=x^{\ln x}$
\انتہا{سوال}
%======================
\ابتدا{سوال}\شناخت{سوال_ماورائی_لوگارتھمی_تفرق_ب}
$y=(\ln x)^{\ln x}$
\انتہا{سوال}
%======================
\موٹا{تکمل}\\
سوال \حوالہ{سوال_ماورائی_تکمل_تلاش_کریں-الف} تا سوال \حوالہ{سوال_ماورائی_تکمل_تلاش_کریں-ب} میں تکمل تلاش کریں۔

\ابتدا{سوال}\شناخت{سوال_ماورائی_تکمل_تلاش_کریں-الف}
$\int5^x\dif x$
\انتہا{سوال}
%======================
\ابتدا{سوال}
$\int(1.3)^x\dif x$
\انتہا{سوال}
%======================
\ابتدا{سوال}
$\int_0^12^{-\theta}\dif \theta$
\انتہا{سوال}
%======================
\ابتدا{سوال}
$\int_{-2}^05^{-\theta}\dif \theta$
\انتہا{سوال}
%======================
\ابتدا{سوال}
$\int_1^{\sqrt{2}}x2^{(x^2)}\dif x$
\انتہا{سوال}
%======================
\ابتدا{سوال}
$\int_1^4 \frac{2^{\sqrt{x}}}{\sqrt{x}}\dif x$
\انتہا{سوال}
%======================
\ابتدا{سوال}
$\int_0^{\pi/2}7^{\cos t}\sin t\dif t$
\انتہا{سوال}
%======================
\ابتدا{سوال}
$\int_0^{\pi/4}\big(\frac{1}{3}\big)^{\tan t}\sec^2t\dif t$
\انتہا{سوال}
%======================
\ابتدا{سوال}
$\int_2^4x^{2x}(1+\ln x)\dif x$
\انتہا{سوال}
%======================
\ابتدا{سوال}\شناخت{سوال_ماورائی_تکمل_تلاش_کریں-ب}
$\int_1^2\frac{2^{\ln x}}{x}\dif x$
\انتہا{سوال}
%======================
سوال \حوالہ{سوال_ماورائی_تکمل_حل_کریں_الف} تا سوال \حوالہ{سوال_ماورائی_تکمل_حل_کریں_ب} میں دیے گئے تکمل حل کریں۔

\ابتدا{سوال}\شناخت{سوال_ماورائی_تکمل_حل_کریں_الف}
$\int 3x^{\sqrt{3}}\dif x$
\انتہا{سوال}
%====================
\ابتدا{سوال}
$\int x^{\sqrt{2}-1}\dif x$
\انتہا{سوال}
%====================
\ابتدا{سوال}
$\int_0^3(\sqrt{2}+1)x^{\sqrt{2}}\dif x$
\انتہا{سوال}
%====================
\ابتدا{سوال}\شناخت{سوال_ماورائی_تکمل_حل_کریں_ب}
$\int_1^ex^{(\ln 2)-1}$
\انتہا{سوال}
%====================
سوال \حوالہ{سوال_ماورائی_دیا_تکمل_الف} تا سوال \حوالہ{سوال_ماورائی_دیا_تکمل_ب} میں دیے تکمل کو حل کریں۔

\ابتدا{سوال}\شناخت{سوال_ماورائی_دیا_تکمل_الف}
$\int\frac{\log_{10}x}{x}\dif x$
\انتہا{سوال}
%=======================
\ابتدا{سوال}
$\int_1^4\frac{\log_2 x}{x}\dif x$
\انتہا{سوال}
%=====================
\ابتدا{سوال}
$\int_1^4\frac{\ln 2\log_2 x}{x}\dif x$
\انتہا{سوال}
%=====================
\ابتدا{سوال}
$\int_1^e\frac{2\ln 10 \log_{10}x}{x}\dif x$
\انتہا{سوال}
%=================
\ابتدا{سوال}
$\int_0^2\frac{\log_2(x+2)}{x+2}\dif x$
\انتہا{سوال}
%===================
\ابتدا{سوال}
$\int_{1/10}^{10}\frac{\log_{10}(10x)}{x}\dif x$
\انتہا{سوال}
%===================
\ابتدا{سوال}
$\int_0^9\frac{2\log_{10}(x+1)}{x+1}\dif x$
\انتہا{سوال}
%===================
\ابتدا{سوال}
$\int_2^3\frac{2\log_2(x-1)}{x-1}\dif x$
\انتہا{سوال}
%===================
\ابتدا{سوال}
$\int\frac{\dif x}{x\log_{10}x}$
\انتہا{سوال}
%===================
\ابتدا{سوال}\شناخت{سوال_ماورائی_دیا_تکمل_ب}
$\int\frac{\dif x}{x(\log_8x)^2}$
\انتہا{سوال}
%===================
سوال \حوالہ{سوال_ماورائی_تکمل_کی_قیمت_الف} تا سوال \حوالہ{سوال_ماورائی_تکمل_کی_قیمت_ب} میں تکمل کی قیمت تلاش کریں۔

\ابتدا{سوال}\شناخت{سوال_ماورائی_تکمل_کی_قیمت_الف}
$\int_1^{\ln x}\frac{1}{t}\dif t,\quad x>1$
\انتہا{سوال}
%=====================
\ابتدا{سوال}
$\int_1^{e^x}\frac{1}{t}\dif t$
\انتہا{سوال}
%=====================
\ابتدا{سوال}
$\int_1^{1/x}\frac{1}{t}\dif t,\quad x>0$
\انتہا{سوال}
%=====================
\ابتدا{سوال}\شناخت{سوال_ماورائی_تکمل_کی_قیمت_ب}
$\frac{1}{\ln a}\int_1^x\frac{1}{t}\dif t,\quad x>0$
\انتہا{سوال}
%=====================
\موٹا{نظریہ اور استعمال}

\ابتدا{سوال}
منحنی \عددی{y=\tfrac{2x}{1+x^2}} اور محور \عددی{x} پر \عددی{-2\le x\le 2}  کے بیچ خطے کا رقبہ معلوم کریں۔
\انتہا{سوال}
%=====================
\ابتدا{سوال}
منحنی \عددی{y=2^{1-x}} اور محور \عددی{x} پر \عددی{-1\le x\le 1}  کے بیچ خطے کا رقبہ معلوم کریں۔
\انتہا{سوال}
%==========================
\ابتدا{سوال}\ترچھا{انسانی خون کا $\pH$}\\
انسانی خون کے \عددی{\pH} کی قیمت \عددی{7.37} سے \عددی{7.44} تک ہوتی ہے۔ انسانی خون میں برق پارہ \عددی{[\ce{H3O^+}]} کے مطابقتی حدود تلاش کریں۔ 
\انتہا{سوال}
%========================
\ابتدا{سوال}\ترچھا{دماغی سیال کا $\pH$}\\
دماغی سیال میں \عددی{[\ce{H3O^+}]} کا گھنا پن تقریباً \عددی{\SI{4.8e-8}{\mole\per\liter}} ہے۔ اس سیال کا \عددی{\pH} تلاش کریں۔
\انتہا{سوال}
%====================
\ابتدا{سوال}
افزائش کار (ایمپلی فائر) سے حاصل صدا کو جزو \عددی{k} سے ضرب دے کر اس سطح صدا کو \عددی{\SI{10}{\deci\bel}} مزید بلن کیا جاتا ہے۔ جزو \عددی{k} کی قیمت تلاش کریں۔
\انتہا{سوال}
%===================
\ابتدا{سوال}
ایک افزائش کار صدا کی شدت کو \عددی{10} سے ضرب دیتا ہے۔ صدا میں کتنے \عددی{\si{\deci\bel}} کا اضافہ پیدا ہو گا؟ 
\انتہا{سوال}
%===================
\ابتدا{سوال}
کسی بھی محلول میں \عددی{[\ce{H3O^+}]}  اور \عددی{[OH^-]} کی گھنا پن کا حاصل ضرب \عددی{10^{-14}} ہوتا ہے۔
\begin{enumerate}[a.]
\item
\عددی{[\ce{H3O^+}]} کی کیا قیمت گھنا پن کی مجموعی \عددی{S=[\ce{H3O^+}]+[\ce{OH^-}]} کو کم سے کم  کرتی ہے؟
\item
اس محلول کی \عددی{\pH} تلاش کریں  جس میں \عددی{S} کی قیمت کم سے کم ہو۔
\item
\عددی{[\ce{H3O^+}]} اور \عددی{[OH^-]} کی کون سی نسبت \عددی{S} کو کم سے کم بناتی ہے؟
\end{enumerate}
\انتہا{سوال}
%==================
\ابتدا{سوال}
کیا \عددی{\log_ab} کی قیمت \عددی{\tfrac{1}{\log_ba}} کے برابر ہو سکتی ہے؟ اپنے جواب کی وجہ   پیش کریں۔
\انتہا{سوال}
%===================
\موٹا{کمپیوٹر کا استعمال}

\ابتدا{سوال}
مساوات \عددی{x^2=2^x} کے دو حل \عددی{x=2} اور  \عددی{x=4} ہیں جبکہ اس کا تیسرا حل بھی پایا جاتا ہے۔ ترسیم کی مدد سے تیسرا حل تلاش کریں۔
\انتہا{سوال}
%===================
\ابتدا{سوال}
کیا \عددی{x>0} کے لئے  \عددی{x^{\ln 2}} اور \عددی{2^{\ln x}} ایک دوسرے کے برابر ہو سکتے ہیں؟دونوں تفاعل ترسیم کرتے ہوئے بتائیں کیا ہوتا ہے۔
\انتہا{سوال}
%====================
\ابتدا{سوال}\ترچھا{$2^x$ کی خط بندی}\\
(ا) نقطہ \عددی{x=0} پر \عددی{f(x)=2^x} کی خط بندی دریافت کریں۔ اس کے بعد عددی سروں کو \عددی{2} اعشاریہ  پور و پور کریں۔ (ب) وقفہ \عددی{-3\le x\le 3} اور وقفہ \عددی{-1\le x\le 1} کے لئے تفاعل اور خط بندی کو ایک ساتھ ترسیم کریں۔
\انتہا{سوال}
%===================
\ابتدا{سوال}\ترچھا{$f(x)=\log_3 x$ کی خط بندی}\\
(ا) نقطہ \عددی{x=3} پر \عددی{f(x)=\log_3 x} کی خط بندی تلاش کریں۔ اس کے بعد عددی سروں کو \عددی{2} اعشاریہ تک پور و پور کریں۔    (ب) وقفہ \عددی{0\le x\le 8} اور \عددی{2\le x\le} کے لئے تفاعل اور خط بندی کو ایک ساتھ ترسیم کریں۔
\انتہا{سوال}
%====================
\موٹا{دیگر اساس کے ساتھ حساب کتاب}

\ابتدا{سوال}
عموماً کیلکولیٹروں میں \عددی{\log_{10}x} اور \عددی{\ln x} پائے جاتے ہیں۔ دیگر اساس کے لوگارتھم تلاش کرنے کی خاطر ہم درج ذیل مساوات استعمال کرتے ہیں۔
\begin{align*}
\log_ax=\frac{\ln x}{\ln a}
\end{align*}
یوں درج ذیل ہو گا۔
\begin{align*}
\log_25=\frac{\ln 5}{\ln 2}\approx 2.3219
\end{align*}
کیلکولیٹر استعمال کرتے ہوئے  \عددی{5} اعشاریہ درستگی تک 
(ا) \عددی{\log_38}، (ب) \عددی{\log_70.5}، (ج) \عددی{\log_{2-}17}، (د) \عددی{\log_{0.5}7} تلاش کریں۔ درج ذیل معلومات استعمال کرتے ہوئے \عددی{\ln x} تلاش کریں۔
(ہ) \عددی{\log_{10}x=2.3}،  (و) \عددی{\log_2x=1.4}،  (ز) \عددی{\log_2x=-1.5}،  (ح) \عددی{\log_{10}x=-0.7}
\انتہا{سوال}
%=====================
\ابتدا{سوال}\ترچھا{تبدیلی پیمانہ}\\
(ا) دکھائیں کہ اساس \عددی{10} لوگارتھم کو اساس \عددی{2} لوگارتھم میں تبدیل کرنے کی مساوات درج ذیل ہے۔
\begin{align*}
\log_2x=\frac{\ln 10}{\ln 2}\log_{10}x
\end{align*}
(ب) دکھائیں کہ اساس  \عددی{a} لوگارتھم کو اساس \عددی{b} لوگارتھم میں تبدیل کرنے  کی مساوات درج ذیل ہے۔
\begin{align*}
\log_bx=\frac{\ln a}{\ln b}\log_ax
\end{align*}
\انتہا{سوال}
%=======================

\حصہ{افزائش اور تنزل}

