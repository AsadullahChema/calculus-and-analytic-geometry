\حصہ{حد اور استمرار}
اس حصہ میں کثیر المتغیر تفاعل کی حد اور استمرار پر غور کیا جائے گا۔

\جزوحصہء{حد}
اگر نقطہ \عددی{(x_0,y_0)} کے قریب   تمام نقاط \عددی{(x,y) } کے لئے تفاعل \عددی{f(x,y)}  کی قیمتیں کسی  مقررہ حقیقی عدد \عددی{L} کے  بہت  زیادہ  قریب ہوں تب ہم کہتے ہیں کہ جیسے جیسے \عددی{(x,y)} نقطہ \عددی{(x_0,y_0)} تک  پہنچنے کی کوشش کرتا ہے، تفاعل \عددی{f} کی قیمت \عددی{L} تک پہنچنے کی کوشش کرتی ہے۔ یہ تعریف، واحد متغیر کے تفاعل کی حد کی تعریف کی مانند ہے۔البتہ، دھیان رہے کہ اگر \عددی{(x_0,y_0)} تفاعل \عددی{f}  کے دائرہ کار  کی اندرون میں پایا جاتا ہو تب \عددی{(x,y)} نقطہ \عددی{(x_0,y_0)} تک کسی بھی رخ سے پہنچنے کی کوشش کر سکتا ہے۔جیسا آپ نیچے  دی گئی   مثالوں میں سے چند  میں دیکھیں گے،  قریب پہنچنے کا رخ بعض اوقات مسئلہ کھڑا کر سکتا ہے۔

\ابتدا{تعریف}
اگر ہر عدد \عددی{\epsilon>0} کے لئے ایسا مطابقتی عدد \عددی{\delta>0} پایا جاتا ہو کہ \عددی{f} کے دائرہ کار میں تمام \عددی{(x,y)} کے لئے 
\begin{align}
0<\sqrt{(x-x_0)^2+(y-y_0)^2}<\delta\implies \abs{f(x,y)-L}<\epsilon
\end{align}
ہو ، تب ہم کہتے ہیں کہ \عددی{(x,y)} کا  \عددی{(x_0,y_0)} تک پہنچنے سے \عددی{f(x,y)} کی قیمت \اصطلاح{حد}\فرہنگ{حد}\حاشیہب{limit}\فرہنگ{limit} \عددی{L} تک پہنچتی ہے جس کو ہم درج ذیل لکھتے ہیں۔
\begin{align*}
\lim_{(x,y)\to(x_0,y_0)}f(x,y)=L
\end{align*}
\انتہا{تعریف}
%====================

حد کی تعریف میں \عددی{\delta \sigma} کی شرط  اس کی  معادل ہے  کہ، کسی بھی \عددی{\epsilon>0} کے لئے ایسا مطابقتی \عددی{\delta>0} پایا جاتا ہو کہ تمام \عددی{x} کے لئے درج ذیل ہو۔
\begin{align}
0<\abs{x-x_0}<\delta\quad \text{اور}\quad 0<\abs{y-y_0}<\delta\implies \abs{f(x,y)-L}<\epsilon
\end{align}
یوں حد کی قیمت  تلاش  ہوئے ہم مستوی میں فاصلوں کی صورت یا محدد میں فرق کی صورت میں سوچ سکتے ہیں۔

حد کی تعریف، تفاعل \عددی{f} کے دائرہ کار کی اندرون  کے ساتھ   سرحدی نقاط \عددی{(x_0,y_0)} کے لئے بھی  قابل استعمال ہے۔  بس اتنا ضروری ہے کہ نقطہ \عددی{(x,y)} ہر وقت دائرہ کار کے اندر رہے۔

واحد متغیر کے تفاعل کی طرح  درج ذیل دکھائے جا  سکتے ہیں۔
\begin{gather}
\begin{aligned}\label{مساوات_کثیر_المتغیر_قواعد_حد}
\lim_{(x,y)\to(x_0,y_0)}x&=x_0\\
\lim_{(x,y)\to(x_0,y_0)}y&=y_0\\
\lim_{(x,y)\to(x_0,y_0)}k&=k\quad \text{\RL{\عددی{k} کوئی بھی عدد ہو سکتا ہے}}
\end{aligned}
\end{gather}
یہ بھی دکھایا جا سکتا ہے کہ دو تفاعل کے مجموعہ کا حد، ان تفاعل کے انفرادی حد   (اگر دونوں موجود ہوں)کا مجموعہ ہو گا۔اسی طرح   کے نتائج   فرق، حاصل ضرب، حاصل تقسیم، مستقل مضرب اور طاقت کے لئے بھی دکھائے جا سکتے ہیں۔

\ابتدا{مسئلہ}\شناخت{مسئلہ_کثیر_المتغیر_قواعد_حد}\موٹا{دو متغیرات کے تفاعل کی حد کے خواص}\\
اگر 
\begin{align*}
\lim_{(x,y)\to(x_0,y_0)}f(x,y)=L\quad \text{اور}\quad \lim_{(x,y)\to(x_0,y_0)}g(x,y)=M
\end{align*}
ہوں تب درج ذیل قواعد کارآمد  ہوں گے۔
\begin{description}
\item{قاعدہ مجموعہ:}
$\lim[f(x,y)+g(x,y)]=L+M$
\item{قاعدہ فرق:}
$\lim[f(x,y)-g(x,y)]=L-M$
\item{قاعدہ مستقل مضرب:}
$\lim kf(x,y)=kL$
جہاں \عددی{k} کوئی مستقل ہے۔
\item{قاعدہ حاصل تقسیم:}
$\lim\frac{f(x,y)}{g(x,y)}=\frac{L}{M}$
اگر  \عددی{M\ne 0} ہو۔
\item{قاعدہ طاقت:}
$\lim[f(x,y)]^{m/n}=L^{m/n}$
اگر  \عددی{m} اور \عددی{n}ا عداد صحیح   اور \عددی{L^{m/n}} ایک حقیقی عدد ہو۔
\end{description}
تمام حد \عددی{(x,y)\to (x_0,y_0)} کی صورت میں حاصل کیے جائیں گے اور \عددی{L}، \عددی{M} کا حقیقی اعداد ہونا لازمی ہے۔
\انتہا{مسئلہ}
%==========

 مساوات \حوالہ{مساوات_کثیر_المتغیر_قواعد_حد} پر   مسئلہ \حوالہ{مسئلہ_کثیر_المتغیر_قواعد_حد} کے اطلاق سے ہمیں معلوم ہوتا ہے کہ  \عددی{(x,y)\to(x_0,y_0)} کرتے ہوئے  کثیر رکنی اور ناطق تفاعل کی حد     ہم \عددی{(x_0,y_0)} پر  تفاعل کی قیمت   سے حاصل کرتے  ہیں۔ بس اتنا ضروری  ہے کہ  نقطہ \عددی{(x_0,y_0)} پر تفاعل  معین ہو۔
