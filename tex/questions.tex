\جزوحصہء{سوالات}
\موٹا{مختلف اعادوں سے تہرا تکمل کی قیمت کا   حصول}\\
\ابتدا{سوال}
چھ مختلف اعادوں سے مثال \حوالہ{مثال_بالکثرت_چھ_اعادے} میں حجم کا حل دیا گیا ہے۔ ان تمام کا مشترک جواب کیا ہے؟
\انتہا{سوال}
%=========
\ابتدا{سوال}
ثُمن  اول میں محددی مستویات اور  مستویات \عددی{x=1}، \عددی{y=2} اور \عددی{z=3} کے بیچ ٹھوس مستطیل  جسم کے حجم  کے چھ مختلف اعادہ   تہرا  تکملات لکھیں۔ ان میں سے ایک تکمل کی قیمت معلوم کریں۔
\انتہا{سوال}
%================
\ابتدا{سوال}
ثُمن  اول  سے مستوی \عددی{6x+3y+2z=6} ایک چو سطحہ کاٹتا ہے۔ اس کے حجم کے چھ مختلف اعادہ تہرا تکملات لکھیں۔ ان میں سے ایک تکمل کی قیمت حاصل کریں۔
\انتہا{سوال}
%===============
\ابتدا{سوال}
ثُمن اول  سے بیلن \عددی{x^2+z^2=4} اور مستوی \عددی{y=3}  ایک خطہ کاٹتے ہیں۔ اس خطہ کے حجم کے چھ مختلف اعادہ تہرا ا تکملات  لکھیں۔ ان میں سے ایک تکمل کی قیمت تلاش کریں۔
\انتہا{سوال}
%==============
\ابتدا{سوال}
قطعات  مکافی \عددی{z=8-x^2-y^2} اور \عددی{z=x^2+y^2} میں محیط خطہ \عددی{D}  کے حجم کا چھ مختلف  تہرا اعادہ تکملات لکھیں۔ان میں سے ایک تکمل کی قیمت معلوم کریں۔
\انتہا{سوال}
%==========
\ابتدا{سوال}
قطع مکافی \عددی{z=x^2+y^2} اور مستوی \عددی{z=2y}  میں  ملفوف خطہ \عددی{D} کے حجم کی تہرا اعادہ تکملات ترتیب  \عددی{\dif z\dif x\dif y} اور  \عددی{\dif z\dif y\dif x} میں لکھیں۔ ان میں سے کسی بھی تکمل کی قیمت حاصل نہ کریں۔
\انتہا{سوال}
%===================
\موٹا{تہرا اعادہ تکمل کی قیمت کی تلاش}\\
سوال \حوالہ{سوال_بالکثرت_تہرا_قیمت_الف} تا سوال \حوالہ{سوال_بالکثرت_تہرا_قیمت_ب} میں تکملات کی قیمتیں تلاش کریں۔

\ابتدا{سوال}\شناخت{سوال_بالکثرت_تہرا_قیمت_الف}
$\int_0^1\int_0^1\int_0^1 (x^2+y^2+z^2)\dif z\dif y\dif x$
\انتہا{سوال}
%====================
\ابتدا{سوال}
$\int_{0}^{\sqrt{2}}\int_{0}^{3y}\int_{x^2+3y^2}^{8-x^2-y^2}\dif z\dif x\dif y$
\انتہا{سوال}
%===================
\ابتدا{سوال}
$\int_{1}^{e}\int_{1}^{e}\int_{1}^{e}\tfrac{1}{xyz}\dif x\dif y\dif z$
\انتہا{سوال}
%===================
\ابتدا{سوال}
$\int_{0}^{1}\int_{0}^{3-3x}\int_{0}^{\pi}\dif z\dif y\dif x$
\انتہا{سوال}
%===================
\ابتدا{سوال}
$\int_{0}^{1}\int_{0}^{\pi}\int_{0}^{\pi}y\sin z\dif x\dif y\dif z$
\انتہا{سوال}
%===================
\ابتدا{سوال}
$\int_{-1}^{1}\int_{-1}^{1}\int_{-1}^{1}(x+y+z)\dif y\dif x\dif z$
\انتہا{سوال}
%===================
\ابتدا{سوال}
$\int_{0}^{3}\int_{0}^{\sqrt{9-x^2}}\int_{0}^{\sqrt{9-x^2}}\dif z\dif y\dif x$
\انتہا{سوال}
%===================
\ابتدا{سوال}
$\int_{0}^{2}\int_{-\sqrt{}4-y^2}^{\sqrt{4-x^2}}\int_{0}^{2x+y}\dif z\dif x\dif y$
\انتہا{سوال}
%===================
\ابتدا{سوال}
$\int_{0}^{1}\int_{0}^{2-x}\int_{0}^{2-x-y}\dif z\dif y\dif x$
\انتہا{سوال}
%===================
\ابتدا{سوال}
$\int_{0}^{1}\int_{0}^{1-x^2}\int_{3}^{4-x^2-y}x\dif z\dif y\dif x$
\انتہا{سوال}
%===================
\ابتدا{سوال}
$\int_{0}^{\pi}\int_{0}^{\pi}\int_{0}^{\pi}\cos(u+v+w)\dif u\dif v\dif w$\quad
(\عددی{uvw} فضا)
\انتہا{سوال}
%===================
\ابتدا{سوال}
$\int_{1}^{e}\int_{1}^{e}\int_{1}^{e}\ln r\ln s\ln t\dif t\dif r\dif s$\quad
(\عددی{rst} فضا)
\انتہا{سوال}
%===================
\ابتدا{سوال}
$\int_{0}^{\pi/4}\int_{0}^{\ln \sec v}\int_{-\infty}^{2t}e^x\dif x\dif t\dif v$\quad
(\عددی{tvx} فضا)
\انتہا{سوال}
%===================
\ابتدا{سوال}\شناخت{سوال_بالکثرت_تہرا_قیمت_ب}
$\int_{0}^{7}\int_{0}^{2}\int_{0}^{\sqrt{4-q^2}}\tfrac{q}{r+1}\dif p\dif q\dif r$\quad
(\عددی{pqr} فضا)
\انتہا{سوال}
%===================

\موٹا{حجم بذریعہ تہرا تکمل}\\
\ابتدا{سوال}
درج  ذیل تکمل کا خطہ دکھایا گیا ہے۔
\begin{align*}
\int_{-1}^1\int_{x^2}^1\int_0^{1-y}\dif z\dif y\dif x
\end{align*}
اس تکمل کو درج ذیل  ترتیب کے اعادہ  معادل روپ میں لکھیں۔
\begin{multicols}{3}
\begin{enumerate}[a.]
\item
$\dif y\dif z\dif x$
\item
$\dif y\dif x\dif z$
\item
$\dif x\dif y\dif z$
\item
$\dif x\dif z\dif y$
\item
$\dif z\dif x\dif y$
\end{enumerate}
\end{multicols}
\انتہا{سوال}
%======
\ابتدا{سوال}
درج ذیل تکمل  کا خطہ دکھایا گیا ہے۔
\begin{align*}
\int_0^1\int_{-1}^0\int_0^{y^2}\dif z\dif y\dif x
\end{align*}
اس تکمل کو درج ذیل  ترتیب کے اعادہ  معادل روپ میں لکھیں۔
\begin{multicols}{3}
\begin{enumerate}[a.]
\item
$\dif y\dif z\dif x$
\item
$\dif y\dif x\dif z$
\item
$\dif x\dif y\dif z$
\item
$\dif x\dif z\dif y$
\item
$\dif z\dif x\dif y$
\end{enumerate}
\end{multicols}
\انتہا{سوال}
%=============

سوال \حوالہ{سوال_بالکثرت_خطہ_کا_حجم_الف} تا سوال \حوالہ{سوال_بالکثرت_خطہ_کا_حجم_ب} میں خطوں کا حجم تلاش کریں۔

\ابتدا{سوال}\شناخت{سوال_بالکثرت_خطہ_کا_حجم_الف}
بیلن \عددی{z=y^2} اور مستوی \عددی{xy}  کے بیچ خطہ  جس کی سرحدیں  مستویات\عددی{x=0}، \عددی{x=1}، \عددی{y=-1} اور \عددی{y=1} ہوں۔
\انتہا{سوال}
%================
\ابتدا{سوال}
ثُمن اول  میں محددی مستویات اور مستویات \عددی{x+z=1}، \عددی{y+2z=2}   کے بیچ خطہ۔
\انتہا{سوال}
%=================
\ابتدا{سوال}
ثُمن اول  میں محددی مستویات اور مستوی \عددی{x+z=2} اور بیلن \عددی{x=4-y^2}   کے بیچ خطہ۔
\انتہا{سوال}
%=================
\ابتدا{سوال}
بیلن \عددی{x^2+y^2=1} سے مستویات \عددی{z=-y} اور \عددی{z=0} جو پچر  کاٹتے ہیں۔
\انتہا{سوال}
%=================
\ابتدا{سوال}
ثُمن اول میں  محددی مستویات اور مستوی \عددی{x+\tfrac{y}{2}+\tfrac{z}{3}=1}  کے بیچ چو سطحہ۔
\انتہا{سوال}
%=================
\ابتدا{سوال}
ثُمن اول میں محددی مستویات، مستوی \عددی{y=1-x}  اور سطح \عددی{z=\cos(\pi x/2),\, 0\le x\le 1} کے بیچ خطہ۔
\انتہا{سوال}
%=================
\ابتدا{سوال}
بیلن \عددی{x^2+y^2=1} اور بیلن \عددی{x^2+z^2=1}  کا مشترک  اندرون۔
\انتہا{سوال}
%=================
\ابتدا{سوال}
ثُمن اول میں محددی مستویات اور سطح \عددی{z=4-x^2-y^2} کے بیچ خطہ۔
\انتہا{سوال}
%=================
\ابتدا{سوال}
ثُمن اول میں محددی مستویات، مستوی \عددی{x+y=4} اور بیلن \عددی{y^2+4z^2=16}  کے بیچ خطہ۔
\انتہا{سوال}
%=================
\ابتدا{سوال}
بیلن \عددی{x^2+y^2=4} سے مستویات \عددی{z=0} اور \عددی{x+z=3} جو خطہ کاٹتے ہیں۔
\انتہا{سوال}
%=================
\ابتدا{سوال}
ثُمن اول میں مستویات \عددی{x+y+2z=2} اور \عددی{2x+2y+z=4} کے بیچ خطہ۔
\انتہا{سوال}
%=================
\ابتدا{سوال}
مستویات \عددی{z=x}، \عددی{x+z=8}، \عددی{z=y}،  \عددی{y=8}  اور \عددی{z=0} کے بیچ متناہی خطہ۔
\انتہا{سوال}
%=================
\ابتدا{سوال}
ٹھوس ترخیمی بیلن \عددی{x^2+4y^2\le 4}  سے \عددی{xy} مستوی اور مستوی \عددی{z=x+2} جو خطہ کاٹتے ہیں۔
\انتہا{سوال}
%=================
\ابتدا{سوال}\شناخت{سوال_بالکثرت_خطہ_کا_حجم_ب}
وہ خطہ جس کا پشت مستوی \عددی{x=0}، سامنے اور اطراف  قطع مکافی بیلن \عددی{x=1-y^2}،  بالا  قطع مکافی  سطح  \عددی{z=x^2+y^2} اور نیچے  مستوی  \عددی{xy} ہوں۔
\انتہا{سوال}
%=================

\موٹا{اوسط قیمتیں}\\
سوال \حوالہ{سوال_بالکثرت_اوسط_تین_بعدی_الف} تا سوال \حوالہ{سوال_بالکثرت_اوسط_تین_بعدی_ب} میں دیے گئے خطہ پر \عددی{F(x,y,z)} کی اوسط قیمت تلاش کریں۔

\ابتدا{سوال}\شناخت{سوال_بالکثرت_اوسط_تین_بعدی_الف}
ثُمن اول میں محددی مستویات اور مستویات \عددی{x=2}، \عددی{y=2} اور \عددی{z=2} کے بیچ مکعب  خطہ اور  تفاعل \عددی{F(x,y,z)=x^2+9}لیں۔
\انتہا{سوال}
%====================
\ابتدا{سوال}
ثُمن اول میں محددی مستویات اور مستویات \عددی{x=1}، \عددی{y=1} اور \عددی{z=2} کے بیچ خطہ  اور  تفاعل \عددی{F(x,y,z)=x+y-z}لیں۔
\انتہا{سوال}
%=================
\ابتدا{سوال}
ثُمن اول میں محددی مستویات اور مستویات \عددی{x=1}، \عددی{y=1}اور \عددی{z=1} کے بیچ  خطہ  اور  تفاعل \عددی{F(x,y,z)=x^2+y^2+z^2}لیں۔
\انتہا{سوال}
%=================
\ابتدا{سوال}\شناخت{سوال_بالکثرت_اوسط_تین_بعدی_ب}
ثُمن اول میں محددی مستویات اور مستویات \عددی{x=2}، \عددی{y=2} اور \عددی{z=2} کے بیچ خطہ  اور  تفاعل \عددی{F(x,y,z)=xyz} لیں۔
\انتہا{سوال}
%=================

\موٹا{تکمل کی ترتیب بدلنا}\\
سوال \حوالہ{سوال_بالکثرت_ترتیب_بدل_کر_حل_الف} تا سوال \حوالہ{سوال_بالکثرت_ترتیب_بدل_کر_حل_ب} میں موزوں طریقہ سے تکمل کی ترتیب تبدیل کر کے تکمل کی قیمت تلاش کریں۔

\ابتدا{سوال}\شناخت{سوال_بالکثرت_ترتیب_بدل_کر_حل_الف}
$\int_0^4\int_0^1\int_{2y}^2 \frac{4\cos(x^2)}{2\sqrt{z}}\dif x\dif y\dif z$
\انتہا{سوال}
%=================
\ابتدا{سوال}
$\int_{0}^{1}\int_{0}^{1}\int_{x^2}^{1}12xze^{zy^2}\dif y\dif x\dif z$
\انتہا{سوال}
%=================
\ابتدا{سوال}
$\int_{0}^{1}\int_{\sqrt[3]{z}}^{1}\int_{0}^{\ln 3}\frac{\pi e^{2x}\sin \pi y^2}{y^2}\dif x\dif y\dif z$
\انتہا{سوال}
%=================
\ابتدا{سوال}\شناخت{سوال_بالکثرت_ترتیب_بدل_کر_حل_ب}
$\int_{0}^{2}\int_{0}^{4-x^2}\int_{0}^{x}\frac{\sin 2z}{4-z}\dif y\dif z\dif x$
\انتہا{سوال}
%=================

\موٹا{نظریہ اور مثالیں}\\
\ابتدا{سوال}
درج ذیل کو \عددی{ a} کے لئے حل کریں۔
\begin{align*}
\int_0^1\int_0^{4-a-x^2}\int_a^{4-x^2-y^2}\dif z\dif y\dif x=\frac{4}{15}
\end{align*}
\انتہا{سوال}
%==========
\ابتدا{سوال}
ترخیمی سطح \عددی{x^2+\tfrac{y^2}{4}+\tfrac{z^2}{c^2}=1} کا حجم  \عددی{} کی کس قیمت کے لئے \عددی{8\pi} ہو گا؟
\انتہا{سوال}
%===================
\ابتدا{سوال}
فضا میں  کونسا دائرہ کار \عددی{D} درج ذیل تکمل کی قیمت کو کم سے کم بناتا ہے؟ اپنے جواب کی وجہ پیش کریں۔
\begin{align*}
\iiint\limits_D (4x^2+4y^2+z^2-4)\dif H
\end{align*}
\انتہا{سوال}
%=============
\ابتدا{سوال}
فضا میں کونسا دائرہ کار \عددی{D} درج ذیل تکمل کی قیمت کو زیادہ سے زیادہ بناتا ہے؟ اپنے جواب کی وجہ پیش کریں۔
\begin{align*}
\iiint\limits_D (1-x^2-y^2-z^2)\dif H
\end{align*}
\انتہا{سوال}
%==========
\موٹا{کمپیوٹر}\\
سوال \حوالہ{سوال_بالکثرت_کمپیوٹر_تہرا_الف} تا سوال \حوالہ{سوال_بالکثرت_کمپیوٹر_تہرا_ب} میں دیے گئے خطہ پر  تفاعل کا تہرا تکمل کمپیوٹر کی مدد سے حل کریں۔

\ابتدا{سوال}\شناخت{سوال_بالکثرت_کمپیوٹر_تہرا_الف}
مستویات \عددی{z=0} اور \عددی{z=1}  اور سطح \عددی{x^2+y^2=1} کے بیچ ٹھوس بیلن پر تفاعل \عددی{F(x,y,z)=x^2y^2z}  لیں۔
\انتہا{سوال}
%=================
\ابتدا{سوال}
ٹھوس خطہ جو نیچے سے   قطع مکافی سطح \عددی{z=x^2+y^2}  اور اوپر سے مستوی \عددی{z=1}  میں ملفوف ہو اور تفاعل \عددی{F(x,y,z)=\abs{xyz}} لیں۔
\انتہا{سوال}
%==================
\ابتدا{سوال}
ٹھوس خطہ جو نیچے سے مخروط \عددی{z=\sqrt{x^2+y^2}} اور اوپر سے مستوی \عددی{z=1} میں ملفوف ہو  اور تفاعل \عددی{ F(x,y,z)=\frac{z}{(x^2+y^2+z^2)^{3/2}}}لیں۔ 
\انتہا{سوال}
%==================
\ابتدا{سوال}\شناخت{سوال_بالکثرت_کمپیوٹر_تہرا_ب}
ٹھوس کرہ \عددی{x^2+y^2+z^2\le 1}  اور تفاعل \عددی{F(x,y,z)=x^4+y^2+z^2} لیں۔
\انتہا{سوال}
%==================
