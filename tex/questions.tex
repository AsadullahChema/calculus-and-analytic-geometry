\حصہء{سوالات}
\موٹا{ارتکاز اور انفراج کی دریافت}\\
سوال \حوالہ{سوال_تسلسل_کیا_مرتکز_یا_منفرج_الف} تا سوال \حوالہ{سوال_تسلسل_کیا_مرتکز_یا_منفرج_ب} میں کون سا تسلسل مرتکز اور کون سا تسلسل منفرج ہے؟ اپنے جواب کی وجہ پیش کریں۔

\ابتدا{سوال}\شناخت{سوال_تسلسل_کیا_مرتکز_یا_منفرج_الف}
$\sum\limits_{n=1}^{\infty}\frac{1}{2\sqrt{n}+\sqrt[3]{n}}$
\انتہا{سوال}
%=========================
\ابتدا{سوال}
$\sum\limits_{n=1}^{\infty}\frac{3}{n+\sqrt{n}}$
\انتہا{سوال}
%=========================
\ابتدا{سوال}
$\sum\limits_{n=1}^{\infty}\frac{\sin^2n}{2^n}$
\انتہا{سوال}
%=========================
\ابتدا{سوال}
$\sum\limits_{n=1}^{\infty}\frac{1+\cos n}{n^2}$
\انتہا{سوال}
%=========================
\ابتدا{سوال}
$\sum\limits_{n=1}^{\infty}\frac{2n}{3n-1}$
\انتہا{سوال}
%=========================
\ابتدا{سوال}
$\sum\limits_{n=1}^{\infty}\frac{n+1}{n^2\sqrt{n}}$
\انتہا{سوال}
%=========================
\ابتدا{سوال}
$\sum\limits_{n=1}^{\infty}\big(\frac{n}{3n+1}\big)^n$
\انتہا{سوال}
%=========================
\ابتدا{سوال}
$\sum\limits_{n=1}^{\infty}\frac{1}{\sqrt{n^3}+2}$
\انتہا{سوال}
%=========================
\ابتدا{سوال}
$\sum\limits_{n=3}^{\infty}\frac{1}{\ln(\ln n)}$
\انتہا{سوال}
%=========================
\ابتدا{سوال}
$\sum\limits_{n=2}^{\infty}\frac{1}{(\ln n)^2}$
\انتہا{سوال}
%=========================
\ابتدا{سوال}
$\sum\limits_{n=1}^{\infty}\frac{(\ln n)^2}{n^3}$
\انتہا{سوال}
%=========================
\ابتدا{سوال}
$\sum\limits_{n=1}^{\infty}\frac{(\ln n)^3}{n^3}$
\انتہا{سوال}
%=========================
\ابتدا{سوال}
$\sum\limits_{n=2}^{\infty}\frac{1}{\sqrt{n}\ln n}$
\انتہا{سوال}
%=========================
\ابتدا{سوال}
$\sum\limits_{n=1}^{\infty}\frac{(\ln n)^2}{n^{3/2}}$
\انتہا{سوال}
%=========================
\ابتدا{سوال}
$\sum\limits_{n=1}^{\infty}\frac{1}{1+\ln^2n}$
\انتہا{سوال}
%=========================
\ابتدا{سوال}
$\sum\limits_{n=1}^{\infty}\frac{1}{(1+\ln n)^2}$
\انتہا{سوال}
%=========================
\ابتدا{سوال}
$\sum\limits_{n=2}^{\infty}\frac{\ln(n+1)}{n+1}$
\انتہا{سوال}
%=========================
\ابتدا{سوال}
$\sum\limits_{n=1}^{\infty}\frac{1}{1+\ln^2n}$
\انتہا{سوال}
%=========================
\ابتدا{سوال}
$\sum\limits_{n=2}^{\infty}\frac{1}{n\sqrt{n^2-1}}$
\انتہا{سوال}
%=========================
\ابتدا{سوال}
$\sum\limits_{n=1}^{\infty}\frac{\sqrt{n}}{n^2+1}$
\انتہا{سوال}
%=========================
\ابتدا{سوال}
$\sum\limits_{n=1}^{\infty}\frac{1-n}{n2^n}$
\انتہا{سوال}
%=========================
\ابتدا{سوال}
$\sum\limits_{n=1}^{\infty}\frac{n+2^n}{n^22^n}$
\انتہا{سوال}
%=========================
\ابتدا{سوال}
$\sum\limits_{n=1}^{\infty}\frac{1}{3^{n-1}+1}$
\انتہا{سوال}
%=========================
\ابتدا{سوال}
$\sum\limits_{n=1}^{\infty}\frac{3^{n-1}+1}{3^n}$
\انتہا{سوال}
%=========================
\ابتدا{سوال}
$\sum\limits_{n=1}^{\infty}\sin\frac{1}{n}$
\انتہا{سوال}
%=========================
\ابتدا{سوال}
$\sum\limits_{n=1}^{\infty}\tan\frac{1}{n}$
\انتہا{سوال}
%=========================
\ابتدا{سوال}
$\sum\limits_{n=1}^{\infty}\frac{10n+1}{n(n+1)(n+2)}$
\انتہا{سوال}
%=========================
\ابتدا{سوال}
$\sum\limits_{n=3}^{\infty}\frac{5n^3-3n}{n^2(n-2)(n^2+5)}$
\انتہا{سوال}
%=========================
\ابتدا{سوال}
$\sum\limits_{n=1}^{\infty}\frac{\tan^{-1}n}{n^{1.1}}$
\انتہا{سوال}
%=========================
\ابتدا{سوال}
$\sum\limits_{n=1}^{\infty}\frac{\sec^{-1}n}{n^{1.3}}$
\انتہا{سوال}
%=========================
\ابتدا{سوال}
$\sum\limits_{n=1}^{\infty}\frac{\coth n}{n^2}$
\انتہا{سوال}
%=========================
\ابتدا{سوال}
$\sum\limits_{n=1}^{\infty}\frac{\tanh n}{n^2}$
\انتہا{سوال}
%=========================
\ابتدا{سوال}
$\sum\limits_{n=1}^{\infty}\frac{1}{n\sqrt[n]{n}}$
\انتہا{سوال}
%=========================
\ابتدا{سوال}
$\sum\limits_{n=1}^{\infty}\frac{\sqrt[n]{n}}{n^2}$
\انتہا{سوال}
%=========================
\ابتدا{سوال}
$\sum\limits_{n=1}^{\infty}\frac{1}{1+2+3+\cdots+n}$
\انتہا{سوال}
%=========================
\ابتدا{سوال}\شناخت{سوال_تسلسل_کیا_مرتکز_یا_منفرج_ب}
$\sum\limits_{n=1}^{\infty}\frac{1}{1+2^2+3^2+\cdots+n^2}$
\انتہا{سوال}
%=========================
\موٹا{نظریہ اور مثالیں}\\
\ابتدا{سوال}
تقابل حد پرکھ کا جزو-ب اور جزو-ج ثابت کریں۔ 
\انتہا{سوال}
%===================
\ابتدا{سوال}
اگر غیر منفی اجزاء کا تسلسل \عددی{\sum_{n=1}^{\infty}a_n} مرتکز ہو تب کیا \عددی{\sum_{n=1}^{\infty}\tfrac{a_n}{n}} کے بارے میں کچھ کہنا ممکن ہو گا؟ وجہ پیش کریں۔
\انتہا{سوال}
%===============
\ابتدا{سوال}
فرض کریں  \عددی{n\ge N} کے لئے \عددی{a_n>0} اور \عددی{b_n>0} ہیں جہاں \عددی{N} عدد صحیح ہے۔ اگر \عددی{\lim\limits_{n\to\infty}\tfrac{a_n}{b_n}=\infty} ہو اور \عددی{\sum a_n} مرتکز ہو تب کیا \عددی{\sum b_n} کے بارے میں کچھ کہنا ممکن ہو گا؟ وجہ پیش کریں۔
\انتہا{سوال}
%==================
\ابتدا{سوال}
ثابت کریں کہ اگر غیر مثبت اجزاء کا تسلسل \عددی{\sum a_n} مرتکز ہو تب \عددی{\sum a_n^2} بھی مرتکز ہو گا۔
\انتہا{سوال}
%=====================
\موٹا{کمپیوٹر کا استعمال}\\
\ابتدا{سوال}
ہم نہیں جانتے ہیں کہ آیا تسلسل \عددی{\sum_{n=1}^{\infty}\tfrac{1}{n^3\sin^2n}} مرتکز کہ منفرج ہے۔ کمپیوٹر کی مدد سے اس تسلسل کا رویہ درج ذیل اقدام سے دیکھیں۔
\begin{enumerate}[a.]
\item
جزوی مجموعات \عددی{s_k=\sum_{n=1}^{k}\tfrac{1}{n^2\sin^2n}} کی ترتیب لیں۔ اس ترتیب کا حد کا رویہ \عددی{k\to \infty} کیسا ہے۔کیا آپ کا کمپیوٹر پروگرام اس ترتیب کے حد کا کلیہ تلاش کر سکتا ہے؟
\item
جزوی مجموعات کے ابتدائی \عددی{100} نقطے \عددی{(k,s-k)} ترسیم کریں۔  کیا یہ مرتکز نظر آتے ہیں؟ آپ اس کے حد کی اندازاً کتنی قیمت لگائیں گے؟
\item
اب ابتدائی \عددی{200} نقطے  \عددی{(k,s_k)} ترسیم کریں۔اس کے رویہ پر تبصرہ کریں۔
\item
ابتدائی \عددی{400} نقطے  \عددی{(k,s_k)} ترسیم کریں۔ \عددی{k=355} پر کیا ہوتا ہے؟ عدد \عددی{\tfrac{355}{113}} کا حساب لگائیں۔ اس حساب کی رو سے \عددی{k=355} پر جزوی مجموعہ کے رویہ پر تبصرہ کریں۔ آپ \عددی{k} کی کن قیمتوں پر اسی رویہ کی توقع کرتے ہیں۔  
\end{enumerate}
\انتہا{سوال}
%===================

\حصہ{غیر منفی اجزاء کے تسلسل کا تناسبی اور جذری پرکھ}
وہ پرکھ ارتکاز  جو دوسرے تسلسل یا تکمل کے ساتھ موازنہ پر منحصر ہو \اصطلاح{بیرونی پرکھ}\فرہنگ{پرکھ!اندرونی}\حاشیہب{extrinsic test}\فرہنگ{test!extrinsic} کہلاتا ہے۔ ایسے پرکھ کار آمد ہوتے ہیں لیکن چند وجوہات کی بنا ہمیں ایسے پرکھ درکار ہیں جو کسی موازنہ پر منحصر نہ ہوں۔ حقیقت میں عین ممکن ہے کہ ہمیں ایسا کوئی تسلسل یا تکمل معلوم نہ ہو جس کے ساتھ موازنہ کرنا ممکن ہو۔ اس کے علاوہ کسی بھی تسلسل کی تمام معلومات اسی کے اجزاء میں پائی جانی چاہیے۔ اسی لئے ہم اپنی توجہ \اصطلاح{اندرونی پرکھ}\فرہنگ{پرکھ!اندرونی}\حاشیہب{intrinsic test}\فرہنگ{test!intrinsic} کی طرف  کرتے ہیں۔اندرونی پرکھ صرف دیے گئے تسلسل پر منحصر ہوتا ہے۔

\جزوحصہء{تناسبی پرکھ}
تناسبی پرکھ ہمارا پہلا اندرونی پرکھ ہے جو تسلسل کے بڑھنے (یا گھٹنے) کی شرح کو نسبت \عددی{\tfrac{a_{n+1}}{a_n}} سے حاصل کرتا ہے۔ ہندسی تسلسل \عددی{\sum a r^n} کے لئے یہ شرح مستقل (\عددی{\tfrac{ar^{n+1}}{ar^n}=r}) ہے اور تسلسل صرف اور صرف اس صورت مرتکز ہو گا جب اس کے نسبت کی مطلق قیمت \عددی{1} سے کم ہو۔ اگر نسبت مستقل نہ ہو تب بھی (اگلی مثال کی طرح) ایسا ہندسی تسلسل معلوم کیا جا سکتا ہے جس کے ساتھ موازنہ کیا جا سکے۔

\ابتدا{مثال}\شناخت{مثال_تسلسل_کیا_تسلسل_مرتکز_ہے}
\عددی{a_1=1} اور تمام \عددی{n} کے لئے \عددی{a_{n+1}=\tfrac{n}{2n+1}a_n} لیں۔ کیا تسلسل \عددی{\sum a_n} مرتکز ہے؟

حل:\quad
ہم تسلسل کے چند ابتدائی اجزاء لکھتے ہیں:
\begin{align*}
a_1=1,\quad a_2=\frac{1}{3}a_1=\frac{1}{3},\quad a_3=\frac{2}{5}a_2=\frac{1\cdot 2}{3\cdot 5},\quad a_4=\frac{3}{7}a_3=\frac{1\cdot 2\cdot 3}{3\cdot 5\cdot 7}
\end{align*}
چونکہ \عددی{\tfrac{n}{2n+1}} کی قیمت \عددی{\tfrac{1}{2}} سے کم ہے لہٰذا ہر جزو گزشتہ جزو کے \عددی{\tfrac{1}{2}} سے بھی کم ہو گا۔ یوں اس  تسلسل کے اجزاء درج ذیل ہندسی تسلسل کے اجزاء سے کم یا برابر ہوں گے
\begin{align*}
1+\big(\frac{1}{2}\big)+\big(\frac{1}{2}\big)^2+\cdots+\big(\frac{1}{2}\big)^{n-1}+\cdots
\end{align*}
 اور یہ ہندسی تسلسل \عددی{2} پر مرتکز ہے۔ یوں ہمارا تسلسل بھی مرتکز ہو گا اور اس کا مجموعہ \عددی{2} سے کم ہو گا۔درج ذیل جدول میں آپ دیکھ سکتے ہیں کہ یہ تسلسل اپنے حد \عددی{\tfrac{\pi}{2}} تک کتنا جلدی پہنچتا ہے۔
\begin{align*}
\begin{array}{rc}
\toprule
n&s_n\\
\midrule
5&\num{1.549206349}\\
10&\num{1.570289085}\\
15&\num{1.570783080}\\
20&\num{1.570795964}\\
25&\num{1.570796317}\\
30&\num{1.570796327}\\
35&\num{1.570796327}\\
\bottomrule
\end{array}
\end{align*}
\انتہا{مثال}
%===============

\ابتدا{پرکھ}\موٹا{تناسبی پرکھ}\\
فرض کریں \عددی{\sum a_n} مثبت اجزاء کا تسلسل ہے اور درج ذیل فرض کریں۔
\begin{align*}
\lim_{n\to\infty}\frac{a_n+1}{a_n}=\rho
\end{align*}
تب درج ذیل ہو گا۔
\begin{enumerate}[a.]
\item
\عددی{\rho<1} کی صورت میں تسلسل مرتکز ہو گا۔
\item
\عددی{\rho>1} یا لامتناہی کے برابر ہونے  کی صورت میں تسلسل منفرج ہو گا۔
\item
\عددی{\rho=1} کی صورت میں یہ پرکھ غیر فیصلہ کن ہو گا۔
\end{enumerate}
\انتہا{پرکھ}
%===================
\ابتدا{ثبوت پرکھ}
تناسبی پرکھ کی ثبوت میں (مثال \حوالہ{مثال_تسلسل_کیا_تسلسل_مرتکز_ہے} کی طرح) موزوں ہندسی تسلسل کے ساتھ موازنہ کیا جائے گا۔ البتہ تناسبی پرکھ استعمال کرتے ہوئے ایسے کسی موازنہ کی ضرورت نہیں ہو گی۔
\begin{enumerate}[a.]
\item
$[\rho<1]$\quad
فرض کریں \عددی{\rho} اور \عددی{1} کے بیچ \عددی{r} ایک عدد ہے۔ یوں \عددی{\epsilon=r-\rho} مثبت ہو گا۔چونکہ
\begin{align*}
\frac{a_{n+1}}{a_n}\to\rho
\end{align*}
ہے لہٰذا بڑے \عددی{n}، مثلاً \عددی{n\ge N}،  کی صورت میں \عددی{\rho} اور  \عددی{\tfrac{a_{n+1}}{a_n}} کے بیچ فرق \عددی{\epsilon} یا اس سے کم ہو گا۔ بالخصوص درج ذیل ہو گا۔
\begin{align*}
\frac{a_{n+1}}{a_n}&<\rho+\epsilon=r&&\text{\RL{جب \عددی{n\ge N}}}
\end{align*}
اس طرح درج ذیل ہو گا۔
\begin{align*}
a_{N+1}&<ra_N,\\
a_{N+2}&<ra_{N+1}<r^2a_N,\\
a_{N+3}&<ra_{N+2}<r^3a_N,\\
\vdots&\\
a_{N+m}&<ra_{N+m-1}<r^ma_n
\end{align*}
ان عدم مساوات سے ظاہر ہے کہ  \عددی{N} جزو کے بعد ہمارے تسلسل کے اجزاء  صفر تک اس ہندسی تسلسل سے زیادہ تیزی سے پہنچتے ہیں جس میں \عددی{r<1} ہو۔ بلکہ  تسلسل \عددی{\sum c_n} پر غور کریں جہاں \عددی{n=1,2,\cdots,N} کے لئے \عددی{c_n=a_n}  اور 
\begin{align*}
c_{N+1}=ra_N,\, c_{N+2}=r^2a_N,\cdots, c_{N+m}=r^ma_N,\cdots
\end{align*}
ہوں۔اب تمام \عددی{n} کے لئے \عددی{a_n\le c_n} اور 
\begin{align*}
\sum_{n=1}^{\infty} c_n&=a_1+a_2+\cdots+a_{N-1}+a_N+ra_N+r^2a_N+\cdots\\
&=a_1+a_2+\cdots+a_{N-1}+a_N(1+r+r^2+\cdots)
\end{align*}
ہے۔ چونکہ \عددی{\abs{r}<1} ہے لہٰذا  ہندسی تسلسل \عددی{1+r+r^2+\cdots} مرتکز ہو گا لہٰذا \عددی{\sum c_n} بھی مرتکز ہو گا۔ چونکہ \عددی{a_n\le c_n} ہے لہٰذا \عددی{\sum a_n} بھی مرتکز ہو گا۔
\item
$[1<\rho\le \infty]$\quad
کسی اشاریہ \عددی{M} سے آگے
\begin{align*}
a_M<a_{M+1}<a_{M+2}<\cdots \quad \text{\RL{اور}}\quad \frac{a_{n+1}}{a_n}>1
\end{align*}
ہو گا۔ تسلسل کے اجزاء \عددی{n} لامتناہی کرنے  سے صفر تک نہیں پہنچتے ہیں لہٰذا \عددی{n} ویں جزو پرکھ کے تحت یہ تسلسل منفرج ہو گا۔
\item
$[\rho=1]$\quad
درج ذیل دو تسلسل
\begin{align*}
\sum_{n=1}^{\infty}\frac{1}{n}\quad \text{}\quad \sum_{n=1}^{\infty}\frac{1}{n^2}
\end{align*} 
دکھاتے ہیں کہ \عددی{\rho=} کی صورت میں کسی دوسرے پرکھ کی ضرورت پیش آئے گی۔
\begin{align*}
\frac{a_{n+1}}{a_n}&=\frac{1/(n+1)}{1/n}=\frac{n}{n+1}\to 1&&\text{\RL{$\sum\limits_{n=1}^{\infty}\frac{1}{n}$ کے لئے}}\\
\frac{a_{n+1}}{a_n}&=\frac{1/(n+1)^2}{1/n^2}=\big(\frac{n}{n+1}\big)^2\to 1^2=1&&\text{\RL{$\sum\limits_{n=1}^{\infty}\frac{1}{n^2}$ کے لئے}}\\
\end{align*}
با وجود اس کے کہ دونوں صورتوں میں \عددی{\rho=1} ہے، پہلا تسلسل منفرج اور دوسرا تسلسل مرتکز ہے۔
\end{enumerate}
\انتہا{ثبوت پرکھ}
%========================

تناسبی پرکھ عموماً اس صورت موثر ہوتا ہے جب اجزاء میں \عددی{n} پر مبنی فقروں کے عدد ضربیہ یا \عددی{n} طاقت کے فقرے پائے جاتے ہوں۔

\ابتدا{مثال}
درج ذیل تسلسل کی ارتکاز پر غور کریں۔
\begin{multicols}{3}
\begin{enumerate}[a.]
\item
$\sum\limits_{n=0}^{\infty}\frac{2^n+5}{3^n}$
\item
$\sum\limits_{n=1}^{\infty}\frac{(2n)!}{n!n!}$
\item
$\sum_{n=1}^{\infty}\frac{4^nn!n!}{(2n)!}$
\end{enumerate}
\end{multicols}
حل:\quad
\begin{enumerate}[a.]
\item
تسلسل \عددی{\sum_{n=0}^{\infty}\tfrac{2^n+5}{3^n}} کے لئے درج ذیل ہو گا۔
\begin{align*}
\frac{a_{n+1}}{a_n}=\frac{(2^{n+1}+5)/3^{n+1}}{(2^n+5)/3^n}=\frac{1}{3}\cdot\frac{2^{n+1}+5}{2^n+5}=\frac{1}{3}\cdot\big(\frac{2+5\cdot 2^{-n}}{1+5\cdot2^{-n}}\big)\to\frac{1}{3}\cdot\frac{2}{1}=\frac{2}{3}
\end{align*}
چونکہ \عددی{p=\tfrac{2}{3}} ہے جو \عددی{1} سے کم ہے لہٰذا یہ تسلسل مرتکز ہو گا۔ اس کا یہ مطلب نہیں کہ تسلسل کا مجموعہ \عددی{\tfrac{2}{3}} ہے۔ در حقیقت اس کا مجموعہ درج ذیل ہے۔
\begin{align*}
\sum_{n=0}^{\infty}\frac{2^n+5}{3^n}=\sum_{n=0}^{\infty}\big(\frac{2}{3}\big)^n+\sum_{n=0}^{\infty}\frac{5}{3^n}=\frac{1}{1-(2/3)}+\frac{5}{1-(1/3)}=\frac{21}{2}
\end{align*}
\item
اگر \عددی{a_n=\tfrac{(2n)!}{n!n!}} ہو تب \عددی{a_{n+1}=\tfrac{(2n+2)!}{(n+1)!(n+1)!}} اور
\begin{align*}
\frac{a_{n+1}}{a_n}&=\frac{n!n!(2n+2)(2n+1)(2n)!}{(n+1)!(n+1)!(2n)!}\\
&=\frac{(2n+2)(2n+1)}{(n+1)(n+1)}=\frac{4n+2}{n+1}\to 4
\end{align*}
ہوں گے۔ چونکہ \عددی{p=4}  ہے  جو \عددی{1} سے بڑا ہے لہٰذا یہ تسلسل منفرج ہو گا۔
\item
اگر \عددی{a_n=\tfrac{4^nn!n!}{(2n)!}} ہو تب
\begin{align*}
\frac{a_{n+1}}{a_n}&=\frac{4^{n+1}(n+1)!(n+1)!}{(2n+2)(2n+1)(2n)!}\cdot\frac{(2n)!}{4^nn!n!}\\
&=\frac{4(n+1)(n+1)}{(2n+2)(2n+1)}=\frac{2(n+1)}{2n+1}\to 1
\end{align*}
ہو گا۔چونکہ حد \عددی{p=1} ہے تناسبی پرکھ ہمیں تسلسل کی ارتکاز یا انفراج کے بارے میں معلومات فراہم نہیں کر سکتا ہے۔ البتہ چونکہ  
 \عددی{\tfrac{a_{n+1}}{a_n}=\tfrac{2n+2}{2n+1}} ہر صورت \عددی{1} سے بڑا ہو گا لہٰذا \عددی{a_{n+1}} ہر صورت \عددی{a_n} سے بڑا ہو گا۔یوں تمام اجزاء \عددی{a_1=2} سے بڑے یا اس کے برابر ہوں گے اور \عددی{n\to\infty} کرنے سے \عددی{n} جزو صفر تک نہیں پہنچتا ہے۔ یوں یہ تسلسل منفرج ہو گا۔
\end{enumerate}
\انتہا{مثال}
%=====================

\جزوحصہء{\عددی{n} واں جذر پرکھ}
اب تک \عددی{\sum a_n} کے لئے جن پرکھ پر غور کیا گیا ان کی بہترین کارکردگی سادہ کلیات کے \عددی{a_n} میں نظر آتی ہے۔ اب درج ذیل پر غور کریں۔

\ابتدا{مثال}\شناخت{مثال_تسلسل_پرکھ_غیر_فیصلہ_کن}
اگر
 $a_n=\begin{cases}
n/2^n&\text{\RL{طاق }n}\\
1/2^n&\text{\RL{جفت }n}
\end{cases}$ 
ہو تب کیا \عددی{\sum a_n} مرتکز ہو گا؟

حل:\quad
ہم اس تسلسل کے ابتدائی چند اجزاء لکھتے ہیں:
\begin{align*}
\sum_{n=1}^{\infty}a_n&=\frac{1}{2^1}+\frac{1}{2^2}+\frac{3}{2^3}+\frac{1}{2^4}+\frac{5}{2^5}+\frac{1}{2^6}+\frac{7}{2^7}+\cdots\\
&=\frac{1}{2}+\frac{1}{4}+\frac{3}{8}+\frac{1}{16}+\frac{5}{32}+\frac{1}{64}+\frac{7}{128}+\cdots
\end{align*}
آپ دیکھ سکتے ہیں کہ یہ ہندسی تسلسل نہیں ہے۔ \عددی{n\to \infty} کرنے سے \عددی{n} واں جزو \عددی{0} تک پہنچتا ہے لہٰذا ہم نہیں جانتے کہ یہ تسلسل منفرج ہو گا۔ یہاں تکملی پرکھ ہماری مدد نہیں کر پاتا۔ تناسبی پرکھ درج ذیل دیتا ہے۔
\begin{align*}
\frac{a_{n+1}}{a_n}=\begin{cases}
\frac{1}{2n}&\text{\RL{$n$ طاق}}\\
\frac{n+1}{2}&\text{\RL{$n$ جفت}}
\end{cases}
\end{align*}
\عددی{n\to\infty} کرنے سے نسبت کم اور زیادہ ہوتی ہے اور کوئی حد نہیں پایا جاتا ہے۔

یہاں ہمیں \عددی{n} واں جذر پرکھ کی ضرورت ہے۔
\انتہا{مثال}
%===================

\ابتدا{پرکھ}\موٹا{\عددی{n} واں جذر پرکھ}\\
فرض کریں تسلسل \عددی{\sum a_n} میں تمام \عددی{n\ge N} کے لئے \عددی{a_n\ge 0} ہیں۔مزید درج ذیل فرض کریں۔
\begin{align*}
\lim_{n\to\infty}\sqrt[n]{a_n}=\rho
\end{align*}
تب
\begin{enumerate}[a.]
\item
\عددی{\rho<1} کی صورت میں یہ تسلسل مرتکز ہو گا،
\item
\عددی{\rho>1} اور لامتناہی \عددی{\rho}  کی صورت میں یہ تسلسل منفرج ہو گا،
\item
\عددی{\rho=1} کی صورت میں پرکھ غیر فیصلہ کن ہو گا۔
\end{enumerate}
\انتہا{پرکھ}
%========================
\ابتدا{ثبوت پرکھ}
\begin{enumerate}[a.]
\item
\عددی{[\rho<1]}
\quad
ہم \عددی{\epsilon} اتنا چھوٹا لیتے ہیں کہ \عددی{\rho+\epsilon<1} ہو۔ چونکہ \عددی{\sqrt[n]{a_n}\to\rho} ہے لہٰذا آخرکار \عددی{\rho} اور اجزاء \عددی{\sqrt[n]{a_n}} کے بیچ  فاصلہ \عددی{\epsilon} سے کم ہو گا۔دوسرے لفظوں میں ایک ایسا اشاریہ \عددی{M\ge N} پایا جاتا ہے جس کے لئے درج ذیل ہو گا۔
\begin{align*}
\sqrt[n]{a_n}&<\rho+\epsilon&& (n\ge M)
\end{align*}
تب درج ذیل بھی درست ہو گا۔
\begin{align*}
a_n&<(\rho+\epsilon)^n&&(n\ge M)
\end{align*}
اب ہندسی تسلسل \عددی{\sum_{n=M}^{\infty}(\rho+\epsilon)^n} جس کی نسبت \عددی{(\rho+\epsilon)<1} ہو مرتکز ہوتا ہے۔ یوں موازنہ کرتے ہوئے ہم دیکھتے ہیں کہ \عددی{\sum_{n=M}^{\infty}a_n} بھی مرتکز ہو گا۔یوں درج ذیل مرتکز ہو گا۔
\begin{align*}
\sum_{n=1}^{\infty}a_n=a_1+a_2+\cdots+a_{M-1}+\sum_{n=M}^{\infty}a_n
\end{align*}
\item
\عددی{[1<\rho\le\infty]}
\quad
کسی عدد صحیح \عددی{M} سے آگے تمام اشاریہ کے لئے  \عددی{\sqrt[n]{a_n}>1} ہو گا لہٰذا تمام \عددی{n>M} کے لئے \عددی{a_n>1} ہو گا۔ اس تسلسل کے اجزاء صفر پر مرکوز نہیں ہیں۔ یوں \عددی{n} ویں جزو پرکھ کے تحت یہ تسلسل منفرج ہو گا۔
\item
\عددی{[\rho=1]}
\quad
تسلسل \عددی{\sum_{n=1}^{\infty}\tfrac{1}{n}} اور \عددی{\sum_{n=1}^{\infty}\tfrac{1}{n^2}} سے ظاہر ہے کہ \عددی{\rho=1} کے لئے یہ پرکھ غیر فیصلہ کن ہے۔ اگرچہ ان دونوں تسلسل میں \عددی{\sqrt[n]{a_n}\to 1} ہے،   پہلا تسلسل منفرج جبکہ دوسرا تسلسل مرتکز ہے۔
\end{enumerate}
\انتہا{ثبوت پرکھ}
%=================

\ابتدا{مثال} \موٹا{(مثال \حوالہ{مثال_تسلسل_پرکھ_غیر_فیصلہ_کن} جاری)}\\

اگر
 $a_n=\begin{cases}
n/2^n&\text{\RL{طاق }n}\\
1/2^n&\text{\RL{جفت }n}
\end{cases}$ 
ہو تب کیا \عددی{\sum a_n} مرتکز ہو گا؟

حل:\quad
ہم \عددی{n} واں جذر پرکھ زیر استعمال لاتے ہیں جو
\begin{align*}
\sqrt[n]{a_n}=\begin{cases}
\frac{\sqrt[n]{n}}{2}&\text{\RL{$n$ طاق}}\\
\frac{1}{2}&\text{\RL{$n$ جفت}}
\end{cases}
\end{align*}
دیتا ہے لہٰذا
\begin{align*}
\frac{1}{2}\le \sqrt[n]{a_n}\le \frac{\sqrt[n]{n}}{2}
\end{align*}
ہو گا۔چونکہ \عددی{\sqrt[n]{n}\to 1} ہے(جدول \حوالہ{جدول_ترتیب_عمومی_حد}) لہٰذا مسئلہ بیچ کے تحت \عددی{\lim_{n\to\infty}\sqrt[n]{a_n}=\tfrac{1}{2}} ہو گا۔ یہ حد \عددی{1} سے کم ہے لہٰذا \عددی{n} ویں جذر پرکھ کے تحت دیا گیا تسلسل مرتکز ہو گا۔
\انتہا{مثال}
%========================
\ابتدا{مثال}
درج ذیل میں کونسا تسلسل مرتکز  اور کونسا منفرج ہے؟
\begin{multicols}{2}
\begin{enumerate}[a.]
\item
$\sum\limits_{n=1}^{\infty}\frac{n^2}{2^n}$
\item
$\sum\limits_{n=1}^{\infty}\frac{2^n}{n^2}$
\end{enumerate} 
\end{multicols}
حل:\quad
\begin{enumerate}[a.]
\item
چونکہ 
\begin{align*}
\sqrt[n]{\frac{n^2}{2^n}}=\frac{\sqrt[n]{n^2}}{\sqrt[n]{2^n}}=\frac{(\sqrt[n]{n})^2}{2}\to\frac{1}{2}<1
\end{align*}
ہے لہٰذا \عددی{\sum_{n=1}^{\infty}\tfrac{n^2}{2^n}} مرتکز ہو گا۔
\item
چونکہ
\begin{align*}
\sqrt[n]{\frac{2^n}{n^2}}=\frac{2}{(\sqrt[n]{n})^2}\to\frac{2}{1}>1
\end{align*}
ہے لہٰذا \عددی{\sum_{n=1}^{\infty}\frac{2^n}{n^2}} منفرج ہو گا۔
\end{enumerate}
\انتہا{مثال}
%================

\حصہء{سوالات}
\موٹا{ارتکاز اور انفراج معلوم کرنا}\\
سوال \حوالہ{سوال_تسلسل_معلوم_کریں_ارتکاز_یا_انفراج_الف} تا سوال \حوالہ{سوال_تسلسل_معلوم_کریں_ارتکاز_یا_انفراج_ب} میں کون سا تسلسل مرتکز اور کون سا منفرج ہے؟ اپنے جواب کی وجہ پیش کریں۔ (جواب حاصل کرنے کے ایک سے زیادہ طریقے ہو سکتے ہیں۔)

\ابتدا{سوال}\شناخت{سوال_تسلسل_معلوم_کریں_ارتکاز_یا_انفراج_الف}
$\sum\limits_{n=1}^{\infty}\frac{n^{\sqrt{2}}}{2^n}$
\انتہا{سوال}
%========================
\ابتدا{سوال}
$\sum\limits_{n=1}^{\infty}n^2e^{-n}$
\انتہا{سوال}
%=======================
\ابتدا{سوال}
$\sum\limits_{n=1}^{\infty}n!e^{-n}$
\انتہا{سوال}
%=======================
\ابتدا{سوال}
$\sum\limits_{n=1}^{\infty}\frac{n!}{10^n}$
\انتہا{سوال}
%=======================
\ابتدا{سوال}
$\sum\limits_{n=1}^{\infty}\frac{n^{10}}{10^n}$
\انتہا{سوال}
%=======================
\ابتدا{سوال}
$\sum\limits_{n=1}^{\infty}\big(\frac{n-2}{n}\big)^n$
\انتہا{سوال}
%=======================
\ابتدا{سوال}
$\sum\limits_{n=1}^{\infty}\frac{2+(-1)^n}{1.25^n}$
\انتہا{سوال}
%=======================
\ابتدا{سوال}
$\sum\limits_{n=1}^{\infty}\frac{(-2)^n}{3^n}$
\انتہا{سوال}
%=======================
\ابتدا{سوال}
$\sum\limits_{n=1}^{\infty}\big(1-\frac{3}{n}\big)^n$
\انتہا{سوال}
%=======================
\ابتدا{سوال}
$\sum\limits_{n=1}^{\infty}\big(1-\frac{1}{3n}\big)^n$
\انتہا{سوال}
%=======================
\ابتدا{سوال}
$\sum\limits_{n=1}^{\infty}\frac{\ln n}{n^3}$
\انتہا{سوال}
%=======================
\ابتدا{سوال}
$\sum\limits_{n=1}^{\infty}\frac{(\ln n)^n}{n^n}$
\انتہا{سوال}
%=======================
\ابتدا{سوال}
$\sum\limits_{n=1}^{\infty}\big(\frac{1}{n}-\frac{1}{n^2}\big)$
\انتہا{سوال}
%=======================
\ابتدا{سوال}
$\sum\limits_{n=1}^{\infty}\big(\frac{1}{n}-\frac{1}{n^2}\big)^n$
\انتہا{سوال}
%=======================
\ابتدا{سوال}
$\sum\limits_{n=1}^{\infty}\frac{\ln n}{n}$
\انتہا{سوال}
%=======================
\ابتدا{سوال}
$\sum\limits_{n=1}^{\infty}\frac{n\ln n}{2^n}$
\انتہا{سوال}
%=======================
\ابتدا{سوال}
$\sum\limits_{n=1}^{\infty}\frac{(n+1)(n+2)}{n!}$
\انتہا{سوال}
%=======================
\ابتدا{سوال}
$\sum\limits_{n=1}^{\infty}e^{-n}(n^3)$
\انتہا{سوال}
%=======================
\ابتدا{سوال}
$\sum\limits_{n=1}^{\infty}\frac{(n+3)!}{3!n!3^n}$
\انتہا{سوال}
%=======================
\ابتدا{سوال}
$\sum\limits_{n=1}^{\infty}\frac{n2^n(n+1)!}{3^nn!}$
\انتہا{سوال}
%=======================
\ابتدا{سوال}
$\sum\limits_{n=1}^{\infty}\frac{n!}{(2n+1)!}$
\انتہا{سوال}
%=======================
\ابتدا{سوال}
$\sum\limits_{n=1}^{\infty}\frac{n!}{n^n}$
\انتہا{سوال}
%=======================
\ابتدا{سوال}
$\sum\limits_{n=2}^{\infty}\frac{n}{(\ln n)^n}$
\انتہا{سوال}
%=======================
\ابتدا{سوال}
$\sum\limits_{n=2}^{\infty}\frac{n}{(\ln n)^{(n/2)}}$
\انتہا{سوال}
%=======================
\ابتدا{سوال}
$\sum\limits_{n=1}^{\infty}\frac{n!\ln n}{n(n+2)!}$
\انتہا{سوال}
%=======================
\ابتدا{سوال}\شناخت{سوال_تسلسل_معلوم_کریں_ارتکاز_یا_انفراج_ب}
$\sum\limits_{n=1}^{\infty}\frac{3^n}{n^32^n}$
\انتہا{سوال}
%=======================
سوال \حوالہ{سوال_تسلسل_مرتکز_منفرج_تسلسل_تلاش_الف} تا سوال \حوالہ{سوال_تسلسل_مرتکز_منفرج_تسلسل_تلاش_ب} میں کون سے تسلسل مرتکز اور کون سے منفرج ہیں؟ اپنے جواب کی وجہ پیش کریں۔

\ابتدا{سوال}\شناخت{سوال_تسلسل_مرتکز_منفرج_تسلسل_تلاش_الف}
$a_1=2,\quad a_{n+1}=\tfrac{1+\sin n}{n}a_n$
\انتہا{سوال}
%======================
\ابتدا{سوال}
$a_1=1,\quad a_{n+1}=\frac{1+\tan^{-1}n}{n}a_n$
\انتہا{سوال}
%======================
\ابتدا{سوال}
$a_1=\tfrac{1}{3},\quad a_{n+1}=\tfrac{3n-1}{2n+5}a_n$
\انتہا{سوال}
%========================
\ابتدا{سوال}
$a_1=3,\quad a_{n+1}=\tfrac{n}{n+1}a_n$
\انتہا{سوال}
%========================
\ابتدا{سوال}
$a_1=2,\quad a_{n+1}=\tfrac{2}{n}a_n$
\انتہا{سوال}
%========================
\ابتدا{سوال}
$a_1=5,\quad a_{n+1}=\tfrac{\sqrt[n]{n}}{2}a_n$
\انتہا{سوال}
%========================
\ابتدا{سوال}
$a_1=1,\quad a_{n+1}=\tfrac{1+\ln n}{n}a_n$
\انتہا{سوال}
%========================
\ابتدا{سوال}
$a_1=\tfrac{1}{2},\quad a_{n+1}=\tfrac{n+\ln n}{n+10}a_n$
\انتہا{سوال}
%========================
\ابتدا{سوال}
$a_1=\tfrac{1}{3},\quad a_{n+1}=\sqrt[n]{a_n}$
\انتہا{سوال}
%========================
\ابتدا{سوال}
$a_1=\tfrac{1}{2},\quad a_{n+1}=(a_n)^{n+1}$
\انتہا{سوال}
%========================
\ابتدا{سوال}
$a_n=\tfrac{2^nn!n!}{(2n)!}$
\انتہا{سوال}
%========================
\ابتدا{سوال}\شناخت{سوال_تسلسل_مرتکز_منفرج_تسلسل_تلاش_ب}
$a_n=\tfrac{(3n)!}{n!(n+1)!(n+2)!}$
\انتہا{سوال}
%========================
سوال \حوالہ{سوال_تسلسل_نشاندہی_الف} تا سوال \حوالہ{سوال_تسلسل_نشاندہی_ب} میں مرتکز اور منفرج تسلسل کی نشاندہی کریں۔ وجہ بھی پیش کریں۔ 

\ابتدا{سوال}\شناخت{سوال_تسلسل_نشاندہی_الف}
$\sum\limits_{n=1}^{\infty}\frac{(n!)^n}{(n^n)^2}$
\انتہا{سوال}
%====================
\ابتدا{سوال}
$\sum\limits_{n=1}^{\infty}\frac{(n!)^n}{n^{(n^2)}}$
\انتہا{سوال}
%===================
\ابتدا{سوال}
 $\sum\limits_{n=1}^{\infty}\frac{n^n}{2^{(n^2)}}$
\انتہا{سوال}
%=====================
\ابتدا{سوال}
 $\sum\limits_{n=1}^{\infty}\frac{n^n}{(2^n)^2}$
\انتہا{سوال}
%=====================
\ابتدا{سوال}
 $\sum\limits_{n=1}^{\infty}\frac{1\cdot3\cdot.\cdots\cdot (2n-1)}{4^n2^nn!}$
\انتہا{سوال}
%=====================
\ابتدا{سوال}\شناخت{سوال_تسلسل_نشاندہی_ب}
 $\sum\limits_{n=1}^{\infty}\frac{1\cdot 3\cdot\cdots \cdot (2n-1)}{[2\cdot4\cdot.\cdots\cdot(2n)](3^n+1)}$
\انتہا{سوال}
%=====================
\موٹا{نظریہ اور مثالیں}\\
\ابتدا{سوال}
\عددی{p} تسلسل کے ساتھ یا تناسبی پرکھ اور نا ہی \عددی{n} واں جذر پرکھ کارآمد ثابت ہوتا ہے۔ انہیں درج ذیل پر لاگو کر کے دکھائیں کہ دونوں پرکھ اس کی ارتکاز یا انفراج دریافت کرنے سے قاصر ہیں۔
\begin{align*}
\sum_{n=1}^{\infty}\frac{1}{n^p}
\end{align*} 
\انتہا{سوال}
%===================
\ابتدا{سوال}
دکھائیں کہ تناسبی پرکھ اور \عددی{n} واں جذر پرکھ درج ذیل کی ارتکاز یا انفراج معلوم نہیں کر سکتے ہیں۔
\begin{align*}
\sum_{n=2}^{\infty}\frac{1}{(\ln n)^p}&&\text{\RL{$p$ مستقل}}
\end{align*}
\انتہا{سوال}
%====================
\ابتدا{سوال}
فرض کریں 
$a_n=\begin{cases}
n/2^n&\text{\RL{$n$ عدد مفرد}}\\
1/2^n&\text{\RL{دیگر صورت}}
\end{cases}$
ہے۔ کیا \عددی{\sum a_n} مرتکز ہے؟اپنے جواب کی وجہ پیش کریں۔
\انتہا{سوال}
%==================
