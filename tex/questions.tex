\حصہء{سوالات}
\موٹا{مقدار معلوم منحنیات کے مماس}\\
سوال \حوالہ{سوال_مخروط_مقدار_معلوم_مماس_الف} تا سوال \حوالہ{سوال_مخروط_مقدار_معلوم_مماس_ب} میں \عددی{t}  پر مقدار معلوم منحنی کے مماس کی مساوات  دریافت کریں گے۔ اس نقطہ پر \عددی{\tfrac{\dif^{\,2}y}{\dif x^2}} بھی معلوم کریں۔

\ابتدا{سوال}\شناخت{سوال_مخروط_مقدار_معلوم_مماس_الف}
$x=2\cos t,\quad y=2\sin t,\quad t=\tfrac{\pi}{4}$
\انتہا{سوال}
%=======================
\ابتدا{سوال}
$x=\sin 2\pi t,\quad y=\cos 2\pi t,\quad t=-\frac{1}{6}$
\انتہا{سوال}
%=======================
\ابتدا{سوال}
$x=4\sin t,\quad y=2\cos t,\quad t=\frac{\pi}{4}$
\انتہا{سوال}
%=======================
\ابتدا{سوال}
$x=\cos t,\quad y=\sqrt{3}\cos t,\quad t=\frac{2\pi}{3}$
\انتہا{سوال}
%=======================
\ابتدا{سوال}
$x=t,\quad y=\sqrt{t},\quad t=\frac{1}{4}$
\انتہا{سوال}
%=======================
\ابتدا{سوال}
$x=\sec^2t,\quad y=\tan t,\quad t=-\frac{\pi}{4}$
\انتہا{سوال}
%=======================
\ابتدا{سوال}
$x=\sec t,\quad y=\tan t,\quad t=\frac{\pi}{6}$
\انتہا{سوال}
%=======================
\ابتدا{سوال}
$x=-\sqrt{t+1},\quad y=\sqrt{3t},\quad t=3$
\انتہا{سوال}
%=======================
\ابتدا{سوال}
$x=2t^2+3,\quad y=t^4,\quad t=-1$
\انتہا{سوال}
%=======================
\ابتدا{سوال}
$x=\frac{1}{t},\quad y=-2+\ln t,\quad t=1$
\انتہا{سوال}
%=======================
\ابتدا{سوال}
$x=t-\sin t,\quad y=1-\cos t,\quad t=\frac{\pi}{3}$
\انتہا{سوال}
%=======================
\ابتدا{سوال}\شناخت{سوال_مخروط_مقدار_معلوم_مماس_ب}
$x=\cos t,\quad y=1+\sin t,\quad t=\frac{\pi}{2}$
\انتہا{سوال}
%=======================

\موٹا{خفی مقدار معلوم مساوات}\\
سوال \حوالہ{سوال_مخروط_خفی_مقدار_معلوم_ڈھلوان_الف} تا سوال \حوالہ{سوال_مخروط_خفی_مقدار_معلوم_ڈھلوان_ب} میں \عددی{x} اور \عددی{y} بطور قابل تفرق خفی مقدار معلوم  تفاعل \عددی{x=f(t)}، \عددی{y=g(t)} دیے گئے ہیں۔  دیے گئے \عددی{t} پر منحنی \عددی{x=f(t),\, t=g(t)} کی ڈھلوان تلاش کریں۔

\ابتدا{سوال}\شناخت{سوال_مخروط_خفی_مقدار_معلوم_ڈھلوان_الف}
$x^2-2tx+2t^2=4,\quad 2y^3-3t^2=4,\quad t=2$
\انتہا{سوال}
%=====================
\ابتدا{سوال}
$x=\sqrt{5-\sqrt{t}},\quad y(t-1)=\ln y,\quad t=1$
\انتہا{سوال}
%=====================
\ابتدا{سوال}
$x+2x^{3/2}=t^2+t,\quad y\sqrt{t+1}+2t\sqrt{y}=4,\quad t=0$
\انتہا{سوال}
%=====================
\ابتدا{سوال}\شناخت{سوال_مخروط_خفی_مقدار_معلوم_ڈھلوان_ب}
$x\sin t+2x=t,\quad t\sin t-2t=y,\quad t=\pi$
\انتہا{سوال}
%=====================

\موٹا{منحنیات کی لمبائیاں}\\
سوال \حوالہ{سوال-مخروط_منحنی_لمبائی_درکار_الف} تا سوال \حوالہ{سوال-مخروط_منحنی_لمبائی_درکار_ب} میں منحنیات کی لمبائیاں تلاش کریں۔

\ابتدا{سوال}\شناخت{سوال-مخروط_منحنی_لمبائی_درکار_الف}
$x=\cos t,\quad y=t+\sin t,\quad 0\le t\le \pi$
\انتہا{سوال}
%=====================
\ابتدا{سوال}
$x=t^3,\quad y=\frac{3t^2}{2},\quad 0\le t\le \sqrt{3}$
\انتہا{سوال}
%=====================
\ابتدا{سوال}
$x=\frac{t^2}{2},\quad y=\frac{(2t+1)^{3/2}}{3},\quad 0\le t\le 4$
\انتہا{سوال}
%=====================
\ابتدا{سوال}
$x=\frac{(2t+3)^{3/2}}{3},\quad y=t+\frac{t^2}{2},\quad 0\le t\le 3$
\انتہا{سوال}
%=====================
\ابتدا{سوال}
$x=8\cos t+8t\sin t,\, y=8\sin t-8t\cos t,\, 0\le t\le \frac{\pi}{2}$
\انتہا{سوال}
%=====================
\ابتدا{سوال}\شناخت{سوال-مخروط_منحنی_لمبائی_درکار_ب}
$x=\ln(\sec t+\tan t),\, y=\cos t,\, 0\le t\le \frac{\pi}{3}$
\انتہا{سوال}
%=====================

\موٹا{سطحی رقبے}\\
سوال \حوالہ{سوال_مخروط_سطح_طواف_الف} تا سوال \حوالہ{سوال_مخروط_سطح_طواف_ب} میں دیے گئے محور کے گرد منحنی گھما کر سطح طواف پیدا کیا جاتا ہے۔ اس سطح کا رقبہ معلوم کریں۔

\ابتدا{سوال}\شناخت{سوال_مخروط_سطح_طواف_الف}
محور \عددی{x}؛ \quad
$x=\cos t,\quad y=2+\sin t,\quad 0\le t\le 2\pi$
\انتہا{سوال}
%======================
\ابتدا{سوال}
محور\عددی{y}؛ \quad
$x=\frac{3}{2}t^{3/2},\quad y=2\sqrt{t},\quad 0\le t\le \sqrt{3}$
\انتہا{سوال}
%======================
\ابتدا{سوال}
محور \عددی{y}؛ \quad
$x=t+\sqrt{2},\quad y=\frac{t^2}{2}+\sqrt{2}t,\quad -\sqrt{2}\le t\le \sqrt{2}$
\انتہا{سوال}
%======================
\ابتدا{سوال}\شناخت{سوال_مخروط_سطح_طواف_ب}
محور \عددی{x}؛ \quad
$x=\ln(\sec t+\tan t)-\sin t,\quad y=\cos t,\quad 0\le t\le \frac{\pi}{3}$
\انتہا{سوال}
%======================
\ابتدا{سوال}\ترچھا{مخروط مقطوع}\\
نقطہ \عددی{(0,1)} اور \عددی{(2,2)} کے بیچ لکیر کو محور \عددی{x} کے گرد گھما کر مخروط مقطوع کا سطح طواف پیدا کیا جاتا ہے۔مقدار معلوم مساوات \عددی{x=2t,\, y=t+1,\, 0\le t\le 1} استعمال کرتے ہوئے سطح طواف کا رقبہ معلوم کریں۔ نتیجہ کا جیومیٹری کے کلیہ \عددی{S=\pi(r_1+r_2)(\text{\RL{ترچھا قد}})} کے ساتھ موازنہ کریں۔
\انتہا{سوال}
%====================
\ابتدا{سوال}\ترچھا{مخروط}\\
مبدا اور نقطہ \عددی{(h,r)} کے بیچ قطع کو محور \عددی{x} کے گرد گھما کر مخروط سطح طواف پیدا کیا جاتا ہے جس کے قاعدے کا رداس \عددی{r} اور  قد \عددی{h} ہوں گے۔ مقدار معلوم مساوات \عددی{x=ht,\, y=rt,\, 0\le t\le 1} استعمال کرتے ہوئے سطح طواف کا رقبہ تلاش کریں۔  نتیجے کا موازنہ جیومیٹری کے کلیہ \عددی{S=\pi r(\text{\RL{ترچھا قد}})} کے ساتھ کریں۔
\انتہا{سوال}
%========================

\موٹا{وسطانی مراکز}\\
\ابتدا{سوال}
(ا) درج ذیل منحنی کے وسطانی مرکز کے محدد تلاش کریں۔
\begin{align*}
x=\cos t+t\sin t,\quad y=\sin t-t\cos t,\quad 0\le t\le\frac{\pi}{2}
\end{align*}
(ب) یہ منحنی شکل \حوالہ{شکل_سوال_دائرے_کا_در_پیچیدہ} میں دکھائی گئی در پیچیدہ کا حصہ ہے۔ اس منحنی کو  ترسیم کریں۔ منحنی کا وسطانی مرکز \عددی{1} اعشاریہ تک تلاش کر کے ترسیم  پر دکھائیں۔
\انتہا{سوال}
%=====================
\ابتدا{سوال}
(ا) درج ذیل منحنی کے وسطانی مرکز کے محدد تلاش کریں۔
\begin{align*}
x=e^t\cos t,\quad y=e^t\sin t,\quad 0\le t\le\pi
\end{align*}
(ب) اس منحنی کو  ترسیم کریں۔ منحنی کا وسطانی مرکز  \عددی{1} اعشاریہ تک تلاش کر کے ترسیم  پر دکھائیں۔
\انتہا{سوال}
%==================
\ابتدا{سوال}
(ا) درج ذیل منحنی کے وسطانی مرکز کے محدد تلاش کریں۔
\begin{align*}
x=\cos t,\quad y=t+\sin t,\quad 0\le t\le\pi
\end{align*}
(ب) اس منحنی کو  ترسیم کریں۔ منحنی کا وسطانی مرکز  \عددی{1} اعشاریہ تک تلاش کر کے ترسیم  پر دکھائیں۔
\انتہا{سوال}
%==================
\ابتدا{سوال}\ترچھا{تکمل کی قیمت}\\
وسطانی مراکز کے مسائل کو عموماً کیلکولیٹر یا کمپیوٹر کی مدد سے حل کیا جاتا ہے۔درج ذیل منحنی کا وسطانی مرکز \عددی{2} اعشاریہ تک کیلکولیٹر یا کمپیوٹر کی مدد سے تلاش کریں۔
\انتہا{سوال}
%===================

\موٹا{نظریہ اور مثالیں}\\
\ابتدا{سوال}\ترچھا{لمبائی کا دارومدار مقدار معلوم مساوات پر نہیں ہوتا ہے۔}\\
نصف دائرہ \عددی{y=\sqrt{1-x^2}} کی لمبائی درج ذیل مقدار معلوم مساوات استعمال کرتے ہوئے تلاش کریں۔
\begin{enumerate}[a.]
\item
$x=\cos 2t,\, y=\sin 2t,\, 0\le t\le \tfrac{\pi}{2}$
\item
$x=\sin \pi t,\, y=\cos \pi t,\, -\frac{1}{2}\le t\le \frac{1}{2}$
\end{enumerate}
آپ دیکھیں گے کہ دونوں جوابات یکساں ہیں۔
\انتہا{سوال}
%=================
\ابتدا{سوال}\ترچھا{ترخیمی تکمل}\\
ترخیم \عددی{x=a\cos t,\, y=b\sin t,\, 0\le t\le 2\pi} کی لمبائی درج ذیل ہے
\begin{align*}
L=4a\int_0^{\pi/2}\sqrt{1-e^2\cos^2t}\dif t
\end{align*}
جہاں \عددی{e} ترخیم کی سنک ہے۔ ماسوائے \عددی{e=0} یا \عددی{e=1} یہ تکمل، جو \اصطلاح{ترخیمی تکمل}\فرہنگ{تکمل!ترخیمی}\فرہنگ{ترخیمی!تکمل}\حاشیہب{elliptic integral}\فرہنگ{elliptic!integral} کہلاتا ہے، غیر بنیادی ہے۔ 
\begin{enumerate}[a.]
\item
قاعدہ ذوزنقہ میں \عددی{n=10} لے کر \عددی{a=1} اور \عددی{e=\tfrac{1}{2}} کے لئے  اس ترخیم کی لمبائی کا اندازہ لگائیں۔
\item
تفاعل \عددی{f(t)=\sqrt{1-e^2\cos^2t}} کے دو گنّا تفرق کی قیمت \عددی{1} سے کم ہے۔ اس حقیقت کو استعمال کرتے ہوئے جزو-ا میں حاصل قیمت میں خلل کا بالائی حد تلاش کریں۔
\end{enumerate}
\انتہا{سوال}
%======================
\ابتدا{سوال}\شناخت{سوال_مخروط_مقدار_معلوم_لمبائی_کلیہ_الف}
جیسا حصہ \حوالہ{حصہ_مخروط_مستوی_منحنیات_کے_مقدار_معلوم_روپ} میں ذکر کیا گیا، وقفہ \عددی{[a,b]} پر تفاعل \عددی{y=f(x)} کے ترسیم کی مقدار معلوم روپ درج ذیل ہو گی۔
\begin{align*}
x=x,\quad y=f(x),\quad a\le x\le b
\end{align*} 
یہاں \عددی{x} از خود مقدار معلوم ہے۔

اس مقدار معلوم روپ کے لئے دکھائیں کہ مقدار معلوم لمبائی
\begin{align*}
L=\int_a^b\sqrt{\big(\frac{\dif x}{\dif t}\big)^2+\big(\frac{\dif y}{\dif t}\big)^2}\dif t
\end{align*}
درج ذیل کارتیسی صورت اختیار کرتی ہے جس کو حصہ \حوالہ{حصہ_استعمال_تکمل_لمبائی_منحنی} میں حاصل کیا گیا۔
\begin{align*}
L=\int_a^b\sqrt{1+\big(\frac{\dif y}{\dif x}\big)^2}\dif x
\end{align*}
یوں کارتیسی کلیہ در حقیقت مقدار معلوم کلیہ کی ایک مخصوص صورت ہے۔
\انتہا{سوال}
%================
\ابتدا{سوال}\شناخت{سوال_مخروط_مقدار_معلوم_لمبائی_کلیہ_ب}
دکھائیں کہ منحنی \عددی{x=g(y),\, c\le y\le d} کی لمبائی کا کارتیسی کلیہ (مساوات \حوالہ{مساوات_تکمل_استعمال_لمبائی_قوس_پ})
\begin{align*}
L=\int_c^d\sqrt{1+\big(\frac{\dif x}{\dif y}\big)^2}\dif y
\end{align*}
در حقیقت درج ذیل مقدار  معلوم کلیہ کی مخصوص صورت ہے۔
\begin{align*}
L=\int_a^b\sqrt{\big(\frac{\dif x}{\dif t}\big)^2+\big(\frac{\dif y}{\dif t}\big)^2}\dif t
\end{align*}
\انتہا{سوال}
%=====================
\ابتدا{سوال}
درج ذیل تدویر کی ایک محراب کے نیچے رقبہ تلاش کریں۔
\begin{align*}
x=a(\theta-\sin\theta),\quad y=a(1-\cos\theta)
\end{align*} \quad
(اشارہ:\عددی{\dif x=\tfrac{\dif x}{\dif \theta}\dif \theta} استعمال کریں۔)
\انتہا{سوال}
%=====================
\ابتدا{سوال}
درج ذیل تدویر کی ایک محراب کی لمبائی معلوم کریں۔
\begin{align*}
x=a(\theta-\sin\theta),\quad y=a(1-\cos\theta)
\end{align*}
\انتہا{سوال}
%===================
\ابتدا{سوال}
تدویر \عددی{x=\theta-\sin\theta,\, y=1-\cos\theta} کی ایک محراب کو محور \عددی{x} کے گرد گھما کر سطح طواف پیدا کیا جاتا ہے۔ اس سطح کا رقبہ تلاش کریں۔
\انتہا{سوال}
%======================
\ابتدا{سوال}
محور \عددی{x} اور تدویر \عددی{x=\theta-\sin\theta,\, y=1-\cos\theta} کے ایک محراب کے بیچ خطہ کو محور \عددی{x} کے گرد گھما کر جسم طواف پیدا کیا جاتا ہے۔ اس ٹھوس جسم کا حجم تلاش کریں۔ (اشارہ: \عددی{\dif H=\pi y^2\dif x=\pi y^2\tfrac{\dif x}{\dif \theta}\dif \theta})
\انتہا{سوال}
%======================

\موٹا{کمپیوٹر کا استعمال}\\
سوال \حوالہ{سوال_مخروط_لساجس_الف} اور سوال \حوالہ{سوال_مخروط_لساجس_الف} میں \اصطلاح{لساجس اشکال}\فرہنگ{لساجس اشکال}\حاشیہب{Lissajous figures}\فرہنگ{Lissajous figures} دکھائی گئی ہیں۔  دونوں سوالات میں ربع اول میں وہ نقطہ تلاش کریں جہاں منحنی کا مماس افقی ہو۔ مبدا پر دو مماس کی مساوات تلاش کریں۔

\ابتدا{سوال}\شناخت{سوال_مخروط_لساجس_الف}
ترسیم شکل \حوالہ{شکل_سوال_مخروط_لساجس_الف} میں دی گئی ہے۔
\انتہا{سوال}
%===================
\ابتدا{سوال}\شناخت{سوال_مخروط_لساجس_ب}
ترسیم شکل \حوالہ{شکل_سوال_مخروط_لساجس_ب} میں دی گئی ہے۔
\انتہا{سوال}
%===================
\begin{figure}
\centering
\begin{minipage}{0.45\textwidth}
\centering
\begin{tikzpicture}[font=\scriptsize,declare function={f(\x)=sin(deg(\x)); g(\x)=sin(deg(2*\x));}]
\pgfmathsetmacro{\k}{2*pi}
\begin{axis}[clip=false,width=4cm,axis lines=middle, xlabel={$x$},ylabel={$y$},xlabel style={at={(current axis.right of origin)},anchor=west},ylabel style={at={(current axis.above origin)},anchor=south},enlargelimits=true,ytick={\empty}]
\addplot[smooth,domain=0:\k]({f(x)},{g(x)});
\addplot[]plot coordinates {(1,1)}node[right]{$\begin{aligned}x&=\sin t\\ y&=\sin 2t  \end{aligned}$};;
\end{axis}
\end{tikzpicture}
\caption{ترسیم سوال \حوالہ{سوال_مخروط_لساجس_الف}}
\label{شکل_سوال_مخروط_لساجس_الف}
\end{minipage}\hfill
\begin{minipage}{0.45\textwidth}
\centering
\begin{tikzpicture}[font=\scriptsize,declare function={f(\x)=sin(deg(2*\x)); g(\x)=sin(deg(3*\x));}]
\pgfmathsetmacro{\k}{2*pi}
\begin{axis}[clip=false,width=4cm,axis lines=middle, xlabel={$x$},ylabel={$y$},xlabel style={at={(current axis.right of origin)},anchor=west},ylabel style={at={(current axis.above origin)},anchor=south},enlargelimits=true,ytick={\empty}]
\addplot[smooth,domain=0:\k]({f(x)},{g(x)});
\addplot[]plot coordinates {(1,1)}node[right]{$\begin{aligned}x&=\sin 2t\\ y&=\sin 3t  \end{aligned}$};;
\end{axis}
\end{tikzpicture}
\caption{ترسیم سوال \حوالہ{سوال_مخروط_لساجس_ب}}
\label{شکل_سوال_مخروط_لساجس_ب}
\end{minipage}
\end{figure}
