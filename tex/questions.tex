\حصہ{لامتناہی تسلسل}
سائنس اور ریاضیات میں تفاعل کو عموماً درج ذیل صورت کی لامتناہی کثیر رکنی کی صورت میں لکھا جاتا ہے۔
\begin{align*}
\frac{1}{1-x}=1+x+x^2+x^3+\cdots+x^n+\cdots,\quad \abs{x}<1
\end{align*}
\عددی{x} کی کسی بھی جائز قیمت کے لئے ہم  لامتناہی تعداد کے مستقلوں کا مجموعہ، جس کو \ترچھا{لامتناہی تسلسل} کہا جاتا ہے، لے کر کثیر رکنی کی قیمت حاصل کرتے ہیں۔  اس حصہ اور اگلے چار حصوں میں ہم لامتناہی تسلسل سے واقف ہونے کی کوشش کرتے ہیں۔

\جزوحصہء{تسلسل اور جزوی مجموعے}
ہم پوچھتے ہیں کہ درج ذیل  فقرے کا کیا مطلب ہے؟
\begin{align*}
1+\frac{1}{2}+\frac{1}{4}+\frac{1}{8}+\frac{1}{16}+\cdots
\end{align*} 
چونکہ ہم لامتناہی مستقلوں کو کبھی بھی جمع نہیں کر سکتے ہیں لہٰذا ہم پہلی جزو سے شروع کر کے بتدریج ایک ایک جزو ساتھ جمع کر کے جزوی مجموعہ میں کسی نقش کو پہچاننے کی کوشش کرتے ہیں۔انہیں جدول \حوالہ{جدول_تسلسل_جزوی_مجموعے} میں دکھایا گیا ہے جن میں یقیناً ایک نقش پایا جاتا ہے۔ جزوی مجموعوں کی ترتیب کا \عددی{n} واں جزو درج ذیل ہے۔
\begin{align*}
a_n=2-\frac{1}{2^{n-1}}
\end{align*}
چونکہ \عددی{\lim_{n\to\infty}(1/{2^n})=0} ہے لہٰذا اس ترتیب کا حد \عددی{2} ہے۔یوں درج ذیل لامتناہی تسلسل کا مجموعہ \عددی{2} ہو گا۔
\begin{align*}
1+\frac{1}{2}+\frac{1}{4}+\cdots+\frac{1}{2^{n-1}}+\cdots
\end{align*}

\begin{table}
\caption{تفاعل کے جزوی مجموعے۔}
\label{جدول_تسلسل_جزوی_مجموعے}
\centering
\renewcommand{\arraystretch}{1.25}
\begin{tabular}{CLL}
\toprule
\text{\RL{جزوی مجموعہ}}&&\text{قیمت}\\
\midrule
\text{پہلا:}&s_1=1&2-1\\
\text{دوسرا:}&s_2=1+\frac{1}{2}&2-\frac{1}{2}\\
\text{تیسرا:}&s_3=1+\frac{1}{2}+\frac{1}{4}&2-\frac{1}{4}\\
\vdots\\
\text{\RL{$n$ واں:}}&s_n=1+\frac{1}{2}+\frac{1}{4}+\cdots+\frac{1}{2^{n-1}}&2-\frac{1}{2^{n-1}}\\
\bottomrule
\end{tabular}
\end{table}

کیا اس تسلسل کے کسی بھی متناہی تعداد کے اجزاء کا مجموعہ \عددی{2} ہو گا؟  نہیں۔ کیا ہم لامتناہی تعداد کے مستقل کو ایک ایک کر کے جمع کر سکتے ہیں؟ نہیں۔ اس کے باوجود ہم تسلسل کے حد کی تعریف کو \عددی{n\to\infty} پر تسلسل کے جزوی مجموعے کا حد لے سکتے ہیں جو مذکورہ بالا تسلسل کے لئے \عددی{2} ہو گا (شکل \حوالہ{شکل_تسلسل_مجموعہ_کی_تعریف})۔ترتیب اور تسلسل کا علم ہمیں متناہی مجموعوں کی قید سے آزاد کرتا ہے۔
\begin{figure}
\centering
\begin{tikzpicture}[font=\small]
\pgfmathsetmacro{\k}{3}
\draw(0,0)--(2.25*\k,0);
\foreach \x in {0,1,1.5,1.75,1.875}{\draw(\x*\k,-0.1)--++(0,0.2);}
\draw(2*\k,0)node[circ]{};
\foreach \x in {0,1,2}{\draw(\x*\k,0)node[below]{$\x$};}
\draw [decoration={brace,raise=0.5cm},decorate](0,0) -- (\k,0) 
node [pos=0.5,anchor=south,yshift=0.55cm] {$1$}; 
\draw [decoration={brace,mirror,raise=0.5cm},decorate](\k,0) -- (1.5*\k,0) 
node [pos=0.5,anchor=north,yshift=-0.55cm] {$1/2$}; 
\draw [decoration={brace,raise=0.5cm},decorate](1.5*\k,0) -- (1.75*\k,0) 
node [pos=0.5,anchor=south,yshift=0.55cm] {$1/4$}; 
\draw [decoration={brace,mirror,raise=0.5cm},decorate](1.75*\k,0) -- (1.875*\k,0) 
node [pos=0.5,anchor=north,yshift=-0.55cm] {$1/8$}; 
\end{tikzpicture}
\caption{جیسے جیسے لمبائیاں \عددی{1}، \عددی{1/2}، \عددی{1/4}، \عددی{1/8}، \نقطے جمع کی جائیں، مجموعہ \عددی{2} کے قریب تر ہوتا جاتا ہے۔}
\label{شکل_تسلسل_مجموعہ_کی_تعریف}
\end{figure}

\ابتدا{تعریف}
دیے گئے اعداد کی ترتیب \عددی{\{a_n\}} کی صورت میں درج ذیل صورت کا فقرہ \اصطلاح{لامتناہی تسلسل}\فرہنگ{تسلسل!لامتناہی}\حاشیہب{infinite series}\فرہنگ{series!infinite} کہلاتا ہے۔
\begin{align*}
a_1+a_2+a_3+\cdots+a_n+\cdots
\end{align*} 
عدد \عددی{a_n} کو اس تسلسل کا \اصطلاح{\عددی{n} واں جزو}\فرہنگ{تسلسل!جزو}\حاشیہب{nth term}\فرہنگ{series!nth term} کہتے ہیں۔ ترتیب \عددی{\{s_n\}} جس کی تعریف درج ذیل ہے
\begin{align*}
s_1&=a_1\\
s_2&=a_1+a_2\\
&\vdots\\
s_n&=a_1+a_2+\cdots+a_n=\sum_{k=1}^n a_k\\
&\vdots
\end{align*}
 کو اس تسلسل کے \اصطلاح{جزوی مجموعوں کی ترتیب} کہتے ہیں اور \عددی{s_n} کو \عددی{n} واں جزوی مجموعہ  کہتے ہیں۔ اگر جزوی مجموعوں کی ترتیب \عددی{L} پر مرتکز ہو تب ہم کہتے ہیں کہ یہ تسلسل \اصطلاح{مرتکز}\فرہنگ{تسلسل!ارتکاز}\فرہنگ{series!convergence} ہے اور اس کا مجموعہ \عددی{L} ہے۔ایسی صورت میں ہم درج ذیل بھی لکھتے ہیں۔
\begin{align*}
a_1+a_2+\cdots+a_n+\cdots=\sum_{k=1}^{\infty} a_n=L
\end{align*}
اگر تسلسل کے جزوی مجموعوں کی ترتیب مرتکز نہ ہو تب ہم کہتے ہیں کہ تسلسل \اصطلاح{منفرج}\فرہنگ{تسلسل!انفراج}\فرہنگ{series!divergence} ہے۔
\انتہا{تعریف}
%=========================


تسلسل \عددی{a_1+a_2+\cdots+a_n+\cdots} پر غور کرنے سے پہلے ضروری نہیں کہ ہمیں معلوم ہو کہ آیا یہ تسلسل مرتکز یا منفرج ہے۔ بہر حال اس تسلسل کو درج ذیل صورت میں لکھنا مفید ہوتا ہے۔
\begin{align*}
\sum_{n=1}^{\infty}a_n,\quad \sum_{k=1}^{\infty}a_k,\quad \sum a_n \,\, \text{\RL{(مجموعہ \عددی{1} تا \عددی{\infty} ہو گا)}}
\end{align*}

\جزوحصہء{ہندسی تسلسل}
درج ذیل صورت کے تسلسل کو \اصطلاح{ہندسی تسلسل}\فرہنگ{تسلسل!ہندسی}\حاشیہب{geometric series}\فرہنگ{series!geometric} کہتے ہیں جہاں \عددی{a} اور \عددی{r} مقررہ حقیقی اعداد ہیں اور \عددی{a\ne 0} ہے۔ 
\begin{align}\label{مساوات_تسلسل_ہندسی_الف}
a+ar+ar^2+\cdots+ar^{n-1}+\cdots=\sum_{n=1}^{\infty}ar^{n-1}
\end{align}
درج ذیل میں نسبت \عددی{r} مثبت ہے
\begin{align*}
a+\frac{1}{2}+\frac{1}{4}+\cdots+\big(\frac{1}{2}\big)^{n-1}+\cdots
\end{align*}
جبکہ درج ذیل میں \عددی{r} منفی  ہے۔
\begin{align*}
a-\frac{1}{3}+\frac{1}{9}-\cdots+\big(-\frac{1}{3}\big)^{n-1}+\cdots
\end{align*}

اگر \عددی{r=1} ہو تب مساوات \حوالہ{مساوات_تسلسل_ہندسی_الف} کا \عددی{n} واں جزوی مجموعہ 
\begin{align*}
s_n=a+a(1)+a(1)^2+\cdots+a(1)^{n-1}=na
\end{align*}
ہو گا جو \عددی{\lim_{n\to\infty}s_n=\mp\infty} کی بنا منفرج ہے جہاں  علامت، \عددی{a} کی علامت پر منحصر ہو گی۔ اگر \عددی{r=-1} ہو تب تسلسل کے  جزوی مجموعے یک بعد دیگرے \عددی{a} اور \عددی{0} ہوں گے لہٰذا تسلسل منفرج ہو گا۔ اگر \عددی{\abs{r}\ne 1} تب تسلسل کا ارتکاز یا انفراج درج ذیل طریقہ سے جاننا ممکن ہو گا۔
\begin{align*}
s_n&=a+ar+ar^2+\cdots+ar^{n-2}+ar^{n-1}\\
rs_n&=ar+ar^2+\cdots+ar^{n-1}+ar^n&&\text{\RL{$s_n$ کو $r$ سے ضرب دیں}}\\
s_n-rs_n&=a-ar^n&&\text{\RL{$s_n$ سے $rs_n$ منفی کریں}}\\
s_n(1-r)&=a(1-r^n)&&\text{\RL{تجزی}}\\
s_n&=\frac{a(1-r^n)}{1-r},\quad (r\ne 1)&& \text{\RL{$r\ne 1$ کی صورت میں $s_n$ کا حل}}
\end{align*}
اگر \عددی{\abs{r}<1} ہو تب \عددی{n\to \infty} سے \عددی{r^n\to 0}  (حصہ \حوالہ{حصہ_ترتیب_حد_تلاش_کے_مسائل}) لہٰذا \عددی{s_n=\tfrac{a}{1-r}} ہوں گے۔ اس کے برعکس \عددی{\abs{r}>1} کی صورت میں \عددی{\abs{r^n}\to\infty} کی بنا تسلسل منفرج ہو گا۔

یوں \عددی{\abs{r}<1} کی صورت میں ہندسی تسلسل \عددی{a+ar+ar^2+\cdots+ar^{n-1}+\cdots} عدد \عددی{\tfrac{a}{1-r}} پر مرتکز ہو گا:
\begin{align}
\sum_{n=1}^{\infty}ar^{n-1}=\frac{a}{1-r},\quad \abs{r}<1
\end{align}
\عددی{\abs{r}>1} کی صورت میں تسلسل منفرج ہو گا۔

\ابتدا{مثال}
درج ذیل ہندسی تسلسل میں \عددی{a=\tfrac{1}{9}} اور \عددی{r=\tfrac{1}{3}} ہیں۔
\begin{align*}
\frac{1}{9}+\frac{1}{27}+\frac{1}{81}+\cdots=\sum_{n=1}^{\infty}\frac{1}{9}\big(\frac{1}{3}\big)^{n-1}=\frac{1/9}{1-(1/3)}=\frac{1}{6}
\end{align*}
\انتہا{مثال}
%========================
\ابتدا{مثال}
درج ذیل ہندسی تسلسل میں \عددی{a=-\tfrac{5}{4}} اور \عددی{r=-\tfrac{1}{4}} ہیں۔
\begin{align*}
\sum_{n=1}^{\infty}\frac{(-1)^n5}{4^n}=-\frac{5}{4}+\frac{5}{16}-\frac{5}{64}+\cdots
\end{align*}
یہ ہندسی تسلسل \عددی{-1} پر مرتکز ہے۔
\begin{align*}
\frac{a}{1-r}=\frac{-5/4}{1+(1/4)}=-1
\end{align*}
\انتہا{مثال}
%========================
\ابتدا{مثال}
آپ ایک گیند کو افقی سطح پر \عددی{a} میٹر بلندی سے گراتے ہیں۔ یہ گیند \عددی{h} بلندی سے گر کر \عددی{rh} بلندی تک اچھلتا  ہے جہاں \عددی{r} مثبت اور \عددی{1} سے کم ہے۔ یہ گیند اوپر اور نیچے سفر کرتے ہوئے کل کتنا فاصلہ طے کرتا ہے؟

حل:\quad
کل فاصلہ درج ذیل ہو گا۔
\begin{align*}
s=a+\underbrace{2ar+2ar^2+2ar^3+\cdots}_{2ar/(1-r)}=a+\frac{2ar}{1-r}=a\,\frac{1+r}{1-r}
\end{align*}
یوں \عددی{a=\SI{6}{\meter}} اور \عددی{r=\tfrac{2}{3}} کی صورت میں طے شدہ فاصل درج ذیل ہو گا۔
\begin{align*}
s=6\,\frac{1+(2/3)}{1-(2/3)}=6\big(\frac{5/3}{1/3}\big)=\SI{30}{\meter}
\end{align*} 
\انتہا{مثال}
%====================
\ابتدا{مثال}\ترچھا{دہراتے اعشاری}\\
دہراتے اعشاری \عددی{5.23\, 23\, 23\cdots} کو دو عدد صحیح کا نسبت لکھیں۔

حل:\quad
\begin{align*}
5.23\,23\,23\cdots&=5+\frac{23}{100}+\frac{23}{(100)^2}+\frac{23}{(100)^3}+\cdots\\
&=5+\frac{23}{100}\underbrace{\big(1+\frac{1}{100}+\big(\frac{1}{100}\big)^2+\cdots\big)}_{1/(1-0.01)}&&a=1,\, r=\tfrac{1}{100}\\
&=5+\frac{23}{100}\big(\frac{1}{0.99}\big)=5+\frac{23}{99}=\frac{518}{99}
\end{align*}
\انتہا{مثال}
%=======================

\جزوحصہء{دوربینی تسلسل}
مرتکز ہندسی تسلسل کے مجموعہ کے کلیہ کی طرح تسلسل کے مجموعوں کے کلیات بہت کم پائے جاتے ہیں لہٰذا ہمیں تسلسل کے مجموعہ کی اندازاً قیمت پر گزارا کرنا ہو گا۔البتہ اگلی مثال میں بھی ایسا تسلسل دیا گیا ہے جس کا بالکل ٹھیک  مجموعہ تلاش کیا جا سکتا ہے۔

\ابتدا{مثال}
تسلسل \عددی{\sum_{n=1}^{\infty}\tfrac{1}{n(n+1)}} کا مجموعہ تلاش کریں۔

حل:\quad
 جزوی مجموعوں کی ترتیب میں ایسا نقش دیکھنے کی کوشش کرتے ہیں جس سے \عددی{s_n} کا کلیہ اخذ کیا جا سکتا ہو۔ہم جزوی کسر 
\begin{align}
\frac{1}{k(k+1)}=\frac{1}{k}-\frac{1}{k+1}
\end{align}
استعمال کر کے جزوی مجموعہ
\begin{align*}
\sum_{n=1}^{k}\frac{1}{n(n+1)}=\frac{1}{1\cdot 2}+\frac{1}{2\cdot 3}+\cdots+\frac{1}{k\cdot(k+1)}
\end{align*}
کو
\begin{align}
s_k=\big(\frac{1}{1}-\frac{1}{2}\big)+\big(\frac{1}{2}-\frac{1}{3}\big)+\cdots+\big(\frac{1}{k}-\frac{1}{k+1}\big)
\end{align}
 لکھتے ہیں۔قوسین کھول کر یکساں اجزاء کاٹ کر درج ذیل حاصل ہوتا ہے۔
\begin{align}
s_n=1-\frac{1}{k+1}
\end{align}
اب \عددی{k\to \infty} سے \عددی{s_k\to 1} حاصل ہو گا۔ یہ تسلسل منفرج ہے اور اس کا مجموعہ \عددی{1} ہے۔
\begin{align*}
\sum_{n=1}^{\infty}\frac{1}{n(n+1)}=1
\end{align*}
\انتہا{مثال}
%===================

\جزوحصہء{منفرج تسلسل}

