\جزوحصہء{سوالات}
\موٹا{رقبہ بذریعہ دوہرا تکمل}\\
سوال \حوالہ{سوال_بالکثرت_رقبہ_بطور_دوہرا_تکمل_الف} تا سوال \حوالہ{سوال_بالکثرت_رقبہ_بطور_دوہرا_تکمل_ب} میں  منحنیات اور لکیروں کے بیچ خطے کا خاکہ بنا کر اس خطے کے رقبہ کو بطور دوہرا بار بار  تکمل لکھیں۔ اس تکمل کی قیمت دریافت کریں۔

\ابتدا{سوال}\شناخت{سوال_بالکثرت_رقبہ_بطور_دوہرا_تکمل_الف}
محددی محور اور لکیر \عددی{x+y=2}
\انتہا{سوال}
%==================
\ابتدا{سوال}
لکیر \عددی{x=0}، \عددی{y=2x} اور \عددی{y=4}
\انتہا{سوال}
%====================
\ابتدا{سوال}
قطع  مکافی \عددی{x=-y^2} اور لکیر \عددی{y=x+2}
\انتہا{سوال}
%====================
\ابتدا{سوال}
قطع مکافی \عددی{x=y-y^2} اور لکیر \عددی{y=-x}
\انتہا{سوال}
%====================
\ابتدا{سوال}
منحنی \عددی{y=e^x} اور لکیر \عددی{y=0}، \عددی{x=0} اور \عددی{x=\ln 2}
\انتہا{سوال}
%====================
\ابتدا{سوال}
ربع اول میں منحنیات \عددی{y=\ln x}، \عددی{y=2\ln x} اور لکیر \عددی{x=e}
\انتہا{سوال}
%====================
\ابتدا{سوال}
قطع مکافی \عددی{x=y^2} اور \عددی{x=2y-y^2}
\انتہا{سوال}
%====================
\ابتدا{سوال}\شناخت{سوال_بالکثرت_رقبہ_بطور_دوہرا_تکمل_ب}
قطع مکافی \عددی{x=y^2-1} اور \عددی{x=2y^2-2}
\انتہا{سوال}
%====================

سوال \حوالہ{سوال_بالکثرت_رقبہ_سے_خطہ_الف} تا سوال \حوالہ{سوال_بالکثرت_رقبہ_سے_خطہ_ب} میں مستوی \عددی{xy} میں   خطوں کے رقبات کو تکمل یا تکملات کے مجموعوں کی کی صورت میں پیش کیا گیا ہے۔  ان خطوں کا خاکہ بنا کر    سرحدی منحنیات  پر ان  کی مساواتیں لکھیں اور ان نقطوں کی نشاندہی کریں جہاں منحنیات ایک دوسرے کو قطع کرتی ہیں۔ اس کے بعد ان خطہ کا رقبہ تلاش کریں۔ 

\ابتدا{سوال}\شناخت{سوال_بالکثرت_رقبہ_سے_خطہ_الف}
$\int_0^6\int_{y^2/3}^{2y}\dif x\dif y$
\انتہا{سوال}
%===================
\ابتدا{سوال}
$\int_0^3\int_{-x}^{x(2-x)}\dif y\dif x$
\انتہا{سوال}
%====================
\ابتدا{سوال}
$\int_0^{\pi/4}\int_{\sin x}^{\cos x}\dif y\dif x$
\انتہا{سوال}
%===============
\ابتدا{سوال}
$\int_{-1}^2\int_{y^2}^{y+2}\dif x\dif y$
\انتہا{سوال}
%===============
\ابتدا{سوال}
$\int_{-1}^0\dif y\dif x+\int_0^2\int_{-x/2}^{1-x}\dif y\dif x$
\انتہا{سوال}
%===============
\ابتدا{سوال}\شناخت{سوال_بالکثرت_رقبہ_سے_خطہ_ب}
$\int_0^2\int_{x^2-4}^0\dif y\dif x+\int_0^4\int_0^{\sqrt{x}}\dif y\dif x$
\انتہا{سوال}
%===============

\موٹا{اوسط قیمت}\\
\ابتدا{سوال}
تفاعل \عددی{f(x,y)=\sin(x+y)} کی اوسط قیمت درج ذیل خطوں پر تلاش کریں۔
\begin{enumerate}[a.]
\item
مستطیل \عددی{0\le x\le \pi,\, 0\le y\le \pi}
\item
مستطیل \عددی{0\le x\le \pi,\, 0\le y\le \pi/2}
\end{enumerate}
\انتہا{سوال}
%==========
\ابتدا{سوال}
کیا چکور \عددی{0\le x\le 1,\, 0\le y\le 1} یا ربع اول میں دائرہ \عددی{x^2+y^2=1}  میں \عددی{f(x,y)=xy}  کی اوسط قیمت زیادہ ہو گی؟ ان دونوں خطوں میں اوسط کی قیمت تلاش کریں۔
\انتہا{سوال}
%================
\ابتدا{سوال}
چکور \عددی{0\le x\le 2,\, 0\le y\le 2} میں قطع مکافی \عددی{z=x^2+y^2} کا  اوسط قد تلاش کریں۔
\انتہا{سوال}
%===========
\ابتدا{سوال}
چکور \عددی{\ln 2\le x\le 2\ln 2,\, \ln 2\le y\le 2\ln 2} میں \عددی{f(x,y)=\tfrac{1}{xy}} کی اوسط قیمت تلاش کریں۔
\انتہا{سوال}
%================
\موٹا{مستقل کثافت}\\
\ابتدا{سوال}
ربع اول میں قطع مکافی \عددی{y=2-x^2} اور لکیر \عددی{x=0}، \عددی{y=x} کے بیچ ایک باریک چادر جس کی کثافت \عددی{\delta=3} ہو  پائی جاتی ہے۔اس کا مرکز کمیت تلاش کریں۔
\انتہا{سوال}
%================
\ابتدا{سوال}
ربع اول میں محددی محور اور  لکیر \عددی{x=3} اور \عددی{y=3}  کے بیچ  مستقل کثافت کی باریک  مستطیل چادر پائی جاتی ہے۔ اس کے جمودی  معیار اثر اور رداس دوار تلاش کریں۔
\انتہا{سوال}
%============
\ابتدا{سوال}
ربع اول میں محور \عددی{x}، قطع مکافی \عددی{y^2=2x} اور لکیر \عددی{x+y=4} کے بیچ خطہ کا وسطانی مرکز تلاش کریں۔
\انتہا{سوال}
%==========
\ابتدا{سوال}
ربع اول سے لکیر \عددی{x+y=3} ایک تکونی خطہ کاٹتی ہے۔ اس خطہ کا وسطانی مرکز تلاش کریں۔
\انتہا{سوال}
%=================
\ابتدا{سوال}
محور \عددی{x} اور  منحنی \عددی{y=\sqrt{1-x^2}} کے بیچ خطہ کا وسطانی مرکز تلاش کریں۔
\انتہا{سوال}
%=================
\ابتدا{سوال}
ربع اول میں قطع مکافی \عددی{y=6x-x^2}  اور لکیر \عددی{y=2x} کے بیچ خطے کا رقبہ \عددی{\tfrac{125}{6}} ہے۔ اس کا وسطانی مرکز تلاش کریں۔
\انتہا{سوال}
%===================
\ابتدا{سوال}
ربع اول سے دائرہ \عددی{x^2+y^2=a^2}  ایک خطہ کاٹتا ہے۔ اس خطہ کا وسطانی مرکز تلاش کریں۔
\انتہا{سوال}
%===========
\ابتدا{سوال}
دائرہ \عددی{x^2+y^2=4} کے بیچ کثافت \عددی{\delta=1}  کی باریک چادر کی محور \عددی{x} کے لحاظ سے جمودی معیار اثر تلاش کریں۔ اس نتیجہ کو استعمال کرتے ہوئے اس خطہ کی \عددی{I_y} اور \عددی{I_0} دریافت کریں۔
\انتہا{سوال}
%==================
\ابتدا{سوال}
محور \عددی{x} اور قوس \عددی{y=\sin x,\, 0\le x\le \pi} کے بیچ خطہ کا وسطانی مرکز تلاش کریں۔
\انتہا{سوال}
%=================
\ابتدا{سوال}
محور \عددی{x} اور  منحنی \عددی{y=\tfrac{\sin^2x}{x^2}} کے بیچ وقفہ \عددی{\pi\le x\le 2\pi} پر کثافت \عددی{\delta=1} کی باریک چادر پائی جاتی ہے۔ محور \عددی{y} کے لحاظ سے اس کی جمودی معیار اثر تلاش کریں۔
\انتہا{سوال}
%=============
\ابتدا{سوال}\ترچھا{لامتناہی خطہ کا وسطانی مرکز}\\
ربع دوم میں محددی محور اور منحنی \عددی{y=e^x} کے بیچ خطہ کا وسطانی مرکز تلاش کریں۔ (کمیت اور معیار اثر کے کلیات میں آپ کو غیر مناسب تکملات استعمال کرنے ہوں گے۔)
\انتہا{سوال}
%============
\ابتدا{سوال}\ترچھا{لامتناہی چادر کا پہلا معیار اثر}\\
ربع اول میں منحنی \عددی{y=e^{-x^2/2}} کے نیچے  کثافت \عددی{\delta=1} کے لامتناہی جسامت کی چادر   کا محور \عددی{y} کے لحاظ سے پہلا معیار اثر تلاش کریں۔
\انتہا{سوال}
%===============
\موٹا{متغیر کثافت}\\
\ابتدا{سوال}
قطع مکافی \عددی{x=y-y^2} اور لکیر \عددی{x+y=0} کے بیچ باریک چادر کی کثافت \عددی{\delta(x,y)=x+y} ہے۔ محور \عددی{x} کے لحاظ سے اس کی جمودی معیار اثر اور رداس دوار تلاش کریں۔ 
\انتہا{سوال}
%===========
\ابتدا{سوال}
ترخیم \عددی{x^2+4y^2=12} سے قطع مکافی \عددی{x=4y^2}  جس چھوٹے حصہ کو کاٹتا ہے، اس کی کثافت \عددی{\delta(x,y=5x} ہے۔ اس کی کمیت تلاش کریں۔
\انتہا{سوال}
%============
\ابتدا{سوال}
محور \عددی{y} اور  لکیر \عددی{y=x} اور \عددی{y=2-x} کے بیچ تکونی چادر کی کثافت \عددی{\delta(x,y)=6x+3y+3} ہے۔ اس چادر کا مرکز کمیت تلاش کریں۔
\انتہا{سوال}
%===================
\ابتدا{سوال}\شناخت{سوال_بالکثرت_درکار_رقبہ}
منحنیات  \عددی{x=y^2} اور \عددی{x=2y-y^2} کے بیچ باریک چادر کی کثافت \عددی{\delta(x,y)=y+1} ہے۔ اس کی کمیت اور محور \عددی{x} کے لحاظ سے جمودی معیار اثر تلاش کریں۔
\انتہا{سوال}
%=============
\ابتدا{سوال}
ربع اول سے خطوط \عددی{x=6} اور \عددی{y=1} ایک مستطیل  باریک چادر کاٹتے ہیں جس کی کثافت \عددی{\delta(x,y)=x+y+1} ہے۔ اس کی مرکز کمیت اور محور \عددی{y} کے لحاظ سے جمودی معیار اثر  اور رداس دوار تلاش کریں۔
\انتہا{سوال}
%============
\ابتدا{سوال}
قطع مکافی \عددی{y=x^2} اور لکیر \عددی{y=x} کے بیچ باریک چادر کی کثافت \عددی{\delta(x,y)=y+1} ہے۔ اس کا مرکز کمیت اور محور \عددی{y} کے لحاظ سے جمودی معیار اثر اور رداس  دوار تلاش کریں۔
\انتہا{سوال}
%=============
\ابتدا{سوال}
قطع مکافی \عددی{y=x^2}، محور \عددی{x} اور  لکیر \عددی{x=\mp 1} کے بیچ باریک چادر کی کثافت \عددی{\delta(x,y)=7y+1} ہے۔ اس کا مرکز کمیت اور محور \عددی{y} کے لحاظ سے جمودی معیار اثر اور رداس  دوار تلاش کریں۔
\انتہا{سوال}
%================
\ابتدا{سوال}
خطوط  \عددی{x=0}، محور \عددی{x=20}، \عددی{y=-1} اور  \عددی{y=1} کے بیچ باریک چادر کی کثافت \عددی{\delta(x,y)=1+x/20} ہے۔ اس کا مرکز کمیت اور محور \عددی{x} کے لحاظ سے جمودی معیار اثر اور رداس  دوار تلاش کریں۔
\انتہا{سوال}
%================
\ابتدا{سوال}\شناخت{سوال_بالکثرت_قطبی_الف}
لکیر \عددی{y=x}، \عددی{y=-x} اور \عددی{y=1} کے بیچ تکونی چادر کی کثافت \عددی{\delta(x,y)=y+1} ہے۔ اس کا مرکز کمیت  اور محددی محوروں کے لحاظ سے جمودی معیار اثر اور رداس دوار تلاش کریں۔ اس کا قطبی جمودی معیار اثر اور  رداس دوار بھی تلاش کریں۔
\انتہا{سوال}
%=================
\ابتدا{سوال}
کثافت \عددی{\delta(x,y)=3x^2+1} لیتے ہوئے سوال \حوالہ{سوال_بالکثرت_قطبی_الف} کو دوبارہ حل کریں۔
\انتہا{سوال}
%=================

\موٹا{نظریہ اور مثالیں}\\
\ابتدا{سوال}
مستوی \عددی{xy} میں جراثیم   کی تعدادی کثافت \عددی{f(x,y)=\tfrac{10000e^y}{1+\abs{x}/2}} ہے جہاں \عددی{x} اور \عددی{y} کی ناپ سنٹی میٹر میں ہے۔ مستطیل \عددی{-5\le x\le 5,\, -2\le y\le 0} میں جراثیم کی کل تعداد تلاش کریں۔
\انتہا{سوال}
%=================
\ابتدا{سوال}
سطح زمین پر  کثافت آبادی \عددی{f(x,y)=100(y+1)} ہے جہاں \عددی{x} اور \عددی{y} کلومیٹر میں ہیں۔منحنیات \عددی{x=y^2} اور \عددی{x=2y-y^2} کے بیچ کل آبادی کتنی ہو گی؟
\انتہا{سوال}
%==============
\ابتدا{سوال}
مستقل کثافت کا ایک برتن مستوی \عددی{xy} میں خطہ \عددی{0\le y\le a(1-x^2),\, -1\le x\le 1} پر واقع ہے۔ یہ برتن \عددی{45^{circ}}   تک ٹیڑھا کرنے تک واپس اپنی جگہ پر آن گرتا ہے۔ مستقل \عددی{a} کی قیمت تلاش کریں۔
\انتہا{سوال}
%================
\ابتدا{سوال}\ترچھا{جمودی معیار اثر کم سے کم کرنا}\\
ربع اول میں کثافت \عددی{\delta*x,y)=1} کی چادر  لکیر \عددی{x=4} اور \عددی{y=2} کے بیچ  پائی جاتی ہے۔ لکیر \عددی{y=a} کے لحاظ سے اس چادر کی جمودی معیار اثر \عددی{I_a} درج ذیل ہے۔
\begin{align*}
I_a=\int_0^4\int_0^2(y-a)^2\dif y\dif x
\end{align*}
مستقل \عددی{a} کی وہ قیمت تلاش کریں جو \عددی{I_a} کو کم سے کم کرتا ہو۔
\انتہا{سوال}
%================
\ابتدا{سوال}
مستوی \عددی{xy} میں لکیر \عددی{y=\tfrac{1}{\sqrt{1-x^2}}}، \عددی{y=-\tfrac{1}{\sqrt{1-x^2}}}، \عددی{x=0} اور \عددی{x=1}  کے بیچ لامتناہی خطہ کا وسطانی مرکز تلاش کریں۔
\انتہا{سوال}
%================
\ابتدا{سوال}
ایک پتلی چھڑی   کی مستقل  خطی کثافت \عددی{\delta\,\si{\gram\per\centi\meter}}  اور لمبائی \عددی{L} ہے۔  اس کا رداس دوار  دیے گئے محور کے لحاظ سے تلاش کریں۔
\begin{enumerate}[a.]
\item
چھڑی کے محور کو عمودی اور اس کی مرکز کمیت سے گزرتے ہوا خط۔
\item
چھڑی کے ایک سر پر چھڑی کے محور کو عمودی خط۔ 
\end{enumerate}
\انتہا{سوال}
%================
\ابتدا{سوال}
مستوی \عددی{xy} میں مستقل کثافت \عددی{\delta}  کی چادر منحنیات \عددی{x=y^2} اور \عددی{x=2y-y^2} کے بیچ پائی جاتی ہے۔
\begin{enumerate}[a.]
\item
ایسا  \عددی{\delta} دریافت کریں کہ چادر کی  کمیت   سوال \حوالہ{سوال_بالکثرت_درکار_رقبہ} کے چادر کی کمیت کے برابر ہو۔
\item
جزو-ا میں حاصل \عددی{\delta} کی قیمت کا اس خطہ پر \عددی{\delta(x,y)=y+1}  کی اوسط قیمت کے ساتھ موازنہ کریں۔
\end{enumerate}
\انتہا{سوال}
%=========
\ابتدا{سوال}
دائرہ \عددی{x^2+(y-1)^2=1}  کی کثافت مستقل ہے۔ محوروں کے لحاظ سے اس کے جمودی معیار اثر تلاش کریں۔ 
\انتہا{سوال}
%================
\موٹا{مسئلہ متوازی محور}\\
مستوی \عددی{xy} میں  ایک خطہ  پر  کمیت \عددی{m} کی باریک چادر پائی جاتی ہے۔ اس کے مرکز کمیت سے خط \عددی{L_{c,m}}  گزرتا ہے۔ خط \عددی{L_{c,m}} کے متوازی   \عددی{h} اکائیاں دور خط \عددی{L} پایا جاتا ہے۔ مسئلہ متوازی محور کہتا ہے  کہ \عددی{L_{c,m}} اور \عددی{L} کے لحاظ سے  بالترتیب جمودی معیار اثر \عددی{I_{c,m}} اور \عددی{I_L} درج ذیل کلیہ کو مطمئن کریں گے۔
\begin{align}
I_L=I_{c,m}+mh^2
\end{align}
اس کلیہ کو استعمال کرتے ہوئے ایک جمودی معیار اثر سے دوسرا با آسانی دریافت کیا جا سکتا ہے۔

\ابتدا{سوال}\ترچھا{مسئلہ متوازی محور کا ثبوت}\\
(ا) دکھائیں کہ باریک چادر کے مرکز کمیت سے گزرتی خط کے لحاظ سے  چادر کا جمودی معیار اثر صفر ہو گا۔ (اشارہ: مرکز کمیت کو مبدا پر رکھیں اور  خط کو محور \عددی{y} پر رکھیں۔ کلیہ \عددی{\bar{x}=\tfrac{M_y}{M}} کیا دیگا؟) (ب)  جزو-ا کے نتیجہ سے مسئلہ متوازی محور  اخذ کریں۔(اشارہ:  خط \عددی{L_{c,m}} کو محور \عددی{y} اور \عددی{L} کو \عددی{x=h} پر رکھ کر \عددی{I_L} کے تکمل کو دو حصوں میں  لکھیں۔)
\انتہا{سوال}
%==================
\ابتدا{سوال}
(ا) مسئلہ متوازی محور استعمال کرتے ہوئے مثال \حوالہ{مثال_بالکثرت_تکون_رداس_دوار} کے نتائج استعمال کرتے ہوئے  اس مثال میں چادر کے مرکز کمیت سے گزرتی افقی اور انتصابی خطوط کے لحاظ سے چادر کی جمودی معیار اثر تلاش کریں۔ (ب) جزو-ا کے نتائج استعمال کرتے ہوئے خطوط \عددی{x=1} اور \عددی{y=2} کے لحاظ سے چادر کی جمودی معیار اثر دریافت کریں۔
\انتہا{سوال}
%==============
\موٹا{کلیہ پاپس}\\
جناب پاپس نے حصہ  \حوالہ{حصہ_استعمال_تکمل_بنیادی_نقش_دیگر_نمونی_استعمال} کا مسئلہ پاپس بیان کیا۔ اس کے علاوہ وہ جانتے تھے کہ  ایک دوسرے کو نہ ڈھانپتے ہوئے دو  مستوی خطوں  کا وسطانی مرکز ان خطوں کے وسطانی مراکز سے گزرتے ہوئے خط پر پایا جاتا ہے۔مستوی \عددی{xy} میں ایک دوسرے کو نہ ڈھانپتی ہوئی   دو باریک چادر  \عددی{P_1} اور \عددی{P_2} فرض کریں،  جن کی کمیت بالترتیب \عددی{m_1} اور \عددی{m_2} ہو۔مبدا سے بالترتیب  ان چادروں کے مراکز کمیت تک سمتیات    \عددی{\kvec{c}_1} اور \عددی{\kvec{c}_2} لیں۔اب اشتراک \عددی{P_1\cup P_2} کا مرکز کمیت درج ذیل سمتیہ دیگا۔
\begin{align}\label{مساوات_بالکثرت_کلیہ_پاپس}
\kvec{c}=\frac{m_1\kvec{c}_1+m_2\kvec{c}_2}{m_1+m_2}
\end{align}
مساوات \حوالہ{مساوات_بالکثرت_کلیہ_پاپس} کو \اصطلاح{کلیہ پاپس}\فرہنگ{کلیہ!پاپس}\حاشیہب{Pappus's formula}\فرہنگ{Pappus!formula} کہتے ہیں۔ایک دوسرے کو نہ ڈھانپتی ہوئی دو سے زیادہ (لیکن متناہی تعداد کی)  چادروں کے لئے درج ذیل کلیہ ہو گا۔
\begin{align}\label{مساوات_بالکثرت_عمومی_کلیہ_پاپس}
\kvec{c}=\frac{m_1\kvec{c}_1+m_2\kvec{c}_2+\cdots+m_n\kvec{c}_n}{m_1+m_2+\cdots+m_n}
\end{align}
یہ کلیہ بالخصوص وہاں فائدہ مند ہو گا جہاں   غیر منظم  شکل و صورت کی چادر کے حصوں کے وسطانی مراکز ہم  جیومیٹری سے علیحدہ علیحدہ طور پر جانتے ہوں اور جہاں  ہر حصہ از خود  مستقل کثافت کا ہو۔ ہم اس کلیہ کو استعمال کرتے ہوئے پوری چادر کا وسطانی مرکز معلوم کر سکتے ہیں۔ 
