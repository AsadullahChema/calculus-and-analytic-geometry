\حصہ{جزوی تفرقات}
جب  ماسوائے ایک غیر تابع متغیر کے ہم باقی تمام  کو برقرار رکھیں اور اس ایک متغیر کے لحاظ سے تفاعل کا تفرق لیں تو ہمیں "جزوی" تفرق حاصل ہوتا ہے۔ اس حصہ میں دکھایا جائے گا کہ جزوی تفرقات کیسے  پائے جاتے ہیں اور واحد متغیر کے تفاعل کے تفرق کے قواعد بروئے کار لاتے ہوئے   جزوی تفرقات   کی قیمت کے حصول کے بارے میں بتایا جائے گا۔

\جزوحصہء{تعریفات اور علامتیت }
اگر تفاعل \عددی{f(x,y)} کے دائرہ کار میں \عددی{(x_0,y_0)} ایک نقطہ ہو تب انتصابی سطح \عددی{y=y_0}  سطح \عددی{z=f(x,y)} کو منحنی \عددی{z=f(x,y_0)} میں مس کرے گا۔ یہ منحنی مستوی \عددی{y=y_0} میں   تفاعل \عددی{z=f(x,y_0)} کی ترسیم ہو گی۔اس مستوی میں  افقی محدد \عددی{x}  ہے؛   انتصابی محدد \عددی{z} ہے۔

ہم نقطہ \عددی{(x_0,y_0)} پر \عددی{x} کے لحاظ سے \عددی{f} کے جزوی تفرق سے مراد  نقطہ \عددی{x=x_0} پر  \عددی{f(x,y_0)} کا سادہ تفرق لیتے ہیں۔

\ابتدا{تعریف}
نقطہ \عددی{(x_0,y_0)} پر\اصطلاح{ \عددی{x} کے لحاظ سے \عددی{f(x,y)} کا جزوی تفرق}\فرہنگ{تفرق!جزوی}\حاشیہب{partial derivative}\فرہنگ{derivative!partial}
\begin{align}\label{مساوات_کثیرالمتغیر_جزوی_تفرق_الف}
\left.\frac{\partial f}{\partial x}\right\vert_{(x_0,y_0)}=\left.\frac{\dif}{\dif x}f(x,y_0)\right\vert_{x=x_0}=\lim_{h\to 0}\frac{f(x_0+h,y_0)-f(x_0,y_0)}{h}
\end{align}
ہو گا بشرطیکہ  یہ حد موجود ہو۔ (آپ \عددی{\partial} کو \عددی{\dif} کی ایک قسم تصور کریں۔)
\انتہا{تعریف}
%================

نقطہ \عددی{N(x_0,y_0,f(x_0,y_0))} پر مستوی \عددی{y=y_0} میں منحنی \عددی{z=f(x,y_0)} کی ڈھلوان  سے مراد نقطہ \عددی{(x_0,y_0)} پر \عددی{x} کے لحاظ سے \عددی{f} کے جزوی تفرق کی قیمت ہے۔  نقطہ \عددی{N} پر منحنی کا مماسی خط، مستوی \عددی{y=y_0} میں وہ خط ہے جو \عددی{N} سے گزرتا ہو اور جس کی ڈھلوان یہی ہو۔ جب \عددی{y} کی قیمت برقرار \عددی{y_0} رکھی جائے  تب \عددی{x} کے لحاظ سے \عددی{f} کی شرح تبدیلی نقطہ \عددی{(x_0,y_0)} پر جزوی تفرق \عددی{\tfrac{\partial f}{\partial x}}    دیتا ہے۔ یہ نقطہ \عددی{(x_0,y_0)} پر \عددی{\ai} کے رخ \عددی{f} کی شرح تبدیلی  ہے۔

جزوی تفرق کی علامت اس چیز  پر منحصر ہو گی جس پر ہم زور دینا چاہتے ہیں۔یوں درج ذیل علامت اس وقت استعمال کیے جائیں گے  جب ہم نقطہ \عددی{(x_0,y_0)} پر زور دینا چاہیں۔
\begin{align*}
\frac{\partial f}{\partial x}(x_0,y_0),\quad f_x(x_0,y_0)
\end{align*}
سائنس اور انجینئری میں درج ذیل علامت مقبول ہے جہاں تفاعل کا صریحاً  ذکر  کیے  بغیر نقطہ \عددی{(x_0,y_0)} پر \عددی{x} کے لحاظ سے  \عددی{z} کا  جزوی تفرق لیا  گیا ہے۔
\begin{align*}
\left.\frac{\partial z}{\partial x}\right\vert_{(x_0,y_0)}
\end{align*}
جہاں جزوی تفرق  کو ایک تفاعل تصور کرنا مقصود  ہو وہاں درج ذیل علامت استعمال کیے جائیں گے، جہاں  \عددی{x} لے لحاظ سے \عددی{f} (یا \عددی{z}) کے جزوی تفرقات لیے گئے ہیں۔
\begin{align*}
f_x,\quad \frac{\partial f}{\partial x},\quad z_x,\quad \frac{\partial z}{\partial x}
\end{align*}

نقطہ \عددی{(x_0,y_0)} پر \عددی{y} کے لحاظ سے \عددی{f(x,y)} کے جزوی تفرق کی تعریف ، \عددی{x} کے لحاظ سے \عددی{f} کی جزوی تفرق کی تعریف کی طرح ہے۔ہم \عددی{x} کو \عددی{x_0} رکھتے ہوئے \عددی{y_0} پر \عددی{y} کے لحاظ سے \عددی{f(x_0,y)} کا سادہ تفرق لیتے ہیں۔

\ابتدا{تعریف}
نقطہ \عددی{(x_0,y_0)} پر\اصطلاح{ \عددی{y} کے لحاظ سے \عددی{f(x,y)} کا جزوی تفرق}\فرہنگ{تفرق!جزوی}\حاشیہب{partial derivative}\فرہنگ{derivative!partial}
\begin{align}\label{مساوات_کثیرالمتغیر_جزوی_تفرق_ب}
\left.\frac{\partial f}{\partial y}\right\vert_{(x_0,y_0)}=\left.\frac{\dif}{\dif y}f(x_0,y)\right\vert_{y=y_0}=\lim_{h\to 0}\frac{f(x_0,y_0+h)-f(x_0,y_0)}{h}
\end{align}
ہو گا بشرطیکہ  یہ حد موجود ہو۔ 
\انتہا{تعریف}
%=================


نقطہ \عددی{N(x_0,y_0,f(x_0,y_0))} پر مستوی \عددی{x=x_0} میں منحنی \عددی{z=f(x_0,y)} کی ڈھلوان  سے مراد نقطہ \عددی{(x_0,y_0)} پر \عددی{y} کے لحاظ سے \عددی{f} کے جزوی تفرق کی قیمت ہے۔  نقطہ \عددی{N} پر منحنی کا مماسی خط، مستوی \عددی{x=x_0} میں وہ خط ہے جو \عددی{N} سے گزرتا ہو اور جس کی ڈھلوان یہی ہو۔ جب \عددی{x} کی قیمت برقرار \عددی{x_0} رکھی جائے  تب \عددی{y} کے لحاظ سے \عددی{f} کی شرح تبدیلی نقطہ \عددی{(x_0,y_0)} پر جزوی تفرق \عددی{\tfrac{\partial f}{\partial y}}    دیتا ہے۔ یہ نقطہ \عددی{(x_0,y_0)} پر \عددی{\aj} کے رخ \عددی{f} کی شرح تبدیلی  ہے۔

متغیر \عددی{y} کے لحاظ سے جزوی تفرق کو \عددی{x} کے لحاظ سے جزوی تفرق کی طرح لکھا جاتا ہے:
\begin{align*}
\frac{\partial f}{\partial y}(x_0,y_0),\quad f_y(x_0,y_0),\quad \frac{\partial f}{\partial y},\quad f_y
\end{align*}

دھیان رہے کہ نقطہ \عددی{(x_0,y_0)} پر اب سطح \عددی{z=f(x,y)} کے ساتھ دو مماسی خط منسلک ہیں۔کیا ان مماسی سطح کا تعین کردہ سطح نقطہ \عددی{N} پر \عددی{z=f(x,y)} کو مماسی ہو گا؟ جزوی تفرق کے بارے میں مزید معلومات جاننے کے بعد ہم اس سوال کا جواب دے پائیں گے۔

\جزوحصہء{حساب}
جیسا ہم مساوات \حوالہ{مساوات_کثیرالمتغیر_جزوی_تفرق_الف} سے جانتے ہیں،   \عددی{y} کو مستقل تصور کرتے ہوئے \عددی{x} کے لحاظ سے \عددی{f} کا سادہ تفرق  ہمیں  \عددی{\tfrac{\partial f}{\partial x}} دیگا۔اسی طرح مساوات \حوالہ{مساوات_کثیرالمتغیر_جزوی_تفرق_ب} کہتی ہے کہ \عددی{x} کو مستقل رکھتے ہوئے \عددی{y} کے لحاظ سے \عددی{f} کا سادہ تفرق ہمیں \عددی{\tfrac{\partial f}{\partial y}} دیگا۔ 

\ابتدا{مثال}
نقطہ \عددی{(4,-5)} پر درج ذیل کے لئے \عددی{\tfrac{\partial f}{\partial x}} اور \عددی{\tfrac{\partial f}{\partial y}} کی قیمتیں تلاش کریں۔
\begin{align*}
f(x,y)=x^2+3xy+y-1
\end{align*}
حل:\quad
ہم \عددی{y} کو مستقل تصور کرتے ہوئے \عددی{x} کے لحاظ سے \عددی{f} کا تفرق لے کر   \عددی{\tfrac{\partial f}{\partial x}} حاصل کرتے ہیں۔
\begin{align*}
\frac{\partial f}{\partial x}=\frac{\partial}{\partial x}(x^2+3xy+y-1)=2x+3\cdot 1\cdot y+0-0=2x+3y
\end{align*}
نقطہ \عددی{(4,-5)} پر \عددی{\tfrac{\partial f}{\partial x}} کی قیمت \عددی{2(4)+3(-5)=-7} ہو گی۔

اسی طرح ہم \عددی{x} کو مستقل تصور کرتے ہوئے \عددی{y} کے لحاظ سے \عددی{f} کا تفرق لے کر   \عددی{\tfrac{\partial f}{\partial y}} حاصل کرتے ہیں۔
\begin{align*}
\frac{\partial f}{\partial y}=\frac{\partial}{\partial x}(x^2+3xy+y-1)=0+3\cdot x\cdot 1+1-0=3x+1
\end{align*}
نقطہ \عددی{(4,-5)} پر \عددی{\tfrac{\partial f}{\partial y}} کی قیمت \عددی{3(4)+1=13} ہو گی۔
\انتہا{مثال}
%=============
\ابتدا{مثال}
تفاعل \عددی{f(x,y)=y\sin xy} کا \عددی{\tfrac{\partial f}{\partial y}} معلوم کریں۔

حل:\quad
ہم \عددی{x} کو مستقل  تصور جبکہ  \عددی{f} کو \عددی{y} اور \عددی{\sin xy} کا حاصل ضرب تصور کرتے ہیں:
\begin{align*}
\frac{\partial f}{\partial y}&=\frac{\partial}{\partial y}(y\sin xy)=y\frac{\partial}{\partial y}\sin xy+(\sin xy)\frac{\partial}{\partial y}(y)\\
&=(y\cos xy)\frac{\partial}{\partial y}(xy) +\sin xy=xy\cos xy+\sin xy
\end{align*}
\انتہا{مثال}
%==================

\جزوحصہء{فنیات}
\ترچھا{جزوی تفرق}\quad کمپیوٹر آپ کو حساب میں کئی بعد  تک مدد فراہم کر سکتا ہے۔ آپ ایک غیر تابع متغیر کے علاوہ تمام متغیرات کی قیمتیں فراہم کر کے واحد  متغیر کے لحاظ سے جزوی تفرق معلوم کر  کے  ترسیم کر سکتے ہیں۔ جزوی تفرق اور سادہ تفرق کے لئے کمپیوٹر کی زبان میں عموماً  ایک جیسی اصطلاح استعمال کی جاتی ہے۔  جزوی تفرقات کے حصول میں کمپیوٹر ضرور استعمال کریں۔

\ابتدا{مثال}
تفاعل \عددی{f(x,y)=\tfrac{2y}{y+\cos x}} کے لئے \عددی{f_x} تلاش کریں۔

حل:\quad
ہم \عددی{f} کو حاصل تقسیم تصور کر کے \عددی{y} کو مستقل رکھ کر درج ذیل حاصل کرتے ہیں۔
\begin{align*}
f_x&=\frac{\partial}{\partial x}\big(\frac{2y}{y+\cos x}\big)=\frac{(y+\cos x)\tfrac{\partial}{\partial x}(2y)-2y\tfrac{\partial}{\partial x}(y+\cos x)}{(y+\cos x)^2}\\
&=\frac{(y+\cos x)(0)-2y(-\sin x)}{(y+\cos x)^2}=\frac{2y\sin x}{(y+\cos x)^2}
\end{align*}
\انتہا{مثال}
%=================
\ابتدا{مثال}
مستوی \عددی{x=1} قطع مکافی سطح  \عددی{z=x^2+y^2} کو قطع مکافی میں قطع کرتا ہے۔نقطہ \عددی{(1,2,5)} پر اس قطع مکافی کے مماس کی ڈھلوان تلاش کریں۔

حل:\quad
مماس کی ڈھلوان نقطہ \عددی{(1,2)} پر جزوی تفرق \عددی{\tfrac{\partial z}{\partial y}} کی قیمت ہو گی:
\begin{align*}
\left.\frac{\partial z}{\partial y}\right\vert_{(1,2)}=\left.\frac{\partial}{\partial y}(x^2+y^2)\right\vert_{(1,2)}=\left.2y\right\vert_{(1,2)}=2(2)=4
\end{align*}
تصدیق کی خاطر ہم قطع مکافی کو  واحد متغیر تفاعل \عددی{z=(1)^2+y^2=1+y^2} کی مستوی \عددی{x=1} میں ترسیم تصور کر کے \عددی{y=2} پر اس کی ڈھلوان حاصل کرتے ہیں۔ یہ ڈھلوان جس کو سادہ تفرق سے حاصل کیا جاتا ہے درج ذیل ہو گا۔
\begin{align*}
\left.\frac{\dif z}{\dif y}\right\vert_{y=2}=\left.\frac{\dif}{\dif y}(1+y^2)\right\vert_{y=2}=\left.2y\right\vert_{y=2}=4
\end{align*}
\انتہا{مثال}
%==============

سادہ تفرق کی طرح جزوی تفرق کے لئے بھی  خفی تفرق کار آمد ہے۔

\ابتدا{مثال}
اگر درج ذیل مساوات  دو غیر تابع متغیرات \عددی{x} اور \عددی{y} کا تفاعل  \عددی{z} دیتی ہو  جس کا  جزوی تفرق موجود ہو تب \عددی{\tfrac{\partial z}{\partial x}} تلاش کریں۔
\begin{align*}
yz-\ln z=x+y
\end{align*}
حل:\quad
ہم \عددی{y} کو مستقل اور \عددی{z} کو \عددی{x} کا تفاعل تصور کرتے ہوئے  مساوات کے دونوں اطراف کا \عددی{x} کے لحاظ سے تفرق لیتے ہیں:
\begin{align*}
\frac{\partial}{\partial x}(yz)-\frac{\partial}{\partial x}\ln z&=\frac{\partial x}{\partial x}+\frac{\partial y}{\partial x}\\
y\frac{\partial z}{\partial x}-\frac{1}{z}\frac{\partial z}{\partial x}&=1+0&&\text{\RL{\عددی{y} مستقل}}\\
\big(y-\frac{1}{z}\big)\frac{\partial z}{\partial x}&=1&&\tfrac{\partial}{\partial x}(yz)=y\tfrac{\partial z}{\partial x}\\
\frac{\partial z}{\partial x}&=\frac{z}{yz-1}
\end{align*}
\انتہا{مثال}
%===================
\جزوحصہء{دو سے زیادہ متغیرات کے تفاعل}
