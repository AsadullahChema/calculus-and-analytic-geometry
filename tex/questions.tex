\حصہ{فلکی سیاروں اور  مصنوعی سیاروں   کی حرکت}
اس حصہ میں ہم    قوانین نیوٹن اور قوت کشش  کی مدد سے سیاروں کی حرکت کے قوانین کپلر   اخذ کریں گے اور زمین کے گرد   مصنوعی سیاروں  کے مدار پر بحث کریں گے۔قوانین نیوٹن سے قوانین کپلر کا حصول احصاء  کی  اہم کامیابی  ہے۔اس میں وہ سب کچھ درکار ہو گا جو ہم نے اب تک پڑھا ہے جیسا فضا میں سمتیات کا الجبرا اور جیومیٹری، سمتی تفاعل  کا احصاء، تفرقی مساوات کے حل،  ابتدائی قیمت مسائل اور ترخیمی حصوں کی قطبی محددی  تشریح۔

\جزوحصہء{قطبی اور نلکی محدد میں حرکت کی سمتی مساواتیں}
ہم  یہاں  قطبی محدد  کو   \عددی{r}، \عددی{\theta} اور  نلکی محدد کو \عددی{r}، \عددی{\theta}، \عددی{z} لکھیں گے۔  ایک ذرہ قطبی محددی مستوی میں حرکت کرتا ہو، ہم اس کے مقام، سمتی رفتار اور اسراع کو متحرک اکائی سمتیات
\begin{align}\label{مساوات_سمتی_تفاعل_قطبی_روپ_الف}
\kvec{u}_r=(\cos\theta)\ai+(\sin\theta)\aj,\quad \kvec{u}_{\theta}=-(\sin\theta)\ai+(\cos\theta)\aj
\end{align}
کی روپ میں لکھتے ہیں (شکل \حوالہ{شکل_سمتی_تفاعل_قطبی_محدد_اکائی_سمتیات})۔ اکائی سمتیہ \عددی{\kvec{u}_r} کا رخ سمتیہ \عددی{\krightharpoonup{OP}} کے رخ ہے لہٰذا \عددی{\kvec{r}=r\kvec{u}_r} ہو گا۔ اکائی سمتیہ \عددی{\kvec{u}_{\theta}} بڑھتے \عددی{\theta}  کے رخ یعنی   سمتیہ \عددی{\kvec{u}_r} کو  عمودی ہے۔

\begin{figure}
\centering
\begin{minipage}{0.45\textwidth}
\centering
\begin{tikzpicture}[font=\small,declare function={f(\x)=(\x-0.5)^2-0.5;}]
\pgfmathsetmacro{\ang}{25}
\pgfmathsetmacro{\len}{2.5}
\draw[-latex](0,0)node[left]{$O$}--(3,0)node[right]{$x$};
\draw[-latex](0,0)--(0,2)node[left]{$y$};
\draw[name path=f,domain=0.6:2] plot (\x,{f(\x)});
\path[name path=r](0,0)--(\ang:\len);%coordinate(ka)node[pos=0.7,above]{$\kvec{r}$};
\draw[-latex,name intersections={of={f and r}}](0,0)--(intersection-1)node[below right]{$P(r,\theta)$}node[pos=0.5,above]{$\kvec{r}$};
\draw[-latex](intersection-1)--++(\ang:1)node[right]{$\kvec{u}_r$};
\draw[-latex](intersection-1)--++(\ang+90:1)node[left,pos=0.7]{$\kvec{u}_{\theta}$};
\draw[-stealth]([shift={(0:0.6)}]0,0) arc (0:\ang:0.6)node[pos=0.6,right]{$\theta$};
\end{tikzpicture}
\caption{نقطہ \عددی{P} کا قطبی محدد \عددی{r} سمتیہ \عددی{\kvec{r}} کی مقدار ہو گی۔یوں \عددی{\kvec{u}_r}، جو \عددی{\tfrac{\kvec{r}}{\abs{\kvec{r}}}} ہے  \عددی{\tfrac{\kvec{r}}{r}} ہو گا۔}
\label{شکل_سمتی_تفاعل_قطبی_محدد_اکائی_سمتیات}
\end{minipage}\hfill
\begin{minipage}{0.45\textwidth}
\centering
\begin{tikzpicture}[font=\small]
\pgfmathsetmacro{\ang}{25}
\pgfmathsetmacro{\len}{2.5}
\draw[-latex](0,0)node[left]{$O$}--(3,0)node[right]{$x$};
\draw[-latex](0,0)--(0,2)node[left]{$y$};
\draw[name path=f](0,-0.6) to [out=10,in=-80](2,2);
\path[name path=r](0,0)--(\ang:\len);%coordinate(ka)node[pos=0.7,above]{$\kvec{r}$};
\draw[-latex,name intersections={of={f and r}}](0,0)--(intersection-1)node[below right]{$P(r,\theta)$}node[pos=0.5,above]{$\kvec{r}$};
\draw[-latex](intersection-1)--++(\ang:1)coordinate(ka)node[below,pos=0.85]{$\dot{r}\kvec{u}_r$};
\draw[-latex](ka)--++(\ang+90:1)coordinate(kb)node[right,pos=0.7]{$r\dot{\theta}\kvec{u}_{\theta}$};
\draw[-latex](intersection-1)--(kb)node[pos=0.5,right]{$\kvec{v}$};
\draw[-stealth]([shift={(0:0.6)}]0,0) arc (0:\ang:0.6)node[pos=0.6,right]{$\theta$};
\end{tikzpicture}
\caption{قطبی محدد میں سمتی رفتار سمتیہ \عددی{\kvec{v}=\dot{r}\kvec{u}_r+r\dot{\theta}\kvec{u}_{\theta}} ہو گا۔}
\label{شکل_سمتی_تفاعل_سمتی_رفتار_سمتیہ}
\end{minipage}
\end{figure}

مساوات \حوالہ{مساوات_سمتی_تفاعل_قطبی_روپ_الف} سے ہمیں   درج ذیل ملتے ہیں۔
\begin{gather}
\begin{aligned}\label{مساوات_سمتی_تفاعل_قطبی_روپ_ب}
\frac{\dif\kvec{u}_r}{\dif \theta}&=-(\sin\theta)\ai+(\cos\theta)\aj=\kvec{u}_{\theta}\\
\frac{\dif\kvec{u}_{\theta}}{\dif\theta}&=-(\cos\theta)\ai-(\sin\theta)\aj=-\kvec{u}_r
\end{aligned}
\end{gather}

ہم  وقت کے لحاض سے \عددی{\kvec{u}_r} اور \عددی{\kvec{u}_{\theta}} کی تبدیلی دیکھنے کی خاطر ان  کا  تفرق \عددی{t} کے لحاض سے زنجیری قاعدہ  سے حاصل کرتے   ہیں۔
\begin{gather}
\begin{aligned}\label{مساوات_سمتی_تفاعل_قطبی_روپ_پ}
\dot{\kvec{u}}_r=\frac{\dif\kvec{u}_r}{\dif\theta}\dot{\theta}=\dot{\theta}\kvec{u}_{\theta},\quad\dot{\kvec{u}}_{\theta}=\frac{\dif\kvec{u}_{\theta}}{\dif\theta}\dot{\theta}=-\dot{\theta}\kvec{u}_r
\end{aligned}
\end{gather}
یوں سمتی رفتار (شکل \حوالہ{شکل_سمتی_تفاعل_سمتی_رفتار_سمتیہ})
\begin{align}\label{مساوات_سمتی_تفاعل_قطبی_روپ_ت}
\kvec{v}=\dot{\kvec{r}}=\frac{\dif}{\dif t}(r\kvec{u}_r)=\dot{r}\kvec{u}_r+r\dot{\kvec{u}}_r=\dot{r}\kvec{u}_r+r\dot{\theta}\kvec{u}_{\theta}
\end{align}
اور اسراع  درج  ذیل ہو گا۔
\begin{align}\label{مساوات_سمتی_تفاعل_قطبی_روپ_ٹ}
\kvec{a}=\dot{\kvec{v}}=(\ddot{r}\kvec{u}_r+\dot{r}\dot{\kvec{u}}_r)+(\dot{r}\dot{\theta}\kvec{u}_{\theta}+r\ddot{\theta}\kvec{u}_{\theta}+r\dot{\theta}\dot{\kvec{u}}_{\theta})
\end{align}

جب \عددی{\dot{\kvec{u}}_r} اور \عددی{\dot{\kvec{u}}_{\theta}} کے حصول کے لئے   مساوات \حوالہ{مساوات_سمتی_تفاعل_قطبی_روپ_پ} استعمال کیا جائے اور اجزاء کو علیحدہ کیے جائیں  تب اسراع کی مساوات   درج ذیل  صورت اختیار کرتی ہے۔
\begin{align}
\kvec{a}=(\ddot{r}-r\dot{\theta}^2)\kvec{u}_r+(r\ddot{\theta}+2\dot{r}\dot{\theta})\kvec{u}_{\theta}
\end{align} 

 ہم مساوات \عددی{\kvec{r}=r\kvec{u}_r} کے دائیں ہاتھ جزو  \عددی{z\ak}  جمع کر کے ان مساواتوں کو وسعت دے کر  فضا میں حرکت کے لئے  قابل استعمال بنا سکتے ہیں۔یوں نلکی محدد میں درج ذیل ہوں گے۔
\begin{gather}
\begin{aligned}
\kvec{r}&=r\kvec{u}_r+z\ak\\
\kvec{v}&=\dot{r}\kvec{u}_r+r\dot{\theta}\kvec{u}_{\theta}+\dot{z}\ak\\
\kvec{a}&=(\ddot{r}-r\dot{\theta}^2)\kvec{u}_r+(r\ddot{\theta}+2\dot{r}\dot{\theta})\kvec{u}_{\theta}+\ddot{z}\ak
\end{aligned}
\end{gather}
دھیان رہے کہ \عددی{z\ne 0} کی صورت میں \عددی{\abs{\kvec{r}}=r} ہو گا۔

سمتیات \عددی{\kvec{u}_r}، \عددی{\kvec{u}_{\theta}} اور \عددی{\ak} دایاں ہاتھ چھوکٹ دیتے ہیں جس میں درج ذیل ہوں گے۔
\begin{align}
\kvec{u}_r\times\kvec{u}_{\theta}=\ak,\quad \kvec{u}_{\theta}\times\ak=\kvec{u}_r,\quad \ak\times\kvec{u}_r=\kvec{u}_{\theta}
\end{align}
