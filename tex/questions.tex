\جزوحصہء{سوالات}
\موٹا{تکمل کے  خطہ کی تلاش اور دوہرا تکملات}\\
سوال \حوالہ{سوال_بالکثرت_خطہ_خاکہ_الف} تا سوال \حوالہ{سوال_بالکثرت_خطہ_خاکہ_ب} میں تکمل کے  خطے کا خاکہ بنائیں اور تکمل کی قیمت تلاش کریں۔ 

\ابتدا{سوال}\شناخت{سوال_بالکثرت_خطہ_خاکہ_الف}
$\int_0^3\int_0^2(4-y^2)\dif y\dif x$
\انتہا{سوال}
%=================
\ابتدا{سوال}
$\int_0^3\int_{-2}^0 (x^2y-2xy)\dif y\dif x$
\انتہا{سوال}
%=================
\ابتدا{سوال}
$\int_{-1}^{0}\int_{-1}^1(x+y+1)\dif x\dif y$
\انتہا{سوال}
%=================
\ابتدا{سوال}
$\int_{\pi}^{2\pi}\int_0^{\pi}(\sin x+\cos y)\dif x\dif y$
\انتہا{سوال}
%=================
\ابتدا{سوال}
$\int_0^{\pi}\int_0^x x\sin y\dif y\dif x$
\انتہا{سوال}
%=================
\ابتدا{سوال}
$\int_0^{\pi}\int_0^{\sin x}y\dif y\dif x$
\انتہا{سوال}
%=================
\ابتدا{سوال}
$\int_1^{\ln 8}\int_0^{\ln y}e^{x+y}\dif x\dif y$
\انتہا{سوال}
%=================
\ابتدا{سوال}
$\int_1^2\int_y^{y^2}\dif x\dif y$
\انتہا{سوال}
%=================
\ابتدا{سوال}
$\int_0^1\int_0^{y^2}3y^3e^{xy}\dif x\dif y$
\انتہا{سوال}
%=================
\ابتدا{سوال}\شناخت{سوال_بالکثرت_خطہ_خاکہ_ب}
$\int_1^4\int_0^{\sqrt{x}}\frac{3}{2}e^{y/\sqrt{x}}\dif y\dif x$
\انتہا{سوال}
%=================

سوال \حوالہ{سوال_بالکثرت_دیا_خطہ_تکمل_الف} تا سوال \حوالہ{سوال_بالکثرت_دیا_خطہ_تکمل_ب} میں \عددی{f} کو دیے ہوئے خطہ پر تکمل کریں۔\\
\ابتدا{سوال}\شناخت{سوال_بالکثرت_دیا_خطہ_تکمل_الف}
ربع اول میں لکیر \عددی{y=x}، \عددی{y=2x}، \عددی{x=1} اور \عددی{x=2}  کے بیچ خطہ پر تفاعل \عددی{f(x,y)=\tfrac{x}{y}} کا تکمل۔
\انتہا{سوال}
%=================
\ابتدا{سوال}
چکور \عددی{1\le x\le 2,\, 1\le y\le 2} پر تفاعل \عددی{f(x,y)=\tfrac{1}{xy}}  کا تکمل۔
\انتہا{سوال}
%==================
\ابتدا{سوال}
مثلث خطہ جس کے راس \عددی{(0,0)}، \عددی{(1,0)} اور \عددی{(0,1)} ہیں میں تفاعل \عددی{f(x,y)=x^2+y^2} کا تکمل۔
\انتہا{سوال}
%=================
\ابتدا{سوال}
مستطیل \عددی{0\le x\le \pi,\, 0\le y\le 1} پر تفاعل \عددی{f(x,y)=y\cos xy} کا تکمل۔
\انتہا{سوال}
%==================
\ابتدا{سوال}
مستوی \عددی{uv} کے ربع اول میں  لکیر \عددی{u+v=1} کے نیچے تفاعل \عددی{f(u,v)=v-\sqrt{u}} کا تکمل۔ 
\انتہا{سوال}
%================
\ابتدا{سوال}\شناخت{سوال_بالکثرت_دیا_خطہ_تکمل_ب}
مستوی \عددی{st} کے ربع اول میں  منحنی \عددی{s=\ln t} کے اوپر جانب \عددی{t=1} سے \عددی{t=2} تک تفاعل \عددی{f(s,t)=e^s\ln t} کا تکمل۔
\انتہا{سوال}
%==============
سوال \حوالہ{سوال_بالکثرت_خاکہ_بنائیں_قیمت_حاصل_کریں_الف} تا سوال \حوالہ{سوال_بالکثرت_خاکہ_بنائیں_قیمت_حاصل_کریں_ب} میں تکملات دیے گئے ہیں۔ ان تکملات کے خطوں کا خاکہ  بنائیں اور تکمل کی قیمت حاصل کریں۔

\ابتدا{سوال}\شناخت{سوال_بالکثرت_خاکہ_بنائیں_قیمت_حاصل_کریں_الف}
$\int_{-2}^0\int_v^{-v}2\dif p\dif v$\quad
مستوی \عددی{pv}
\انتہا{سوال}
%=================
\ابتدا{سوال}
$\int_0^1\int_0^{\sqrt{1-s^2}}8t\dif t\dif s$\quad
مستوی \عددی{st}
\انتہا{سوال}
%====================
\ابتدا{سوال}
$\int_{-\pi/3}^{\pi/3}\int_0^{\sec t}3\cos t\dif u\dif t$\quad
مستوی \عددی{tu}
\انتہا{سوال}
%====================
\ابتدا{سوال}\شناخت{سوال_بالکثرت_خاکہ_بنائیں_قیمت_حاصل_کریں_ب}
$\int_0^3\int_{-2}^{4-2u}\frac{4-2u}{v^2}\dif v\dif u$\quad
مستوی \عددی{uv}
\انتہا{سوال}
%====================

\موٹا{تکمل کی الٹ ترتیب}\\
سوال \حوالہ{سوال_بالکثرت_الٹ_لکھیں_الف} تا سوال \حوالہ{سوال_بالکثرت_الٹ_لکھیں_ب} میں تکمل کے خطہ کا خاکہ بنا کر معادل الٹ ترتیب کا تکمل لکھیں۔

\ابتدا{سوال}\شناخت{سوال_بالکثرت_الٹ_لکھیں_الف}
$\int_0^1\int_2^{4-2x}\dif y\dif x$
\انتہا{سوال}
%===============
\ابتدا{سوال}
$\int_0^2\int_{y-2}^0\dif x\dif y$
\انتہا{سوال}
%================
\ابتدا{سوال}
$\int_0^1\int_y^{\sqrt{y}}\dif x\dif y$
\انتہا{سوال}
%================
\ابتدا{سوال}
$\int_0^1\int_{1-x}^{1-x^2}\dif y\dif x$
\انتہا{سوال}
%================
\ابتدا{سوال}
$\int_0^1\int_1^{e^x}\dif y\dif x$
\انتہا{سوال}
%================
\ابتدا{سوال}
$\int_0^{\ln 2}\int_{e^y}^2\dif x\dif y$
\انتہا{سوال}
%================
\ابتدا{سوال}
$\int_0^{3/2}\int_0^{9-4x^2}16x\dif y\dif x$
\انتہا{سوال}
%================
\ابتدا{سوال}
$\int_0^2\int_0^{4-y^2}y\dif x\dif y$
\انتہا{سوال}
%================
\ابتدا{سوال}
$\int_0^1\int_{-\sqrt{1-y^2}}^{\sqrt{1-y^2}}3y\dif x\dif y$
\انتہا{سوال}
%================
\ابتدا{سوال}\شناخت{سوال_بالکثرت_الٹ_لکھیں_ب}
$\int_0^2\int_{-\sqrt{4-x^2}}^{\sqrt{4-x^2}}6x\dif y\dif x$
\انتہا{سوال}
%================

\موٹا{دوہرا تکمل کی قیمت کا حصول}\\
سوال \حوالہ{سوال_بالکثرت_ترتیب_تعین_الف} تا سوال \حوالہ{سوال_بالکثرت_ترتیب_تعین_ب} میں  تکمل کے خطہ کا خاکہ بنا کر تکمل کی ترتیب تعین کرتے ہوئے  تکمل کی قیمت تلاش کریں۔

\ابتدا{سوال}\شناخت{سوال_بالکثرت_ترتیب_تعین_الف}
$\int_0^{\pi}\int_x^{\pi}\frac{\sin y}{y}\dif y\dif x$
\انتہا{سوال}
%==================
\ابتدا{سوال}
$\int_0^2\int_x^22y^2\sin xy\dif y\dif x$
\انتہا{سوال}
%================
\ابتدا{سوال}
$\int_0^1\int_y^1 x^2e^{xy}\dif x\dif y$
\انتہا{سوال}
%================
\ابتدا{سوال}
$\int_0^2\int_0^{4-x^2}\frac{xe^{2y}}{4-y}\dif y\dif x$
\انتہا{سوال}
%================
\ابتدا{سوال}
$\int_0^{2\sqrt{\ln 3}}\int_{y/2}^{\sqrt{\ln 3}}e^{x^2}\dif x\dif y$
\انتہا{سوال}
%================
\ابتدا{سوال}
$\int_0^3\int_{\sqrt{x/3}}^1e^{y^3}\dif y\dif x$
\انتہا{سوال}
%================
\ابتدا{سوال}
$\int_0^{1/{16}}\int_{y^{1/4}}^{1/2}\cos(16\pi x^5)\dif x\dif y$
\انتہا{سوال}
%================
\ابتدا{سوال}
$\int_0^8\int_{\sqrt[3]{x}}^2\frac{\dif y\dif x}{y^4+1}$
\انتہا{سوال}
%================
\ابتدا{سوال}
$\iint\limits_R (y-2x^2)\dif S$
جہاں \عددی{R} چکور \عددی{\abs{x}+\abs{y}=1} کا اندرونی خطہ  ہے۔
\انتہا{سوال}
%================
\ابتدا{سوال}\شناخت{سوال_بالکثرت_ترتیب_تعین_ب}
$\iint\limits_R xy\dif S$
جہاں لکیر \عددی{y=x}، \عددی{y=2x} اور \عددی{x+y=2} کے بیچ خطہ \عددی{R} ہے۔
\انتہا{سوال}
%================

\موٹا{سطح \عددی{z=f(x,y)} کے نیچے حجم}\\
\ابتدا{سوال}
مستوی \عددی{xy} میں لکیر \عددی{y=x}، \عددی{x=0} اور \عددی{x+y=2} کے بیچ مثلث کے اور  قطع مکافی سطح  \عددی{z=x^2+y^2} کے نیچے  خطہ کا حجم تلاش کریں۔
\انتہا{سوال}
%===========
\ابتدا{سوال}
ایک ٹھوس جسم  اوپر سے بیلن \عددی{z=x^2} اور نیچے سے مستوی \عددی{xy} میں لکیر \عددی{y=x} اور قطع مکافی \عددی{y=2-x^2}  کے بیچ مثلث   خطہ کے درمیان پایا جاتا ہے۔ اس جسم  کا حجم تلاش کریں۔
\انتہا{سوال}
%===============
\ابتدا{سوال}
ایک ٹھوس جسم  کا قاعدہ مستوی \عددی{xy} میں  لکیر \عددی{y=3x} اور قطع مکافی \عددی{y=4-x^2} کے بیچ خطہ ہے جبکہ اس کا بالائی سر مستوی \عددی{z=x+4} پر مشتمل ہے۔  اس جسم کا حجم تلاش کریں۔
\انتہا{سوال}
%=============
\ابتدا{سوال}
ثُمن  اول میں  محددی مستویات،  بیلن \عددی{x^2+y^2=4} اور مستوی \عددی{z+y=3} کے بیچ ٹھوس جسم کا حجم تلاش کریں۔ 
\انتہا{سوال}
%============
\ابتدا{سوال}
ثُمن اول میں  محددی مستویات، مستوی \عددی{x=3}  اور قطع مکافی بیلن \عددی{z=4-y^2} کے بیچ ٹھوس جسم کا حجم تلاش کریں۔
\انتہا{سوال}
%===========
\ابتدا{سوال}
ثُمن اول سے  سطح \عددی{z=4-x^2-y^2} ایک ٹھوس جسم کاٹتی ہے۔ اس جسم کا حجم تلاش کریں۔
\انتہا{سوال}
%=================
\ابتدا{سوال}
ثُمن اول سے بیلن \عددی{z=12-3y^2} اور مستوی \عددی{x+y=2}  ایک  پچر کاٹتے ہیں۔ اس پچر کا حجم تلاش کریں۔
\انتہا{سوال}
%===========
\ابتدا{سوال}
چکور ستون   \عددی{\abs{x}+\abs{y}\le 1} سے مستویات  \عددی{z=0} اور \عددی{3x+z=3}  جس ٹھوس جسم کو کاٹتے ہیں اس کا حجم تلاش کریں۔
\انتہا{سوال}
%==============
\ابتدا{سوال}
ایک ٹھوس جسم  سامنے اور پشت سے  مستویات \عددی{x=2} اور \عددی{x=1}،  اطراف سے بیلن \عددی{y=\mp\tfrac{1}{x}}، اوپر سے مستوی \عددی{z=x+1} اور نیچے سے مستوی \عددی{z=0} میں گھیرا ہوا ہے۔ اس جسم کا حجم تلاش کریں۔
\انتہا{سوال}
%=============
\ابتدا{سوال}
ایک جسم سامنے اور پشت سے مستویات \عددی{x=\pm \tfrac{\pi}{3}}، اطراف سے بیلن \عددی{y=\mp\sec x} ، اوپر سے بیلن \عددی{z=1+y^2} اور نیچے سے مستوی \عددی{xy} میں گھیرا ہوا ہے۔ اس جسم کا حجم تلاش کریں۔
\انتہا{سوال}
%==========
\موٹا{غیر محدود خطوں پر تکملات}\\
سوال \حوالہ{سوال_بالکثرت_غیر_مناسب_الف} تا سوال \حوالہ{سوال_بالکثرت_غیر_مناسب_ب} میں غیر مناسب تکملات کو بار بار تکمل تصور کرتے ہوئے ان  کی قیمت تلاش کریں۔

\ابتدا{سوال}\شناخت{سوال_بالکثرت_غیر_مناسب_الف}
$\int_1^{\infty}\int_{e^{-x}}^1\frac{1}{x^3y}\dif y\dif x$
\انتہا{سوال}
%==================
\ابتدا{سوال}
$\int_{-1}^1\int_{-1/\sqrt{1-x^2}}^{1/\sqrt{1-x^2}}(2y+1)\dif y\dif x$
\انتہا{سوال}
%=======================
\ابتدا{سوال}
$\int_{-\infty}^{\infty}\int_{-\infty}^{\infty}\frac{1}{(x^2+1)(y^2+1)}\dif x\dif y$
\انتہا{سوال}
%=======================
\ابتدا{سوال}\شناخت{سوال_بالکثرت_غیر_مناسب_ب}
$\int_0^{\infty}\int_0^{\infty}xe^{-(x+2y)}\dif x\dif y$
\انتہا{سوال}
%=======================

\موٹا{دوہرا تکملات کی تخمین}\\
سوال \حوالہ{سوال_بالکثرت_تخمینی_دوہرا_الف} اور سوال \حوالہ{سوال_بالکثرت_تخمینی_دوہرا_ب} میں تفاعل  \عددی{f(x,y)} کے  دوہرا تکمل   کے خطہ \عددی{R} کو انتصابی خط \عددی{x=a} اور افقی خط \عددی{y=c} خانہ بند کرتی ہیں۔ ہر  ذیلی مستطیل میں دکھائے گئے   \عددی{(x_k,y_k)} لیتے ہوئے  درج  ذیل تخمین استعمال کر کے دوہرا تکملات کی تخمینی قیمتیں تلاش کریں۔
\begin{align*}
\iint\limits_R f(x,y)\dif S\approx \sum\limits_{k=1}^{n}f(x_k,y_k)\Delta S_k
\end{align*}

\ابتدا{سوال}\شناخت{سوال_بالکثرت_تخمینی_دوہرا_الف}
تفاعل \عددی{f(x,y)=x+y}  اور خطہ \عددی{R}،   جو   نصف دائرہ \عددی{y=\sqrt{1-x^2}} اور    محور \عددی{x} کے بیچ ہے۔ خانہ بندی \عددی{x=-1,-1/2,0,1/4,1/2,1} اور \عددی{y=0,1/2,1} لیں۔ نقطہ \عددی{(x_k,y_k) } کو \عددی{k} واں  خانے  کا نچلا بایاں کونا لیں بشرطیکہ یہ مستطیل \عددی{R} کے اندر پایا جاتا ہو ۔
\انتہا{سوال}
%================
\ابتدا{سوال}\شناخت{سوال_بالکثرت_تخمینی_دوہرا_ب}
تفاعل \عددی{f(x,y)=x+2y} ہے جبکہ   اور دائرہ \عددی{(x-2)^2+(y-3)^2=1} کا اندرونی  خطہ \عددی{R} ہے۔ خانہ بندی \عددی{x=1,3/2,2,5/2,3} اور \عددی{y=2,5/2,3,7/2,4} لیں ۔بشرطیکہ \عددی{k} واں مستطیل \عددی{R} میں پایا جاتا ہو، \عددی{k} ویں مستطیل کے وسطانی مرکز کو   \عددی{(x_k,y_k)}  لیں۔
\انتہا{سوال}
%=======================
\موٹا{نظریہ اور مثالیں}\\
\ابتدا{سوال}
قرص \عددی{x^2+y^2\le 4}  کو  شعاع \عددی{\theta=\tfrac{\pi}{6}} اور \عددی{\theta=\tfrac{\pi}{2}}  دو ٹکڑوں میں تقسیم کرتے ہیں۔ ان میں سے چھوٹے ٹکڑے پر \عددی{f(x,y)=\sqrt{4=x^2}}  کا تکمل لیں۔ 
\انتہا{سوال}
%===========
\ابتدا{سوال}
لا متناہی مستطیل \عددی{2\le x\le \infty,\, 0\le y\le 2} پر \عددی{f(x,y)=\tfrac{1}{(x^2-x)(y-1)^{2/3}}} کا تکمل لیں۔ 
\انتہا{سوال}
%===================
\ابتدا{سوال}
ایک ٹھوس (غیر دائری) قائمہ بیلن کا قاعدہ \عددی{xy} مستوی ہے جبکہ اس کی بالائی سرحد قطع مکافی سطح  \عددی{z=x^2+y^2} ہے۔ اس بیلن کا حجم
\begin{align*}
H=\int_0^1\int_0^y (x^2+y^2)\dif x\dif y+\int_1^2\int_0^{2-y}(x^2+y^2)\dif x\dif y
\end{align*}
ہے۔ خطہ \عددی{R}  کا خاکہ بنائیں اور بیلن کے  حجم کو ،  تکمل کی ترتیب الٹ کرتے ہوئے ،  ایک بار بار تکمل کی صورت میں لکھ کر  حل کریں۔
\انتہا{سوال}
%======================
\ابتدا{سوال}
درج  ذیل کی قیمت تلاش کریں۔ (اشارہ: متکمل کو ایک تکمل کی صورت میں لکھیں۔)
\begin{align*}
\int_0^2 (\tan^{-1}\pi x-\tan^{-1}x)\dif x
\end{align*}
\انتہا{سوال}
%================
\ابتدا{سوال}
مستوی \عددی{xy} میں کونسا خطہ \عددی{R} درج ذیل تکمل کی قیمت کو زیادہ سے زیادہ بناتا ہے؟
\begin{align*}
\iint\limits_R (4-x^2-2y^2)\dif S
\end{align*}
اپنے جواب کی وجہ پیش کریں۔
\انتہا{سوال}
%=============
\ابتدا{سوال}
مستوی \عددی{xy} میں کونسا خطہ \عددی{R} درج ذیل تکمل کی قیمت کو کم سے کم  بناتا ہے؟
\begin{align*}
\iint\limits_R (x^2+y^2-9)\dif S
\end{align*}
اپنے جواب کی وجہ پیش کریں۔
\انتہا{سوال}
%=============
\ابتدا{سوال}
کیا استمراری تفاعل \عددی{f(x,y)} کا مستوی \عددی{xy} میں مستطیل خطہ پر  تکمل کی ترتیب بدلتے ہوئے مختلف نتائج کا حصول ٹھیک ہو گا؟ اپنے جواب کی وجہ بنائیں۔
\انتہا{سوال}
%================
\ابتدا{سوال}
ایک مثلث جس کے راس  \عددی{(0,1)}، \عددی{(2,0)} اور \عددی{(1,2)} ہوں پر استمراری تفاعل \عددی{f(x,y)} کے  دوہرا  تکمل  کی قیمت درکار ہے۔ آپ یہ قیمت کیسے حاصل کریں گے؟ اپنے جواب کی وجہ پیش کریں۔
\انتہا{سوال}
%================
\ابتدا{سوال}
درج ذیل تعلق کو ثابت کریں۔
\begin{align*}
\int_{-\infty}^{\infty}\int_{-\infty}^{\infty} e^{-x^2-y^2}\dif x\dif y=\lim_{b\to\infty}\int_{-b}^b\int_{-b}^b e^{-x^2-y^2}\dif x\dif y=4\left(\int_0^{\infty}e^{-x^2}\dif x\right)^2
\end{align*}
\انتہا{سوال}
%=============
\ابتدا{سوال}
درج ذیل غیر مناسب تکمل کی قیمت تلاش کریں۔
\begin{align*}
\int_0^1\int_0^3\frac{x^2}{(y-1)^{2/3}}\dif y\dif x
\end{align*}
\انتہا{سوال}
%===============
\موٹا{اعدادی تراکیب سے تکمل کی قیمت کی تلاش}\\
سوال \حوالہ{سوال_بالکثرت_اعدادی_تراکیب_الف} تا سوال \حوالہ{سوال_بالکثرت_اعدادی_تراکیب_ب} میں کمپیوٹر استعمال کرتے ہوئے اعدادی تراکیب سے دوہرا تکملات کی قیمتیں دریافت کریں۔

\ابتدا{سوال}\شناخت{سوال_بالکثرت_اعدادی_تراکیب_الف}
$\int_1^3\int_1^x\frac{1}{xy}\dif y\dif x$
\انتہا{سوال}
%===============
\ابتدا{سوال}
$\int_0^1\int_0^1e^{-x^2-y^2}\dif y\dif x$
\انتہا{سوال}
%==============
\ابتدا{سوال}
$\int_0^1\int_0^1\tan^{-1}xy\dif y\dif x$
\انتہا{سوال}
%==================
\ابتدا{سوال}\شناخت{سوال_بالکثرت_اعدادی_تراکیب_ب}
$\int_{-1}^1\int_0^{\sqrt{1-x^2}}3\sqrt{1-x^2-y^2}\dif y\dif x$
\انتہا{سوال}
%===============
