\حصہ{قاعدہ لھوپیٹال}
ایسا کسر جس کا نسب نما اور شمار کنندہ دونوں تحدیدی نقطہ پر  صفر کو پہنچتے ہوں، کا حد تلاش کرنے کا قاعدہ یعقوب برنولی نے دریافت کیا جس کو \اصطلاح{قاعدہ لھوپیٹال} کہتے ہیں۔

\جزوحصہء{وسطی حاصل تقسیم}
اگر نقطہ \عددی{x=a} پر تفاعل \عددی{f(x)} اور \عددی{g(x)} کی قیمتیں صفر ہوں تب \عددی{x=a} پر کرتے ہوئے  \عددی{\lim_{x\to a}\tfrac{f(x)}{g(x)}} کا حصول ممکن نہیں ہو گا چونکہ ایسا کرنے سے \عددی{\tfrac{0}{0}} ملتا ہے جو بے معنی ہے اور  جس کو \اصطلاح{وسطی روپ}\فرہنگ{وسطی روپ}\حاشیہب{intermediate form}\فرہنگ{intermediate form} کہتے ہیں۔ اب تک ہم دیکھ چکے ہیں کہ جو حد وسطی روپ دیں ان کا حد تلاش کرنا کبھی آسان اور کبھی مشکل ہوتا ہے۔ہم نے حصہ \حوالہ{حصہ_تفرق_تکونیاتی_تفاعل_کا_تفرق} میں کافی محنت کے بعد \عددی{\lim_{x\to 0}\tfrac{\sin x}{x}} کی قیمت حاصل کی۔ اس کے برعکس تفرق کے حصول میں استعمال ہونے والا درج ذیل حد تلاش کرنے میں ہمیں کوئی دشواری پیش نہیں ہوئی،
\begin{align*}
f'(a)=\lim_{x\to a}\frac{f(x)-f(a)}{x-a}
\end{align*}
اگرچہ اس حد میں \عددی{x=a} پر کرنے سے ہر صورت \عددی{\tfrac{0}{0}} حاصل ہوتا ہے۔ قاعدہ لھوپیٹال  کی مدد سے ہم تفرق کے حصول میں حد کے استعمال سے استفادہ کرتے ہوئے ان حد کو تلاش کرتے ہیں جو وسطی روپ کو جنم دیتے ہیں۔

\ابتدا{مسئلہ}\موٹا{قاعدہ لھوپیٹال (پہلی صورت)}\\
فرض کریں کہ \عددی{f(a)=g(a)=0}  ہے جبکہ  \عددی{f'(a)} اور \عددی{g'(a)} موجود ہیں جہاں \عددی{g'(a)\ne 0} ہے۔ تب درج ذیل ہو گا۔
\begin{align}\label{مساوات_ماورائی_قاعدہ_لھوپیٹال_اول}
\lim_{x\to a}\frac{f(x)}{g(x)}=\frac{f'(a)}{g'(a)}
\end{align}
\انتہا{مسئلہ}
%=========================
\ابتدا{ثبوت}
ہم \عددی{f'(a)} اور \عددی{g'(a)}، جو از خود حد کو ظاہر کرتے ہیں، سے شروع کرتے ہوئے  واپس چلتے ہیں۔یوں درج ذیل ہو گا۔
\begin{align*}
\frac{f'(a)}{g'(a)}&=\frac{\lim_{x\to a}\frac{f(x)-f(a)}{x-a}}{\lim_{x\to a}\frac{g(x)-g(a)}{x-a}}=\lim_{x\to a}\frac{\frac{f(x)-f(a)}{x-a}}{\frac{g(x)-g(a)}{x-a}}\\
&=\lim_{x\to a}\frac{f(x)-f(a)}{g(x)-g(a)}\\
&=\lim_{x\to a}\frac{f(x)-0}{g(x)-0}\\
&=\lim_{x\to a}\frac{f(x)}{g(x)}
\end{align*}
\انتہا{ثبوت}
%=======================
 
قاعدہ لھوپیٹال استعمال کرتے ہوئے \عددی{f} کے تفرق \عددی{f'} کو \عددی{g} کے تفرق \عددی{g'} سے تقسیم کریں۔ یاد رہے کہ \عددی{\tfrac{f}{g}} کا تفرق \عددی{(\tfrac{f}{g})'}  درست نتیجہ نہیں دیگا۔

\ابتدا{مثال}\شناخت{مثال_ماورائی_قاعدہ_لھوپیٹال_الف}
\begin{align*}
\text{\RL{(ا)}}\quad \lim_{x\to 0}\frac{3x-\sin x}{x}&=\left. \frac{3-\cos x}{a}\right\vert_{x=0}=2\\
\text{\RL{(ب)}}\quad \lim_{x\to 0}\frac{\sqrt{1+x}-1}{x}&=\left.\frac{\tfrac{1}{2\sqrt{1+x}}}{1}\right\vert_{x=0}=\frac{1}{2}\\
\text{\RL{(ج)}}\quad \lim_{x\to 0}\frac{x-\sin x}{x^3}&=\left.\frac{1-\cos x}{3x^2}\right\vert_{x\to 0}=?\quad \text{\RL{اب بھی $\tfrac{0}{0}$ ملتا ہے}}
\end{align*}
\انتہا{مثال}
%======================

ہم دیکھتے ہیں کہ مثال \حوالہ{مثال_ماورائی_قاعدہ_لھوپیٹال_الف} کے جزو-ج میں قاعدہ لھوپیٹال کے استعمال کے باوجود \عددی{\tfrac{0}{0}} حاصل ہوتا ہے۔ قاعدہ لھوپیٹال کی بہتر روپ کہتی ہے کہ جب تک ہمیں \عددی{\tfrac{0}{0}}  حاصل ہو ہم اس قاعدہ کو بار بار استعمال کر سکتے ہیں۔ یوں اس مثال کو حل کرتے ہیں:
\begin{align*}
\lim_{x\to 0}\frac{x-\sin x}{x^3}&=\lim_{x\to 0}\frac{1-\cos x}{3x^2}&&\text{\RL{اب بھی $\tfrac{0}{0}$ ملتا ہے}}\\
&=\lim_{x\to 0}\frac{\sin x}{6x}&&\text{\RL{اب بھی $\tfrac{0}{0}$ ملتا ہے}}\\
&=\lim_{x\to 0}\frac{\cos x}{6}=\frac{1}{6}&&\text{\RL{اس بار $\tfrac{0}{0}$ نہیں ملا}}
\end{align*}

\ابتدا{مسئلہ}\شناخت{مسئلہ_ماورائی_لھوپیٹال_دوم}\موٹا{قاعدہ لھوپیٹال (بہتر روپ)}\\
فرض کریں کہ \عددی{f(a)=g(a)=0} ہے جبکہ   \عددی{f} اور \عددی{g} کھلے وقفہ  \عددی{I} پر قابل تفرق ہیں۔ اس وقفہ پر نقطہ \عددی{a} پایا جاتا ہے۔مزید فرض کریں کہ \عددی{x\ne a} کی صورت میں \عددی{I} پر \عددی{g'(x)\ne 0} ہے۔ تب درج ذیل ہو گا
\begin{align}
\lim_{x\to 0}\frac{f(x)}{g(x)}=\lim_{x\to a}\frac{f'(x)}{g'(x)}
\end{align}
اگر دائیں ہاتھ حد موجود ہو یا یہ \عددی{\infty} اور یا \عددی{-\infty} ہو۔
\انتہا{مسئلہ}
%========================

اس مسئلے کا ثبوت کتاب کے آخر میں ضمیمہ میں پیش کیا گیا ہے۔ 

\ابتدا{مثال}
\begin{align*}
\lim_{x\to 0}&\frac{\sqrt{1+x}-\tfrac{x}{2}}{x^2}&&\tfrac{0}{0}\\
&=\lim_{x\to a}\frac{\tfrac{1}{2}(1+x)^{-1/2}-\tfrac{1}{2}}{2x}&&\tfrac{0}{0}\\
&=\lim_{x\to a}\frac{-\tfrac{1}{4}(1+x)^{-3/2}}{2}=-\frac{1}{8}&&\text{\RL{$\tfrac{0}{0}$ نہیں ہے}}
\end{align*}
\انتہا{مثال}
%===========================
قاعدہ لھوپیٹال استعمال کرتے ہوئے جیسے  \عددی{\tfrac{0}{0}} سے کچھ ہٹ کر ملتا ہے آپ حد تلاش کر پائیں گے۔

\ابتدا{مثال}
\begin{align*}
\lim_{x\to 0}&\frac{1-\cos x}{x+x^2}&&\text{\RL{$\tfrac{0}{0}$ ملتا ہے}}\\
&=\lim_{x\to 0}\frac{\sin x}{1+2x}=\frac{0}{1}=0&&\text{\RL{$\tfrac{0}{0}$ نہیں ہے}}
\end{align*}
اگر \عددی{\tfrac{0}{0}} ملنے کے بعد رکنے کی بجائے ہم مزید ایک بار قاعدہ لھوپیٹال استعمال کریں تب ہمیں درج ذیل \موٹا{غلط} نتیجہ حاصل ہو گا۔
\begin{align*}
\lim_{x\to0}\frac{1-\cos x}{x+x^2}=\lim_{x\to 0}\frac{\sin x}{1+2x}=\lim_{x\to 0}\frac{\cos x}{2}=\frac{1}{2}
\end{align*} 
\انتہا{مثال}
%===================
\ابتدا{مثال}
\begin{align*}
\lim_{x\to 0^+}&\frac{\sin x}{x^2}&&\tfrac{0}{0}\\
&=\lim_{x\to 0^+}\frac{\cos x}{2x}=\infty&&\text{\RL{$\tfrac{0}{0}$ نہیں ملا}}
\end{align*}
\انتہا{مثال}
%===================

قاعدہ لھوپیٹال وہاں بھی قابل استعمال ہو گا جہاں وسطی روپ \عددی{\tfrac{\infty}{\infty}} ہو۔ اگر \عددی{x\to a} کرنے سے \عددی{f(x)} اور \عددی{g(x)} دونوں لامتناہی تک پہنچتے ہوں تب اگر درج ذیل میں دایاں حد موجود ہو تب
\begin{align*}
\lim_{x\to a}\frac{f(x)}{g(x)}=\lim_{x\to a}\frac{f'(x)}{g'(x)}
\end{align*}
ہو گا۔یہاں \عددی{a} از خود متناہی یا لا متناہی ہو سکتا ہے۔

\ابتدا{مثال}
\begin{align*}
\text{\RL{(ا)}}\quad \lim_{x\to (\tfrac{\pi}{2})^-}&\,\,\frac{\sec x}{1+\tan x}&&\tfrac{\infty}{\infty}\\
&=\lim_{x\to (\tfrac{\pi}{2})^-}\frac{\sec x\tan x}{\sec^2 x}=\lim_{x\to (\tfrac{\pi}{2})^-}\sin x=1\\
\text{\RL{(ب)}}\quad \lim_{x\to \infty}&\frac{\ln x}{2\sqrt{x}}=\lim_{x\to \infty}\frac{\tfrac{1}{x}}{\tfrac{1}{\sqrt{x}}}=\lim_{x\to \infty}\frac{1}{\sqrt{x}}=0
\end{align*}
\انتہا{مثال}
%================
\جزوحصہء{وسطی حاصل ضرب اور فرق}
بعض اوقات ہم وسطی روپ \عددی{0\cdot\infty} اور \عددی{\infty-\infty} کو الجبرا کی مدد سے \عددی{\tfrac{0}{0}} یا \عددی{\tfrac{\infty}{\infty}} لکھ سکتے ہیں۔ یاد رہے کہ ہم یہ نہیں  کہتے ہیں کہ عدد \عددی{0\cdot \infty} یا \عددی{\infty-\infty} موجود ہے اور نا ہی ہم کہتے ہیں کہ عدد \عددی{\tfrac{0}{0}} یا \عددی{\tfrac{\infty}{\infty}} موجود ہے۔ یہ روپ کسی بھی عدد کو ظاہر نہیں کرتے ہیں بلکہ محض تفاعل کے رویہ کو بیان کرتے ہیں۔

\ابتدا{مثال}
\begin{align*}
\lim_{x\to 0^+}&x\cot x&&\text{\RL{$0\cdot \infty $ کی بنا $x\cot x$ کو مختلف روپ میں لکھیں}}\\
&=\lim_{x\to 0^+}x\cdot \frac{1}{\tan x}\\
&=\lim_{x\to 0^+}\frac{x}{\tan x}&&\text{\RL{اب $\tfrac{0}{0}$ ملتا ہے}}\\
&=\lim_{x\to 0^+}\frac{1}{\sec^2x}=\frac{1}{1}=1
\end{align*}
\انتہا{مثال}
%====================
\ابتدا{مثال}
تلاش کریں: 
$\lim_{x\to 0}\big(\frac{1}{\sin x}-\frac{1}{x}\big)$ \\
حل:\quad
اگر \عددی{x\to 0^+} ہو تب \عددی{\sin x\to 0^+} اور درج ذیل ہو گا۔
\begin{align*}
\frac{1}{\sin x}-\frac{1}{x}\to \infty-\infty
\end{align*}
اسی طرح اگر \عددی{x\to 0^-} ہو تب \عددی{\sin x\to 0^-} اور درج ذیل ہو گا۔
\begin{align*}
\frac{1}{\sin x}-\frac{1}{x}=-\infty-(\infty)=-\infty+\infty
\end{align*}
دونوں صورتوں میں ہم حد جاننا ممکن نہیں ہے۔ہمیں  تفاعل کو نئی صورت
\begin{align*}
\frac{1}{\sin x}-\frac{1}{x}=\frac{x-\sin x}{x\sin x}
\end{align*} 
میں لکھ کر قاعدہ لھوپیٹال استعمال کرتے ہیں:
\begin{align*}
\lim_{x\to 0}\big(\frac{1}{\sin x}-\frac{1}{x}\big)&=\lim_{x\to0}\frac{x-\sin x}{x\sin x}&&\tfrac{0}{0}\\
&=\lim_{x\to 0}\frac{1-\cos x}{\sin x+x\cos x}&&\text{\RL{اب بھی $\tfrac{0}{0}$ ملتا ہے}}\\
&=\lim_{x\to0}\frac{\sin x}{2\cos x-x\sin x}=\frac{0}{2}=0
\end{align*}
\انتہا{مثال}
%===================
\جزوحصہء{وسطی طاقت}
