\حصہ{مخروط حصوں کے قطبی مساوات}
چاند، سیارے، مصنوعی سیارے اور دم دار ستارے ترخیم، قطع مکافی اور قطع زائد پر حرکت کرتے ہیں۔ ان کی حرکت کو قطبی محدد میں ایک آسان مساوات سے ظاہر کیا جا سکتا ہے لہٰذا فلکیات اور فلکیاتی انجینئری میں قطبی محدد  اہمیت رکھتے ہیں۔ ہم اس مساوات کو یہاں حاصل کرتے ہیں۔

\جزوحصہء{خطوط}
فرض کریں مبدا \عددی{O} سے خط \عددی{L} تک عمودی لکیر، \عددی{L} پر  نقطہ \عددی{N_0(r_0,\theta_0)} پہنچتی ہے جہاں \عددی{ r\ge0} ہے (شکل \حوالہ{شکل_مخروط_خط_قطبی_مساوات})۔ اب اگر \عددی{L} پر \عددی{N(r,\theta)} کوئی دوسرا نقطہ ہو تب نقاط \عددی{O}، \عددی{N_0} اور \عددی{N} ایک قائمہ الزاویہ مثلث کے راس ہوں گے جس سے
\begin{align*}
\frac{r_0}{r}=\cos(\theta-\theta_0)
\end{align*}
یا
\begin{align*}
r\cos(\theta-\theta_0)=r_0
\end{align*}
لکھا جا سکتا ہے۔
\begin{figure}
\centering
\begin{tikzpicture}[]
\draw[-latex](-0.25,0)--(2.5,0)node[right]{$x$};
\draw[-latex](0,-0.2)--(0,2.5)node[above]{$y$};
\draw(2,-0.2)--(0.5,2.5)coordinate[pos=0.85](kb)node[above]{$L$};
\draw(0,0)--($(2,-0.2)!(0,0)!(0.5,2.5)$)coordinate(ka)node[circ]{}node[right]{$N_0(r_0,\theta_0)$}node[pos=0.6,above]{$r_0$};
\RightAngle{(0,0)}{(ka)}{(0.5,2.5)}
\draw(0,0)node[below left]{$O$}--(kb)node[circ]{}node[right]{$N(r,\theta)$}node[pos=0.7,left]{$r$};
\draw[-stealth]([shift={(0:0.7)}]0,0)arc(0:30:0.7);
\draw(13:1)node[]{$\theta_0$};
\draw[-stealth]([shift={(0:0.5)}]0,0)arc(0:70:0.5);
\draw(50:0.75)node[]{$\theta$};
\end{tikzpicture}
\caption{خط کی قطبی مساوات}
\label{شکل_مخروط_خط_قطبی_مساوات}
\end{figure}
\موٹا{خط کی معیاری قطبی مساوات}\\
اگر مبدا سے خط \عددی{L} تک عمود نقطہ \عددی{N_0(r_0,\theta_0)} پر بیٹھتا ہو اور \عددی{r_0\ge 0} ہو تب \عددی{L} کی مساوات درج ذیل ہو گی۔
\begin{align*}
r\cos(\theta-\theta_0)=r_0
\end{align*}

\ابتدا{مثال}\شناخت{مثال_مخروط_خط_قطبی_مساوات_الف}
مماثل \عددی{\cos(A-B)=\cos A\cos B-\sin A\sin B} استعمال کر کے  شکل \حوالہ{شکل_مثال_مخروط_خط_قطبی_مساوات_الف} میں دیے خط کی مساوات تلاش کریں۔

حل:\quad
\begin{align*}
r\cos\big(\theta-\frac{\pi}{3}\big)&=2\\
r\big(\cos\theta\cos\frac{\pi}{3}-\sin\theta\sin\frac{\pi}{3}\big)&=2\\
\frac{1}{2}r\cos\theta+\frac{\sqrt{3}}{2}r\sin\theta&=2\\
\frac{1}{2}x+\frac{\sqrt{3}}{2}y&=2\\
x+\sqrt{3}y&=4
\end{align*}
\انتہا{مثال}
%=========================
\begin{figure}
\centering
\begin{minipage}{0.45\textwidth}
\centering
\begin{tikzpicture}[]
\draw[-latex](-0.25,0)--(2.5,0)node[right]{$x$};
\draw[-latex](0,-0.2)--(0,2.5)node[above]{$y$};
\draw(0,0)node[below left]{$O$}--++(60:1.5)node[pos=0.75,left]{$2$}node[circ]{}node[above right]{$(2,\tfrac{\pi}{3})$};
\draw(60:1.5)++(-30:1)coordinate(ka)--++(150:2);
\RightAngle{(0,0)}{(60:1.5)}{(ka)}
\draw[-stealth]([shift={(0:0.5)}]0,0) arc (0:60:0.5);
\draw(30:0.7)node[]{$\tfrac{\pi}{3}$};
\end{tikzpicture}
\caption{خط برائے مثال \حوالہ{مثال_مخروط_خط_قطبی_مساوات_الف}}
\label{شکل_مثال_مخروط_خط_قطبی_مساوات_الف}
\end{minipage}\hfill
\begin{minipage}{0.45\textwidth}
\centering
\begin{tikzpicture}[font=\small]
\draw[-latex](-0.25,0)--(4,0)node[right]{$x$};
\draw[-latex](0,-0.2)--(0,3)node[above]{$y$};
\draw(3.75,1.5)coordinate(ka)node[circ]{}node[below,xshift=2ex]{$N_0(r_0,\theta_0)$} circle (1.25);
\draw(0,0)node[below left]{$O$}--(ka)node[pos=0.6,below]{$r_0$};
\draw(ka)--++(70:1.25)node[above]{$N(r,\theta)$}coordinate(kb)node[pos=0.5,right]{$a$};
\draw(0,0)--(kb)node[pos=0.5,above left]{$r$};
\draw[-stealth]([shift={(0:0.9)}]0,0) arc (0:20:0.9);
\draw(10:1.2)node[]{$\theta_0$};
\draw[-stealth]([shift={(0:1.5)}]0,0) arc (0:32:1.5);
\draw(13:1.7)node[]{$\theta$};
\end{tikzpicture}
\caption{دائری کی قطبی مساوات۔}
\label{شکل_مخروط_دائرہ_قطبی_مساوات}
\end{minipage}
\end{figure}

\جزوحصہ{دائرے}
ایک دائرہ جس کا مرکز \عددی{N_0(r_0,\theta_0)} اور رداس \عددی{a} ہو کی قطبی مساوات حاصل کرنے کی خاطر ہم مثلث \عددی{ON_0N} پر قاعدہ کوسائن لاگو کرتے ہیں (شکل \حوالہ{شکل_مخروط_دائرہ_قطبی_مساوات}) جس سے درج ذیل حاصل ہو گا۔
\begin{align}\label{مساوات_مخروط_دائرہ_قطبی_الف}
a^2=r_0^2+r^2-2r_0r\cos(\theta-\theta_0)
\end{align}  
اگر یہ دائرہ مبدا سے گزرتا ہو تب  \عددی{r_0=a} ہو گا جس سے مساوات \حوالہ{مساوات_مخروط_دائرہ_قطبی_الف} درج ذیل سادہ صورت اختیار کرتی ہے۔
\begin{gather}
\begin{aligned}\label{مساوات_مخروط_دائرہ_قطبی_ب}
a^2&=a^2+r^2-2ar\cos(\theta-\theta_0)&&\text{\RL{(مساوات \حوالہ{مساوات_مخروط_دائرہ_قطبی_الف} میں \عددی{r_0=a})}}\\
r^2&=2ar\cos(\theta-\theta_0)\\
r&=2a\cos(\theta-\theta_0)
\end{aligned}
\end{gather}
اگر دائرے کا مرکز مثبت \عددی{x} محور پر پایا جاتا ہو تب مساوات \حوالہ{مساوات_مخروط_دائرہ_قطبی_ب} درج ذیل دے گی۔
\begin{align}\label{مساوات_مخروط_دائرہ_قطبی_پ}
r=2a\cos\theta
\end{align}
اگر دائرے کا مرکز مثبت \عددی{y} محور پر پایا جاتا ہو تب \عددی{\theta_0=\tfrac{\pi}{2}} اور \عددی{\cos(\theta-\tfrac{\pi}{2})=\sin\theta} ہوں گے لہٰذا  مساوات \حوالہ{مساوات_مخروط_دائرہ_قطبی_ب} درج ذیل دے گی۔
\begin{align}\label{مساوات_مخروط_دائرہ_قطبی_ت}
r=2a\sin\theta
\end{align}
مساوات \حوالہ{مساوات_مخروط_دائرہ_قطبی_پ} اور مساوات \حوالہ{مساوات_مخروط_دائرہ_قطبی_ت} میں \عددی{r} کی جگہ \عددی{-r} پر کر کے ان دائروں کی مساواتیں حاصل ہوں گی جن کے مرکز منفی \عددی{x} محور یا منفی \عددی{y} محور پر ہوں (شکل \حوالہ{شکل_مخروط_محددی_مرکوز_دائرے})۔

\begin{figure}
\centering
\begin{subfigure}{0.22\textwidth}
\centering
\begin{tikzpicture}[font=\small]
\pgfmathsetmacro{\r}{1}
\draw[-latex](-0.2,0)--(2.25,0)node[right]{$x$};
\draw[-latex](0,-1)--(0,1.5)node[above]{$y$};
\draw(\r,0)node[circ]{}node[below]{$(a,0)$} circle(\r);
\draw(\r,\r)node[above]{$r=2a\cos\theta$};
\end{tikzpicture}
\end{subfigure}\hfill
\begin{subfigure}{0.22\textwidth}
\centering
\begin{tikzpicture}[font=\small]
\pgfmathsetmacro{\r}{1}
\draw[-latex](-1,0)--(1.25,0)node[right]{$x$};
\draw[-latex](0,-0.2)--(0,2.25)node[above]{$y$};
\draw(0,\r)node[circ]{}node[right]{$(a,\tfrac{\pi}{2})$} circle(\r);
\draw(0,2*\r)node[above right]{$r=2a\sin\theta$};
\end{tikzpicture}
\end{subfigure}\hfill
\begin{subfigure}{0.22\textwidth}
\centering
\begin{tikzpicture}[font=\small]
\pgfmathsetmacro{\r}{1}
\draw[-latex](-2,0)--(0.25,0)node[right]{$x$};
\draw[-latex](0,-1)--(0,1.25)node[above]{$y$};
\draw(-\r,0)node[circ]{}node[below]{$(-a,0)$} circle(\r);
\draw(-\r,\r)node[above,xshift=-1ex]{$r=-2a\cos\theta$};
\end{tikzpicture}
\end{subfigure}\hfill
\begin{subfigure}{0.22\textwidth}
\centering
\begin{tikzpicture}[font=\small]
\pgfmathsetmacro{\r}{1}
\draw[-latex](-1,0)--(1.25,0)node[right]{$x$};
\draw[-latex](0,-2.25)--(0,0.25)node[above]{$y$};
\draw(0,-\r)node[circ]{}node[right]{$(-a,\tfrac{\pi}{2})$} circle(\r);
\draw(0,-2*\r)node[below right]{$r=-2a\sin\theta$};
\end{tikzpicture}
\end{subfigure}
\caption{کارتیسی محور پر مرکز والے دائروں کے قطبی مساوات۔}
\label{شکل_مخروط_محددی_مرکوز_دائرے}
\end{figure}

\ابتدا{مثال}\ترچھا{مبدا سے گزرتے ہوئے دائرے}\\
\begin{align*}
\begin{array}{ccc}
\text{رداس}&\text{مرکز}&\text{مساوات}\\
\midrule
3&(3,0)&r=6\cos\theta\\
2&(2,\tfrac{\pi}{2})&r=4\sin\theta\\
\tfrac{1}{2}&(-\tfrac{1}{2},0)&r=-\cos\theta\\
1&(-1,\tfrac{\pi}{2})&r=-2\sin\theta
\end{array}
\end{align*}
\انتہا{مثال}
%===========================

\جزوحصہء{ترخیم، قطع مکافی اور قطع زائد یکجا}
ترخیم، قطع مکافی اور قطع زائد کے قطبی مساوات معلوم کرنے کی خاطر ہم ایک ماسکہ کو مبدا پر رکھتے ہیں اور مطابقتی ناظمہ کو مبدا کے دائیں، انتصابی لکیر \عددی{x=k} پر رکھتے ہیں۔ یوں 
\begin{align*}
NF=r
\end{align*}
اور
\begin{align*}
ND=k-FB=k-r\cos\theta
\end{align*}
ہوں گے۔ یوں مخروط کی ماسکہ ناظمہ مساوات \عددی{NF=e\cdot ND} درج ذیل  صورت اختیار کرتی ہے
\begin{align*}
r=e(k-r\cos\theta)
\end{align*}
جس کو \عددی{r} کے لئے حل کرتے ہیں:
\begin{align}\label{مساوات_مخروط_ترخیم_قطع_مکافی_قطع_زائد}
r=\frac{ke}{1+e\cos\theta}
\end{align}
یہ مساوات \عددی{0<e<1} کے لئے ترخیم، \عددی{e=1} کے لئے قطع مکافی اور \عددی{e>1} کے لئے قطع زائد کو ظاہر کرتی ہے۔ اس طرح ترخیم، قطع مکافی اور قطع زائد کو ایک ہی مساوات ظاہر کرتی ہے۔

