
\حصہء{سوالات}
\موٹا{ترتیب کے اجزاء کی تلاش}\\
سوال \حوالہ{سوال_تسلسل_کلیہ_دی_ہے_الف} تا سوال \حوالہ{سوال_تسلسل_کلیہ_دی_ہے_ب} میں ترتیب کی \عددی{n} ویں جزو کا کلیہ دیا گیا ہے۔ اس کے ابتدائی اجزاء \عددی{a_1}، \عددی{a_2}، \عددی{a_3} اور \عددی{a_4} تلاش کریں۔

\ابتدا{سوال}\شناخت{سوال_تسلسل_کلیہ_دی_ہے_الف}
$a_n=\frac{1-n}{n^2}$
\انتہا{سوال}
%=====================
\ابتدا{سوال}
$a_n=\frac{1}{n!}$
\انتہا{سوال}
%====================
\ابتدا{سوال}
$a_n=\frac{(-1)^{n+1}}{2n-1}$
\انتہا{سوال}
%====================
\ابتدا{سوال}
$a_n=2+(-1)^n$
\انتہا{سوال}
%====================
\ابتدا{سوال}
$a_n=\frac{2^n}{2^{n+1}}$
\انتہا{سوال}
%====================
\ابتدا{سوال}\شناخت{سوال_تسلسل_کلیہ_دی_ہے_ب}
$a_n=\frac{2^n-1}{2^n}$
\انتہا{سوال}
%====================
سوال \حوالہ{سوال_تسلسل_کلیہ_توالی_دیا_ہے_الف} تا سوال \حوالہ{سوال_تسلسل_کلیہ_توالی_دیا_ہے_ب} میں ابتدائی ایک یا دو اجزاء اور کلیہ توالی دی گئی ہے۔ ابتدائی دس اجزاء تلاش کریں۔

\ابتدا{سوال}\شناخت{سوال_تسلسل_کلیہ_توالی_دیا_ہے_الف}
$a_1=1,\quad a_{n+1}=a_n+\big(\frac{1}{2^n}\big)^n$
\انتہا{سوال}
%==========================
\ابتدا{سوال}
$a_1=1,\quad a_{n+1}\frac{a_n}{n+1}$
\انتہا{سوال}
%======================
\ابتدا{سوال}
$a_1=2,\quad a_{n+1}=(-1)^{n+1}\frac{a_n}{2}$
\انتہا{سوال}
%======================
\ابتدا{سوال}
$a_1=-2,\quad a_{n+1}=\frac{na_n}{n+1}$
\انتہا{سوال}
%======================
\ابتدا{سوال}
$a_1=a_2=1,\quad a_{n+2}=a_{n+1}+a_n$
\انتہا{سوال}
%======================
\ابتدا{سوال}\شناخت{سوال_تسلسل_کلیہ_توالی_دیا_ہے_ب}
$a_1=2,\quad a_2=-1,\quad a_{n+2}=\frac{a_{n+1}}{a_n}$
\انتہا{سوال}
%======================
\موٹا{ترتیب کے کلیہ کی تلاش}\\
سوال \حوالہ{سوال_تسلسل_کلیہ_تلاش_الف} تا سوال \حوالہ{سوال_تسلسل_کلیہ_تلاش_ب} میں دیے گئے ترتیب کے \عددی{n} ویں جزو کا کلیہ تلاش کریں۔

\ابتدا{سوال}\شناخت{سوال_تسلسل_کلیہ_تلاش_الف}
$1,-1,1,-1,1,\cdots$\quad
ہر بار \عددی{1} کی علامت تبدیل ہوتی ہے۔
\انتہا{سوال}
%==================
\ابتدا{سوال}
$-1,1,-1,1,-1,\cdots$\quad
ہر بار \عددی{1} کی علامت تبدیل ہوتی ہے۔
\انتہا{سوال}
%==========================
\ابتدا{سوال}
$1,-4,9,-16,25,\cdots$\quad
مثبت عدد صحیح کا مربع جس کی علامت ہر بار تبدیل ہوتی ہے۔ 
\انتہا{سوال}
%======================
\ابتدا{سوال}
$1,-\frac{1}{4},\frac{1}{9},-\frac{1}{16},\frac{1}{25},\cdots$\quad
مثبت عدد صحیح کے مربع کا بالعکس متناسب جس کی علامت ہر بار تبدیل ہوتی ہے۔
\انتہا{سوال}
%======================
\ابتدا{سوال}
$0,3,8,15,24,\cdots$\quad
مثبت عدد صحیح کے مربع سے \عددی{1} کم۔
\انتہا{سوال}
%======================
\ابتدا{سوال}
$-3,-2,-1,0,1,\cdots$\quad
\عددی{-3} سے شروع کرتے ہوئے عدد صحیح۔
\انتہا{سوال}
%======================
\ابتدا{سوال}
$1,5,9,13,17,\cdots$\quad
ہر دوسرا طاق مثبت عدد صحیح۔
\انتہا{سوال}
%======================
\ابتدا{سوال}
$2,6,10,14,18,\cdots$\quad
ہر دوسرا جفت مثبت عدد صحیح۔
\انتہا{سوال}
%======================
\ابتدا{سوال}
$1,0,1,0,1,\cdots$\quad
باری باری \عددی{1} اور \عددی{0}
\انتہا{سوال}
%======================
\ابتدا{سوال}\شناخت{سوال_تسلسل_کلیہ_تلاش_ب}
$1,1,2,2,3,3,4,\cdots$\quad
ہر مثبت عدد صحیح دو بار۔
\انتہا{سوال}
%======================
\موٹا{کیلکولیٹر کی مدد سے حد کی تلاش}\\
سوال \حوالہ{سوال_تسلسل_کیلکولیٹر_حد_الف} تا سوال \حوالہ{سوال_تسلسل_کیلکولیٹر_حد_ب} میں کیلکولیٹر کے  ساتھ تجربات کرتے ہوئے \عددی{N} کی وہ قیمت تلاش کریں جو دی گئی عدم مساوات کو تمام \عددی{n>N} کے لئے مطمئن کرتا ہو۔ دی گئی عدم مساوات، تسلسل کی حد کی با ضابطہ تعریف کے تحت ہے۔ تسلسل کی تفصیل پیش کریں اور اس کی حد تلاش کریں۔

\ابتدا{سوال}\شناخت{سوال_تسلسل_کیلکولیٹر_حد_الف}
$\abs{\sqrt[n]{0.5}-1}<10^{-3}$
\انتہا{سوال}
%=====================
\ابتدا{سوال}
$\abs{\sqrt[n]{n}-1}<10^{-3}$
\انتہا{سوال}
%========================
\ابتدا{سوال}
$(0.9)^n<10^{-3}$
\انتہا{سوال}
%========================
\ابتدا{سوال}\شناخت{سوال_تسلسل_کیلکولیٹر_حد_ب}
$\frac{2^n}{n!}<10^{-7}$
\انتہا{سوال}
%========================
\ابتدا{سوال}\شناخت{سوال_تسلسل_ترکیب_نیوٹن}\ترچھا{ترکیب نیوٹن سے حاصل ترتیبات}\\
ترکیب نیوٹن کی قابل تفرق تفاعل \عددی{f(x)} پر اطلاق  سے ابتدائی قیمت \عددی{x_0} اور اس کے بعد اعداد کی ترتیب \عددی{\{x_n\}} حاصل ہوتی ہے جو موزوں صورت میں \عددی{f} کے صفر پر مرتکز ہو گی۔ اس ترتیب کا کلیہ توالی درج ذیل ہے۔
\begin{align*}
x_{n+1}=x_n-\frac{f(x_n)}{f'(x_n)}
\end{align*}
\begin{enumerate}[a.]
\item
دکھائیں کہ \عددی{f(x)=x^2-a^2,\, a>0} کا کلیہ توالی \عددی{x_{n+1}=\tfrac{x_n+a/{x_n}}{2}} ہے۔
\item
ابتدائی قیمت \عددی{x_0=1} اور \عددی{a=3} لیتے ہوئے وہاں تک یک بعد دیگرے اجزاء تلاش کریں جب اجزاء دہرانے شروع ہو جاتے ہیں۔ کون سے عدد کی تخمین حاصل ہوتی ہے؟ وجہ پیش کریں۔ 
\end{enumerate}
\انتہا{سوال}
%=====================
\ابتدا{سوال} 
گزشتہ سوال (سوال \حوالہ{سوال_تسلسل_ترکیب_نیوٹن}) میں \عددی{a=3} کی بجائے \عددی{a=2} لیتے ہوئے جزو-ب دوبارہ حل کریں۔
\انتہا{سوال}
%========================
\ابتدا{سوال}\شناخت{سوال_تسلسل_پائے}\ترچھا{$\tfrac{\pi}{2}$ کی تعریف توالی}\\
اگر آپ \عددی{x_1=1} سے شروع کر کے  \عددی{\{a_n\}} کے باقی اجزاء کو قاعدہ \عددی{x_n=x_{n-1}+\cos x_{n-1}} سے حاصل کریں تب ایک ایسی ترتیب حاصل ہو گی جو بہت تیزی سے \عددی{\tfrac{\pi}{2}} پر مرتکز ہو گی۔ (ا) ایسا کر کے دیکھیں۔ (ب) اتنی تیز ارتکاز کی وجہ شکل \حوالہ{شکل_سوال_تسلسل_پائے} کی مدد سے پیش کریں۔
\انتہا{سوال}
%====================
\begin{figure}
\centering
\begin{tikzpicture}
\pgfmathsetmacro{\r}{1.5}
\pgfmathsetmacro{\ang}{50}
\draw[-latex](-0.25,0)--(\r+0.5,0)node[right]{$x$};
\draw[-latex](0,-0.2)--(0,2)node[above]{$y$};
\draw(\r,0)node[below]{$1$}  (0,\r)node[left]{$1$};
\draw([shift={(0:\r)}]0,0) arc (0:90:\r);
\draw(0,0)--++(\ang:\r)coordinate(kT)--($(0,0)!(kT)!(0,1.5)$)node[pos=0.6,pin=60:{$\cos x_{n-1}$}]{};
\draw[-stealth]([shift={(0:0.5)}]0,0) arc (0:\ang:0.5);
\draw(1/2*\ang:0.5)node[right]{$x_{n-1}$};
\draw[thick]([shift={(0:\r)}]0,0) arc (0:\ang:\r);
\draw(1/2*\ang:\r)node[right]{$x_{n-1}$};
\end{tikzpicture}
\caption{اکائی دائرہ برائے سوال \حوالہ{شکل_سوال_تسلسل_پائے}}
\label{شکل_سوال_تسلسل_پائے}
\end{figure}
\ابتدا{سوال}
گاڑیاں بنانے والا ایک کارخانہ دھاتی چادر کو دبا کر ایک گاڑی  کا ڈھانچہ  اوسطاً \عددی{7.25} گھنٹوں میں تیار کرتا ہے۔ اگر ڈھانچہ تیار کرنے کے لئے درکار وقت میں سالانہ \عددی{\SI{6}{\percent}}  کمی رونما ہو تب \عددی{n} سالوں بعد
\begin{align*}
S_n=7.25(0.94)^n
\end{align*} 
اتنا وقت درکار ہو گا۔ کتنے سالوں بعد تقریباً \عددی{3.5} گھنٹے درکار ہوں گے؟ جواب کو دو مختلف طریقوں سے تلاش کریں:
\begin{enumerate}[a.]
\item
تسلسل \عددی{S_n} کا وہ پہلا جزو تلاش کریں جو \عددی{3.5} کے برابر یا اس سے کم ہو۔
\item
تفاعل \عددی{f(x)=7.25(0.96)^x} ترسیم کر کے دیکھیں یہ کہاں لکیر \عددی{y=3.5} کو مس کرتی ہے۔
\end{enumerate}
\انتہا{سوال}
%=====================
\موٹا{نظریہ اور مثالیں}\\
سوال \حوالہ{سوال_تسلسل_اوپر_سے_محدود_الف} تا سوال \حوالہ{سوال_تسلسل_اوپر_سے_محدود_ب} میں معلوم کریں کہ آیا تسلسل غیر گھٹتی ہے اور کیا یہ اوپر سے محدود ہے۔

\ابتدا{سوال}\شناخت{سوال_تسلسل_اوپر_سے_محدود_الف}
$a_n=\frac{3n+1}{n+1}$
\انتہا{سوال}
%======================
\ابتدا{سوال}
$a_n=\frac{(2n+3)!}{(n+1)!}$
\انتہا{سوال}
%====================
\ابتدا{سوال}
$a_n=\frac{2^n3^n}{n!}$
\انتہا{سوال}
%====================
\ابتدا{سوال}\شناخت{سوال_تسلسل_اوپر_سے_محدود_ب}
$a_n=2-\frac{2}{n}-\frac{1}{2^n}$
\انتہا{سوال}
%====================
سوال \حوالہ{سوال_تسلسل_مرتکز_منفرج_الف} تا سوال \حوالہ{سوال_تسلسل_مرتکز_منفرج_ب} میں کون سی ترتیب مرتکز ہے اور کون سی منفرج؟ اپنے جواب کی وجہ پیش کریں۔

\ابتدا{سوال}\شناخت{سوال_تسلسل_مرتکز_منفرج_الف}
$a_n=1-\frac{1}{n}$
\انتہا{سوال}
%===================
\ابتدا{سوال}
$a_n=n-\frac{1}{n}$
\انتہا{سوال}
%====================
\ابتدا{سوال}
$a_n=\frac{2^n-1}{2^n}$
\انتہا{سوال}
%====================
\ابتدا{سوال}
$a_n\frac{2^n-1}{3^n}$
\انتہا{سوال}
%====================
\ابتدا{سوال}
$a_n=[(-1)^n+1]\big(\frac{n+1}{n}\big)$
\انتہا{سوال}
%====================
\ابتدا{سوال}\شناخت{سوال_تسلسل_مرتکز_منفرج_ب}
ایک ترتیب کا پہلا جزو \عددی{x_1=\cos(1)}، اگلا جزو \عددی{x_2=x_1} یا \عددی{\cos(2)} میں سے جو بھی بڑا ہے، اس سے اگلا جزو \عددی{x_3=x_2} یا \عددی{\cos(3)} میں سے جو بھی بڑا (دائیں جانب زیادہ دور) ہے۔ یوں عمومی جزو درج ذیل ہو گا۔
\begin{align*}
x_{n+1}=\{x_n,\cos(n+1)\}_{\text{\RL{زیادہ بڑا}}}
\end{align*}
\انتہا{سوال}
%==================
\ابتدا{سوال}\شناخت{سوال_تسلسل_غیر_بڑھتا_ترتیب}\ترچھا{غیر بڑھتے ترتیبات}\\
ایک ترتیب جس میں تمام \عددی{n} کے لئے \عددی{a_n>a_{n+1}} ہو \اصطلاح{غیر بڑھتا ترتیب}\فرہنگ{ترتیب!غیر بڑھتا}\حاشیہب{nonincreasing sequence}\فرہنگ{sequence!nonincreasing} کہلاتا ہے۔ اگر ہر \عددی{n} کے لئے \عددی{M\le a_n} ہو جہاں \عددی{M} کوئی عدد ہو تب \عددی{M} کو ترتیب \عددی{\{a_n\}} کی \اصطلاح{زیریں حد بندی}\فرہنگ{حد بندی!زیریں}\حاشیہب{lower bound}\فرہنگ{bound!lower} کہتے ہیں اور ہم کہتے ہیں کہ یہ ترتیب \اصطلاح{نیچے سے محدود}\فرہنگ{ترتیب!نیچے سے محدود}\حاشیہب{bounded from below}\فرہنگ{bounded!from below} ہے۔ مسئلہ \حوالہ{مسئلہ_تسلسل_غیر_گھٹتا_تسلسل} سے اخذ کریں کہ ایسا غیر بڑھتا تسلسل  جو نیچے سے محدود ہو مرتکز ہو گا جبکہ غیر بڑھتا تسلسل جو نیچے سے محدود نہ ہو منفرج ہو گا۔
\انتہا{سوال}
%========================
سوال \حوالہ{سوال_تسلسل_غیر_بڑھتا_الف} تا سوال \حوالہ{سوال_تسلسل_غیر_بڑھتا_ب} میں سوال \حوالہ{سوال_تسلسل_غیر_بڑھتا_ترتیب} کا نتیجہ استعمال کرتے ہوئے معلوم کریں کہ کونسی ترتیب مرتکز اور کونسی سی منفرج ہے۔

\ابتدا{سوال}\شناخت{سوال_تسلسل_غیر_بڑھتا_الف}
$a_n=\frac{n+1}{n}$
\انتہا{سوال}
%===================
\ابتدا{سوال}
$a_n=\frac{1+\sqrt{2n}}{\sqrt{n}}$
\انتہا{سوال}
%===================
\ابتدا{سوال}
$a_n=\frac{1-4^n}{2^n}$
\انتہا{سوال}
%===================
\ابتدا{سوال}
$a_n=\frac{4^{n+1}+3^n}{4^n}$
\انتہا{سوال}
%===================
\ابتدا{سوال}\شناخت{سوال_تسلسل_غیر_بڑھتا_ب}
$a_1=1,\quad a_{n+1}=2a_n-3$
\انتہا{سوال}
%=============
\ابتدا{سوال}
ترتیب \عددی{\{\tfrac{n}{n+1}\}} کی کم سے کم بالائی حد بندی \عددی{1} ہے۔ دکھائیں کہ اگر عدد \عددی{M} ایک سے کم ہو تب   \عددی{\{\tfrac{n}{n+1}\}} کے اجزاء آخر کار \عددی{M} سے تجاوز کر جائیں گے۔ یعنی \عددی{M<1} کی صورت میں ایسا عدد صحیح \عددی{N} موجود  ہو گا کہ جب \عددی{n>N} ہو تب \عددی{\tfrac{n}{n+1}>M} ہو گا۔ چونکہ ہر \عددی{n} کے لئے \عددی{\tfrac{n}{n+1}<1} ہے  لہٰذا یوں ثابت ہوتا ہے کہ  \عددی{\{\tfrac{n}{n+1}\}} کی بالائی حد بندی \عددی{1} ہو گی۔
\انتہا{سوال}
%===================
\ابتدا{سوال}\ترچھا{کم سے کم بالائی حد بندی کی یکتائی}\\
دکھائیں کہ اگر \عددی{M_1} اور \عددی{M_2} ترتیب \عددی{\{a_n\}} کے کم سے کم بالائی حد بندی ہوں تب \عددی{M_1=M_2} ہو گا، یعنی، کسی بھی ترتیب کے دو مختلف کم سے کم بالائی حد بندی نہیں ہو سکتی ہیں۔
\انتہا{سوال}
%======================
\ابتدا{سوال}
کیا ضروری ہے کہ اوپر سے محدود، مثبت اعداد کی ترتیب \عددی{\{a_n\}} لازماً  مرتکز ہو گی؟ اپنے جواب کی وجہ پیش کریں۔
\انتہا{سوال}
%======================
\ابتدا{سوال}
اگر \عددی{\{a_n\}} مرتکز ترتیب ہو تب دکھائیں کہ ہر مثبت عدد \عددی{\epsilon} کے لئے ایسا مطابقتی عدد صحیح \عددی{N} ہو گا کہ تمام \عددی{m} اور \عددی{n} کے لئے درج ذیل ہو۔
\begin{align*}
m>N\quad\text{اور}\quad n>N\quad \implies \quad \abs{a_m-a_n}<\epsilon
\end{align*}

\انتہا{سوال}
%==================
\ابتدا{سوال}\ترچھا{حد کی یکتائی}\\
ثابت کریں کہ ہر ترتیب کا حد یکتا ہو گا، یعنی، دکھائیں کہ اگر \عددی{L_1} اور \عددی{L_2} ایسے اعداد ہوں کہ \عددی{a_n\to L_1} اور \عددی{a_m\to L_2} ہوں تب \عددی{L_1=L_2} ہو گا۔
\انتہا{سوال}
%=====================
\ابتدا{سوال}\ترچھا{ترتیبات اور حد}\\
دکھائیں کہ اگر ترتیب \عددی{\{a_n\}} کے دو ذیلی ترتیبات کے حد مختلف ہوں، \عددی{L_1\ne L_2} تب \عددی{\{a_n\}} منفرج ترتیب ہو گی۔ 
\انتہا{سوال}
%=====================
\ابتدا{سوال}
ترتیب \عددی{\{a_n\}} کے جفت اشاریہ کے اجزاء کو \عددی{a_{2k}} اور طاق اشاریہ کے اجزاء کو \عددی{a_{2k+1}} سے ظاہر کیا جاتا ہے۔ ثابت کریں کہ \عددی{a_{2k}\to L} اور \عددی{a_{2k+1}\to L} کی صورت میں \عددی{a_n\to L} ہو گا۔
\انتہا{سوال}
%==================
\ابتدا{سوال}
دکھائیں کہ ترتیب \عددی{\{a_n\}} اس صورت \عددی{0} کو مرتکز ہو گا جب مطلق قیمتیں \عددی{\{\abs{a_n}\}} صفر کو مرتکز ہوں۔
\انتہا{سوال}
%===================
\موٹا{کمپیوٹر کا استعمال}\\
سوال \حوالہ{سوال_تسلسل_کمپیوٹر_اقدام_الف} تا سوال \حوالہ{سوال_تسلسل_کمپیوٹر_اقدام_ب} میں کمپیوٹر کی مدد سے درج ذیل اقدام کریں۔
\begin{enumerate}[a.]
\item
ابتدائی \عددی{25} اجزاء کا حساب لگا کر انہیں ترسیم کریں۔ کیا ترتیب اوپر یا نیچے سے محدود نظر آتی ہے؟ کیا یہ منفرج یا مرتکز نظر آتی ہے؟ ارتکاز کی صورت میں حد \عددی{L} کتنا ہے؟
\item
اگر تسلسل مرتکز ہو تب ایسا عدد صحیح \عددی{N} تلاش کریں کہ \عددی{n\ge N} کے لئے \عددی{\abs{a_n-L}\le 0.01} ہو۔ ترتیب میں کتنا آگے جا کر  \عددی{L} اور اجزاء کے بیچ فاصلہ \عددی{0.0001} سے کم ہو گا؟
\end{enumerate}

\ابتدا{سوال}\شناخت{سوال_تسلسل_کمپیوٹر_اقدام_الف}
$a_n=\sqrt[n]{n}$
\انتہا{سوال}
%=======================
\ابتدا{سوال}
$a_n=\big(2+\frac{0.5}{n}\big)^n$
\انتہا{سوال}
%=====================
\ابتدا{سوال}
$a_1=1,\quad a_{n+1}=a_n+\frac{1}{5^n}$
\انتہا{سوال}
%=====================
\ابتدا{سوال}
$a_1=1,\quad a_{n+1}=a_n+(-2)^n$
\انتہا{سوال}
%=====================
\ابتدا{سوال}
$a_n=\sin n$
\انتہا{سوال}
%=====================
\ابتدا{سوال}
$a_n=n\sin\frac{1}{n}$
\انتہا{سوال}
%=====================
\ابتدا{سوال}
$a_n\frac{\sin n}{n}$
\انتہا{سوال}
%=====================
\ابتدا{سوال}
$a_n=\frac{\ln n}{n}$
\انتہا{سوال}
%=====================
\ابتدا{سوال}
$a_n=(0.9999)^n$
\انتہا{سوال}
%=====================
\ابتدا{سوال}
$a_n=123456^{1/n}$
\انتہا{سوال}
%=====================
\ابتدا{سوال}
$a_n=\frac{8^n}{n!}$
\انتہا{سوال}
%=====================
\ابتدا{سوال}\شناخت{سوال_تسلسل_کمپیوٹر_اقدام_ب}
$a_n=\frac{n^{41}}{19^n}$
\انتہا{سوال}
%=====================
\ابتدا{سوال}\ترچھا{سود در سود}\\
آپ ایک بینک میں مستقل رقم \عددی{A_0} جمع کرتے ہیں جو سالانہ \عددی{r} فی صد سود کا ایک سال میں \عددی{m} مرتبہ حساب لگا کر  آپ کے رقم میں جمع کرتی ہے۔ مزید آپ ہر سال \عددی{b} رقم بھی بینک میں جمع کرتے ہیں یا \عددی{b<0} کی صورت میں بینک سے نکالتے ہیں۔یوں \عددی{n+1} سال بعد کل رقم درج ذیل ہو گی۔
\begin{align}\label{مساوات_تسلسل_سود_در_سود_الف}
A_{n+1}=\big(1+\frac{r}{m}\big)A_n+b
\end{align}
\begin{enumerate}[a.]
\item
اگر \عددی{A_0=1000}، \عددی{r=0.02015}، \عددی{m=12} اور \عددی{b=50} ہوں تب ابتدائی \عددی{100} نقطوں \عددی{(n,A_n)} کو ترسیم کریں۔ پانچ سال کے آخر میں آپ کی رقم کتنی ہو گی؟ کیا \عددی{\{A_n\}} مرتکز ہے؟ کیا \عددی{\{A_n\}} محدود ہے۔
\item
اگر \عددی{A_0=5000}، \عددی{r=0۔0589}، \عددی{m=12} اور \عددی{b=-50} ہوں تب ابتدائی \عددی{100} نقطوں \عددی{(n,A_n)} کو ترسیم کریں۔
\item
اگر آپ بینک میں \عددی{5000} رقم مستقل طور پر جمع کریں جس  پر سالانہ \عددی{\SI{4.5}{\percent}} سود ہو جس کا ایک سال میں چار مرتبہ  حساب کیا جاتا ہو تب کتنے سالوں بعد آپ کی رقم \عددی{20000} ہو گی۔ اگر سود \عددی{\SI{6.25}{\percent}} ہو؟
\item
سود در سود کا تعلق مساوات \حوالہ{مساوات_تسلسل_سود_در_سود_الف} میں پیش کیا گیا ہے جو \عددی{k\ge 0} کے لئے درج ذیل تعلق کو مطمئن کرتی  ہے
\begin{align}\label{مساوات_تسلسل_سود_در_سود_ب}
A_k=(1+r/m)^k(A_0+mb/r)-\frac{mb}{r}
\end{align}
جس کی تصدیق کی خاطر مساوات \حوالہ{مساوات_تسلسل_سود_در_سود_الف} اور مساوات \حوالہ{مساوات_تسلسل_سود_در_سود_ب} کی ابتدائی \عددی{50} اجزاء کا آپس میں موازنہ کریں۔ اس کے بعد مساوات \حوالہ{مساوات_تسلسل_سود_در_سود_ب} سے مساوات \حوالہ{مساوات_تسلسل_سود_در_سود_الف} اخذ کریں۔
\end{enumerate}
\انتہا{سوال}
%======================
