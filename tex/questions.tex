\جزوحصہء{سوالات}
\موٹا{ترسیم کی پہچان}\\
سوال \حوالہ{سوال_مخروط_ہم_پلہ_الف} تا سوال \حوالہ{سوال_مخروط_ہم_پلہ_الف} میں دیے قطع مکافی کا ہم پلہ درج ذیل میں تلاش کریں۔
\begin{align*}
x^2=2y,\quad x^2=-6y,\quad y^2=8x,\quad y^2=-4x
\end{align*}
اس کے بعد قطع مکافی کے ماسکہ اور ناظمہ دریافت کریں۔

\ابتدا{سوال}\شناخت{سوال_مخروط_ہم_پلہ_الف}
شکل \حوالہ{شکل_مخروط_ترخیم_بائیں_دائیں}-ا
\انتہا{سوال}
%===================
\ابتدا{سوال}
شکل \حوالہ{شکل_مخروط_ترخیم_بائیں_دائیں}-ب
\انتہا{سوال}
%===================
\ابتدا{سوال}
شکل \حوالہ{شکل_مخروط_ترخیم_بائیں_دائیں}-ج
\انتہا{سوال}
%===================
\ابتدا{سوال}\شناخت{سوال_مخروط_ہم_پلہ_ب}
شکل \حوالہ{شکل_مخروط_ترخیم_بائیں_دائیں}-د
\انتہا{سوال}
%===================
\begin{figure}
\centering
\begin{subfigure}{0.45\textwidth}
\centering
\begin{tikzpicture}[declare function={f(\x)=2*sqrt(\x);}]
\begin{axis}[small,axis lines=middle,xlabel={$x$},ylabel={$y$},xlabel style={at={(current axis.right of origin)},anchor=west},ylabel style={at={(current axis.above origin)},anchor=south},xtick={\empty},ytick={\empty},enlargelimits]
\addplot[domain=0:0.2]{f(x)};
\addplot[domain=0.2:2]{f(x)};
\addplot[domain=0:0.2]{-f(x)};
\addplot[domain=0.2:2]{-f(x)};
\end{axis}
\end{tikzpicture}
\caption{}
\end{subfigure}\hfill
\begin{subfigure}{0.45\textwidth}
\centering
\begin{tikzpicture}[declare function={f(\x)=2*sqrt(-\x);}]
\begin{axis}[small,axis lines=middle,xlabel={$x$},ylabel={$y$},xlabel style={at={(current axis.right of origin)},anchor=west},ylabel style={at={(current axis.above origin)},anchor=south},xtick={\empty},ytick={\empty},enlargelimits=true]
\addplot[domain=0:-0.2]{f(x)};
\addplot[domain=-0.2:-2]{f(x)};
\addplot[domain=0:-0.2]{-f(x)};
\addplot[domain=-0.2:-2]{-f(x)};
\end{axis}
\end{tikzpicture}
\caption{}
\end{subfigure}
\begin{subfigure}{0.45\textwidth}
\centering
\begin{tikzpicture}[declare function={f(\x)=-1/4*\x^2;}]
\begin{axis}[small,axis lines=middle,xlabel={$x$},ylabel={$y$},xlabel style={at={(current axis.right of origin)},anchor=west},ylabel style={at={(current axis.above origin)},anchor=south},xtick={\empty},ytick={\empty},enlargelimits]
\addplot[domain=0:0.2]{f(x)};
\addplot[domain=0.2:2]{f(x)};
\addplot[domain=0:0.2](-x,{f(x)});
\addplot[domain=0.2:2](-x,{f(x)});
\end{axis}
\end{tikzpicture}
\caption{}
\end{subfigure}\hfill
\begin{subfigure}{0.45\textwidth}
\centering
\begin{tikzpicture}[declare function={f(\x)=1/4*\x^2;}]
\begin{axis}[small,axis lines=middle,xlabel={$x$},ylabel={$y$},xlabel style={at={(current axis.right of origin)},anchor=west},ylabel style={at={(current axis.above origin)},anchor=south},xtick={\empty},ytick={\empty},enlargelimits=true]
\addplot[domain=0:0.2]{f(x)};
\addplot[domain=0.2:2]{f(x)};
\addplot[domain=0:0.2](-x,{f(x)});
\addplot[domain=0.2:2](-x,{f(x)});
\end{axis}
\end{tikzpicture}
\caption{}
\end{subfigure}
\caption{ترسیم برائے سوال \حوالہ{سوال_مخروط_ہم_پلہ_الف} تا سوال \حوالہ{سوال_مخروط_ہم_پلہ_ب}}
\label{شکل_مخروط_ترخیم_بائیں_دائیں}
\end{figure}

سوال \حوالہ{سوال_مخروط_ہم_پلہ_تلاش_الف} تا سوال \حوالہ{سوال_مخروط_ہم_پلہ_تلاش_ب} میں دیے مخروط کا درج ذیل میں ہم پلہ مساوات تلاش کریں۔
\begin{align*}
\frac{x^2}{4}+\frac{y^2}{9}=1,\quad \frac{x^2}{2}+y^2=1,\quad \frac{y^2}{4}-x^2=1,\quad \frac{x^2}{4}-\frac{y^2}{9}=1
\end{align*}
دیے گئے مخروط کا ماسکہ اور راس تلاش کریں۔ اگر قطع زائد دیا گیا ہو تب اس کے متقارب بھی دریافت کریں۔

\ابتدا{سوال}\شناخت{سوال_مخروط_ہم_پلہ_تلاش_الف}
ترسیم شکل \حوالہ{شکل_مخروط_کئی_الف}-ا میں دیا گیا ہے
\انتہا{سوال}
%===================
\ابتدا{سوال}
ترسیم شکل \حوالہ{شکل_مخروط_کئی_الف}-ب میں دیا گیا ہے
\انتہا{سوال}
%========================
\ابتدا{سوال}
ترسیم شکل \حوالہ{شکل_مخروط_کئی_الف}-ج میں دیا گیا ہے
\انتہا{سوال}
%========================
\ابتدا{سوال}\شناخت{سوال_مخروط_ہم_پلہ_تلاش_ب}
ترسیم شکل \حوالہ{شکل_مخروط_کئی_الف}-د میں دیا گیا ہے
\انتہا{سوال}
%========================
\begin{figure}
\centering
\begin{subfigure}{0.45\textwidth}
\centering
\begin{tikzpicture}[declare function={f(\x)=sqrt(5)/2*sqrt(\x^2-4);}]
\begin{axis}[small,axis lines=middle,xlabel={$x$},ylabel={$y$},xlabel style={at={(current axis.right of origin)},anchor=west},ylabel style={at={(current axis.above origin)},anchor=south},xtick={\empty},ytick={\empty},enlargelimits]
\addplot[domain=2:2.2]{f(x)};
\addplot[domain=2.2:6]{f(x)};
\addplot[domain=2:2.2]{-f(x)};
\addplot[domain=2.2:6]{-f(x)};
\addplot[domain=2:2.2](-x,{f(x)});
\addplot[domain=2.2:6](-x,{f(x)});
\addplot[domain=2:2.2](-x,{-f(x)});
\addplot[domain=2.2:6](-x,{-f(x)});
\end{axis}
\end{tikzpicture}
\caption{}
\end{subfigure}\hfill
\begin{subfigure}{0.45\textwidth}
\centering
\begin{tikzpicture}[declare function={f(\x)=4/3*sqrt(9-\x^2);}]
\begin{axis}[axis equal,small,axis lines=middle,xlabel={$x$},ylabel={$y$},xlabel style={at={(current axis.right of origin)},anchor=west},ylabel style={at={(current axis.above origin)},anchor=south},xtick={\empty},ytick={\empty},enlargelimits]
\addplot[domain=-3:-2.8]{f(x)};
\addplot[domain=-2.8:2.8]{f(x)};
\addplot[domain=2.8:3]{f(x)};
\addplot[domain=-3:-2.8]{-f(x)};
\addplot[domain=-2.8:2.8]{-f(x)};
\addplot[domain=2.8:3]{-f(x)};
\end{axis}
\end{tikzpicture}
\caption{}
\end{subfigure}
\begin{subfigure}{0.45\textwidth}
\centering
\begin{tikzpicture}[declare function={f(\x)=3/4*sqrt(16-\x^2);}]
\begin{axis}[axis equal,small,axis lines=middle,xlabel={$x$},ylabel={$y$},xlabel style={at={(current axis.right of origin)},anchor=west},ylabel style={at={(current axis.above origin)},anchor=south},xtick={\empty},ytick={\empty},enlargelimits]
\addplot[domain=-4:-3.8]{f(x)};
\addplot[domain=-3.8:3.8]{f(x)};
\addplot[domain=3.8:4]{f(x)};
\addplot[domain=-4:-3.8]{-f(x)};
\addplot[domain=-3.8:3.8]{-f(x)};
\addplot[domain=3.8:4]{-f(x)};
\end{axis}
\end{tikzpicture}
\caption{}
\end{subfigure}\hfill
\begin{subfigure}{0.45\textwidth}
\centering
\begin{tikzpicture}[declare function={f(\x)=2/sqrt(5)*sqrt(5+\x^2);}]
\begin{axis}[axis equal,small,axis lines=middle,xlabel={$x$},ylabel={$y$},xlabel style={at={(current axis.right of origin)},anchor=west},ylabel style={at={(current axis.above origin)},anchor=south},xtick={\empty},ytick={\empty},enlargelimits]
\addplot[domain=-4:4]{f(x)};
\addplot[domain=-4:4]{-f(x)};
\end{axis}
\end{tikzpicture}
\caption{}
\end{subfigure}
\caption{ترسیمات برائے سوال \حوالہ{سوال_مخروط_ہم_پلہ_تلاش_الف} تا سوال \حوالہ{سوال_مخروط_ہم_پلہ_تلاش_ب}}
\label{شکل_مخروط_کئی_الف}
\end{figure}

\موٹا{قطع مکافی}\\
سوال \حوالہ{سوال_مخروط_قطع_مکافی_دیا_الف} تا سوال \حوالہ{سوال_مخروط_قطع_مکافی_دیا_ب} میں دیے گئے قطع مکافی کا ماسکہ اور ناظمہ تلاش کرنے کے بعد  اس کو ترسیم کریں۔ ماسکہ اور ناظمہ کو بھی ترسیم میں شامل کریں۔

\ابتدا{سوال}\شناخت{سوال_مخروط_قطع_مکافی_دیا_الف}
$y^2=12x$
\انتہا{سوال}
%=====================
\ابتدا{سوال}
$x^2=6y$
\انتہا{سوال}
%=====================
\ابتدا{سوال}
$x^2=-8y$
\انتہا{سوال}
%=====================
\ابتدا{سوال}
$y^2=-2x$
\انتہا{سوال}
%=====================
\ابتدا{سوال}
$y=4x^2$
\انتہا{سوال}
%=====================
\ابتدا{سوال}
$y=-8x^2$
\انتہا{سوال}
%=====================
\ابتدا{سوال}
$x=-3y^2$
\انتہا{سوال}
%=====================
\ابتدا{سوال}\شناخت{سوال_مخروط_قطع_مکافی_دیا_ب}
$x=2y^2$
\انتہا{سوال}
%=====================

\موٹا{ترخیم}\\
سوال \حوالہ{سوال_مخروط_ترخیم_دیا_الف} تا سوال \حوالہ{سوال_مخروط_ترخیم_دیا_ب} میں دیے گئے ترخیم کی مساوات کو معیاری روپ میں لکھ کر ترسیم کر کے ترسیم پر ماسکہ دکھائیں۔ 

\ابتدا{سوال}\شناخت{سوال_مخروط_ترخیم_دیا_الف}
$16x^2+25y^2=400$
\انتہا{سوال}
%===================
\ابتدا{سوال}
$7x^2+16y^2=112$
\انتہا{سوال}
%=====================
\ابتدا{سوال}
$2x^2+y^2=2$
\انتہا{سوال}
%=====================
\ابتدا{سوال}
$2x^2+y^2=4$
\انتہا{سوال}
%=====================
\ابتدا{سوال}
$3x^2+2y^2=6$
\انتہا{سوال}
%=====================
\ابتدا{سوال}
$9x^2+10y^2=90$
\انتہا{سوال}
%=====================
\ابتدا{سوال}
$6x^2+9y^2=54$
\انتہا{سوال}
%=====================
\ابتدا{سوال}\شناخت{سوال_مخروط_ترخیم_دیا_ب}
$169x^2+25y^2=4225$
\انتہا{سوال}
%=====================

سوال \حوالہ{سوال_مخروط_ترسیم_معلومات_الف} اور سوال \حوالہ{سوال_مخروط_ترسیم_معلومات_ب} میں \عددی{xy} مستوی میں پائے جانے والے  ترخیم کے ماسکہ اور راس کی معلومات دی گئی ہے جس کا مرکز  \عددی{xy} مستوی کے مبدا پر ہے۔ ترخیم کی معیاری مساوات تلاش کریں۔

\ابتدا{سوال}\شناخت{سوال_مخروط_ترسیم_معلومات_الف}
ماسکے \عددی{(\pm \sqrt{2},0)} اور راس \عددی{(\pm 2,0)}
\انتہا{سوال}
%=======================
\ابتدا{سوال}\شناخت{سوال_مخروط_ترسیم_معلومات_ب}
ماسکے \عددی{(0,\pm 4)} اور راس \عددی{(0,\pm 5)}
\انتہا{سوال}
%========================

\موٹا{قطع زائد}\\
سوال \حوالہ{سوال_مخروط_قطع_زائد_معلومات_الف} تا سوال \حوالہ{سوال_مخروط_قطع_زائد_معلومات_ب} میں قطع زائد کی مساواتیں دی گئی ہیں۔ مساوات کو معیاری روپ میں لکھیں اور قطع زائد کا متقارب دریافت کریں۔ قطع زائد کا خاکہ کھینچ کر متقارب اور ماسکہ بھی دکھائیں۔ 

\ابتدا{سوال}\شناخت{سوال_مخروط_قطع_زائد_معلومات_الف}
$x^2-y^2=1$
\انتہا{سوال}
%=========================
\ابتدا{سوال}
$9x^2-16y^2=144$
\انتہا{سوال}
%=================================
\ابتدا{سوال}
$y^2-x^2=8$
\انتہا{سوال}
%=================================
\ابتدا{سوال}
$y^2-x^2=4$
\انتہا{سوال}
%=================================
\ابتدا{سوال}
$8x^2-2y^2=16$
\انتہا{سوال}
%=================================
\ابتدا{سوال}
$y^2-3x^2=3$
\انتہا{سوال}
%=================================
\ابتدا{سوال}
$8y^2-2x^2=16$
\انتہا{سوال}
%=================================
\ابتدا{سوال}\شناخت{سوال_مخروط_قطع_زائد_معلومات_ب}
$64x^2-36y^2=2304$
\انتہا{سوال}
%=================================

سوال \حوالہ{سوال_مخروط_قطع_زائد_معلومات_سے_مساوات_الف} تا سوال \حوالہ{سوال_مخروط_قطع_زائد_معلومات_سے_مساوات_ب} میں \عددی{xy} مستوی پر قطع زائد کے ماسکہ، راس اور متقارب کی معلومات دی گئی ہے۔ قطع زائد کا مرکز \عددی{xy} مستوی کے مبدا پر ہے۔قطع زائد کی معیاری مساوات حاصل کریں۔

\ابتدا{سوال}\شناخت{سوال_مخروط_قطع_زائد_معلومات_سے_مساوات_الف}
ماسکے \عددی{(0,\pm \sqrt{2})} اور متقارب \عددی{y=\pm x}
\انتہا{سوال}
%========================
\ابتدا{سوال}
ماسکے \عددی{(\pm 2,0)} اور متقارب \عددی{y=\pm\tfrac{1}{\sqrt{3}}x}
\انتہا{سوال}
%========================
\ابتدا{سوال}
ماسکے \عددی{(\pm 3,0)} اور متقارب \عددی{y=\pm\tfrac{4}{3}x}
\انتہا{سوال}
%========================
\ابتدا{سوال}\شناخت{سوال_مخروط_قطع_زائد_معلومات_سے_مساوات_ب}
ماسکے \عددی{(0,\pm 2)} اور متقارب \عددی{y=\pm\tfrac{1}{2}x}
\انتہا{سوال}
%========================

\موٹا{مخروطی حصوں کا انتقال}\\
\ابتدا{سوال}
قطع مکافی \عددی{y^2=8x} کو \عددی{2} اکائیاں نیچے اور \عددی{1} اکائی دائیں منتقل کر کے قطع مکافی  \عددی{(y+2)^2=8(x-1)} پیدا کیا جاتا ہے۔ (الف) نئے قطع مکافی کے راس، ماسکہ اور ناظمہ دریافت کریں۔ (ب) نئے راس، ماسکہ اور ناظمہ کو ترسیم کرتے ہوئے نئے قطع مکافی کا خاکہ بنائیں۔
\انتہا{سوال}
%====================
\ابتدا{سوال}
قطع مکافی \عددی{x^2=-4y} کو \عددی{1} اکائی بائیں اور \عددی{3} اکائیاں اوپر منتقل کرتے ہوئے قطع مکافی \عددی{(x+1)^2=-4(y-3)} پیدا کیا جاتا ہے۔   (الف) نئے قطع مکافی کا راس، ماسکہ اور ناظمہ دریافت کریں۔ (ب) نئے راس، ماسکہ اور ناظمہ کو ترسیم کرتے ہوئے نئے قطع مکافی کا خاکہ بنائیں۔
\انتہا{سوال}
%=====================
\ابتدا{سوال}
ترخیم \عددی{\tfrac{x^2}{16}+\tfrac{y^2}{9}=1} کو \عددی{4} اکائیاں دائیں اور \عددی{3} اکائیاں اوپر منتقل کر کے ترخیم \عددی{\tfrac{(x-4)^2}{16}+\tfrac{(y-3)^2}{9}=1} پیدا کیا جاتا ہے۔ (الف) نئے ترخیم کا ماسکے، راس اور مرکز دریافت کریں۔ (ب) نئے ماسکے، راس اور مرکز ترسیم کرتے ہوئے نئے ترخیم کا خاکہ بنائیں۔
\انتہا{سوال}
%=====================
\ابتدا{سوال}
ترخیم \عددی{\tfrac{x^2}{9}+\tfrac{y^2}{25}=1} کو \عددی{3} اکائیاں بائیں اور \عددی{2} اکائیاں نیچے منتقل کر کے
 ترخیم \عددی{\tfrac{(x+3)^2}{9}+\tfrac{(y+2)^2}{25}=1} پیدا کیا جاتا ہے۔ (الف) نئے ترخیم کا ماسکے، راس اور مرکز دریافت کریں۔ (ب) نئے ماسکے، راس اور مرکز ترسیم کرتے ہوئے نئے ترخیم کا خاکہ بنائیں۔
\انتہا{سوال}
%=====================
\ابتدا{سوال}
قطع زائد \عددی{\tfrac{x^2}{16}-\tfrac{y^2}{9}=1} کو \عددی{2} اکائیاں دائیں منتقل کر کے قطع زائد
 \عددی{\tfrac{(x-2)^2}{16}-\tfrac{y^2}{9}=1} پیدا کیا جاتا ہے۔ (الف) نئے قطع زائد کے مرکز، ماسکے، راس اور متقارب دریافت کریں۔ (ب) نئے مرکز، ماسکے، راس اور متقارب ترسیم کرتے ہوئے نئے قطع زائد کا خاکہ بنائیں۔
\انتہا{سوال}
%======================
\ابتدا{سوال}
قطع زائد \عددی{\tfrac{y^2}{4}-\tfrac{x^2}{5}=1} کو \عددی{2} اکائیاں نیچے منتقل کرتے ہوئے قطع زائد \عددی{\tfrac{(y+2)^2}{4}-\tfrac{x^2}{5}=1} پیدا کیا جاتا ہے۔ (الف) نئے قطع زائد کا مرکز، ماسکے اور متقارب دریافت کریں۔ (ب) نیا مرکز، ماسکے اور متقارب ترسیم کر کے نئے قطع زائد کا خاکہ بنائیں۔
\انتہا{سوال}
%===================

سوال \حوالہ{سوال_مخروط_قطع_مکافی_منتقل_الف} تا سوال \حوالہ{سوال_مخروط_قطع_مکافی_منتقل_ب} میں قطع مکافی کی مساوات اور اس کی منتقلی کی معلومات دی گئی ہے۔ نئے قطع مکافی کی مساوات تلاش کر کے نئے قطع مکافی کا راس، ماسکہ اور ناظمہ معلوم کریں۔

\ابتدا{سوال}\شناخت{سوال_مخروط_قطع_مکافی_منتقل_الف}
\عددی{y^2=4x}، 
\quad
\عددی{2} اکائیاں بائیں اور \عددی{3} اکائیاں  نیچے۔
\انتہا{سوال}
%=====================
\ابتدا{سوال}
\عددی{y^2=-12x}، 
\quad
\عددی{4} اکائیاں دائیں اور \عددی{3} اکائیاں  اوپر۔
\انتہا{سوال}
%=====================
\ابتدا{سوال}
\عددی{x^2=8y}، 
\quad
\عددی{1} اکائی دائیں اور \عددی{7} اکائیاں  نیچے۔
\انتہا{سوال}
%=====================
\ابتدا{سوال}\شناخت{سوال_مخروط_قطع_مکافی_منتقل_ب}
\عددی{x^2=6y}، 
\quad
\عددی{3} اکائیاں بائیں اور \عددی{2} اکائیاں  نیچے۔
\انتہا{سوال}
%=====================

سوال \حوالہ{سوال_مخروط_ترخیم_منتقل_الف} تا سوال \حوالہ{سوال_مخروط_ترخیم_منتقل_ب} میں ترخیم کی مساوات اور اس کی منتقلی کی معلومات دی گئی ہے۔ نئے ترخیم کی مساوات تلاش کر کے نئے ترخیم کے ماسکے، راس اور مرکز معلوم کریں۔

\ابتدا{سوال}\شناخت{سوال_مخروط_ترخیم_منتقل_الف}
\عددی{\tfrac{x^2}{6}+\tfrac{y^^2}{9}=1}، 
\quad
\عددی{2} اکائیاں بائیں اور \عددی{1} اکائی نیچے۔
\انتہا{سوال}
%=====================
\ابتدا{سوال}
\عددی{\tfrac{x^2}{2}+y^2=1}، 
\quad
\عددی{3} اکائیاں دائیں اور \عددی{4} اکائیاں اوپر۔
\انتہا{سوال}
%=====================
\ابتدا{سوال}
\عددی{\tfrac{x^2}{3}+\tfrac{y^2}{2}=1}، 
\quad
\عددی{2} اکائیاں دائیں اور \عددی{3} اکائیاں اوپر۔
\انتہا{سوال}
%=====================
\ابتدا{سوال}\شناخت{سوال_مخروط_ترخیم_منتقل_ب}
\عددی{\tfrac{x^2}{16}+\tfrac{y^2}{25}=1}، 
\quad
\عددی{43} اکائیاں بائیں اور \عددی{5} اکائیاں نیچے۔
\انتہا{سوال}
%=====================


سوال \حوالہ{سوال_مخروط_قطع_زائد_منتقل_الف} تا سوال \حوالہ{سوال_مخروط_قطع_زائد_منتقل_ب} میں قطع زائد کی مساوات اور اس کی منتقلی کی معلومات دی گئی ہے۔ نئے قطع زائد کی مساوات تلاش کر کے نئے قطع زائد کا مرکز، ماسکے، راس اور متقارب معلوم کریں۔

\ابتدا{سوال}\شناخت{سوال_مخروط_قطع_زائد_منتقل_الف}
$\tfrac{x^2}{4}-\tfrac{y^2}{5}=1$\quad
دائیں \عددی{2} اکائیاں اور اوپر \عددی{2} اکائیاں
\انتہا{سوال}
%=======================
\ابتدا{سوال}
$\tfrac{x^2}{16}-\tfrac{y^2}{9}=1$\quad
بائیں \عددی{5} اکائیاں اور نیچے \عددی{1} اکائی
\انتہا{سوال}
%=======================
\ابتدا{سوال}
$y^2-x^2=1$\quad
بائیں \عددی{1} اکائیاں اور نیچے \عددی{1} اکائی
\انتہا{سوال}
%=======================
\ابتدا{سوال}\شناخت{سوال_مخروط_قطع_زائد_منتقل_ب}
$\tfrac{y^2}{3}-x^2=1$\quad
دائیں \عددی{1} اکائیاں اور اوپر \عددی{3} اکائیاں
\انتہا{سوال}
%=======================

سوال \حوالہ{سوال_مخروط_مساوات_منتقلی_الف} تا سوال \حوالہ{سوال_مخروط_مساوات_منتقلی_ب} میں دیے گئے مخروط حصوں کا (جیسا مناسب ہو) مرکز، ماسکے، راس، متقارب اور رداس دریافت کریں۔

\ابتدا{سوال}\شناخت{سوال_مخروط_مساوات_منتقلی_الف}
$x^2+4x+y^2=12$
\انتہا{سوال}
%====================
\ابتدا{سوال}
$2x^2+2y^2-28x+12y+144$
\انتہا{سوال}
%====================
\ابتدا{سوال}
$x^2+2x+4y-3=0$
\انتہا{سوال}
%====================
\ابتدا{سوال}
$y^2-4y-8x-12=0$
\انتہا{سوال}
%====================
\ابتدا{سوال}
$x^2+5y^2+4x=1$
\انتہا{سوال}
%====================
\ابتدا{سوال}
$9x^2+6y^2+36y=0$
\انتہا{سوال}
%====================
\ابتدا{سوال}
$x^2+2y^2-2x-4y=-1$
\انتہا{سوال}
%====================
\ابتدا{سوال}
$4x^2+y^2+8x-2y=-1$
\انتہا{سوال}
%====================
\ابتدا{سوال}
$x^2-y^2-2x+4y=4$
\انتہا{سوال}
%====================
\ابتدا{سوال}
$x^2-y^2+4x-6y=6$
\انتہا{سوال}
%====================
\ابتدا{سوال}
$2x^2-y^2+6y=3$
\انتہا{سوال}
%====================
\ابتدا{سوال}\شناخت{سوال_مخروط_مساوات_منتقلی_ب}
$y^2-4x^2+16x=24$
\انتہا{سوال}
%====================

\موٹا{عدم مساوات}\\
سوال \حوالہ{سوال_مخروط_خطے_الف} تا سوال \حوالہ{سوال_مخروط_خطے_ب} میں عدم مساوات یا عدم مساوات کی جوڑی دی گئی ہے۔ \عددی{xy} مستوی میں اس خطہ کو ترسیم کریں۔

\ابتدا{سوال}\شناخت{سوال_مخروط_خطے_الف}
$9x^2+16y^2\le 144$
\انتہا{سوال}
%========================
\ابتدا{سوال}
$x^2+y^2\ge 1,\quad 4x^2+y^2\le 4$
\انتہا{سوال}
%========================
\ابتدا{سوال}
$x^2+4y^2\ge 4,\quad 4x^2+9y^2\le 36$
\انتہا{سوال}
%========================
\ابتدا{سوال}
$(x^2+y^2-4)(x^2+9y^2-9)\le 0$
\انتہا{سوال}
%========================
\ابتدا{سوال}
$4y^2-x^2\ge 4$
\انتہا{سوال}
%========================
\ابتدا{سوال}\شناخت{سوال_مخروط_خطے_ب}
$\abs{x^2-y^2}\le 1$
\انتہا{سوال}
%========================


\موٹا{نظریہ اور مثالیں}

\ابتدا{سوال}\شناخت{سوال_مخروط_قطع_مکافی_حجم}\ترچھا{قطع مکافی ٹھوس جسم کے حجم کا کلیہ آرشمیدسی }\\
قطع مکافی \عددی{y=\tfrac{4h}{b^2}x^2} اور لکیر \عددی{y=h} میں  گھیرے ہوئے خطے کو \عددی{y} محور کے گرد گھما کر جسم طواف پیدا کیا جاتا ہے۔ دکھائیں کہ اس جسم کا حجم مطابقتی مخروط کے حجم کا\عددی{\tfrac{3}{2}} گنّا ہو گا (شکل \حوالہ{شکل_سوال_مخروط_قطع_مکافی_حجم})۔ 
\انتہا{سوال}
%=======================


\begin{figure}
\centering
\begin{minipage}{0.45\textwidth}
\centering
\begin{tikzpicture}[yscale=0.75,declare function={f(\x)=\x^2;}]
\draw[-latex](0,0)--(2.5,0)node[right]{$x$};
\draw[-latex](0,{f(2)})--++(0,0.75)node[above]{$y$};
\draw[thick,domain=0:2]plot ({\x},{f(\x)});
\draw[domain=0:2]plot ({-\x},{f(\x)});
\draw (0,4)circle (2cm and 0.25cm);
\draw(-2,4)--(0,0)--(2,4)coordinate[pos=0.5](ka);
\draw(ka)node[pin={[pin distance=1cm]30:{مخروط}}]{};
\draw(1.25,{f(1.25)})node[right]{$y=\frac{4h}{b^2}x^2$};
\draw(2,{f(2)})node[right]{$(\tfrac{b}{2},h)$};
\end{tikzpicture}
\caption{جسم طواف برائے سوال \حوالہ{سوال_مخروط_قطع_مکافی_حجم}}
\label{شکل_سوال_مخروط_قطع_مکافی_حجم}
\end{minipage}\hfill
\begin{minipage}{0.45\textwidth}
\centering
\begin{tikzpicture}[declare function={f(\x)=sqrt(\x);}]
\begin{axis}[clip=false,small, axis lines=middle,xtick={\empty},ytick={\empty},xlabel={$x$},ylabel={$y$},xlabel style={at={(current axis.right of origin)},anchor=west},ylabel={$y$},ylabel style={at={(current axis.above origin)},anchor=south},enlargelimits=true]
\addplot[domain=0:0.5]{f(x)};
\addplot[domain=0.5:4]{f(x)}node[pos=0.5,below right]{$y^2=kx$};
\addplot[]plot coordinates {(3.75,{f(3.75)})}node[circ]{}node[below right]{$N$};
\addplot[]plot coordinates {(3.75,{f(3.75)})(3.75,0)};
\addplot[]plot coordinates {(3.75,{f(3.75)})(0,{f(3.75)})};
\addplot[]plot coordinates {(2,0.75)}node[]{$B$}  {(1,1.5)}node[]{$A$};
\end{axis}
\end{tikzpicture}
\caption{خطے برائے سوال \حوالہ{سوال_مخلوط_قطع_مکافی_خطے}}
\label{شکل_سوال_مخلوط_قطع_مکافی_خطے}
\end{minipage}
\end{figure}

\ابتدا{سوال}\ترچھا{معلق پل کی رسیاں قطع مکافی کی صورت میں لٹکی ہوتی ہیں۔}\\
ایک معلق پل کی کمیت \عددی{m} کلو گرام فی میٹر ہے۔ اس پل کو رسیوں سے لٹکایا گیا ہے۔ اگر مبدا پر رسی کا افقی تناو \عددی{H} ہو تب رسی کی منحنی درج ذیل مساوات کو مطمئن کرتی ہے۔
\begin{align*}
\frac{\dif y}{\dif x}=\frac{mg}{H}x
\end{align*}
اس تفرقی مساوات کو حل کرتے ہوئے دکھائیں کہ رسی کی منحنی کی مساوات ایک قطع مکافی ہے۔ \عددی{x=0} پر \عددی{y=0} ابتدائی معلومات ہے۔
\انتہا{سوال}
%========================
\ابتدا{سوال}
نقاط \عددی{(1,0)}، \عددی{(0,1)} اور \عددی{(2,2)} سے گزرتے دائرے کی مساوات دریافت کریں۔
\انتہا{سوال}
%=======================
\ابتدا{سوال}
نقاط \عددی{(2,3)}، \عددی{(3,2)} اور \عددی{(-4,3)} سے گزرتے دائرے کی مساوات دریافت کریں۔
\انتہا{سوال}
%=======================
\ابتدا{سوال}
ایک دائرہ جس کا مرکز \عددی{(-2,1)} پر ہے نقطہ \عددی{(1,3)} سے گزرتا ہے۔  کیا نقطہ \عددی{(1.1,2.8)} اس دائرے پر، اس کے اندر یا اس کے باہر پایا جاتا ہے؟
\انتہا{سوال}
%====================
\ابتدا{سوال}
جہاں دائرہ \عددی{(x-2)^2+(y-1)^2=5} محددی محوروں کو قطع کرتا ہے وہاں اس دائرے کے مماس معلوم کریں۔ 
\انتہا{سوال}
%===================
\ابتدا{سوال}\شناخت{سوال_مخلوط_قطع_مکافی_خطے}
قطع مکافی \عددی{y^2=kx,\, k>0} پر نقطہ \عددی{N} سے محددی محور کے متوازی لکیریں کھینچی جاتی ہیں۔ ان لکیروں اور محددی محوروں کے کے بیچ مستطیل خطہ کو قطع مکافی دو حصوں \عددی{A} اور \عددی{B} میں تقسیم کرتا ہے (شکل \حوالہ{شکل_سوال_مخلوط_قطع_مکافی_خطے})۔ (الف) دکھائیں کہ ان خطوں کو \عددی{y} محور کے گرد گھما کر حاصل اجسام طواف کے حجم کی نسبت \عددی{4:1} ہے۔ (ب) ان خطوں کو \عددی{x} محور کے گرد گھما کر حاصل اجسام طواف کے حجم کی نسبت کیا ہو گی؟ 
\انتہا{سوال}
%===================
\ابتدا{سوال}
دکھائیں کہ لکیر \عددی{x=-p} پر  کسی بھی نقطہ سے منحنی \عددی{y^2=4px} پر دو مماس، آپس میں عمودی ہوں گے۔  
\انتہا{سوال}
%=================
\ابتدا{سوال}
ترخیم \عددی{x^2+4y^2=4} میں محصور زیادہ سے زیادہ رقبے کے مستطیل کے اضلاع معلوم کریں۔مستطیل کے اضلاع محددی محور کے متوازی ہیں۔ 
\انتہا{سوال}
%====================
\ابتدا{سوال}
ترخیم \عددی{9x^2+4y^2=36} کو (الف) \عددی{x} محور، (ب) \عددی{y} محور کے گرد گھما کر جسم طواف پیدا کیا جاتا ہے۔ اس کا حجم معلوم کریں۔
\انتہا{سوال}
%=====================
\ابتدا{سوال}
ربع اول میں \عددی{x} محور، لکیر \عددی{x=4} اور قطع زائد \عددی{9x^2-4y^2=36} کے بیچ تکونی خطہ کو \عددی{x} محور کے گرد گھما کر جسم طواف پیدا کیا جاتا ہے۔  اس جسم کا حجم تلاش کریں۔
\انتہا{سوال}
%====================
\ابتدا{سوال}
ایک خطہ کا بایاں سرحد محور \عددی{y}، دایاں سرحد قطع زائد \عددی{x^2-y^2=1}  جبکہ اس کا نچلا اور بالائی سرحد لکیر \عددی{y=\pm 3} ہیں۔ اس خطہ کو \عددی{y} محور کے گرد گھما کر جسم طواف پیدا کیا جاتا ہے۔ اس جسم کا حجم تلاش کریں۔ 
\انتہا{سوال}
%=======================
\ابتدا{سوال}
محور \عددی{x} کے بالائی اور ترخیم \عددی{\tfrac{x^2}{9}+\tfrac{y^2}{16}=1} کے نیچے  خطے کا وسطانی مرکز تلاش کریں۔ 
\انتہا{سوال}
%================
\ابتدا{سوال}
قطع زائد \عددی{y^2-x^2=1} کے بالائی شاخ \عددی{y=\sqrt{x^2+1},\, 0\le x\le \sqrt{2}} کو \عددی{x} محور کر گرد گھما کر سطح طواف پیدا کیا جاتا ہے۔ اس سطح کا رقبہ تلاش کریں۔
\انتہا{سوال}
%==================
\ابتدا{سوال}\شناخت{سوال_مخروط_امواج}
پانی کی سطح کو پہلے \عددی{A} اور بعد میں \عددی{B} پر چھو کر شکل \حوالہ{شکل_سوال_مخروط_امواج} میں دکھائے گئے امواج پیدا کئے گئے۔ جیسے جیسے یہ امواج پھیلتے ہیں، ان کا نقطہ قطع ایک منحنی بناتا ہے جو قطع زائد کی طرح معلوم ہوتا ہے۔ کیا ایسا حقیقتاً ہو گا؟  یہ جاننے کے کئے ہم \عددی{A} اور \عددی{B} پر مرکز دائروں کو امواج کا نمونہ لے سکتے ہیں۔

لمحہ \عددی{t} پر نقطہ \عددی{N} مرکز \عددی{A} سے \عددی{r_A(t)}  اور \عددی{B} سے \عددی{r_B(t)} فاصلہ پر ہو گا۔ چونکہ دائروں کے رداس ایک مستقل رفتار (موج کی رفتار) سے بڑھتے ہیں لہٰذا \عددی{\tfrac{\dif r_A}{\dif t}=\tfrac{\dif r_B}{\dif t}} ہو گا۔ اس سے اخذ کریں کہ
 \عددی{r_A-r_B} ایک مستقل ہو گا لہٰذا \عددی{N} اس قطع زائد پر پایا جائے گا جس کے ماسکہ \عددی{A} اور \عددی{B} ہیں۔
\انتہا{سوال}
%============
\begin{figure}
\centering
\begin{minipage}{0.45\textwidth}
\centering
\begin{tikzpicture}
\draw[name path=a] (0,0)node[circ]{}node[below]{$A$} circle (1.5);
\draw[name path=b] (0.75,0)node[circ]{}node[below]{$B$} circle (1.25);
\draw[name intersections={of={a and b}}](0,0)--(intersection-1)node[circ]{}node[above]{$N(t)$}node[pos=0.5,shift={(-1ex,1ex)},font=\scriptsize]{$r_A(t)$};
\draw(0.75,0)--(intersection-1)node[pos=0.4,xshift=1ex,,fill=white,font=\scriptsize]{$r_B(t)$};
\end{tikzpicture}
\caption{امواج برائے سوال \حوالہ{سوال_مخروط_امواج}}
\label{شکل_سوال_مخروط_امواج}
\end{minipage}\hfill
\begin{minipage}{0.45\textwidth}
\centering
\begin{tikzpicture}[font=\scriptsize,declare function={f(\x)=sqrt(\x);ft(\x)=1/2*(\x+1);}]
\pgfmathsetmacro{\B}{atan(0.5)}
\pgfmathsetmacro{\P}{atan(1/0.75)}
\pgfmathsetmacro{\D}{\P-\B}
\draw[-latex](0,0)--(2,0)node[right]{$x$};
\draw[-latex](0,0)--(0,1.5)node[above]{$y$};
\draw[domain=0:2]plot (\x,{ft(\x)});
\draw[domain=0.2:2]plot (\x,{f(\x)});
\draw[domain=0:0.2]plot(\x,{f(\x)});
\draw[domain=0:0.2]plot(\x,{-f(\x)});
\draw[domain=0.2:2]plot (\x,{-f(\x)});
\draw[]plot coordinates {(0.25,0)(1,{f(1)})};
\draw[]plot coordinates {(0,{f(1)})(2,{f(1)})}node[right]{$L'$};
\draw([shift={(0:0.7)}]1,1) arc (0:\B:0.7);
\draw(1,1)++(0.9,0.15)node[]{$\beta$};
\draw(1.75,{-f(1.25)})node[above]{$y^2=4px$};
\draw(0.25,0)node[circ]{}node[below,xshift=2ex]{$F(p,0)$};
\draw(1,1)node[above,xshift=-2ex]{$N(x_0,y_0)$};
\draw(2,{ft(2)})node[above]{$L$};
\draw(1,1)--(1,0)node[pos=0.5,right]{$y_0$};
\draw([shift={(0:0.3)}]0.25,0) arc (0:\P:0.3);
\draw(20:0.7)node[]{$\phi$};
\draw([shift={(180:0.5)}]1,1) arc (180:180+\D:0.5);
\draw(1,1)++(-0.6,-0.15)node[]{$\beta$};
\draw([shift={(180+\B:0.7)}]1,1) arc (180+\B:180+2*\B:0.7);
\draw(1,1)++(-0.6,-0.6)node[]{$\alpha$};
\end{tikzpicture}
\caption{قطع مکافی میں انعکاس (سوال \حوالہ{سوال_مخروط_قطع_مکافی_انعکاس})}
\label{شکل_سوال_مخروط_قطع_مکافی_انعکاس}
\end{minipage}
\end{figure}


\ابتدا{سوال}\شناخت{سوال_مخروط_قطع_مکافی_انعکاس}\ترچھا{قطع مکافی کے خواص انعکاس}\\
قطع مکافی \عددی{y^2=4px} پر عمومی نقطہ \عددی{N(x_0,y_0)} کو شکل \حوالہ{شکل_سوال_مخروط_قطع_مکافی_انعکاس} میں دکھایا گیا ہے۔ نقطہ \عددی{N} پر لکیر \عددی{L} اس قطع مکافی کا مماس ہے۔ قطع مکافی کا ماسکہ \عددی{F(p,0)} ہے۔نقطہ \عددی{N} سے دائیں منعکس شعاع \عددی{L'}، محور \عددی{x} کے متوازی  ہے۔ ہم دکھاتے ہیں کہ \عددی{F} سے خارج، \عددی{N} پر پہنچتا شعاع انعکاس کے بعد \عددی{L'} کا ہم مکان ہو گا۔ یہ دکھانے کی خاطر ہم دکھاتے ہیں کہ \عددی{\beta=\alpha} ہو گا۔ اس مساوات کی تصدیق درج ذیل اقدام کے ذریعہ کریں۔
\begin{enumerate}[a.]
\item
دکھائیں کہ \عددی{\tan \beta=\tfrac{2p}{y_0}} ہو گا۔
\item
دکھائیں کہ \عددی{\tan \phi=\tfrac{y_0}{x_0-p}} ہو گا۔
\item
درج ذیل مماثل
\begin{align*}
\tan \alpha=\frac{\tan \phi-\tan\beta}{1+\tan\phi\tan\beta}
\end{align*}
استعمال کرتے ہوئے دکھائیں کہ \عددی{\tan\alpha=\tfrac{2p}{y_0}} ہو گا۔چونکہ \عددی{\alpha} اور \عددی{\beta} دونوں زاویہ حادہ ہیں لہٰذا
 \عددی{\tan\beta=\tan\alpha} یعنی \عددی{\beta=\alpha} ہو گا۔ 
\end{enumerate}  
\انتہا{سوال}
%======================
