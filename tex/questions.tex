\جزوحصہء{سوالات}
\موٹا{مقامی انتہا کی تلاش}\\
سوال \حوالہ{سوال_کثیرالمتغیر_تمام_زیادہ_کم_زین_الف} تا سوال  \حوالہ{سوال_کثیرالمتغیر_تمام_زیادہ_کم_زین_ب} میں تفاعل کے  تمام مقامی زیادہ سے زیادہ قیمت کے نقاط، مقامی کم سے کم قیمت کے نقاط  اور نقاط زین تلاش کریں۔

\ابتدا{سوال}\شناخت{سوال_کثیرالمتغیر_تمام_زیادہ_کم_زین_الف}
$f(x,y)=x^2+xy+y^2+3x-3y+4$
\انتہا{سوال}
%==================
\ابتدا{سوال}
$f(x,y)=x^2+3xy+3y^2-6x+3y-6$
\انتہا{سوال}
%====================
\ابتدا{سوال}
$f(x,y)=2xy-5x^2-2y^2+4x+4y-4$
\انتہا{سوال}
%====================
\ابتدا{سوال}
$f(x,y)=2xy-5x^2-2y^2+4x-4$
\انتہا{سوال}
%====================
\ابتدا{سوال}
$f(x,y)=x^2+xy+3x+2y+5$
\انتہا{سوال}
%====================
\ابتدا{سوال}
$f(x,y)=y^2+xy-2x-2y+2$
\انتہا{سوال}
%====================
\ابتدا{سوال}
$f(x,y)=5x-7x^2+3x-6y+2$
\انتہا{سوال}
%====================
\ابتدا{سوال}
$f(x,y)=2xy-x^2-2y^2+3x+4$
\انتہا{سوال}
%====================
\ابتدا{سوال}
$f(x,y)=x^2-4xy+y^2+6y+2$
\انتہا{سوال}
%====================
\ابتدا{سوال}
$f(x,y)=3x^2+6xy+7y^2-2x+4y$
\انتہا{سوال}
%====================
\ابتدا{سوال}
$f(x,y)=2x^2+3xy+4y^2-5x+2y$
\انتہا{سوال}
%====================
\ابتدا{سوال}
$f(x,y)=4x^2-6xy+5y^2-20x+26y$
\انتہا{سوال}
%====================
\ابتدا{سوال}
$f(x,y)=x^2-y^2-2x+4y+6$
\انتہا{سوال}
%====================
\ابتدا{سوال}
$f(x,y)=x^2-2xy+2y^2-2x+2y+1$
\انتہا{سوال}
%====================
\ابتدا{سوال}
$f(x,y)=x^2+2xy$
\انتہا{سوال}
%====================
\ابتدا{سوال}
$f(x,y)=3+2x+2y-2x^2-2xy-y^2$
\انتہا{سوال}
%====================
\ابتدا{سوال}
$f(x,y)=x^3-y^3-2xy+6$
\انتہا{سوال}
%====================
\ابتدا{سوال}
$f(x,y)=x^3+3xy+y^3$
\انتہا{سوال}
%====================
\ابتدا{سوال}
$f(x,y)=6x^2-2x^3+3y^2+6xy$
\انتہا{سوال}
%====================
\ابتدا{سوال}
$f(x,y)=3y^2-2y^3-3x^2+6xy$
\انتہا{سوال}
%====================
\ابتدا{سوال}
$f(x,y)=9x^3+\tfrac{y^3}{3}-4xy$
\انتہا{سوال}
%====================
\ابتدا{سوال}
$f(x,y)=8x^3+y^3+6xy$
\انتہا{سوال}
%====================
\ابتدا{سوال}
$f(x,y)=x^3+y^3+3x^2-3y^2-8$
\انتہا{سوال}
%====================
\ابتدا{سوال}
$f(x,y)=2x^3+2y^3-9x^2+3y^2-12y$
\انتہا{سوال}
%====================
\ابتدا{سوال}
$f(x,y)=4xy-x^4-y^4$
\انتہا{سوال}
%====================
\ابتدا{سوال}
$f(x,y)=x^4+y^4+4xy$
\انتہا{سوال}
%====================
\ابتدا{سوال}
$f(x,y)=\frac{1}{x^2+y^2-1}$
\انتہا{سوال}
%====================
\ابتدا{سوال}
$f(x,y)=\frac{1}{x}+xy+\frac{1}{y}$
\انتہا{سوال}
%====================
\ابتدا{سوال}
$f(x,y)=y\sin x$
\انتہا{سوال}
%====================
\ابتدا{سوال}\شناخت{سوال_کثیرالمتغیر_تمام_زیادہ_کم_زین_ب}
$f(x,y)=e^{2x}\cos y$
\انتہا{سوال}
%====================

\موٹا{مطلق انتہا کی تلاش}\\
سوال \حوالہ{سوال_کثیرالمتغیر_مطلق_انتہا_الف} تا سوال \حوالہ{سوال_کثیرالمتغیر_مطلق_انتہا_ب} میں تفاعل کی مطلق انتہا تلاش کریں۔

\ابتدا{سوال}\شناخت{سوال_کثیرالمتغیر_مطلق_انتہا_الف}
ربع اول میں بند تکون، جس کے  سرحد   \عددی{x=0}، \عددی{y=2} اور \عددی{y=2x} ہیں، میں تفاعل  \عددی{f(x,y)=2x^2-4x+y^2-4y+1} ہے۔
\انتہا{سوال}
%=================
\ابتدا{سوال}
ربع اول میں بند تکون، جس کے اطراف \عددی{x=0}، \عددی{y=4} اور \عددی{y=x} ہیں،  میں تفاعل \عددی{f(x,y)=x^2-xy+y^2+1} ہے۔
\انتہا{سوال}
%===================
\ابتدا{سوال}
ربع اول میں بند تکون، جس کے اطراف \عددی{x=0}، \عددی{y=0} اور \عددی{y+2x=2} ہیں، میں تفاعل \عددی{f(x,y)=x^2+y^2} ہے۔ 
\انتہا{سوال}
%===========
\ابتدا{سوال}
مستطیل پٹی \عددی{0\le x\le 5,\, -3\le y\le 3} پر تفاعل \عددی{T(x,y)=x^2+xy+y^2-6x} ہے۔
\انتہا{سوال}
%==================
\ابتدا{سوال}
مستطیل \عددی{0]le x\le 5,\, -3\le y\le 0} پر تفاعل \عددی{T(x,y)=x^2+xy+y^2-6x+2} ہے۔
\انتہا{سوال}
%=================
\ابتدا{سوال}
مستطیل \عددی{0\le x\le 1,\, 0\le y\le 1} پر تفاعل \عددی{f(x,y)=48xy-32x^3-24y^2} ہے۔
\انتہا{سوال}
%=================
\ابتدا{سوال}
مستطیل \عددی{1\le x\le 3,\, -\tfrac{\pi}{4}\le \tfrac{\pi}{4}} پر تفاعل \عددی{f(x,y)=(4x-x^2)\cos y} ہے۔
\انتہا{سوال}
%===============
\ابتدا{سوال}\شناخت{سوال_کثیرالمتغیر_مطلق_انتہا_ب}
 تفاعل \عددی{f(x,y)=4x-8xy+2y+1}،   اضلاع \عددی{x=0}، \عددی{y=0} اور \عددی{x+y=1} میں بند خطہ میں ہے۔
\انتہا{سوال}
%===================
\ابتدا{سوال}
دو ایسے  اعداد \عددی{a} اور \عددی{b}، جہاں \عددی{a\le b} ہے،  تلاش کریں تا کہ درج ذیل کی قیمت زیادہ سے زیادہ ہو۔
\begin{align*}
\int_a^b(6-x-x^2)\dif x
\end{align*}
\انتہا{سوال}
%====================
\ابتدا{سوال}
دو ایسے  اعداد \عددی{a} اور \عددی{b}، جہاں \عددی{a\le b} ہے،  تلاش کریں تا کہ درج ذیل کی قیمت زیادہ سے زیادہ ہو۔
\begin{align*}
\int_a^b(24-2x-x^2)^{1/3}\dif x
\end{align*}
\انتہا{سوال}
%====================
\ابتدا{سوال}\ترچھا{درجہ حرارت}\\
ایک دائری پٹی \عددی{x^2+y^2\le 1}  اور اس کی سرحد \عددی{x^2+y^2=1} کو یوں گرم کیا جاتا ہے کہ نقطہ \عددی{(x,y)} پر درجہ حرارت \عددی{T(x,y)x^2+2y^2-x} ہو۔اس پٹی پر زیادہ سے زیادہ اور کم سے کم درجہ حرارت  تلاش کریں۔
\انتہا{سوال}
%=============
\ابتدا{سوال}
کھلا  ربع اول \عددی{x>0,\, y>0} میں  \عددی{f(x,y)=xy+2x-\ln x^2y} کا نقطہ  فاصل تلاش کریں اور دکھائیں کہ اس نقطہ پر تفاعل کی قیمت کم سے کم ہو گی۔
\انتہا{سوال}
%==========

\موٹا{نظریہ اور مثالیں}\\
\ابتدا{سوال}
درج ذیل معلومات استعمال کرتے ہوئے    زیادہ سے زیادہ قیمت کے نقاط، کم سے کم قیمت کے نقاط اور نقاط زین، اگر موجود ہوں، تلاش کریں۔
\begin{enumerate}[a.]
\item
$f_x=2x-4y,\, f_y=2y-4x$
\item
$f_x=2x-2,\, f_y=2y-4$
\item
$f_x=9x^2-9,\, f_y=2y+4$
\end{enumerate}
ہر جواب کی وجہ بیان کریں۔
\انتہا{سوال}
%===============
\ابتدا{سوال}
درج ذیل تفاعل کے لئے مبدا پر ممیز \عددی{f_{xx}f_{yy}-f_{xy}^2}   صفر ہے لہٰذا دو رتبی تفرقی پرکھ غیر  فیصلہ کن ہو گا۔ مبدا پر سطح \عددی{z=f(x,y)} کی ذہنی  تصویر کشی کرتے ہوئے  دریافت کریں کہ مبدا پر زیادہ سے زیادہ قیمت کا نقطہ، کم سے کم قیمت کا نقطہ یا نقطہ زین پایا جاتا ہے۔ ہر جواب کی وجہ پیش کریں۔
\begin{multicols}{2}
\begin{enumerate}[a.]
\item
$f(x,y)=x^2y^2$
\item
$f(x,y)=1-x^2y^2$
\item
$f(x,y)=xy^2$
\item
$f(x,y)=x^3y^2$
\item
$f(x,y)=x^3y^3$
\item
$f(x,y)=x^4y^4$
\end{enumerate}
\end{multicols}
\انتہا{سوال}
%============
\ابتدا{سوال}
دکھائیں کہ \عددی{k}  کی ہر قیمت کے لئے  \عددی{(0,0)} تفاعل \عددی{f(x,y)=x^2+kxy+y^2} کا نقطہ فاصل ہو گا۔ (اشارہ: دو صورتوں پر غور کریں: \عددی{k=0} اور \عددی{k\ne 0})
\انتہا{سوال}
%==================
\ابتدا{سوال}
مستقل \عددی{k} کی کن قیمتوں کے لئے دو رتبی تفرقی پرکھ ضمانت دیتا ہے کہ  \عددی{(0,0)} پر \عددی{f(x,y)=x^2+kxy+y^2} کا (ا)  نقطہ زین (ب) مقامی کم سے کم قیمت  کا نقطہ   پایا جائے گا؟  مستقل \عددی{k} کی کن قیمتوں کے لئے دو رتبی تفرقی پرکھ غیر  فیصلہ کن ہو گا؟ اپنے جوابات کی وجہ پیش کریں۔
\انتہا{سوال}
%==============
\ابتدا{سوال}
(ا) کیا \عددی{f_x(a,b)=f_y(a,b)=0} ہوتے ہوئے ہر   صورت \عددی{(a,b)} پر \عددی{f} کا مقامی زیادہ سے ی زیادہ قیمت کا نقطہ یا کم سے کم قیمت کا نقطہ پایا جائے گا؟ اپنے جواب کی وجہ پیش کریں۔   (ب)  اگر ایک قرص  میں ، جس کا مرکز  \عددی{(a,b)} ہو،  ہر نقطہ پر \عددی{f}  اور اس کے یک رتبی اور دو رتبی جزوی تفرقات استمراری ہوں، اور \عددی{f_{xx}(a,b)} اور \عددی{f_{yy}(a,b)} کی علامتیں ایک دوسرے سے مختلف ہوں  تب کیا \عددی{f} کے بارے میں کچھ کہنا ممکن ہو گا؟ اپنے جواب کی وجہ پیش کریں۔
\انتہا{سوال}
%=================
\ابتدا{سوال}
نقطہ \عددی{(a,b)} پر  \عددی{f} کا   مقامی  زیادہ سے زیادہ قیمت کا نقطہ ہونے کی صورت میں  مسئلہ \حوالہ{مسئلہ_کثیرالمتغیر_مقامی_انتہائی_قیمت_یک_رتبی_تفرقی_پرکھ} کا دیا گیا  ثبوت استعمال کرتے ہوئے  اس مسئلہ کو \عددی{(a,b)} پر مقامی کم سے کم قیمت کا نقطہ ہونے کی صورت کے لئے ثابت کریں۔
\انتہا{سوال}
%================
\ابتدا{سوال}
مستوی \عددی{x+2y+3z=0}  سے   زیادہ بلندی  پر  \عددی{z=10-x^2-y^2} کی ترسیم کے تمام نقاط میں وہ نقطہ تلاش کریں جو مستوی سے دور ترین ہو۔
\انتہا{سوال}
%=================
\ابتدا{سوال}
مستوی \عددی{x+2y-z=0} سے \عددی{z=x^2+y^2+10} کی ترسیم کا قریب ترین نقطہ تلاش کریں۔
\انتہا{سوال}
%===============
\ابتدا{سوال}
بند ربع اول \عددی{x\ge 0,\, y\ge 0} میں تفاعل \عددی{f(x,y)=x+y} کی کوئی مطلق زیادہ سے زیادہ قیمت نہیں پائی جاتی ہے۔ کیا  اس حقیقت میں اور    کتاب میں  مطلق انتہا کی تلاش پر کی گئی گفتگو  میں تضاد پایا جاتا ہے؟ اپنے جواب کی وجہ پیش کریں۔
\انتہا{سوال}
%============
\ابتدا{سوال}
مربع \عددی{0\le x\le 1,\, 0\le y\le 1} میں تفاعل \عددی{f(x,y)=x^2+y^2+2xy-x-y+1} پر غور کریں۔
\begin{enumerate}[a.]
\item
دکھائیں کہ  اس مربع میں خطی قطع \عددی{2x+2y=1} پر \عددی{f} کی مطلق کم سے کم  قیمت پائی جاتی ہے۔ اس کم سے کم قیمت کو تلاش کریں۔
\item
مربع پر \عددی{f} کی مطلق زیادہ سے زیادہ قیمت تلاش کریں۔
\end{enumerate}
\انتہا{سوال}
%==============

\موٹا{مقدار معلوم منحنیات پر انتہائی قیمتیں}\\
