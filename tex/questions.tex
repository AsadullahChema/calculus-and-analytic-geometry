\حصہء{سوالات}
\موٹا{مستوی \عددی{xy} میں حرکت}\\
سوال \حوالہ{سوال_سمتی_تفاعل_مقام_سے_سمتی_رفتار_اسراع_الف} تا سوال \حوالہ{سوال_سمتی_تفاعل_مقام_سے_سمتی_رفتار_اسراع_ب} میں مستوی \عددی{xy} میں لمحہ \عددی{t} پر  ایک ذرے کا مقام \عددی{\kvec{r}(t)} ہے۔ اس ذرے کی راہ کی ترسیم کے \عددی{x} اور \عددی{y} محدد کی مساواتیں تلاش کریں۔ اس کے بعد دیے گئے لمحہ  پر ذرے کی سمتی رفتار اور اسراع سمتیات دریافت کریں۔

\ابتدا{سوال}\شناخت{سوال_سمتی_تفاعل_مقام_سے_سمتی_رفتار_اسراع_الف}
$\kvec{r}(t)=(t+1)\ai+(t^2-1)\aj,\quad t=1$
\انتہا{سوال}
%===================
\ابتدا{سوال}
$\kvec{r}(t)=(t^2+1)\ai+(2t-1)\aj,\quad t=\frac{1}{2}$
\انتہا{سوال}
%===================
\ابتدا{سوال}
$\kvec{r}(t)=e^t\ai+\frac{2}{9}e^{2t}\aj,\quad t=\ln 3$
\انتہا{سوال}
%===================
\ابتدا{سوال}\شناخت{سوال_سمتی_تفاعل_مقام_سے_سمتی_رفتار_اسراع_ب}
$\kvec{r}(t)=(\cos 2t)\ai+(3\sin 2t)\aj,\quad t=0$
\انتہا{سوال}
%===================

سوال \حوالہ{سوال_سمتی_تفاعل_تعین_گر_سمتیہ_الف} تا سوال \حوالہ{سوال_سمتی_تفاعل_تعین_گر_سمتیہ_ب} میں مستوی \عددی{xy} میں مختلف منحنیات پر حرکت کرتے ہوئے  ایک ذرے کا  تعین گر سمتیہ  دیا گیا ہے۔دیے گئے لمحات  پر اس ذرے کے  سمتی رفتار اور اسراع کے سمتیات دریافت کریں۔ ان سمتیات کو منحنی پر ترسیم کریں۔

\ابتدا{سوال}\شناخت{سوال_سمتی_تفاعل_تعین_گر_سمتیہ_الف}\ترچھا{دائرہ \عددی{x^2+y^2=1} پر حرکت}\\
$\kvec{r}(t)=(\sin t)\ai+(\cos t)\aj,\quad t=\frac{\pi}{4},\frac{\pi}{2}$
\انتہا{سوال}
%==================
\ابتدا{سوال}\ترچھا{دائرہ \عددی{x^2+y^216} پر حرکت}\\
$\kvec{r}(t)=(4\cos\tfrac{t}{2})\ai+(4\sin\tfrac{t}{2})\aj,\quad t=\pi,\frac{3\pi}{2}$
\انتہا{سوال}
%=================
\ابتدا{سوال}\ترچھا{تدویر \عددی{x=t-\sin t,\, y=1-\cos t} پر حرکت}\\
$\kvec{r}(t)=(t-\sin t)\ai+(1-\cos t)\aj,\quad t=\pi,\frac{3\pi}{2}$
\انتہا{سوال}
%===================
\ابتدا{سوال}\شناخت{سوال_سمتی_تفاعل_تعین_گر_سمتیہ_ب}\ترچھا{قطع مکافی \عددی{y=x^2+1} پر حرکت}\\
$\kvec{r}(t)=t\ai+(t^2+1)\aj,\,\, t=-1,0,1$
\انتہا{سوال}
%================

\موٹا{فضا میں سمتی رفتار اور اسراع}\\
سوال \حوالہ{سوال_سمتی_تفاعل_رفتار_رخ_الف} تا سوال \حوالہ{سوال_سمتی_تفاعل_رفتار_رخ_ب} میں لمحہ \عددی{t} پر ایک ذرے کا تعین گر سمتیہ \عددی{\kvec{r}(t)} ہے۔اس ذرے کی سمتی رفتار اور اسراع تلاش کریں۔ دئے گئے لمحہ  پر اس کی   رفتار اور رخ کی قیمت تلاش کریں۔اس لمحہ پر ذرے کی سمتی رفتار کو رفتار اور رخ کا حاصل ضرب لکھیں۔

\ابتدا{سوال}\شناخت{سوال_سمتی_تفاعل_رفتار_رخ_الف}
$\kvec{r}(t)=(t+1)\ai+(t^2-1)\aj+2t\ak,\quad t=1$
\انتہا{سوال}
%===================
\ابتدا{سوال}
$\kvec{r}(t)=(1+t)\ai+\frac{t^2}{\sqrt{2}}\aj+\frac{t^2}{3}\ak,\quad t=1$
\انتہا{سوال}
%==================
\ابتدا{سوال}
$\kvec{r}(t)=(2\cos t)\ai+(3\sin t)\aj+4t\ak,\quad t=\frac{\pi}{2}$
\انتہا{سوال}
%====================
\ابتدا{سوال}
$\kvec{r}(t)=(\sec t)\ai+(\tan t)\aj+\frac{4}{3}t\ak,\quad t=\frac{\pi}{6}$
\انتہا{سوال}
%=================
\ابتدا{سوال}
$\kvec{r}(t)=(2\ln(t+1))\ai+t^2\aj+\frac{t^2}{2}\ak,\quad t=1$
\انتہا{سوال}
%====================
\ابتدا{سوال}\شناخت{سوال_سمتی_تفاعل_رفتار_رخ_ب}
$\kvec{r}(t)=(e^{-t})\ai+(2\cos 3t)\aj+(2\sin 3t)\ak,\quad t=0$
\انتہا{سوال}
%====================

سوال \حوالہ{سوال_سمتی_رفتار_زاویہ_الف} تا سوال \حوالہ{سوال_سمتی_رفتار_زاویہ_ب} میں  لمحہ \عددی{t} پر فضا میں ایک ذرے کا تعین گر سمتیہ \عددی{\kvec{r}(t)} ہے۔لمحہ \عددی{t=0} پر اس کی سمتی رفتار اور اسراع کے بیچ زاویہ تلاش کریں۔

\ابتدا{سوال}\شناخت{سوال_سمتی_رفتار_زاویہ_الف}
$\kvec{r}(t)=(3t+1)\ai+\sqrt{3}t\aj+t^2\ak$
\انتہا{سوال}
%===================
\ابتدا{سوال}
$\kvec{r}(t)=(\tfrac{\sqrt{2}}{2})\ai+(\tfrac{\sqrt{2}}{2}t-16t^2)\aj$
\انتہا{سوال}
%========================
\ابتدا{سوال}
$\kvec{r}(t)=(\ln(t^2+1))\ai+(\tan^{-1}t)\aj+\sqrt{t^2+1}\ak$
\انتہا{سوال}
%========================
\ابتدا{سوال}\شناخت{سوال_سمتی_رفتار_زاویہ_ب}
$\kvec{r}(t)=\tfrac{4}{9}(1+t)^{3/2}\ai+\tfrac{4}{9}(1-t)^{3/2}\aj+\tfrac{1}{3}t\ak$
\انتہا{سوال}
%========================

سوال \حوالہ{سوال_سمتی_تفاعل_عمودی_رفتار_اسراع_الف} اور سوال \حوالہ{سوال_سمتی_تفاعل_عمودی_رفتار_اسراع_ب} میں لمحہ \عددی{t} پر فضا میں ایک  ذرے کا تعین گر سمتیہ \عددی{\kvec{r}(t)} ہے۔دیے گئے وقفہ میں وہ لمحہ یا لمحات تلاش کریں جن پر سمتی رفتار سمتیہ اور اسراع سمتیہ ایک دوسرے کے عمودی  ہوں گے۔

\ابتدا{سوال}\شناخت{سوال_سمتی_تفاعل_عمودی_رفتار_اسراع_الف}
$\kvec{r}(t)=(t-\sin t)\ai+(1-\cos t)\aj,\quad 0\le t\le 2\pi$
\انتہا{سوال}
%================
\ابتدا{سوال}\شناخت{سوال_سمتی_تفاعل_عمودی_رفتار_اسراع_ب}
$\kvec{r}(t)=(\sin t)\ai+t\aj+(\cos t)\ak,\quad t\ge 0$
\انتہا{سوال}
%=====================

\موٹا{سمتی  قیمت تفاعل کا تکمل}\\
سوال \حوالہ{سوال_سمتی_تفاعل_سمتی_قیمت_الف} تا سوال \حوالہ{سوال_سمتی_تفاعل_سمتی_قیمت_ب} میں تکمل حاصل کریں۔

\ابتدا{سوال}\شناخت{سوال_سمتی_تفاعل_سمتی_قیمت_الف}
$\int_0^1[t^3\ai+7\aj+(t+1)\ak]\dif t$
\انتہا{سوال}
%=====================
\ابتدا{سوال}
$\int/-1^2\big[(6-6t)\ai+3\sqrt{t}\aj+(\tfrac{4}{t^2})\ak\big]\dif t$
\انتہا{سوال}
%==========================
\ابتدا{سوال}
$\int_{-\pi/4}^{\pi/4}[(\sin t)\ai+(1+\cos t)\aj+(\sec^2 t)\ak]\dif t$
\انتہا{سوال}
%==========================
\ابتدا{سوال}
$\int_0^{\pi/3}[(\sec t\tan t)\ai+(\tan t)\aj+(2\sin t\cos t)\ak]\dif t$
\انتہا{سوال}
%==========================
\ابتدا{سوال}
$\int_1^4[\tfrac{1}{t}\ai+\tfrac{1}{5-t}\aj+\tfrac{1}{2t}\ak]\dif t$
\انتہا{سوال}
%==========================
\ابتدا{سوال}\شناخت{سوال_سمتی_تفاعل_سمتی_قیمت_ب}
$\int_0^1[\tfrac{2}{\sqrt{1-t^2}}\ai+\tfrac{\sqrt{3}}{1+t^2}\ak]\dif t$
\انتہا{سوال}
%==========================

\موٹا{سمتی تفاعل کے ابتدائی قیمت مسائل}\\
سوال \حوالہ{سوال_سمتی_تفاعل_ابتدائی_قیمت_مسئلہ_الف} تا سوال \حوالہ{سوال_سمتی_تفاعل_ابتدائی_قیمت_مسئلہ_ب} میں \عددی{t} کے سمتی تفاعل \عددی{\kvec{r}} کے ابتدائی قیمت مسائل دیے گئے ہیں۔ انہیں حل کریں۔

\ابتدا{سوال}\شناخت{سوال_سمتی_تفاعل_ابتدائی_قیمت_مسئلہ_الف}
\begin{align*}
\frac{\dif \kvec{r}}{\dif t}&=-t\ai-t\aj-t\ak&&\text{\RL{تفرقی مساوات}}\\
\kvec{r}(0)&=\ai+2\aj+3\ak&&\text{\RL{ابتدائی شرط}}
\end{align*}
\انتہا{سوال}
%===================
\ابتدا{سوال}
\begin{align*}
\frac{\dif \kvec{r}}{\dif t}&=(180t)\ai+(180t-16t^2)\aj&&\text{\RL{تفرقی مساوات}}\\
\kvec{r}(0)&=100\aj&&\text{\RL{ابتدائی شرط}}
\end{align*}
\انتہا{سوال}
%===================
\ابتدا{سوال}
\begin{align*}
\frac{\dif \kvec{r}}{\dif t}&=\tfrac{3}{2}(t+1)^{1/2}\ai+e^{-t}\aj+\tfrac{1}{t+1}\ak&&\text{\RL{تفرقی مساوات}}\\
\kvec{r}(0)&=\ak&&\text{\RL{ابتدائی شرط}}
\end{align*}
\انتہا{سوال}
%===================
\ابتدا{سوال}
\begin{align*}
\frac{\dif \kvec{r}}{\dif t}&=(t^3+4t)\ai+t\aj+2t^2\ak&&\text{\RL{تفرقی مساوات}}\\
\kvec{r}(0)&=\ai+\aj&&\text{\RL{ابتدائی شرط}}
\end{align*}
\انتہا{سوال}
%===================
\ابتدا{سوال}
\begin{align*}
\frac{\dif ^{\,2}\kvec{r}}{\dif t^2}&=-32\ak&&\text{\RL{تفرقی مساوات}}\\
\kvec{r}(0)&=100\ak&&\text{\RL{ابتدائی شرائط}}\\
\left. \tfrac{\dif \kvec{r}}{\dif t}\right\vert_{t=0}&=8\ai+8\aj
\end{align*}
\انتہا{سوال}
%===================
\ابتدا{سوال}\شناخت{سوال_سمتی_تفاعل_ابتدائی_قیمت_مسئلہ_ب}
\begin{align*}
\frac{\dif ^{\,2}\kvec{r}}{\dif t^2}&=-(\ai+\aj+\ak)&&\text{\RL{تفرقی مساوات}}\\
\kvec{r}(0)&=10\ai+a0\aj+a0\ak&&\text{\RL{ابتدائی شرائط}}\\
\left. \tfrac{\dif \kvec{r}}{\dif t}\right\vert_{t=0}&=0
\end{align*}
\انتہا{سوال}
%=====================

\موٹا{ہموار منحنیات کے مماسی خط}\\
جیسا متن میں بتایا گیا ہے، ہموار منحنی \عددی{\kvec{r}(t)=f(t)\ai+g(t)\aj+h(t)\ak} کا \عددی{t=t_0}  پر مماسی خط نقطہ \عددی{(f(t_0),g(t_0),h(t_0))} سے گزرتا ہے اور،  \عددی{t_0} پر  اس منحنی کے سمتی رفتار سمتیہ \عددی{\kvec{v}(t_0)}،  کا متوازی ہوتا ہے۔ سوال \حوالہ{سوال_سمتی_تفاعل_مماس_الف} تا سوال \حوالہ{سوال_سمتی_تفاعل_مماس_ب} میں \عددی{t=t_0} پر  دیے گئے منحنی کے مماسی خط کی مقدار معلوم مساوات حاصل کریں۔

\ابتدا{سوال}\شناخت{سوال_سمتی_تفاعل_مماس_الف}
$\kvec{r}(t)=(\sin t)\ai+(t^2-\cos t)\aj+e^t\ak,\quad t_0=0$
\انتہا{سوال}
%===============
\ابتدا{سوال}
$\kvec{r}(t)=(2\sin t)\ai+(2\cos t)\aj+5t\ak,\quad t_0=4\pi$
\انتہا{سوال}
%===============
\ابتدا{سوال}
$\kvec{r}(t)=(a\sin t)\ai+(a\cos t)\aj+bt\ak,\quad t_0=2\pi$
\انتہا{سوال}
%===============
\ابتدا{سوال}\شناخت{سوال_سمتی_تفاعل_مماس_ب}
$\kvec{r}(t)=(\cos t)\ai+(\sin t)\aj+(\sin 2t)\ak,\quad t_0=\tfrac{\pi}{2}$
\انتہا{سوال}
%===============

\موٹا{دائری راہ پر حرکت}\\
\ابتدا{سوال}
اکائی دائرہ \عددی{x^2+y^2=1} پر ایک ذرہ کے  حرکت کو  (ا) تا (د) میں دی گئی  مساوات ظاہر کرتی ہیں۔اگرچہ (ا) تا (د) میں ذرے کا  راہ ایک ہے، ان راہ پر  اس کا حرکی رویہ مختلف ہے۔ ہر راہ پر درج ذیل کے جوابات دیں۔
\begin{enumerate}[1.]
\item
کیا ذرے کی رفتار مستقل ہے؟ اگر ایسا ہو، تب اس کی رفتار کتنی ہے؟
\item
کیا ذرے کا سمتی رفتار سمتیہ اور اسراع آپس میں ہر جگہ  عمودی ہیں؟
\item
کیا یہ ذرہ اکائی دائرے پر گھڑی کے رخ   یا اس کے مخالف رخ گھومتا ہے؟ 
\item
کیا ذرہ نقطہ \عددی{(1,0)} سے ابتدا کرتا ہے؟
\end{enumerate}  
\begin{enumerate}[a.]
\item
$\kvec{r}(t)=(\cos t)\ai+(\sin t)\aj,\quad t\ge 0$
\item
$\kvec{r}(t)=\cos (2t)\ai+\sin(2t)\aj,\quad t\ge 0$
\item
$\kvec{r}(t)=(\cos t)\ai-(\sin t)\aj,\quad t\ge 0$
\item
$\kvec{r}(t)=\cos(t^2)\ai+\sin(t^2)\aj,\quad t\ge 0$
\end{enumerate}
\انتہا{سوال}
%===========
\ابتدا{سوال}
دکھائیں کہ درج ذیل ابتدائی قیمت سمتی قیمت تفاعل، مستوی \عددی{x+y-2z=2}  میں  رداس \عددی{1} کے دائرہ پر حرکت کو ظاہر کرتا ہے جہاں دائرے کا مرکز \عددی{(2,2,1)} ہے۔
\begin{align*}
\kvec{r}(t)=(2\ai+2\aj+\ak)+\cos t(\tfrac{1}{\sqrt{2}}\ai-\tfrac{1}{\sqrt{2}}\aj)+\sin t(\tfrac{1}{\sqrt{3}}\ai+\tfrac{1}{\sqrt{3}}\aj+\tfrac{1}{\sqrt{3}}\ak)
\end{align*} 
\انتہا{سوال}
%==================

\موٹا{خط مستقیم پر حرکت}\\
\ابتدا{سوال}
لمحہ \عددی{t=0} پر ایک ذرہ نقطہ \عددی{(1,2,3)} پر واقع ہے۔ یہ خط مستقیم پر حرکت کرتا ہوا نقطہ \عددی{(4,1,4)} پہنچتا ہے۔ اس کا رفتار \عددی{(1,2,3)} پر \عددی{2} اور  اس کی اسراع مستقل \عددی{3\ai-\aj+\ak} ہے۔ لمحہ \عددی{t} پر اس کا تعین گر سمتیہ \عددی{\kvec{r}(t)} دریافت کریں۔
\انتہا{سوال}
%================
\ابتدا{سوال}
لمحہ \عددی{t=0} پر ایک ذرہ نقطہ \عددی{(1,-1,2)} پر پایا جاتا ہے  اور  اس کا رفتار \عددی{2} ہے۔ یہ نقطہ \عددی{(3,0,3)} کی طرف یکساں اسراع  \عددی{2\ai+\aj+\ak}سے بڑھتا ہے۔ لمحہ \عددی{t} پر اس کا تعین گر سمتیہ \عددی{\kvec{r}(t)} تلاش کریں۔
\انتہا{سوال}
%=================

\موٹا{نظریہ اور مثالیں}\\
\ابتدا{سوال}
ایک ذرہ قطع مکافی \عددی{y^2=2x} کے بالائی حصہ پر بائیں سے دائیں رخ، \عددی{5} اکائیاں فی سیکنڈ کے  مستقل رفتار   سے  حرکت کرتا ہے۔اس ذرہ کی سمتی رفتار اس لمحہ  پر تلاش کریں جب یہ نقطہ \عددی{(2,2)} سے گزرتا ہے۔
\انتہا{سوال}
%===================
\ابتدا{سوال}
ایک ذرہ  مستوی \عددی{xy} میں ایک  تدویر پر یوں حرکت کرتا ہے کہ لمحہ \عددی{t} اس کا تعین گر سمتیہ 
\begin{align*}
\kvec{r}(t)=(t-\sin t)\ai+(1-\cos t)\aj
\end{align*}
ہوتا ہے۔ \عددی{\abs{\kvec{r}}} اور \عددی{\abs{\kvec{a}}} کی کم سے کم اور زیادہ سے زیادہ قیمتیں تلاش کریں۔(اشارہ: پہلے \عددی{\abs{\kvec{v}}^2} اور \عددی{\abs{\kvec{a}}^2} کی انتہائی قیمتیں تلاش کریں اور بعد میں جذر لیں۔)
\انتہا{سوال}
%=============
\ابتدا{سوال}
ایک ذرہ  مستوی \عددی{yz} میں ترخیم \عددی{\tfrac{y^2}{9}+\tfrac{z^2}{4}=1} پر یوں  حرکت  کرتا ہے کہ لمحہ \عددی{t} پر اس کا  تعین گر سمتیہ 
\begin{align*}
\kvec{r}(t)=(3\cos t)\aj+(2\sin t)\ak
\end{align*}
ہوتا ہے۔ \عددی{\abs{\kvec{r}}} اور \عددی{\abs{\kvec{a}}} کی کم سے کم اور زیادہ سے زیادہ قیمتیں تلاش کریں۔ (بالائی سوال میں اشارہ  دیکھیں۔)
\انتہا{سوال}
%====================
\ابتدا{سوال}\ترچھا{مصنوعی سیارہ کی دائری  حرکت}\\
ایک مصنوعی سیارہ جس کی کمیت \عددی{m} ہے ایک جسم جس کی کمیت \عددی{M} ہے کے گرد دائری مدار  پر  مستقل رفتار \عددی{v} سے  طواف کرتا ہے۔دائری مدار کا رداس \عددی{r_0} ہے۔ اس مصنوعی سیارہ کے مدار کا    دوری عرصہ  \عددی{T} (ایک چکر کے لئے درکار وقت)   درج ذیل اقدام کے ذریعہ تلاش کریں۔
\begin{enumerate}[a.]
\item
کمیت \عددی{M} کے جسم کو مبدا پر   اور لمحہ \عددی{t=0} پر   مصنوعی سیارہ کو محور  \عددی{x} پر رکھیں۔حرکت کو گھڑی کے رخ تصور کریں۔لمحہ \عددی{t} پر سیارہ کا تعین گر سمتیہ \عددی{\kvec{r}(t)} لیں۔ دکھائیں کہ \عددی{\theta=\tfrac{vt}{r_0}} ہو گا  لہٰذا درج ذیل  ہو گا۔
\begin{align*}
\kvec{r}(t)=(r_0\cos\tfrac{vt}{r_0})\ai+(r_0\sin\tfrac{vt}{r_0})\aj
\end{align*}
\item
سیارے کی اسراع معلوم کریں۔
\item
نیوٹن کے قانون تجاذب  کے تحت سیارہ پر قوت درج ذیل ہو گی جہاں \عددی{G}  تجاذب کا عالمگیر مستقل ہے۔
\begin{align*}
\kvec{F}=\big(-\frac{GMm}{r_0^2}\big)\frac{\kvec{r}}{r_0}
\end{align*}
نیوٹن کے دوسرے قانون سے \عددی{\kvec{F}=m\kvec{a}} ہو گا  جس سے \عددی{v^2=\tfrac{GM}{r_0}} حاصل کریں۔ 
\item
دکھائیں کہ \عددی{T} درج ذیل کو مطمئن کرتا ہے۔
\begin{align*}
vT=2\pi r_0
\end{align*}
\item
جزو ج اور جزو د سے درج ذیل حاصل کریں جو دوری عرصہ کا مربع ہے۔
\begin{align*}
T^2=\frac{4\pi^2}{GM}r_0^3
\end{align*}
\end{enumerate}
\انتہا{سوال}
%=================
