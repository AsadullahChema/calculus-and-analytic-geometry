\حصہ{طاقتی تسلسل کے استعمال}
اس حصہ میں ثنائی تسلسل متعارف کرایا جائے گا جو طاقت اور جذر کا اندازہ  کرنے میں مددگار ثابت ہوتا ہے۔ مزید ابتدائی قیمت مسئلے کے حل کو تخمیناً تسلسل سے ظاہر کرنا اور غیر بنیادی تکمل کے قیمت کے حصول میں تسلسل کا کردار دکھایا جائے گا۔ ایسے حد جو غیر معین صورت دیتے ہوں کا حل بھی سکھایا جائے گا۔  ہم \عددی{\tan^{-1}x} کے مکلارن تسلسل کا ایک مختصر طریقہ دکھائیں گے اور بار بار استعمال ہونے والے تسلسلوں کی جدول کا ذکر کریں گے۔

\جزوحصہء{طاقت اور جذر کے لئے ثنائی تسلسل} 
تفاعل \عددی{f(x)=(1+x)^m} جہاں \عددی{m} مستقل ہے، کا مکلارن تسلسل درج ذیل ہے
\begin{multline}\label{مساوات_تسلسل_ثنائی_الف}
1+mx+\frac{m(m-1)}{2!}x^2+\frac{m(m-1)(m-2)}{3!}x^3\\
+\frac{m(m-1)(m-2)\cdots (m-k+1)}{k!}x^k+\cdots
\end{multline}
جس کو \اصطلاح{ثنائی تسلسل}\فرہنگ{ثنائی!تسلسل}\فرہنگ{تسلسل!ثنائی}\حاشیہب{binomial series}\فرہنگ{binomial!series} کہتے ہیں اور جو \عددی{\abs{x}<1} کے لئے مطلق مرتکز ہے۔ یہ تسلسل حاصل کرنے کی خاطر ہم تفاعل اور اس کے تفرقات لکھتے ہیں:
\begin{align*}
f(x)&=(1+x)^m\\
f'(x)&=m(1+x)^{m-1}\\
f''(x)&=m(m-1)(1+x)^{m-2}\\
f'''(x)&=m(m-1)(m-2)(m-3)(1+x)^{m-3}\\
\vdots\\
f^{(k)}(x)&=m(m-1)(m-2)\cdots(m-k+1)(1+x)^{m-k}
\end{align*}
نقطہ  \عددی{x=0} پر ان کی قیمتیں دریافت کر کے مکلارن تسلسل کے کلیہ میں پر کرتے ہوئے مساوات \حوالہ{مساوات_تسلسل_ثنائی_الف} کا تسلسل حاصل ہو گا۔

اگر \عددی{m} عدد صحیح ہو جو صفر یا اس سے بڑا ہو تب \عددی{k=m+1} عددی سر سے تمام عددی سر صفر ہوں گے لہٰذا \عددی{(m+1)} اجزاء کے بعد یہ تسلسل رک جاتا ہے۔

اگر \عددی{m} صفر یا مثبت عدد صحیح نہ ہو تب یہ تسلسل لامتناہی اجزاء پر مشتمل ہو گا جو \عددی{\abs{x}<1} کے لئے مرتکز ہو گا۔ اس کی وجہ دیکھنے کی خاطر فرض کریں \عددی{u_k} وہ جزو ہے جس میں \عددی{x^k}  پایا جاتا ہو۔ اب مطلق ارتکاز کے تناسبی پرکھ سے آپ دیکھ سکتے ہیں کہ درج ذیل ہو گا۔ 
\begin{align*}
\abs{\frac{u_{k+1}}{u_k}}&=\abs{\frac{m-k}{k+1}x}\to \abs{x}&&k\to \infty
\end{align*}

ثنائی تسلسل کا حصول ہمیں صرف اتنا بتاتا ہے کہ  \عددی{(1+x)^m} اس کو پیدا کرتا ہے اور \عددی{\abs{x}<1} کے لئے یہ تسلسل مرتکز ہے۔ تسلسل کا حصول ہمیں یہ نہیں دکھاتا ہے کہ یہ تسلسل \عددی{(1+x)^m} کو مرکوز ہے۔ حقیقت میں یہ تسلسل \عددی{(1+x)^m} کو مرکوز ہے، جس کا ثبوت پیش نہیں کیا جائے گا۔

\begin{gather}
\begin{aligned}\label{مساوات_تسلسل_ثنائی_علامتیں_الف}
(1+x)^m&=1+\sum_{k=1}^{\infty}\binom{m}{k}x^k,&&-1<x<1\\
\binom{m}{1}&=m,\quad \binom{m}{2}=\frac{m(m-1)}{2!} &&\text{جہاں}\\
\binom{m}{k}&=\frac{m(m-1)(m-2)\cdots(m-k+1)}{k!}&&\text{\RL{کے لئے}}\quad k\ge 3 \quad \text{اور}
\end{aligned}
\end{gather}
ہوں گے۔

\ابتدا{مثال}
اگر \عددی{m=-1} ہو تب
\begin{align*}
\binom{-1}{1}&=-1,\quad \binom{-1}{2}=\frac{(-1)(-2)}{2!}=1,\\
\binom{-1}{k}&=\frac{(-1)(-2)(-3)\cdots(-1-k+1)}{k!}=(-1)^k\binom{k!}{k!}=(-1)^k
\end{align*}
ہوں گے اور مساوات \حوالہ{مساوات_تسلسل_ثنائی_علامتیں_الف} درج ذیل ثنائی تسلسل دے گی۔
\begin{align*}
(1+x)^{-1}=1+\sum_{k=1}^{\infty}(-1)^kx^k=1-x+x^2-x^3+\cdots+(-1)^k x^k+\cdots
\end{align*}
\انتہا{مثال}
%========================
\ابتدا{سوال}
ہم مثال \حوالہ{مثال_استعمال_تخمینی_صورت_الف} سے جانتے ہیں کہ  چھوٹے \عددی{\abs{x}} کے لئے  \عددی{\sqrt{1+x}\approx 1+\tfrac{x}{2}} ہو گا۔ثنائی تسلسل میں \عددی{m=\tfrac{1}{2}} لیتے ہوئے دو قدری اور بلند رتبی تخمین حاصل ہوتے ہیں، اور ساتھ ہی اندازہ خلل بھی حاصل ہوتا ہے جو مسئلہ بدلتے تسلسل کا اندازہ خلل  دیتا ہے:
\begin{align*}
(1+x)^{1/2}&=1+\frac{x}{2}+\frac{(\tfrac{1}{2})(-\tfrac{1}{2})}{2!}x^2+\frac{(\tfrac{1}{2})(-\tfrac{1}{2})(-\tfrac{3}{2})}{3!}x^3\\
&\quad\quad\quad\quad\quad\quad\quad+\frac{(-\tfrac{1}{2})(-\tfrac{1}{2})(-\tfrac{3}{2})(-\tfrac{5}{2})}{4!}x^4+\cdots\\
&=1+\frac{x}{2}-\frac{x^2}{8}+\frac{x^3}{16}-\frac{5x^4}{128}+\cdots
\end{align*}
دیگر تخمین \عددی{x} کی مختلف قیمتیں پر کرتے ہوئے حاصل ہوں گی۔ مثال کے طور پر:
\begin{align*}
\sqrt{1-x}&\approx 1-\frac{x^2}{2}-\frac{x^4}{8} &&\text{\RL{چھوٹے \عددی{\abs{x^2}} کے لئے}}\\
\sqrt{1-\frac{1}{x}}&\approx 1-\frac{1}{2x}-\frac{1}{8x^2}&&\text{\RL{چھوٹا \عددی{\abs{\tfrac{1}{x}}} یعنی بڑا \عددی{\abs{x}}}}
\end{align*}
\انتہا{سوال}
%=========================
\جزوحصہء{تفرقی مساوات کے طاقتی تسلسل حل اور ابتدائی قیمت مسائل}
