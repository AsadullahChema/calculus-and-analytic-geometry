\حصہ{مستوی میں مسئلہ گرین}
ہم اب ایک ایسا مسئلہ پیش کرتے ہیں جو مستوی خطہ کی سرحد کی ہمراہ یا اس کو عبور کرتے ہوئے داب نا پذیر سیال کی بہاو اور خطہ کے اندر اس کی حرکت  کے بیچ تعلق پیش کرتا ہے۔ پھیلاو  اور گردش  کے تصورات  سیال کے سرحدی رویہ اور اندرونی رویہ کے بیچ تعلق پیش کرنا ممکن بناتے ہیں۔ سیال کی سمتی رفتاری میدان  کا پھیلاو کسی نقطہ پر  خطہ میں سیال کے دخول یا خروج کی ناپ ہے جبکہ گردش اس نقطہ پر سیال کے گھومنے کی شرح کی ناپ ہے۔

مسئلہ گرین کہتا ہے کہ، چند ایسے شرائط مطمئن ہونے کی صورت میں جو عملی استعمال میں عموماً پورے ہوتے ہیں، مستوی خطہ کی سرحد سے خارجی بہاو سرحد کے اندر  میدان کے  پھیلاو کے  دوہرا تکمل کے برابر ہو گا۔  اس مسئلے کا دوسرا روپ کہتا ہے کہ خطہ کی سرحد کی ہمراہ خلاف گھڑی دائری بہاو اس خطہ میں میدان کی گردش کے دوہرا تکمل کے برابر ہو گا۔ 

مسئلہ گرین، علم احصاء کے عظیم مسائل میں سے ایک ہے۔ یہ گہرا اور حیرت کن ہے اور اس کے  دور رس نتائج  پائے جاتے ہیں۔ خالص ریاضیات میں مسئلہ گرین کی اہمیت، احصاء کے بنیادی مسئلہ کے برابر ہے۔ عملی ریاضیات میں مسئلہ گرین کا تین ابعادی روپ برقی، مقناطیسی، اور سیالی حرکیات کے مسائل کا بنیاد مہیا کرتا ہے۔  

ہم سیال کی حرکت کی سمتی رفتاری میدان  کی بات اس لئے کرتے ہیں کہ سیال کی حرکت کا ذہنی خاکہ بنانا آسان ہوتا ہے۔ یاد رہے کہ مسئلہ گرین کسی بھی سمتی میدان، جو چند شرائط کو مطمئن کرتا ہو، کے لئے درست ہو گا۔ اس کی درستگی میدان کے کسی خصوصی طبیعی خاصیت کے ہونے  پر منحصر نہیں ہے۔  

\جزوحصہء{نقطہ پر کثافت بہاو: پھیلاو}
مسئلہ گرین کے لئے ہمیں دو نئے تصورات کی ضرورت پیش آتی ہے۔ پہلا تصور، ایک نقطہ پر سمتی میدان کی کثافت بہاو ہے جس کو ریاضیات میں سمتی میدان کا \ترچھا{پھیلاو} کہتے ہیں۔ اس کو حاصل کرنے کا طریقہ درج ذیل ہے۔

فرض کریں مستوی میں سمتی میدانی کا سیالی بہاو \عددی{\kvec{F}(x,y)=M(x,y)\ai+N(x,y)\aj} ہے  اور خطہ \عددی{R} کے ہر نقطہ پر \عددی{M} اور \عددی{N} کے یک رتبی جزوی تفرقات استمراری ہیں۔ خطہ \عددی{R} میں \عددی{(x,y)} ایک نقطہ ہے اور \عددی{S} ایک ایسا چھوٹا مستطیل ہے جو مکمل طور پر \عددی{R} میں پایا جاتا ہے اور جس کا ایک راس \عددی{(x,y)} ہے۔ اس مستطیل کے اطراف محددی محور کے متوازی ہیں اور ان کی لمبائیاں \عددی{\Delta x} اور \عددی{\Delta y} ہیں۔ مستطیل کی نچلی سرحد کو عبور کرتا ہوا خارجی سیال کی شرح تخمیناً نقطہ \عددی{(x,y)} پر باہر رخ عمودی سمتی رفتار کے غیر سمتی جزو ضرب لمبائی قطع کے برابر ہو گی:
\begin{align}
\kvec{F}(x,y)\cdot(-\aj)\Delta x=-N(x,y)\Delta x
\end{align}
یوں اگر سمتی رفتار کی اکائی میٹر فی سیکنڈ ہو تب خارجی شرح کی اکائی میٹر فی سیکنڈ ضرب میٹر یعنی مربع میٹر فی سیکنڈ ہو گی۔ باقی تین اطراف کے عمودی باہر رخ خارجی سیال کی شرح بھی اسی طرح حاصل کی جا سکتی ہیں۔ چاروں اطراف کے نتائج کو یہاں پیش کرتے ہیں۔
\begin{gather}
\begin{aligned}
\kvec{F}(x,y+\Delta y)\cdot \aj\Delta x&=N(x,y+\Delta y)\Delta x&&\text{\RL{بالائی}}\\
\kvec{F}(x,y)\cdot (-\aj)\Delta x&=-N(x,y)\Delta x&&\text{\RL{نچلی}}\\
\kvec{F}(x+\Delta x,y)\cdot \ai\Delta y&=M(x+\Delta x,y)\Delta y&&\text{\RL{دائیں}}\\
\kvec{F}(x,y)\cdot (-\ai)\Delta y&=-M(x,y)\Delta y&&\text{\RL{بائیں}}
\end{aligned}
\end{gather}
مخالف اضلاع کی شرح کے مجموعات درج ذیل ہوں گے۔
\begin{align}
[N(x,y+\Delta y)-N(x,y)]\Delta x\approx \big(\frac{\partial N}{\partial y}\Delta y\big)\Delta x&&\text{\RL{نچلی اور بالائی}}\label{مساوات_سمتی_تکمل_نچلی_بالائی}\\
[M(x+\Delta x,y)-M(x,y)]\Delta y\approx \big(\frac{\partial M}{\partial x}\Delta x\big)\Delta y&&\text{\RL{دائیں اور بائیں}}\label{مساوات_سمتی_تکمل_دایاں_بایاں}
\end{align}
مساوات \حوالہ{مساوات_سمتی_تکمل_نچلی_بالائی} اور مساوات \حوالہ{مساوات_سمتی_تکمل_دایاں_بایاں} کا مجموعہ کل اخراج دیگا:
\begin{align}
\text{\RL{مستطیل کی سرحد کو عبور کرتا ہوا بہاو}}\approx \big(\frac{\partial M}{\partial x}+\frac{\partial N}{\partial y}\big)\Delta x\Delta y
\end{align}
ہم اب \عددی{\Delta x\Delta y} سے تقسیم کر کے بہاو فی اکائی رقبہ  یعنی بہاو کی کثافت حاصل کرتے ہیں۔
\begin{align*}
\frac{\text{\RL{مستطیل کی سرحد کو عبور کرتا ہوا بہاو}}}{\text{\RL{مستطیل کا رقبہ}}}\approx \big(\frac{\partial M}{\partial x}+\frac{\partial N}{\partial y}\big)
\end{align*}
آخر میں ہم \عددی{\Delta x} اور \عددی{\Delta y} کو صفر تک پہنچا کر نقطہ \عددی{(x,y)} پر \عددی{\kvec{F}} کے \ترچھا{کثافت بہاو} کی تعریف اخذ کرتے ہیں۔

ریاضیات میں ہم  کثافت بہاو کو \عددی{\kvec{F}} کا \ترچھا{پھیلاو} کہتے ہیں۔میدان \عددی{\kvec{F}} کے پھیلاو کو ہم پھیلاو \عددی{\kvec{F}} لکھتے ہیں۔

\ابتدا{تعریف}
نقطہ \عددی{(x,y)} پر سمتی میدان \عددی{\kvec{F}=M\ai+N\aj} کا کثافت بہاو یا \اصطلاح{پھیلاو}\فرہنگ{پھیلاو}\حاشیہب{divergence}\فرہنگ{divergence}  درج ذیل ہو گا۔
\begin{align}\label{مساوات_سمتی_تکمل_پھیلاو_کی_تعریف}
\text{\RL{پھیلاو \(\kvec{F}\)}}=\frac{\partial M}{\partial x}+\frac{\partial N}{\partial y}
\end{align} 
\انتہا{تعریف}
%====================

اگر نقطہ \عددی{(x_0,y_0)} پر ایک باریک سوراخ سے سیال ایک خطہ میں داخل ہو تب اس سوراخ سے سیال پھیلے گا جس کی بنا اس کو یہی نام دیا گیا ہے۔ چھوٹے مستطیل میں نقطہ \عددی{(x_0,y_0)} پر سوراخ سے سیال داخل ہونے کی صورت میں پھیلاو مثبت ہو گا جبکہ خطہ سے سیال کی اخراج کی صورت میں پھیلاو کی قیمت منفی ہو گی۔  

\ابتدا{مثال}
سمتی میدان \عددی{\kvec{F}(x,y)=(x^2-y)\ai+(xy-y^2)\aj} کا پھیلاو تلاش کریں۔ 

حل:\quad
ہم مساوات \حوالہ{مساوات_سمتی_تکمل_پھیلاو_کی_تعریف} استعمال کرتے ہیں۔
\begin{align*}
\text{\RL{پھیلاو \(\kvec{F}\)}}&=\frac{\partial M}{\partial x}+\frac{\partial N}{\partial y}=\frac{\partial}{\partial x}(x^2-y)+\frac{\partial}{\partial y}(xy-y^2)\\
&=2x-x-2y=3x-2y
\end{align*}
\انتہا{مثال}
%====================

\جزوحصہء{ایک نقطہ پر کثافت دائری بہاو}
مسئلہ گرین کے لئے درکار دوسرا نیا تصور ایک نقطہ پر سمتی میدان \عددی{\kvec{F}} کی کثافت دائری بہاو ہے جس کو ریاضیات میں \عددی{\kvec{F}} کی \ترچھا{گردش} کہتے ہیں۔ اس کو حاصل کرنے کی خاطر ہم وہی سمتی میدان
\begin{align*}
\kvec{F}(x,y)=M(x,y)\ai+N(x,y)\aj
\end{align*}
اور نقطہ \عددی{(x,y)} پر چھوٹا مستطیل رقبہ \عددی{S} لیتے ہیں۔ اس مستطیل کو یہاں دوبارہ پیش کیا گیا ہے۔

رقبہ \عددی{S} کے گرد خلاف گھڑی \عددی{\kvec{F}} کا دائری بہاو مستطیل \عددی{S} کی اطراف کی ہمراہ  بہاو کی شرح کا مجموعہ ہو گا۔اکائی مماسی سمتیہ \عددی{\ai} کے رخ سمتی رفتار \عددی{\kvec{F}} کا غیر سمتی جزو ضرب لمبائی قطع تخمیناً نچلے ضلع کی ہمراہ بہاو کے برابر ہو گا: 
\begin{align}
\kvec{F}(x,y)\cdot\ai\Delta x=M(x,y)\Delta x
\end{align}
باقی اطراف کی ہمراہ خلاف گھڑی  بہاو اسی طرح حاصل کی جا سکتی ہیں۔چاروں اضلاع کے نتائج درج ذیل ہوں گے۔
\begin{gather}
\begin{aligned}
\kvec{F}(x,y+\Delta y)\cdot (-\ai)\Delta x&=-M(x,y+\Delta y)\Delta x&&\text{\RL{بالائی}}\\
\kvec{F}(x,y)\cdot \ai\Delta x&=M(x,y)\Delta x&&\text{\RL{نچلی}}\\
\kvec{F}(x+\Delta x,y)\cdot \aj\Delta y&=N(x+\Delta x,y)\Delta y&&\text{\RL{دائیں}}\\
\kvec{F}(x,y)\cdot (-\aj)\Delta y&=-N(x,y)\Delta y&&\text{\RL{بائیں}}
\end{aligned}
\end{gather}
ہم مخالف اطراف کے اجزاء کا مجموعہ لیتے ہیں۔
\begin{align}
-[M(x,y+\Delta y)-M(x,y)]\Delta x&\approx -\big(\frac{\partial M}{\partial y}\Delta y\big)\Delta x&&\text{\RL{بالائی اور نچلی}}\label{مساوات_سمتی_تکمل_بالائی_نچلی_گردش}\\
 [N(x+\Delta x,y)-N(x,y)]\Delta y&\approx \big(\frac{\partial N}{\partial x}\Delta x\big)\Delta x&&\text{\RL{دائیں اور بائیں}}\label{مساوات_سمتی_تکمل_دائیں_بائیں_گردش}
\end{align}
مساوات \حوالہ{مساوات_سمتی_تکمل_بالائی_نچلی_گردش} اور مساوات \حوالہ{مساوات_سمتی_تکمل_دائیں_بائیں_گردش} کے مجموعہ کو \عددی{\delta x\Delta y} سے تقسیم کر کے مستطیل کی کثافت دائری بہاو کی تخمینی قیمت حاصل ہو گی:
\begin{align*}
\frac{\text{\RL{مستطیل کے گرد دائری بہاو}}}{\text{\RL{مستطیل کا رقبہ}}}\approx \frac{\partial N}{\partial x}-\frac{\partial M}{\partial y}
\end{align*}
آخر میں ہم \عددی{\Delta x} اور \عددی{\Delta y} کو صفر تک پہنچا کر نقطہ \عددی{(x,y)} پر \عددی{\kvec{F}} کی \ترچھا{کثافت دائری بہاو} کی تعریف اخذ کرتے ہیں جس کو ریاضیات میں \عددی{\kvec{F}} کی \ترچھا{گردش} کہتے ہیں۔

\ابتدا{تعریف}
نقطہ \عددی{(x,y)} پر سمتی میدان \عددی{\kvec{F}} کی کثافت دائری بہاو یا \اصطلاح{گردش}\فرہنگ{گردش}\حاشیہب{curl}\فرہنگ{curl} درج ذیل ہو گا۔
\begin{align}
\text{\RL{گردش \(\kvec{F}\)}}=\frac{\partial N}{\partial x}-\frac{\partial M}{\partial y}
\end{align}
\انتہا{تعریف}
%========================
