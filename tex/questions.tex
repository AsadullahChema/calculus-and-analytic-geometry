\حصہء{سوالات}
\موٹا{$n$ ویں جزوی مجموعہ کی تلاش}\\
سوال \حوالہ{سوال_تسلسل_کلیہ_جزوی_مجموعہ_الف} تا سوال \حوالہ{سوال_تسلسل_کلیہ_جزوی_مجموعہ_ب} میں دیے تسلسل کے \عددی{n} ویں جزوی مجموعہ کا کلیہ تلاش کریں۔اس کلیہ کو استعمال کرتے ہوئے تسلسل (اگر مرتکز ہو) کا مجموعہ حاصل کریں۔

\ابتدا{سوال}\شناخت{سوال_تسلسل_کلیہ_جزوی_مجموعہ_الف}
$2+\frac{2}{3}+\frac{2}{9}+\frac{2}{27}+\cdots+\frac{2}{3^{n-1}}+\cdots$
\انتہا{سوال}
%=========================
\ابتدا{سوال}
$\frac{9}{100}+\frac{9}{100^2}+\frac{9}{100^3}+\cdots+\frac{9}{100^n}+\cdots$
\انتہا{سوال}
%=========================
\ابتدا{سوال}
$1-\frac{1}{2}+\frac{1}{4}-\frac{1}{8}+\cdots+(-1)^{n-1}\frac{1}{2^{n-1}}+\cdots$
\انتہا{سوال}
%=========================
\ابتدا{سوال}
$1-2+4-8+\cdots+(-1)^{n-1}2^{n-1}+\cdots$
\انتہا{سوال}
%=========================
\ابتدا{سوال}
$\frac{1}{2\cdot 3}+\frac{1}{3\cdot 4}+\frac{1}{4\cdot 5}+\cdots+\frac{1}{(n+1)(n+2)}+\cdots$
\انتہا{سوال}
%=========================
\ابتدا{سوال}\شناخت{سوال_تسلسل_کلیہ_جزوی_مجموعہ_ب}
$\frac{5}{1\cdot 2}+\frac{5}{2\cdot 3}+\frac{5}{3\cdot 4}+\cdots+\frac{5}{n(n+1)}+\cdots$
\انتہا{سوال}
%=========================
\موٹا{ہندسی اجزاء والے تسلسل}\\
سوال \حوالہ{سوال_تسلسل_ہندسی_مجموعہ_الف} تا سوال \حوالہ{سوال_تسلسل_ہندسی_مجموعہ_ب} میں تسلسل کے ابتدائی چند اجزاء لکھنے کے بعد تسلسل کا مجموعہ تلاش کریں۔

\ابتدا{سوال}\شناخت{سوال_تسلسل_ہندسی_مجموعہ_الف}
$\sum\limits_{n=0}^{\infty}\frac{(-1)^n}{4^n}$
\انتہا{سوال}
%=====================
\ابتدا{سوال}
$\sum\limits_{n=2}^{\infty}\frac{1}{4^n}$
\انتہا{سوال}
%=====================
\ابتدا{سوال}
$\sum\limits_{n=1}^{\infty}\frac{7}{4^n}$
\انتہا{سوال}
%=====================
\ابتدا{سوال}
$\sum\limits_{n=0}^{\infty}(-1)^n\frac{5}{4^n}$
\انتہا{سوال}
%=====================
\ابتدا{سوال}
$\sum\limits_{n=0}^{\infty}\big(\frac{5}{2^n}+\frac{1}{3^n}\big)$
\انتہا{سوال}
%=====================
\ابتدا{سوال}
$\sum\limits_{n=0}^{\infty}\big(\frac{5}{2^n}-\frac{1}{3^n}\big)$
\انتہا{سوال}
%=====================
\ابتدا{سوال}
$\sum\limits_{n=0}^{\infty}\big(\frac{1}{2^n}+\frac{(-1)^n}{5^n}\big)$
\انتہا{سوال}
%=====================
\ابتدا{سوال}\شناخت{سوال_تسلسل_ہندسی_مجموعہ_ب}
$\sum\limits_{n=0}^{\infty}\big(\frac{2^{n+1}}{5^n}\big)$
\انتہا{سوال}
%=====================
\موٹا{دور بینی تسلسل}\\
سوال \حوالہ{سوال_تسلسل_دور_بینی_الف} تا سوال \حوالہ{سوال_تسلسل_دور_بینی_ب} میں جزوی کسر استعمال کرتے ہوئے تسلسل کا مجموعہ تلاش کریں۔

\ابتدا{سوال}\شناخت{سوال_تسلسل_دور_بینی_الف}
$\sum\limits_{n=1}^{\infty}\frac{4}{(4n-3)(4n+1)}$
\انتہا{سوال}
%========================
\ابتدا{سوال}
$\sum\limits_{n=1}^{\infty}\frac{6}{(2n-1)(2n+1)}$
\انتہا{سوال}
%========================
\ابتدا{سوال}
$\sum\limits_{n=1}^{\infty}\frac{40n}{(2n-1)^2(2n+1)^2}$
\انتہا{سوال}
%========================
\ابتدا{سوال}
$\sum\limits_{n=1}^{\infty}\frac{2n+1}{n^2(n+1)^2}$
\انتہا{سوال}
%========================
\ابتدا{سوال}
$\sum\limits_{n=1}^{\infty}\big(\frac{1}{\sqrt{n}}-\frac{1}{\sqrt{n+1}}\big)$
\انتہا{سوال}
%========================
\ابتدا{سوال}
$\sum\limits_{n=1}^{\infty}\big(\frac{1}{2^{1/n}}-\frac{1}{2^{1/(n+1)}}\big)$
\انتہا{سوال}
%========================
\ابتدا{سوال}
$\sum\limits_{n=1}^{\infty}\big(\frac{1}{\ln(n+2)}-\frac{1}{\ln(n+1)}\big)$
\انتہا{سوال}
%========================
\ابتدا{سوال}\شناخت{سوال_تسلسل_دور_بینی_ب}
$\sum\limits_{n=1}^{\infty}(\tan^{-1}(n)-\tan^{-1}(n+1))$
\انتہا{سوال}
%========================
\موٹا{ارتکاز اور انفراج}\\
سوال \حوالہ{سوال_تسلسل_منفرج_تب_مجموعہ_الف} تا سوال \حوالہ{سوال_تسلسل_منفرج_تب_مجموعہ_ب} میں سے کون سے تسلسل مرتکز اور کون سے منفرج ہیں؟ اپنے جواب کی وجہ پیش کریں۔ مرتکز تسلسل کے مجموعے تلاش کریں۔

\ابتدا{سوال}\شناخت{سوال_تسلسل_منفرج_تب_مجموعہ_الف}
$\sum\limits_{n=0}^{\infty}\big(\frac{1}{\sqrt{2}}\big)^n$
\انتہا{سوال}
%============================ 
\ابتدا{سوال}
$\sum\limits_{n=0}^{\infty}(\sqrt{2})^n$
\انتہا{سوال}
%============================ 
\ابتدا{سوال}
$\sum\limits_{n=1}^{\infty}(-1)^{n+1}\frac{3}{2^n}$
\انتہا{سوال}
%============================ 
\ابتدا{سوال}
$\sum\limits_{n=1}^{\infty}(-1)^{n+1}n$
\انتہا{سوال}
%============================ 
\ابتدا{سوال}
$\sum\limits_{n=0}^{\infty}\cos n\pi$
\انتہا{سوال}
%============================ 
\ابتدا{سوال}
$\sum\limits_{n=0}^{\infty}\frac{\cos n\pi}{5^n}$
\انتہا{سوال}
%============================ 
\ابتدا{سوال}
$\sum\limits_{n=0}^{\infty}e^{-2n}$
\انتہا{سوال}
%============================ 
\ابتدا{سوال}
$\sum\limits_{n=1}^{\infty}\ln \frac{1}{n}$
\انتہا{سوال}
%============================ 
\ابتدا{سوال}
$\sum\limits_{n=1}^{\infty}\frac{2}{10^n}$
\انتہا{سوال}
%============================ 
\ابتدا{سوال}
$\sum\limits_{n=0}^{\infty}\frac{1}{x^n},\quad \abs{x}>1$
\انتہا{سوال}
%============================ 
\ابتدا{سوال}
$\sum\limits_{n=0}^{\infty}\frac{2^n-1}{3^n}$
\انتہا{سوال}
%============================ 
\ابتدا{سوال}
$\sum\limits_{n=1}^{\infty}\big(1-\frac{1}{n}\big)^n$
\انتہا{سوال}
%============================ 
\ابتدا{سوال}
$\sum\limits_{n=0}^{\infty}\frac{n!}{1000^n}$
\انتہا{سوال}
%============================ 
\ابتدا{سوال}
$\sum\limits_{n=1}^{\infty}\frac{n^n}{n!}$
\انتہا{سوال}
%============================ 
\ابتدا{سوال}
$\sum\limits_{n=1}^{\infty}\ln\big(\frac{n}{n+1}\big)$
\انتہا{سوال}
%============================ 
\ابتدا{سوال}
$\sum\limits_{n=1}^{\infty}\ln \big(\frac{n}{2n+1}\big)$
\انتہا{سوال}
%============================ 
\ابتدا{سوال}
$\sum\limits_{n=0}^{\infty}\big(\frac{e}{\pi}\big)^n$
\انتہا{سوال}
%============================ 
\ابتدا{سوال}\شناخت{سوال_تسلسل_منفرج_تب_مجموعہ_ب}
$\sum\limits_{n=0}^{\infty}\frac{e^{n\pi}}{\pi^{ne}}$
\انتہا{سوال}
%============================ 
\موٹا{ہندسی تسلسل}\\
سوال \حوالہ{سوال_تسلسل_ہندسی_عدم_مساوات_الف} تا سوال \حوالہ{سوال_تسلسل_ہندسی_عدم_مساوات_ب} میں ہندسی تسلسل دیے گئے ہیں۔ تسلسل کے چند ابتدائی اجزاء لکھ کر \عددی{a} اور \عددی{r} تلاش کر کے تسلسل کو مجموعہ معلوم کریں۔ اس کے بعد عدم مساوات \عددی{\abs{r}<1} کو \عددی{x} کی صورت میں لکھ کر \عددی{x} کی وہ قیمت دریافت کریں جو عدم مساوات کو مطمئن کرتی ہو اور جس کے لئے تسلسل مرتکز ہو۔  

\ابتدا{سوال}\شناخت{سوال_تسلسل_ہندسی_عدم_مساوات_الف}
$\sum\limits_{n=0}^{\infty}(-1)^nx^n$
\انتہا{سوال}
%======================
\ابتدا{سوال}
$\sum\limits_{n=0}^{\infty}(-1)^nx^{2n}$
\انتہا{سوال}
%======================
\ابتدا{سوال}
$\sum\limits_{n=0}^{\infty}\big(\frac{x-1}{2}\big)^n$
\انتہا{سوال}
%======================
\ابتدا{سوال}\شناخت{سوال_تسلسل_ہندسی_عدم_مساوات_ب}
$\sum\limits_{n=0}^{\infty}\frac{(-1)^n}{2}\big(\frac{1}{3+\sin x}\big)^n$
\انتہا{سوال}
%======================
سوال \حوالہ{سوال_تسلسل_ایکس_قیمت_الف} تا سوال \حوالہ{سوال_تسلسل_ایکس_قیمت_ب} میں \عددی{x} کی وہ قیمتیں معلوم کریں جن کے لئے دیا گیا ہندسی تسلسل مرتکز ہو۔ \عددی{x} کی ان قیمتوں کے لئے تسلسل کے مجموعہ کو \عددی{x} کی صورت میں لکھیں۔

\ابتدا{سوال}\شناخت{سوال_تسلسل_ایکس_قیمت_الف}
$\sum\limits_{n=0}^{\infty}2^nx^n$
\انتہا{سوال}
%==========================
\ابتدا{سوال}
$\sum\limits_{n=0}^{\infty}(-1)^nx^{-2n}$
\انتہا{سوال}
%==========================
\ابتدا{سوال}
$\sum\limits_{n=0}^{\infty}(-1)^n(x+1)^n$
\انتہا{سوال}
%==========================
\ابتدا{سوال}
$\sum\limits_{n=0}^{\infty}\big(-\frac{1}{2}\big)^n(x-3)^n$
\انتہا{سوال}
%==========================
\ابتدا{سوال}
$\sum\limits_{n=0}^{\infty}\sin^n x$
\انتہا{سوال}
%==========================
\ابتدا{سوال}\شناخت{سوال_تسلسل_ایکس_قیمت_ب}
$\sum\limits_{n=0}^{\infty}(\ln x)^n$
\انتہا{سوال}
%==========================
