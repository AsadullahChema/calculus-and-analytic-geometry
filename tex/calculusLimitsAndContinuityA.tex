\حصہ{تصور حد کی توسیع}
اس حصے میں ہم حد کی تصور کو وسعت دیتے ہیں۔
\begin{enumerate}[1.]
\item
\ترچھا{یک طرفہ حد۔} جب \عددی{x} نقطہ \عددی{a} تک بائیں ہاتھ سے پہنچنے کی کوشش کرے تب \اصطلاح{بائیں ہاتھ حد}\فرہنگ{حد!بائیں ہاتھ}\حاشیہب{left-handed limit}\فرہنگ{limit!left-handed} حاصل ہو گا۔اسی طرح جب \عددی{x} نقطہ \عددی{a} تک دائیں ہاتھ سے پہنچنے کی کوشش کرے تب \اصطلاح{دائیں ہاتھ حد}\فرہنگ{حد!دائیں ہاتھ}\حاشیہب{left-handed limit}\فرہنگ{limit!left-handed} حاصل ہو گا۔
\item
\ترچھا{لامنتاہی حد۔} اگرچہ یہ حقیقی حد نہیں ہے لیکن یہ ان تفاعل کا رویہ بیان کرنے میں مدد دیتی ہے جن کی قیمت بہت زیادہ، مثبت یا منفی، ہو جاتی ہو۔  
\end{enumerate}

\جزوحصہء{یک طرفہ حد}
تفاعل \عددی{f} کا نقطہ \عددی{a} پر حد اص صورت \عددی{L} کے برابر ہو گا جب \عددی{a} کے دونوں اطراف \عددی{f} معین ہو اور \عددی{a} کے دونوں اطراف سے نزدیک تر ہونے کی صورت میں \عددی{f} کی قیمت \عددی{L} کے نزدیک تر پہنچتی ہو۔اسی لئے عام حد کو بعض اوقات \اصطلاح{دو طرفہ حد}\فرہنگ{حد!دو طرفہ}\حاشیہب{two-sided limit}\فرہنگ{limit!two-sided} بھی کہتے ہیں۔

عین ممکن ہے کہ صرف بائیں ہاتھ یا صرف دائیں ہاتھ سے  \عددی{a} کے نزدیک تر ہونے سے \عددی{f} کا حد پایا جاتا ہو۔ایسی صورت میں ہم کہتے ہیں کہ \عددی{f} کا \عددی{a} پر یک طرفہ (بائیں ہاتھ یا دائیں ہاتھ)  حد پایا جاتا ہے۔اگر \عددی{x} نقطہ صفر تک دائیں ہاتھ سے پہنچنے کی کوشش کرے تب تفاعل \عددی{f(x)=\tfrac{x}{\abs{x}}} کا حد \عددی{1} ہو گا جبکہ اگر صفر کو \عددی{x} بائیں ہاتھ سے پہنچنے کی کوششش کرے تب تفاعل کا حد \عددی{-1} ہو گا (شکل \حوالہ{شکل_حد_دایاں_بایاں_مختلف})۔
\begin{figure}
\centering
\begin{minipage}{0.45\textwidth}
\centering
\begin{tikzpicture}
\draw[-latex](-2,0)--(2,0)node[right]{$x$};
\draw[-latex](0,-1.3)--(0,1.5)node[above]{$y$};
\draw(-2,-1)--(0,-1)node[ocirc]{}node[right]{$-1$};
\draw(2,1)--(0,1)node[ocirc]{}node[left]{$1$};
\draw(-1.25,0.75)node[]{$y=\tfrac{x}{\abs{x}}$};
\end{tikzpicture}
\caption{مبدا پر بائیں ہاتھ حد اور دائیں ہاتھ حد مختلف ہیں۔}
\label{شکل_حد_دایاں_بایاں_مختلف}
\end{minipage}\hfill
\begin{minipage}{0.45\textwidth}
\centering
\begin{tikzpicture}
\begin{axis}[axis equal,small,axis lines=middle,xlabel={$x$},ylabel={$y$},xmin=-2.5,xmax=2.5,ymin=-0.2,ymax=2.2,xtick={\empty},ytick={\empty},xlabel style={at={(current axis.right of origin)},anchor=west}]
\addplot[domain=0:180]({2*cos(x)},{2*sin(x)});
\draw(axis cs:-2,0)node[circ]{}node[below]{$-2$} (axis cs:2,0)node[circ]{}node[below]{$2$};
\end{axis}
\end{tikzpicture}
\caption{تفاعل کے دائرہ کار کے آخری سروں پر یک طرفہ حد۔}
\label{شکل_حد_دایاں_بایاں_مختلف_نصف_دائرہ}
\end{minipage}%
\end{figure}

\ابتدا{تعریف}\موٹا{دائیں ہاتھ اور بائیں ہاتھ حد کی غیر رسمی تعریف}\\
فرض کریں کہ وقفہ \عددی{(a,b)}، جہاں \عددی{a<b} ہے ، پر تفاعل \عددی{f(x)} معین ہے۔اگر  اس وقفہ کے اندر سے \عددی{a} تک \عددی{x} پہنچنے کی کوشش کرنے سے \عددی{f(x)} کی قیمت \عددی{L} تک پہنچنے کی کوشش کرتی ہو تب ہم کہتے ہیں کہ \عددی{a} پر \عددی{f(x)} کا \اصطلاح{دائیں ہاتھ حد} \عددی{L} ہے جس کو ہم درج ذیل لکھاتےہیں۔
\begin{align*}
\lim\limits_{x\to a^+} f(x)=L
\end{align*}
فرض کریں کہ وقفہ \عددی{(c,a)}، جہاں \عددی{c<a} ہے ، پر تفاعل \عددی{f(x)} معین ہے۔اگر  اس وقفہ کے اندر سے \عددی{a} تک \عددی{x} پہنچنے کی کوشش کرنے سے \عددی{f(x)} کی قیمت \عددی{M} تک پہنچنے کی کوشش کرتی ہو تب ہم کہتے ہیں کہ \عددی{a} پر \عددی{f(x)} کا \اصطلاح{بائیں ہاتھ حد} \عددی{M} ہے جس کو ہم درج ذیل لکھاتےہیں۔
\begin{align*}
\lim\limits_{x\to a^-} f(x)=M
\end{align*}
شکل \حوالہ{شکل_حد_دایاں_بایاں_مختلف} میں تفاعل \عددی{f(x)=\tfrac{x}{\abs{x}}} کے لئے درج ذیل ہیں۔
\begin{align*}
\lim\limits_{x\to a^+} f(x)=1,\quad \lim\limits_{x\to a^-} f(x)=-1
\end{align*}
\انتہا{تعریف}
%==========================

\عددی{x\to a^+} سے مراد ہے کہ  \عددی{a} تک پہنچتے ہوئے \عددی{x} کی قیمت \عددی{a} سے بڑی رہتی ہے۔ اسی طرح \عددی{x\to a^-} سے مراد ہے کہ  \عددی{a} تک پہنچتے ہوئے \عددی{x} کی قیمت \عددی{a} سے چھوٹی رہتی ہے۔ 

دائرہ کار کے آخری سروں پر تفاعل کا عمومی حد نہیں ہو سکتا ہے البتہ دائرہ کار کے آخری سروں پر تفاعل کا یک طرفہ حد  ہو سکتا ہے۔

\ابتدا{مثال}
تفاعل \عددی{f(x)=\sqrt{4-x^2}} کا دائرہ کار \عددی{[-2,2]} ہے۔تفاعل کی ترسیم نصف دائرہ ہے جس کو شکل \حوالہ{شکل_حد_دایاں_بایاں_مختلف_نصف_دائرہ} میں دکھایا گیا ہے۔دائرہ کار کے آخری سروں پر یک طرفہ حد درج ذیل ہیں۔
\begin{align*}
\lim\limits_{x\to -2^+} \sqrt{4-x^2}=0,\quad  \lim\limits_{x\to 2^-} \sqrt{4-x^2}=0
\end{align*}
\انتہا{مثال}
%=========================
