%this file describes urdu commands for the commonly used english latex commands

%chapter, section etc
%  \newcommand*{newcommand}[arguments]{actual command}
\newcommand*{\باب}[1]{\chapter{#1}}                                      %defining commonly used commands
\newcommand*{\حصہ}[1]{\section{#1}}
\newcommand*{\جزوحصہ}[1]{\subsection{#1}}
\newcommand*{\جزوجزوحصہ}[1]{\subsubsection{#1}}

\newcommand*{\بابء}[1]{\chapter*{#1}}                                      %defining commonly used commands
\newcommand*{\حصہء}[1]{\section*{#1}}
\newcommand*{\جزوحصہء}[1]{\subsection*{#1}}
\newcommand*{\جزوجزوحصہء}[1]{\subsubsection*{#1}}


%english text in urdu mode
\newcommand*{\تحریر}[1]{\textenglish{#1}}	% english text in urduMode
%\newcommand*{\موٹا}[1]{\textbf{#1}}
%\newcommand*{\ترچھا}[1]{{\textit{#1}}}
\newcommand*{\موٹا}[1]{{\urduTechTermsfont{#1}}}
%\newcommand*{\ترچھا}[1]{{\urduTechTermsfont{#1}}}
\newcommand*{\ترچھا}[1]{{\small{#1}}}

%\newcommand*{\اصطلاح}[1]{{\color{red}{#1}}}   %colours spills to next word when there is index or footnote entry with the word
%\newcommand{\اصطلاح}[1]{{\urdufontBig{#1}}}

\newcommand{\اصطلاح}[1]{{\urduTechTermsfont{#1}}}


%end commands cannot be redefined and as such these two are not usable
\providecommand*{\ابتدا}[1]{\begin{#1}}
\providecommand*{\انتہا}[1]{\end{#1}}

%include and input directives for adding external files into the main document 
\newcommand*{\بشمول}[1]{\includeonly{#1}}
\newcommand*{\شامل}[1]{\include{#1}}
\newcommand*{\داخل}[1]{\input{#1}}

%to use extra latex packages
\newcommand*{\استعمال}[1]{\usepackage{#1}}

%footnotes and indexes
\newcommand*{\حاشیہب}[1]{{\raggedright{\footnote{\textenglish{#1}}}} }               %footnote to the left hand side
\newcommand*{\حاشیہد}[1]{{\raggedleft{\footnote{#1}}}}
\newcommand*{\حاشیہط}[1]{\marginpar{#1}}

\newcommand*{\فرہنگ}[1]{\index{#1}}

%references and labels
\newcommand*{\شناخت}[1]{\label{#1}}
\newcommand*{\حوالہ}[1]{\ref{#1}}
\newcommand*{\حوالہصفحہ}[1]{\pageref{#1}}

%counters
\newcommand*{\فاصلہ}{\vspace*{10mm}}
\newcommand*{\فاصلہء}{\quad}

%itemize, bullets and numbered items   
\newcommand*{\اشیاء}{itemize}                               %used in   \begin{itemize}
\newcommand*{\شے}[1]{\item {#1}}			%used in    \item, \description
%description
\newcommand*{\جزو}[1]{\item[#1]}                      %used in \begin{description}
%maths commands
\newcommand*{\عددی}[1]{\: \ensuremath{#1} \:} % in-line math & inside math mode
\newcommand*{\عددیء}[1]{\ensuremath{#1}}
\newcommand*{\سمتیہ}[1]{\ensuremath{{\bf{#1}}}}
\newcommand*{\سمتیازیرنوشت}[2]{\ensuremath{{\boldsymbol{#1}}_{\textup{#2}}}}

\newcommand*{\ضرب}{\time}					%multiplication symbol
\newcommand*{\نکطہد}{\cdot}
\newcommand*{\نقطے}{\ensuremath{\cdots}}

\newcommand*{\زیرنوشت}[3]{\: \ensuremath{{#1_{#2 \textrm {#3}}}} \:}   %english+urdu subscript \زیرنوشت{V}{CE}{غیرافزائندہ}
\newcommand*{\سیدھازیرنوشت}[2]{\: \ensuremath{{#1_{\textup{#2}}}} \:} %RC

\newcommand*{\قریب}[1]{\mbox{#1}}
\newcommand{\سن}[1]{؁\,\ensuremath{#1}}


\renewcommand{\indexname}{فرہنگ}        %does nothing here. must be placed within begin{urdufont} environment to be the last to take effect 
%===============================


%numbering scheme
\renewcommand*{\thefigure}{\arabic{figure}.\thechapter}
\renewcommand*{\thetable}{\arabic{table}.\thechapter}
\renewcommand*{\theequation}{\arabic{equation}.\thechapter}
\renewcommand*{\thesection}{\arabic{section}.\thechapter}
\renewcommand*{\thesubsection}{\arabic{subsection}.\arabic{section}.\thechapter}
\renewcommand*{\thesubsubsection}{\arabic{subsubsection}.\arabic{subsection}.\arabic{section}.\thechapter}
%=======================================
%the following pertains to theorem environment. added to maths sty
\renewcommand*{\thetheorem}{\arabic{theorem}.\thechapter}
\renewcommand*{\thecorollary}{\arabic{corollary}.\arabic{theorem}.\thechapter}
\renewcommand*{\thelemma}{\arabic{lemma}.\thechapter}
\renewcommand*{\thedefinition}{\arabic{definition}.\thechapter}

%================
%my environments
%================

%environment for examples مثال
%\newcounter{examplecounter}[section]
%\renewcommand{\theexamplecounter}{\arabic{examplecounter}}
\newcounter{examplecounter}[chapter]
\renewcommand{\theexamplecounter}{\arabic{examplecounter}.\thechapter}

\newenvironment*{مثال}
{\par\noindent\ignorespaces    مثال \refstepcounter{examplecounter} \theexamplecounter :\quad}%
{\hfill\qedsymbol  \vspace{\baselineskip}\par}
%{\noindent\ignorespaces \vspace{\baselineskip} \hrule \vspace{\baselineskip}  مثال \refstepcounter{examplecounter} \theexamplecounter :}%
%{\par\noindent \hrule  \vspace{\baselineskip}}

%------
%practice problems environment مشق

%\newcounter{practicecounter}[section]                              %practice here means مشق
%\renewcommand{\thepracticecounter}{\arabic{practicecounter}}
\newcounter{practicecounter}[chapter]                              %practice here means مشق
\renewcommand{\thepracticecounter}{\arabic{practicecounter}.\thechapter}

\newenvironment*{مشق}
{\par\noindent\ignorespaces \vspace{\baselineskip} \hrule \vspace{\baselineskip} مشق \refstepcounter{practicecounter} \thepracticecounter :\quad}%
{\par\noindent \hrule  \vspace{\baselineskip}}
%---------

%end of chapter questions environment سوال

\newcounter{questioncounter}[section]                                           %for reseting at every section. to be used only during writing stage
\renewcommand{\thequestioncounter}{\arabic{questioncounter}}
%\newcounter{questioncounter}[chapter]						%when book is finished, use this instead of the above
%\renewcommand{\thequestioncounter}{\arabic{questioncounter}.\thechapter}

\newenvironment*{سوال}				
{\noindent\ignorespaces  سوال \refstepcounter{questioncounter} \thequestioncounter :\quad}%
{\par\noindent }
%--------------------

%defining a LAW   قانون

%\newcounter{lawcounter}[section]
%\renewcommand{\thelawcounter}{\arabic{lawcounter}}
\newcounter{lawcounter}[chapter]
\renewcommand{\thelawcounter}{\arabic{lawcounter}.\thechapter}

\newenvironment*{قانون}				
{\par\medskip \refstepcounter{lawcounter} \quad\nopagebreak}%
{\par\hfill\qedsymbol  \vspace{\baselineskip}\par }
%{\par\medskip }
%--------------------
%--------------------

%defining a THEOREM   مسئلہ

%\newcounter{kthcounter}[section]
%\renewcommand{\thekthcounter}{\arabic{kthcounter}}
\newcounter{kthcounter}[chapter]
\renewcommand{\thekthcounter}{\arabic{kthcounter}.\thechapter}

\newenvironment*{مسئلہ}				
{\par\noindent\ignorespaces  مسئلہ \refstepcounter{kthcounter} \thekthcounter :\quad}%
{\par\noindent }
%--------------------
%--------------------

%defining a Proof   ثبوت

%\newcounter{kprcounter}[chapter]
%\renewcommand{\thekprcounter}{\arabic{kprcounter}.\thechapter}

\newenvironment*{ثبوت}				
{\noindent\ignorespaces  ثبوت :\quad}%
{\par\hfill\qedsymbol  \vspace{\baselineskip}\par }
%{\par\noindent\qedsymbol  \vspace{\baselineskip} }
%--------------------
%--------------------
%++++++++++++++++++++++++++++++

%--------------------

%defining a COROLLARY   ضمنی نتیجہ


\newcounter{kcocounter}[chapter]
\renewcommand{\thekcocounter}{\arabic{kcocounter}.\thechapter}

\newenvironment*{ضمنی نتیجہ}				
{\par\noindent\ignorespaces  ضمنی نتیجہ \refstepcounter{kcocounter} \thekcocounter :\quad}%
{\par\noindent }
%--------------------
%--------------------

%defining a Proof   ثبوت ضمنی نتیجہ

%\newcounter{kprcocounter}[chapter]
%\renewcommand{\thekprcocounter}{\arabic{kprcocounter}.\thechapter}

\newenvironment*{ثبوت ضمنی نتیجہ}				
{\noindent\ignorespaces  ثبوت ضمنی نتیجہ :\quad}%
{\par\hfill\qedsymbol  \vspace{\baselineskip}\par }
%{\par\noindent\qedsymbol  \vspace{\baselineskip} }
%--------------------
%--------------------
%++++++++++++++++++++++++++++++++++++++++++++

%defining a Definition تعریف

%\newcounter{kdfcounter}[chapter]
%\renewcommand{\thedfrcounter}{\arabic{kdfcounter}.\thechapter}

\newenvironment*{تعریف}				
{\par\noindent\ignorespaces  تعریف :\quad}%
{\par\hfill\qedsymbol  \vspace{\baselineskip}\par }
%{\par\noindent }
%--------------------
%--------------------

%defining a Definition مفروضہ

%\newcounter{kAscounter}[chapter]
%\renewcommand{\theAsrcounter}{\arabic{kAscounter}.\thechapter}

\newenvironment*{مفروضہ}				
{\par\noindent\ignorespaces  مفروضہ \quad}%
{\par\hfill\qedsymbol  \vspace{\baselineskip}\par }
%{\par\noindent }
%--------------------
%--------------------

%defining a RULE   قاعدہ

%\newcounter{krucounter}[section]
%\renewcommand{\thekrucounter}{\arabic{krucounter}}
\newcounter{krucounter}[chapter]
\renewcommand{\thekrucounter}{\arabic{krucounter}.\thechapter}

\newenvironment*{قاعدہ}				
{\par\noindent\ignorespaces  قاعدہ \refstepcounter{krucounter} \thekrucounter :\quad}%
{\par\noindent }
%--------------------
%--------------------

%defining a Proof   ثبوت قاعدہ

%\newcounter{kprRcounter}[chapter]
%\renewcommand{\thekprRcounter}{\arabic{kprRcounter}.\thechapter}

\newenvironment*{ثبوت قاعدہ}				
{\noindent\ignorespaces  ثبوت قاعدہ :\quad}%
{\par\hfill\qedsymbol  \vspace{\baselineskip}\par }
%{\par\noindent\qedsymbol  \vspace{\baselineskip} }
%--------------------
%--------------------

%defining a TEST   پرکھ

%\newcounter{kttcounter}[section]
%\renewcommand{\thekttcounter}{\arabic{kttcounter}}
\newcounter{kttcounter}[chapter]
\renewcommand{\thekttcounter}{\arabic{kttcounter}.\thechapter}

\newenvironment*{پرکھ}				
{\par\noindent\ignorespaces  \refstepcounter{kttcounter} \quad \nopagebreak}%
{\par\hfill\qedsymbol  \vspace{\baselineskip}\par }
%--------------------

\newenvironment*{ثبوت پرکھ}				
{\noindent\ignorespaces  ثبوت پرکھ :\quad}%
{\par\hfill\qedsymbol  \vspace{\baselineskip}\par }
%{\par\noindent\qedsymbol  \vspace{\baselineskip} }

