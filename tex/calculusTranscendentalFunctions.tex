\باب{ماورائی تفاعل}
        ریاضیات میں بہت سے تفاعل ایک دوسرے کے الٹ ہیں۔ غالباً سب سے زیادہ جانی پہچانی الٹ تفاعل کی جوڑی \عددی{\ln x} اور \عددی{e^x} ہے۔ موزوں وقفہ پر پابند تکونیاتی تفاعل کے اہم الٹ پائے  جاتے ہیں۔ اسی طرح  لوگارتھمی اور قوت نمائی تفاعل کے دیگر الٹ جوڑیاں پائی جاتی ہیں۔ ہذلولی تفاعل اور ان کے الٹ تفاعل کا استعمال آویزاں رسی، منتقلی حرکی توانائی، اور ہوا میں گرتے ہوئے جسم پر قوت رگڑ کے مسائل میں کام آتے ہیں۔ اس باب میں ان تمام تفاعل پر غور کیا جائے گا۔ ان مسئلوں کا بھی ذکر کیا جائے گا جنہیں یہ تفاعل حل کرنے میں مدد گار ثابت ہوتے ہیں۔

\حصہ{الٹ تفاعل اور ان کے تفرق}
اس حصہ میں ہم الٹ تفاعل کی تعریف پیش کرتے ہیں اور ان کی کلیات، ترسیمات، اور الٹ جوڑیوں کے تفرق پر غور کرتے ہیں۔

\جزوحصہء{ایک ایک تفاعل}
تفاعل سے مراد وہ قاعدہ ہے جو اپنی دائرہ کار کے ہر نقطہ کو اپنی سعت میں ایک قیمت مختص کرتا ہو۔بعض تفاعل ایک ہی قیمت کو ایک سے زیادہ  نقطوں  کے لئے مختص کرتے ہیں۔ یوں \عددی{-1} کا مربع اور \عددی{1} کا مربع \عددی{1} ہے؛ اسی طرح \عددی{\tfrac{\pi}{3}} اور \عددی{\tfrac{2\pi}{3}} کا سائن \عددی{\tfrac{\sqrt{3}}{2}} ہے۔ اس کے بر عکس دیگر تفاعل کسی ایک قیمت کو کبھی بھی دو بار مختص نہیں کرتے ہیں۔ مختلف اعداد کے جذر المربع اور جذر الکعب ہر صورت ایک دوسرے سے مختلف ہوتے ہیں۔ ایسا تفاعل جس کے انفرادی نقطوں پر منفرد قیمت ہو کو \اصطلاح{ایک ایک تفاعل}\فرہنگ{ایک ایک تفاعل}\حاشیہب{one to one function}\فرہنگ{one to one} کہتے ہیں۔

\ابتدا{تعریف}
دائرہ کار \عددی{D} پر تفاعل \عددی{f(x)}  تب \اصطلاح{ایک ایک} ہو گا جب \عددی{x_1\ne x_2} کی صورت میں \عددی{f(x_1)\ne f(x_2)} ہو۔
\انتہا{تعریف}
%=====================

\ابتدا{مثال}
چونکہ کسی بھی غیر منفی اعداد کے لئے \عددی{x_1\ne x_2} کی صورت میں \عددی{\sqrt{x_1}\ne \sqrt{x_2}} ہے  لہٰذا \عددی{f(x)=\sqrt{x}} غیر منفی اعداد کے کسی بھی دائرہ کار پر یہ ایک ایک تفاعل ہے۔
\انتہا{مثال}
%====================
\ابتدا{مثال}
چونکہ \عددی{\sin(\tfrac{\pi}{6})=\sin(\tfrac{5\pi}{6})} ہے لہٰذا وقفہ \عددی{[0,\pi]} پر \عددی{g(x)=\sin x} ایک ایک تفاعل نہیں ہے۔ اس کے برعکس چونکہ ربع اول میں تمام زاویوں کے سائن مختلف ہیں لہٰذا وقفہ \عددی{[0,\tfrac{\pi}{2}]} پر \عددی{g(x)=\sin x} ایک ایک تفاعل ہے۔
\انتہا{مثال}
%====================

ایک ایک تفاعل \عددی{y=f(x)} کی ترسیم کسی بھی افقی لکیر کو زیادہ سے زیادہ ایک بار قطع کرتی ہے۔ اگر کسی تفاعل کی ترسیم کسی افقی لکیر کو ایک سے زیادہ مرتبہ قطع کرتی ہو تب یہ تفاعل \عددی{y} کی اس قیمت کو ایک سے زیادہ مرتبہ اختیار کرتا ہے لہٰذا یہ ایک ایک تفاعل نہیں ہو گا۔

\موٹا{افقی لکیر کا پرکھ}\\
کوئی بھی تفاعل \عددی{y=f(x)} صرف اور صرف اس صورت ایک ایک تفاعل ہو گا جب اس کی ترسیم ہر افقی لکیر کو زیادہ سے زیادہ ایک بار قطع کرتی ہو۔ 

\جزوحصہء{الٹ}
چونکہ ایک ایک تفاعل کا ہر مخارج  انفرادی مداخل  سے آتا ہے لہٰذا ایک ایک تفاعل کو الٹ کرتے ہوئے ہر مخارج کو واپس اس مداخل پر بھیجا جا سکتا ہے جس سے یہ مخارج حاصل ہوتا ہے۔ ایک ایک تفاعل \عددی{f} کو الٹ کر کے جو تفاعل حاصل ہوتا ہے اس کو \عددی{f} کا \اصطلاح{الٹ}\فرہنگ{الٹ}\حاشیہب{inverse}\فرہنگ{inverse} کہتے ہیں جس کو \عددی{f^{-1}} سے ظاہر کیا جاتا ہے جہاں \عددی{f^{-1}} میں \عددی{-1} کو طاقت نہ سمجھا جائے: یعنی \عددی{f(x)^{-1}} سے مراد \عددی{\tfrac{1}{f(x)}} نہیں ہے۔ ہم \عددی{f^{-1}} کو "\عددی{f} کا الٹ" پڑھتے ہیں۔

جیسا شکل سے ظاہر ہے، \عددی{f} سے \عددی{f^{-1}} یا \عددی{f^{-1}} سے \عددی{f} حاصل کیا جا سکتا ہے۔ یوں کسی بھی \عددی{x} کے لئے \عددی{f(x)} حاصل کر کے اس \عددی{f(x)} کا الٹ \عددی{f^{-1}(f(x))} حاصل کیا جا سکتا ہے جو \عددی{x} ہو گا۔ تفاعل \عددی{f^{-1}(f(x))} یا تفاعل \عددی{f(f^{-1}(x))} میں \عددی{x} پر کرنے سے واپس \عددی{x} ملتا ہے۔ ایسا تفاعل جو ہر عدد کو اسی عدد کے لئے مختص کرتا ہو  \اصطلاح{شناختی تفاعل}\فرہنگ{شناختی تفاعل}\فرہنگ{تفاعل!شناختی}\حاشیہب{identity function}\فرہنگ{identity function}\فرہنگ{function!identity} کہلاتا ہے۔ یوں تفاعل \عددی{f} اور \عددی{g} کو ایک دوسرے  کا الٹ تفاعل ہونے کے لئے پرکھا جا سکتا ہے۔اگر \عددی{(f\circ g)(x)=(g\circ f)(x)=x} ہو تب \عددی{f} اور \عددی{g} ایک دوسرے کے الٹ تفاعل ہوں گے ورنہ یہ ایک دوسرے کے الٹ تفاعل نہیں ہوں گے۔ اگر \عددی{f} اپنے دائرہ کار کا مکعب لیتا ہو تب \عددی{g} اس صورت \عددی{f} کا الٹ ہو گا اگر \عددی{g} جذر الکعب لیتا ہو ورنہ یہ \عددی{f} کا الٹ نہیں ہو گا۔

تفاعل \عددی{f} اور \عددی{g} ایک دوسرے کے الٹ صرف اور صرف اس صورت ہوں گے جب
\begin{align*}
f(g(x))=x\quad \text{اور}\quad g(f(x))=x
\end{align*}
ہوں۔ایسی صورت میں \عددی{g=f^{-1}} اور \عددی{f=g^{-1}} ہوں گے۔

ایک تفاعل کا الٹ صرف اور صرف اس صورت ہو گا جب یہ ایک ایک تفاعل ہو۔ یوں بڑھتے تفاعل کا الٹ تفاعل ہو گا اور  گھٹتے تفاعل کا بھی الٹ تفاعل ہو گا۔ جن تفاعل کا تفرق مثبت ہو وہ اپنے دائرہ کار میں بڑھتے ہیں لہٰذا ان کا الٹ ہو گا (صفحہ \حوالہصفحہ{نتیجہ_صریح_استعمال_سوم} پر مسئلہ اوسط قیمت کا ضمنی نتیجہ \حوالہ{نتیجہ_صریح_استعمال_سوم})۔اسی طرح جن تفاعل کا تفرق منفی ہو وہ اپنے دائرہ کار میں گھٹتے ہیں لہٰذا ان کا الٹ ہو گا۔

\جزوحصہء{الٹ کی تلاش}
تفاعل کے الٹ کی ترسیم کا تفاعل کے ترسیم کے ساتھ کیا تعلق ہے؟ فرض کریں ایک تفاعل کی ترسیم شکل کی طرح  بڑھتا ہو، یعنی یہ بائیں سے دائیں اوپر اٹھتی  ہو۔ کسی بھی \عددی{x} کے لئے ترسیم سے قیمت پڑھنے کے لئے ہم محور \عددی{x} پر نقطہ \عددی{x} سے شروع ہو کر محور \عددی{y} کے متوازی چل کر ترسیم تک پہنچتے ہیں اور یہاں سے محور \عددی{x} کے متوازی چل کر محور \عددی{y} تک پہنچ کر تفاعل کی قیمت \عددی{y} پڑھتے ہیں۔ہم اس عمل کو الٹ کرتے ہوئے \عددی{y} سے شروع کرتے ہوئے \عددی{x} پڑھ سکتے ہیں۔

تفاعل \عددی{f} کی ترسیم حاصل کرنے کی خاطر ہم \عددی{f^{-1}} کی ترسیم میں مداخل مخارج جوڑیوں کا  کا آپس میں تبادلہ  کرتے ہیں۔ اس ترسیم کو عمومی طرز پر دکھانے کی خاطر ہمیں ان جوڑیوں کا \عددی{45^{\circ}} کی لکیر \عددی{y=x} میں عکس لینا ہو گا اور ساتھ ہی  حرف \عددی{x} اور حرف \عددی{y} کا ایک دوسرے کے ساتھ تبادلہ کرنا ہو گا۔ یوں غیر تابع متغیر، جس کو اب \عددی{x} کہتے ہیں، افقی محور پر دکھایا جائے گا اور تابع متغیر، جس کو اب \عددی{y} کہتے ہیں، کو انتصابی محور پر دکھایا جائے گا۔ تفاعل \عددی{f(c)} اور \عددی{f^{-1}(x)} کی ترسیمات لکیر \عددی{y=x} کے لحاظ سے تشاکلی ہیں۔

شکل میں \عددی{f^{-1}} کو متغیر \عددی{x} کا تفاعل لکھنا دکھانا گیا ہے جس کو درج ذیل بیان کیا جا سکتا ہے۔
\begin{enumerate}[a.]
\item
مساوات \عددی{y=f(x)} کو  \عددی{x} کے لئے حل کریں۔ یوں \عددی{x} کو \عددی{y} کی صورت میں لکھا جائے گا۔
\item
جزو-ا میں حاصل مساوات میں \عددی{x} اور \عددی{y} کا آپس میں تبادلہ کریں۔ یوں حاصل کلیہ \عددی{y=f^{-1}(x)} ہو گا۔
\end{enumerate} 

\ابتدا{مثال}\شناخت{مثال_ماورائی_الٹ}
تفاعل \عددی{y=\tfrac{x}{2}+1} کا الٹ حاصل کریں جہاں غیر تابع متغیر \عددی{x} ہو۔

حل:\quad
قدم ا:\quad
\عددی{x} کے لئے حل کرتے ہیں۔
\begin{align*}
y&=\frac{x}{2}+1\\
2y&=x+2\\
x&=2y-2
\end{align*}
قدم ب:\quad
حاصل مساوات میں \عددی{x} اور \عددی{y} کا آپس میں تبادلہ کرتے ہیں۔
\begin{align*}
y=2x-2
\end{align*}
یوں تفاعل \عددی{f(x)=\tfrac{x}{2}+1} کا الٹ تفاعل \عددی{f^{-1}(x)=2x-2} ہو گا۔

اس کی تصدیق کرنے کی خاطر ہم دیکھتے ہیں کہ آیا دونوں مرکب تفاعل شناختی تفاعل دیتے ہیں:
\begin{align*}
f^{-1}(f(x))&=2\big(\frac{x}{2}+1\big)-2=x+2-2=x\\
f(f^{-1}(x))&=\frac{1}{2}(2x-2)+1=x-1+1=x
\end{align*}
\انتہا{مثال}
%===================
\ابتدا{مثال}
تفاعل \عددی{y=x^2,\, x\ge 0} کا الٹ تلاش کریں جہاں غیر تابع متغیر \عددی{x} ہو۔

حل:\quad
قدم ا:\quad
دیے گئے مساوات کو حل کر کے \عددی{x} کو \عددی{y} کی صورت میں لکھتے ہیں۔
\begin{align*}
y&=x^2\\
\sqrt{y}&=\sqrt{x^2}=\abs{x}=x&&\text{\RL{$x\ge 0$ کی بنا $\abs{x}=x$ ہو گا}}
\end{align*}
قدم ب:\quad
جزو-ا میں حاصل نتیجہ میں \عددی{x} اور \عددی{y} کا آپس میں تبادلہ کرتے ہیں۔
\begin{align*}
y=\sqrt{x}
\end{align*}
یوں تفاعل \عددی{y=x^2,\, x\ge 0} کا الٹ \عددی{y=\sqrt{x}} ہو گا۔

یہاں دھیان رہے کہ پابند تفاعل \عددی{y=\sqrt{x},\,x\ge 0} ایک ایک تفاعل ہے لہٰذا اس کا الٹ پایا جاتا ہے جبکہ تفاعل \عددی{y=x^2} ایک غیر پابند تفاعل ہے جو ایک ایک تفاعل نہیں ہے لہٰذا اس کا الٹ نہیں پایا جاتا ہے۔
\انتہا{مثال}
%=====================                            

\موٹا{کمپیوٹر کا استعمال}\\
تفاعل \عددی{y=f(x)} کا الٹ تفاعل نہایت آسانی سے درج ذیل مقدار معلوم روپ استعمال کرتے ہوئے ترسیم کیا جا سکتا ہے۔
\begin{align*}
x(t)=f(t),\quad y(t)=t
\end{align*}
آپ تفاعل اور تفاعل کے الٹ کو ساتھ ساتھ ترسیم کر سکتے ہیں:
\begin{align*}
x_1(t)&=t,\quad y_1(t)=f(t)&&\text{\RL{تفاعل}}\\
x_2(t)& =f(t),\quad y_2(t)=t&&\text{\RL{تفاعل کا الٹ}}
\end{align*}
اس سے بھی زیادہ بہتر ہو گا کہ تفاعل، تفاعل کا الٹ اور شناختی تفاعل \عددی{y=x} کو ساتھ ساتھ ترسیم کریں جہاں شناختی تفاعل درج ذیل ہو گا۔
\begin{align*}
x_3(t)&=t,\quad y_3(t)=t&&\text{\RL{شناختی تفاعل}}
\end{align*}

تفاعل \عددی{y=\tfrac{x^5}{x^2+1}} اور \عددی{y=x+\cos x} کے ساتھ ان کے الٹ تفاعل اور شناختی تفاعل ایک ساتھ ترسیم کر کے دیکھیں۔ ترسیم میں \عددی{x} اور \عددی{y} محور کے اکائی فاصلے برابر نظر آنے چاہیے تا کہ لکیر \عددی{y=x} کے لحاظ سے تفاعل اور اس کا الٹ تشاکلی  نظر آئیں۔ 

\جزوحصہء{قابل تفرق تفاعل کے الٹ کے تفرق}
تفاعل \عددی{f(x)=\tfrac{x}{2}+1}  اور اس کے الٹ \عددی{f^{-1}(x)=2x-2} (مثال \حوالہ{مثال_ماورائی_الٹ}) کے تفرق درج ذیل ہیں۔
\begin{align*}
\frac{\dif}{\dif x}f(x)&=\frac{\dif}{\dif x}\big(\frac{x}{2}+1\big)=\frac{1}{2}\\
\frac{\dif}{\dif x}f^{-1}(x)&=\frac{\dif}{\dif x}(2x-2)=2
\end{align*}
یہ تفرقات ایک دوسرے کے بالعکس متناسب ہیں۔ تفاعل \عددی{f} کی ترسیم لکیر \عددی{y=\tfrac{x}{2}+1} اور \عددی{f^{-1}} کی ترسیم لکیر \عددی{y=2x-2} ہے۔ ان لکیروں کے ڈھلوان ایک دوسرے کے بالعکس متناسب ہیں۔

یہ نتیجہ کسی مخصوص تفاعل کے لئے نہیں ہے۔ لکیر \عددی{y=x} میں کسی بھی غیر افقی یا غیر انتصابی لکیر کے عکس کا ڈھلوان اس لکیر کے ڈھلوان کے بالعکس متناسب ہو گا۔ یوں اگر دیے گئے لکیر کا ڈھلوان \عددی{m\ne 0} ہو تب منعکس لکیر کا ڈھلوان \عددی{\tfrac{1}{m}} ہو گا۔ 

تفاعل اور اس کے الٹ کے ڈھلوانوں  کا بالعکس متناسب تعلق دیگر تفاعل کو بھی مطمئن کرتا ہے۔ اگر نقطہ \عددی{(a,f(a))} پر \عددی{y=f(x)} کا ڈھلوان \عددی{f'(a)\ne 0} ہو تب مطابقتی نقطہ \عددی{(f(a),a)} پر \عددی{y=f^{-1}(x)} کا ڈھلوان \عددی{\tfrac{1}{f'(a)}} ہو گا۔ یوں \عددی{f(a)} پر \عددی{f^{-1}} کا تفرق \عددی{a} پر \عددی{f} کے تفرق کا بالعکس متناسب ہو گا۔ یہ تعلق اس صورت درست ہو گا جب \عددی{f} درج ذیل مسئلہ میں پیش   شرائط کو مطمئن کرتا ہو۔ یہ شرائط اعلٰی احصاء سے حاصل ہوتے ہیں۔ 

\ابتدا{مسئلہ}\موٹا{الٹ تفاعل کے تفرق کا قاعدہ}\\

\انتہا{مسئلہ}

